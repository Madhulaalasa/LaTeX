\documentclass[11pt, openany]{book}
\usepackage[text={4.65in,7.45in}, centering, includefoot]{geometry}
\usepackage[table, x11names]{xcolor}
\usepackage{fontspec,realscripts}
\usepackage{polyglossia}
\setdefaultlanguage{sanskrit}
\setotherlanguage{english}
\setmainfont[Scale=1]{Shobhika}
\newfontfamily\s[Script=Devanagari, Scale=0.9]{Shobhika}
\newfontfamily\regular{Linux Libertine O}
\newfontfamily\en[Language=English, Script=Latin, Scale=0.98]{Linux Libertine O}
\newfontfamily\vy[Script=Devanagari, Scale=1, Color=purple]{Shobhika-Bold}
\newfontfamily\qt[Script=Devanagari, Scale=1, Color=violet]{Shobhika-Regular}
\newcommand{\devanagarinumeral}[1]{
\devanagaridigits{\number \csname c@#1\endcsname}} % for devanagari page numbers
\XeTeXgenerateactualtext=1 % for searchable pdf
\usepackage{enumerate}
\pagestyle{plain}
\usepackage{fancyhdr}
\pagestyle{fancy}
\renewcommand{\headrulewidth}{0pt}
\usepackage{afterpage}
\usepackage{multirow}
\usepackage{multicol}
\usepackage{wrapfig}
\usepackage{vwcol}
\usepackage{microtype}
\usepackage{amsmath,amsthm, amsfonts,amssymb}
\usepackage{mathtools}% < -- new package for rcases
\usepackage{graphicx}
\usepackage{longtable}
\usepackage{setspace}
\usepackage{footnote}
\usepackage{perpage}
\MakePerPage{footnote}
\usepackage{xspace}
\usepackage{array}
\usepackage{emptypage}
\usepackage{hyperref}% Package for hyperlinks
\hypersetup{colorlinks, citecolor=black, filecolor=black, linkcolor=blue, urlcolor=black}
\setlength{\parskip}{0.6em}

\begin{document}
\thispagestyle{empty}

\begin{center}
{\large THE}\\

\vspace{5mm}
\textbf{\Huge VYAVAHĀRAMAYŪKHA}\\

\vspace{5mm}
{\large OF}\\

\vspace{5mm}
\textbf{\huge BHAṬṬA NĪLAKAṆṬHA}\\

\vspace{3mm}
\textbf{WITH AN INTRODUCTION, NOTES AND APPENDICES}\\

\vspace{4mm}
BY\\

\vspace{4mm}
\textbf{\Large P.V.KANE, M.A.LL.M.,}\\

\vspace{4mm}
VAKIL, HIGH COURT, BOMBAY;\\

\vspace{3mm}
\textbf{\en ZALA VEDANTA PRIZEMAN; MANDLIK GOLD MEDALIST; SOMETIME}\\

\vspace{3mm}
\textbf{\en PROFESSOR OF SANSKRIT, ELPHINSTONE COLLEGE, BOMBAY;}\\

\vspace{3mm}
\textbf{\en FELLOW OF THE BOMBAY ASIATIC SOCIETY, MEMBER OF}\\

\vspace{3mm}
\textbf{\en THE SENATE OF THE BOMBAY UNIVERSITY; AUTHOR OF}\\

\vspace{3mm}
\textbf{\en {\qt THE HISTORY OF SANSKRIT POETICS} \& c.}\\

\vspace{8mm}
\rule{0.4\linewidth}{0.5pt}\\

\textbf{\large First Editlon, 1000 copies.}\\

\rule{0.4\linewidth}{0.5pt}\\

\vspace*{\fill}
\onehalfspacing

{\LARGE 1926}\\

\rule{0.2\linewidth}{0.5pt}\\

\textbf{\emph \Large \en Price Rupees Ten.}
\end{center}

\newpage
\thispagestyle{empty}

\vspace*{\fill}
\onehalfspacing
\begin{center}
Published by Dr. V. G. Paranjpe, M. A., LL. B, D.Litt.,\\

Secretary, Bhandarkar Oriental Research Institute, Poona.\\

\rule{0.3\linewidth}{0.5pt}\\

Printed by Ramchandra Yesu Shedge, at the {\qt Nirnaya\textendash\ sagar} Press,\\

26\textendash\ 28, Kolbhat Lane, Bombay.
\end{center}
\vspace*{\fill}
\onehalfspacing

\newpage
\thispagestyle{empty}
\begin{center}
\textbf{\LARGE भट्टश्रीनीलकठकृतः}\\

\vspace{5mm}
\textbf{\huge व्यवहारमयूखः~।}\\

\vspace*{\fill}
\onehalfspacing
\rule{0.4\linewidth}{0.5pt}\\
\vspace*{\fill}
\onehalfspacing

{\large एम्. ए., एल्एल्. एम्., इत्युपपदधारिणा}\\

\vspace{7mm}
\textbf{\Large काणेकुलसमुद्भवेन वामनसूनुना पाण्डुरङ्गशर्मणा}\\

\vspace{5mm}
\textbf{\Large पाठान्तरप्रदर्शनपूर्वकं संशोध्याङ्ग्लभाषाटिप्पन्या परिष्कृतः}\\

\vspace*{\fill}
\onehalfspacing
\rule{0.4\linewidth}{0.5pt}\\
\vspace*{\fill}
\onehalfspacing

\textbf{पुण्यपत्तनस्थ\textendash\ }\\

\vspace{7mm}
\textbf{\LARGE प्राच्यविद्यासंशोधनमन्दिराधिकृतैः}\\

\vspace{7mm}
\textbf{मुम्बय्यां निर्णयसागरमुद्रणालये मुद्रयित्वा प्राकाश्यं नीतः}\\

\vspace{7mm}
\rule{0.4\linewidth}{0.5pt}\\

\vspace{7mm}
\textbf{शाके १८४७, ख्रिस्ताब्दे १९२६}\\

\vspace{7mm}
\rule{0.4\linewidth}{0.5pt}

\vspace{7mm}
\textbf{\Large मूल्यं दश रूप्यकाः~।}
\end{center}

\newpage
\thispagestyle{empty}
\begin{center}
\textbf{\LARGE PREFACE.}\\

\vspace{4mm}
\rule{0.3\linewidth}{0.5pt}
\end{center}

This edition of the Vyavahāramayūkha was entrusted to me by the late Prof. S. R. Bhandarkar. I am very sorry that the edition took so many years. But, owing to various causes over many of which I had no control, I could not finish the work quickly. At one time I had almost made up my mind to give up the undertaking altogether. For one reason I do not regret the long delay that has occurred. The years that I spent in collecting materials have been of great help to me in making the annotations exhaustive and have also induced me to undertake another work, viz. the history of Dharmaśāstra Literature.

I am under a deep debt of gratitude to several friends for help in various directions. I must make special mention of Dr. S. K. Belvalkar, Prof. H. D. Velankar of the Wilson College, Bombay, Mr. V. C. Koparkar of Nagpur and Mr. D. K. Karandikar of Dapoli.

\begin{center}
\rule{0.2\linewidth}{0.5pt}
\end{center}

\newpage
\thispagestyle{empty}
\begin{center}
\textbf{\LARGE TABLE OF CONTENTS.}\\
\rule{0.5\linewidth}{0.5pt}
\end{center}

\noindent {\small 
\begin{tabular}{p{7cm} p{0.5cm} p{3cm}r}
Introduction~~~~~~\ldots~~~~\ldots~~~~~~\ldots~~~~~~\ldots & pp. & I\textendash\ XLVII\\
Critical apparatus~~~~~~~\ldots~~~~~~\ldots~~~~~~\ldots & ,, & I\textendash\ IV\\
Family and personal history of Nīlakaṇṭha~~~~~~~~~~~~~\ldots~~~~~\ldots~~~~~~\ldots~~~~~~\ldots& ,,& V\textendash\ XVI\\
The works of Nīlakaṇṭha~~~~~~~\ldots~~~~~~\ldots& ,,& XVII\textendash\ XXIV\\
Period of Nīlakaṇṭha's literary Activity &,,& XXV\textendash\ XXVII \\
The contents of the twelve Mayūkhas &,,& XXVIII\textendash\ XXXIV \\
The position of Nīlakaṇṭha in Dharmaśāstra Literature~~~~~~~~\ldots~~~~~~\ldots~~~~~~\ldots~~~~~~\ldots& ,,& XXXV\textendash\ XXXVII\\
Nīlakaṇṭha and other writers on Vyavahāra& ,, & XXXVIII\textendash\ XL\\
The Position of Vyavahāramayūkha in modern Hindu Law~~\ldots~~~~~~\ldots~~~~~~\ldots~~~~~~\ldots& ,,& XLI\textendash\ XLV\\
The present edition~~~~~~~\ldots~~~~~~\ldots~~~~~~\ldots& ,,& XLVI\textendash\ XLVII\\
Analysis of the contents of the Text~~~~~\ldots&,, &XLIX\textendash\ LIX \\
Errata~~~~~~~~~~~~~\ldots~~~~~~\ldots~~~~~~\ldots~~~~~~\ldots& ,,& LX\\
Text~~~~~~~~~~~~~~~\ldots~~~~~~\ldots~~~~~~\ldots~~~~~~\ldots& ,, &1\textendash\ 256\\
Notes~~~~~~~~~~~~~~\ldots~~~~~~\ldots~~~~~~\ldots~~~~~~\ldots& ,,& 1\textendash\ 440\\
Appendix A (Text of Vyavahāratattva)~~~~~~\ldots &,,& 441\textendash\ 473 \\
Appendix B (Information about authors and works quoted in the work)~~~~~~~~\ldots~~~~~~\ldots & ,,& 475\textendash\ 486\\
Appendix C (List of works and authors quoted in the twelve Mayūkhas)~~~~~~\ldots~~~~~~\ldots &,,& 487\textendash\ 507\\
Appendix D (Mitākṣarā passages expressly criticized or quoted )~~~~~~~\ldots~~~~~~\ldots~~~~~~\ldots &,,& 509\textendash\ 511\\
Appendix E (Madanaratna passages quoted or criticized )~~~~~~~~\ldots~~~~~~\ldots~~~~~~\ldots~~~~~~\ldots& ,,& 513\textendash\ 515\\
Appendix F (Pūrvamīmāmsā doctrines referred to)~~~~~~~~~~~~~~~\ldots~~~~~~\ldots~~~~~~\ldots~~~~~~\ldots& ,,& 517\textendash\ 518\\
Appendix G (Index of quotations)~~\ldots~~~~~~\ldots& ,,& 519\textendash\ 539\\
General Index~~~~~~~~\ldots~~~~~~\ldots~~~~~~\ldots~~~~~~\ldots& ,,& 541\textendash\ 560
\end{tabular}}

\newpage
\thispagestyle{empty}
\begin{center}
\textbf{\Large List of abbreviations and of some of the works relied upon in this edition.}\\

\rule{0.4\linewidth}{0.5pt}
\end{center}

\vspace{-6mm}
\noindent
{\small B.G. $=$ Bombay Gazetteer volumes.\\
B.I.$=$Bibliotheca Indica (edition of a work)\\
Bik.$=$ Bikaner (catalogue of mss. at).\\
Bom.L.R.$=$ Bombay Law Reporter.\\
Cat.$=$ Catalogue. \\
E.I.$=$ Epigraphia Indica.\\
I.A.$=$ Indian Antiquary.\\
I.L.R.$=$ Indian Law Reports.\\
I.O.Cat.$=$ Catalogue of mss. at the India office in London\\
J.B.B.R.A.S.$=$Journal of the Bombay Branch Royal Asiatic Society.\\
L.R.I.A.$=$ Law Reports, Indian Appeals.\\
Moo.I.A.$=$ Moore's Indian Appeals.\\
Nirn.$=$ Nirṇaya-sāgar edition (of a work).\\
S.B.E.$=$ Sacred Books of the East (series, edited by Prof. Max Múller).\\
अग्निपुराण\textendash\ (Ānandāśrama edition, Poona).\\
अपरार्क\textendash\ (Ānandāśrama edition, Poona).\\
आप.ध.सू.$=$ आपस्तम्बधर्मसूत्र (Būhler's edition of 1868).\\
आप.गृ.सू.$=$आपस्तम्बगृह्यसूत्र (Mysore Govt. Bibliographica Sanskritica No.1).\\
आप.श्रौ.सू.$=$ आपसम्बश्रौतसूत्र (B.I. edition).\\
आश्व.गृ.सू. $=$ आश्वलायनगृह्यसूत्र.\\
आश्व.श्रौ.सू.$=$ आश्वलायनश्रौतसूत्र (Ānandāśrama edition).\\
ऋ.$=$ ऋग्वेद.\\
का.श्रौ.सू $=$ कात्यायनश्रौतसूत्र (Weber's edition).\\
गौ.ध.सूः $=$ गौतमधर्मसूत्र (Ānandāśrama edition).\\
चतुर्वर्ग\textendash\ चतुर्वर्गचिन्तामणि of Hemādri(Bibliotheca Indica edition)\\
चतुर्विंशतिमतसंग्रह\textendash\ Benares Sanskrit (Pandit) series.\\
तन्त्रवार्तिक\textendash\ Benares Sanskrit (Pandit) series.\\
तै.सं.$=$ तैत्तिरीयसंहिता.}

\fancyhead[CE,CO]{List of Abbreviations.}
\fancyhead[RE,LO]{\thepage}
\cfoot{}
\newpage
\renewcommand{\thepage}{\arabic{page}}
\setcounter{page}{2}

% List of Abbreviations. 2

\noindent
{\small दाय$=$ दायभाग of जीमूतवाहन(edition of 1829 with the commentary of श्रीकृष्णतर्कालंकार ).\\
द्वैतनिर्णय\textendash\ A ms. from the Deccan College Collection.\\
नारद\textendash\ नारदस्मृति (edited by Dr. Jolly).\\
निर्णय\textendash\ निर्णयसिन्धु of कमलाकरभट्ट (Nirn. edition).\\
न्यायसुधा\textendash\ Commentary on the तन्त्रवार्तिक (Chaukhamba Sanskrit series).\\
परा.मा.$=$ पराशरमाधवीय ( Bombay Sanskrit Series ).\\
पा.$=$ The अष्टाध्यायी of पाणिनि.\\
पू.मी.सू.$=$ पूर्वमीमांसासूत्र of जैमिनि (B. I. edition, with the भाष्य of शबर).\\
प्र.$=$ प्रकरण.\\
बौ.ध.सू.$=$ बौधायनधर्मसूत्र ( Mysore Govt. Sanskrit Series)\\
मद.पा.$=$ मदनपारिजात (B.I. edition).\\
मनु.$=$ मनुस्मृति (Nirn.edition).\\
महा$=$ महाभारत (Bombay edition).\\
मिता.$=$ मिताक्षरा, commentary on the याज्ञवल्क्यस्मृति ( ed. by Mr. Gharpure).\\
मी.परि.$=$ मीमांसापरिभाषा (Nirn. edition)\\
मेधा$=$ मेधातिथि's भाष्य on the मनुस्मृति (Mandlik's edition).\\
याज्ञ$=$ The याज्ञवल्क्यस्मृति (ed. by Mr. Gharpure).\\
व.ध.सू.$=$ वसिष्ठधर्मसूत्र (Bombay Sanskrit Series).\\
वि.चि.$=$ विवादचिन्तामणि.\\
वि.ध.सू. or विष्णु.ध.सू$=$ विष्णुधर्मसूत्र (edition by Dr. Jolly)\\
वि.र.$=$ विवादरत्नाकर ( B. I. edition).\\
विश्वरूप $=$ Trivandrum edition of the याज्ञवल्क्यस्मृति.\\
वीर$=$ वीरमित्रोदय (Jivānanda's edition of 1875 of the व्यवहार portion).\\
व्य.$=$ व्यवहार\\
व्यव.मा. or व्य.मा. or व्यवहारमा$=$ व्यवहारमातृका of जीमूतवाहन\\
सरस्वती$=$ सरस्वतीविलास (edited by Foulkes).\\
सि.कौ $=$ सिद्धान्तकौमुदी of भट्टोजिदीक्षित ( Nirn. edition).\\
सुबोधिनी$=$ edited by Mr. Gharpure.\\
स्मृतिच$=$ स्मृतिचन्द्रिका of देवण्णभट्ट (edited by Mr. Gharpure).}

\begin{center}
\rule{0.2\linewidth}{0.5pt}
\end{center}

\newpage
\thispagestyle{empty}
\begin{center}
\textbf{\LARGE INTRODUCTION}\\

\rule{0.4\linewidth}{0.5pt}\\

\vspace{1mm}
\textbf{\large I.}\\

\vspace{1mm}
\textbf{\large Critical Apparatus}\\

\rule{0.4\linewidth}{0.5pt}
\end{center}

The present edition of the Vyavahāramayūkha of Nīlakaṇṭha is based on the following edition; and manuscripts:\textendash

\hangindent=1.5cm (A.) The oblong lithographed edition of 1826 published at Bombay by {\qt Shreecrustna Jagannathjee} under the patronage of the Government of Bombay and printed at the Courier Press. This edition is, for the time when it was published, a very accurate one. There are a few misprints and mistakes. It does not say what mss. were consulted and no various readings are given. It gives references to editions of the Manusmṛti and Yājñavalkyasmṛti that were published before it. At the end there is a table of contents and a list of errata is given at the beginning. This edition contains 244 Pages with eight lines on each page.

\hangindent=1.5cm (B.) This is a paper ms. belonging to the Deccan College Collection, No. 67 of 1879\textendash\ 80, written on 73 folios, having 16 lines on each Page up to folio 32 and 12\textendash\ 15 thereafter. There is no date at the beginning or at the end. It looks to be about 100 years old. The handwriting is not good. Red vertical double lines are used to indicate quotations.

\hangindent=1.5cm (C.) This ms. is No. 120 of the Vis'rambag Collection (i) written on 85 folios. There are generally eleven

\fancyhead[CE]{INTRODUCTION TO VYAVAHĀRAMAYŪKHA}
\fancyhead[CO]{CRITICAL APPARATUS}
\fancyhead[RE,LO]{\thepage}
\cfoot{}
\newpage
%%%%%%%%%%%%%%%%%%%%%%%%%%%%%%%%%%%%%%%%%%%%%%%%%%%%%
\renewcommand{\thepage}{\Roman{page}}
\setcounter{page}{2}

% INTRODUCTION TO VYAVAHĀRAMAYŪKHA II 

\hangindent=1.5cm \indent \hspace{0.8cm} lines on each page. It is written very carelessly, though in a good hand. There is no date at the beginning or at the end. The ms. appears to be a hundred years old. There are many omissions of words and lines through oversight.

\hangindent=1.5cm (D.) This ms. is No. 121 of the Viśrambag collection (i). There are 100 leaves with 10 or 11 lines on each page. It is written in a clear bold hand, but rather carelessly. The copyist was probably altogether ignorant of Sanskrit and wrote to dictation. The colophon at the end shows that it was copied in saṁvat 1820 i. e. 1764 A. D.

\hangindent=1.5cm (E.) This ms. is No 296 of the Viśrambag collection (ii).It is incomplete and contains 98 folios; out of which 1, 5\textendash\ 34, 48 and 85\textendash\ 94 are wanting. The writer was an illiterate and careless scribe, though he wrote a good hand. This ms. omits very frequently words and sentences through oversight.

\hangindent=1.5cm (F.) This is a ms. belonging to the Bhau Daji collection of the Bombay Branch of the Royal Asiatic society. It contains 91 folios with 9 or 10 lines on each page. Is is well written and is tolerably correct, but frequently omits words and even lines. Corrections are made in a smaller and more beautiful hand, probably by another scribe. The original readings of F agree remarkably with B and D, but the corrections make it differ from them. In a few cases whole pages are omitted, though the ms. itself presents consecutively numbered pages.

\hangindent=1.5cm (G.) A ms. from the Library of the Calcutta Sanskrit College, containing 95 folios with 12 lines (sometimes only 8 or 10) on each page. It is written in

\newpage
% III CRITICAL APPARATUS

\hangindent=1.5cm \indent \hspace{0.8cm} in another ink but probably by the same hand. Two folios, 44\textendash\ 45, are missing, though on the first page it is described as complete. From folio 80 there is confusion. Probably the leaves of the original were carried off by the wind when the scribe was copying. He collected the leaves together but changed their order and copied down the leaves so shuffled up. The ms. looks modern and must not be more than 100 years old.

\hangindent=1.5cm (H.) A ms. from the Library of the Calcutta Sanskrit College in Bengali characters containing 78 folios with 8 or 9 lines on each page. Though described as complete on the title page, it stops at the title called स्त्रीसंग्रहण. This ms. is very incorrect and full of lacuna, very often due to the fact that the scribe's eye ran from one word to the same word occurring a few lines later. The ms. is modern, about 50 years old.

\hangindent=1.5cm (K.) This is the Benares lithographed edition of 1879 printed at the Kāśī\textendash\ Sanskṛta Yantrālaya. This edition often confounds the letters प and य, ब and व, त and न. There are numerous mistakes arising from the inability to read correctly the original from which this edition was printed. This edition does not give various readings and was probably based upon a single ms. This edition agrees remarkably with A, C and G, particularly with C even in the matter of omissions.

\hangindent=1.5cm (M.) This is the edition of the late Raosaheb V.N.Mandlik published in 1879 containing the text, translation and critical notes. This is a scholarly edition. It is based on six mss. and two printed editions. This edition is not now available in the

\newpage
% INTRODUCTION TO VYAVAHĀRAMAYŪKHA IV 

\hangindent=1.5cm \indent \hspace{0.8cm} market. It gives in the footnotes various readings and also references to some of the works quoted or referred to in the text.

\hangindent=1.5cm (N.) This is a ms. belonging to the library of Srimant Raje Lakshmanrao Saheb Bhonsle of Nagpur (junior ). It is well written and is tolerably correct. It has 136 folios with fourteen lines on each page. It is about a hundred years old. It is full of omissions. From the section on स्त्रीसंग्रहण, a great confusion is visible. Probably the leaves of the original were blown away by the wind when the scribe was copying. The leaves were collected without any attempt at arranging them in consecutive order. 

It will be seen from the above that mss. belonging to different parts of India have been utilised in preparing this edition. Among the mss. B, D, E, and F agree very closely, even in their mistakes and are probably copies of the same codex archetypus. C sometimes agrees with B D F and sometimes with G. C G and K show a remarkable agreement even in omissions. H is akin to G. M very often follows A. N seems to be an independent ms, though it generally presents the same readings as C and K and sometimes agrees with A and M. In the footnotes all important readings have been collected, only very palpable mistakes of copyists being generally omitted. Even such mistakes sometimes be found in tha footnotes purposely given for the sake of comparison.

The Vyavahāramayūkha quotes very largely from the Manusmṛti, the Yājñavalkya\textendash\ smṛti, the Nārada\textendash\ smṛti and other smṛti works. In the footnotes important variations from the Printed editions of these works have been pointed out.

\begin{center}
\rule{0.2\linewidth}{0.5pt}
\end{center}

\fancyhead[CE]{INTRODUCTION TO VYAVAHĀRAMAYŪKHA}
\fancyhead[CO]{FAMILY OF NĪLAKAṆṬHA}
\fancyhead[RE,LO]{\thepage}
\cfoot{}
\newpage
%%%%%%%%%%%%%%%%%%%%%%%%%%%%%%%%%%%%%%%%%%%%%%%%%%%%%
\renewcommand{\thepage}{\Roman{page}}
\setcounter{page}{5}

% V FAMILY OF NĪLAKAṆṬHA

\begin{center}
\textbf{\large II}\\

\vspace{1mm}
\textbf{\large The family and personal history of Nīlakaṇṭha.}\\

\rule{0.2\linewidth}{0.5pt}
\end{center}

For several generations the family of which Nīlakaṇṭha was a worthy scion held the first place among learned men in that ancient and far\textendash\ famed seat of Sanskrit learning, the city of Benares. The Pūrvamīmāñsā system and religious and ceremonial lore were the special forte of this family. Although biographies of learned men are very rare in India, as regards this family the case is somewhat different. Mahāmahopādhyāya Haraprasad Śāstri has brought to light a biography of this family written by a distinguished member of the family, Śaṁkarabhaṭṭa, son of Nārāyaṇabhaṭṭa and father of Nīlakaṇṭha (vide Indian Antiquary for 1912 vol. 41. pp 7\textendash\ 13 ).

Unfortunately the copy supplied to the Mahāmahopadhyāya does not contain the first folio and the work, which is full of inaccuracies and omissions, comes abruptly to an end. The last chapter shows that Śaṁkarabhaṭṭa, who was a very old man then, was weighed down with grief for the loss of a promising nephew. The work is styled Gādhivaṁśānucarita from the fact that the gotra of the family was Viśvāmitra. The family migrated to Benares from the Deccan. According to tradition the home of the family was in the ancient and famous town of Paithan. The first member of the family, of whom some notices are preserved in works that were beyond doubt composed by the members of the family, was Govinda\renewcommand{\thefootnote}{1}\footnote{श्रीमद्दक्षिणदेशेगस्त्य इवासीत्स भट्टगोविन्दः~। तत्सूनुः श्रीरामेश्वरभट्टोभूत्स सर्वदिक्ख्यातः~॥ Introduction to ज्योतिष्टोमपद्धति of रामकृष्ण. But it has to be noted that in the commentary on the वृत्तरत्नाकर composed by नारायणभट्ट, two more ancestors are mentioned. {\qt भट्टः श्रीनागपाशात् समजनि विबुधश्चाङ्गदेवाख्यभट्टः प्रासोष्ठासौ तनूजं रघुपतिनिरतं भट्टगोविन्दसंज्ञम्~। विश्वामित्रान्ववायाम्बुधिविधुरधिकं वर्धते तत्तनूजो विद्याब्धौ लब्धपारः प्रथितपृथुयशा भट्टरामेश्वराख्यः~॥} I.O.Cat. part II p. 303. I take चाङ्गदेव as the name and not अङ्गदेव as some do. चाङ्गदेव was a famous name in Mahārāṣtra.}. As the first folio of the


\newpage
% INTRODUCTION TO VYAVAHĀRAMAYŪKHA VI 

\noindent
Gādhivaṁśānucarita is not available, information about the founder of the family and its early fortunes is not forthcoming from that work. In the Tristhalīsetu of Nārāyaṇabhaṭṭa, the author refers to his ancestor Govinda and informs us that the gotra of the family was Viśvāmitra\renewcommand{\thefootnote}{1}\footnote{विश्वामित्रकुलोदधौ विधुरिवाखण्डः कलानां निधिर्वाग्गुम्फे निखिलेऽपि यस्य वसुधा शिष्यप्रशिष्यैश्चिता~। विद्यापद्मविकासनैकतरणिः श्रीभट्टगोविन्दजः संख्यावद्गणनाग्रणीर्विजयते श्रीभट्टरामेश्वरः~॥ त्रिस्थलीसेतु (mss. Deccan College No. 104 of 1892\textendash\ 95 and Viśrambag i. No. 149).}. Rāmeśvara was the son of Govinda. The copy of the Gādhivaṁśānucarita opens (on its second page) with a panegyric of Rāmeśvarabhaṭṭa. He is said to have been very strong in Mīmāñsā, grammar, logic and in philosophy. He wrote a poem styled Rāmakutūhala in order to eclipse the fame of the Naiṣadhīya of Śrīharṣa. Numerous pupils flocked to him at Paithan on the Godāvarī. He is said to have cured of leprosy the son of an infuential Mahomedan officer of the Ahmednagar state. He went to Kolhapur and thence to Vijayanagar, which was then ruled over by the famous Kṛṣṇarāya. He then started on a pilgrimage to Dvārakā. On his way to Dvārakā a son was born to him in śake 1435 caitra i.e. March 1513 A. D. This son later on became famous as Nārāyaṇabhaṭṭa. Rāmeśvara, after staying for four years at Dvārakā, came back to Paithan. After spending four more years at Paithan, Rāmeśvarabhaṭṭa left for Benares\renewcommand{\thefootnote}{2}\footnote{The Introduction to the त्रिस्थलीसेतु bears this out, श्रिता वाराणसी तेन नगरी न गरीयसी~। यतोन्वा नगरी हेमनगरीतिमुपेयुषी~॥}. A second son named Śrīdhara was born on the way and a third named Mādhava at Kāśī\renewcommand{\thefootnote}{3}\footnote{यस्याचिता ब्रह्महरीशकल्पैः स्वसूनुभिर्यश्च तुरीयरूपः~॥ नारायणश्रीधरमाधवाख्यै रामेश्वरः सोजनि गाधिवंशे~। vide I.O. Cat p.531, Nos 1667\textendash\ 68 कालतत्त्वविवेचन of रघुनाथ, son of माधव.}. Rāmeśvara was advanced in age when his first son Nārāyaṇabhaṭṭa was born. So he must have been quite an old 

\newpage
% VII Family of NĪLAKAṆṬHA 

\noindent
man when he came to Benares. For some of his famous pupils; vide Indian Antiquary for 1912, p 9. Students from all parts of India came in crowds to Benares to learn at his feet and spread his fame throughout the length and breadth of India. Rāmeśvara died at a very advanced age and his wife became a satī.

Nārāyaṇabhaṭṭa learnt all the śāstras at the feet of his father. He is said to have engaged in constant disputations with the pandits of Eastern India. He vanquished Maithila and Gauda pandits at the house of Todarmal. It was he who raised Dākṣinātya pandits to that position of high eminence which they hold even now at Benares. He was the most illustrious member of his family and shed lustre on it by his giant intellect, his holiness and his ceaseless efforts in the cause of Sanskrit literature. Pandits all over India looked upon him as their patron and he spared neither money nor pains to help them. He was very fond of collecting and copying manuscripts. It is related that, when the Mussalmans razed the temple of Viśveśvara at Benares to the ground from religious bigotry and hatred, there was severe drought for a long time and that the Mahomedan ruler implored Nārāyaṇabhaṭṭa to propitiate Viśveśvara. Nārāyaṇabhaṭṭa propitiated Viśveśvara and copious rain fell in a day. Thereupon Nārāryaṇabhaṭṭa induced the Mahomedan ruler to allow him to rebuild the temple of Viśveśvara. For his piety and learning Nārāryaṇabhaṭṭa was given the title of {\qt Jagadguru} and his family was given the first place of honour in the assembly of learned Brāhmaṇas and at the recitations of the Vedas (mantrajāgaras). The latter distinction continues in the family, it is said,

\newpage
% INTRODUCTION TO VYAVAHĀRAMAYŪKHA VIII

\noindent
even now. That Nārāyaṇabhaṭṭa was concerned with the rebuilding of the temple of Visveśvara is vouched for by Divākarabhaṭṭa, a daughter's son of Nīlakaṇṭha, who was the grand\textendash\ son of Nārāyaṇabhaṭṭa\renewcommand{\thefootnote}{1}\footnote{श्रीरामेश्वरसूरिसूनुरभवन्नारायणाख्यो महान्येनाकार्यविमुक्तके सुविधिना विश्वेश्वरस्थापना~। Introductory verse 4 to the दानहीरावलिप्रकाश. Vide I.O.Cat. part III p. 547, No. 1708.}. But it is rather strange that the Gādhivaṁśānucarita is silent on this point (I. A. vol. 41 at p. 10). In the colophons to the several works of his descendants, Nārāyaṇabhaṭṭa is frequently styled {\qt Jagadguru}.\renewcommand{\thefootnote}{2}\footnote{e.g. {\qt श्रीमज्जगद्गुरुमीमांसकनारायणभट्टसूरिसूनुरामकृष्णभट्टात्मजदिनकरभट्टानुजकमला\textendash\ करभट् टकृते शूद्रधर्मतत्त्वे} \& c. Aufrecht's cat. of Sanskrit mss. in the Bodleian Library p. 277, No. 654.} Nārāyaṇabhaṭṭa wrote the Prayogaratna, the Tristhalīsetu, the Antyeṣṭipaddhati, Rudrapaddhati, Divyānuṣṭhānapaddhati\renewcommand{\thefootnote}{3}\footnote{It is probably to this work that नीलकण्ठ in his व्यवहारतत्त्व refers in the words {\qt एतदनुष्ठानपद्धतिस्तु भट्टकृता द्रष्टव्या} p.457.}, and numerous other works. He wrote a commentary on the Vṛttaratnākara in the year 1602 of Vikramārka i. e. 1546 A.D.\renewcommand{\thefootnote}{4}\footnote{{\qt याति विक्रमशके द्विखषड्भूसंमिते सितगकार्तिकरुद्रे~। ग्रन्थपूर्तिसुकृतं किल कुर्मो रामचन्द्रपदपूज (न?) पुष्पम्~॥} I. O. Cat. part II p. 304.} His works are even now used all over India and regulate the performance of religious ceremonial in modern times. His descendants speak of him as almost an avatāra\renewcommand{\thefootnote}{5}\footnote{वेदार्थधर्मरक्षायै मायामानुषरूपिणम्~। पितामहं हरिं वन्दे भट्टनारायणाह्वयम्~॥ Introductory 3rd verse to the निर्णयसिन्धु of कमलाकरभट्ट (Nirn. ed.).} of the Deity and as a prafound Mīmāṁsaka\renewcommand{\thefootnote}{6}\footnote{मीमांसासरसीसरोजमकरन्दास्वादनैकव्रतो हंसःस्वीययशःसिताद्वयमतेरस्तान्यरूपाभिधः~। वाग्देव्या गतमत्सरां श्रियमयं नित्यानुरक्तां भजन्नो मुञ्चन्नविमुक्तं विजयते श्रीभट्टनारायणः~॥ Intro. 4th verse to the द्वैतनिर्णय of शंकरभट्ट (No. 109 of the Deccan College collection of 1895\textendash\ 1902).}. He appears to have composed a commentary on the Śāstradīpikā of Pārthasārathimis'ra,

\newpage
% IX FAMILY OF NĪLAKAṆṬHA 

\noindent
as his son Śaṁkarabhaṭṭa informs us\renewcommand{\thefootnote}{1}\footnote{My friend, Pandit Bakres'āstri of Bombay, has a copy of the comment of शंकरभट्ट on the शास्त्रदीपिका. While commenting on the first अध्याय, शंकरभट्ट says {\qt प्रागङ्घ्रौ गुरुचरणैर्व्याख्या संख्यावतां तु विख्याता~। विहिता हिता ततस्तं विहाय विवृणोमि संवृतं भावम्~॥} At the end of the 6th अध्याय we have these words {\qt प्रागङ्घ्रियुगले टीका कृता गुरुभिरेव यतः~। लक्षणेत्रावशिष्टा षट्पादी व्याख्यायते मया~॥} This shows tha नारायणभट्ट commented on the first pāda of the first अध्याय and the first two pādas of the sixth अध्याय ( of the पूर्वमीमांसासूत्र ).}. As he was born in 1513 and wrote a work in 1546 A.D. the literary activity of Nārāyaṇabhaṭṭa must be ascribed to the period between 1540 A.D. and 1580 A.D.

Nārāyaṇbhaṭṭa had three sons, Rāmakṛṣṇabhaṭṭa, Śaṁkarabhaṭṭa and Govindabhaṭṭa, the first being the eldest\renewcommand{\thefootnote}{2}\footnote{In the प्रयोगरत्न (Nirn. ed. ) we read in one place इति श्री \ldots प्रयोगरत्ने तज्ज्येष्ठसुतरामकृष्णोन्नीता दुष्टरजोदर्शनशान्तिः समाप्ता.}. Rāmakṛṣṇa also was a very learned man. He is spoken of as a helmsman in the deep ocean of the phiosophy of the Bhāṭṭa ( Kumārilabhaṭṭa) school and as unravelling the knotty points in other Śāstras also and as having made his opponents look like glow\textendash\ worms in the brilliance of his lore\renewcommand{\thefootnote}{3}\footnote{यो भाट्टतन्त्रगहनार्णवकर्णधारः शास्त्रान्तरेषु निखिलेष्वपि मर्मभेत्ता~। योऽत्र श्रमः किल कृतः कमलाकरेण प्रीतोमुनास्तु सुकृती बुधरामकृष्णः~॥ Introductory verse to the शूद्रकमलाकरः विद्याप्रद्योतनोद्धोतखद्योतीकृतवादिनम्~। पितरं रामकृष्णाख्यं वन्दे स्नेहमरान्वितम्~॥}. He wrote a commentary on the Tantravārtika, the Jīvatpitṛka\textendash\ kartavya\textendash\ nirṇaya, the Jyotiṣṭomapaddhati, the Māsikas'rāddhanirṇaya and other works. The Gādhivaṁs'ānucarita says that he died at the age of 52.

Śaṁkarabhaṭṭa, the second son of Nārāyaṇabhaṭṭa, was a profound Mīmāñsaka. He wrote a commentary on the Śāstradīpikā, to which frequent reference is made in his own work called the Dvaitanirṇaya and in the

\newpage
% INTRODUCTION TO VYAVAHĀRAMAYŪKHA X

\noindent
Saṁskāramayūkha, where it is styled Śāstradīpikāprakās'a. For an account of his Dvaitanirṇaya, the Annals of the Bhandarkar Institute (vol. III, part 2, pp. 67\textendash\ 72) may be consulted. In this latter work, distinctly states that he will conform to the views of southern writers\renewcommand{\thefootnote}{1}\footnote{दाक्षिणात्ये स्थित्वा धर्मद्वैतेषु निर्णयम्~। तनुतेसौ विगाह्यैव नावमान्यः कथंचन~॥ 8th intro. verse in the ms. of the द्रैतनिर्णय (No. 109 of 1895\textendash\ 1902 of the Deccan College Collection).}. He wrote a work called Mīmāñsābālaprakās'a (printed at Benares), in which he summarises the conclusions established in the twelve chapters of the Pūrvamīmāñsāsūtra. Another work of his is the Dharmaprakās'a or Sarvadharmaprakās'a, in which his mother's name is given as Pārvatī\renewcommand{\thefootnote}{2}\footnote{संपूज्यः प्रङ्मुखतः श्रीहरिमुदधिसुतां देवतां चैव वाचं स्वां विद्यां सूत्रकारान्गुरुमथ सकलान्धर्मशास्त्रप्रवक्तॄन्~। श्रीमन्नारायणाह्वं गुरुमथ जननीं पार्वतीं शंकरः श्रीमीमांसान्यायसारं शिवपुरि तनुते सर्वधर्मप्रकाशम्~॥ I.O. Cat. part III p. 482 No. 1564.} and in which he refers to his Śāstradīpikāprakās'a. Some of his other works are Vidhirasāyanadūṣaṇa, in which he refutes the Vidhirasāyana of Appayyadīkṣita, the Nirṇayacandrikā, Vratamayūkha. Bhaṭṭoji Dīkṣita, author of the Siddhānta\textendash\ kaumudī, was the most famous of his pupils.

The third son of Nārāyaṇabhaṭṭa was Govinda who died at the age of 48, leaving four sons (vide I. A. vol.41 at p. 11 ).

Rāmakṛṣṇa, the son of Nārāyaṇa, had three sons, Dinakara alias Divākara, Kamalākara and Lakṣmaṇa. The oldest of these was Dinakara\renewcommand{\thefootnote}{3}\footnote{विन्दुमाधवपादाब्जरोलम्बीकृतविग्रहम्~। ज्यायांसं भ्रातरं भट्टदिवाकरमुपास्महे~॥ 6th intro. verse to the निर्णयसिन्धु; येनोद्धृता स्वस्य कुलप्रतिष्ठा महावराहेण महीव तोयात्~। गङ्गेव विद्याभिससार यस्माद्दिवाकरं नौमि तमग्रजाग्र्यम्~॥ 6th intro. verse to the आचाररत्न of लक्ष्मणभट्ट.} and Lakṣmaṇa was the youngest\renewcommand{\thefootnote}{4}\footnote{अधीत्य लक्ष्मणाख्येन कमलाकरसोदरात्~। आचारत्नं सुधिया यथामति वितन्यते~॥ 7th intro. verse, आचाररत्न.}. Their mother's name was Umā and she

\newpage
% XI FAMILY OF ~NĪLAKAṆṬHA 

\noindent
seems to have immolated herself on the funeral pyre of her husband. The sons offer most touching reverence to their mother in their works. Dinakara alias Divākara wrote the Bhāṭṭadinakarī or Bhātṭadinakaramīmāñsā which is a commentary on the Śāstradīpikā, the S'āntisāra, the Dinakaroddyota. This latter was a comprehensive digest, commenced by Dinakara and completed by his son Vis'ves'vara or Gāgābhaṭṭa. Kamalākarabhaṭṭa wrote no less than twenty\textendash\ two works. Next to Nārāyaṇabhaṭṭa, Kamalākara and his cousin Nīlakaṇṭha stand out as the most prominent and far\textendash\ famed scions of this family of Bhaṭṭas. In some of Kamalākara's works such as the Śāntikamalākara and the commentary on the Kāvyaprakās'a verses occur highly eulogising his proficience in all the s'āstras. He tells us that he composed his commentary on the Kāvyaprakāśa for the diversion of his son Ananta. He composed his Nirṇayasindhu in the year 1668 of the Vikrama era. i. e. in 1612 A. D. We learn from another source that this was his first work. Therefore his literary activity must have fallen between 1610 A.D.

% \footnote text missing 

\newpage
% INTRODUCTION TO VYAVAHĀRAMAYŪKHA XII

\noindent
and 1640 A. D. Some of his important works are the Nirṇayasindhu, the S'ūdrakamalākara, the Vivādatāṇḍava, tho Sāntikamalākara, the Vratakamalākara, the Pūrtakamalākara and the commentary on the Kāvyaprakās'a. For a complete list vide the foot\textendash\ note\renewcommand{\thefootnote}{1}\footnote{In the विवादताण्डव he says at the end (I. 0. Cat. part. III, p. 455 No.1502) {\qt येनाकारि प्रोद्भटा वार्तिकस्य टीका वान्या (चान्या?) विंशतिग्रन्थमाला~। श्रीरामाङ्घ्रयोरर्पिता निर्णयेषु सिन्धुः शास्त्रे तत्त्वकौतूहले च~॥}; at the end of the शान्तिरत्न (also called शान्तिकमलाकर) after the verse तर्के दुस्तर्कमेधः there is a list of 22 works आदौ निर्णयसिन्धुस्तु वार्तिके टिप्पणी पुनः~। काव्यप्रकाशगा व्याख्या दाने च कमलाकरः~॥ शान्तिरत्नं ततः पूर्तव्रतयोः कमलाकरः~। ग्रन्थो वेदान्तरत्नं च सभादर्शकुतूहलम्~। प्रायश्चित्ते रत्नमेकं व्यवहारे तथापरम्~। बद्धवाह्निक (?) मन्यच्च गोत्रप्रवरदर्पणः~। रत्नं कर्मविपाकाख्यं कार्तवीर्यस्य पद्धतिः~। सोमप्रयोगः शूद्राणां धर्मो रुद्रस्य पद्धतिः~। टिप्पणी च तथा शास्त्रदीपिकालोकसंज्ञिता~। मीमासायां तथा शास्त्रतत्त्वस्य कमलाकरः~। सर्वतीर्थविधिश्चैव भक्तिरत्नं तथोत्तमम्~। रामकृष्णसुतेनेत्थं कमलाकरशर्मणा~। द्व्यधिका विंशतीनां च ग्रन्थानां रत्नमालिका~। सेवां कर्तुमशक्तेन ह्यर्पिता समपादयोः~॥}. The youngest of the three brothers, Lakṣmaṇa, studied under Kamalākara and wrote the Ācāraratna,the Gotrapravararatna and a few other works.

S'aṁkarabaṭṭa had four sons, Ranganātha, Dāmodara, Nṛsiṁha and Nīlakaṇṭha. Mandlik is not right in saying that S'aṁkarabhaṭṭa had two sons. In the Vyavahāratattva (vide appendix A) the colophon makes it clear that Nīlakaṇṭha was the younger brother of the first three mentioned above. Similarly in the colophon to the Nītimayūkha (Benares edition of 1880) Nīlakaṇṭha is described as the younger brother of the first three mentioned above. The Dvaitanirṇaya of S'aṁkarabhaṭṭa says that the author's son Dāmodara wrote a supplement to the Dvaitanirṇaya\renewcommand{\thefootnote}{2}\footnote{अस्मत्सुतदामोदरभट्टकृतेस्मत्कृतद्वैतनिर्णयस्य परिशिष्टे दत्ताक्षतायाः कन्यायाः पुनर्दानं परस्य चेति कलिनिषेधव्याख्याने उक्तम्. vide Annals of the Bhandarkar Institute, vol.III, part 2, p.72.}. In the Vyavahāratattva Nīlakaṇṭha refers to the work on matters forbidden in the

\newpage
% XIII FAMILY OF ~NĪLAKAṆṬHA

\noindent
Kali age composed by his eldest brother Dāmodara\renewcommand{\thefootnote}{1}\footnote{एषां कलिवर्ज्यावर्ज्यादिविवेकोस्मज्ज्येष्ठभ्रातृभट्टदामोदरकृतकलिवर्ज्यनिर्णयादवसेयः~। p.465}. In the Ācāramayūkha Nīlakaṇṭha refers to the Kalivarjyanirṇaya of his elder brother (bhrātṛcaraṇāḥ) and in the Prāyas'cittamayūkha to his eldest brother, without naming him. In the other Mayūkhas also (such as those on S'rāddha and Samaya ) references occur to an elder brother. It is dificult to reconcile the fact that Dāmodara is spoken of as the eldest brother in the Vyavahāratattva with the fact that Raṅganātha's name occurs before that of Dāmodara in the colophon to the same work. An explanation may be hazarded that Raṅganātha probably died early so that Dāmodara became the eldest or that Raṅganātha might have been given away in adoption. It is also possible that the colophon is not exact as to the seniority among the brothers. It is significant that the Gādhivaṁs'anucarita speaks of only Dāmodara, Nṛsiṁha and Nīlakaṇṭha. So it looks very probable that when S'aṁkarabhaṭṭa wrote the work in his old age, Raṅganātha had passed away. The works of Nīlakaṇtha will be dealt with separately later on.

Dinakara alias Divākara had a son Vis'ves'vara better known as Gāgābhaṭṭa. He officiated at the coronation of S'ivaji, the founder of the Maratha empire. Besides completing his father's digest, the Uddyota, he wrote the Bhāṭṭacintāmaṇi, the Mīmāñsākusumāñjali\renewcommand{\thefootnote}{2}\footnote{In the भाट्टचिन्तामणि (तर्कपाद p.88 Chowkhamba series) he says {\qt इदं गुरुमतं तु मद्विहितसूत्रवृत्तौ मया विचार्य कुसुमाञ्जलौ बहु च दूषितं भूषितम्~। अपेक्षितमथाप्युपेक्षितमिदं कठोरत्वतः शिशुप्रतिविबोधनार्थकृतभाट्टचिन्तामणौ~॥}}, the Kāyasthadharmadīpa and other works. His S'ivārkodaya is modelled on the lines of the S'lokavārtika of Kumārila. In the Kāyasthadharmadīpa reference is made to Aurangzeb, to Rājagiri (Raigad

\newpage
% INTRODUCTION TO VYAVAHĀRAMAYŪKHA XIV

\noindent
fort) as the capital of s'iva (S'ivaji), to S'ahaji and JĪjā (the mother of S'ivaji) and to Bālāji Kāyastha, a minister of S'ivaji at whose instance the work was composed by Gāgābhaṭṭa\renewcommand{\thefootnote}{1}\footnote{I. 0. Cat vol III, pp. 525\textendash\ 527, No. 1653.} Kamalākarabhaṭṭa had three sons one of whom Ananta wrote a digest styled Rāmakalpadruma on ācāra, samaya, s'rāddha, utsarga, prāyas'citta and similar matters.

Dāmodarabhaṭṭa had a son Siddhes'vara, who wrote a work called Saṁskāramayūkha in saṁvat 1736 (i.e. 1679\textendash\ 80 A.D.). Nīlakaṇṭha had two sons, S'aṁkara and Bhānu and a daughter. His wife's name was Gaṅgā\renewcommand{\thefootnote}{2}\footnote{श्रीभास्करं शिवकरं शिरसा प्रणम्य श्रीनीलकण्ठपितरं जननीं च गङ्गाम्~। तत्पादचिन्तनबलो बुधशंकराख्यः संस्कारभास्करममुं वितनोति काश्याम्~। Intro. 2nd verse to the संस्कारभास्कर of शंकर (I. 0. Cat. part III p. 433 No. 1464); लक्ष्मीं नारायणं नत्वा सीतया सहितं रघुम्~। गङ्गायुतं नीलकण्ठं गुरोः पादाम्बुजं तथा~। पितामहकृतद्वैतनिर्णयार्थस्य संग्रहम्~। नीलकण्ठात्मजेनेह क्रियते भनुनाधुना~॥ द्वैतनिर्णयसिद्धान्तसंग्रह I. 0. Cat. part III p. 488 No. 1575.}. Nīlakaṇṭha's son S'aṁkarabhaṭṭa had a hand in editing the Saṁskāramayūkha, as will be seen later on. He wrote the Kuṇḍoddyotādarśana (or Kuṇḍabhāskara ) in 1671 A.D.\renewcommand{\thefootnote}{3}\footnote{I.O. Cat. part III p.427 (footnote). Peterson in his cat. of U1war mss. says that Kuṇdārka was printed in the कुण्डग्रन्थविंशति (p.2) that that work was commented upon by रघुवीरदीक्षित, son of विठ्ठल and that रघुवीरदीक्षित wrote one of his works, the मुहूर्तसर्वस्व, in 1636 (?)} Besides these he wrote the Vratārka, the Kuṇḍārka, the Karmavipāka. Bhānubhaṭṭa, another son of Nīlakaṇṭha, wrote the Dvaitanirṇayasiddha ntasaṁgraha, which is an epitome of the Dvaitanirṇya, the ekavastrasnānavidhi, and the Homanirṇaya. The name of Nīlakaṇṭha's daughter Was Gaṅgā ( probably in her husband's family). She was married to Bhaṭṭa Mahā. deva, of the Bhāradvāja gotra, surnamed Kāla ( Kaḷe in

\newpage
% XV FAMILY OF ~NĪLAKAṆṬHA

\noindent
Marathi). Her son Divākarabhaṭṭa was a very learned man and compiled an extensive digest called Dharmas'āstrasudhānidhi. Parts of that work are Ācārārka, the Dānacandrikā, the Āhnikacandrikā, the Dānahīrāvaliprakās'a \& c. He composed his Ācārārka in the (Vikrama) year 1743 (i. e. 1686\textendash\ 87 A.D.)\renewcommand{\thefootnote}{1}\footnote{vide. I. 0. . part III, p 509\textendash\ 510, No. 1616. The verses at the end are {\qt यद्वाक्याद्विधिवाक्यार्थमपूर्वार्थाभिधातृता~। नीलकण्ठो जयत्येष मीमाīlसकधुरंधरः~॥ श्रीबालकृष्णात्मजसूनुनिर्मितं वर्षेग्निवेदाश्वहिमांशुसंयुते~। जना प्रकुर्युः किल धर्मकार्यमाचारसूर्ये किल संविलोक्य~॥}.}. In that work he speaks of his maternal grandfather as the foremost among Mīmāñsakas. In the Dānahīrāvaliprakāsa he speaks of Nīlakaṇṭha as possessed of the unclouded wisdom of Bṛhaspati and S'ukra. From the introductory verses to the Dānasaṁkṣepacandrikā we find that his mother's name was Gaṅgā and father's name Mahādeva. In that Work be distinctly says that he follows the Dānoddyota, Dānaratna and Dānamayūkha. The last is one of the twelve mayūkhas of Nīlakaṇṭha.

It is not neccessary to pursue the pedigree of the family beyond the immediate descendents of Nīlakaṇṭha.

% footnote 2-4 text missing

\newpage
% INTRODUCTION TO VYAVAHĀRAMAYŪKHA XVI

Therefore the pedigree of the family is 

\begin{center}
\includegraphics[width=1.1\linewidth]{Cropped_Images/Figures/Figure24-1.jpg}
\end{center}

For a more detailed pedigree Mandlik's edition may be consulted. It is however to be remembered that the pedigree of the family given by Mandlik on information supplied by modern s'āstris is not quite accurate. Dr.Ganganatha Jha was not able to find recently any living descendant of Nīlakaṇṭha in Benares. In Mandlik's edition Gāgābhaṭṭa~is shown to have had no dascendants, while Dr. Gaganatha Jha says that a descendant of Gāgābhaṭṭa by name Rāmabhaṭṭa~lives at Benares near {\qt Ratanphātak}.

\begin{center}
\rule{0.2\linewidth}{0.5pt}
\end{center}

\fancyhead[CE]{INTRODUCTION TO VYAVAHĀRAMAYŪKHA}
\fancyhead[CO]{THE WORKS OF NĪLAKAṆṬHA}
\fancyhead[RE,LO]{\thepage}
\cfoot{}
\newpage
%%%%%%%%%%%%%%%%%%%%%%%%%%%%%%%%%%%%%%%%%%%%%%%%%%%%%
\renewcommand{\thepage}{\Roman{page}}
\setcounter{page}{17}

% XVII THE WORKS OF NĪLAKAṆṬHA 

\begin{center}
\textbf{\large III}\\

\vspace{1mm}
\textbf{\large THE WORKS OF NĪLAKAṆṬHA}

\rule{0.2\linewidth}{0.5pt}
\end{center}

Nīlakaṇṭha composed an encyclopedia embracing various topics connected with ancient and medieval Hindu civil and religious law, ceremonial, politics and cognate matters. That encyclopedia is generally styled Bhagavanta\textendash\ bhāskara in honour of the author's patron, Bhagavantadeva, a Bundella chieftain of the Sengara (S'ṛṅgivara) clan that ruled at Bhareha near the confluence of the Jumna and the Chambal (carmaṇvatī). Some variation in the name of the encyclopaedia is perceptible in the various colophons to the different parts of it. That the patron's name was Bhagavanta (\textendash\ deva or \textendash\ varman ) is quite clear\renewcommand{\thefootnote}{1}\footnote{vide the concluding verse of the Vyavahāramayūkha and the 12th verse to the प्रायश्चित्तमयूख (Benares ed.) {\qt तिथेर्मयूखं प्रतिपाद्य सम्यगाराध्य धामाथ गिरामगोचरम्~। श्राद्धं वदत्यत्र स नीलकण्ठः संप्रेरितः श्रीभगवन्तवर्मणा~॥}.}. It is therefore natural to expect that the work should be styled Bhagavanta\textendash\ bhāskara\renewcommand{\thefootnote}{2}\footnote{e.g. in the शान्तिमयूख the 14th introductory verse (in the Benares ed. of 1879) is {\qt भगवन्तभास्कराख्ये ग्रन्थेऽस्मिन् शिष्टसम्मते च ततः~। शान्तिविवेकमयूखः प्रतन्यते नीलकण्ठेन~॥}.}. But in the colophons to some of the Mayūkhas the work is called Bhagavantabhāskara\renewcommand{\thefootnote}{3}\footnote{Vide Mandlik's edition of the व्यवहारमयूख.} or simply Bhāskara\renewcommand{\thefootnote}{4}\footnote{Vide the श्राद्धमयूख (Benares ed. of 1879) the नीतिमयूख (Benaresed of 1880).}. In some other colophons it is called Vidvadbhāskara\renewcommand{\thefootnote}{5}\footnote{Vide the समयमयूख ( Benares ed. of saṁvat 1937).}. As the whole work was styled Bhāskara (the Sun) it was divided into twelve parts, just as there were twelve Ādityas and each part came to be called a Mayūkha (a ray) by a continuation of the metaphor. Nīlakaṇṭha distinctly says in most of the Mayūkhas that he composed

\newpage
% INTRODUCTION TO VYAVAHĀRAMAYŪKHA XVIII

\noindent
the work at the command of Bhagavantadeva or that he was urged or inspired by his patron to do so.\renewcommand{\thefootnote}{1}\footnote{Note the word भगवन्तदेवादिष्ट in the colophon to the व्यवहारमयूख and the प्रतिष्ठामयूख (Bombay edition of 1891 printed at the ज्ञानदर्पण press), the word संप्रेरित in the introduction to the प्रायश्चित्तमयूख (p.XVII. note 1 above), the word आज्ञप्त in the 11th intro. verse to the शान्तिमयूख (Benares ed. of 1879).}

The introductory verses in the mss. of all the Mayūkhas present a perplexing problem. Hardly any two mss, of the same Mayūkha contain the same introductory verses. For example, one of the three mss. of the Samayamayūkha in the Bhau Daji collection of the Bombay Royal Asiatic Society has only one introductory verse; while in the other two that verse does not occur at all. In one of these two latter there are four introductory verses\renewcommand{\thefootnote}{2}\footnote{भगवन्तभास्कराख्ये ग्रन्थेऽस्मिन् शिष्टसंमते च ततः~। समयस्य विधिमयूखः प्रतन्यते नीलकण्ठेन~॥} and in the other there are five, the Benares edition agreeing with the last. The Benares edition of the S'āntimayūkha (of 1879 ) contains fourteen introductory verses, nine of which (from the second) give the genealogy of the family of Bhagavantadeva; while one ms. of the S'āntimayūkha in the Bhau Daji collection has only one introductory verse which is not found in the Benares edition; and another ms. of the same in the same collection has three verses, only one of which is found in the Benares edition. In the same way the printed editions of the Prāyas'cittamayūkha and the Ācāramayūkha (Benares, 1879 ) contain fourteen introductory verses each; while mss. of these two Mayūkhas in the Gaṭṭulalji collection in Bombay have only two and three verses respectively. This perplexing variance in the number of introductory verses cannot be satisfactoriy explained by supposing that in all cases of such differences the scribes of the mss. and others introduced unauthorised

\newpage
% XIX WORKS OF NĪLAKAṆṬHA 

\noindent
interpolations. The hypothesis which, after a careful consideration of all the introductions, seems most probable is that Nīlakaṇṭha himself (or probably his son) from time to time revised his works, recast the introductory verses, added to them and also made slight alterations and additions in the body of the works.

Some of the Mayūkhas such as the printed editions of the S'ānti, Prāyas'citta, s'rāddha and Ācāra Mayūkhas contain the genealogy of the family of Bhagavantadeva. The genealogy is more or less mythical, but there are no weighty reasons to suppose that the verses are spurious and not from the pen of Nīlakaṇṭha himself\renewcommand{\thefootnote}{1}\footnote{The verses are:\textendash\ जज्ञे पितामहतनोः खलु कश्यपो यस्तस्मादजायत मुनिस्तु विभाण्डकाख्यः~। तं पुत्रिणां धुरमरोपयदृष्यशृङ्गस्तस्यान्वयेप्यजनि शृङ्गिवराभिधानः~॥ तस्मिन्वंशे महति वितते सेङ्गराख्ये नृपाणां राजा कर्णः समजनि यथा सागरे शीतरश्मिः~। कीर्त्या यस्य प्रथिततरया श्रोत्रजातेभिपूर्णे कर्णस्यापि प्रविततकथा नावकाशं लभन्ते~॥ विशोकाख्यदेवस्ततस्तत्सुतोभूद्विशोकीकृता येन सर्वा धरित्री~। ततोप्यास राजास्तशत्रुस्ततोभूद्रयाख्यो रयेणैव सर्वाहितघ्नः~॥ बभूवाथ वैराटराजस्ततोभून्नृपो मेदिनीवल्लभो वीठुराजः (v. 1. मेदिनीशो बभौ वीठुराजः)~। नरब्रह्मदेवस्ततो मन्युदेवस्ततोभून्नृपश्चन्द्रपालाभिधानः ( v.l.श्चेन्द्र )~॥ शिवगणाख्यनृपः समजन्यथो शिवगणाख्यपुरं प्रचकार यः~। शिवगणेन समः सकलैर्गुणैः शिवशिवप्रथमो गणनासु यः~॥ रोलिचन्द्र इति तत्तनयोभूत्कर्मसेननृपतिस्तमथानु~। लोकपो नरहरिर्नृपराजो रामचन्द्र इति तत्तनुजातः~॥ यशोदेवस्ततो जातस्ताराचन्द्रनृपस्ततः~। चक्रसेनस्ततो राजा राहसिंहनृपो यतः~॥ ततोप्यभूद्भूपतिसाहिदेवः स्वकीर्तिभिर्निर्जितदुग्धसिन्धुः~। अभूत्ततः श्रीभगवन्तदेवः सदैव भाग्योदयवान् क्षितीशः~॥ यद्दानद्रविणाद्रिनिर्जितवपु रत्नाचलो लज्जया दूरे स्तब्ध इलावृते निविशते नो यत्र पुंसां गतिः~॥ किं च त्रस्यदरातिवामनयनानेत्राम्बुभिर्वर्धितस्तेजोग्निर्वडवामुखोत्थहुतभुक्तुल्यः कथं नो भवेत्~॥ आज्ञप्तस्तेन राज्ञा विबुधकुलमणिर्दाक्षिणात्यावतंसो भट्टः श्रीनीलकण्ठः स्मृतिषु दृढमतिर्जैमिनीयेद्वितीयः~। आज्ञामादाय मूर्ध्ना सविनयममुना तस्य सर्वान्निबन्धान्दृष्ट्वा सम्यग्विविच्य प्रविततकिरणस्तन्यते भास्करोयम्~॥. Vide also Aufrecht's Bod. Cat., p.280. No.656 and I.O.Cat. part III, p.429, No.1444 and Mandlik's Introduction LXXVII.}. The genealogy is: from Brahmā was born Kas'yapa, whose son was Vibhāṇḍaka, whose son was Rṣyas'ṛiṅga. In the family of the latter was born S'ṛṅgivara, after whom the family came to be kaown as Sengara. King Karṇa was born in that family. Then follows a line of eighteen kings, the last being Bhagavantadeva.

\newpage
% INTRODUCTION TO VYAVAHĀRAMAYŪKHA XX

The order in which the twelve Mayūkhas were composed is an interesting question. In the introductory verses to the Benares editions of the Ācāramayūkha, the Prāyaścittamayūkha and the S'āntimayūkha, the order is given as follows\renewcommand{\thefootnote}{1}\footnote{संस्काराचारकालाः समुचितरचनाः श्राद्धनीतिः विवादो दानोत्सर्गप्रतिष्ठा जगति जयकराः संगतार्थानुबद्धाः~। प्रायश्चित्तं विशुद्धस्तदनु निगदिता शान्तिरेवं क्रमेण ख्याता ग्रन्थेऽत्र शुद्धे बुधजनसुखदा द्वादशैते मयूखाः~॥}:\textendash\ (1) Saṁskāra; (2) Ācāra; (3) Samaya; (4) S'rāddha; (5 ) Nīti; (6) Vyavahāra; (7) Dāna; (8) Utsarga; (9) Pratiṣṭhā; (10) Prāyas'citta; (11) S'uddhi; (12) S'ānti. The same order occurs in another verse in the introduction to the Samayamayūkha\renewcommand{\thefootnote}{2}\footnote{संस्काराचारकालाः श्राद्धं नीतिर्विवाददाने च~। अत्रोत्सर्गः प्रतिष्ठा प्रायश्चित्तं विशुद्धिशान्ती च~। (Benares ed. of saṁvat 1937). vide I.O.Cat part III, p.428 No.1441.}. In the colophon at the end of the Ācāramayūkha it is described as the second; while the S'āntimayūkha is described as the twelfth. But it is worthy of note that in the colophon to the edition of the Pratiṣṭhāmayūkha published in Bombay in 1891, it is described as the eighth while it is the ninth according to the order set forth above. The introductory verses to many of the Mayūkhas and the internal evidence contained in them is sufficient to establish the order in which almost all the Mayūkhas were written\renewcommand{\thefootnote}{3}\footnote{तिरथेर्मयूखं प्रतिपाद्य सम्यगाराध्य धामाथ गिरामगोचरम्~। श्राद्धं वदत्यत्र स नीलकण्ठः संप्रेरितः श्रीभगवन्तवर्मणा~॥ Intro. to the श्राद्धमयूख. This shows that the श्राद्ध was composed after the समयमयूख that speaks of this. पितृसौहित्यसंसिद्धिहेतुमुक्त्वाथ तत्सुतः~। नीलकण्ठः प्रकुरुते राजेनीतिं नृपाट्टताम्~॥ नीतिमयूख. The first verse of the व्यवहारमयूख shows that it was written after the नीतिमयूख. उक्त्वा दानविधिं नाम राममाराध्य भास्करम्~। उत्सर्गविषये भट्टनीलकण्ठो वदत्यथ~॥ उत्सर्गमयूख; महो महत्समाराध्य जलोत्सर्गमथोक्तवान्~। प्रतिष्ठां सर्वदेवानां नीलकण्ठो वदत्यसौ~॥ प्रतिष्ठमयूख\textendash }. Nīlakaṇṭha very frequently says that a Particular subject has been already treated of by him in another Mayūkha or that he will dilate on it in a subsequent Mayūkha. From the cross

\newpage
% XXI THE WORKS OF ~NĪLAKAṆṬHA

\noindent
references contained in the several Mayūkhas it appears that the order set forth above is tolerably correct\renewcommand{\thefootnote}{1}\footnote{e.g. the श्राद्धमयूख (Benares ed.) says {\qt प्रपञ्चितं चेदं संस्कारमयूखे} (p.46); in the आचारमयूख (Benares ed.) we find {\qt एषां प्रतिष्ठापूजादि सर्वं प्रतिष्ठामयूखे} (p.69); दुःस्वप्नाश्चाचारमयूखे उक्ताः (नीतिp. 48); {\qt दानमयूखे वक्ष्यते} (p.87 0f समय); {\qt निर्णीतं चेदं समयमयूखेऽस्माभिः} व्यवहार}. Considerations of space and utility require that the cross references should not be set out here in detail.

The next question is whether Nīlakaṇṭha composed other works than the twelve Mayūkhas. In appendix A. there is a work called Vyavahāratattva. Four different reasons lead irresistibly to the conclusion that that work was composed by Nīlakaṇṭha. In the first place the colophon at the end of that work describes it as the composition of Nīlakaṇṭha, the son of the Mīmāñsaka S'aṅkarabhaṭṭa. In the second place, in the section on dattāpradānika, the author of the Vyavahāratattva speaks of the Dvaitanirṇaya as composed by his father. Besides, at the beginning of the section on Dāyavibhāga, the author of the Vyavahāratattva says that the proposition that the sources of ownership are those well known from worldly dealings has been established by him in the discussion on ownership. This is obviously a reference to the Vyavahāramayūkha wherein there is an elaborate discourse on {\qt svatva}. Besides there is a very elose correspondence in language and doctrines between the Vyavahāratattva and the Vyavahāramayūkha. Therefore there can be no room for doubt that both works are by the same author\renewcommand{\thefootnote}{2}\footnote{For a discussion about the व्यवहारतत्त्व vide 21 Bom, L.R. p.1\textendash\ 4 ( Journal portion ).}. The Nirṇayasindhu of Kamalākara several times quotes a Vyavahāratattva, which, however, is certainly a different work altogether as the quotations show that that work dwelt upon ceremonial matters and

\newpage
% INTRODUCTION TO VYAVAHĀRAMAYŪKHA XXII

\noindent
religious rites. The only important points in which the Vyavahāratattva differs from the Vyavahāramayūkha are two, viz. the former work places the mother before the father in the matter of succession, while the latter reverses the order and the former makes no reference to the sister as an heir, while the latter assigns her a high place among gotraja heirs. The reason probably lies in the fact that the Vyavahāratattva was a mere epitome and the author rather followed in both matters the orthodox school of Vijñānes'vara, who was a southern writer like Nīlakaṇṭha himself; while in the Vyavahāramayūkha he propounded the views prevalent or favoured in the territories of his patron or at his court. About the position of the father, he distinctly states that the eastern writers preferred him to the mother. It is noteworthy that neither Mandlik in his learned introduction nor the learned authors of the Digest of West and Būhler refer to the Vyavahāratattva of Nīlakaṇṭha. That work is for the first time placed in print before Sanskrit scholars. It is not possible to say that the Vyavahāratattva is an abridgment of the Mitākṣarā. A comparison of the contents of the former with the latter shows that the topics dealt with are arranged in different ways in two works.

Nīlakaṇṭha seems to have also composed a work on adoption styled Dattakanirṇaya. In the Vyavahāratattva the author refers to a Dattakanirṇaya written by himself. The Dharmsindhu also says that the Dattakanirṇaya of Nīlakaṇṭha prescribes that on the death of an adopted son his natural and adoptive fathers had both to observe mourning for three nights and the sapiṇḍas for one night, while on the death of an adoptive son whose thread\textendash\ ceremony had been performed (in the family of

\newpage
% XXIII THE WORKS OF NILAKAṆṬHA

\noindent
adoption) the adoptive father and his sapiṇḍas would have had to observe mourning for ten days\renewcommand{\thefootnote}{1}\footnote{{\qt दत्तकस्य मरणे पूर्वापरपित्रोस्त्रिरात्रं सपिण्डानामेकाहमाशौचमुपनीतदत्तकमरणादौ पालक सपिण्डानां दशाहादीति नीलकण्ठीये दत्तकनिर्णये} धर्मसिन्धु (परिच्छेद III, पूर्वार्ध).}. The quotation from the Dharmasindhu shows that what is referred to is not the section on adoption in the Vyavahāramayūkha, but an independent work, since in the Vyavahāramayūkha there is nothing coresponding to the quotation.

Nīlakaṇṭha is said to have written (according to Aufrecht) two works styled Dharmaprakāśa and S'rāddhaprakāśa. The former is referred to in the Saṁskāramayūkha\renewcommand{\thefootnote}{2}\footnote{p.37 of the edition issued by the Gujarati Press in 1913 (धर्मप्रकाशे तातचरणाः).}. It is extremely doubtful whether the Dharmaprakās'a is a work of Nīlakaṇṭha. We saw above that S'aṁkarabhaṭṭa wrote a work called Dharmaprakāśa. It is probable that there is some confusion owing to the defective text of the Saṁskāramayūkha, wherein editorial additions were made by the son of Nīlakaṇṭha.

The edition of the Saṁskāramayūkha published by the Gujarati Press in Bombay presents a curious problem. The introductory verses make it clear that the work was composed by S'aṁkara, the son of Niakaṇṭha and not by Nīlakaṇṭha himself. The colophon at the end also makes this clear. In the body of the work the other Mayūkhas are in several places referred to as {\qt composed by my father}. For example, on pp. 7 and 10 of the printed edition we have श्राद्धमयूखे तातचरणाः. In other places such expressions as the following are met with: \textendash\ {\qt शान्तिस्तु शान्तिमयूखे वक्ष्यते} (pp.14 and 23 ); {\qt कलिवर्ज्यानि च समयमयूखे दर्शयिष्यामः} (p.70); {\qt पुत्रिकाकरणप्रकारादिकं व्यवहारमयूखे दर्शयिष्यामः} (p.82); {\qt विशेषस्तु आचारमयूखे वक्ष्यते} (p.129). In

\newpage
% INTRODUCTION TO VYAVAHĀRAMAYŪKHA XXIV

\noindent
most of these places, there are different readings in some mss., as the footnotes point out, to some such effect as {\qt तातचरणकृतमयूखे ज्ञेयम्}. On p. 120 We read {\qt विवाहोत्तरं पञ्चयज्ञानुष्ठानमुक्तमाचारमयूखे दर्शयिष्यामः~। अष्टकादिश्राद्धज्ञानं श्राद्धमयूखे} and on p. 130 {\qt अन्ये शूद्रधर्मा आशौचादयस्तत्र तत्र वक्ष्यन्ते शान्तिकादावप्यधिकारः शान्तिमयूखे स्थापयिष्यते}. In these cases there are no different readings pointed out in the footnotes. In this state of the printed text, several mss. of the Saṁskāramayūkha were consulted. It was found that they all contained the introductory verses and the colophon ascribing the work to Nīlakaṇṭha's son. In the present state of our knowledge all that can be said is that the Saṁskāramayūkha of Nīakaṇṭha was edited by his son S'aṁkara with additions of his own, but that what we now have is substantially the work of Nīlakaṇṭha. If ever a ms. of the Saṁskāramayūkha comes to light containing the text as it left the hand of Nīlakaṇṭha, it will afford an interesting comparison with the printed text.

\begin{center}
\rule{0.2\linewidth}{0.5pt}
\end{center}

\fancyhead[CE]{INTRODUCTION TO VYAVAHĀRAMAYŪKHA}
\fancyhead[CO]{THE PERIOD OF NĪLAKṆṬHA'S ACTIVITY}
\fancyhead[RE,LO]{\thepage}
\cfoot{}
\newpage
%%%%%%%%%%%%%%%%%%%%%%%%%%%%%%%%%%%%%%%%%%%%%%%%%%%%%
\renewcommand{\thepage}{\Roman{page}}
\setcounter{page}{25}

% XXV THE PERIOD OF NĪLAKṆṬHA'S ACTIVITY

\begin{center}
\textbf{\large IV}\\

\vspace{1mm}
\textbf{\large The period of Nīlakaṇṭha's literary activity.}\\

\rule{0.2\linewidth}{0.5pt}
\end{center}

The period of the literary activity of Nīlakaṇṭha can be determined with tolerable precision. Nīlakaṇṭha frequently quotes his father's Dvaitanirṇaya in the Mayūkhas on Vyavahāra, Prāyas'citta, Samaya, S'rāddha, and S'ānti. The Dvaitanirṇaya quotes the Toḍarānanda, an encyclopaedia of religious and civil law, astronomy and medicine, compiled by Todarmal, the famous finance minister of Akbar. The Jyotiṣasaukhya, a portion of the Toḍarānanda, was composed in 1572 A.D. and a ms. of the Vyavahārasaukhya was copied in 1581 A. D. Therefore it is reasonable to suppose that the Dvaitanirṇaya could not have been composed much earlier than 1600 A. D. Kamalākara, who was the first cousin (paternal uncle's son) of Nīlakaṇṭha composed the Nirṇayasindhu, which was one of the earliest of his numerous works, in 1612 A. D. Nīlakaṇṭha, who was the youngest of the four sons of S'aṁkarabhaṭṭa, could not have begun his literary career earlier than Kamalākara who was only the second son of his father Rāmakṛṣṇa, the latter again being older than Saṁkarabhaṭṭa. Therefore it is highly probable that Nīlakaṇṭha's literary activities began later than 1610 A. D. One ms. of the Vyavahāratattva bears the date saṁvat 1700 (i.e. 1644 A.D.). This may be the date when the ms. was copied or it may be the date when the work was composed. At all events the Vyavahāratattva is not later than 1644 A.D. The Vyavahāratattva presupposes the Vyavahāramayā\textendash

\newpage
% INTRODUOTION TO VYAVAHĀRAMAYŪKHA XXVI

\noindent
kha and refers to the author's Dattakanirṇaya. Hence Nīlakaṇṭha must be deemed to have written a good deal before 1644 A.D. A ms. of the S'āntimayūkha in the Bhau Daji collection (in the Bombay Royal A. Society ) seams to bear the date saṁvat 1706, i.e. 1650 A.D.\renewcommand{\thefootnote}{1}\footnote{The colophon is श्रीनीलकण्ठरचितः स्मृतिभास्कराख्यो ग्रन्थो मया रसखर्षिकुसंमितेब्दे~। चैत्रे सिते रवितिथौ रविपादपद्मे पद्मिकृतौ विकसता जनतोयकृष्णैः~। चेन्नादृतः कतिपयैरपि दुष्टभावैः किं तेन भावनिपुणाः खलु सन्ति सन्तः~। किं दग्धचञ्चुपुटकाककदम्बकेन पक्कं रसालफलमुज्झितमेव सम्यक्~॥ इति श्री मीमांसकभट्टनीलकण्ठकृते भास्करे शान्तिमयूखः समाप्ति मगमत्~।. It will be noticed that one letter after मया is wanting.} Whether this is the date of the composition of the work or only the date of its being copied does not make much difference to the argument. The Sāntimayūkha is the last of the twelve Mayūkhas that Nīlakaṇṭha composed. Hence the above quotation makes it clear that the last of the Mayūkhas was composed not later than 1650 A. D. Therefore the literary activity of Nīlakaṇṭha must be placed between 1610 and 1650 A.D. Since the Vyavahāratattva was either composed or copied in 1644 A.D. and presupposes the Vyavahāramayūkha, the latter could not have been composed later than 1640 A.D. This conclusion about the period of the literary activity of Nīlakaṇṭha and the date of the Vyavahāramayūkha is corroborated by several circumstances. Gāgābhaṭṭa, who was the son of Dinakara, the first cousin of Nīlakaṇṭha, was a famous man in 1674 when he officiated at the coronation of S'ivaji. Nīlakaṇṭha, being of the same generation as Gāgābhaṭṭa's father, must have attained eminence about 1650 at the latest. s'aṁkara, the son of Nīlakaṇṭha, wrote the Kuṇḍabhāskara in 1671 A. D. Divākarabha~~a, the son of Nīlakaṇṭha's daughter, wrote his Ācārārka in 1686 A.D. Therefore

\newpage
% XXVII THE PERIOD OF NĪLAKAṆṬHA'S ACTIVITY

\noindent
Nīlakaṇṭha must have been a man of mature years in 1650. In the same direction points the fact that Siddhes'vara, the son of Dāmodara and nephew of Nīlakaṇṭha, wrote his Saṁskāramayūkha in 1680 A. D. It is significant to note that Puruṣottamaji, perhaps the most illustrious descendant of Vallabhācārya, who was born in saṁvat 1724 (i.e. 1668 A.D.) and who wrote at Surat, refers to the S'uddhimayūkha in his work styled Dravyas'uddhi.

\vspace{2cm}
\begin{center}
\rule{0.2\linewidth}{0.5pt}
\end{center}

\newpage
% INTRODUCTION TO VYAVAHĀRAMAYŪKHA XXVIII 

\begin{center}
\textbf{\large V}\\

\vspace{1mm}
\textbf{\large The contents of the twelve Mayūkhas.}\\

\rule{0.2\linewidth}{0.5pt}
\end{center}

It will not be out of place to give a brief outline of the contents of the twelve Mayūkhas.

(1) The \emph{\en Saṁskāramayūkha}: The worship of Gaṇes'a and Svastivācana (which are necessary in all saṁskāras); the enumeration of saṁskāras; the procedure and details about Garbhādhāna, Puṁsavana, Jātakarma, Nāmakaraṇa, Cūdākaraṇa, Upanayana, Samāvartana (return of the student from the teacher's house), and marriage; the duties of Brahmacārins; holidays; \emph{\en gotrās} and \emph{\en pravaras}; sapiṇḍa relationship; different forms of marriage, viz. Brāhma, Āsura \& c; the time of marriage; the duties of married women and of widows; the dutīes of the four castes and of the orders of householdor, of the forest hermit, and of the ascetic.

(2) The \emph{\en Ācaramayūkha}: the use of right hand in all ritual; the time of rising from bed; meditation on various deities, immortal persons \& c; directions as to the time and the place of answering calls of nature and as to the manner of purification thereafter; sipping water by way of purification \emph{\en (ācamana)}; rinsing the mouth; daiy bath and baths on special occasions; applying \emph{\en tilakas} and ashes; the performance of the daily \emph{\en saṁdhyā}; offering water to the Sun; muttering of prayers (japa); offering of oblations to fire \emph{\en (homa)}; division of the day into eight parts with the actions and engagements appropriate to each; the five great daily \emph{\en yajñas}; offering water to sages, heroes and ancestors; worship of deities such as Hara, Hari, S'ālagrāma; the flowers and leaves 

\fancyhead[CE]{INTRODUCTION TO VYAVAHĀRAMAYŪKHA}
\fancyhead[CO]{CONTENTS OF THE MAYŪKHAS}
\fancyhead[RE,LO]{\thepage}
\cfoot{}
\newpage
%%%%%%%%%%%%%%%%%%%%%%%%%%%%%%%%%%%%%%%%%%%%%%%%%%%%%
\renewcommand{\thepage}{\Roman{page}}
\setcounter{page}{29}

% XXIX CONTENTS OF THE MAYŪKHAS 

\noindent
appropriate to the worship of each deity ; the offering called Vais'vadeva; midday meal and accessory matters; engagements after dinner; sleep; dreams, good and evil.

(3) The \emph{\en Samayamayūkha}: division of \emph{\en tithis} into \emph{\en pūraṇa} and \emph{\en khaṇda}; the sāstrārtha as to each \emph{\en tithi} from the \emph{\en pratipad} to the \emph{\en amāvāsyā}; important festivals like Kṛṣṇajanmāṣṭamī, Rāmanavamī, Navarātra, Mahās'ivarātra, and the rites to be performed on each of these; the \emph{\en utsarjana} and \emph{\en upākarma} rites on the full moon of S'rāvaṇa; the time for performing an \emph{\en iṣṭi}; offering of \emph{\en piṇda}( rice\textendash\ ball ) to the Manes on the \emph{\en amāvāsyā}; eclipses and the rites to be performed when they occur; the fortnight ( bright or dark ) appropriate to diferent rites; three kinds of months, \emph{\en cāndra, sāvana}, and \emph{\en saura} differing in their duration; the rites appropriate to each month from \emph{\en caitra}; the intercalary month, the rites appropriate to it and the actions to be eschewed in it; the determination of the seasons; the sixty years cycle; rites to be performed on the birthday of a person; proper and improper times for shaving; ; things prohibited in the Kali age.

(4) The \emph{\en S'rāddhamayūkha}: the definition of S'rāddha; two varieties of it, \emph{\en pārvaṇa} and \emph{\en ekoddiṣṭa}; the proper time and place for S'rāddha; persons compe tent to perform S'rāddha; cases in which women were competent to perform S'rāddha; such S'rāddhas as mahālaya; materials to be used in S'rāddha ; use of flesh prohibited in S'rāddha though allowed in former ages; discourse on kus'a and sesame; brāhmaṇas unfit to be invited at S'rāddhas; the way in which the sacred thread was to be worn at the time of performing S'rāddhas other rites; the places where \emph{\en piṇḍas} were o be offered and

\newpage
% INTRODUCTION TO VYAVAHĀRAMAYŪKHA XXX

\noindent
the size of \emph{\en piṇḍas}; gifts to brāhmaṇas; places where \emph{\en piṇḍas} are ultimately to be cast ; the \emph{\en prayoga} (procedure ) of S'rāddhas; how S'rāddha is to be performed by him who is unable to go through the whole ritual of it; the letting loose of a bull ; the sixteen S'rāddhas that led to \emph{\en sapiṇḍana}; S'rāddhas on auspicious occasions; daily S'rāddha as one of the Mahāyajñas.

(5) The \emph{\en Nitimayūkha}: defnition of king (rājan); the proper time for coronation; characteristics of a throne; the king's crown; the seven constituent elements of a state, viz. the king, the ministers, allies, people, forts, treasury and army; the principal vices of kings and their effects; evils of gambling and drinking; the qualities of a good king; the duties of kings; the five great \emph{\en yajñas} in the case of kings, viz punishing the wicked, honouring the good, increase of wealth by lawful means, impartiality and protection of the state; messengers and envoys \emph{\en (dūtas)}, their qualities and three classes; fate and human effort; eulogy of the brave that sacrifice their lives in battle; varieties of elephants; the game of chess.

(6) The \emph{\en Vyavahāramayūkha}: defnition of Vyavahāra; eighteen titles of vyavahāra; the courts of justice; judge and assessors; other tribunals than the king's courts; conflict between \emph{\en dharmaśāstra} and \emph{\en arthaśāstra} and between different rules of dharmaśāstra; force of local or family usage; the plaintiff or complainant; the defendant or opponent; the plaint, its contents and defects; the defence and its four varieties; suraties for the litigants; the \emph{\en pramāṇas}, viz. documents, possession and witnesses; description of various kinds of documents; characteristics of false witnesses; ordeals in the absence of. Other means of proof; principal ordeals viz. of fire, water, poison and balance; persons fit to undergo ordeals and

\newpage
% XXXI CONTENTS OF THE MAYŪKHAS 

\noindent
the proper times and places for ordeals; other ordeals such as holy water, rice, heated golden māṣa \& c.; ownership; meaning of \emph{\en dāya}; two kinds of heritage, \emph{\en sapratibandha} and \emph{\en apratibandha}; partition of heritage; time for partition ; shares on partition ; the rights of the father, mother, and eldest son on partition ; partition after father's death; twelve kinds of sons; adopted son; who should adopt, when one should adopt; persons competent to give in adoption; who should be adopted; the ceremonial of adoption ; two kinds of \emph{\en dattaka; dvyāmuṣyāyaṇa} defined; how far \emph{\en sapiṇḍa} relationship of the adopted son extends in the family of adoption and in the family of birth; property not liable to partition ; evidence of separation; order of succession to \emph{\en sapratibandha} heritage; the compact series of heirs from the wife to the brother's son; \emph{\en gotrajas} as heirs; sister's right to succeed; \emph{\en samanodakas} and \emph{\en bandhus}; strangers as heirs; re\textendash\ union; definition of \emph{\en strīdhana}; its varieties; sūccession to \emph{\en strīdhana}; persons excluded from inheritance; debts, recovery of debts and rates of interest; mortgages and pledges; suretyship; three or four kinds of sureties; deposit; sale by one not an owner; partnership resumption of gifts; non\textendash\ payment of wages; rescission of sales; disputes as to boundaries; assault and abuse; theft; adultery; violent offences; gambling and other miscellaneous matters.

(7) The \emph{\en Dānamayūkha}: definition of \emph{\en dāna}; eulogy of dāna; Persons competent to make gifts and receive them; things proper to be given as gifts; \emph{\en iṣṭa} and \emph{\en pūrta}; proper times and places for making gifts; measures of corn and distance; postures of idols of various deities such as Gaṇes'a, Nārāyaṇa, Kāma; the \emph{\en manḍapa} described; settling the four principal directions; the ceremonial of the worship of the planets; the sīxteen \emph{\en mahādānas}

\newpage
% INTRODUCTION TO VYAVAHĀRAMAYŪKHA XXXII

\noindent
such as weighing against gold \& c; gifts of lands, houses, elephants, horses; prohibition against the resumption of gifts; description of a \emph{\en prapā} (where water was distributed gratis ) for travellers.

(8) The \emph{\en Utsargamayūkha}: eulogy of the dedication of a reservoir of water to the public; proper time for making such a dedication; the ritual of such a gift; wells and tanks; pandal to be erected near the reservoir at the time of dedication ; twenty\textendash\ four priests required in the dedication and their duties; the deities invoked at such a dedication; purification of wells and tanks polluted by dogs, cats, asses, pigs or corpses; the planting of trees and rites appropriate thereto.

(9) The \emph{\en Pratiṣṭhāmayūkha}: the time for consecrating temples; the preparations for consecration, such as collecting firesticks, saffron, musk \& c; worship of the maṇdapa ( pavilion or pandal); bathing the image; consecrating the image in two ways, \emph{\en cala} and \emph{\en acala}; the procedure of repairing old temple buildings and re\textendash\ establishing idols pulled down or carried away by river or defaced by accident \& c.

(10) The \emph{\en Prāyas'cittamayūkha}:\textendash\ definition of \emph{\en prāyas'citta}; no necessity for \emph{\en prāyaścitta} in certain cases such as killing an ātatāyin; the description of various hells in order to induce sinners to repent; the different births to which sinners are condemned; the constitution of an assembly that is to prescribe prāyas'citta; the preliminaries to undergoing a prāyas'citta, such as shaving, applying cowdung and mud to the body; the rites common to all prāyas'citta; the various kinds of \emph{\en Kṛcchras} as prāyas'cittas, description of Brahmakūrca. Parāka, Sāntapana, Cāndrāvyaṇa; visits to various

\newpage
% XXXIII CONTINTS OF THE MAYŪKHAS 

\noindent
\emph{\en tīrthas} prescribed in the case of various classes of sinners; various causes of sinfulness and pollution, such as murder, drinking, theft, adultery, eating forbidden things, giving up vedic study, contact with certain persons; prāyas'cittas for killing a Brāhmaṇa and members of other castes, for killing various male and female relatives, for relatives of persons committing suicide and for those that attempt suicide, for killing a cow and other animals, for drinking liquors and eating flesh, onions, garlic and other prohibited articles; prāyas'cittas for taking food in certain S'rāddhas from men of other castes or from S'ūdras; prāyas'cittas for thefts of various articles and for adultery ; prāyaścitta for contact of nine kinds; no sin arises from contact at tīrthas, in marriage processions, fairs, battles, national calamities, burning of a village; prāyaścittas for lesser transgressions of various kinds such as selling oil, honey or salt by Brāhmaṇas, for receiving forbidden gifts, for being an actor \& c.

(11) The \emph{\en S'uddhimayūkha}: purification of vessels of gold, silver, copper, iron, lead \& c.; purification of vessels scratched by birds or beasts, or plates licked by S'ūdras or cows, or soiled by contact with wine \& c.; purification of cloth of various kinds when soiled; rules as to purification left to local usage by Marīci; periods of impurity on account of miscarriage or still\textendash\ birth or ordinary birth; periods of impurity on death before the first year, before the thread ceremony or marriage in the case of women; instantaneous purification in the case of persons killed in battle or killed by a stroke of lighthing \& c.; how the sick are to be purified in case of impurity due to birth or death; prāyas'citta for death on a cot, death due to snake bite; the death of a \emph{\en brahmacārin} specially ominous; the merit of helping to carry the corpse of an 

\newpage
% INTRODUCTION TO VYAVAHĀRAMAYŪKHA XXXIV

\noindent
unknown or poor person; no impurity on the death of a \emph{\en saṁnyāsin}; when the ashes are to be collected after cremation of a body; the merit of casting the ashes in the Ganges at Benares or at Prayāga; the nine S'rāddhas to be performed on death; the letting loose of a bull on the 11th day after death; procedure about S'rāddha if the day or month of death not known; practice of \emph{\en satī}; women that were unft to perform \emph{\en sahagamana}; procedure, if before one impurity ceases, another occurs; periods of impurity on hearing of the death of a sapiṇḍa abroad after the lapse of three months, six months \& c.; the period of impurity on the death of \emph{\en samānodakas} and on the death of one's teacher; purification on the death of a married sister and other relatives.

(12) The \emph{\en S'āntimayūkha}: defnition of S'ānti; even S'ūdras authorised to perform the propitiatory rites for averting evil; Vināyakas'ānti; characteristics of the nine \emph{\en grahas} (the sun, the moon, Mars and the rest, Rāhu and Ketu ); propitiatory rites on the conjunctions of certain planets; how heroes like Saudāsa, Nala, Rāma, the Pāṇḍavas suffared from the evil aspects of planets, rites on the birth of an infant with teeth or for the birth of a child on the 14th day of the dark half of a month or when the moon is in the constellation of Mūla, or when a child is born on certain \emph{\en Yogas} like Vaidhṛti and Vyatipāta; rites on the birth of a son after three daughters or \emph{\en vice versa}, and on the birth of twins; rites for birth on particular \emph{\en tithis} or days of the week or particular lunar mansions; rites on certain extraordinary events (such as weeping or laughing of trees); solemn propitiatory rites at the time of coronation \& c.

\begin{center}
\rule{0.2\linewidth}{0.5pt}
\end{center}

\fancyhead[CE]{INTRODUCTION TO VYAVAHĀRAMAYŪKHA}
\fancyhead[CO]{NĪLAKAṆṬHA'S PLACE IN DHARMAS'ĀSTRA}
\fancyhead[RE,LO]{\thepage}
\cfoot{}
\newpage
%%%%%%%%%%%%%%%%%%%%%%%%%%%%%%%%%%%%%%%%%%%%%%%%%%%%%
\renewcommand{\thepage}{\Roman{page}}
\setcounter{page}{35}

% XXXV NĪLAKAṆṬHA'S PLACE IN DHARMAS'ĀSTRA

\begin{center}
\textbf{\large VI}\\

\vspace{1mm}
\textbf{\large The position of Nīlakaṇṭha in Dharmas'āstra Literature.}\\

\rule{0.2\linewidth}{0.5pt}
\end{center}

The developmant of religious and civil law in India falls into four well\textendash\ marked but somewhat overlapping periods. The first period starts in the midst of antiquity and culminates in the ancient Gṛhya and Dharma sūtras. Most of the Gṛhya and Dharma sūtras even in their extant form are several centuries earlier than the Christian era. The present writer is not one of those who hold that metrical \emph{\en smṛtis} in continuous s'loka metre are in a body later than the sūtra works (at least the older ones among those extant ). It seems very probable that metrical \emph{\en smṛtis} were composed even before the sūtra style attained its full vigour. It may be readily admitted that most of the extant metrical \emph{\en smṛtis} are much later than some of the extant Dharmasūtras (such as those of Gautama, Baudhāyana, Āpastamba). But the same cannot be said of the smṛti material contained in the Mahābhārata and of the Manusmṛti. Therefore it must be said that while several attempts were being made to compose sūtra works on ritual and law, metrical works also were being composed for the same purpose. The sccond period is that of the metrical smṛtis like those of Yājñavalkya, Nārada, Bṛhaspati, Kātyāyana and a host of other writers. This period extends from the centuries immediately preceding the Christian era to about 600 A.D. The third period is that of eminent commenta tors and it extends from the 7th century to the 12th. Among its earlier representatives are Asahāya, Vis'varūpa and Medhātithi. To this period belong several well\textendash\ known names such as those of Bhāruci, S'rikara

\newpage
% INTRODUCTION TO VYAVAHĀRAMAYŪKHA XXXVI

\noindent
Govindarāja and Dhāres'vara. But the best exponents of this period are Vijñānes'vara and Aparārka, who respectively flourished in the latter half of the 11th and the first half of the 12th century. From the 13th century to the 18th is the period of the Nibandhakāras, the writers of digests and encyclopeadias. One finds that many writers in this period compose treatises in which they review all the work done by their predecessors from the earliest times, introduce order and system in the heterogeneous and scattered mass of material that had accumulated during the lapse of centuries, examine the views of different authors, express their adherence to some one view and discard or refute the rest. Nīlakaṇṭha is one of the foremost representatives of this period. His position is analogous to that of Bhaṭṭoji Dīkṣita in Grammar or of Jagannātha in Poetics. Nīlakaṇṭha makes a difference in dealing with the conflicting views of writers believed to be inspired sages like Atri, Aṅgiras, Devala, Manu and of later writers Iike Medhātithi, Hemādri, Mādhava and others. As regards the former class of writers, he hardly, if ever, says that they are wrong, but tries to reconcile the diffrences amongst them as best as he can and, where the confict is utterly irreconcilable has recourse to the theory that their views had reference to a different yuga. As regards writers of the second class his method is different. He has the highest admiration and reverence for authors like Vijñānes'vara, Mādhava and Hemādri. But his veneration for these authorities does not make him slavishly follow their dicta and bow to their authority in everything. He very often expresses frank dissent from their views. But his criticisms of these writers are always impartial and most courteous as befits a scholar whose passion is the search of truth as it presents it to himself. He boldly critcizes the

\newpage
% XXXVII NĪLAKAṆAṬH'S PLACE IN DHARMAS'ĀSTRA 

\noindent
opinions of every one, not sparing even his own father who was a profound mīmāñsaka\renewcommand{\thefootnote}{1}\footnote{For example, vide श्राद्धमयूख (p. 25 Benares ed.) अत्र
{\qt कुलद्वयेऽपि चोच्छिन्ने स्त्रीभिः कार्या नृप क्रियाः} इति वाक्ये स्त्रीग्रहणं भार्यापरमभिप्रेत्य मातुलाद्यभावे आसुरादिविवाहोढा क्रियाकारिणीति तातचरणाः~। प्रमाणं त्वत्र न जाने~। किं च यथा {\qt पितुः पुत्रेण कर्तव्यः} इत्यत्र पुत्रशब्देन द्वादशविधपुत्रग्रहणं तथा पत्नीशब्देनापि सकलविवाहोढाग्रहणमपि प्रतिभाति~।; शान्तिमयूख (p. 25 Benares ed. ) {\qt तातचरणास्तु अयुतहोमादीनामेव प्राधान्यं ग्रहहोमस्त्वङ्गमित्याहुस्तदाशयं न जाने~। \ldots अत एव ग्रहपूजाहोमयोरपि प्राधान्यं भाति~।}.}. He is profuse in acknowledging the debt he owes to others. Wherever he did not personally verify a quotation from an ancient work but took it over from one of his predecessors, he distinctly says so\renewcommand{\thefootnote}{2}\footnote{For example, vide the words {\qt हेमाद्रौ पारिजाते, हेमाद्रौ वह्निपुराणे, दानहेमाद्रौ गारुडपुराणे, माधवीये नारायणाः~।} \& c.}. In the vastness of the material drawn upon, in the ease and flow of style, in the conciseness and perspicuity of his remarks, in sobriety of judgment, in acuteness of vision, in the orderly presentation of various topics for discussion, Nīlakaṇṭha is hardly rivalled, much less surpassed, by any writer of this period. When this is said, it is not meant that all the twelve Mayūkhas are equal in execution and workmanship. The best are the Mayūkhas on Vyavahāra, S'rāddha, Prāyaścitta, Ācāra and Samaya. The weakest of the whole lot are the Mayūkhas on Nīti and Utsarga. From appendix C it will be seen that Nīlakaṇṭha quotes no less than about a hundred smṛtis and several hundred other works on Dharmas'āstra.

Nīlakaṇṭha, being bred and brought up in an atmosphere redolent with the Pūrvamīmāñsa system, very frequently discusses the doctrines of that system and makes very acute use of them in all the Mayūkhas. In the Vyavaḥāramayūkha alone he draws upon the Pūrvamīmāñsā in dozens of places. In Appendix F are brought together most of the passages from the Vyava hāramayūkha in which the Pūrvamīmāñsā system is relied upon or appealed to by Nīlakaṇṭha.

\vspace{-5mm}
\begin{center}
\rule{0.2\linewidth}{0.5pt}
\end{center}

\newpage
% INTRODUCTION OF VYAVAHĀRAMAYūKHA XXXVIII 

\begin{center}
\textbf{\large VII}\\

\vspace{1mm}
\textbf{\large Nīlakaṇṭha and other writers on Vyavahāra.}\\

\rule{0.2\linewidth}{0.5pt}
\end{center}

The Vyavahāramayūkha stands in a special relation to the Mitākṣarā of Vijñānes'vara and the Madanaratna. When Nīlakaṇṭha wrote it appears that Vijñāneśvara had come to be looked upon as the most authoritative writer on Dharmaśāstra. In the Dvaitanirṇaya his father speaks of Vijñāneśvara as the foremost among writers of {\qt nibandhas}\renewcommand{\thefootnote}{1}\footnote{The words are {\qt सर्वनिबन्धकृद्वरिष्ठेन विज्ञानेश्वरयोगिना}.}. Nīlakaṇṭha himself looked upon Vijñāneśvara as the first among {\qt sāmpradāyikas} (those who are repositories of traditional lore)\renewcommand{\thefootnote}{2}\footnote{Mark the words {\qt साम्प्रदायिका विज्ञानेश्वरादयः} (text p.171, II 6\textendash\ 7)}. Of all the Mayūkhas it is in the Vyavahāramayūkha that Nīlakaṇṭha most frequently quotes and also criticises the Mitākṣarā. In appendix D are collected together all those passages from the Vyavahāramayūkha wherein the Mitākṣarā is either quoted or criticised. It will be seen that the most important points on which the Vyavahāramayūkha differs from the Mitākṣarā are the preference of the father over the mother, the high place assigned to the sister as an heir, the postponement of the half brother and his son to the paternal grand\textendash\ mother and sister, the various kinds of strīdhana and the different rules of succession as to each. It seems that Nīlakaṇṭha highly esteemed the Madanaratna. He quotes that work as frequently as he does the Mitākṣarā and in most places follows its views in preference to those of others. In appendix E all those passages where the Madanaratna is quoted or referred to have been brought together. Unfortunately it was not possible to secure a copy of the Madanaratna (Vyavahāroddyota) even after a good deal

\fancyhead[CE]{INTRODUCTION TO VYAVAHĀRAMAYŪKHA}
\fancyhead[CO]{NĪLAKAṆṬHA AND HIS PREDECESSORS}
\fancyhead[RE,LO]{\thepage}
\cfoot{}
\newpage
%%%%%%%%%%%%%%%%%%%%%%%%%%%%%%%%%%%%%%%%%%%%%%%%%%%%%
\renewcommand{\thepage}{\Roman{page}}
\setcounter{page}{39}

% XXXIX NĪLAKAṆṬHA AND HIS PREDECESSORS 

\noindent
of inquiry and search. A comparison of the original text of the Madanaratna with the Vyavahāramayūkha would have cleared up many difficult points. The Vīramitrodaya, however, has been of great help in pointing certain views as those of the Madanaratna.

In the division of his encyclopaedic work into twelve parts and in the general method of treatment Nīlakaṇṭha had several predecessors. Hemādri, minister of the Devagiri Yādāva kings Mahādeva (1260\textendash\ 1271 A.D.) and Rāmacandra (1271\textendash\ 1309 A. D.), composed a vast encyclopaedia styled Caturvargacintāmaṇi on Vrata, Dāna, Tīrtha, Mokṣa, Kāla \&c. Caṇḍesvara, minister of the king of Mithilā, wrote a voluminous work divided into seven parts called ratnākaras ( oceans, as in Hindu mythology there are seven oceans) on Dāna, Vyavahāra, S'uddhi, Pūjā, Vivāda \& c. He weighed himself against gold in \emph{\en s'ake} 1276 i.e. 1314 A.D. His Vivādaratnākara is a work of paramount authority in Mithilā and is quoted in the Vyavahāramayūkha. King Madanasiṁha composed a large work called Madanaratna in seven Uddyotas on Samaya, Ācāra, Vyavahāra, Prāyas'citta, Dāna, S'uddhi and S'ānti. The Madanaratna and Hemādri are quoted at every step by Nīakaṇṭha. The Nrisiṁhaprasāda is a work of enormous extent, being nearly half as much in bulk as the Mahābhārata. It was composed by Dalapati (is it a proper name? ) who was the chief minister of king Nizamshah\renewcommand{\thefootnote}{1}\footnote{Vide I.O.Cat.part III. pp.434\textendash\ 435 No.1467, Mitra's Bik. Cat. pp. 429\textendash\ 430, Benares {\qt Pandit}, New Series, vol. V.p.377 for an account of the work.}, probably the founder of the Nizamshāhi dynasty of Ahmednagar ( 1489\textendash\ 1508 ). A ms. of that work was written in \emph{\en saṁvat} 1568 i. e. 1512 A. D. This

\newpage
% INTRODUCTION OF VYAVAHĀRAMAYŪKHA XL

\noindent
work is divided into twelve parts called sāras on Saṁskāra, Āhnika, S'rāddha, Kālanirṇaya, Vyavahāra, Prāyas'citta, Karmavipāka, Vrata, Dāna, S'ānti, Tīrtha and Pratiṣṭhā. It is remarkable how closely the parts of this work agree in number and nomenclature with those of the Bhagavantabhāskara. The Nṛsiṁhaprasāda is quoted in the Samayamayūkha and the Dvaitanirṇaya. Raghunandana, who is later than 1450 and earlier than 1600 A. D. and who wrote a commentary on the Dāyabhāga, is the author of a comprehensive work called Smṛtitattva, divided into twenty\textendash\ eight parts styled \emph{\en tattvas} on Dāya, Divya, Saṁskāra, S'uddhi, Prāyas'citta, Tīrtha, Vyavahāra, Pratiṣṭhā \& c. His Divyatattva is quoted in the Vyavahāramayūkha and the other \emph{\en tattvas} also are frequently referred to in the other Mayūkhas. He is designated Smārtabhaṭṭācārya and Gaudamīmāñsaka by Nīlakaṇṭha. Ṭoḍaramalla, the famous finance minister of Akbar, compiled an encyclopaedia of religious and civil laws, medicine and astronomy styled Ṭodarānanda. The various sections of this work are called saukhyas and deal with Ācāra, Dāna, Vyavahāra, Prāyaścitta, Samaya, S'uddhi, Vrata \& c. We saw above that his Jyotistattva was composed in 1572 A.D. and a ms. of his Vyavahārasaukhya was copied in 1581 A.D. The Ṭoḍarānanda is quoted in the Vyavahāramayūkha and other Mayūkhas. 

\begin{center}
\rule{0.2\linewidth}{0.5pt}
\end{center}

\fancyhead[CE]{INTRODUCTION TO VYAVAHĀRAMAYŪKHA}
\fancyhead[CO]{MAYŪKHA AND MODERN HINDU LAW}
\fancyhead[RE,LO]{\thepage}
\cfoot{}
\newpage
%%%%%%%%%%%%%%%%%%%%%%%%%%%%%%%%%%%%%%%%%%%%%%%%%%%%%
\renewcommand{\thepage}{\Roman{page}}
\setcounter{page}{41}

% XLI MAYŪKHA AND MODERN HINDU LAW

\begin{center}
\textbf{\large VIII}\\

\vspace{1mm}
\textbf{\large The position of the Vyavahāramayūkha in modern Hindu Law.}\\

\rule{0.2\linewidth}{0.5pt}
\end{center}

It has been repeatedly laid down by the Bombay High Court and by the Privy Council, the highest judicial tribunal for India, that the three books of chief authority in western India are Manu, the Mitākṣarā and the Mayūkha\renewcommand{\thefootnote}{1}\footnote{Vide Murarji v Parvatibai I.L.R.1 Bom. 177 at p.187; Savitribai v Laxmibai I.L.R.2. Bom. 573 at p. 606; Lallubdhai v Cossibai I.L.R.5 Bom. 110 at p.117 (P.C.); Pranjivandas v Devkuvarabai 1 Bom. H.C.R. (O.C.J.) 130 at p.131.}. In the Maratha country and in the Ratnagiri district the Mitākṣarā is of paramount authority and a subordinate place, though still a very important one, is assigned to the Vyavahāramayūkha\renewcommand{\thefootnote}{2}\footnote{Krisnaji v Pandurang 12 Bom.H.C.R.65, 67\textendash\ 68 vide also 5 Bom.H.C.R (A.C.J.) 181, 185, 7 Bom. H.C.R. (A.C.J.) at p.169 and Jankibai v Sundra I.L.R.14 Bom. 612, 616 (Ratnagiri District)}. The Vyavahāramayūkha is of paramount authority in Guzerat, the town and island of Bombay and in northern Konkan\renewcommand{\thefootnote}{3}\footnote{Lallubhai v Mankuvarbai I.L.R.2 Bom. 388 at p.418; I.L.R. 6 Bom 541, 546 ; Jankibai v Sundra I.L.R.14 Bom. 612, at pp.623\textendash\ 24; Vyas Chimanlal v Vyas Ramachandra I.L.R.24 Bom. 367 (F.B.) at p.373}. Though the pre\textendash\ eminence of the Mitākṣarā in the Maratha country is admitted, yet its doctrines have in several instances been set aside in favour of those put forward in the Vyavahāramayūkha\renewcommand{\thefootnote}{4}\footnote{Bhagirthibai v Kahnujirao I.L.R.11 Bom. 285 (F.B.), at p.293.}. For example, though the Mitākṣarā nowhere recognises the sister as a

\newpage
% INTRODUCTION OF VYAVAHĀRAMAYŪKHA XLII

\noindent
\emph{\en gotraja sapiṇḍa}, the courts, following the Mayūkha, have assigned her a high place as heir even in the Maratha country. It is interesting to see how the Vyavahāramayūkha came to be recognised as an authoritative work in Guzerat. We saw above that the family of Nīlakaṇṭha came from the Deccan. Naturally all the members of that family preferred the usages of the Deccan and Saṁkarabhaṭṭa expressly says in his Dvaitanirṇaya that he will conform to the views of Deccan writers. Therefore the works of these Bhaṭṭas of Benares were highly esteemed by the learned men of the Maratha country. When the Marathas extended their sway over Guzerat in the 18th century, the works of Kamalākara (particularly the Nirṇayasindhu) and of Nīlakaṇṭha (particularly the Vyavahāramayūkha ) were relied upon by the S'āstris at the court of the Maratha rulers of Guzerat. Thus the Vyavahāramayūkha had come to be looked upon as a work of high authority in Guzerat before the advent of the British in the beginning of the 19th century\renewcommand{\thefootnote}{1}\footnote{Vide Lallubhai v Mankuvarbai I.L.R.2 Bom.388, 418\textendash\ 19 and Bhagirthibai v Kahnujirao I.L.R. 11 Bom. 285 (F.B.) 294\textendash\ 95 for the reasons of the ascendancy of the in Vyavahāramayūkha in Guzerat.}. The result was that so early as 1827 Borradaile translated the Vyavahāramayūkha in English. That the Mayūkhas of Nīlakaṇṭha were eagerly sought for even as far to the south as the Belgaum district in the times of the Peshwas is establishad by a letter of Naro Vinayak, Mamlatdar of Athni in the present Belgaum District, dated 28th June 1797. In that letter reference is made to the copying of the six Mayūkhas on Saṁskāra, Ācāra, Samaya, S'rāddha, Nīti and Vyavahāra and a request is made that the other

\newpage
% XLIII MAYūKHA AND MODERN HINDU LAW 

\noindent
six Mayūkhas may be sent for a copy being made\renewcommand{\thefootnote}{1}\footnote{Vide ऐतिहासिकलेखसंग्रह vol X . P. 5172 1etter No. 4006 (edited by) Mr. Vāsudevs'āstri Khare, 1920). As the letter is interesting the whole of it is reproduced below. "राजश्रिया विराजित राजमान्य राजश्री गंगाधर रावजी स्वामीचे शेवेसी पोष्य नारो विनायक कृतानेक साष्टांग नमस्कार वि~॥ येथील कुशल तागायत छ ३ मोहोरम. यथास्थित असे. विशेष. आपण पत्र पाठविलें तें पावलें {\qt सहा मयूखांचे पुस्तक नेलें त्याला फार दिवस झाले. हल्लीं पुस्तकांचे प्रयोजन आहे तरी बंदोवस्तानें पाठवावें,} ह्मणोन लिहिलें त्यास येथें मयूखांचे पुस्तक लिहविलें आहे तें शोधाववाचें होतें याजमुळें पोथी ठेविली होती. हल्लीं पाठविली आहे, पत्रें सुमार\textendash

७१ संस्कार

७३ आचार

९२ समय

९८ श्राद्ध

७९ नीति

७६ व्यवहार

\rule{0.05\linewidth}{0.5pt}

४८९ 

सहा मयूखांची ४८९ पत्रें खालीं वर फळ्या घालून वरतें मेणकापड व त्यावर आणखी रुमाल बांधून आपल्याकडील जासूद एक व येथून दोन गडी देऊन पुस्तक पाठविलें आहे. येऊन पोहोंचेल पुढील सहा मयूखांचे पुस्तक पाहिजे तें पाठवून द्यार्वेः, त्याजवरून येथील पुस्तक सम प्रत करून माघारें पाठवून देऊं. बहुत काय लिहिणें कृपालोभ करावा हे विनंति.}. It appears that even in Northern India the Vyavahāramayūkha was referred to by the British courts as early as 1813 A.D\renewcommand{\thefootnote}{2}\footnote{Bhagwan Singh v Bhagwan I.L.R.17 All 294, at p.314.}.

The general principle on which the courts of Western India act in construing the rules laid down by the Mitākṣarā and the Vyavahāramayūkha is that they are to be harmonised with one another, wherever and so far as that is reasonably possible\renewcommand{\thefootnote}{3}\footnote{Gojabai v Shrimant Shahajirao I.L.R.17 Bom.114, 118 quoted with approval in Kesserbai v Hunsraj I.L.R. 30 Bom. 431, 432 (P.C.)}.

It was said above that the Vyavahāramayūkha is of paramount authority in Northern Konkan. As there is divergence between the views of the Mitākṣarā and the Mayūkha in matters of succession, it becomes of

\newpage
% INTRODUCTION OF VYAVAHĀRAMAYŪKHA XLIV

\noindent
great practical importance to settle with precision the exact limits in Northern Konkan up to which the Mayūkha must be regarded as a work of paramount authority. It has been judicially decided that Karanja, which is an island opposite the Bombay harbour, is governed by the principles of the Mayūkha, that Mahad, the southernmost Taluka of the Kolaba District, is not so governed and that the predominance of the Mayūkha\renewcommand{\thefootnote}{1}\footnote{Sakharam Sadashiva Adhikari v Sitabai I.L.R.3 Bom.353.} cannot either on principle or authority be taken further south than Chaul and Nagothna\renewcommand{\thefootnote}{2}\footnote{Vide Narhar v Bhau I.L.R. 40 Bom. 621. (where the authorities are collected ).} (in the northem part of the Kolaba District).

Though the authority of the Vyavahāramayūkha is supreme in Guzerat, the island of Bombay and northern Konkan and high in the Maratha country, it is not to be supposed that the whole of it has been either adopted by the people or accepted by the courts. There are several matters, such as the twelve kinds of sons and the fifteen kinds of slaves and the marriage of a person with girls belonging to lower castes than his own, on which Nīlakaṇṭha expatiates with as much learning, patience and zest as any ancient writer, although those usages had become obsolete centuries before his time\renewcommand{\thefootnote}{3}\footnote{Vide the remarks in Rahi v Govind I.L.R. 1 Bom. 97 at p 112 and Lallubhai v Mankuvarbai I.L.R.2 Bom. 388 at pp.420 and 447.}. Nīlakaṇṭha says that the paternal great\textendash\ grand\textendash\ father, the paternal uncle and the half\textendash\ brother's son succeed together. But the courts have never recognised this rule, nor has it ever been made the foundation of a claim in a court of law. On the other hand, the views of Nīlakaṇṭha

\newpage
% XLV MAYŪKHA AND MODERN HINDU LAW

\noindent
that the sister is a gotraja sapiṇḍa and that even a married man may be taken in adoption have been followed by the courts, although hardly any eminent writer before him propounded these views. Kamalākara, a first cousin of Nīlakaṇṭha, criticises those who would include sisters in the term brothers (vide Sarvādhikari's Tagore Law Lectures p. 664, ed. of 1882). Among the other Mayūkhas, the Saṁskāramayūkha is frequently cited in the law reports\renewcommand{\thefootnote}{1}\footnote{Vide I.L.R. 2 Bom. 388 at p.425, 3 Bom. 353at p.361. 4 Bom.219 at p.221, 32 Bom.81 at pp.88,96.}. In a case reported in 22 Bom. L. R. (p. 334) both sides seem to have relied upon theṭ Pratiṭhṣāmayūkha.


\begin{center}
\rule{0.2\linewidth}{0.5pt}
\end{center}

\newpage
% INTRODUCTION OF VYAVAHāRAMAYŪKHA XLVI

\begin{center}
\textbf{\large IX}\\

\textbf{\large The present edition}\\

\rule{0.2\linewidth}{0.5pt}
\end{center}

In section I above the material on which the text of the present edition is based has been indicated. No efforts have bean spared to arrive at a correct text of the Vyavahāramayūkha. Great labour was spent in trying to trace the quotations to their sources. Some of the quotations had to be found out from mss. Only those who have ever done the work of identifying quotations can form an adequate idea of the labour involved in this task. In spite of this there are still several quotations that have defied all efforts to trace them. Often times there is great divergence between the printed texts of the authors quoted by Nīlakaṭha and the readings of the mss. In most of such cases, the readings of the mss. have been given in the text and in the footnotes the readings of the printed editions are indicated. As regards various readings only the important ones have been pointed out. The footnotes would have been encumbered with unnecessary details if every variation and every omission contained in the mss. had been indicated.

The annotations have been purposely made copious. The Vyavahāramayūkha is full of difficulties. An attempt has been made to fully explain every possible difficulty. The numerous references to the doctrines and technical terms of the Pūrvamīmāñsa have been explained at length. Parallel passages from other works have been added at every step. References to modern developments of the Hindu law have been frequently given.

In order to enhance the utility of the work several appendices have been added. Appendix A contains the

\fancyhead[CE]{INTRODUCTION TO VYAVAHĀRAMAYŪKHA}
\fancyhead[CO]{THE PRESENT EDITION}
\fancyhead[RE,LO]{\thepage}
\cfoot{}
\newpage
%%%%%%%%%%%%%%%%%%%%%%%%%%%%%%%%%%%%%%%%%%%%%%%%%%%%%
\renewcommand{\thepage}{\Roman{page}}
\setcounter{page}{47}

% XLVII THE PRESENT EDITION

\noindent
text of the Vyavahāratattva which is based on two mss. Appendix B contains the names of all the authors and works quoted in the Vyavahāramayūkha with brief notes in the case of some. Appendix C contains a consolidated list of all the authors and works occurring in the twelve Mayūkhas. Appendix D collects together all those passages in which the Mitākṣarā and Vijñānes'vara are quoted or criticized. In Appendix E are gathered together the passages where the Madanaratna is quoted, criticised or refrred to. Appendix F contains the passagas where the doctrines of the Pūrvamīmāñsā have been appealed to or relied upon in the Vyavahāramayūkha. Appendix G gives an index of the pratīkas of the verses occurring in the work.

As regards the system of transliteration, the one adopted by the Bhandarkar Institute has been followed in the Introduction. Unfortunately in the notes this system was not consistently followed with regard to four letters viz. ऋ,च्, छ् and ष्.

\begin{center}
\rule{0.2\linewidth}{0.5pt}
\end{center}

\newpage
\thispagestyle{empty}

\begin{center}
\textbf{\large A brief analysis of the contents of the text.}\\

\rule{0.2\linewidth}{0.5pt}
\end{center}

\noindent {\small 
\begin{tabular}{m{30em} m{5em}}
Subject & PAGE\\
मङ्गलाचरणम् & 1\\
\textbf{\large व्यवहारलक्षणम्} & 1\\
व्यवहारपदानि & 1\textendash\ 2\\
व्यवहारमातृकाः & 2\textendash\ 21\\
सभास्थानम् & 2\textendash\ 3\\
प्राङ्विवाकामात्यसभ्यादीनां स्वरूपम् & 3\textendash\ 4\\
राजसभातो निर्णयकान्तराणि पूगादीनि & 5\textendash\ 6\\
व्यवहारदर्शनकालः & 6\\
धर्मशास्त्रार्थशास्त्रयोर्विप्रतिपत्तौ कार्यम् & 6\\
स्मृतिविरोधे कार्यम् & 6\\
देशाचारादीनामवेक्षणम् & 7\\
कार्यार्थिनि समागते कर्तव्यम् & 8\\
असेधस्य चातुर्विध्यम् & 8\\
आसेधातिक्रमे दण्डः & 8\\
क्वचिदासेद्धुरेव दण्डः & 9\\
अनासेध्याः & 9\\
आह्वानानाह्वाने & 9\textendash\ 10\\
आहूतस्यानागच्छतो दण्डः & 10\\
आहूतागमने कार्यम & 10\\
प्रतिनिधिनियोजनम् & 11\\
क्वचित् प्रतिनिध्यभावः & 11\\
भाषालक्षणम् & 12\\
भाषायाः शोधनस्यावधिः & 13\\
पक्षाभासाः & 13\\
उत्तरलक्षणम् & 14\\
उत्तरस्य चातुर्विध्यम् & 14\\
मिथ्योत्तरलक्षणम् & 15\\
संप्रतिपत्त्युत्तरलक्षणम् & ,,\\
प्रत्यवस्कन्दनोत्तरलक्षणम् & ,,\\
प्राङ्न्यायोत्तरलक्षणम् & ,,
\end{tabular}}

\fancyhead[CO]{INTRODUCTION TO VYAVAHĀRAMAYŪKHA}
\fancyhead[CE]{SANSKRIT ANAIYSIS Of VYAVAHĀRAMAYŪKHA}
\fancyhead[RO,LE]{\thepage}
\fancyhead[LO,RE]{}
\cfoot{}
\newpage
%%%%%%%%%%%%%%%%%%%%%%%%%%%%%%%%%%%%%%%%%%%%%%%%%%%%%
\renewcommand{\thepage}{\Roman{page}}
\setcounter{page}{50}

% SANSKRIT ANAIYSIS Of VYAVAHĀRAMAYŪKHA L

\noindent {\small 
\begin{tabular}{m{30em} m{5em}}
& PAGE\\
उत्तराभासाः & 15\textendash\ 16\\
अनुत्तरत्वे कारणम् & 17\\
साधनोक्तौ क्रमः & 17\textendash\ 18\\
व्यवहारस्य चतुष्पात्त्वम् & 18\\
हीनवादिलक्षणम् & 19\\
प्रतिभूः & ,,\\
प्रतिभूत्वेनाग्राह्याः & 19\textendash\ 20\\
लग्नकाभावे कार्यम् & 20\textendash\ 21\\
हीनवादिलक्षणानि & 21\textendash\ 24\\
\textbf{\large प्रमाणनिरूपणम्} & 21\\
प्रमाणप्रकाराः & 21\textendash\ 22\\
प्रमाणबलविचारः & 22\\
क्वचिद्दिव्यप्राबल्यम् & 22\\
क्वचित्साक्षिदिव्ययोर्विकल्पः & 23\\
\textbf{\large लेख्यम्} & 24\textendash\ 30\\
लेख्यभेदाः & 24\\
लेख्यभेदानां लक्षणानि & 24\textendash\ 25\\
राजशासनस्य प्रकाराः & 27\\
अन्यलेख्यकरणे कारणानि & 29\\
संदिग्घलेख्यशुद्धिः & 29\\
दुष्टलेख्यम् & 30\\
\textbf{\large भुक्तिः} & 30\textendash\ 33\\
भोगप्रामाण्यचिन्ता & 30\textendash\ 31\\
भोगागमयोर्बलाबलविचारः & 31\\
अनागमोपभोगः & 31\textendash\ 32\\
भोगमात्रानाश्यानि & 33\\
\textbf{\large साक्षिणः} & 33\textendash\ 44\\
साक्षिभेदाः & 34\\
साक्षिभेदस्वरूपनिरूपणम् & 35\\
वर्ज्याः साक्षिणः & 36\textendash\ 37\\
साक्षिदोषाः & 38\textendash\ 39\\
साक्षिदोषकारणमब्रुवतो दण्डः & 39\\
कूटसाक्षिकर्तुर्दण्डः & 40
\end{tabular}}

\newpage
% LI INTRODUCTION OF VYAVAHĀRAMAYŪKHA

\noindent {\small 
\begin{tabular}{m{30em} m{5em}}
& PAGE.\\
दुष्टसाक्षिणां निश्चयोपायाः & 40\\
साक्षिप्रश्नप्रकारः & 40\textendash\ 41\\
शपथप्रकाराः & 42\\
साक्षिणां विप्रतिपत्तौ नियमः & 43\\
साक्ष्यमङ्गीकृत्यानभिधाने दण्डः & 43\\
साक्ष्यानङ्गीकारे दण्डः & 43\\
अनृतवादिनां साक्षिणां दण्डः & ,,\\
क्वचिदनृतवचनाभ्यनुज्ञा & 44\\
\textbf{\large दिव्यम्} & 44\textendash\ 88\\
दिव्यभेदाः & 45\\
विविधेषु विवादेषु दिव्ययोजना & ,, \\
शपथाः & 46\\
शपथप्रकाराः & ,,\\
दिव्याधिकारिव्यवस्था & 47\textendash\ 48\\
दिव्यकारिणोऽशक्तौ प्रतिनिधिग्रहणम् & 49\textendash\ 50\\
दिव्यकालः & 50\textendash\ 51\\
दिव्ययोग्यो देशः & 51\\
सर्वदिव्यसाधारणो विधिः & 52\textendash\ 55\\
धटविधिः & 56\textendash\ 61\\
धटदिव्यप्रयोगः & 62\textendash\ 68\\
अग्निदिव्यविधिः & 69\textendash\ 74\\
अग्निदिव्यप्रयोगः & 74\textendash\ 76\\
जलविधिः & 76\textendash\ 79\\
जलविधिप्रयोगः & 79\textendash\ 80\\
विषदिव्यविधिः & 80\textendash\ 81\\
कोशदिव्यविधिः & 82\\
तण्डुलदिव्यविधिः & 83\\
तप्तामाषदिव्यविधिः & 83\textendash\ 84\\
फालदिव्यविधिः & 84\textendash\ 85\\
धर्मजविधिः & 85\textendash\ 86\\
धर्मजविधिप्रयोगः & 86\textendash\ 87\\
शपथाः & 87\textendash\ 88
\end{tabular}}

\newpage
% SANSKRIT ANALYSIS Of VYAVAHĀRAMAYŪKHA LII

\noindent {\small
\begin{tabular}{m{30em} m{5em}}
& Page.\\
\textbf{\large दायनिर्णयः} &89\textendash\ 166\\
स्वत्वलक्षणम् & 89\\
स्वत्वकारणानि & 89\textendash\ 93\\
दायलक्षणम् & 93\\
सप्रतिबन्धो दायः & 93\\
अप्रतिबन्धो दायः & ,,\\
दायस्य विभागः & 94\\
विभागकालाः & 94\textendash\ 96\\
पितर्यसमर्थे ज्येष्ठपुत्रानुमत्या विभागः & 96\\
ज्येष्ठविभागे विशेषः & 97\\
ज्यैष्ठ्यनिर्णयः & 97\\
यमलयोर्ज्यैष्ठ्यविचारः & ,,\\
उद्धारविभागस्य कलौ निषिद्धत्वम् & 98\\
विभागे नारदोक्तं पितुरंशद्वयम् & ,,\\
पितामहार्जिते पितुः पुत्रस्य च समांशित्वम् & 98\\
विभागे माताप्यंशभागिनी & 99\\
पितुरूर्ध्वं विभागः & 100\\
अनेकभ्रातृपुत्राणां विभागप्रकारः & ,,\\
दायग्रहणे पुत्रपौत्रादीनामवधिः & 101\\
क्वचिन्मातृतो विभागः & 102\\
विजातीयपुत्रविभागः & 102\textendash\ 103\\
त्रैवर्णिकानां शूद्रपुत्रो न रिक्थभाक् & 103\\
विभागानन्तरोत्पन्ने विशेषः & 104\textendash\ 105\\
भ्रातृभगिनीसंस्कारे विशेषः & 105\textendash\ 106\\
द्वादशविधपुत्राः & 106\textendash\ 107\\
\textbf{\large दत्तकनिरूपणम्} & 107\textendash\ 122\\
शौनकोक्तो दत्तकपुत्रप्रतिग्रहप्रकारः & 109\textendash\ 111\\
दौहित्रभगिनेयौ एव शूद्रस्य मुख्यौ & 111\\
दानप्रतिग्रहाधिकारविचारः & 112\\
स्त्रीणां दत्तकविधौ अधिकारविचारः & 112\textendash\ 113\\
परिणीतोऽपि दत्तको ग्राह्यः & 114\\
केवलदत्तकः & 114\textendash\ 115\\
व्द्यामुष्यायणदत्तकः & ,,
\end{tabular}}

\newpage
% LIII INTRODUCTION OF VYAVAHĀRAMAYŪKHA 

\noindent {\small 
\begin{tabular}{m{30em} m{5em}}
& PAGE.\\
दत्तकसापिण्ड्यविचारः &117\textendash\ 120\\
पुत्रदानप्रतिग्रहविधिः & 120\textendash\ 122\\
ऋणविभागे विशेषः & 122\textendash\ 123\\
अविभाज्यधनभेदाः & 124\textendash\ 128\\
क्रमागतोद्धृतम् & ,,\\
विद्याधनस्वरूपम् & 125\textendash\ 126\\
अन्यदप्यविभाज्यम् & 128\textendash\ 130\\
वञ्चनया स्थापितस्य पुनर्विभागः & 131\\
विभागसंदेहे निर्णायकानि & 132\\
विभक्तकृत्यम् & 133\textendash\ 136\\
\textbf{\large सप्रतिबन्धदायहरणक्रमः} & 137\textendash\ 145\\
पत्नी & 137\textendash\ 140\\
दुहिता &141\\
दौहित्रः & 141\\
पिता & 141\textendash\ 2\\
माता & ,,\\
सोदरो भ्राता & 142\\
सोदरभ्रातृपुत्रः & 143\\
गोत्रजाः सपिण्डाः & ,,\\
पितामही & ,, \\
भगिनी & ,,\\
पितामहसपत्नभ्रातरौ & ,,\\
अन्ये गोत्रजाः सपिण्डाः & ,,\\
समानोदकाः & ,,\\
बन्धवः & 144\\
आचार्यशिष्यसहाध्यायिश्रोत्रियादीनामधिकारः & 144\\
एतेषामभावेन्यो ब्राह्मणो दायग्रहणाधिकारी & ,,\\
यत्यादिरिक्थे विशेषः & 145\\
\textbf{\large संसृष्टिनिर्णयः} & 145\textendash\ 152\\
संसर्गलक्षणम् & 146\\
केषां संसृष्टत्वं भवति & ,,\\
संसृष्टिनां पुनर्विभागः & ,,\\
संसृष्टिधनाधिकारिणः & 147\textendash\ 152\\
\textbf{\large स्त्रीधनम्} & 152\textendash\ 162
\end{tabular}}

\newpage
% SANSKRIT ANAIYSIS Of VYAVAHĀRAMAYŪKHA LIV

\noindent {\small 
\begin{tabular}{m{30em} m{5em}}
& Page. \\
तद्भेदाः &152\textendash\ 153\\
स्रीधनभेदानां लक्षणानि & 153 \\
स्त्रीभ्यो धनदाने विशेषः & 154\\
कस्मिंश्चिद्धने स्त्रियाः स्वातन्त्र्यम् & 155\\
भर्तृपुत्रादीनां स्त्रीधनेऽस्वातन्त्र्यम् & 155\textendash\ 156\\
एतस्यापवादाः & 156\\
प्रतिदाने भर्तुः विकल्पः & ,,\\
क्वचिदकामोपि भर्ता दाप्यः & ,,\\
अन्वाधेयाख्यस्त्रीधनग्रहणक्रमः & 157\\
भगिनीषु विशेषः & 158\\
यौतके विशेषः & ,,\\
अन्वाधेयभर्तृप्रीतिदत्तभिन्ने पारिभाषिके स्त्रीधनेऽधिकारिणः & 159\\
दुहितृदौहित्रीदौहित्रपुत्रादीनामधिकारक्रमः & 159\textendash\ 160\\
एतेषामभावे विवाहभेदेन स्त्रीधनग्रहणाधिकारिव्यवस्था & 160\textendash\ 161\\
मातृतुल्याः स्त्रियः & 161\\
शुल्के विशेषः & 162\\
मृतकन्यायाः धनेऽधिकारिणः & ,,\\
\textbf{\large अनंशाः} & 162\textendash\ 166\\
तेषामुद्देशः & 162\textendash\ 163\\
अनंशा यावज्जीवं पोषणीयाः & 164\textendash\ 165\\
तत्पुत्रा निर्दोषा भागहारिणः & 165\\
अनंशपत्नीकन्या भर्तव्याः & 166\\
\textbf{\large ऋणादानम्} & 166\textendash\ 190\\
धनिकस्य प्रयोगप्रकारः & 166\\
वृद्धिः & 167\\
तद्भेदाः& 167\textendash\ 168\\
वृद्धिविषये नियमः & 168\textendash\ 170\\
वृद्धिः सकृत्प्रयोगे द्वैगुण्यं नात्येति & 171\\
\textbf{\large आधिविधिः} & 171\textendash\ 175\\
आधिलक्षणम् & 171\\
तद्भेदाः & ,,\\
आधिभोगे विशेषः & 172\\
आधिनाशे मूल्यदानम् & ,,
\end{tabular}}

\newpage
% LV INTRODUCTION OF VYAVAHĀRAMAYŪKHA 

\noindent {\small 
\begin{tabular}{m{30em} m{5em}}
& Page.\\
अस्यापवादः & 172\textendash\ 173\\
आधिमोक्षः &174\\
धनामोचने आधिनाशः & ,,\\
गोप्याधौ विशेषः & ,,\\
चरित्रबन्धककृते विशेषः & 174\textendash\ 175\\
\textbf{\large प्रतिभूः} & 175\textendash\ 178\\
तत्त्रैविध्यम् & 175\textendash\ 176\\
बृहस्पतिमतेन तच्चातुर्विध्यम् & 176\\
नष्टर्णिकान्वेषणाय कालो देयः & ,,\\
प्रतिभूस्तत्पुत्रश्चर्णं दाप्यः & ,,\\
प्रातिभाव्यकृतमृणं पौत्रैर्न देयम् & ,,\\
बहुषु प्रतिभूष्वकेन ॠणदानम् & 177\\
ॠणिकः प्रतिभूदत्तं प्रतिदाप्यः & 178\\
उत्तमर्णस्यर्णग्रहणप्रकारः & 178\\
सामादय उपायाः & 178\textendash\ 182\\
अनेकोत्तमर्णानां युगपदुपस्थितौ क्रमः & 183\\
पुत्रपौत्रैरृणं देयम् & 184\textendash\ 185\\
पुत्रादिभिरदेयमृणम् &186\\
ऋणादानाधिकारिणां क्रमः & 186\textendash\ 187\\
प्रोषितविषयेन्यैरृणदानम् &188\\
उत्तमर्णतत्पुत्राद्यभावे ऋणदानप्रकारः &189\\
\textbf{\large निक्षेपः} &190\textendash\ 195\\
निक्षेपस्वरूपम् &190\\
उपनिधिलक्षणम् &,,\\
याचितादाने &191\\
देवराजोपघाते न दोषः &192\\
निक्षेपविधेरन्वाहितादिष्वतिदेशः &,,\\
सुवर्णादिषु ह्रासनियमः &193\\
और्णकार्पासादिषु वृद्धिनियमः &194\\
\textbf{\large अस्वामिविक्रयः} &195\textendash\ 200\\
तल्लक्षणम् &195\\
क्रेतुः कृत्यम् &195\textendash\ 196\\
नष्टापहृतोद्धारे नियमः &198
\end{tabular}}

\newpage
% SANSKRIT ANAIYSIS Of VYAVAHĀRMAYŪKHA LVI 

\noindent {\small 
\begin{tabular}{m{30em} m{5em}}
& Page. \\
एतद्विषये राज्ञो भृतिरूपो भागः &198\\
परस्वामिकनष्टलब्धपशूनां भृतिः &199\\
निधिप्राप्तौ कर्तव्यम् &,,\\
चौरापहृतधने विशेषः &200\\
\textbf{\large संभूयसमुत्थानम्} &200\textendash\ 202\\
\textbf{\large दत्ताप्रदानिकम्} &202\textendash\ 206\\
तत्स्वरूपम् &202\\
अदेयानि &,,\\
देयम् &203\\
दत्तम् &204\\
अदत्तम् &,,\\
उत्कोचस्वरूपम् &205\\
उपधिप्रयुक्तदानाधमनविषये नियमः &205\textendash\ 206\\
\textbf{\large अभ्युपेत्याशुश्रूषा} &206\textendash\ 211\\
तत्स्वरूपम् &206\\
तत्त्रैविध्यम् &206\\
दास्यं विप्रं नैव कारयेत् &206\textendash\ 208\\
दासभेदाः &208\textendash\ 209\\
दासमोक्षः कथं कदा च भवति& 209\textendash\ 210\\
दासमोक्षस्य विधिः &210\textendash\ 211\\
\textbf{\large वेतनादानम्} &211\textendash\ 214\\
तत्स्वरूपम् &211\\
भृतिमनिश्चित्य कर्मकरणे भृतिनियमः &212\\
प्रतिश्रुत्य कर्माकरणे विशेषः &212\\
स्वामिसेवकयोर्विवादे &212\\
प्रस्थानविघ्नकृत्सेवकादौ विशेषः &213\\
भाटकम् &213\textendash\ 214\\
\textbf{\large संविद्व्यतिक्रमः} &214\textendash\ 216\\
तत्स्वरूपम् &214\\
तत्र राजकृत्यम् &215\\
संविल्लङ्घने दण्डाः &216\\
\textbf{\large क्रीतानुशयः} &216\textendash\ 217\\
तत्स्वरूपम् &216
\end{tabular}}

\newpage
% INTRODUCTION TO VYAVAHĀRAMAYŪKHA LVII

\noindent {\small 
\begin{tabular}{m{30em} m{5em}}
& PAGES\\
पण्यपरीक्षाकालः &216\\
परिभुक्ते पण्यं क्रेतुरेव &217\\
\textbf{\large विक्रीयासंप्रदानम्} &217\textendash\ 219\\
तत्स्वरूपम् &217\\
मूल्यं गृहीत्वा पण्यदाने दण्डः &218\\
तदपवादः &,,\\
पण्ये दीयमाने अगृह्णतो विशेषः &,,\\
मत्तोन्मत्तादिविक्रीते विशेषः &,,\\
सदोषपण्यविक्रये दण्डः &219\\
\textbf{\large स्वामिपालविवादः} &219\textendash\ 221\\
पालदोषेण पशुनाशे दण्डः &219\\
पश्वादिमरणनिश्चायकानि लिङ्गानि &219\\
गवादिप्रचारार्था कियती भूः कर्तव्या &219\textendash\ 221\\
परसस्यादिभक्षणे पशुस्वामिनो दण्डः &220\\
तदपवादाः &221\\
\textbf{\large सीमाविवादः} &221\textendash\ 226\\
सीमाज्ञानोपायाः &221\\
तत्र साक्षिणां विशेषः &,,\\
ज्ञातृचिह्नाभावे राज्ञा सीमा निश्चेतव्या &222\\
गृहादिनिवेशप्रभृतिभोगो रक्षणीयः &223\\
मेखलाभ्रमादिषु विशेषः &,,\\
संसरणनिरोधप्रतिषेधः &224\\
मर्यादायाः प्रभेदे दण्डः &225\\
सीमामध्यजातवृक्षादीनामुबोपगः साधारणः & ,,\\
स्वामिनोनिवेद्यैव सेतोः प्रवर्तयितरि &225\textendash\ 226\\
\textbf{\large वाक्पारुष्यम्} &227\textendash\ 229\\
प्रथमोत्तममध्यमभेदेन त्रैविध्यम् &227\\
दण्डाः &227\\
ब्राह्मणाद्याक्रोशे दण्डाः &227\\
शूद्रस्य विप्रादीनाक्रोशतो दण्डः &227\\
मात्रादीनामाक्षारणे दण्डः &228\\
अङ्गादिविनाशवाचिके दण्डः &228
\end{tabular}}

\newpage
% LVIII SANSKRIT ANALIYSIS OF VYAVAHĀRAMAYŪKHA 

\noindent {\small 
\begin{tabular}{m{30em} m{5em}}
& Page.\\
अश्लीलाक्षेपे दण्डः &229\\
क्वचिद्दण्डार्धकल्पना &229\\
\textbf{\large दण्डपारुष्यम्} &229\textendash\ 232\\
विविधा दण्डाः &229\textendash\ 231\\
पशुताडनादौ दण्डाः &232\\
वृक्षोपघाते दण्डः &,,\\
\textbf{\large स्तेयम्} &232\textendash\ 233\\
क्षुद्रमध्यमोत्तमद्रव्यादिभेदाः &232\textendash\ 233\\
प्रकाशतस्कराः &233\textendash\ 234\\
अप्रकाशतस्कराः &235\\
नवविधाश्चौराः &,,\\
स्त्रीहरणे दण्डः &236\\
पशुहरणे दण्डः &,,\\
सुवर्णरजतरत्नादिस्तेये दण्डाः &237\\
विविधा दण्डाः स्तेयविषये& 237\textendash\ 238\\
\textbf{\large साहसम्} &238\textendash\ 245\\
तल्लक्षणम्& 238\\
तद्भेदाः &239\\
प्रकाशघातकाः &,,\\
उपांशुघातकाः &,,\\
आरम्भकृत्सहायादीनां दण्डः &240\\
विप्रदण्डे विशेषः &,,\\
आततायिस्वरूपम् &240\textendash\ 241\\
आततायिवधविचारः &241\textendash\ 243\\
हीनमध्योत्कृष्टद्रव्यहरणे दण्डाः &243\\
साहसप्रयोजकस्य दण्डः &,,\\
साध्वीं विप्रां बलाद्गच्छतो दण्डः &244\\
बलात्सजातीयभार्यागमने दण्डः &,,\\
हीनमध्यमोत्तमसाहसेषु दण्डाः &244\textendash\ 245\\
\textbf{\large स्त्रीसंग्रहणम्} &245\\
छलेन सजातीयपरभार्यागमने दण्डः &245\\
हीनमध्यमोत्तमत्रिविधस्त्रीसंग्रहणे दण्डः &245
\end{tabular}}

\newpage
% INTRODUCTION TO VYAVAHĀRAMAYŪKHA LIX

\noindent {\small 
\begin{tabular}{m{30em} m{5em}}
& Page. \\
दुर्वृत्तस्य परस्त्रिया सह संभाषणे दण्डः &246\\
निवारितयोः स्त्रीपुरुषयोर्दण्डः &,,\\
उभयोरनुरागकृते संभोगे दण्डः &,,\\
व्यभिचरितस्वनुलोमासु गमने दण्डः &,,\\
व्यभिचरितास्वनुलोमासु गमने दण्डः &247\\
प्रातिलोम्येन गमने दण्डः &247\textendash\ 248\\
मातृतुल्यस्त्रीगमने दण्डः &248\\
ब्राह्मणस्य दास्यादिगमने दण्डः &249\\
स्त्रीकृते संग्रहणे तस्या दण्डः &250\\
ब्राह्मण्यादीनां शूद्रादिगमने दण्डः &250\\
व्यभिचारनिश्चयोपायाः &,,\\
\textbf{\large स्त्रीपुंधर्मः} &251\\
\textbf{\large द्यूतसमाह्वयौ} &252\\
द्यूते कपटकर्तुर्दण्डः &,,\\
राजाज्ञां विना द्यूते दण्डः& ,,\\
द्यूतधर्मस्य समाह्वयेतिदेशः &,,\\
\textbf{\large प्रकीर्णकम्} &253\textendash\ 256\\
विविधेषु अपराधेषु दण्डाः &253\textendash\ 255\\
सर्वस्वापहारदण्डे विशेषः &255 \textendash\ 256\\
अन्यायाद्गृहीतस्य दण्डस्य विधिः &256
\end{tabular}}

\begin{center}
\rule{0.2\linewidth}{0.5pt}
\end{center}

\newpage
\thispagestyle{empty}

\begin{center}
\textbf{\LARGE ERRATA.}\\

\vspace{2mm}
\textbf{\large TEXT}\\

\rule{0.2\linewidth}{0.5pt}
\end{center}

\noindent {\small 
\begin{tabular}{m{1em} m{1em} m{1em} m{2em} m{15em}}
58,& 1.& 2& read& धटकर्क for घकटर्क \\
70,& 1.& 1& ,,& तदा for तद\\
90,& 1.& 18& ,,& यत्तु for यत्त \\
92,& 1.& 10& ,,& क्लृप्तम् for कप्तम् \\
175,& 1.& 8& ,,& प्रयोजकेसति for प्रयोजके सति\\
179,& 1.& 16& ,,& निबन्धं वा for निबन्ध वा \\
185,& 1.& 9& ,,& ६.२७ for ६.२५ \\
194,& 1.& 9& ,,& याचितेदत्ते for याचिते दत्ते \\
197,& 1.& 6 &,,& वाप्यविशोधयन् for वप्याविशोधयन् \\
206,& 1.& 9& ,,& शुश्रूषां for शुश्रषां\\
214,& 1.& 3& ,,& स भाटकम् for सभाटकम् \\
227,& 1.& 18& ,,& माक्रुश्य for माक्रश्य \\
230,& 1.& 2 &,,& आक्रष्टस्तु for आक्रष्टस्तु \\
247,& 1.& 10& ,,& शूद्रोगुप्तां for शद्रो गुप्तां \\
252,& 1.& 15& ,,& तदभिन्न for तद्भिन्न\\
256,& 1.& 10&,,& चर्मण्वतीतरणिजाशुभ for चर्मण्वती तरणिजा शुभ
\end{tabular}}

\begin{center}
\textbf{\large Notes}
\end{center}

\noindent {\small 
\begin{tabular}{m{1em} m{1em} m{1em} m{2em} m{15em}}
1,& 1.& 10& read& द्विज for दिज\\
14,& 1.& 22& ,,& पौर for पैर\\
64,& 1.& 6& ,,& पश्यतोब्रुवत for पश्यतो ब्रुवत \\
79,& 1.& 30& ,,& तत्पूर्वमावेदितं for तत्षूर्वमावेदितं \\
96,& 1.& 3& ,,& शिक्यद्वय for शिक्यद्व \\
168,& 1.& 19& ,,& त्रिभिरृणैरृणवा for त्रिभिरृणैर्वा \\
196,& 1.& 19& ,,& सापिण्ड्यं for सापिण्ढ्यं \\
229,& 1.& 31 &,,& ms. for mss. \\
345,& 1.& 13& ,,& भ्रातृ for भातृ
\end{tabular}}

\begin{center}
\rule{0.2\linewidth}{0.5pt}
\end{center}

\newpage
\thispagestyle{empty}

\begin{center}
\textbf{\large भट्टनीलकण्ठकृतः}\\

\vspace{2mm}
\textbf{\huge व्यवहारमयूखः~।}
\end{center}

\begin{quote}
{\vy \renewcommand{\thefootnote}{1}\footnote{G reads before the verse उक्त्वा नृप elaven other verses for which see Introduction.}उक्त्वा नृपनयरीतिं नत्वा भास्वत्पदाम्बुजं सम्यक्~।\\
विरचयति नीलकण्ठो व्यवहारविनिर्णये किंचित्~॥~१~॥

\renewcommand{\thefootnote}{2}\footnote{C, G, H, K, N omit the verses {\qt द्विज \ldots हीयते}.}द्विजराजैकमूर्धन्यं वृषाध्यक्षं शिवान्वितम्~।\\
काश्यां सर्वोपदेष्टारं भावये शंकरं गुरुम्~॥~२~॥

विरोधिमार्गद्वयदर्शनार्थं द्वेधा बभूवात्र परः पुमान्यः~।\\
श्रीशंकरो भट्ट इहैकरूपो मीमांसकाद्वैतमुरीचकार~॥~३~॥

प्रतारकैरादृतमत्र किंचिन्मया तु निर्मूलतया तदुज्झितम्~।\\
\renewcommand{\thefootnote}{3}\footnote{B, D ऊनोक्तितानेन हि तेन.}ऊनोक्तिता नात्र हि तेन काचित्खपुष्पहीनापचितिर्न हीयते~॥~४~॥}
\end{quote}

विप्रतिपद्यमाननरान्तरगताज्ञाताधर्मज्ञापनानुकूलो व्यापारो व्यवहारः~। वादिप्रतिवादिकर्तृकः \renewcommand{\thefootnote}{4}\footnote{H संभोगसाक्षि०}संभवद्भोगसाक्षिप्रमाणको विरोधिकोटिव्यवस्थापनानुकूलो वा व्यापारः सः~। संप्रतिपत्त्युत्तरे तु \renewcommand{\thefootnote}{5}\footnote{H, K .कृूलो व्यापारः}व्यवहारपदप्रयोगो भाक्त इति मदनरत्ने~। \renewcommand{\thefootnote}{6}\footnote{G व्यावृत्त्यर्थं द्वितीयदलं for व्यावृत्त्यर्थमुत्तरदलम्; H व्यावृत्त्यर्थमनन्तरद्वितीयपटलम्.}वादवितण्डादिव्यावृत्त्यर्थमुत्तरदलम्~। 

\begin{center}
\textbf{\Large ॥~अथ व्यवहारपदानि~॥~१~॥}
\end{center}

अथ तत्पदम्~। याज्ञवल्क्यः ( २. ५ )

\begin{quote}
{\vy स्मृत्याचारव्यपेतेन मार्गेणाधर्षितः परैः~।\\
आवेदयति \renewcommand{\thefootnote}{7}\footnote{B, C, D, K यद्राज्ञे for चेद्राज्ञे.}चेद्भाज्ञे व्यवहारपदं हि तत्~॥}
\end{quote}

\fancyhead[CE,CO]{व्यवहारपदानि}
\fancyhead[RO]{[ $\S$ १}
\fancyhead[LE]{$\S$ १ ]}
\fancyhead[RE,LO]{\thepage}
\cfoot{}
\newpage
%%%%%%%%%%%%%%%%%%%%%%%%%%%%%%%%%%%%%%%%%%%%%%%%%%%%%
\renewcommand{\thepage}{\devanagarinumeral{page}}
\setcounter{page}{2}

% § १ ] व्यवहारपदानि २

आधर्षितस्तिरस्कृतः~॥ तस्याष्टादश भेदानाह मनुः ( ८. ४\textendash\ ७ )

\begin{quote}
{\vy तेषामाद्यमृणादानं निक्षेपोऽस्वामिविक्रयः~।\\
संभूय च समुत्थानं दत्तस्यानपकर्म च~॥

वेतनस्यैव चादानं संविदश्च व्यतिक्रमः~।\\
क्रयविक्रयानुशयो विवादः स्वामिपालयोः~॥

सीमाविवादधर्मश्च पारुष्ये दण्डवाचिके~।\\
स्तेयं च साहसं चैव स्त्रीसंग्रहणमेव च~॥

स्त्रीपुन्धर्मो विभागश्च द्यूतमाह्वयं\renewcommand{\thefootnote}{1}\footnote{B, D, E, F, H द्यूतमाह्वयमेव for द्यूतमाह्वय एव.} एव च~।\\
पदान्यष्टादशैतानि व्यवहारस्थिताविह~॥} इति~॥
\end{quote}

अनपकर्माप्रदानम्~। अनुशयः पश्चात्तापः~। द्यूतमप्राणिकरणिका क्रीडा~। प्राणिकरणिका समाह्वयः\renewcommand{\thefootnote}{2}\footnote{G, H साह्वयः for समाह्वयः,}~॥ अत्र

\begin{quote}
{\vy मनुष्यमारणं स्तेयं परदाराभिमर्शनम्~।\\
पारुष्यमुभयं चेति साहसं स्याच्चतुर्विधम्~॥} इति
\end{quote}

बृहस्पत्युक्तेः \renewcommand{\thefootnote}{3}\footnote{B, D, E अत्र मनुष्यदण्डपारुष्याणां for स्त्रीसंग्रहणवाक्यारुष्यदण्डपारुष्याणाम्; A, F स्त्रीसंग्रहणवाग्दण्डपारुष्याणाम्}स्त्रीसंग्रहणवाक्पारुष्यदण्डपारुष्याणां साहसभेदत्वेऽपि पृथङ्निर्देशो गोबलीवर्दन्यायेन~। एतेषां चाष्टादशपदानां स्वरूपमग्रे व्यक्तीकरिष्यते~॥~१~॥

\begin{center}
\textbf{\Large ॥~अथ व्यवहारमातृकाः~॥~२~॥}
\end{center}

बृहस्पतिः~।

\begin{quote}
{\vy दुर्गमध्ये गृहं कुर्याज्जलवृक्षाश्रितं\renewcommand{\thefootnote}{4}\footnote{वीर (p. 10) reads {\qt वृक्षान्वितं} and परा मा वृक्षावृतं, and स्मृतिच {\qt जलमध्योच्छ्रितं पृथु}} पृथक्~।\\
प्राग्दिशि प्राङ्मुखीं तस्य लक्षण्यां कल्पयेत्सभाम्~॥}
\end{quote}

\newpage
\fancyhead[CE,CO]{व्यवहारमातृकाः\textendash\ सभाक्षलणम्}
\fancyhead[RO]{[ $\S$ २}
\fancyhead[LE]{$\S$ २ ]}

% ३ व्यवहारमातृकाः \textendash\ सभाक्षलणम् [ § २

सैव च धर्माधिकरणम्~। 

\begin{quote}
{\vy धर्मशास्त्रविचारेण सारासारविवेचनम्\renewcommand{\thefootnote}{1}\footnote{B, D, E विवेचना for विवेचनम्; C विवेचने. N omits the Verse धर्मशास्त्र. स्मृतिच (p. 19 ) reads मूलसारविवेचनं, परा मा reads मूलशास्त्रविवेचनं and वीर (p. 10) धर्मशास्त्रानुसारेण अर्थशास्त्रविवेचनम्.}~।\\
यत्राधिक्रियते स्थाने धर्माधिकरणं हि तत्~॥}
\end{quote}

इति कात्यायनोक्तेः~। मनुः ( ८. १\textendash\ २ )

\begin{quote}
{\vy व्यवहारान्दिदृक्षुस्तु ब्राह्मणैः सह पार्थिवः~।\\
मन्त्रज्ञैर्मन्त्रिभिश्चैव विनीतः प्रविशेत्सभाम्~॥

विनीतवेषाभरणः पश्येत्कार्याणि कार्यिणाम्~॥}
\end{quote}

याज्ञवल्क्यः ( २. १)

\begin{quote}
{\vy व्यवहारान्नृपः पश्येद्विद्वद्भिर्ब्राह्मणैः सह~।\\
धर्मशास्त्रानुसारेण क्रोधलोभविवर्जितः~॥}
\end{quote}

नृपः प्रजापालनाधिकृतो यः कश्चिन्न क्षत्रिय एव~। कात्यायनः

\begin{quote}
{\vy सप्राङ्विवाकः सामात्यः सब्राह्मणपुरोहितः~।\\
ससभ्यः प्रेक्षको राजा स्वर्गे तिष्ठति धर्मतः~॥}
\end{quote}

अत्र ब्राह्मणोऽनियुक्तः~। सभ्यास्तु नियुक्ताः~। तथा चोक्तम् (नारद १५ ३.२)

\begin{quote}
{\vy नियुक्तो वानियुक्तो वा धर्मज्ञो वक्तुमर्हति~॥} इति~।
\end{quote}

प्राङ्विवाकस्वरूपमाह बृहस्पतिः

\begin{quote}
{\vy विवादे पृच्छति प्रश्नं प्रतिप्रश्नं तथैव च~।\\
प्रियपूर्वं प्राग्वदति प्राङ्विवाकस्ततः स्मृतः~॥} इति~।
\end{quote}

\newpage
\fancyhead[CE,CO]{व्यवहारमातृकाः\textendash\ अमात्यस्वरूपम् }

% §२ ] व्यवहारमातृकाः \textendash\ अमात्यस्वरूपम् ४

अमात्यस्वरूपमाह व्यासः 

\vspace{-2mm}
\begin{quote}
{\vy सर्वशास्त्रार्थवेत्तारमलुब्धं न्यायभाषिणम्~।\\
विप्रं प्राज्ञं क्रमायातममात्यं स्थापयेद्द्विजम्~॥} इति~।
\end{quote}

अत्र पुनर्द्विजग्रहणं विप्राभावे क्षत्रियवैश्ययोरुपादानार्थम्~। तथा च कात्यायनः

\vspace{-2mm}
\begin{quote}
{\vy यत्र \renewcommand{\thefootnote}{1}\footnote{The व्यव मा, reads यदि विप्रो न विद्वान् स्यात् (p. 279)}विद्वान्न विप्रः स्यात्क्षत्रियं तत्र योजयेत्~।\\
वैश्यं वा धर्मशास्त्रज्ञं शूद्रं यत्नेन वर्जयेत्~॥} इति~।
\end{quote}

सभ्यांश्चाह याज्ञवल्क्यः ( २. २. )

\vspace{-2mm}
\begin{quote}
{\vy श्रुताध्ययनसंपन्ना धर्मज्ञाः सत्यवादिनः~।\\
राज्ञा सभासदः कार्या रिपौ मित्रे च ये समाः~॥} इति~।
\end{quote}

एतेषां संख्यामाह बृहस्पतिः

\vspace{-2mm}
\begin{quote}
{\vy लोकवेदज्ञधर्मज्ञाः सप्त पञ्च त्रयोऽपि वा~।\\
यत्रोपविष्टा विप्राः स्युः सा यज्ञसदृशी सभा~॥} इति~।
\end{quote}

\renewcommand{\thefootnote}{2}\footnote{B, F बृहस्पतिः for स एव; E omits स एव.}स एव

\vspace{-2mm}
\begin{quote}
{\vy शब्दाभिधानतत्त्वज्ञौ गणनाकुशलौ शुची~।\\
नानालिपिज्ञौ कर्तव्यौ राज्ञा गणकलेखकौ~॥}
\end{quote}

शब्दः शब्दशास्त्रम्~। अभिधानं कोशः~। कात्यायनः

\vspace{-2mm}
\begin{quote}
{\vy श्रोतारो वणिजस्तत्र कर्तव्या न्यायदर्शिनः~॥}
\end{quote}

तत्र सभायाम्~। बृहसतिः

\vspace{-2mm}
\begin{quote}
{\vy आकारणे रक्षणे च साक्ष्यर्थिप्रतिवादिनाम्~।\\
सभ्याधीनः सत्यवादी कर्तव्यस्तु स्वपूरुषः~॥}
\end{quote}

\newpage
\fancyhead[CE,CO]{व्यवहारमातृकाः\textendash\ साध्यपालः}
% ५ व्यवहारमातृकाः \textendash\ साध्यपालः [§ २

अयं च शूद्र एव~। तथा च व्यासः 

\vspace{-2mm}
\begin{quote}
{\vy साध्यपालस्तु कर्तव्यो राज्ञा साध्यस्य साधकः~।\\
क्रमायातो दृढः शूद्रः सभ्यानां च मते स्थितः~॥} इति~।
\end{quote}

याज्ञवल्क्यः ( २. ३. )

\vspace{-2mm}
\begin{quote}
{\vy अपश्यता कार्यवशाद्व्यवहारान्नृपेण तु~।\\
सभ्यैः सह नियोक्तव्यो ब्राह्मणः सर्वधर्मवित्\renewcommand{\thefootnote}{1}\footnote{H सत्यधर्मवित् for सर्वधर्मवित्.}~॥}
\end{quote}

नृपाध्यक्षादीनां कार्यमाह बृहस्पतिः 

\vspace{-2mm}
\begin{quote}
{\vy वक्ताध्यक्षो नृपः शास्ता सभ्याः कार्यपरीक्षकाः~।\\
गणको गणयेदर्थं लिखेन्न्यायं च लेखकः~॥}
\end{quote}

स एव

\vspace{-2mm}
\begin{quote}
{\vy पूर्वामुखस्तूपविशेद्राजा सभ्या उदङ्मुखाः~।\\
गणकः पश्चिमास्यस्तु लेखको दक्षिणामुखः~॥}
\end{quote}

राजसभातो \renewcommand{\thefootnote}{2}\footnote{B निर्णायकं तमाह for निर्णायकान्तरमाह; D, E निर्णायकान्तमाह}निर्णायकान्तरमाह याज्ञवल्क्यः ( २. ३० )

\vspace{-2mm}
\begin{quote}
{\vy \renewcommand{\thefootnote}{3}\footnote{E omits the words from नृपेणाधिकृताः to पूग}नृपेणाधिकृताः पूगाः श्रेणयोऽथ कुलानि च~।\\
पुुूर्वं पूर्वं गुरु ज्ञेयं व्यवहारविधौ नृणाम्~॥}
\end{quote}

नृपेणाधिकृताः प्राङ्विवाकादयः~। पूगाः \renewcommand{\thefootnote}{4}\footnote{C, G, K एकग्रामस्थविजातीयानां for एकग्रामस्थानां विजातीयानां; D, B omit एकग्रामस्थानाम्.}नानाकर्मोपजीविनामेकग्रामस्थानां विजातीयानां समूहाः~। पूगविपरीताश्च श्रेणयः~। कुलानि ज्ञातिसंबन्धिबन्धूनां समूहाः~॥ बृहस्पतिरपि 

\vspace{-2mm}
\begin{quote}
{\vy ये \renewcommand{\thefootnote}{5}\footnote{B, D, E ये वारण्यचराः for ये चारण्यचराः}चारण्यचरास्तेषामरण्ये करणं भवेत्~।\\
सेनायां सैनिकानां तु सार्थेषु वणिजां तथा~॥}
\end{quote}

\newpage
\fancyhead[CE,CO]{व्यवहारमातृकाः\textendash\ व्यवहारदर्शनकालः}
% §२ ] व्यवहारमातृकाः \textendash\ व्यवहारदर्शनकालः ६

\noindent
करणं सभा~।

व्यवहारदर्शने कालमाह कात्यायनः

\begin{quote}
{\vy \renewcommand{\thefootnote}{1}\footnote{B, C, D, E, G, H, M सभास्थानेषु for सभास्थाने तु; H सभ्यां स्थाने तु; K सभ्याः स्थानेषु.}सभास्थाने तु पूर्वाह्णे कार्याणां निर्णयं नृपः~।\\
कुर्याच्छास्त्रप्रणीतेन मार्गेणामित्रकर्शनः~॥

दिवसस्याष्टमं भागं मुक्त्वा भागत्रयं तु यत्~।\\
स कालो व्यवहाराणां शास्त्रदृष्टः परः स्मृतः~॥} इति~।
\end{quote}

\renewcommand{\thefootnote}{2}\footnote{B, G, K आद्ययामाद्यार्धमष्टमो भाग एतदुत्तरमावर्तनात् for आद्ययामार्धमष्टमो भागः तदुत्तरमावर्तनात्.}आद्ययामार्थमष्टमो भागस्तदुत्तरमावर्तनात्प्राचीनं भागत्रयम्~॥ वर्ज्यास्तिथीश्चाह संवर्तः

\begin{quote}
{\vy चतुर्दशी ह्यमावास्या पौर्णमासी तथाष्टमी~।\\
तिथिष्वासु न \renewcommand{\thefootnote}{3}\footnote{परा मा reads पश्येत्तु व्यवहारान्न नित्यशः वीर {\qt पश्येत्तु व्यवहारविचक्षणः}}पश्येत व्यवहारान्विचक्षणः~॥} इति~।
\end{quote}

\renewcommand{\thefootnote}{4}\footnote{C omits बृहस्थतिः before पूर्वाह्णे.}बृहस्पतिः

\begin{quote}
{\vy पूर्वाह्णे तामधिष्ठाय बृद्धामात्यानुजीविभिः~।\\
पश्येत्पुराणधर्मार्थशास्त्राणि शृणुयात्तथा~॥

तां सभाम्~। अर्थशास्त्रं नीतिशास्त्रम्~॥}
\end{quote}

धर्मशास्त्रार्थशास्त्रयोर्विप्रतिपत्तावाह नारदः ( १. ३९)

\begin{quote}
{\vy यत्र विप्रतिपत्तिः स्याद्धर्मशास्त्रार्थशास्त्रयोः~।\\
अर्थशास्त्रोक्तमुत्सृज्य धर्मशास्त्रोक्तमाचरेत्~॥} इति~।
\end{quote}

धर्मशास्त्रयोर्विरोधे त्वाह याज्ञवल्क्यः ( २. २१ )

\begin{quote}
{\vy स्मृत्योर्विरोधे न्यायस्तु बलवान्व्यवहारतः~।}
\end{quote}

\newpage
\fancyhead[CE,CO]{व्यवहारमातृकाः\textendash\ न्यायानालोचने दोषः}
% ७ व्यवहारमातृकाः \textendash\ न्यायानालोचने दोषः [§ २ 

न्यायमनालोचयतो दोषमाह बृहस्पतिः 

\begin{quote}
{\vy केवलं शास्त्रमाश्रित्य न कर्तव्यो हि\renewcommand{\thefootnote}{1}\footnote{परा मा and व्यव मा reads न कर्तव्यो विनिर्णयः} निर्णयः~।\\
युक्तिहीने विचारे तु धर्महानिः प्रजायते~॥} इति~।
\end{quote}

देशाचाराद्यप्यालोचयेदित्याह बृहस्पतिः

\begin{quote}
{\vy देशजातिकुलानां च ये धर्माः प्राक्प्रवर्तिताः~।\\
तथैव ते पालनीयाः प्रजा\renewcommand{\thefootnote}{2}\footnote{ H, K प्रजाः प्रक्षुभ्यते for प्रजा प्रक्षुभ्यते.} प्रक्षुभ्यतेऽन्यथा~॥

जनापरक्तिर्भवति बलं कोशश्च नश्यति~।\\
\renewcommand{\thefootnote}{3}\footnote{A, E, F, K उदूह्यते for उदुह्यते. E reads सुतो forसुता.}उदुह्यते दाक्षिणात्यैर्मातुलस्य सुता द्विजैः~॥

\renewcommand{\thefootnote}{4}\footnote{A, F मध्ये देशे for मध्यदेशे}मध्यदेशे कर्मकराः शिल्पिनश्च \renewcommand{\thefootnote}{5}\footnote{D, E गवादिनः for गवाशिनः}गवाशिनः~।\\
मत्स्यादाश्च नराः \renewcommand{\thefootnote}{6}\footnote{H सर्वे for पूर्वे.}पूर्वे व्यभिचाररताः स्त्रियः~॥

उत्तरे मद्यपा नार्यः स्पृश्या नॄणां रजस्वलाः~।\\
अनेन कर्मणा नैते प्रायश्चित्तदमार्हकाः~॥} इति~।
\end{quote}

पूर्वे प्राच्याः~। क्वचित्सर्व इति पाठः~। सर्वे ब्राह्मणादयः~। दमो दण्डः~। \renewcommand{\thefootnote}{7}\footnote{C यत्त्वन्यत्र प्रायश्चित्तादि for यत्त्वत्र प्रायश्चित्तादि.}यत्त्वत्र प्रायश्चित्तादिस्मरणं \renewcommand{\thefootnote}{8}\footnote{B, D, E व्याख्यानुपात्त for वाक्यानुपात्तः C for तदन्तवाक्यानुपात्त for तदेतद्वाक्यानुपात्त; K वाक्यमुपात्तदेश.}तदेतद्वाक्यानुपात्तदेशपरमिति केचित्~। अन्ये तु प्रायश्चित्तरूपो\renewcommand{\thefootnote}{9}\footnote{C रूपो दमः for रूपो यो दमः} यो दम इति व्याचक्षाणा दण्डाभावमात्रम् अन्यदेशे दण्डः प्रायश्चित्तं चेत्याहुः~। व्यासः

\newpage
\fancyhead[CE,CO]{
व्यवहारमातृकाः\textendash\ वणिक्शिल्पिप्रभृतिषु तज्ज्ञैर्निर्णयः}
% § २ ] व्यवहारमातृकाः \textendash\ वणिक्शिल्पिप्रभृतिषु तज्ज्ञैर्निर्णयः ८

\begin{quote}
{\vy वणिक्शिल्पिप्रभृतिषु कृषिरङ्गोपजीविषु~।\\
अशक्यो निर्णयो ह्यन्यैस्तज्ज्ञैरेव तु कारयेत्~॥}
\end{quote}

मनुः ( ८. ३९० )

\begin{quote}
{\vy आश्रमेषु द्विजातीनां कार्ये विवदतां मिथः~।\\
न विब्रूयान्नृपो धर्मं चिकीर्षन्हितमात्मनः~॥} इति~।
\end{quote}

कात्यायनः

\begin{quote}
{\vy काले कार्यार्थिनं पृच्छेत्प्रणतं पुरतः स्थितम्~।\\
किं कार्यं का च ते पीडा मा\renewcommand{\thefootnote}{1}\footnote{व्यव मा reads भैषीर्याहि मानद.} भैषीर्ब्रूहि मानव~॥

केन कस्मिन्कदा कस्मात्पृच्छेदेवं \renewcommand{\thefootnote}{2}\footnote{B, D, E, F, G, H, K समागतं for सभागतम् व्यव मा reads सभां गतः}सभागतम्~।\\
एवं पृष्टः स यद्ब्रूयात्ससभ्यैर्ब्राह्मणैः सह~॥

विचार्य कार्यं न्याय्यं चेदाह्वानार्थमतः परम्~।\\
मुद्रां वा निक्षिपेत्तस्मिन्पुरुषं वा समादिशेत्~॥}
\end{quote}

नारदः ( १. ४७ ) 

\begin{quote}
{\vy \renewcommand{\thefootnote}{3}\footnote{B, C, D, E, G, K वक्तव्यार्थे for वक्तव्येर्थे. नारद reads न तिष्ठन्तम्.}वक्तव्येर्थे ह्यतिष्ठन्तमुत्क्रामन्तं च तद्वचः~।\\
आसेधयेद्विवादार्थी\renewcommand{\thefootnote}{4}\footnote{B विवादार्थं for विवादार्थी. अपरार्क reads विवादार्थम्.} यावदाह्वानदर्शनम्~॥}
\end{quote}

आसेधस्य चातुर्विध्यमाह स एव ( नारद १. ४८ ) 

\begin{quote}
{\vy स्थानासेधः कालकृतः प्रवासात्कर्मणस्तथा~।\\
चतुर्विधः स्यादासेधस्तमासिद्धो\renewcommand{\thefootnote}{5}\footnote{नारद reads {\qt सेधो नासिद्धस्तं विलङ्घयेत्}} न लङ्घयेत्~॥}
\end{quote}

आसिद्धस्यासेधातिक्रमे दण्डमाह स एव (नारद १. ५१ ) 

\begin{quote}
{\vy आसेधयोग्य आसिद्ध उत्क्रामन्दण्डमर्हति~।}
\end{quote}

\newpage
\fancyhead[CE,CO]{व्यवहारमातृकाः\textendash\ क्वचिदासेद्धुरेव दण्डः}
%९ व्यवहारमातृकाः \textendash\ क्वचिदासेद्धुरेव दण्डः [§ २ 

क्वचिदासेद्धुरेव दण्डमाह स एव\renewcommand{\thefootnote}{1}\footnote{Vide S. B. E. vol. 33 p. 235.}

\begin{quote}
{\vy यस्त्विन्द्रियनिरोधेन व्याहारोच्छ्वसनादिभिः~।\\
\renewcommand{\thefootnote}{2}\footnote{B, E, G असेधयेदनासेध्ये for आसेधयेदनासेध्यैः स्मृति च reads न त्वतिक्रमी.}आसेधयेदनासेध्यैः स दण्ड्यो न त्वतिक्रमात्~॥} इति~।
\end{quote}

क्वचिदासेधातिक्रमे दण्डाभावमाह नारदः ( १. ४९)

\begin{quote}
{\vy नदीसन्तारकान्तारदुर्देशोपप्लवादिषु~।\\
आसिद्धस्तं परासेधमुत्क्रामन्ना\renewcommand{\thefootnote}{3}\footnote{B उत्क्रमन् for उत्क्रामन्.}पराध्रुयात्~॥} इति~।
\end{quote}

\renewcommand{\thefootnote}{4}\footnote{G अनासेध्यासेद्धुः for अनासेध्यानासेद्धुः}अनासेध्यानासेद्धुर्दण्डमाह कात्यायनः 

\begin{quote}
{\vy आसेधयंस्त्वनासेध्यं राज्ञा शास्य इति स्थितिः~।} इति~।
\end{quote}

अनासेध्यानाह स एव~। 

\begin{quote}
{\vy \renewcommand{\thefootnote}{5}\footnote{A, F, G, K वृक्षं पर्वत for वृक्षपर्वत; G, K पर्वतं वारूढा for पर्वतमारूढा.}वृक्षपर्वतमारूढा हस्त्यश्वरथनौस्थिताः~।\\
विषमस्थाश्च ते सर्वे नासेध्याः कार्यसाधकैः~॥

व्याध्यार्ता व्यसनस्थाश्च यजमानस्तथैव च~।}
\end{quote}

\renewcommand{\thefootnote}{6}\footnote{A, B. D, E, F, G, M omit आह्वाने व्यवस्थामाह नारदः}आह्वाने व्यवस्थामाह नारदः

\begin{quote}
{\vy अकल्पबालस्थविरविषमस्थक्रियाकुलान्~॥ 

कार्यातिपातिव्यसनिनृपकार्योत्सवाकुलान्~।\\
मत्तोन्मत्तप्रमत्तार्तभृत्यान्नाह्वानयेन्नृपः~॥

न हीनपक्षां युवतिं कुलेजातां प्रसूतिकाम्~।\\
सर्ववर्णोत्तमां कन्यां ता ज्ञातिप्रभुकाः स्मृताः~॥}
\end{quote}

\newpage
\fancyhead[CE,CO]{व्यवहारमातृकाः\textendash\ आह्वाने व्यवस्था}
% §२] व्यवहारमातृकाः \textendash\ आह्वाने व्यवस्था १०

\begin{quote}
{\vy तदधीनकुटुम्बिन्यः स्वैरिण्यो गणिकाश्च याः~।\\
निष्कुला याश्च पतितास्तासामा\renewcommand{\thefootnote}{1}\footnote{अपरार्क ( p. 606 ) reads आह्वानमर्हति.}ह्वानमिष्यते~॥

ज्ञात्वाभियोगं येऽपि स्युर्वने प्रव्रजितादयः~।\\
तानप्याह्वानयेद्राजा गुरुकार्येष्वकोपयन्~॥

कालं देशं च विज्ञाय कार्याणां च बलाबलम्~।\\
अकल्पादीनपि शनैः शनैराह्वानयेन्नृपः~॥}
\end{quote}

\noindent
यानैरिति क्वचित्पाठः~॥

आहूतोनागच्छन् दण्ड्यः~। तथा च बृहस्पतिः~।

\begin{quote}
{\vy आहूतो \renewcommand{\thefootnote}{2}\footnote{{\qt यस्तु} for यत्र in व्यव. मा. and परा. मा.}यत्र नागच्छे\renewcommand{\thefootnote}{3}\footnote{B, D नागच्छेदर्थाद्बन्धु for नागच्छेद्दर्पाद्बन्धु.}द्दर्पाद्बन्धुकुलान्वितः~।\\
अभि\renewcommand{\thefootnote}{4}\footnote{B, D अभियोगादिरूपेण for अभियोगानुरूपेण.}योगानुरूपेण तस्य दण्डं प्रकल्पयेत्~॥}
\end{quote}

अभियोगभेदेन दण्डभेदमाह कात्यायनः~।

\begin{quote}
{\vy हीने कर्मणि पञ्चाशन्मध्यमे तु शतावरः~।\\
गुरुकार्येषु दण्डः स्यान्नित्यं पञ्चश\renewcommand{\thefootnote}{5}\footnote{C, K पञ्चशतावरः for पञ्चशतावरः.}तावरः~॥} इति~।
\end{quote}

\noindent
आहूतस्यागमने यत्कार्यं तदाह पितामहः~।

सभायाः पुरतः स्थाप्योऽभियुक्तो वादिना तथा~। इति~। तृतीया सहार्थे~। कात्यायनः

\begin{quote}
{\vy \renewcommand{\thefootnote}{6}\footnote{B, F ततोऽभियोक्ता for तत्राभियोक्ता; K यत्राभियोक्ता}तत्राभियोक्ता \renewcommand{\thefootnote}{7}\footnote{The व्यव मा १ स्मृतिच. reads प्राग् ब्रूयात् for प्रब्रूयात्.}प्रब्रूयादभियुक्तस्त्वनन्तरम्~।\\
तयोरन्ते सदस्यास्तु प्राड्विवाकस्ततः परम्~॥}
\end{quote}

\newpage
\fancyhead[CE,CO]{व्यवहारमातृकाः\textendash\ आहूतस्यागमने कार्यम्}
% ११ व्यवहारमातृकाः \textendash\ आहूतस्यागमने कार्यम् [§ २

बृहस्पतिः

\begin{quote}
{\vy अहंपूर्विकया यातावर्थिप्रत्यर्थिनौ यदा~।\\
वादो \renewcommand{\thefootnote}{1}\footnote{B, D दण्डो वर्णानु for वादो वर्णानु.}वर्णानुरूपेण ग्राह्यः पीडामवेक्ष्य वा~॥

अप्रगल्भजडोन्मत्तवृद्धस्त्रीबालरोगिणाम्~।\\
पूर्वो\renewcommand{\thefootnote}{2}\footnote{B, C, G, H, K पूर्वोत्तरं for पूर्वोत्तर., वीर. (p.53) reads पूर्वोत्तरम् \textendash\ A, G, H, K वहेत् for वदेत्.}त्तरे वदेद्बन्धुर्नियुक्तोन्योथवा नरः~॥}
\end{quote}

नारदः\renewcommand{\thefootnote}{3}\footnote{H omits नारदः before अर्थिना.} ( २. २२ )

\begin{quote}
{\vy अर्थिना सन्नियुक्तो वा प्रत्यर्थिप्रे\renewcommand{\thefootnote}{4}\footnote{मिता, परा. मा. read प्रहितः for प्रेरितः.}रितोऽपि वा~।\\
यो यस्यार्थे विवदते तयोर्जयपराजयौ~॥}
\end{quote}

यत्तु कात्यायनः ( $=$नारद २. २३ )

\begin{quote}
{\vy यो न भ्राता न च पिता न पुत्रो न नियोगकृत्~।\\
परार्थवादी दण्ड्यः स्याद्व्यवहारेषु विब्रुवन्~॥} इति
\end{quote}

तदनियुक्तपरम्~॥ क्वचित्प्रतिनिध्यभावस्तेनैवोक्तः~। 

\begin{quote}
{\vy ब्रह्महत्यासुरापाने स्तेये गुर्वङ्गनागमे~।\\
मनुष्यमारणे स्तेये परदाराभिमर्शने~॥

अभक्ष्यभक्षणे चैव कन्याहरणदूषणे~।\\
पारुष्ये कूटकरणे नृपद्रोहे तथैव च~॥

प्रति\renewcommand{\thefootnote}{5}\footnote{F प्रतीवादी for प्रतिवादी.}वादी न \renewcommand{\thefootnote}{6}\footnote{H दान्तः for दाप्यः.}दाप्यः स्यात्कर्ता \renewcommand{\thefootnote}{7}\footnote{E, M तद्धि वदेत् for तद्विवदेत्; C, G, K कर्ता तु विवदेत्. वीर (p.54) reads न दातव्यः कर्तापि विवदेत्स्वयम्.}तद्विवदेत्स्वयम्~।} इति~।
\end{quote}

\newpage
\fancyhead[CE,CO]{व्यवहारमातृकाः\textendash\ प्रत्यर्थिनि स्थापिते अर्थिकार्यम्}
% §२ ] व्यवहारमातृकाः \textendash\ प्रत्यर्थिनि स्थापिते अर्थिकार्यम् १२ 

\noindent
आत्यन्तिकप्रतिनिधिनिषेधार्थं पुनः स्तेयग्रहणम्~। प्रति\renewcommand{\thefootnote}{1}\footnote{B, C, D, F, K, N omit प्रतिवादी प्रतिनिधिः.}वादी प्रतिनिधिः~॥

प्रत्यर्थिनि स्थापितेर्थिनः कार्यमाह याज्ञवल्क्यः ( २. ६. )

\begin{quote}
{\vy प्रत्यर्थिनोग्रतो लेख्यं यथा\renewcommand{\thefootnote}{2}\footnote{F, G, K, N यदावेदितं for यथावेदितम्.}वेदितमर्थिना~।\\
समामासतदर्धाहर्नामजात्यादिचिह्नि\renewcommand{\thefootnote}{3}\footnote{H जात्यादिनिश्चितं for जात्यादिचिह्नितम्.}तम्~॥}
\end{quote}

स्मृत्य\renewcommand{\thefootnote}{4}\footnote{F omits स्मृत्यन्तरे.}न्तरे

\begin{quote}
{\vy अर्थवद्धर्मसंयुक्तं परिपूर्णमनाकुलम्~।\\
साध्यवद्वाचकपदं प्रकृतार्थानुबन्धि च~॥ 

प्रसिद्धमविरुद्धं च निश्चितं साधनक्षमम्~।\\
संक्षिप्तं निखिला\renewcommand{\thefootnote}{5}\footnote{D लिखितार्थ for निखिलार्थम्.}र्थं च देशकालाविरोधि च~॥

वर्षर्तुमासपक्षाहोवेलादेशप्रदेशवत्~।\\
स्थानावसथसाध्याख्याजात्याकारवयोयुतम्~॥

साध्यप्रमाणसंख्यावदात्मप्रत्यर्थिनामवत्~।\\
परात्मपूर्वजानेकराजनामभिरङ्कितम्~॥

क्षमालिङ्गात्मपीडावत्कथिताहर्तृदायकम्~।\\
यदावेदयते राज्ञे तद्भाषेत्यभिधीयते~॥} इति~।
\end{quote}

अत्रोक्तवर्षादीनामाध्यादिषूपयोगो वक्ष्यते~॥ देशादीनां च क्वचिदुपयोग उक्तः स्मृत्यन्तरे

\begin{quote}
{\vy देश\renewcommand{\thefootnote}{6}\footnote{वीर (p. 64) reads देशं for देशः.}श्चैव तथा स्थानं संनिवेशस्तथैव च~।\\
जातिः संज्ञाधि\renewcommand{\thefootnote}{7}\footnote{स्मृतिच reads निवासः for अधिवासः.}वासश्च प्रमाणं क्षेत्र\renewcommand{\thefootnote}{8}\footnote{B, D क्षत्रनाम for क्षेत्रनाम.}नाम च~॥}
\end{quote}

\newpage
\fancyhead[CE,CO]{व्यवहारमातृकाः\textendash\ भाषालक्षणम्}
%१३ व्यवहारमातृकाः \textendash\ भाषालक्षणम् [§ २ 

\begin{quote}
{\vy पितृपैतामहं चैव पूर्वराजा\renewcommand{\thefootnote}{1}\footnote{H पूर्वराजप्रकीर्तितं for पूर्वराजानुकीर्तनम्; K पूर्वराजायुकीर्तनम्.}नुकीर्तनम्~।\\
स्थावरेषु विवादेषु दशैतानि प्रवेश\renewcommand{\thefootnote}{2}\footnote{अपरार्क, स्मृतिच and वीर read निवेशयेत्.}येत्~॥} इति~।
\end{quote}

कात्यायनः

\begin{quote}
{\vy पूर्वपक्षं स्वभावोक्तं प्राड्विवाकोऽभिलेखयेत्~।\\
पाण्डुलेखेन फलके ततः पत्रे विशोधितम्~॥}
\end{quote}

शोधनस्यावधिमाह नारदः 

\begin{quote}
{\vy शोधयेत्पूर्ववादं तु यावन्नोत्तरदर्शनम्~।\\
अवष्टब्धस्योत्तरेण निवृत्तं शोधनं भवेत्~॥

भाषाया उत्तरं यावत्प्रत्यर्थी न निवेशयेत्~।\\
अर्थी तु लेखयेत्तावद्यावद्वस्तु \renewcommand{\thefootnote}{3}\footnote{D विचक्षणं for विवक्षितम्.}विवक्षितम्~॥} इति (नारद २. ७)
\end{quote}

एवं पक्षलक्षणे स्थिते तद्विरुद्धाः पक्षाभासा अर्थसिद्धा अप्युक्ताः स्मृत्यन्तरे

\begin{quote}
{\vy अप्रसिद्धं निराबाधं निरर्थं निष्प्रयोजनम्~।\\
असाध्यं वा विरुद्धं वा पक्षाभासं विवर्जयेत्~॥} इति~।
\end{quote}

अप्रसिद्धं स्वपुष्पं ममापहृतमिति~। निराबाधं मम दीपप्रकाशेनायं व्यवहरतीति~। निरर्थं कचटतपं ममापहृतमिति~। निष्प्रयोजनं मत्प्रातिवेशिकः सुस्वरमधीत इति~। असाध्यं सभ्रूभङ्गं \renewcommand{\thefootnote}{4}\footnote{F उपहसितोनेन for हसितोनेन.}हसितोऽनेनेत्यादि~। विरुद्धं मूकेनाहं शप्त इति~। पुरराष्ट्रादिविरुद्धं वा~। तथा च स्मर्यते

\begin{quote}
{\vy \renewcommand{\thefootnote}{5}\footnote{B, C, D, G, K राज्ञा तु वर्जितो for राज्ञा विवर्जितो.}राक्षा विवर्जितो यस्तु यश्च पौरविरोधकृत्~।\\
राष्ट्रस्य वा समस्तस्य प्रकृतीनां तथैव च~॥}
\end{quote}

\newpage
\fancyhead[CE,CO]{व्यबहारमातृकाः\textendash\ पक्षाभासाः}
% §२ ] व्यबहारमातृकाः \textendash\ पक्षाभासाः १४

\begin{quote}
{\vy अन्ये वा ये पुरग्राम\renewcommand{\thefootnote}{1}\footnote{H पौरविरोधकाः~। अनादेयास्तु omitting by oversight the intervening words.}महाजनविरोधकाः~।\\
अनादेयास्तु ते सर्वे व्यवहाराः प्रकीर्तिताः~॥} इति~।
\end{quote}

\renewcommand{\thefootnote}{2}\footnote{B, C omit न before चानेकप्रतिज्ञत्वमति.}न चानेकप्रतिज्ञत्वमपि पक्षाभासत्वमिति

\begin{quote}
{\vy बहुप्रतिज्ञं यत्कार्यं व्यव\renewcommand{\thefootnote}{3}\footnote{अपरार्क and वीर read व्यवहारेषु निश्चितम्, and व्यव मा. (p. 296) reads व्यवहारेष्वनिश्चितम्.}हारे सुनिश्चितम्~।\\
कामं तदपि गृह्णीयाद्राजा तत्त्वबुभुत्सया~॥} इति
\end{quote}

कात्यायनीयेन विरोधापत्तेः~॥ यत्तु \textendash\ अनेकपदसंकीर्णः पूर्वपक्षो न सिध्यति \textendash\ इति त\renewcommand{\thefootnote}{4}\footnote{N omits तद्युगपन्न सिध्यति before क्रमेण तु सिध्यति.}द्युगपन्न सिध्यति क्रमेण तु सिध्यतीति व्याख्येयम्~॥

एवं पूर्वपक्षे लिखिते यत्कार्यं तदाह याज्ञवल्क्यः ( २. ७ )

\begin{quote}
{\vy श्रुतार्थस्योत्तरं लेख्यं पूर्वावेदकसंनिधौ~॥} इति~।
\end{quote}

उत्तरं लक्षयति नारदः

\begin{quote}
{\vy पक्षस्य व्यापकं सारमसंदिग्घमनाकुलम्~।\\
अव्याख्यागम्यमित्येतदुत्तरं तद्विदो विदुः~॥}
\end{quote}

अस्य चातुर्विध्यमाह कात्यायनः ( $=$नारद २. ४ )

\begin{quote}
{\vy मिथ्या संप्रति\renewcommand{\thefootnote}{5}\footnote{नारद (II. 4) reads संप्रतिपत्तिर्वा प्रत्यवस्कन्दमेव वा~। प्राङ्न्यायविधिसाध्यं वा. अपरार्क reads प्राङ्न्यायविधिसिद्ध्या वा.}पत्त्या वा प्रत्यवस्कन्दनेन वा~।
प्राङ्न्यायप्रतिसिद्ध्या वा उत्तरं स्याच्चतुर्विधम्~॥} इति~।
\end{quote}

\newpage
\fancyhead[CE,CO]{व्यवहारमातृकाः\textendash\ मिथ्योत्तरलक्षणम्}
% १५ व्यवहारमातृकाः \textendash\ मिथ्योत्तरलक्षणम् [§२ 

मिथ्योत्तरं च लक्षयति स एव

\begin{quote}
{\vy अभि\renewcommand{\thefootnote}{1}\footnote{B अभियुक्तो नियोगस्य for अभियुक्तोभियोगस्य.}युक्तोभियोगस्य यदि कुर्यादपह्नवम्~।\\
मिथ्या तत्तु विजानीयादुत्तरं व्यवहारतः~॥} इति~।
\end{quote}

तच्च चतुर्विधमित्याह स एव ( नारद २. ५ ) 

\begin{quote}
{\vy मिथ्यैतन्नाभिजानामि तदा मेभूदसंनिधिः~।\\
अजा\renewcommand{\thefootnote}{2}\footnote{B, D, F अज्ञातश्चास्मि for अजातश्चास्मि. D तत्कालं for तत्काले.}तश्चास्मि तत्काल इति मिथ्या चतुर्विधम्~॥} इति~।
\end{quote}

संप्रतिपत्त्युत्तरं तूक्तं स्मृत्यन्तरे 

\begin{quote}
{\vy साध्यस्य सत्यवचनं प्रतिपत्तिरुदाहृता~॥} इति~।
\end{quote}

प्रत्यवस्कन्दनोत्तरं लक्षयति नारदः 

\begin{quote}
{\vy \renewcommand{\thefootnote}{3}\footnote{व्यव. मा. and परा मा. read अर्थिनाभिहितः.}अर्थिना लिखितो योऽर्थः प्रत्यर्थो यदि तं तथा~।\\
प्रपद्य कारणं ब्रूयात्प्रत्यवस्कन्दनं स्मृतम्~॥} इति~।
\end{quote}

प्राङ्न्यायोत्तरं लक्षयति कात्यायनः 

\begin{quote}
{\vy आचारेणावसन्नोऽपि पुनर्लेखयते यदि~।\\
सोऽभिधेयो जितः पूर्वं प्राङ्न्यायस्तु स उच्यते~॥} इति~।
\end{quote}

एवमुत्तरलक्षणे स्थिते तद्रहितानामुत्तराभासत्वमर्थसिद्धमपि स्पष्टीकृतं स्मृत्यन्तरे

\begin{quote}
{\vy सन्दिग्घमन्यत्प्रकृतादत्यल्पमतिभूरि च~।\\
पक्षैकदेशव्याप्यन्यत्तथा नैवोत्तरं भवेत्~॥

यद्व्यस्तपदमव्यापि निगूढार्थं तथाकुलम्~।\\
व्याख्यागम्यमसारं च नोत्तरं स्वार्थसिद्धये~॥} इति~।
\end{quote}

\newpage
\fancyhead[CE,CO]{व्यवहारमातृकाः\textendash\ उत्तराभासाः}
% § 2] व्यवहारमातृकाः \textendash\ उत्तराभासाः १६ 

कात्यायनोऽपि 

\begin{quote}
{\vy पक्षैकदेशे यत्सत्यमेकदेशे च कारणम्~।\\
मिथ्या चैवै\renewcommand{\thefootnote}{1}\footnote{C चैवैकदेशं for चैवैकदेशे.}कदेशे च संकरात्तदनुत्तरम्~॥} इति~।
\end{quote}

अनुत्तरत्वे च कारणमाह स एव 

\begin{quote}
{\vy न चैकस्मिन्विवादे तु क्रिया स्याद्वादिनोर्द्वयोः~।\\
न चार्थसिद्धिरुभयोर्न चैकत्र क्रियाद्वयम्~॥} इति~।
\end{quote}

अत्रायमर्थः~। मिथ्याकारणोत्तरयोः संकरे द्वयोरपि \renewcommand{\thefootnote}{2}\footnote{G द्वयोरपि वादिप्रतिवादिनोः for द्वयोरपि वादिनोः.}वादिनोः क्रिया प्राप्नोति \textendash\ मिथ्या क्रिया पूर्ववादे कारणे प्रतिवादिनि \textendash\ इति नारदोक्तेः~। तदुभयमेकस्मिन्व्यवहारे विरुद्धम्~। तथा कारणप्राङ्न्यायसंकरे तु प्रत्यर्थिन एव क्रियाद्वयम्~। प्राङ्न्यायकारणोक्तौ तु प्रत्यर्थी निर्दिशेत्क्रियाम् \textendash\ इति व्यासोक्तेः~। अत्र च \textendash\ प्राङ्न्याये जयपत्रेण प्राङ्विवाकादिभिस्तथा \textendash\ इति व्यासोक्त्यैव प्राङ्न्याये जयपत्रेण प्राङ्न्यायदर्शिभिर्वा भावयितव्यम्~। कारणोत्तरे तु साक्षिलेख्यादिभिरित्यत्रापि विरोधः~। एवं त्रिचतुस्संकरेपि द्रष्टव्यम्~। एतेषां \renewcommand{\thefootnote}{3}\footnote{D एषां वानुत्तरत्वेन for एतेषां चानुत्तरत्वं; K एषां चानुत्तरत्वं.}चानुत्तरत्वं यौगपद्येन~। क्रमेण तूत्तरत्वमेव~॥ \renewcommand{\thefootnote}{4}\footnote{N reads क्रमश्चार्थिसभ्येच्छया भवति.}क्रमश्चार्थिप्रत्यर्थिसभ्येच्छया भवति~। तथा च हारीतः

\begin{quote}
{\vy मिथ्योत्तरं कारणं च स्यातामेकत्र चेदुभे~।\\
सत्यं \renewcommand{\thefootnote}{5}\footnote{B, C, D सत्यं वापि for सत्यं चापि.}चापि सहान्येन तत्र ग्राह्यं किमुत्तरम्~॥

यत्प्रभूतार्थविषयं यत्र वा स्यात्क्रियाफलम्~।\\
उत्तरं तत्र तज्ज्ञेयमसंकीर्णमतोन्यथा~॥}
\end{quote}

\newpage
\fancyhead[CE,CO]{व्यवहारमातृकाः\textendash\ उत्तरभेदेषु क्रियाव्यवस्था}
% १७ व्यवहारमातृकाः \textendash\ उत्तरभेदेषु क्रियाव्यवस्था [§ २ 

\noindent
संकीर्णं भवतीति शेषः~। अस्यार्थः~। सुवर्णवस्त्राभियोगे सुवर्णं न गृहीतं वस्त्रं तु गृहीतं प्रतिदत्तं \renewcommand{\thefootnote}{1}\footnote{A, E, M दत्तं वेत्यत्रादौ for दत्तं चेत्यत्रादौ; B चेत्यादौ for चेत्यत्रादौ.}चेत्यत्रादौ सुवर्णविषये व्यवहृत्य पश्चाद्वस्त्रविषये व्यवहर्तव्यम्~। एवं मिथ्याप्राङ्न्याययोः कारणप्राङ्न्याययोश्च संकरे योज्यम्~। तथा तत्रैवाभियोगे सुवर्णं गृहीतं वस्त्रं तु न गृहीतं प्रतिदत्तमिति वा वस्त्रविषये पूर्वं जित इति \renewcommand{\thefootnote}{2}\footnote{C चोच्यते for वोच्यते}वोच्यते~। तत्र वस्त्रविषय एव व्यवहर्तव्यं न सुवर्णविषये भूरिविषयत्वेपि क्रियाभावात्~। यत्र त्वियं गौर्मदीयामुकस्मिन्काले \renewcommand{\thefootnote}{3}\footnote{B, C, D, G, H नष्टस्याद्य गृहे for नष्टाद्यास्य गृहे, K नष्टा व्याधगृहे.}नष्टाद्यास्य गृहे दृष्टेत्यभियोगे मिथ्यैतदेतत्प्रदर्शितकालात्पूर्वमेवास्मद्गृहे स्थितेति मिथ्याकारणयोः कृत्स्नपक्षव्यापित्वं तत्र नानुत्तरत्वम्~। \renewcommand{\thefootnote}{4}\footnote{F, G सकारणमिथ्योत्तरं for सकारणं मिथ्योत्तरम्.}सकारणं मिथ्योत्तरमिदम्~। अत्र च प्रत्यर्थिन एव क्रिया \renewcommand{\thefootnote}{5}\footnote{A, E, M नार्थिनः for नार्थिनोऽपि.}नार्थिनोऽपि मिथ्याकारणयोर्वापि ग्राह्यं कारणमुत्तरम् \textendash\ इति हारीतोक्तेः~॥ एवं मिथ्याप्राङ्न्याययोः कारणप्राङ्न्याययोश्च \renewcommand{\thefootnote}{6}\footnote{B, C, D, F, G, H, K, N कृत्स्नपदव्यापित्वे for कृत्स्नपक्षव्यापित्वे.}कृत्स्नपक्षव्यापित्वे सति नानुत्तरत्वम्~। अत्रोभयत्रापि प्रत्यर्थिन एव क्रियेति \renewcommand{\thefootnote}{7}\footnote{N omits न क्वाप्येकस्मिन्\ldots क्रियेत्यलम्.}न क्वाप्येकस्मिन्व्यवहारे द्वयोः क्रियेत्यलम्~॥

उत्तरलेखनोत्तरं साधनोक्तौ क्रममाह याज्ञवल्क्यः (२. ७\textendash\ ८ )

\begin{quote}
{\vy ततोऽर्थी लेखयेत्सद्यः प्रतिज्ञातार्थसाधनम्~।\\
तत्सिद्धौ सिद्धिमाप्नोति विपरीतमतोऽन्यथा~॥} इति~।
\end{quote}

\newpage
\fancyhead[CE,CO]{व्यवहारमातृकाः\textendash\ उत्तरलेखनोत्तरं साधनक्रमः}
%§२ ] व्यवहारमातृकाः \textendash\ उत्तरलेखनोत्तरं साधनक्रमः १८ 

इदं च मिथ्योत्तरविषयम्~। उत्तरान्तरे तु प्रत्यर्थिन एव साधनोपन्यासः~। \renewcommand{\thefootnote}{1}\footnote{D, F तथा च हारीतः for यथाह हारीतः.}यथाह हारीतः

\begin{quote}
{\vy प्राङ्न्यायकारणोक्तौ तु प्रत्यर्थी निर्दिशेत्क्रियाम्~।\\
मिथ्योक्तौ पूर्ववादी तु प्रतिपत्तौ न सा भवेत्~॥} इति~।
\end{quote}

एवं व्यवहारस्य चतुष्पात्त्वमाह याज्ञवल्क्यः (२. ८. )

\begin{quote}
{\vy चतुष्पाद्व्यवहारोऽयं विवादेषूपदर्शितः~॥} इति~।
\end{quote}

\renewcommand{\thefootnote}{2}\footnote{B, F सि पदचतुष्टयं for पादचतुष्टयम्.}पादचतुष्टयं तु व्यक्तीकृतं स्मृत्यन्तरे~। 

\begin{quote}
{\vy भाषोत्तरक्रियासाध्यसिद्धिभिः क्रमवृत्तिभिः~।\\
आक्षिप्तचतुरंशस्तु चतुष्पादभिधीयते~॥} इति~।
\end{quote}

इदं च संप्रतिपत्त्यतिरिक्तोत्तरविषयम्~। \renewcommand{\thefootnote}{3}\footnote{F संप्रतिपत्तौ for संप्रतिपत्तेः.}संप्रतिपत्तेर्द्विपात्त्वात्~। \renewcommand{\thefootnote}{4}\footnote{C, K यथा बृहस्पतिः for यथाह बृहस्पतिः.}यथाह बृहस्पतिः

\begin{quote}
{\vy मि\renewcommand{\thefootnote}{5}\footnote{व्यव. मा. reads मिथ्योक्तौ च चतुष्पात्स्यात् प्रत्यवस्कन्दने तथा~। प्राङ्न्याये च स विज्ञेयः.}थ्योत्तरे चतुष्पात्स प्रत्यवस्कन्दने तथा~।\\
व्यवहारस्तु विज्ञेयो द्विपात्संप्रतिपत्तिषु~॥} इति~।
\end{quote}

याज्ञवल्क्यः (२. ९\textendash\ १० ) 

\begin{quote}
{\vy अभियोगमनिस्तीर्य नैनं प्रत्यभियोजयेत्~।\\
अभियुक्तं च नान्येन नोक्तं विप्रकृतिं नयेत्~॥

कुर्यात्प्रत्यभियोगं च कलहे साहसेषु\renewcommand{\thefootnote}{6}\footnote{G साहसेषु वा for साहसेषु च.} च~॥}
\end{quote}

\newpage
\fancyhead[CE,CO]{व्यवहारमातृकाः\textendash\ हीनवादिलक्षणम्}
%१९ व्यवहारमातृकाः \textendash\ हीनवादिलक्षणम् [§ २ 

नारदः ( २. २४ ) 

\begin{quote}
{\vy पूर्ववादं परित्यज्य योन्यमालम्बते पुनः~।\\
वादसंक्रमणाञ्ज्ञेयो हीनवादी स वै नरः~॥}
\end{quote}

हीनवादी दण्ड्यो भवति न प्रकृतादर्थाद्धीयत इत्यर्थः~। एतच्चार्थव्यवहारे ज्ञेयम्~। यथाह स एव ( नारद २. २५ )

\begin{quote}
{\vy सर्वेष्व\renewcommand{\thefootnote}{1}\footnote{नारदस्मृति reads सर्वेष्वपि विवादेषु and पशुस्त्री.}र्थविवादेषु वाक्छले नावसीदति~।\\
परस्त्रीभूम्यृणादाने शास्योऽप्यर्थान्न हीयते~॥} इति~।
\end{quote}

\renewcommand{\thefootnote}{2}\footnote{B, G omit पूर्वार्ध.. .र्धम्.}पूर्वार्धस्योदाहरणार्थमुत्तरार्धम्~। याज्ञवल्क्यः (२. १७) 

\begin{quote}
{\vy साक्षिषूभयतः सत्सु साक्षिणः पूर्ववादिनः~।\\
पूर्वपक्षेधरीभूते भवन्त्युत्तरवादिनः~॥}
\end{quote}

पूर्ववादिनः प्रतिज्ञावादिनः~। पूर्वपक्षः प्रतिज्ञा~। अधरीभूते करणोत्तरोपन्यासेन प्रत्यर्थिस्वीकारेणासाध्ये~। साक्षिग्रहणं
प्रमाणान्तरस्याप्युपलक्षणार्थम्~। स एव ( याज्ञवल्क्य २. १० )

\begin{quote}
{\vy उभयोः प्रतिभूर्ग्राह्यः समर्थः कार्यनिर्णये~।}
\end{quote}

निर्णयस्य कार्यं कार्यनिर्णय इति~॥ प्रतिभूत्वेनाग्राह्यानाह कात्यायनः

\begin{quote}
{\vy न स्वामी न च वै शत्रुः स्वामिनाधिकृतस्तथा~।\\
निरु\renewcommand{\thefootnote}{3}\footnote{C, K, N विरुद्धो for निरुद्धो.}द्धो दण्डितश्चैव संशयस्थो न च क्वचित्~॥

नैव रिक्थी न रिक्तश्च \renewcommand{\thefootnote}{4}\footnote{स्मृतिच, परा मा. and वीर read न चैवात्यन्तवासिनः (वीर. also notices the reading अन्यत्र वासिनः ).}न चैवान्यत्र वासितः~।\\
राजकार्यनियुक्तश्च ये च प्रव्रजिता नराः~॥}
\end{quote}

\newpage
\fancyhead[CE,CO]{व्यवहारमातृकाः\textendash\ प्रतिभूत्वेनाग्राह्याः}
% §२ ] व्यवहारमातृकाः \textendash\ प्रतिभूत्वेनाग्राह्याः २० 

\begin{quote}
{\vy नाशक्तो ध\renewcommand{\thefootnote}{1}\footnote{B, D, G धनिनो दातुं for धनिने दातुम्; F नाशक्तौ for नाशक्तो.}निने दातुं दण्डं राज्ञे च तत्समम्~।\\
नाविज्ञातो ग्रहीतव्यः प्रतिभूत्वक्रियां प्रति~॥} इति~।
\end{quote}

निरु\renewcommand{\thefootnote}{2}\footnote{D, B omit the words निरुद्धो\ldots धिकारी and अन्यत्र\ldots ष्कृतः. F, G also did the same, but the words were added in another hand.}द्धो निगडादिबद्धः~। संशयस्थो व्यसनी~। रिक्थी पुत्रपौत्रादिर्द्रव्यग्रहणाधिकारी~। रिक्तो दरिद्रः~। अन्यत्र वासितो देशाद्बहिष्कृतः~॥

याज्ञवल्क्यः (२. ५२ )

\begin{quote}
{\vy भ्रातॄणामथ दम्पत्योः पितुः पुत्रस्य चैव हि~।\\
प्रातिभाव्यमृणं साक्ष्यमविभक्ते \renewcommand{\thefootnote}{3}\footnote{G, K न तु स्मृतं for तु न स्मृतम्. The printed editions of याज्ञ read न तु स्मृतम्.}तु न स्मृतम्~॥}
\end{quote}

लग्नकाभावे त्वाह कात्यायनः

\begin{quote}
{\vy अथ चैत्प्रतिभूर्नास्ति कार्ययो\renewcommand{\thefootnote}{4}\footnote{C, H, K, N कार्ययोग्यस्य वादिनः for कार्ययोग्यस्तु वादिनः स्मृतिच reads वादयोग्यस्य and दूताय for भृत्याय.}ग्यस्तु वादिनः~।\\
स रक्षितो दिनस्यान्ते दद्याद्भृत्याय वेतनम्~॥} इति~।
\end{quote}

स एव 

\begin{quote}
{\vy द्विजातिः प्रतिभूहीनो रक्ष्यः स्याद्बाह्यचारिभिः~।\\
शूद्रादीन्प्रतिभूहीनान्बन्धयेन्निगडेन तु~॥} इति~।
\end{quote}

हीन\renewcommand{\thefootnote}{5}\footnote{C, H, K. N omit हीनवादि\ldots नर इति.}वादिलक्षणान्याह नारदः (२. २४ )

\begin{quote}
{\vy पूर्ववादं परित्यज्य योऽन्यमालम्बते पुनः~।\\
वादसंक्रमणाज्ज्ञेयो हीनवादी स वै नरः~॥} इति~।
\end{quote}

\newpage
\fancyhead[CE,CO]{व्यवहारमातृकाः\textendash\ दुष्टवादिलक्षणम्}
% २१ व्यवहारमातृकाः \textendash\ दुष्टवादिलक्षणम् [§ २

दुष्टं लक्षयति याज्ञवल्क्यः (२. १३\textendash\ १५ ) 

\begin{quote}
{\vy देशाद्देशान्तरं याति सृक्किणी परिलेढि च~।\\
ललाटं स्विद्यते \renewcommand{\thefootnote}{1}\footnote{G स्विद्यते वास्य for स्विद्यते चास्य.}चास्य मुखं वैवर्ण्यमेति च~॥

परिशुष्यत्स्खलद्वाक्यो विरुद्धं बहु भाषते~।\\
वाक्चक्षुः पूजयति नो तथोष्ठौ निर्भुजत्यपि~॥

स्वभावाद्विकृतिं गच्छेन्मनोवाक्कायकर्मभिः~।\\
अभि\renewcommand{\thefootnote}{2}\footnote{B, C, D, F, K अभियोगे च साक्ष्ये च for अभियोगेथ साक्ष्ये वाः ,A, G अभियोगेथ साक्ष्ये च.}योगेथ साक्ष्ये वा दुष्टः स परिकीर्तितः~॥}
\end{quote}

सृक्किणी ओष्ठप्रान्तौ~॥~२~॥

\begin{center}
\textbf{\Large ॥~\renewcommand{\thefootnote}{3}\footnote{C, G, K अथ प्रमाणं for अथ प्रमाणनिरूपणम्.}अथ प्रमाणनिरूपणम्~॥~३~॥}
\end{center}

याज्ञवल्क्यः ( २. २२ )

\begin{quote}
{\vy प्रमाणं लिखितं भुक्तिः साक्षिणश्चेति कीर्तितम्~।\\
एषामन्यतमाभावे दिव्यान्यतममुच्यते~॥}
\end{quote}

कात्यायनोपि

\begin{quote}
{\vy यद्येको मानुषीं ब्रू\renewcommand{\thefootnote}{4}\footnote{अपरार्क reads मानुषीं कुर्यादन्यः कुर्यात्तु.}यादन्यो ब्रूयात्तु दैविकीम्~।\\
मानुषीं तत्र गृह्णीयान्न तु दैवीं क्रियां नृपः~॥

यद्येकदेश\renewcommand{\thefootnote}{5}\footnote{अपरार्क reads देशप्राप्तापि and न्याय्या for ग्राह्या.}व्याप्तापि क्रिया विद्येत मानुषी~।\\
सा ग्राह्या न तु \renewcommand{\thefootnote}{6}\footnote{B, C, D, H, K पूर्वापि for पूर्णापि.}पूर्णापि दैविकी वदतां नृणाम्~॥}
\end{quote}

\newpage
\fancyhead[CE,CO]{प्रमाणनिरूपणं\textendash\ प्रमाणबलविचारः}
\fancyhead[RO]{[ $\S$ ३ }
\fancyhead[LE]{$\S$ ३ ]}

%§ 3] प्रमाणनिरूपणं \textendash\ प्रमाणबलविचारः 22

\begin{quote}
{\vy क्रिया न दैविकी प्रोक्ता विद्यमानेषु साक्षिषु~।\\
लेख्ये च सति वादेषु न स्याद्दिव्यं न साक्षिणः~॥

\renewcommand{\thefootnote}{1}\footnote{C, G, H, K. N omit the verse पूग\ldots साक्षिणः.}पूगश्रेणिगणादीनां या स्थितिः परिकीर्तिता~।\\
तस्यास्तु साधनं लेख्यं न दिव्यं न च साक्षिणः~॥

\renewcommand{\thefootnote}{2}\footnote{मिता reads दत्तादत्तेथ भृत्यानां स्वामिना; अपरार्क {\qt दत्तादत्ते तथादत्ते स्वामिना}. वीर (p. 112) reads {\qt दत्तादत्तेषु भृत्यानां स्वामिनां}; अपरार्क and वीर (p. 112) read {\qt विक्रीयादानसंबन्धे} A, B, C, F, K स्वामिनां for स्वामिनः; G स्वामिना. वीर reads {\qt धनमयच्छति}.}दत्तादत्ते तथादत्ते स्वामिनो निर्णये सति~॥

विक्रयादानसंबन्धे क्रीत्वा धनमनिच्छति~॥

द्यूते समाह्वये चैव विवादे समुपस्थिते~।\\
साक्षिणः साधनं प्रोक्तं न दिव्यं न च लेख्यकम्~॥

द्वारमार्गक्रियाभोगे जलवाहादिके तथा~।\\
भुक्तिरेव हि गुर्वी स्यान्न लेख्यं न च साक्षिणः~॥}
\end{quote}

क्वचि\renewcommand{\thefootnote}{3}\footnote{C, H, K, N before क्वचिद्दिव्य reads दत्तादत्ते (K, N दत्त्वादत्ते) दत्तमिति प्रतिश्रुत्यानर्पिते तथादत्ते दत्त्वा पुनराच्छिद्य गृहीते~। स्वामिनां निर्णये सपीति ( तीति ? ) एतत्स्वामिकमेतदिति निश्चये सति.}द्दिव्यप्राबल्यमाह बृहस्पतिः 

\begin{quote}
{\vy मणिमुक्ता\renewcommand{\thefootnote}{4}\footnote{B, D मुक्ताप्रवालानां for मुक्तानाणकानाम्; F also read प्रवालानां but corrects to नाणकानां in another hand. The परा मा. and वीर (p. 114) read मुक्ताप्रवालानां.}नाणकानां कूटकृन्न्यासहारकः~।\\
हिंसकोन्या\renewcommand{\thefootnote}{5}\footnote{H अन्यागमासेवी for अन्याङ्गनासेवी.}ङ्गनासेवी परीक्ष्याः शपथैः सदा~॥

महापापाभिशापेषु विद्यमानेषु साक्षिषु~।\\
दिव्यमालम्बते वादी न पृच्छेत्तत्र साक्षिणः~॥} इति~।
\end{quote}

\newpage
\fancyhead[CE,CO]{प्रमाणनिरूपणं\textendash\ दिव्यप्राबल्यविचारः}
% २३ प्रमाणनिरूपणं \textendash\ दिव्यप्राबल्यविचारः [§ ३

व्यासः

\begin{quote}
{\vy न मयैतत्कृतं लेख्यं कूटमेतेन कारितम्~।\\
अधरीकृत्य तत्पत्रमर्थे दिव्येन निर्णयः~॥

अरण्ये \renewcommand{\thefootnote}{1}\footnote{C विजये for विजने.}विजने रात्रावन्तर्वेश्मनि साहसे~।\\
न्यासापह्नवने चैव दिव्या संभवति क्रिया~॥}
\end{quote}

वृहस्पतिः

\begin{quote}
{\vy लिखिते साक्षिवादे च संदिग्धं यत्र जायते~।\\
अनुमाने च संभ्रान्ते तत्र दिव्यं विशोधनम्~॥}
\end{quote}

क्वचित्साक्षिदिव्ययोर्विकल्पमाह स एव

\begin{quote}
{\vy प्रक्रान्ते साहसे वादे पारुष्ये दण्ड\renewcommand{\thefootnote}{2}\footnote{D. F. दण्डवाचके for दण्डवाचिके}वाचिके~।\\
बलोद्भूतेषु कार्येषु साक्षिणो दिव्यमेव\renewcommand{\thefootnote}{3}\footnote{B, D, H, K दिव्यमेव च for दिव्यमेव वा.} वा~॥

ऋणे लेख्यं साक्षिणो वा युक्तिलेशाद\renewcommand{\thefootnote}{4}\footnote{D लेखादयोपि for लेखादयोपि.}योऽपि वा~।\\
दैविकी वा क्रिया प्रोक्ता प्रजानां हितकाम्यया~॥} इति~।
\end{quote}

युक्तिलेशो युक्त्येकदेशः~। वाचिके पारुष्ये ब्रह्महासीत्येवमाद्याक्रोशात्मक इत्यर्थः~। यत्तु कात्यायनः \textendash\ वाक्पारुष्ये च भूमौ च दिव्यं न परिकल्पयेत् \textendash\ इति तदल्पवाक्पारुष्यपरम्~। भूमिग्रहणं \renewcommand{\thefootnote}{5}\footnote{C, D, F, G, K मात्रोपलक्षणं for मात्रोपलक्षणार्थम्, H मात्रोपलेशनम्}स्थावरमात्रोपलक्षणार्थम्~। यथाह पितामहः \textendash\ स्थावरेषु विवादेषु दिव्यानि परिवर्जयेत् \textendash\ इति~। साक्ष्यादिसत्त्वेयं दिव्यनिषेधः~। तथा च \renewcommand{\thefootnote}{6}\footnote{C, F, G, H, K पितामहः for स एव.}स एवसाक्षिभिर्लिखितेनाथ भुक्त्या चैतान्प्रसाधयेत् \textendash\ इति~। स एव

\newpage
\fancyhead[CE,CO]{प्रमाणनिरूपणं\textendash\ सर्वप्रमाणाभावे कार्यम्}
% § ३ ] प्रमाणनिरूपणं \textendash\ सर्वप्रमाणाभावे कार्यम् २४ 

\begin{quote}
{\vy लेख्यं यत्र न विद्येत न भुक्तिर्न च साक्षिणः~।\\
न च\renewcommand{\thefootnote}{1}\footnote{G न वा दिव्या for न च दिव्या.} दिव्यावतारोस्ति प्रमाणं तत्र पार्थिवः~॥

निश्चेतुं ये न शक्याः स्युर्वादाः संदिग्धरूपिणः~।\\
तेषां नृपः प्रमाणं स्यात्स सर्वस्य प्रभुर्यतः~॥}
\end{quote}

\begin{center}
\textbf{\Large ॥~\renewcommand{\thefootnote}{2}\footnote{A, M, N add the words इति व्यवहारमातृकाः समाप्ताः before अथ लेख्यम्.}अथ लेख्यम्~॥~४~॥}
\end{center}

\renewcommand{\thefootnote}{3}\footnote{B, D, K omit तत्र.}तत्र बृहस्पतिः

\begin{quote}
{\vy राजलेख्यं स्थानकृतं स्वहस्तलिखितं तथा~।\\
लेख्यं तु त्रिविधं प्रोक्तं भिन्नं तद्बहुधा पुनः~॥} इति~।
\end{quote}

यत्तु वसिष्ठेन ( व. ध. सूत्र appendix 10 ) \textendash\ लौकिकं राजकीयं च लेख्यं विद्याद् द्विलक्षणम् \textendash\ इति द्वैविध्यमुक्तं तत्स्थानकृतस्वहस्तलि\renewcommand{\thefootnote}{4}\footnote{B, C, D, G, K लिखितयोर्भेदमाश्रित्य for लिखितयोरभेदमाश्रित्य.}खितयोरभेदमाश्रित्य~। लौकिकं जानपदमिति पर्यायौ राजकीयं जानपदं लिखितं द्विविधं स्मृतम् \textendash\ इति संग्र\renewcommand{\thefootnote}{5}\footnote{B, D, F संग्रहोक्तेः for संग्रहकारोक्तेः.}हकारोक्तेः~॥ बृहस्पतिः

\begin{quote}
{\vy \renewcommand{\thefootnote}{6}\footnote{B, D, G भागधान 01 भागदान; C भागप्रदान. B दासऋणादिभिः for दासक्रणादिभिः.}भागदानक्रयाधानसंविद्दासऋणादिभिः~।\\
सप्तधा लौकिकं लेख्यं त्रिविधं राजशासनम्~॥

भ्रातरः संविभक्ता ये स्वरूच्या तु परस्परम्~।\\
विभाग\renewcommand{\thefootnote}{7}\footnote{H विभागं यत्र for विभागपत्रं.}पत्रं कुर्वन्ति भागलेख्यं तदुच्यते~॥}
\end{quote}

\newpage
\fancyhead[CE,CO]{लेख्यभेदाः}
% २५ लेख्यभेदाः [§ ४ 

\begin{quote}
{\vy भूमिं दत्त्वा तु यत्पत्रं कुर्याच्चन्द्रार्क\renewcommand{\thefootnote}{1}\footnote{B, D चन्द्रार्कसाक्षिकं for चन्द्रार्ककालिकम्; G has कालिकं in the text and साक्षिकं in the margin.}कालिकम्~।\\
अनाच्छेद्यमनाहार्यं दानलेख्यं तु तद्विदुः~॥

गृहक्षेत्रादिकं क्रीत्वा तुल्यमूल्याक्षरान्वितम्~।\\
पत्रं कारयते यत्तु क्रयलेख्यं\renewcommand{\thefootnote}{2}\footnote{D क्रमलेख्यं for क्रयलेख्यं.} तदुच्यते~॥

जङ्गमं स्थावरं बन्धं दत्त्वा लेख्यं करोति यत्~।\\
गोप्यभोग्यक्रियायुक्तमाधिलेख्यं तदुच्यते~॥

\renewcommand{\thefootnote}{3}\footnote{परा. मा. reads ग्रामादिसमयात्कुर्यान्मतं; अपरार्क reads मतलेख्यं, वीर reads सत्यलेख्यम्.}ग्रामो देशश्च यत्कुर्यान्मतं लेख्यं परस्परम्~।\\
राजाविरोधि धर्मार्थं संवित्पत्रं वदन्ति तत्~॥

वस्त्रान्नहीनः कान्तारे लिखितं कुरुते तु यत्~।\\
१० कर्माणि ते करिष्यामि दासपत्रं तदुच्यते~॥

धनं वृद्ध्या गृहीत्वा तु स्वयं कु\renewcommand{\thefootnote}{4}\footnote{H कुर्यात्प्रकारयेत् for कुर्याच्च कारयेत्.}र्याच्च कारयेत्~।\\
उद्धारपत्रं तत्प्रोक्तमृणलेख्यं मनीषिभिः~॥}
\end{quote}

\noindent
आदिशब्दाद्विशुद्ध्यादिपत्रग्रहणम्~॥ 

विशुद्ध्यादिप\renewcommand{\thefootnote}{5}\footnote{B, D, F विशुद्धिपत्राण्याह for विशुद्ध्यादिपत्राण्याह.}त्राण्याह कात्यायनः 

\begin{quote}
{\vy अभिशापे समुत्तीर्णे प्रायश्चित्ते कृते जनैः~।\\
विशुद्धिपत्रकं ज्ञेयं तेभ्यः साक्षिसमन्वितम्~॥

उत्तमेषु समस्तेषु अभिशापे समागते~।\\
वृत्तानुवादे लेख्यं यत्तज्ज्ञेयं सन्धिपत्रकम्~॥

सीमाविवादे निर्णीते सीमापत्रं विधीयते~॥} इति~।
\end{quote}

\newpage
\fancyhead[CE,CO]{अन्वाधिपत्रादिलक्षणम्}
\fancyhead[RO]{[ $\S$ ४}
\fancyhead[LE]{$\S$ ४ ]}
% § ४ ] अन्वाधिपत्रादिलक्षणम् २६

अन्वाधिपत्रमाह प्रजापतिः 

\begin{quote}
{\vy धनी धनेन तेनैव परमाधिं नयेद्यदि~।\\
कृत्वा तदाधिलिखितं \renewcommand{\thefootnote}{1}\footnote{पूर्वं वास्य for पूर्वं चास्य.}पूर्वं चास्य समर्पयेत्~॥}
\end{quote}

याज्ञवल्क्योऽपि (२. ९४ )

\begin{quote}
{\vy दत्त्वर्णं पाटयेल्लेख्यं शुद्धै वान्यत्तु कारयेत्~॥}
\end{quote}

पूर्वोक्तयोः स्वहस्त\renewcommand{\thefootnote}{2}\footnote{A, D, E, M स्वहस्तान्यहस्तकृतयोः for सहस्तकृतान्यहस्तकृतयोः.}कृतान्यहस्तकृतयोर्विशेषमाह नारदः (४. १३५ )

\begin{quote}
{\vy लेख्यं तु द्विविधं प्रोक्तं स्वहस्तान्यकृतं तथा~।\\
असाक्षिमत्साक्षिमच्च सिद्धिर्देशस्थितेस्तयोः~॥} इति~।
\end{quote}

याज्ञवल्क्यः (२. ८९ ) 

\begin{quote}
{\vy विनापि साक्षिभिर्लै\renewcommand{\thefootnote}{3}\footnote{D साक्षिमल्लेख्यं for साक्षिभिर्लेख्यं.}ख्यं स्वहस्तलिखितं तु यत्~।\\
तत्प्रमाणं स्मृतं लेख्यं बलोपधिकृतादृते~॥}
\end{quote}

बलं बलात्कारः~। उपाधिर्लोभादिः~। अन्यकृते विशेष\renewcommand{\thefootnote}{4}\footnote{B, C, D, G , K omit अन्यकृते before विशेषमाह; F also did so, but corrected it.}माह स एव (या. २. ८४\textendash\ ८५)

\begin{quote}
{\vy यः कश्चिदर्थो नि\renewcommand{\thefootnote}{5}\footnote{B, D कश्चिदर्थोभिमतः for कष्चिदर्थो निष्णातः}ष्णातः स्वरुच्या तु परस्परम्~।\\
लेख्यं तु साक्षिमत्कार्यं तस्मिन्धनिक\renewcommand{\thefootnote}{6}\footnote{D धनिकपूर्वके for धनिकपूर्वकम्.}पूर्वकम्~॥

समामासतदुर्धाहर्नामजातिस्वगोत्रकैः~।\\
सब्रह्मचारिकात्मीयपितृनामादिचिह्नितम्~॥} इति~।
\end{quote}

\newpage
\fancyhead[CE,CO]{स्वहस्तपरहस्तलिखितलेख्यविचारः}
% २७ स्वहस्तपरहस्तलिखितलेख्यविचारः [ §४ 

सब्रह्मचारिकं बह्वृचादिशाखाप्रयुक्तं गुणनाम बह्वृचः कठ इत्यादि स एव (या. २. ८६\textendash\ ८८) 

\begin{quote}
{\vy समाप्तेर्थे ऋणी नाम खहस्तेन निवेशयेत्~।\\
मतं मेमुकपुत्रस्य यदत्रोपरि लेखितम्~॥

साक्षिणश्च स्बहस्तेन पितृनामकपूर्वकम्~।\\
अत्राहममुकः साक्षी लिखेयुरिति ते समाः~॥

उभयाभ्यर्थितेनैतन्मया ह्यमुकसूनुना~।
लिखितं ह्यमुकेनेति लेखकोन्ते ततो लिखेत्~॥}
\end{quote}

समाः संख्यातो गुणतश्च~। क्वचिदसमा\renewcommand{\thefootnote}{1}\footnote{N reads क्वचिदसमाः समा इत्यकारे प्रश्लेषः.} इत्यकारप्रश्लेषः~॥ नारदः

\begin{quote}
{\vy अलिपिज्ञ ऋणी यः स्यात्स्वमतं तु स लेखयेत्~।\\
साक्षी वा साक्षिणान्येन सर्वसाक्षिसमीपतः~॥}
\end{quote}

त्रिविधं राजशासनमित्युक्तं तद्दर्शयतो याज्ञवल्क्यबृहस्पती (या. १. ३१८\textendash\ ३२० )

\begin{quote}
{\vy दत्त्वा भूमिं निबन्धं वा कृत्वा लेख्यं तु कारयेत्\renewcommand{\thefootnote}{2}\footnote{H धारयेत् for कारयेत्}~।\\
आगामिभद्रनृपतिपरिज्ञानाय पार्थिवः~॥

पटे वा ताम्रपट्टे वा स्वमुद्रोपरिचिह्नितम्~।\\
अभिलेख्यात्मनो वंश्यानात्मानं च महीपतिः~॥

प्रतिग्रहपरीमाणं दान\renewcommand{\thefootnote}{3}\footnote{F दानं च्छेदोपवर्णनं for दानच्छेदोपवर्णनम्.}च्छेदोपवर्णनम्~।\\
स्वहस्तकालसंपन्नं शासनं कारयेत्स्थिरम्~॥}
\end{quote}

\newpage
\fancyhead[CE,CO]{राजशासनविचारः}
% § ४ ] राजशासनविचारः २८

\noindent
निबन्धः आक\renewcommand{\thefootnote}{1}\footnote{C आकारादौ for आकरादौ.}रादौ राजादिदत्तं नियतलभ्यम्~। प्रतिगृह्यते \renewcommand{\thefootnote}{2}\footnote{B, D यत्प्रतिग्रहः for यत्स प्रतिग्रहः; G, K यत्तत्प्रतिग्रहः.}यत्स प्रतिग्रहो भूम्यादिः तस्य परिमा\renewcommand{\thefootnote}{3}\footnote{B, D omit the words परिमाणमियत्ता.. .यस्य; F also did the same, but corrects in another hand.}णमियत्ता~। दीयते यत्तद्दानं गृहादि \renewcommand{\thefootnote}{4}\footnote{F तस्योपवर्णनं for तस्या उपवर्णनम्.}तस्य छेदो मर्यादा तस्या उपवर्णनं कथनम्~। तथा

\begin{quote}
{\vy देशादिकं यस्य राजा लिखितेन प्रयच्छति~।\\
सेवाशौर्यादिना तुष्टः प्रसादलिखितं तु तत्~॥

पूर्वोत्तरक्रियावादनिर्णयान्ते यदा नृपः~।\\
प्रदद्याज्जयिने लेख्यं जयपत्रं तदुच्यते~॥} इति~।
\end{quote}

राज्ञोनुकल्पमाह व्यासः 

\begin{quote}
{\vy \renewcommand{\thefootnote}{5}\footnote{C, D राजा तु स्वयमादिष्टसन्धि for राज्ञा तु स्वयमादिष्टः सन्धि; K दिष्टसन्धि. दिष्टसन्धि,}राज्ञा तु स्वयमादिष्टः सन्धिविग्रहलेखकः~।\\
\renewcommand{\thefootnote}{6}\footnote{A ताम्रपट्टे वापि पटे for ताम्रपट्टे पटे वापि}ताम्रपट्टे पटे वापि प्रलिखेद्राजशासनम्~॥}
\end{quote}

अत्र च राज्ञा स्वमतं लेख्यमित्याह स एव

\begin{quote}
{\vy संनिवेशं प्रमाणं \renewcommand{\thefootnote}{7}\footnote{H has a confused text here. It reads प्रमाणं कुर्याच्चन्द्रार्ककालिकम्. H seems to have gone back several pages, copied a few lines and then returns to मतं मे.}च स्वहस्तेन लिखेत्स्वयम्~।\\
मतं मेमुकपुत्रस्य अमुकस्य महीपतेः~॥}
\end{quote}

संनिवेशं प्रमाणं चेति पूर्ववाक्येनान्वयः~। वसिष्ठस्तु राजलेख्यस्य चातुर्विध्यमाह

\begin{quote}
{\vy शासनं प्रथमं ज्ञेयं जयपत्रं तथापरम्~।\\
आज्ञाप्रज्ञापनापत्रे राजकीयं चतुर्विधम्~॥}
\end{quote}

\newpage
\fancyhead[CE,CO]{वसिष्ठमतेन राजशासनस्य चातुर्विध्यम्}
%२९ वसिष्ठमतेन राजशासनस्य चातुर्विध्यम् [§ ४ 

\begin{quote}
{\vy सामन्तेष्वथ भृत्येषु राष्ट्रपालादिकेषु च~।\\
कार्यमादिश्यते\renewcommand{\thefootnote}{1}\footnote{H reads प्रदिश्यते येन पत्रे प्रज्ञापनाय तत्, omitting two lines.} येन तदाज्ञापत्रमुच्यते~॥

ऋत्विक्पुरोहिताचार्यमान्येष्वभ्यर्हितेषु\renewcommand{\thefootnote}{2}\footnote{B, D अभ्यर्हितेषु तु for अभ्यर्हितेषु च; F also read तु but corrected it to च.} च~।\\
कार्यं निवेद्यते येन पत्रं प्रज्ञापनाय तत्~॥}
\end{quote}

शासनजयपत्रे प्रागुक्ते~॥ याज्ञवल्क्यः (२. ९१ )

\begin{quote}
{\vy देशान्तरस्थे दुर्लेख्ये नष्टोन्मृष्टे हृते तथा\renewcommand{\thefootnote}{3}\footnote{B, C, D, G, H, K तथा हृते for हृते तथा; F also at first wrote तथा हृते but corrected into हृते तथा.}~।\\
भिन्ने दग्धे तथा छिन्ने लेख्यमन्यत्तु कारयेत्~॥}
\end{quote}

नारदः (४. १४२ )

\begin{quote}
{\vy लेख्ये देशान्त\renewcommand{\thefootnote}{4}\footnote{B, D देशान्तरे न्यस्तेः for देशान्तरन्यस्तेः, F देशान्तरं न्यस्ते.}रन्यस्ते शीर्णे दुर्लिखिते हृते~।\\
सतस्तत्कालकरणमसतो द्रष्टृदर्शनम्~॥}
\end{quote}

द्रष्टारः साक्षिणः~। तदभावे दिव्यम्~। अलेख्यसाक्षिके दिव्यं\renewcommand{\thefootnote}{5}\footnote{B, D, K दैवीं 0 दिव्यम्} व्यवहारे विनिर्दिशेत् \textendash\ इति कात्यायनोक्तेः~। याज्ञवल्क्यः (२. ९२ )

\begin{quote}
{\vy संदिग्धलेख्यशुद्धिः स्यात्स्वहस्तलिखितादिभिः~।\\
युक्तिप्राप्तिक्रियाचिह्नसंबन्धागमहेतुभिः~॥}
\end{quote}

युक्तिरर्थापत्तिः~। प्राप्तिर्द्वयोरेकत्रावस्थानम्~। चिह्नं मुद्रादि~। \renewcommand{\thefootnote}{6}\footnote{B, D, F place क्रिया साक्ष्यादिः before चिह्नम्.}क्रिया \renewcommand{\thefootnote}{7}\footnote{G, K साक्ष्यादि for साक्ष्यादिः.}साक्ष्यादिः~। संबन्धोनागतः~। आगमः संभावितः प्राप्त्युपायः~। हेतुरनुमानम्~॥ प्रजापतिः 

\newpage
\fancyhead[CE,CO]{दुष्टलेख्यविचारः}
% §४ ] दुष्टलेख्यविचारः ३० 

\begin{quote}
{\vy कार्यो यत्नेन महता निर्णयो राजशासने~।\\
राज्ञां स्वहस्ततो मुद्रालेखका\renewcommand{\thefootnote}{1}\footnote{G लेखकान्तरदर्शनात् for लेखकाक्षरदर्शनात्.}क्षरदर्शनात्~॥}
\end{quote}

दुष्ठलेख्यान्याह बृहस्पतिः 

\begin{quote}
{\vy मुमूर्षुशत्रुभीतार्तस्त्रीमत्तव्यसनातुरैः~।\\
निशोपधिवलात्कारकृतं लेख्यं न सिध्यति~॥

दूषितो गर्हितः साक्षी यत्रैको विनिवेशितः~।\\
कूटलेख्यं तु तत्प्रोक्तं लेखको वापि तद्विधः~॥} इति~॥
\end{quote}

\begin{center}
\textbf{\Large ॥~\renewcommand{\thefootnote}{2}\footnote{A, B, C, D, F, G, H, K omit इति लेख्यप्रकरणम्}इति लेख्यप्रकरणम्~॥~४~॥}\\

\vspace{2mm}
\textbf{\Large अथ भुक्तिः~॥~५~॥}
\end{center}

१० नारदः (४. ८५)

\begin{center}
{\vy आगमेन विशुद्धेन भोगो याति प्रमाणताम्~।\\
अविशुद्धागमो भोगः प्रामाण्यं नैव गच्छति~॥}
\end{center}

सागमत्ववद्विशेषणान्तरवैशिष्ट्यमप्याह व्यासः

\begin{center}
{\vy सागमो दीर्घकालश्चा\renewcommand{\thefootnote}{3}\footnote{अपरार्क reads {\qt च्छेदोपाधिविवर्जितः}; व्यव. मा. {\qt निश्छिद्रोन्यरवोज्झितः}; परा. मा. and vईष्र {\qt कालश्च विच्छेदोपरमो}. D, F विच्छेदोपरबोधितः for विच्छेदोपरवोज्झितः; H विच्छेदोपनबोधितः (?).}विच्छेदोपरवोज्झितः~।\\
प्रत्यर्थिसंनिधी\renewcommand{\thefootnote}{4}\footnote{H अन्यार्थसंनिधानश्च for प्रत्यर्थिसंनिधानश्च.}नश्च पञ्चाङ्गो भोग इष्यते~॥}
\end{center}

भुक्तिमात्रेण साध्यासिद्धिमाह नारदः ( ४. ८६ )

\begin{quote}
{\vy \renewcommand{\thefootnote}{5}\footnote{नारद reads भोगं केवलतो यस्तु and स विज्ञेयस्तु तस्करः.}संभोगं केवलं यस्तु कीर्तयेन्नागमं क्वचित्~।\\
भोगच्छलापदेशेन विज्ञेयः स तु तस्करः~॥} इति
\end{quote}

\newpage
\fancyhead[CE,CO]{भुक्तिविचारः}
\fancyhead[RO]{[ $\S$ ५ }
\fancyhead[LE]{$\S$ ५ ]}
% ३१ भुक्तिविचारः [ §५

\noindent
इदं चागमस्मरणयोग्ये काले ज्ञेयम्~। तदयोग्ये तु भुक्तिमात्रमपि \renewcommand{\thefootnote}{1}\footnote{A प्रमाणमाह for प्रमाणमित्याह.}प्रमाणमित्याह स एव (नारद ४. ८९)

\begin{quote}
{\vy स्मार्ते काले क्रिया \renewcommand{\thefootnote}{2}\footnote{नारदस्मृति reads भुक्तेः for भूमेः.}भूमेः सागमा भुक्तिरिष्यते~।\\
अस्मार्तेनुगमाभावात्क्रमात्रिपुरुषागता~॥} इति~।
\end{quote}

\renewcommand{\thefootnote}{3}\footnote{B, D, F. omit अस्मार्ते\ldots संभवात्.}अनुगमाभावाद्योग्यानुपलब्ध्या \renewcommand{\thefootnote}{4}\footnote{C निश्चयस्य संभवात् for निश्चयस्यासंभवात्.}आगमाभावनिश्चयस्यासंभवात्~॥ अस्मार्तेप्यागमाभावस्मरणानुवृत्तावाह स एव (नारद ४. ८७)

\begin{quote}
{\vy अनागमं तु यो भुङ्क्ते बहून्यव्दशतान्यपि~।\\
चौरदण्डेन तं पापं दण्डयेत्पृथिवीपतिः~॥} इति~।
\end{quote}

यत्तु स एव (नारद ४. ९१)

\begin{quote}
{\vy अन्यायेनापि यद्भुक्तं \renewcommand{\thefootnote}{5}\footnote{नारदस्मृति reads पितुः for पित्रा and व्यव. मा. reads अपाकर्तुम्.}पित्रा पूर्वतरैस्त्रिभिः~।\\
न तच्छक्यमपाहर्तुं क्रमात्रिपुरुषागतम्~॥} इति
\end{quote}

तदागमं विनाप्यन्यायेनापि यत्पित्रा सह पूर्वतरैस्त्रिभिर्भुक्तं तदपाहर्तुं न शक्यं, \renewcommand{\thefootnote}{6}\footnote{B omits किं before पुनः;}किं पुनरागमाभावनिश्चयासंभव इति योज्यम्~। यदपि हारीतीयम्\renewcommand{\thefootnote}{7}\footnote{B, D हारीतायां for हारीतीयम्; F हारीततायाम्.}

\begin{quote}
{\vy यद्विनागममत्यन्तं भुक्तं पूर्वैस्त्रिभिर्भवेत्~।\\
न तच्छक्यम\renewcommand{\thefootnote}{8}\footnote{B, C, D, F, G, H, K अपाकर्तुं for अपाहर्तुम्.}पाहर्तुं क्रमात्रिपुरुषागतम्~॥} इति
\end{quote}

\newpage
page content missing

\newpage
\fancyhead[CE,CO]{भोगस्य प्रामाण्यम्}
% ३३ भोगस्य प्रामाण्यम् [§५

\noindent
उच्यते~। एतत्पश्यतोप्रतिषेधतस्ताव\renewcommand{\thefootnote}{1}\footnote{B, D, F तावत्कालीनभोगे तद्भूम्यादि० for तावत्कालीन\textendash\ तद्भूम्यादि०.}त्कालीनतद्भूम्यादिजन्यफलहानिर्भवतीत्येवंपरम्~। न तु तद्भूम्यादिवस्तुहानिरपीति~। अनागमं तु यो भुङ्क्ते (नारद ४. ८७ ) \textendash\ इत्युदाहृतवचनविरोधात्~॥ कात्यायनः 

\begin{quote}
{\vy नोपभोगे बलं कार्यमाहर्त्रा तत्सुतेन वा~।\\
पशुस्त्रीपुरुषादीनामिति धर्मो व्यवस्थितः~॥}
\end{quote}

नारदः ( ४. ८१ ) 

\begin{quote}
{\vy आधिः सीमा बालधनं निक्षेपोपनिधिस्त्रियः~।\\
राजस्वं श्रोत्रि\renewcommand{\thefootnote}{2}\footnote{नारदस्मृति reads श्रोत्रियद्रव्यं न भोगेन प्रणश्यति.}यस्वं च नोपभोगेन नश्यति~॥} ( $=$ मनु ८. १४९ )
\end{quote}

मनुः ( ८. १४६ ) 

\begin{quote}
{\vy संप्रीत्या भुज्यमानानि न नश्यन्ति कदाचन~।\\
धेनुरुष्ट्रवो \renewcommand{\thefootnote}{3}\footnote{B, D, F वहः शाखी यश्च for वहन्नश्वो यश्च ; H,G,Kवहत्राख्यो यश्च.}वहन्नश्वो यश्च दम्यः प्रयुज्यते~॥}
\end{quote}

दम्यः प्रयुज्यते दमनार्थं यः समर्प्यते~॥

\begin{center}
\textbf{\Large ॥~\renewcommand{\thefootnote}{4}\footnote{4.A,B,C,D,F,G,H,K,omit इति \ldots प्रकरणम्.}इति भुक्तिप्रकरणम्~॥~५~॥}\\

\vspace{2mm}
\textbf{\Large ॥~अथ साक्षिणः~॥~६~॥}
\end{center}

टोडरानन्दे नारदः ( ४. १४७ )

\begin{quote}
{\vy संदिग्धेषु\renewcommand{\thefootnote}{5}\footnote{नारद reads सन्दिग्धेषु च, श्रुतदृष्टानुभूतार्थात्, व्यक्तदर्शनम्.} तु कार्येषु द्वयोर्विवदमानयोः~।\\
दृष्टश्रुतानुभूतत्वात्साक्षिभ्यो व्यक्तदर्शनम्~॥}
\end{quote}

\newpage
\fancyhead[CE,CO]{साक्षिभेदाः}
\fancyhead[RO]{[$\S$ ६}
\fancyhead[LE]{$\S$ ६]}
% §६ ] साक्षिभेदाः ३४

तद्भेदानाह बृहस्पतिः 

\begin{quote}
{\vy लिखितो\renewcommand{\thefootnote}{1}\footnote{The व्यव. मा. (p. 321) reads लिखितोऽलिखितो गूढः.} लेखितो गूढः \renewcommand{\thefootnote}{2}\footnote{D, F स्मारितस्तुल्य० for स्मारितः कुल्य०.}स्मारितः कुल्यदूतकौ~।\\
यादृच्छिकश्चोत्तरश्च कार्यमध्य\renewcommand{\thefootnote}{3}\footnote{कार्यपृष्ठगतः.}गतोपरः~॥

नृपोध्यक्षस्तथा ग्रामः साक्षी द्वादशधा स्मृतः~॥}
\end{quote}

अर्थिना\renewcommand{\thefootnote}{4}\footnote{B, C, D, F, K. N omit अर्थिना. B, D,F पत्रनिवेशितः for पत्रे निवेशितः G.अर्थिना स्वयं पत्रे निवेशितः.} पत्रे निवेशितो लिखितः~। प्रत्यर्थिनार्थिप्रेरणया निवेशितो लेखितः~। \renewcommand{\thefootnote}{5}\footnote{H omits कुड्यादि \ldots श्रावणाद्वा.}कुड्यादिव्यवधानेन श्रावितो गूढः~। पुनः पुनः कार्यं \renewcommand{\thefootnote}{6}\footnote{F स्मर्यमाणः.}स्मार्यमाणः स्मारितः~। यदृच्छयैवागतः साक्षीक्रियमाणो यादृच्छिकः~। श्रवणाच्छ्रावणाद्वा साक्षिणामप्युपर्युपरि भाषमाण उत्तरः~। अध्यक्षः\renewcommand{\thefootnote}{7}\footnote{In G, H the words अध्यक्षः.\ldots वाक्यात् Precede कुड्यादि above.} प्राङ्विवाकः~। इदं च सभ्यादीनामुपलक्षणम्~। लेखकः प्राङ्विवाकश्च सभ्याश्चैवानुपूर्वशः \textendash\ इति कात्यायनवाक्यात्~। स एव

\begin{quote}
{\vy नव सप्त पञ्च वा स्युश्चत्वारस्त्रय एव वा~।\\
उभौ वा श्रोत्रियौ ग्राह्यौ नैकं पृच्छेत्कदाचन~॥

लिखितौ द्वौ तथा गूढौ त्रिचतुःपञ्च लेखिताः~।\\
यदृच्छास्मारिताः कुल्यास्तथा चोत्तरसाक्षिणः~॥

दूतकः खटिकाग्राही कार्यमध्यगतस्तथा~।\\
एक एव प्रमाणं स्यान्नृपोध्यक्षस्तथैव च~॥}
\end{quote}

न लिखितादीनामुभयानुमतत्व एकस्यापि ग्रहणमाह याज्ञवल्क्यः (२. ७२) 

\begin{quote}
{\vy \renewcommand{\thefootnote}{8}\footnote{omits उभयानु\ldots व्यासः.}उभयानुमतः साक्षी \renewcommand{\thefootnote}{9}\footnote{B,C,D,G,K. भवेदेकोपि for भवत्येकोपि.}भवत्येकोपि धर्मवित्~।}
\end{quote}

\newpage
\fancyhead[CE,CO]{साक्षिसंख्या}
% ३५ साक्षिसंख्या [ §६ 

व्यासः

\begin{quote}
{\vy शुचिक्रियश्च धर्मज्ञः साक्षी यस्त्वनुभूतवाक्~।\\
प्रमाण\renewcommand{\thefootnote}{1}\footnote{C,K. प्रमाणमेको भवति.}मेकोपि भवेत्साहसेषु विशेषतः~॥}
\end{quote}

अनुभूतवाक् प्रायो दृष्ट\renewcommand{\thefootnote}{2}\footnote{B, C, D, F,G, K ०सत्यवचनः.}सत्यवचाः~॥ निक्षेपादिष्वनाप्तमप्येकमाह कात्यायनः

\begin{quote}
{\vy अभ्यन्तरस्थनिक्षेपे साक्ष्यमेकोपि वाच्यते~।\\
अर्थिना प्रहितः साक्षी \renewcommand{\thefootnote}{3}\footnote{The वीर० reads अभ्यन्तरस्तु निक्षेपे and the स्मृतिच० {\qt आभ्यन्तरस्तु.} The स्मृतिच० reads भवत्येकोपि दूतकः.}भवेदेकोपि याचिते~॥}
\end{quote}

याचितं विवाहाद्यर्थमानीत\renewcommand{\thefootnote}{4}\footnote{K, नीतं for आनीतम्.}माभरणादिकं कुण्डलादि~। पण्यविवादेप्येकमाह स एव

\begin{quote}
{\vy संस्कृतं येन यत्पण्यं तत्तेनैव विभावयेत्~।\\
एक एव प्रमाणं स विवादे तत्र कीर्तितः~॥}
\end{quote}

तेषां च स्वरूपमाह व्यासः

\begin{quote}
{\vy धर्मज्ञाः \renewcommand{\thefootnote}{5}\footnote{F पौत्रिणः for पुत्रिणः.}पुत्रिणो मौलाः कुलीनाः सत्यवादिनः~।\\
श्रौतस्मार्तक्रियायुक्ता विगतद्वेषमत्सराः~॥

श्रोत्रिया न \renewcommand{\thefootnote}{6}\footnote{B, D, F न पदाधीनाः for न पराधीनाः}पराधीनाः सूर्यश्चाप्रवासिनः~।\\
युवानः साक्षिणः कार्या ऋणादिषु विजानता~॥}
\end{quote}

\newpage
\fancyhead[CE,CO]{साक्षिस्वरूपम्}
% §6] साक्षिस्वरूपम् ३६ 

नारदः ( ४. १५५ ) 

\begin{quote}
{\vy श्रेणिषु\renewcommand{\thefootnote}{1}\footnote{B, C, D, F, K श्रेणीषु for श्रेणिषु.} श्रेणिपुरुषाः स्वेषु वर्गेषु वर्गिणः~।\\
बहिर्वासिषु बाह्याः स्युः स्त्रियः स्त्रीषु च साक्षिणः~॥}
\end{quote}

वर्गिण आह कात्यायनः

\begin{quote}
{\vy लिङ्गिनः श्रेणिपूगाश्च वणिग्जा\renewcommand{\thefootnote}{2}\footnote{अपरार्क and वीर०read वणिग्व्राताः.}तास्तथापरे~।\\
समूहस्थाश्च ये चान्ये वर्गास्तानब्रवीद्भृगुः~॥

दासचारणमल्लानां हस्त्यश्वरथ\renewcommand{\thefootnote}{3}\footnote{C ,G,H,K० रथवाजिनां for ०रथजीविनाम् .अपरार्क and वीर० read इस्त्यश्वायुधजीविनाम् andप्रत्येकैकं समू०.}जीविनाम्~।\\
प्रत्येकैकसमूहानां नायका वर्गिणः स्मृताः~॥}
\end{quote}

विजातीया\renewcommand{\thefootnote}{4}\footnote{C, D, F,G,H,K विजातीयानप्याह\textendash\ }नाह याज्ञवल्क्यः ( २. ६९ ) 

\begin{quote}
{\vy त्र्यवराः साक्षिणो ज्ञेयाः श्रौतस्मार्तक्रियापराः~।\\
यथाजाति यथावर्णं सर्वे सर्वेषु वा स्मृताः~॥}
\end{quote}

\renewcommand{\thefootnote}{5}\footnote{A, E, M वर्ज्यानाह.}वर्ज्यानप्याह स एव ( याज्ञवल्क्य २. ७० \textendash\ ७१ ) 

\begin{quote}
{\vy स्त्रीबालवृद्धकितवमत्तोन्मत्ताभिशस्तकाः~।\\
रङ्गावतारिपाखण्डिकूटकृद्विकलेन्द्रियाः~॥

पतिताप्ता\renewcommand{\thefootnote}{6}\footnote{B, D, F, G,N पतितार्थाभिसंबन्धि०}र्थसंबन्धिसहायरिपुतस्कराः~।\\
साहसी द्वष्टदोषश्च निर्धूताद्यास्त्वसाक्षिणः~॥}
\end{quote}

निर्धूतःस्वजनैस्त्यक्तः~। आदिशब्देन दासाद्याः~। बृहस्पतिः

\begin{quote}
{\vy मातुः पिता पितृव्यश्च भार्याया भ्रातृमातुलौ~।\\
भ्राता सखा च जामाता सर्ववादेष्वसाक्षिणः~॥}
\end{quote}

\newpage
\fancyhead[CE,CO]{साक्षित्वेन वर्ज्याः}
% ३७ साक्षित्वेन वर्ज्याः [§६

नारदः ( ४. १६१ )

\begin{quote}
{\vy अनिर्दिष्टस्तु साक्षित्वे स्वयमेवैत्य यो वदेत्~।\\
स्वय\renewcommand{\thefootnote}{1}\footnote{नारदस्मृति reads\textendash\ सूचीत्युक्तःस शास्त्रेषु.}मुक्तिः स शास्त्रेषु न स साक्षित्वमर्हति~॥}
\end{quote}

कात्यायनः

\begin{quote}
{\vy साक्षिणां लिखितानां च निर्दिष्टानां च वादिना\renewcommand{\thefootnote}{2}\footnote{व्यव. मा. and वीर०read वादिनां for वादिना and भेदात्सर्वे न साक्षिण}~।\\
तेषामेकोन्यथावादी भवेत्सर्वे न साक्षिणः~॥}
\end{quote}

एतेषामपि कचित्साक्षित्वमाह नारदः ( ४. १८८ ) 

\begin{quote}
{\vy असाक्षिणो ये निर्दिष्टा दास\renewcommand{\thefootnote}{3}\footnote{A,E, M नैष्कृतिकादयः for नैकृतिकादयः नारदस्मृति and अपरार्क read नैकृतिकादयः. अपरार्क reads आश्रित्य for आसाद्य.}नैकृतिकादयः~।\\
कार्यगौरवमासाद्य भवेयुस्तेपि साक्षिणः~॥}
\end{quote}

असंभवे मनुः (८. ७० ) 

\begin{quote}
{\vy \renewcommand{\thefootnote}{4}\footnote{ B, D,F स्त्रियाथ संभवे.}स्त्रियाप्यसंभवे कार्यं बालेन स्थविरेण वा~।\\
शिष्येण बन्धुना वापि दासेन भृतकेन वा~॥}
\end{quote}

याज्ञवल्क्यः (२. ७२ ) 

\begin{quote}
{\vy सर्वः साक्षी संग्रहणे चौर्यपारुष्यसाहसे~॥}
\end{quote}

अत्र स्त्री\renewcommand{\thefootnote}{5}\footnote{C, D, F, G, H, K. स्त्रीसंग्रहादीनां.}संग्रहणादीनां साहसत्वेपि पृथगुपादानं गुप्ततया क्रियमाणस्त्रीसंग्र\renewcommand{\thefootnote}{6}\footnote{C,D,F,G,H,K, ०संग्रहपरम्.}हादिपरम्~। उशनाः

\begin{quote}
{\vy दासोन्धो बधिरः कुष्ठी स्त्रीबालस्थविरादयः~।\\
एतेप्यनभिसंबद्धाः साहसे साक्षिणो मताः~॥}
\end{quote}

\newpage
\fancyhead[CE,CO]{साक्षिदोषाः कदा वक्तव्याः}
% §६ ] साक्षिदोषाः कदा वक्तव्याः ३८ 

अनभिसंबद्धा अपक्षपातिनः~॥ बृहस्पतिः

\begin{quote}
{\vy \renewcommand{\thefootnote}{1}\footnote{स्मृतिच०reads साक्षिणोर्थिसमु० and वीर०reads साक्षिणोर्थी समुपद्दिष्टान्. B,C,D,F साक्षिणोर्थसमुद्दिष्टान्.}साक्षिणोर्थे समुद्दिष्टान्सत्सु दोषेषु दूषयेत्~।\\
अदुष्टान्दूषयन्वादी तत्समं दण्डमर्हति~॥}
\end{quote}

वाद्यत्र प्रत्यर्थी~। तत्समं विवादविषयीभूतद्रव्यसमम्~। व्यासः

\begin{quote}
{\vy साक्षिदोषाः प्रयोक्तव्याः संसदि प्रतिवादिना~।\\
पत्रेभिलिखितान्सर्वान् वाच्याः प्रत्युत्तरं तु ते~॥}
\end{quote}

पत्रे लिखितान्साक्षिदोषान्प्रतिलक्षीकृत्य तद्विषयं परिहारं ते साक्षिणः सभ्यैर्वाचनीया इत्यर्थः~। स एव

\begin{quote}
{\vy प्रतिपत्तौ न साक्षित्वमर्हन्ति तु कथंचन~।\\
अतोन्यथा भावनीयाः क्रियया प्रतिवादिना~॥

\renewcommand{\thefootnote}{2}\footnote{अपरार्क reads असाधयन् for अभावयन्.}अभावयन्दमं दाप्यः प्रत्यर्थी साक्षिणः स्फुटम्~।\\
भाविताः साक्षिणो वर्ज्याः साक्षिधर्मनिराकृताः~॥

तथैव\renewcommand{\thefootnote}{3}\footnote{B, C, D, F, H, K, N omit the words तथैव\ldots व्यवस्थित इति अपरार्क and वीर० read जितः स विनयं दाप्यः.} विनयं दाप्यः शास्त्रदृष्टेन कर्मणा~।
यदि वादी निराकाङ्क्षः साक्षि\renewcommand{\thefootnote}{4}\footnote{A साक्षिसत्ये for साक्षिसत्त्वे. अपरार्क and परा. मा read साक्षिसत्ये.}सत्त्वे व्यवस्थितः~॥} इति~।
\end{quote}

अतोन्यथा असंप्रतिपत्तौ~। भावनीयाः अङ्गीकारयितव्याः~। क्रियया प्रमाणेन~। स्फुटं यथा स्यात्तथा\renewcommand{\thefootnote}{5}\footnote{C,D तथा भावयन् for तथाभावयन्.}भावयन्नित्यन्वयः~॥ यत्तु

\begin{quote}
{\vy सभासदां दूषणं यल्लोकसिद्धमथापि वा~।\\
साक्षिणां दूषणं ग्राह्यं न\renewcommand{\thefootnote}{6}\footnote{अपरार्क, स्मृतिच०, परा. मा. and वीर०read असाध्यम् परा. मा. reads नान्यदिष्यते for दोष० and स्मृतिच० reads दोषवर्णनात् ; while व्यव. मा. reads दोषदर्शनात्.} साध्यं दोषवर्जनात्~॥} इति
\end{quote}

\newpage
\fancyhead[CE,CO]{साक्षिदोषाः कदा केन वक्तव्याः}
%३९ साक्षिदोषाः कदा केन वक्तव्याः [ §६ 

\renewcommand{\thefootnote}{1}\footnote{B, D, F ०वर्जनादिति तदा सत्वेन ( तत्त्वेन ? ) लोकेवधारितसाक्षिविषयम्}तल्लोकावधारिताप्तसाक्षिविषयम्~। प्रतिवादिना दोषाज्ञाने स एव

\begin{quote}
{\vy प्रमाणस्य हि ये दोषा वक्तव्यास्ते विवादिना~।\\
\renewcommand{\thefootnote}{2}\footnote{N omits गूढास्तु\ldots. प्रदर्शनात्.}गूढास्तु प्रकटाः सभ्यैः काले शास्त्रप्रदर्शनात्~॥}
\end{quote}

गूढाः शास्त्रप्रदर्शनेन साक्षिवादात्पूर्वकाले वक्तव्या इत्यर्थः~। अन\renewcommand{\thefootnote}{3}\footnote{B, D, F, H अनन्तरं तु वक्तव्याः.}न्तरं तु न वक्तव्या इत्याह बृहस्पतिः

\begin{quote}
{\vy \renewcommand{\thefootnote}{4}\footnote{लेख्ये दोषास्तु for लेख्यदोषास्तु.}लेख्यदोषास्तु ये केचित्साक्षिणां चैव ये स्मृताः~।\\
वादकाले तु वक्तव्याः पश्चादुक्तान्न दूषयेत्~॥}
\end{quote}

\renewcommand{\thefootnote}{5}\footnote{D, F omit the words उक्तानुक्त० \ldots कर्तरि क्तः B, C,H, K, N omit वक्तुमारब्ध\ldots..कर्तरि क्तः.}उक्तान् उक्तवतः वक्तुमारव्धवत इत्यर्थः~। आदिकर्मणि क्तः कर्तरि च ( पा. ३.४. ७१ ) \textendash\ इति कर्तरि क्तः~॥

अत्र दण्डमाह कात्यायनः

\begin{quote}
{\vy उक्तेर्थे साक्षिणो यस्तु दूषयेत्प्रागदूषितान्~।\\
\renewcommand{\thefootnote}{6}\footnote{परा. मा.reads स च तत्कारणं for न च \& c.}न च तत्कारणं ब्रूयात्प्राप्नुयात्पूर्वसाहसम्~॥}
\end{quote}

साक्षिणां स्वदूषणपरिहाराशक्तावर्थी तं कुर्यादित्याह बृहस्पतिः

\begin{quote}
{\vy लेख्यं वा साक्षिणो वापि विवादे यस्य दूषिताः~।\\
तस्य कार्य न \renewcommand{\thefootnote}{7}\footnote{F न सिध्येत.}सिध्येत्तु याव\renewcommand{\thefootnote}{8}\footnote{C. यावन्नात्र ; H यावत्तत्र.}त्तन्न विशोधयेत्~॥}
\end{quote}

\noindent
तल्लेख्यादिकम्~॥

कूटसाक्षिकर्तुर्दण्डमाह कात्यायनः

\newpage
\fancyhead[CE,CO]{कूटसाक्षिकर्तुर्दण्डः}
% §६ ] कूटसाक्षिकर्तुर्दण्डः ४०

\begin{quote}
{\vy येन कार्यस्य लोभेन निर्दिष्टाः कूटसाक्षिणः~।\\
गृहीत्वा तस्य सर्वस्वं कुर्यान्निर्विषयं ततः~॥}
\end{quote}

\noindent
निर्विषयं विवादविषयीभूतार्थरहितम्~॥ 

दुष्टसाक्षि\renewcommand{\thefootnote}{1}\footnote{D,F.०साक्षिनिश्चयो० for साक्षिणां निश्चयो०.}णां निश्चयोपायमाह नारदः ( ४. १९३ \textendash\ १९६ )

\begin{quote}
{\vy यस्त्वात्मदोषदुष्ट\renewcommand{\thefootnote}{2}\footnote{नारद reads भिन्नत्वात् for दुष्टत्वात्.}त्वादस्वस्थ इव लक्ष्यते~।\\
स्थानात्स्थानान्तरं गच्छेदेकैकं \renewcommand{\thefootnote}{3}\footnote{A,C,E,G,H,K,M वानुधावति.}चानुधावति~॥

का\renewcommand{\thefootnote}{4}\footnote{D काकत्य० for कासत्य०. नारद reads क्रामत्यनिभृतोकस्मात्.}सत्यकस्माच्च भृशमभीक्ष्णं\renewcommand{\thefootnote}{5}\footnote{F अभीक्ष्णे for अभीक्ष्णम्.} निश्वसित्यपि~।\\
विलिखत्यवनिं पद्भ्यां बाहू वासश्च धूनयेत्~॥

भिद्यते\renewcommand{\thefootnote}{6}\footnote{D,F विद्यते for भिद्यते.} मुखवर्णोस्य ललाटं स्विद्यते तथा~।\\
शोषमागच्छतश्चौष्ठावूर्ध्वं तिर्यक् च वीक्षते~॥

त्वरमाण \renewcommand{\thefootnote}{7}\footnote{नारद reads इवाकस्मादपृष्टो.}इवाबद्धमपृष्टो बहु भाषते~।\\
कूटसाक्षी स विज्ञेयस्तं पापं\renewcommand{\thefootnote}{8}\footnote{D omits पापं\ldots पृच्छेदृतम्.} विनयेद्भृशम्~॥}
\end{quote}

साक्षिणां प्रश्नप्रकारमाह कात्यायनः \renewcommand{\thefootnote}{9}\footnote{F,K,N omit मनुरपि.}मनुरपि ( ८. ८७ , ७९ \textendash\ ८०)

\begin{quote}
{\vy देवब्राह्मणसान्निध्ये साक्ष्यं पृच्छेदृतं\renewcommand{\thefootnote}{10}\footnote{वीर० reads ऋते द्विजान्.} द्विजान्~।\\
उदङ्ग\renewcommand{\thefootnote}{11}\footnote{A,E,M.प्राङ्मुखांश्चfor प्राङ्मुखान्वा.}मुखान्वा \renewcommand{\thefootnote}{12}\footnote{B,D,F,G,H. पूर्वाहे वा for पूर्वाह्ने च.}पूर्वाह्ने च शुचिः शुचीन्~॥}
\end{quote}

\newpage
\fancyhead[CE,CO]{साक्षिणां प्रश्नप्रकारः}
% ४१ साक्षिणां प्रश्नप्रकारः [§६

\begin{quote}
{\vy सभान्तः साक्षिणः \renewcommand{\thefootnote}{1}\footnote{मनु० reads प्राप्तान् for सर्वान्}सर्वानर्थिप्रत्यर्थिसंनिधौ~।\\
प्राड्विवाकोनुयुञ्जीत विधिनानेन सान्त्वयन्~॥

यद्वयोरनयोर्वेत्थ कार्येस्मिं\renewcommand{\thefootnote}{2}\footnote{D,F कार्येस्मिन्धिष्ठितम्.}श्चेष्टितं मिथः~।\\
तद्ब्रूत सर्वं सत्येन युष्माकं ह्यत्र साक्षिता~॥}
\end{quote}

गवाश्वादिविवादेषु विप्रति\renewcommand{\thefootnote}{3}\footnote{H विप्रतिपत्त्यर्थ० for विप्रतिपन्नार्थ०.}पन्नार्थसान्निध्यमप्याह स एव 

\begin{quote}
{\vy अर्थिप्रत्यर्थिसान्निध्ये साध्यार्थस्य च संनिधौ~।\\
प्रत्यक्षं चोदयेत्\renewcommand{\thefootnote}{4}\footnote{वीर० reads वादयेत् for चोदयेत्.}साक्ष्यं न परोक्षं कथंचन~॥

अर्थस्योपरि कर्तव्यं तयोरपि विना क्वचित्~।\\
चतुष्पादेषु धर्मोयं द्विपदस्थावरेषु च~॥

तौल्यगणिममेयानामभावेपि हि वादयेत्~।\\
क्रियाकारेषु \renewcommand{\thefootnote}{5}\footnote{H. पूर्वेषु for सर्वेषु०}सर्वेषु साक्षित्वं न \renewcommand{\thefootnote}{6}\footnote{A., B, C, D, E, M तत्त्वतो० for न त्वतो०.}त्वतोन्यथा~॥}
\end{quote}

तयोरपि विना क्वचित् तयोर्वादि\renewcommand{\thefootnote}{7}\footnote{B, C, D, H तयोर्वादिनोर्विनाप्यधसंनिधाने ; K तयोर्वादिनो विना अपि.}प्रतिवादिनोर्विनाप्यर्थसंनिधाने क्वचिच्चतुष्पादादिष्वित्यर्थः~। तौल्यं तोलनयोग्यं सुवर्णादि~। गणिमं गणनीयं नाणकादि~। मेयं मानयोग्यं व्रीहिगोधूमादि~। अभावेपि साध्यार्थसंनिधानाभावेपि~। क्रियाकारेषु व्यवहारेषु~॥ वधरूपे विवादपदे साक्षिभाषणं शिवसंनिधावित्याह स एव

\begin{quote}
{\vy वधे चेत्प्राणिनां साक्ष्यं \renewcommand{\thefootnote}{8}\footnote{परा. मा. and वीर०read वादयेच्छवसंनिधौ.}वादयेच्छिवसंनिधौ~।\\
तदभावे तु चिह्नस्य\renewcommand{\thefootnote}{9}\footnote{विप्रस्य for चिह्नस्य.} नान्यथै\renewcommand{\thefootnote}{10}\footnote{B, C, D, H, K नान्वथैव for नान्यथैतत् स्मृतिच०, परा. मा. and वीर० read नान्यथैव. ,B,D विवादयेत् for प्रवादयेत्.}तत्प्रवादयेत्~॥}
\end{quote}

\newpage
% §६ ] साक्षिप्रभाषणे कालहरणं न कार्यम् ४२

\noindent
तत्साक्षि\renewcommand{\thefootnote}{1}\footnote{B,D भाषणं च चिह्वस्याभावे;~। F चिह्वस्य भावे कार्ये}भाषणं वधचिह्नस्याभावे कार्यम्~। अन्यथा वधचिह्नसद्भावे~॥ स एव

\begin{quote}
{\vy न कालहरणं कार्य राज्ञा\renewcommand{\thefootnote}{2}\footnote{F राज्ञां साक्षिप्रभाषणं for राज्ञा साक्षिप्रभाषणे.} साक्षिप्रभाषणे~।\\
महादोषो भवेत्कालाद्धर्मव्यावृत्तिलक्षणः~॥}
\end{quote}

नारदः ( ४. १९८ )

\begin{quote}
{\vy आहूय साक्षिणः पृच्छेन्नियम्य शपथैर्भृशम्~।\\
समस्तान्विदिताचारान्विज्ञातार्थान्पृथक् पृथक्~॥}
\end{quote}

वसिष्ठः

\begin{quote}
{\vy समवेतैस्तु यद्दृष्टं वक्तव्यं तु तथैव तत्~।\\
विभिन्नेनैर्व\renewcommand{\thefootnote}{3}\footnote{अपरार्क reads विभिन्नैककार्य for विभिन्नेनैव तत्कार्य ; परा मा. and वीर० read यत्कार्य for तत्कार्यम्} तत्कार्यं वक्तव्यं तु पृथक् पृथक्~॥

भिन्नकाले तु यत्कार्यं विज्ञातं यत्र साक्षिभिः~।\\
एकैकं वादयेत्तत्र विधिरेष प्रकीर्तितः~॥}
\end{quote}

मनुः ( ८. ११३, १०२ )

\begin{quote}
{\vy सत्येन शापयेद्विप्रं क्षत्रियं वाहनायुधैः~।\\
गोबीजकाश्चनैर्वैश्यं शूद्रं सर्वैस्तु पातकैः~॥

गोरक्षकान्वाणिजकांस्तथा कारुकुशीलवान्~।\\
प्रेष्यान्वार्धुषिकांश्चैव विप्रान् शूद्रवदाचरेत्~॥

ये व्यपेताः स्वकर्मभ्यः पर\renewcommand{\thefootnote}{4}\footnote{B,C,D,F,G परपीडोपजीविनः for परपिण्डोपजीविनः}पिण्डोपजीविनः~।\\
द्विजत्वमभि\renewcommand{\thefootnote}{5}\footnote{B,D,F अभिकाङ्क्षन्ति for अभिवाञ्छन्ति.}वाञ्छन्ति तांस्तु शूद्रवदाचरेत्~॥} इति~।
\end{quote}

\newpage 
\fancyhead[CE,CO]{साक्षिवचनपरीक्षा}
% ४३ साक्षिवचनपरीक्षा [ §६

\noindent
अन्यथा ब्रुवतः सत्यं ते नश्यतीत्येवमादिप्रकारेणेत्यर्थः~॥ साक्षिवचन\renewcommand{\thefootnote}{1}\footnote{B, D, F परीक्षोपायमाह for परीक्षामाह}परीक्षामाह~।

\begin{quote}
{\vy देशकालवयो\renewcommand{\thefootnote}{2}\footnote{वीर०reads वयोजातिसंज्ञाद्रव्यप्रमाणतः and परा.मा.reads जाति for जाति.}द्रव्यसंज्ञाज्ञातिप्रमाणता~।\\
अन्यूनं चेन्निगदितं सिद्धं साध्यं विनिर्दिशेत्~॥}
\end{quote}

साक्षिणां\renewcommand{\thefootnote}{3}\footnote{D,F साक्षिणः for साक्षिणाम्.} विप्रतिपत्तौ निर्णयमाह याज्ञवल्क्यः (२. ७८ )

\begin{quote}
{\vy द्वैधे बहूनां वचनं समेषु गुणिनां तथा~।\\
गुणिद्वैधे तु वचनं ग्राह्यं ये गुणवत्तमाः~॥}
\end{quote}

साक्ष्यमङ्गीकृत्यानभिधाने दण्डमाह स एव ( याज्ञवल्क्य २. ७६ )

\begin{quote}
{\vy अब्रुवन् हि नरः साक्ष्यमृणं सदशबन्धकम्~।\\
राज्ञा\renewcommand{\thefootnote}{4}\footnote{H omits राज्ञा\ldots सदशबन्धकम्} सर्वं प्रदाप्यः स्यात् षट्चत्वारिंशकेहनि~॥}
\end{quote}

सर्वं वृद्धिसहितम्~। सदशबन्धकं दशमांशसहितम्~। दशमांशो राज्ञा ग्राह्यः सवृद्धिकमृणमुत्तमर्णेन ग्राह्यमिति मिताक्षरायाम्~॥ जानतः साक्ष्यानङ्गीकारे दण्डमाह स एव ( याज्ञवल्क्य २. ८२ )

\begin{quote}
{\vy यः साक्ष्यं श्रावितोन्येभ्यो निह्नुते तत्तमोवृतः~।\\
स दाप्योष्टगुणं दण्डं ब्राह्मणं तु विवासयेत्~॥}
\end{quote}

विवादपराजये यो \renewcommand{\thefootnote}{5}\footnote{B,C,G,K तं दण्डमष्टगुणं ; D यो दण्डमष्टगुणम्}दण्डस्तमष्टगुणं दाप्यः~। दण्डदानासमर्थस्तु ब्राह्मणो निर्वास्यः क्षत्रियादिः स्वोचितं कर्म कारयितव्य इति मिताक्षरायाम्~। मनुः (८. १०८ )

\begin{quote}
{\vy यस्य दृश्येत सप्ताहादुक्तवाक्यस्य साक्षिणः~।\\
रोगोग्निर्ज्ञातिमरणमृणं दाप्यो दमं च सः~॥}
\end{quote}

\newpage
\fancyhead[CE,CO]{साक्षिषु तारताम्यविचारः}
% §६] साक्षिषु तारताम्यविचारः ४४

याज्ञवल्क्यः ( २. ८० )

\begin{quote}
{\vy \renewcommand{\thefootnote}{1}\footnote{B, F read before उक्तेपि साक्षिभिः the verse वर्णिनां हि \ldots द्विजैः and add शूद्रस्य प्रायश्चित्तमाह विष्णुः~। उक्तेपि साक्षिभिः \& c.}उक्तेपि साक्षिभिः साक्ष्ये यद्यन्ये गुणवत्तमा\renewcommand{\thefootnote}{2}\footnote{B, C, D, F, G, K गुणवत्तराः.}~।\\
द्विगुणा वान्यथा ब्रूयुः कूटाः स्युः पूर्वसाक्षिण\renewcommand{\thefootnote}{3}\footnote{B, D, F add after साक्षिणः {\qt  अनृतवदनानुक्षिणः} (?) ; G adds अनृतवादिनः}~॥}
\end{quote}

नारदः ( १. ६२ )

\begin{quote}
{\vy निर्णिक्ते\renewcommand{\thefootnote}{4}\footnote{नारदस्मृति reads {\qt निर्णिक्तव्यवहारेषु}.}व्यवहारे तु प्रमाणमफलं भवेत्~।\\
लिखितं साक्षिणो वापि पूर्वमावेदितं न चेत्~॥}
\end{quote}

क्वचित्साक्षिणोनृतवचनानुज्ञां प्रायश्चित्तं चाह याज्ञवल्क्यः (२. ८३)

\begin{quote}
{\vy वर्णिनां\renewcommand{\thefootnote}{5}\footnote{C,H,K वर्णिनो हि for वर्णिनां हि.} हि वधो यत्र तत्र साक्ष्यनृतं वदेत्~।\\
तत्पावनाय निर्वाप्यश्चरुः सारस्वतो द्विजैः~॥}
\end{quote}

शूद्रस्य प्रायश्चित्तमाह विष्णुः ( विष्णुधर्मसूत्र ८. १७ ) शूद्रश्चै\renewcommand{\thefootnote}{6}\footnote{B, D, F ०श्चैकाह्रिकं ; विष्णुधर्मसूत्र reads शूद्र एकाह्निकम्}काहिकं गोदशकस्य ग्रासं दद्यात् \textendash\ इति~। ऐकाहिकमेकस्मिन्नहनि भक्षयितुं पर्याप्तम्~॥

\begin{center}
\textbf{\Large ॥~\renewcommand{\thefootnote}{7}\footnote{A, F omit इति\ldots समाप्तम् C, G, H, K read इति साक्षिनिरूपणं for इति\ldots समाप्तम्.}इति साक्षिप्रकरणं समाप्तम्~॥~६~॥}\\

\vspace{2mm}
\textbf{\Large ॥~अथ दिव्यम्~॥~७~॥}
\end{center}

तत्तु मानुषप्रमाणानिर्णीतार्थनिर्णायकम्~। तच्च तत्कालकालान्तरनिर्णायकतया द्वेधा~। तत्राद्यभेदानाह बृहस्पतिः

\newpage
\fancyhead[CE,CO]{दिव्यभेदाः}
\fancyhead[LE]{$\S$ ७]}
\fancyhead[RO]{[$\S$ ७}
% ४५ दिव्यभेदाः [§७

\begin{quote}
{\vy \renewcommand{\thefootnote}{1}\footnote{B.D.घटो for धटो;F घटाग्नि० for धटोग्नि०;K घटोनिरुदकं}घटोग्निरुदकं चैव विषं कोशश्च पञ्चमम्~।\\
षष्ठं च तण्डुलाः प्रोक्तं सप्तमं तप्तमाषकाः\renewcommand{\thefootnote}{2}\footnote{A, C,G, H ॰माषकः for०माषकाः~। D माषकम्; K समाषकः ; F omits तप्तमाषकाः\ldots नवमम्.}~॥

अष्टमं फालमित्युक्तं नवमं धर्मजं स्मृतम्~।}
\end{quote}

तत्राद्यानि पञ्च महाभियोग एवेत्याह याज्ञवल्क्यः ( २. ९५ )

\begin{quote}
{\vy तुलाग्र्यापो विषं कोशो दिव्यानीह विशुद्धये~।\\
महाभियोगेष्वेतानि शीर्षकस्थेभियोक्तरि~॥}
\end{quote}

शीर्षकस्थे \renewcommand{\thefootnote}{3}\footnote{A पराजयिक०for पाराजयिक०}पाराजयिकदण्डभाजि~॥ पितामहः

\begin{quote}
{\vy सावष्टम्भाभियुक्तानां धटादीनि विनिर्दिशेत्~।\\
तण्डुलाश्चैव कोशश्च \renewcommand{\thefootnote}{4}\footnote{A शङ्कास्वेव नियोजयेत् for शङ्कास्वेतानि योजयेत्}शङ्कास्वेतानि योजयेत्~॥} इति~।
\end{quote}

अवष्टम्भो\renewcommand{\thefootnote}{5}\footnote{B, C, F, K omit अवष्टम्भो निश्चयः.} निश्चयः~। तेन कोशोवष्टम्भे शङ्कायां च भवति~। कालिकापुराणे

\begin{quote}
{\vy परदाराभिशापे च \renewcommand{\thefootnote}{6}\footnote{H omits चौर्या \ldots शापे च.}चौर्यागम्यागमेषु च~।\\
महापातकशस्ते च स्याद्दिव्यं नृपसाहसे~॥

विप्रतिपत्तौ विवादेवर्णस्य स्थापने कृते~।\\
तत्रैव स्थापयेद्दिव्यं शिरःपूर्वं महीपतिः~॥

परदाराभिशापे च बहवो यत्र वादिनः~।\\
शिरोहीनं भवेद्दिव्यमात्मनः शुद्धिकारणात्~॥}
\end{quote}

अगम्याः परदारभिन्नाः \renewcommand{\thefootnote}{7}\footnote{G परदाराभिन्नाः for परदारभिन्नाः.}साधारणा वेश्याद्याः~। शस्तेभिशस्ते~। साहसं बलादन्यायः~। अवर्णो निन्दा~। \renewcommand{\thefootnote}{8}\footnote{B, C, D, F. K omit शिरो दण्डः}शिरो दण्डः~। परदाररूपं

\newpage
\fancyhead[CE,CO]{दिव्यभेदव्यवस्था}
% §७ ] दिव्यभेदव्यवस्था ४६

\noindent
विशेषणमविवक्षितं अभिशापस्यानुवाद्यत्वात्~। एवं \renewcommand{\thefootnote}{1}\footnote{C एवं च बहवः.}बहव इत्याद्यपि~। तेन वाद्यभावेपि सर्वाभिशापे\renewcommand{\thefootnote}{2}\footnote{D, F भिशापेपि दिव्यं for भिशापे दिव्यम्} दिव्यं भवति~। शुद्धिकारणादिति हेतुनिर्देशोप्येवं संगच्छते~। विनापि शीर्षकं कुर्यान्नृपद्रोहेथ पातके ( याज्ञवल्क्य २. ९६ ) \textendash\ इति सामान्योक्तिश्च\renewcommand{\thefootnote}{3}\footnote{G omits च after सामान्योक्तिः.}~॥ नारदः\renewcommand{\thefootnote}{4}\footnote{H reads after नारदः प्रायश्चित्तं चाह याज्ञवल्क्यः~। वर्णिनो हि वधो (the verse above ).}

\begin{quote}
{\vy राजभि शङ्कितानां च निर्दिष्टानां च दस्युभिः~।\\
आत्मशुद्धिपराणां च देयं दिव्यं शिरो विना~॥} इति~।
\end{quote}

कालान्तरे निर्णायकं दिव्यं \renewcommand{\thefootnote}{5}\footnote{B, D, F शापः for शपथः.}शपथः~॥ तद्भेदानाह नारदः (४. २४८)

\begin{quote}
{\vy सत्यं \renewcommand{\thefootnote}{6}\footnote{D, F,G सत्यवाहन० for सत्यं वाहन०, K सस्यवाहन०.}वाहनशस्त्राणि गोवीजकनकानि च~।\\
देवतापितृपादांश्च दत्तानि सुकृतानि च~॥

स्पृशेच्छिरांसि पुत्राणां दाराणां सुहृदां तथा~।\\
अभियोगेषु सर्वेषु कोश\renewcommand{\thefootnote}{7}\footnote{B, K कोशयान० for कोशपान०; C कोशपात०.}पानमथापि वा~॥}

इत्येते शपथाः प्रोक्ता मनुना स्वल्प\renewcommand{\thefootnote}{8}\footnote{नारद० read कारणे for ०कारणात् and दिव्यतत्त्व {\qt ०कारणाः}}कारणात्~॥ (नारद ४. २५०)~।
\end{quote}

कोशस्य कालान्तरनिर्णायकत्वेपि महाभियोगविषयतया आद्येषु पाठः~। याज्ञवल्क्यः ( २. ९६ ) 

\begin{quote}
{\vy रुच्या वान्यतरः कुर्यादितरो \renewcommand{\thefootnote}{9}\footnote{B, D,F वर्जयेत्.}वर्तयेच्छिरः~॥}
\end{quote}

\newpage
\fancyhead[CE,CO]{दिव्येष्वधिकारिणः}
% ४७ दिव्येष्वधिकारिणः [ §७ 

\noindent
अयं च विकल्पोभियोक्तु\renewcommand{\thefootnote}{1}\footnote{B, C, D. F, K अभियोज्येच्छाया० for अभियोक्तुरिच्छाया०.}रिच्छायामेव~। तदनिच्छायां त्वभियोज्यस्यैव दिव्यम्~।

\begin{quote}
{\vy न कश्चिदभियोक्तारं दिव्येषु विनियोजयेत्~।\\
अभियुक्ताय दातव्यं दिव्यं दिव्यविशारदैः~॥} इति
\end{quote}

\noindent
दिव्यतत्त्वे \renewcommand{\thefootnote}{2}\footnote{B, C, D, F, K कातीयोक्तेः.}कात्यायनीयोक्तेः~॥

अथाधिकारिव्यवस्था\renewcommand{\thefootnote}{3}\footnote{A०व्यवस्थाः for ०व्यवस्था.}~। याज्ञवल्क्यः ( २. ९८ )

\begin{quote}
{\vy तुला स्त्रीबालवृद्धान्धपङ्गुब्राह्मणरोगिणाम्~।\\
अग्निर्जलं वा शूद्रस्य यवाः सप्त विषस्य वा~॥}
\end{quote}

स्त्री जातिवयोवस्थाविशेषानादरेण~। षोडशाब्दावधिको बालः सर्वजातीयोपि~। अशीतिवर्षाधिको वृद्धः~। अत्र ब्राह्मणस्य वक्ष्यमाण\renewcommand{\thefootnote}{4}\footnote{B, D. F वक्ष्यमाणासाधारण० for वक्ष्यमाणसाधारण.} साधारणकाले तुलैव नियता~। अग्न्यादिकाले तु तान्यपि भवन्ति~।

अत एव पितामहः
\begin{quote}
{\vy सर्वेषामेव वर्णानां कोशाच्छुद्धिविर्धीयते~।\\
सर्वाण्येतानि सर्वेषां ब्राह्मणस्य विना\renewcommand{\thefootnote}{5}\footnote{B, C, D, F,G, K विषं विना for विना विषम्. मिता० and दिव्यत्व read विषं विना.} विषम्~॥} इति~।
\end{quote}

कालिकापुराणे

\begin{quote}
{\vy वर्णान्त्यस्य सदा देयं मापकं तप्तहेमजम्~॥}
\end{quote}

नारदः 

\begin{quote}
{\vy क्लीबान्नरान्सत्त्वहीनान्परितश्चार्दितान्तरान्\\
बालवृद्धातुरान्स्त्री\renewcommand{\thefootnote}{6}\footnote{वीर० reads क्लीबातुरान् सत्त्व and ०र्दितान्नरान्}श्चपरीक्षेत घटे सदा~॥}
\end{quote}

\newpage 
\fancyhead[CE,CO]{दिव्यभेदेषु अधिकारिव्यवस्था}
% §७ ] दिव्यभेदेषु अधिकारिव्यवस्था ४८

\begin{quote}
{\vy स्त्रीणां तु न विषं प्रोक्तं न चापि सलिलं स्मृतम्~।\\
घटकोशादिभिस्तासामन्त\renewcommand{\thefootnote}{1}\footnote{C,K अतस्तत्त्वं for अन्तस्तत्त्वम्.}स्तत्त्वं विचारयेत्~॥

नार्तानां तोयशुद्धिः स्यान्न विषं पित्तरोगिणाम्~।\\
श्वित्र्यन्धकुनखादीनां नाग्निक\renewcommand{\thefootnote}{2}\footnote{C,K नाग्निकार्यम्.}र्म विधीयते~॥

न मज्जनीया\renewcommand{\thefootnote}{3}\footnote{नारद reads न मज्जनीयं स्त्रीबालम्.} स्त्रीबाला धर्मशास्त्रविचक्षणैः~।\\
रोगिणो ये च वृद्धाः स्युः पुमांसो ये च दुर्बलाः~॥

निरुत्साहान्व्याधिक्लिष्टा\renewcommand{\thefootnote}{4}\footnote{A, E, H,M क्लिष्टांस्तांस्तोये न निमज्जयेत् for ०क्लिष्टान्नार्तास्तोये निमज्जयेत्.}न्नार्तांस्तोये निमज्जयेत्~।\\
सद्यो म्रियन्ते मज्जन्तः स्वल्पप्राणा हि ते स्मृताः~॥

साहसेप्यागतानेतान्नैव तोये निमज्जयेत्~।\\
न चापि हारये\renewcommand{\thefootnote}{5}\footnote{F अग्निमविशेषे न शोधयेत्.}दग्निं न विषेण विशोधयेत्~॥} ( last four are नारद ४. २५५ , ३१३ \textendash\ ३१५ )
\end{quote}

विष्णु ( $=$ विष्णुधर्मसूत्र ९. २९ ) 

\begin{quote}
{\vy न श्लेष्मिणां\renewcommand{\thefootnote}{6}\footnote{C,G,K श्लेष्मिकाणां ; , D,F श्लेष्मिकानाम्} व्याधितानां भीरूणां श्वासा\renewcommand{\thefootnote}{7}\footnote{B श्वासकासिकानां.}कासिनाम्~।}
\end{quote}

कात्यायनः

\begin{quote}
{\vy न लोहशिल्पिनामग्निः सलिलं नाम्बुसेविनाम्~।\\
मन्त्रयोगविदां चैव विषं दद्यात्तु न क्वचित्~॥

तण्डुलैर्न नियुञ्जीत\renewcommand{\thefootnote}{8}\footnote{C न तु युञ्जीत for न नियुञ्जीत.} व्रतिनं मुखरोगिणम्~॥}
\end{quote}

\newpage
\fancyhead[CE,CO]{दिव्यभेदव्यवस्था}
% ४९ दिव्यभेदव्यवस्था [ §७ 

व्रती पयोव्रतादिमान्~। पितामहः

\begin{quote}
{\vy मद्यपस्त्री\renewcommand{\thefootnote}{1}\footnote{B, D,F व्यवसिनां for व्यसनिनाम्.}व्यसनिनां कितवानां तथैव च~।\\
कोशः प्राज्ञैर्न दातव्यो ये च नास्तिकवृत्तयः~॥}
\end{quote}

नारदः (४. ३३२)

\begin{quote}
{\vy महापराधे निर्धर्मे कृतघ्ने क्लीबकुत्सिते~।\\
\renewcommand{\thefootnote}{2}\footnote{नारद reads नास्तिकव्रात्यदासेषु कोशपानम्.}नास्तिके दृष्टदोषे च कोशदानं विवर्जयेत्~॥}
\end{quote}

कात्यायनः

\begin{quote}
{\vy अस्पृश्याधमदासानां म्लेच्छानां पापकारिणाम्~।\\
प्रातिलोम्यप्रसूतानां निश्चयो न तु राजनि~॥

तत्प्रसिद्धानि दिव्यानि \renewcommand{\thefootnote}{3}\footnote{मिता०, स्मृतिच०, वीर० read संशये for समये}समये तेषु निर्दिशेत्~।}
\end{quote}

तत्प्रसिद्धानि घटसर्पादीनि~॥ दिव्यकारिणोशक्तौ प्रतिनिधिमाह दिव्य\renewcommand{\thefootnote}{4}\footnote{B, E, M दिव्यतन्त्रे.}तत्त्वे स एव

\begin{quote}
{\vy देशकाला\renewcommand{\thefootnote}{5}\footnote{G ०कालविरोधे.}विरोधे तु यथायुक्तं प्रकल्पयेत्~।\\
अन्येन हारयेद्दिव्यं विधिरेष विपर्यये~॥}
\end{quote}

अन्येन हारयेत्प्रतिनिधिना कारयेत्~। विपर्यये दिव्यकारिणोसामर्थ्ये~। \renewcommand{\thefootnote}{6}\footnote{B, C, F, K, N omit युक्तं प्रकल्पयेत्.}युक्तं प्रकल्पयेत्~। \renewcommand{\thefootnote}{7}\footnote{A, B, D, M read विपर्यये bofore पूर्वम्.}पूर्वं पित्रादिघातो महापापं वा कृतमिति निश्चये\renewcommand{\thefootnote}{8}\footnote{B निश्चये वा for निश्चये.} कालान्तरे चार्थान्तरसंदेहेपि प्रतिनिधिद्वारैव दिव्यमाह स एव\renewcommand{\thefootnote}{9}\footnote{After स एव B,D,F add दिव्यप्रकरणे कात्यायनः.}

\newpage
\fancyhead[CE,CO]{दिव्येषु प्रतिनिधि}
% §७] दिव्येषु प्रतिनिधि ५० 

\begin{quote}
{\vy मातापितृद्विज\renewcommand{\thefootnote}{1}\footnote{स्मृतिच०, दिव्यतत्त्व read ०द्विजगुरुबालस्त्रीराजघातिनाम्.}गुरुवृद्धस्त्रीबालघातिनाम्~।\\
महापातकयुक्तानां नास्तिकानां विशेषतः~॥

लिङ्गिनां प्रम\renewcommand{\thefootnote}{2}\footnote{स्मृतिच०reads प्रशठानां for प्रमदानाम्}दानां च मन्त्रयोगक्रियाविदाम्~।\\
वर्णसंकरजातानां पापाभ्यासप्रवर्तिनाम्~॥

एतेष्वेवाभियोगेषु निन्द्येष्वेव\renewcommand{\thefootnote}{3}\footnote{वीर० reads निन्द्येष्वेवं प्रवर्तितः,}प्रयत्नतः~।\\
दिव्यं प्रकल्पयेन्नैव राजा धर्मपरायणः~॥

एतैरेव \renewcommand{\thefootnote}{4}\footnote{दिव्यतत्त्व reads प्रयुक्तानां for नियुक्तानाम्.}नियुक्तानां साधूनां दिव्यमर्हति~।\\
न\renewcommand{\thefootnote}{5}\footnote{B, D,F न संनिधानसाधवः for न सन्ति साधवः.} सन्ति साधवो यत्र तत्र शोध्याः स्वकैर्नरैः~॥}
\end{quote}

\noindent
स्वकैराप्तैः~॥

अथ कालः~। पितामहः

\begin{quote}
{\vy चैत्रो मार्गशिरश्चैव वैशाखश्च तथैव च~।\\
एते साधारणा मासा दिव्यानामविरोधिनः~॥

घटः सर्वर्तुकः प्रोक्तो वाति वाते विवर्जयेत्~।\\
अग्निः शिशिरहेमन्तवर्षासु परिकीर्तितः~॥

शरद्ग्रीष्मे तु सलिलं हेमन्ते शिशिरे विषम्~॥}
\end{quote}

\renewcommand{\thefootnote}{6}\footnote{C, K omit विषे before हेमन्त०}विषे हेमन्तशिशिरयोरेवोपादानं कालान्तरस्याप्युपलक्षणं \textendash\ वर्षे चतुर्यवा मात्रा ( $=$ नारद ४. ३२४ ) \textendash\ इत्याद्यग्रे वक्ष्यमाणत्वात्~॥ नारदः

\begin{quote}
{\vy कोश\renewcommand{\thefootnote}{7}\footnote{C.,B, D, F,G कोशं तु सर्वदा देयं ( B ,D देयः ) for कोशस्तु सर्वदा देयः.}स्तु सर्वदा देयस्तुला स्यात्सार्वकालिकी~॥}
\end{quote}

\newpage
\fancyhead[CE,CO]{दिव्येषु कालनियमः}
% ५१ दिव्येषु कालनियमः [ §७ 

पितामहः 

\begin{center}
{\vy पूर्वाह्नेग्निपरीक्षा स्यात्पूर्वाह्ने च घटो भवेत्~।\\
मध्याह्ने तु जलं देयं धर्मतत्त्वमभीप्स\renewcommand{\thefootnote}{1}\footnote{A, E, M अभीप्सतां ०८ अभीप्सता}ता~॥

दिवसस्य तु पूर्वाह्ने कोशशुद्धिर्विधीयते~।\\
रात्रौ तु पश्चिमे यामे विषं देयं सुशीतलम्~॥} इति~।
\end{center}

\noindent
एतानि\renewcommand{\thefootnote}{2}\footnote{B,C,D,K एतानि च दिव्यानि for एतानि दिव्यानि.} दिव्यान्यादित्यवारे कार्याणीति शिष्टाः~॥ 

अथ देशः~। पितामहः

\begin{quote}
{\vy प्राङ्मुखो निश्चलः कार्यः शुचौ देशे घटः सदा~।\\
इन्द्रस्थाने सभायां वा राजद्वारे चतुष्पथे~॥}
\end{quote}

नारदः ( ४. २६५ )

\begin{quote}
{\vy सभाराजकुलद्वारे देवायतनचत्वरे~॥} इति~।
\end{quote}

कात्यायनः

\begin{quote}
{\vy इन्द्रस्थानेभिशस्तानां महापातकिनां नृणाम्~।\\
नृपद्रोहप्रवृत्तानां राजद्वारे प्रयोजयेत्~॥

प्रति\renewcommand{\thefootnote}{3}\footnote{D, K. प्रतिलोम्य० for प्रतिलोम०}लोमप्रसूतानां दिव्यं देयं चतुष्पथे~।\\
अतोन्येषु तु कार्येषु सभामध्ये विदुर्बुधाः~॥} इति~।
\end{quote}

नारदः

\begin{quote}
{\vy अदेशकालदत्तानि बहि\renewcommand{\thefootnote}{4}\footnote{अपरार्क, स्मृतिच० and वीर० read बहिर्वासकृतानि and वीर० reads व्यभिचारे सदर्थेषु.}र्वादिकृतानि च~।\\
व्यभिचारं सदार्थेषु कुर्वन्तीह न संशयः~॥} इति~।
\end{quote}

\newpage
\fancyhead[CE,CO]{सर्वदिव्यसाधारणो विधिः}
% §७] सर्वदिव्यसाधारणो विधिः ५२ 

अथ सर्वदिव्यसाधारणो विधिः~। पितामहः 

\begin{quote}
{\vy तत आवाहयेद्देवान्विधिना\renewcommand{\thefootnote}{1}\footnote{C,D,F विधिना तेन for विधिनानेन.}नेन धर्मवित्~।\\
प्राङ्मुखः प्राञ्जलिर्भूत्वा प्राङ्विवाकस्ततो वदेत्~॥

एह्येहि भगवन्धर्म अस्मिन्दिव्ये समाविश~।\\
सहितो लोकपालैश्च वस्वादित्यमरुद्गणैः~॥

आवाह्य तु \renewcommand{\thefootnote}{2}\footnote{C धटं धर्म for धटे धर्मम्.}घटे धर्मं पश्चादङ्गानि विन्यसेत्~॥}
\end{quote}

स एव

\begin{quote}
{\vy इन्द्रं पूर्वे तुं \renewcommand{\thefootnote}{3}\footnote{पूर्वं तु for पूर्वे तु.}संस्थाप्य प्रेतेशं दक्षिणे तथा~।\\
वरुणं पश्चिमे भागे कुबेरं चोत्तरे तथा~॥

अग्र्यादिलोकपालांश्च कोणभागेषु विन्यसेत्~।\\
इन्द्रः पीतो यमः श्यामो वरुणः स्फटिकप्रभः~॥

कुबेरस्तु सुवर्णाभः अग्निश्चा\renewcommand{\thefootnote}{4}\footnote{B omits अग्निश्चापि सुवर्णभः.}पि सुवर्णभः~।\\
तथैव निर्ऋतिः श्यामो वायुर्धूम्रः प्रशस्यते~॥

ईशानस्तु भवेद्रक्त\renewcommand{\thefootnote}{5}\footnote{F भवेद्रक्तं for भवेद्रक्तः.} एवं ध्यायेत्क्रमादिमान्~।\\
इन्द्रस्य दक्षिणे पार्श्वे वसूनावाहयेद्बुधः~॥

धरो ध्रुवस्तथा सोम आपश्चैवानिलो नलः~।\\
प्रत्यूषश्च प्रभासश्च वसवोष्टौ प्रकीर्तिताः~॥

देवेशेशानयोर्मध्य आदित्यानां तथा\renewcommand{\thefootnote}{6}\footnote{मिता० reads तथा गणम् for तथायनम्.}यनम्~।\\
धातार्यमा च मित्रश्च\renewcommand{\thefootnote}{7}\footnote{B, D,F वरुणोंशौ for वरुणोंशो; मिता० and दिव्यतत्त्व read वरुणोंशुः and परा. मा. reads वरुणेशौ.}वरुणोंशो भगस्तथा~॥}
\end{quote}

\newpage
\fancyhead[CE,CO]{सर्वदिव्यसाधारणो विधिः}
%५३ सर्वदिव्यसाधारणो विधिः [§७ 

\begin{quote}
{\vy इन्द्रो विवस्वान्पूषा च पर्जन्यो दशम\renewcommand{\thefootnote}{1}\footnote{B, D, F दशमः स्मृतः for दशमस्तथा}स्तथा~।\\
ततस्त्वष्टा ततो विष्णुरजघन्यो जघन्यजः\renewcommand{\thefootnote}{2}\footnote{H ०जघन्यतः for ०जघन्यजः.}~॥

इत्येते द्वादशादित्या नामभिः\renewcommand{\thefootnote}{3}\footnote{दिव्यतत्त्व reads मनुना for नामभिः} परिकीर्तिताः~।\\
अग्नेः पश्चिमभागे तु रुद्राणामयनं विदुः~॥

वीरभद्रश्च शंभुश्च गिरिशश्च महायशाः\renewcommand{\thefootnote}{4}\footnote{C omits महायशाः \ldots भवश्च.}~।\\
अजैकपादहिर्बुध्नयः\renewcommand{\thefootnote}{5}\footnote{A अहिर्बुध्नः.} पिनाकी चापराजितः~॥

भुवनाधीश्वरश्चैव कपाली च विशांपतिः~।\\
स्थाणुर्भवश्च\renewcommand{\thefootnote}{6}\footnote{G.भगश्च for भवश्चः परा मा. for भगश्च.} भगवान् रुद्रास्त्वेकादश स्मृताः~॥

प्रेतेशरक्षोमध्ये च मातृस्थानं प्रकल्पयेत्~।\\
ब्राह्मी माहेश्वरी चैव कौमारी वैष्णवी तथा~॥

वाराही चैव माहेन्द्री चामुण्डा गणसंयुता~।\\
निर्ऋतेरुत्तरे भागे गणेशायतनं विदुः~॥

वरुणस्योत्तरे भागे मरुतां स्थानमुच्यते~।\\
गगन\renewcommand{\thefootnote}{7}\footnote{मिता० reads गगनः स्पर्शनः , वीर० reads पवनः स्पर्शनः and दिव्यतत्त्व reads श्वसनः.}स्पर्शनो वायुरनिलो मारुत\renewcommand{\thefootnote}{8}\footnote{B, C, D, F, H, K मरुतः for मारुतः.}स्तथा~॥

प्राणः प्राणेशजीवौ च मारुताः\renewcommand{\thefootnote}{9}\footnote{G मरुतोष्टौ. स्मृतिच०, मिता०, परा मा, दिव्यतत्त्व, वीर० read मरुतोष्टौ.} सप्त कीर्तिताः~।\\
धटस्योत्तरभागे तु दुर्गामावाहयेद्बुधः~॥

एतासां देवतानां\renewcommand{\thefootnote}{10}\footnote{B,C,D,F,H,K देवतानां तु for देवतानां च.} च स्वनाम्ना पूजनं विदुः~।}
\end{quote}

\newpage
\fancyhead[CE,CO]{सर्वदिव्यसाधारणो विधिः}
% §७] सर्वदिव्यसाधारणो विधिः ५४ 

\begin{quote}
{\vy भूषाव\renewcommand{\thefootnote}{1}\footnote{D,F भूमावसानं for भूषावसानम्.}सानं धर्माय दत्त्वा \renewcommand{\thefootnote}{2}\footnote{C,K चार्घादिकं ; H चार्थादिकम्.}चार्घ्यादिकं क्रमात्~।\\
\renewcommand{\thefootnote}{3}\footnote{B अर्घादि for अर्ध्यादि.}अर्घ्यादि पश्चादङ्गानां भूषान्तमु\renewcommand{\thefootnote}{4}\footnote{B, D,F भूषान्तरमुप० for भूषान्तमुप०.}पकल्पयेत्~।\\
गन्धादिकां निवेद्यन्तां\renewcommand{\thefootnote}{5}\footnote{मिता० and परा. मा. read नैवेद्यान्ताम्.} परिचर्या प्रकल्पयेत्~॥

चतुर्दिक्षु तथा होमः कर्तव्यो वेदपारगैः~।\\
आज्येन हविषा चैव समिद्भिर्होमसाधनैः~॥

सावित्र्या प्रणवेनाथ स्वाहान्तेनैव होमयेत्~॥}
\end{quote}

हविश्चरुः~। आज्यहविःसमिधां\renewcommand{\thefootnote}{6}\footnote{C ०समिधो for समिधाम्.} संप्रतिपन्नदेवताकत्वान्मिलितानामेव होमः सान्नाय्ययोरिवेति दिव्यतत्त्वे गौडमीमांसकाः\renewcommand{\thefootnote}{7}\footnote{B, C, F, H, K ०मीमांसकः.}~। तन्न~। आज्ये स्रुवेणावद्यतीति स्रुवस्य\renewcommand{\thefootnote}{8}\footnote{G स्रुवस्थचरौ for स्रुवस्य चरौ.}, चरौ सकृदुपस्तृणाति मध्यात्पूर्वार्धाच्च द्विर्हविषोवद्यति अभिघारयति हविरवत्तं चैषोवदान\renewcommand{\thefootnote}{9}\footnote{B, D, F अवदानं धर्मः for अवदानधर्मः.}धर्म इत्याश्वलायनादिसूत्रात्स्रुचः\renewcommand{\thefootnote}{10}\footnote{A, C, G, H, M read स्रुवः for स्रुचः.}, समित्सु सामर्थ्यात् हस्तस्येति भिन्नसाधनकत्वात्तन्त्रतानुपपत्तेः~। सान्नाय्यहोमयोस्तु जुह्वा\renewcommand{\thefootnote}{11}\footnote{D, F जुह्वान एव for जुह्वा एव.} एव साधनत्वाद्युक्ता तन्त्रतेति~। स एव~। 

\begin{quote}
{\vy यमर्थमभियुक्तः स्याल्लिखित्वा तं तु पत्रके~।\\
मन्त्रेणानेन \renewcommand{\thefootnote}{12}\footnote{दिव्यतत्त्व reads सहितं कुर्यात्तस्य शिरोगतम्.}सहितं तत्कार्यं च शिरोगतम्~॥}
\end{quote}

\newpage
\fancyhead[CE,CO]{सर्वदिव्यसाधारणो विधिः}
% ५५ सर्वदिव्यसाधारणो विधिः [ §७ 

मन्त्रश्च\renewcommand{\thefootnote}{1}\footnote{D, F omit च after मन्त्रः.} ( महाभारते आदिपर्वणि ७४. ३ ० ) 

\begin{quote}
{\vy आदित्यचन्द्रावनिलोनलश्च द्यौर्भूमिरापो हृदयं यमश्च~।\\
अहश्च रात्रिश्च उभे च सन्ध्ये धर्मश्च जानाति नरस्य वृत्तम्~॥} इति~।
\end{quote}

नारदः

\begin{quote}
{\vy प्राङ्विवाकस्ततो विप्रो वेदवेदाङ्गपारगः~।\\
श्रुतवृत्तोपसंपन्नः शान्तचित्तो विमत्सरः~॥

सत्यसंधः शुचिर्दक्षः सर्वप्राणिहिते रतः~।\\
उपोषितश्चार्द्रवासाः\renewcommand{\thefootnote}{2}\footnote{मिता० and दिव्यतत्त्व read शुद्धवासाः for आर्द्रवासाः and कुर्यात् for कृत्वा.} कृतदन्तानुधावनः~॥

सर्वासां देवतानां च पूजां कृत्वा यथाविधि~॥} इति~।
\end{quote}

याज्ञवल्क्यः ( २. ९७)

\begin{quote}
{\vy सचैलं स्नातमाहूय सूर्योदय उपोषितम्\renewcommand{\thefootnote}{3}\footnote{D उपोषितः for उपोषितम् ; F उपोषिताः.}~।\\
कारयेत्सर्वदिव्यानि नृपब्राह्मणसंनिधौ~॥}
\end{quote}

पितामहोपि 

\begin{quote}
{\vy एकरात्रोपोषिताय त्रिरा\renewcommand{\thefootnote}{4}\footnote{B omits त्रि\ldots ताय. मिता० reads {\qt  त्रिरात्रोपोषिताय स्युरेकरात्रोषिताय वा}.}त्रोपोषिताय वा~।\\
नित्यं देयानि दिव्यानि शुचये चार्द्रवाससे~॥} इति~।
\end{quote}

स एव

\begin{quote}
{\vy सद्भिः परिवृतो राजा एतां \renewcommand{\thefootnote}{5}\footnote{B , C, D, F, G, K शुद्धिमेतां for एतां शुद्धिम् .}शुद्धिं प्रपूजयेत्~।\\
ऋत्विक्पुरोहिताचार्यान् दक्षिणाभिस्तु तोषयेत्~॥

एवं कारयिता राजा भुक्त्वा भोगान्मनोरमान्~।\\
महतीं कीर्तिमाप्नोति ब्रह्मभूयाय कल्पते~॥}
\end{quote}

\newpage
\fancyhead[CE,CO]{दिव्येषु धटविधिः}
% §७ ] दिव्येषु धटविधिः ५६ 

\begin{center}
\textbf{\Large ॥~अथ धटविधिः~॥}
\end{center}

\noindent
पितामहः 

\begin{quote}
{\vy विशालामुच्छ्रितां शुभ्रां धटशालां तु कारयेत्~।\\
यत्र\renewcommand{\thefootnote}{1}\footnote{D,F , K यत्रस्था नोपहन्यन्ते ( K०हन्येत ) for यत्रस्थो नोपहन्येत.}स्थो नोपहन्येत श्वमिश्चाण्डालवायसैः~॥

कपाटबीजसंयुक्तां परिचारकरक्षिताम्~।\\
पानीयादिसमायुक्तामशून्यां कारयेन्नृपः~॥}
\end{quote}

\noindent
नारदः ( ४. २६४ ) 

खादिरं कारयेत्तत्र निव्रणं \renewcommand{\thefootnote}{2}\footnote{नारद and वीर० read शुष्कवर्जितम्}शुक्लवर्जितम्~॥

शुक्लखादिरवर्जितमित्यर्थः~॥

\begin{quote}
{\vy शांशपं तदभावे वा \renewcommand{\thefootnote}{3}\footnote{B, D शालां for शालं. B, C,D,F,G,K. वा for च after शालं.}शालं च कोटरैर्विना~।\\
अञ्जनं तिन्दु\renewcommand{\thefootnote}{4}\footnote{B, D, F तिदुका० for तिन्दुकी०; F सारतिनिशं for ०सारं तिनिशम्.}कीसारं तिनिशं रक्तचन्दनम्~॥}
\end{quote}

\noindent
माधवस्तु\textendash\ अर्जुनस्तिलकोशोक\renewcommand{\thefootnote}{5}\footnote{B, D, F अशोकतिनिशौ for अशोकस्तिनिशः. परा. मा. reads तिमियो for तिनिशो.}स्तिनिशो रक्तचन्दनः\renewcommand{\thefootnote}{6}\footnote{C ०चन्दनमिति for०चन्दनः इति.} \textendash\ इति पपाठ~॥

एवंविधानि काष्ठानि धटार्थं परिकल्पयेत्~॥ (नारदः ४. २६५) 

एवंविधानीत्यादीन्यन्यान्यपि यज्ञियान्यौदुम्बरादीनीति \renewcommand{\thefootnote}{7}\footnote{D, F omit इति before मदनः.}मदनः~।

\noindent
अत एव पितामहः

\begin{quote}
{\vy छित्त्वा तु \renewcommand{\thefootnote}{8}\footnote{परा. मा. reads याज्ञिकं वृक्षं हेतुवन्मन्त्र०; वीर० reads यज्ञियं काष्ठं; स्मृतिच० has विधिवत् for यूपवत्.}यज्ञियं वृक्षं यूपवन्मन्त्रपूर्वकम्~।\\
प्रणम्य लोकपालेभ्यस्तुला कार्या मनीषिभिः~॥}
\end{quote}

\newpage 
% ५७ दिव्येपु धटविधिः [ §७

\begin{quote}
{\vy मन्त्रः सौम्यो वानस्परत्य\renewcommand{\thefootnote}{1}\footnote{D,F वानस्पत्यश्चन्दने for वानस्पत्यश्छेदने ; G वानस्पत्यक्वेदने\textendash }श्छेदने जप्य एव च~।}
\end{quote}

यूपवत्\textendash\ ओषधे त्रायस्यैनम् (वाज. सं. ४. १ ) इत्यादिमन्त्रपूर्वकम्~। सौम्यवानस्पत्ययोर्जप\renewcommand{\thefootnote}{2}\footnote{A, M जाप्यमन्त्रत्वेन for जपमन्त्रत्वेन.}मन्त्रत्वेनादृष्टार्थत्वात्समुच्चयः~। सौम्याः प्रसिद्धाः~। वानस्पत्यो\textendash\ वनस्पते शतवल्शो विरोह\renewcommand{\thefootnote}{3}\footnote{A, B, F, M वरोह for विरोह.} (ऋ. सं. ३. ८. ११ )~। अस्य यूपवदित्यतिदेशा\renewcommand{\thefootnote}{4}\footnote{B, D, F अतिदेशादेव for अतिदेशात् , A,H,M सिद्धस्य for सिद्धस्यैव.}त्सिद्धस्यैवानुवादः~॥ पितामहः

\begin{quote}
{\vy चतु\renewcommand{\thefootnote}{5}\footnote{B omits चतुर्हस्ता\textendash\ रपीति पितामहः.}र्हस्ता तुला कार्या \renewcommand{\thefootnote}{6}\footnote{दिव्यतत्त्व reads {\qt पादौ चोपरि तत्समौ}.}पादौ कार्यौ तथाविधौ~।\\
अन्तरं तु तयोर्हस्तौ भवेदध्य\renewcommand{\thefootnote}{7}\footnote{मिता०,परा०मा,स्मॄतिच०read भवेदध्यर्थमेव वा; अपरार्क reads {\qt अभ्यर्ध एव वा}}र्धमेव च~॥}
\end{quote}

व्यासः

\begin{quote}
{\vy हस्तद्वयं निखेयं तु\renewcommand{\thefootnote}{8}\footnote{स्मृतिच०, वीर०, परा. मा.read {\qt तु प्रोक्तं मुण्डकयोर्द्वयोः}.} पादयोरुभयोरपि~॥} इति
\end{quote}

पितामहः

\begin{quote}
{\vy चतुरस्रा तुला कार्या दृढा ऋज्वी तथैव च~।\\
कटकानि च देयानि त्रिषु स्थानेषु यत्नतः~॥}
\end{quote}

स एव

\begin{quote}
{\vy शिक्यद्वयं समास\renewcommand{\thefootnote}{9}\footnote{दिव्यतत्त्व readsसमासाद्य for समासज्य.}ज्य पार्श्वयोरुभयोरपि~।\\
प्राग\renewcommand{\thefootnote}{10}\footnote{F omits प्रागग्रान्\ldots भयोरपि. \\
\indent \textbf{८ [व्यवहारमयूख]}}ग्रान्कल्पयेद्दर्भांश्छिक्ययोरुभयोरपि~॥

पश्चिमे तोलयेत्कर्तृनन्यस्मिन्मृत्तिकां शुभाम्~।\\
इष्टकाभस्मपाषाणकपालास्थिविवर्जिताम्~॥}
\end{quote}

\newpage
% §७ ] दिव्येषु धटविधिः ५८ 

नारदः (४.२७१\textendash\ २७२ ) 

\begin{quote}
{\vy शिक्यद्वयं समासज्य धकटर्कटयोर्दृढम्~।\\
एकत्र शिक्ये पुरुषमन्यत्र तुलयेच्छिला\renewcommand{\thefootnote}{1}\footnote{F शिवां for शिलाम्.}म्~॥

धारयेदुत्तरे पार्श्वे पुरुषं दक्षिणे शिलाम्~।\\
पिटकं\renewcommand{\thefootnote}{2}\footnote{A, H M पीठकं for पिटकम्.} पूरयेत्तस्मिन्निष्ट\renewcommand{\thefootnote}{3}\footnote{स्मृतिच०reads इष्टकाग्रावपांसुभिः; नारद reads इष्टकालोष्टपंसुभिः.}कापांसुलोष्टकैः~॥}
\end{quote}

परीक्षाप्रकारमाह स एव 

\begin{quote}
{\vy कार्यः परीक्षकैर्नित्यमवलम्बसमो धटः~।\\
उदकं च प्रदातव्यं धटस्योपरि पण्डितैः~॥

यस्मिन्न प्लवते तोयं स विज्ञेयः समो धटः~॥}
\end{quote}

साम्या\renewcommand{\thefootnote}{4}\footnote{C,G, अवालम्बावाह ; K सामर्थ्यमवलम्बा०.}र्थमवलम्बावाह पितामहः

\begin{quote}
{\vy तोरणे च तथा कार्ये पार्श्वयोरुभयोरपि~।\\
धटादुच्चतरे स्यातां नित्यं दशभिरङ्गुलैः~॥

अवलम्बौ च कर्तव्यौ तोरणाभ्यामधोमुखौ~।\\
मृन्मयौ सूत्रसम्बद्धौ \renewcommand{\thefootnote}{5}\footnote{धटामस्तक० for धटमस्तक०.}धटमस्तकचुम्बिनौ~॥}
\end{quote}

पितामहः

\begin{quote}
{\vy तोलयित्वा नरं पूर्वं तस्मा\renewcommand{\thefootnote}{6}\footnote{अपरार्क, स्मृतिच०, परा मा वीर० read पश्चात्तम् for तस्मात्तम्.}त्तमवतार्य\renewcommand{\thefootnote}{7}\footnote{F अवतीर्य for अवतार्य.} तु~।\\
धटं तु कारयेन्नित्यं पताकाध्वजशोभितम्~॥

तत आवाहयेद्देवान् विधिनानेन मन्त्रवित्~।\\
वादित्रतूर्यनिर्घोषैर्गन्धमाल्यानुलेपनैः~॥}
\end{quote}

\newpage 
% ५९ दिव्येषु धटविधिः [§ ७

नारदः 

\begin{quote}
{\vy रक्तैर्गन्धैश्च माल्यैश्च दध्यपूपाक्षतादिभिः~।\\
अर्चयेत्तु धटं पूर्वं ततः शिष्टांस्तु पूजयेत्~॥}
\end{quote}

याज्ञवल्क्यः (२. १००\textendash\ १०२ )

\begin{quote}
{\vy तुलाधारणविद्वद्भिरभियुक्तस्तुलाश्रितः~।\\
प्रतिमानसमीभूतो रेखां कृत्वावतारितः~॥

त्वं तुले सत्यधामासि पुरा देवैर्विनिर्मिता~।\\
तत्सत्यं वद कल्याणि संशयान्मां विमोचय\renewcommand{\thefootnote}{1}\footnote{G.विमोचयेति for विमोचय.}~॥

यद्यस्मि पापकृन्मातस्ततो मां त्वमधो नय~।
शुद्धश्चेद्गमयोर्ध्वं \renewcommand{\thefootnote}{2}\footnote{B, D,F ०र्ध्वं तु for ०र्ध्वं माम्.}मां तुलामित्यभिमन्त्रयेत्~॥}
\end{quote}

नारदः (४. २७६ )

\begin{quote}
{\vy समयैः परिगृह्याथ\renewcommand{\thefootnote}{3}\footnote{B, D,F परिमृज्य for परिगृह्य.} पुनरारोपयेन्नरम्~।\\
निर्वाते वृष्टिरहिते शिरस्यारोप्य पत्रकम्~॥}
\end{quote}

\renewcommand{\thefootnote}{4}\footnote{B, D, F समयैः for शपथैः.}समयैः परिगृह्य शपथैर्नियम्य~। तानाह विष्णुः ( विष्णुध. सू. १५ १०. ९. )

\begin{quote}
{\vy ब्रह्मघ्नां ये स्मृता लोका ये लोकाः कूटसाक्षिणः~।\\
तुलाधारस्य ते लोकास्तुलां धारयतो मृषा~॥} इति~।
\end{quote}

पुनरारोहणकालेभिम\renewcommand{\thefootnote}{5}\footnote{D, F निमन्त्रणं for अभिमन्त्रणम्,}न्त्रणमाह नारदः ($=$४. २७८\textendash\ २७९ ; विष्णुध. सू. १०\textendash\ १०\textendash\ ११)

\newpage 
% §७ ] दिव्येषु धटविधिः ६०

\begin{quote}
{\vy त्वं वेत्सि सर्वभूतानां पापानि सुकृतानि च~।\\
त्वमेव देव जानीषे न विदुर्यानि मानवाः~॥

व्यवहाराभिशस्तोयं मानवस्तोल्यते त्वयि~।\\
तदेनं संशयारूढं धर्मतस्त्रातुमर्हसि~॥

देवासुरमनुष्याणां सत्यैस्त्वमतिरि\renewcommand{\thefootnote}{1}\footnote{D,F अतिरिष्यते; H अतिरिच्यते.}च्यसे~।\\
सत्यसंधोसि भगवञ् शुभाशुभविभावने~॥

आदित्यचन्द्रावनिलोनलश्च द्यौर्भूमिरापो हृदद्यं यम\renewcommand{\thefootnote}{2}\footnote{मनश्च for यमश्चः ; धर्मश्च; F नमश्च.}श्च~।\\
अहश्च रात्रिश्च उभे च सन्ध्ये धर्मश्च जानाति नरस्य वृत्तम्~॥}
\end{quote}

इति~।

पितामहः

\begin{quote}
{\vy ज्योतिर्विद्ब्रा\renewcommand{\thefootnote}{3}\footnote{ब्राह्मणश्रेष्ठः f०r ब्राह्मणः श्रेष्ठः}ह्मणः श्रेष्ठः कुर्यात्कालपरीक्षणम्~।\\
विनाड्यः पञ्च विज्ञेयाः परीक्षाकालकोविदैः~॥

साक्षिणो ब्राह्मणश्रेष्ठा यथाद्दष्टार्थवादिनः~।\\
ज्ञानिनः शुचयोलुब्धा नियोक्तव्या नृपेण तु~॥

शंसन्ति साक्षिणः \renewcommand{\thefootnote}{4}\footnote{B, D, F सर्व for सर्वे.}सर्वे शुद्ध्यशुद्धी नृपे तदा~॥}
\end{quote}

विनाड्यः पलानि~। दशगुर्वक्षरः प्राणः षट् प्राणाः स्याद्विनाडिकाइति स्मृतेः~॥ नारदः ( ४. २८३ )

\begin{quote}
{\vy तुलितो यदि वर्धेत विशुद्धः स्यान्न संशयः~।\\
समो वा हीयमानो वा न विशुद्धो भवेन्नरः~॥}
\end{quote}

वृद्धिरूर्ध्वगतिः~। \renewcommand{\thefootnote}{5}\footnote{D omits हानिरधोगतिः.}हानिरधोगतिः~। पितामहः

\begin{quote}
{\vy अल्पपापः समो ज्ञेयो बहुपापस्तु हीयते~॥}
\end{quote}

\newpage
% ६१ दिव्येषु धटविधिः [ §७

\noindent
अल्पत्वं सकृदमतिकृतत्वेन~। यत्र तु शिरः\renewcommand{\thefootnote}{1}\footnote{ B, D ,F शिष्टावचनेनैव for शिरःस्थायिवचनेनैव.}स्थायिवचनेनैव सकृदमति कृतत्वनिश्चयः पापमात्रे च विप्रतिपत्त्या \renewcommand{\thefootnote}{2}\footnote{B, D, F, G दिव्यकृते. for तद्दिव्ये कृते; C,H विप्रतिपत्त्या च दिव्ये; K पापमात्रे च प्रतिपत्त्या च दिव्ये कृते.}तद्दिव्ये कृते समता तत्र दोषाल्पत्वासंभवात्पुनः करणम्~। अत एव वृहस्पतिः

\begin{quote}
{\vy तत्समस्तु पुनस्तोल्यो \renewcommand{\thefootnote}{3}\footnote{B, C, D, F. K वर्धितो for वृद्धिगो; H वर्णितो.}वृद्धिगो विजयी भवेत्~।} इति~।
\end{quote}

पुनःकरणे कारणान्तरमाह कात्यायनः

\begin{quote}
{\vy शिक्यच्छेदे तुलाभङ्गे तथा चा\renewcommand{\thefootnote}{4}\footnote{G चाधिगुणस्य for चापि गुणस्य.}पि गुणस्य वा~।\\
\renewcommand{\thefootnote}{5}\footnote{B, F शुद्धेस्तु for शुद्धेश्च; Dशुद्धेस्य.}शुद्धेश्च संशये चैनं परीक्षेत पुनर्नरम्~॥}
\end{quote}

व्यासः

\begin{quote}
{\vy कक्षच्छेदे तुलाभङ्गे धटकर्कटयोस्तथा~।\\
रज्जुच्छेदेक्षभङ्गे वा दद्याच्छुद्धिं पुनर्नृपः~॥}
\end{quote}

इदं तु दृष्टकारणकभङ्गपरम्~। आकस्मिकभङ्गादौ त्वशुद्ध एव

\begin{quote}
{\vy कक्षच्छेदे तुलाभङ्गे धट\renewcommand{\thefootnote}{6}\footnote{H omits धटकर्कटयो\ldots तथैवाशुद्धि.}कर्कटयोस्तथा~।\\
रज्जु\renewcommand{\thefootnote}{7}\footnote{F च्छेदेक्षयो भागस्तथैवा० for च्छेदेक्षभङ्गे वा तथैवा०.}च्छेदेक्षभङ्गे वा तथैवाशुद्धिमादिशेत्~॥} इति
\end{quote}

स्मृत्यन्तरात्~। कक्षं शिक्यतलम्~। अक्षः स्तम्भोपरि स्थापितस्तुलाधारः~। आवृत्तिस्तोलनमात्रस्य न \renewcommand{\thefootnote}{8}\footnote{F न योगप्रयोगस्य for न साङ्गप्रयोगस्य.}साङ्गप्रयोगस्येति प्राच्याः~। \renewcommand{\thefootnote}{9}\footnote{A,M एवं वैगुण्या०for एवं तु वैगुण्या०.}एवं तु वैगुण्यापरिहारात्साङ्गः प्रयोग आवर्तनीय इति मदनः~॥

\newpage
\fancyhead[CE,CO]{धटदिव्यप्रयोगः}
% §७ ] धटदिव्यप्रयोगः ६२ 

अथ\renewcommand{\thefootnote}{1}\footnote{B, D, F add पितामहः before अथ प्रयोगः.} प्रयोगः~॥ कर्ता शुभेह्नि \renewcommand{\thefootnote}{2}\footnote{B omits पूर्वाह्ने}पूर्वाह्ने पूर्वोक्तान्यतमवृक्षसमीपे गत्वा\textendash\ ओषधे त्रायस्वैनम् (वाज. सं ४. १ ) इति मन्त्रेण वृक्षं छित्त्वा सोमो धेनुं गौतमः सोम\renewcommand{\thefootnote}{3}\footnote{सौमस्त्रिष्टुप् for सोमस्त्रिष्टुप्.}त्रिष्टुप् जपे विनियोगः~। सोमो धेनुं (ऋ. सं. १. ९१. २० )~। वनस्पते \renewcommand{\thefootnote}{4}\footnote{D,F गाधिनो for गाथिनो.}गाथिनो विश्वामित्रो वनस्पतिस्त्रिष्टुप् जपे विनियोगः~। वनस्पते शतवल्शो (ऋ. सं. ३. ८. ११ ) \textendash\ इति जप्त्वा इन्द्रादिलोकपालान्प्रत्येकं नत्वा चतुर्हस्तदीर्घां\renewcommand{\thefootnote}{5}\footnote{C,H,K चतुर्हस्तां दीर्घां for चतुर्हस्तदीर्घाम्; Gचतुर्हस्तां तुलां दीर्घाम्.} चतुरङ्गुलस्थूलां \renewcommand{\thefootnote}{6}\footnote{D चतुरङ्गुलां for चतुरङ्गुलस्थूलाम्; F omits चतु\ldots स्थूलाम्.}चतुरस्रां मध्येन्तयोश्चतुरङुलवृत्तां मध्य उपरिभागे तयोश्चाधोभागे च त्रिभिः कर्कटैः कट\renewcommand{\thefootnote}{7}\footnote{C कण्टकैः for कटकैः B, D omit ततः.}कैर्वा युतां तुलां कुर्यात्~। ततः सप्तहस्तां पञ्च\renewcommand{\thefootnote}{8}\footnote{D,F omit पञ्चहस्ताम्. D,F, वेदिं कुर्यात्for वेदिं च कुर्यात्}हस्तां वा चतुरङ्गुलोन्नतां वेदिं च कुर्यादिति केचित्~। ततस्तस्यां शुचिदेशान्तरे वा चतुरस्त्रं षडुस्तं षडुस्ताधिकांशकृतचूडं स्तम्भ\renewcommand{\thefootnote}{9}\footnote{B, F omit स्तम्भद्वयं\ldots सार्धहस्तम्.}द्वयं हस्तद्वयं भूमौ निखेयम्~। भूमेरुपरि हस्तचतुष्ट्यं चूडांशश्चाधिकः~। स्तम्भान्तरं तु हस्तद्वयं सार्धहस्तं वा~। त\renewcommand{\thefootnote}{10}\footnote{F मध्ये भागे for मध्येधोभागे.}च्चूडयोर्मध्येधोभागे लोहकर्कटकटकवडिशाद्याकारतुलावलम्बनयुतं काष्ठं निवेश्यम्~। तस्मिंस्तुला\renewcommand{\thefootnote}{11}\footnote{C तुलास्थोपरि for तुलास्वोपरि.} स्वोपरितनकर्कटबडिशादिनावलम्बिंनी\renewcommand{\thefootnote}{12}\footnote{B, F omit लम्बिनी.. हस्तद्वयान्तरा} प्रान्तयोश्च फलकद्वयं तिसृभिस्तिसृभी रज्जुभिर्बन्धनीयम्~। तुलायाः प्राच्यां हस्तद्वयान्तरालं दक्षिणोत्तरयोः स्तम्भद्वयं\renewcommand{\thefootnote}{13}\footnote{D,F स्तम्भद्वये.} निखाय तयोरुप

\newpage
% ६३ धटदिव्यप्रयोगः [ §७ 


\noindent
र्युत्तराङ्गं काष्ठं दद्यात्~। तदेतत्तोरणम्~। तच्च धटाद्दशाङ्गुलोच्चम्~। एतादृशं\renewcommand{\thefootnote}{1}\footnote{B, D, F एतादृशं च पश्चिमा० for एतादृशं पश्चिमा०} पश्चिमायामपि कार्यम्~। तोरणाभ्यामधो लम्बमानौ मृन्मयौ सूत्रसंबद्धौ गोलकरूपावलम्बौ तुलाप्रान्तस्पृशौ कार्यौ साम्यज्ञानाय~। फलकयोश्च प्रागग्रान् कुशा\renewcommand{\thefootnote}{2}\footnote{B, D, F. G, K कुशांस्तृणीयात् for कुशानास्तृणीयात्.}नास्तृणीयात्~। ततः कृतैकोपवासं शक्तौ महाभि\renewcommand{\thefootnote}{3}\footnote{C,G , K ०योगे वा कृतो०for०योगे कृतो०.}योगे कृतोपवासत्रयं रविवारे सूर्योदयोत्त\renewcommand{\thefootnote}{4}\footnote{G ०द्वयानन्तरं for द्वयोत्तरम्.}रं सचैलस्नातं शोध्यं कृतैकोपवासः प्राड्विवाकः \renewcommand{\thefootnote}{5}\footnote{B, D, F पश्चिमशिक्ये.}पश्चिमे शिक्य आरोप्य पूर्वशिक्ये पाषाणेष्टकामृदाद्यारोप्य समं तोलयेत्~। तत्परीक्षामुदकप्रक्षेपादिना सत्यवादिनो विप्रा हेमकाराश्च कुर्युः~। ततस्तोलनकाले येन संनिवेशेनोपविष्टं तत्संनिवेशज्ञानाय तादृशीं रेखां कृत्वा तमवतारयेत्~। ततः शोध्यो देशकालौ संकीर्त्यात्मशुद्धिज्ञापनायामुकदिव्यं करिष्य इति संकल्प्य प्राड्विवाकमेकं चतुर ऋत्विजश्च वस्त्रादिना वृणुयात्~। स्वस्तिवाचनाद्यपि कार्यमिति स्मार्तभट्टाचार्याः~। प्राड्विवाकश्च प्राज्जलिः सतूर्यनादं\renewcommand{\thefootnote}{6}\footnote{C,K०नाद ओमेत्द्येहि for ०नादमोमेह्येहि ; G सन् तूर्यनाद०.}

\begin{quote}
{\vy ओम्~। एह्येहि भगवन्धर्म अस्मिन्दिव्ये समाविश~।\\
सहितो लोकपालैश्च वस्वादित्यमरुद्गणैः~॥} इति
\end{quote}

एवं धटे धर्ममावाह्य पश्चादङ्गदेवता आवाहयेत्~॥ इन्द्रं विश्वा माधुच्छन्दस इन्द्रोनुष्टुप् इन्द्रावाहने विनियोगः~। एवं सर्वत्र विनियोगः~। इन्द्रं\renewcommand{\thefootnote}{7}\footnote{B, C, D, F, H, K add ओम् before इन्द्रं विश्वा.} विश्वा० (ऋ. सं. १. ११. १ ) इन्द्र इहागच्छेह तिष्ठेति पूर्व\renewcommand{\thefootnote}{8}\footnote{B, D, F, H omit पूर्वं इन्द्र\ldots..तिष्ठेति ( p.641. 1)} इन्द्रमावाह्य पीतं ध्यायेत्~। यमाय सोमं यमो यमोनुष्टुप्~।

\newpage
% §७] धटदिव्यप्रयोगः ६४

\noindent
यमाय\renewcommand{\thefootnote}{1}\footnote{1.C G, K add ओम् before यमाय सोमम्.} सोमं० ( ऋ. सं. १०. १४. १३) यम इहागच्छेह तिष्ठेति दक्षिणे यममावाह्य श्यामं ध्यायेत्~। त्वं नो वामदेवो वरुणस्त्रिष्टुप्~। \renewcommand{\thefootnote}{2}\footnote{C, G, K add ओम् before त्वं नो अग्ने.}त्वं नो अग्ने वरुणस्य० (ऋ. सं. ४. १. ४ ) वरुणेहागच्छेह तिष्ठेति पश्चिमे वरुणमावाह्य \renewcommand{\thefootnote}{3}\footnote{C स्फाटिकाभं 07 स्फटिकाभम्.}स्फटिकाभं ध्यायेत्~। राजाधिराजाय (तै. आ. १. ३१. ६ ) इति यजुषा कुबेर इहागच्छेह तिष्ठेति उत्तरे कुबेरमावाह्य सुवर्णवर्णं\renewcommand{\thefootnote}{4}\footnote{B, C, D, F ,G, स्वर्णाभं for सुवर्णवर्णम्.} ध्यायेत्~। अग्निं\renewcommand{\thefootnote}{5}\footnote{अग्निर्मेधा० for अग्निं मेधा०.} मेधातिथिरग्निर्गायत्री~। ओमग्निं दूतं० (ऋ. सं. १. १२. १ ) अग्ने इहागच्छेह तिष्ठेति आग्नेये अग्निमावाह्य सुवर्णवर्णं ध्यायेत्~। \renewcommand{\thefootnote}{6}\footnote{D, F, G., K add ओम् before मो षु णो.}मो षु णो घोरः\renewcommand{\thefootnote}{7}\footnote{G काण्वो for कण्वो.} कण्वो निर्ऋतिर्गायत्री~। मो षु णो० (ऋ. सं. १. ३८. ६) इहेत्याद्याबाह्य श्यामं ध्यायेत्~। तव वायो व्यश्वो वायुर्गायत्री~। \renewcommand{\thefootnote}{8}\footnote{तवा वायोfor तव वायो.}तव वायो० (ऋ. सं. ८. २६. २१ ) इहेति पूर्ववत् धूम्रं ध्यायेत्~। तमीशानं गौतम ईशानो जगती~। \renewcommand{\thefootnote}{9}\footnote{F, D, G, K add ओम् before \textendash\ तमीशानम्.}तमीशानं० (ऋ. सं. १. ८९. ५ ) इहेत्या\renewcommand{\thefootnote}{10}\footnote{B, F omits त्याबाह्ये तिष्ठते (1. 15).}वाह्य रक्तं ध्यायेत्~॥ इन्द्राद्दक्षिणतः ज्मया अत्र वसवो मैत्रावरूणो वसिष्ठो वसवस्विष्टुप्~। ज्मया अत्र वसवो० (ऋ. सं. ७. ३९. ३) इहागच्छतेह तिष्ठतेत्यष्टौ वसून्~।

\begin{quote}
{\vy धरो घ्रुवस्तथा सोम आपश्चैवानिलोनलः~।\\
प्रत्यूषश्च प्रभासश्च\renewcommand{\thefootnote}{11}\footnote{प्रभावश्च for प्रभासश्च; F प्रभाषश्च.} वसवोष्टौ प्रकीर्तिताः~॥}
\end{quote}

\newpage
%६५ धटदिव्यप्रयोगः [ §७ 

इन्द्रेशानयोर्मध्ये त्यान्नु सांमदो मत्स्यो द्वादशादित्या गायत्री~। त्यान्नु\renewcommand{\thefootnote}{1}\footnote{B, D, F add ओम् before त्यान्नु.} क्षत्रियान् (ऋ. सं. ८. ६७. १ ) इत्यादि द्वादशादित्यानावाह्य~।

\begin{quote}
{\vy धातार्यमा च मित्रश्च वरुणोंशो भगस्तथा~।\\
इन्द्रो विवस्वान्पूषा च पर्जन्यो दशम\renewcommand{\thefootnote}{2}\footnote{D,F, दशमः स्मृतः for दशमस्तथा.}स्तथा~॥

ततस्त्वष्टा ततो विष्णुरजघन्यो जघन्यजः~।\\
इत्येते द्वादशादित्या नामभिः परिकीर्तिताः~॥}
\end{quote}

अग्नेः पश्चिमे आ रुद्रासः श्यावाश्च एकादश रुद्रा जगती\renewcommand{\thefootnote}{3}\footnote{B, F, G add ओम् before आ रुद्रासः.} आ रुद्रासः० (ऋ.स. ५. ५७. १ ) इदेति रुद्रानावाह्य~। 

\begin{quote}
{\vy वीरभद्रश्च शंभुश्च गिरिशश्च महायशाः~।\\
अजैकपादहिर्बुध्न्यः\renewcommand{\thefootnote}{4}\footnote{A अहिर्बुध्नः for अहिर्बुध्न्यः\textendash } पिनाकी चापराजितः~॥

भुवनाधीश्वरश्चैव कपाली च विशांपति~।\\
स्थाणु\renewcommand{\thefootnote}{5}\footnote{भर्गश्च for भवश्च; C, D, F., G,भगश्च for भवश्च.}र्भवश्च भगवान् रुद्रा एकादश स्मृताः~॥}
\end{quote}

यमनिर्ऋतिमध्ये ब्रह्म\renewcommand{\thefootnote}{6}\footnote{B, D, F omit ब्रह्म जज्ञानं \ldots वाह्य.} जज्ञानं गौतमो \renewcommand{\thefootnote}{7}\footnote{H वासुदेवो for वामदेवो.}वामदेवो ब्रह्मा त्रिष्टुप् ब्रह्म जज्ञानं० (वाज. सं. १३. ३ ) इहेत्यादि ब्रह्माणमावाह्य~। १५ गौरीर्मिमायेत्यस्य दीर्घतमा उमा जगती \renewcommand{\thefootnote}{8}\footnote{C, G, H,K omit ा् ौगौरीर्मिमायेत्यस्य \ldots र्मिमाय~।.(1.15). \\ \indent \textbf{९ [व्यवहारमयूख]}}गौरीर्मिमाय० (ऋ. सं. १०. १६४.४१) मातर इहागच्छतेह तिष्ठतेति मातृराबाह्य~।

\begin{quote}
{\vy ब्राह्मी माहेश्वरी चैव कौमारी वैष्णवी तथा~।\\
वाराही च तथेन्द्राणी चामुण्डा सप्त मातरः~॥}
\end{quote}

\newpage
% §७ ] धटदिव्यप्रयोगः ६६

\noindent
निर्ऋतेरुत्तरतो गणा\renewcommand{\thefootnote}{1}\footnote{A,D, F गणानां गृत्समदो.}नां त्वा गृत्समदो गणाधिपतिर्जगती \renewcommand{\thefootnote}{2}\footnote{D, F, K add ओम् before गणानाम्.}गणानां त्वा० (ऋ. सं. २. २३. १) इहेत्यादि गणेशम्~। वरुणादुत्तरे मरुतो यस्य राहूगणो मरुतो गायत्री मरुतो\renewcommand{\thefootnote}{3}\footnote{D, C,F , K read ओम् before मरुतो.} यस्य० (ऋ. सं. १. ८६.१)~। इहेत्यादि मरुतः~।

\begin{quote}
{\vy गगनस्पर्शनो वायुरनिलो मारुतस्तथा~।\\
प्राणः प्राणेशजीवौ च मरुतः\renewcommand{\thefootnote}{4}\footnote{B, C, D, F, G,K मरुतोष्टौ for मरुतः सप्त.} सप्त कीर्तिताः~॥}
\end{quote}

\begin{sloppypar}
\noindent
धटादुत्तरे जातवेदसे कश्यपो दुर्गा त्रिष्टुप् जातवेदसे० ( ऋ. सं. १. ९९. १ ) इहेति दुर्गाम्~॥ एवमेता देवता आवाह्य पूजयेत्~। ओं\renewcommand{\thefootnote}{5}\footnote{B तत्र धर्माय; D,F तत्र ओं धर्माय; A, G, K, M omit ओं before धर्माय.} धर्मायार्घ्यं प्रकल्पयामि नमः इत्यादिप्रयोगेणार्घ्यपाद्याचमनीयमधुपर्काचमनीयस्नानवस्त्रयज्ञोपवीताचमनीयमुकुटकटकादिभूषान्तं धर्माय दत्त्वा इन्द्रादीनां प्रणवाद्यैः स्वस्वनामभिश्चतुर्थ्यन्तनमोन्तैरर्घ्या\renewcommand{\thefootnote}{6}\footnote{C अर्घादि for अर्घ्यादि; B अर्घ्याभूषान्तंfor अर्घ्यादिभूषणान्तं ;D भूषान्तं for भूषणान्तम् , चतुर्थ्यन्तैः for ०र्थ्यन्तनमोन्तैः.}दिभूषणान्तं पदार्थानुसमयेन दत्त्वा धर्माय गन्धपुष्पधूपदीपनैवेद्यानि दध्यपूपाक्षतादि\renewcommand{\thefootnote}{7}\footnote{B,C,D,K क्षतानि or ०क्षतादि} च दत्त्वा इन्द्रादीनां गन्धादीनि \renewcommand{\thefootnote}{8}\footnote{D,F, मधुपूर्ववत्.}पूर्ववद्दद्यात्~॥ \renewcommand{\thefootnote}{9}\footnote{B, D, F omit गन्धपुष्पा\ldots कुर्यात् (1. 15).}गन्धपुष्पादीनि च धटे धर्मपूजायां च रक्तानि कार्याणि~। इन्द्रादीनां यथालाभम्~। एतदन्तं प्राड्विवाकः कुर्यात्~। ततश्चतुर्भिर्ऋत्विग्भिश्चतसृषु दिक्षु लौकिकाग्निस्थापनादिपूर्वं होमःकार्यः~। तत्र सप्रणवां 
\end{sloppypar}

\newpage
% ६७ धटदिव्यप्रयोगः [§ ७ 

\noindent
गायत्रीमुच्चार्य पुनः स्वाहा\renewcommand{\thefootnote}{1}\footnote{D, F ०कारान्तः for०कारान्तं.}कारान्तं प्रणवमुच्चार्य आज्यचरुसमिधः प्रत्येकमष्टोत्तर\renewcommand{\thefootnote}{2}\footnote{G ०शतमर्काय for ०शतं सवित्रे.}शतं सविन्रे जुहुयुः~। ततोभियुक्तमर्थं पत्रे लिखेदभियुक्तः~।

\begin{quote}
{\vy आदित्यचन्द्रावनिलोनलश्च द्यौर्भूमिरापो हृदयं यमश्च~।\\
अहश्च रात्रिश्च उभे च सन्ध्ये धर्मश्च जानाति नरस्य \renewcommand{\thefootnote}{3}\footnote{B, D, F सत्त्वमिति for वृत्तमिति.}वृत्तम्~॥} इति 
\end{quote}

\noindent
(महाभारते आदि. ७४. १६ ) मन्त्रं च लिखेत्~। तत्र पत्रं संशोध्याभियुक्तशिरोगतं कुर्यात्~। इदं च धर्मावाहनादि शिरःपन्नदानान्तं सर्वदिव्यसाधारणम्~। ततः प्राड्विवाको धटमभिमन्त्र्येत्

\begin{quote}
{\vy त्वं धट ब्रह्मणा सृष्टः परीक्षार्थं दुरात्मनाम्~।\\
धकाराद्धर्म\renewcommand{\thefootnote}{4}\footnote{मिता०, अपरार्क, स्मृतिच०, वैजयन्ती read धर्ममूर्तिः.}भूतस्त्वं टकारात्कुटिलं नरम्~॥

धृतो भाव\renewcommand{\thefootnote}{5}\footnote{C.भावयते for भावयसे. दिव्यतत्व reads धारयसे.}यसे यस्माद्धटस्तेनाभिधीयसे~।\\
त्वं वेत्सि सर्वभूतानां पापानि सुकृतानि च~॥

त्वमेव सर्वं जानीषे न विदुर्यानि मानवाः~।\\
व्यवहाराभिशस्तोयं मानुषः शुद्धिमिच्छति~॥

तदेनं\renewcommand{\thefootnote}{6}\footnote{B, D, F तदेवं.} संशयादस्माद्धर्मतस्त्रातु\renewcommand{\thefootnote}{7}\footnote{D,F अर्हतीति for अर्हसीति.}मर्हसि~॥} इति
\end{quote}

\noindent
मन्त्रेण~। ततोभियुक्तः

\begin{quote}
{\vy त्वं तुले सत्यधामासि पुरा देवैर्विनिर्मिता\renewcommand{\thefootnote}{8}\footnote{D ,K देवविनिर्मिता for देवैर्विनिर्मिता.}~।\\
तत्सत्यं वद कल्याणि संशयान्मां विमोचय~॥}
\end{quote}
 
\newpage
%§७] धटदिव्यप्रयोगः ६८ 

\begin{quote}
{\vy यद्यहं पापकृन्मातस्ततो मां त्वमधो नय~।\\
शुद्धश्चैद्गमयोर्ध्व \renewcommand{\thefootnote}{1}\footnote{ B, C, D, F, G,K मामिति तुलामधिमन्त्र्येयेत.}मां तुलामित्यभिमन्त्रयेत्~॥} (याज्ञवल्क्य. २. १० १\textendash\ १०२ )
\end{quote}

ततः प्राड्विवाकः शिरोगतपत्रकमभियुक्तं यथास्थानं\renewcommand{\thefootnote}{2}\footnote{F omits यथास्थानम्} यथासंनिवेशं च धटमारोप्य पञ्चपलकं तथैव तत्र स्थापयेत्\renewcommand{\thefootnote}{3}\footnote{B, D, F read after स्थापयेत् \textendash\ पलं च षष्टिगुर्वक्षरोच्चारमितःकालः~। तद्यथात्माकांते (मा त्वं कान्ते?) पक्षस्यान्ते पर्याकाशे स्वाप्सीः~। कान्तं वक्त्रं वृत्तं पूर्ण चन्द्रं मत्वा रौत्रौ ( रात्रौ? ) चेत् भुत्थाम (क्षुत्क्षामः? ) प्राटन्नेटः (?) ( प्राटंश्चेतः ? ) खेटो राहुः प्राद्यात्क्रूरस्तस्माद् ध्वान्ते हर्म्यस्यान्ते शय्यैकान्त (न्ते?) कर्तव्या.}~। तस्मिंश्च काले शुद्ध्यशुद्धी परीक्ष्य शुचिभिर्ब्राह्मणै राज्ञे सभ्येभ्यश्च निवेदनीये\renewcommand{\thefootnote}{4}\footnote{After निवेदनीये B, D, F read तल्लक्षणं च पितामहः \textendash\ साक्षिणो ब्राह्मणाः श्रेष्ठाः यथादृष्टार्धवादिनः~। ज्ञानिनः शुचयोलुब्धा नियोक्तव्या नृपेण तु~॥ शंसन्तु साक्षिणः सर्वं शुद्ध्यशुद्धी नृपे तदा~। इति~। शुद्ध्यशुद्धिनिर्णयहेतुश्च~। तुलितो यदि वर्धेत स शुद्धः स्यान्न संशयः~। समो वा हीयमानो वा न विशुद्धो भवेन्नरः~॥ वर्धत ऊर्ध्वः स्यात्~। हीयमानः अधोगामी इत्यर्थः~। तेन मन्त्रलिङ्गाविरोधः~। तत्र समत्वे सकृदमतिपूर्वतेता (?) स (अ?) दोषता~। हीनत्वे सकृन्मति पूर्वत्वे न बहुदोषतेति विवेकः~। यदा त्वसति दृष्टकारणे शिक्यतररज्जुच्छेदौ तुलाकर्कटपादस्तम्भोत्तरङ्गादीनां भङ्गश्च तदाप्यशुद्धता~। यदा तु दृष्टकारणकौ एषां भङ्गच्छेदौ तदा पुनरारोपयेत्~। ततस्ततोवतीर्य \&c. This Passage is very corrupt in the three Mss. I have not noted all these corrupt readings.} इति~। ततस्ततो वतीर्य प्राड्विवाकब्रह्मर्त्विजो यथाविभवं \renewcommand{\thefootnote}{5}\footnote{D,K दक्षिणादिभिः for दक्षिणाभिः.}दक्षिणाभिस्तोषयेत्~। ततो देवताः उत्तिष्ठं ब्रह्मणस्पते (. सं. १. ४०. १ ) यान्तु देवगणाः\renewcommand{\thefootnote}{6}\footnote{B, D, F add सर्वे after देवगणाः.} इति विसृज्य सर्व प्राड्विवाकाय समर्पयेत्~।

\newpage
\fancyhead[CE,CO]{दिव्येषु अग्निदिव्यविधिः}
% ६९ दिव्येषु अग्निदिव्यविधिः [§७ 

\begin{center}
\textbf{\Large ॥~अथाग्निविधिः~॥}
\end{center}

पितामहः

\begin{quote}
{\vy अग्नेर्विधिं प्रवक्ष्यामि यथावच्छास्त्रचोदितम्~।\\
कारयेन्मण्डलान्यष्टौ पुरस्तान्नवमं तथा~॥ 

आग्नेयं मण्डलं चाद्यं\renewcommand{\thefootnote}{1}\footnote{D वार्धं for चाद्यम्.} द्वितीयं वारुणं तथा~।
तृतीयं वायुदैव\renewcommand{\thefootnote}{2}\footnote{B, D, F ०देवत्यं for दैवत्यं or दैवतम् \textendash\ everywhere.}त्यं चतुर्थ यमदैवतम्~॥

पञ्चमं त्विन्द्रदैवत्यं षष्ठं कौवेरमुच्यते~।\\
सप्तमं सोमदैवत्यं सावित्रं त्वष्टमं तथा~॥

नवमं सर्वदैवत्यमिति वेदविदो विदुः~।}
\end{quote}

मदनस्तु\textendash\ अष्टमं सर्वदैवतम्~। पुरस्ताग्वमं यत्तु तन्महत्पार्थिवं विदुः~॥ गोमयेन कृतानि स्युरद्भिः पर्युक्षितानि\renewcommand{\thefootnote}{3}\footnote{G पर्युक्षितानि वेति.} च \textendash\ इति पपाठ~॥ मण्डलानां परिमाणमाह स एव ($=$नारद ४. २८५\textendash\ २ ८६) 

\begin{quote}
{\vy द्वात्रिंशदङ्गुलं प्राहुर्मण्डलान्मण्डलान्तरम्~॥

अष्टभिर्मण्डलैरेवमङ्गुलानां शतद्वयम्~।\\
षट्पञ्चाशत्समधिकं भूमेस्तु परिकल्पना~॥}
\end{quote}

मण्डलान्मण्डलादेः~। मण्डलभुवोन्तरालभुवश्च मिलित्वा द्वात्रिंशदङ्गुलानीत्यर्थः~। तत्र षोडशाङ्गुलं मण्डलं मण्डलयोरन्तरालं च तावदेव~। षोडशा\renewcommand{\thefootnote}{4}\footnote{H omits षोडशाङ्गुलकं\ldots शोध्यपदम्.}ङ्गुलकं ज्ञेयं मण्डलं तावदन्तरम्\textendash\ इति याज्ञवल्क्योक्तेः\renewcommand{\thefootnote}{5}\footnote{B, D, F याज्ञवल्क्योक्तं for याज्ञवल्क्योक्तेः} ( २. १०६)~। यदि \renewcommand{\thefootnote}{6}\footnote{C शोध्यं पदं for शोध्यपदं}शोध्यपदं षोडशाङ्गुलाधिकं तदान्तरालं\renewcommand{\thefootnote}{7}\footnote{B, D,H तदान्तरालं (D ०ले)षोडशाङ्गुलकं ज्ञेयं न्यूनमेव ; F तदान्तराले षोडशाङ्गुलाधिकंतदान्तराले षोडशाङ्गुलकं ज्ञेयं (न्यू?)नमेव.}

\newpage
% §७ ] दिव्येषु अग्निदिव्यविधिः ७० 

\noindent
षोडशाङ्गुलन्यूनमेव भवति~। यदि शोध्यपदं षोडशाङ्गुलादल्पं तद षोडशाङ्गुलमण्डलमध्ये पदपरिमाणमवान्तरमण्डलं कार्यम्~॥ यत्तु नारदः\textendash\ एवं शतद्वयम् ( नारद ४. २८६ ) इत्यस्याग्रे चत्वारिंशत्समधिकं भूमेरङ्गुलमानतः\textendash\ इति पपाठ तदष्टमनवममण्डलान्तरालभूभागं विहाय योज्यं तस्य गमनानर्हत्वात्~। एवं चतुर्विंशतिराख्याता भूमेस्तु परिकल्पना \textendash\ इति कल्पतरूक्ते\renewcommand{\thefootnote}{1}\footnote{C,G, कल्पतरूक्तेपि पाठे;~। K कल्पनसूक्तेपि पाठे.} पाठेप्याद्यमवस्थितिमण्डलं विहायाङ्गुल\renewcommand{\thefootnote}{2}\footnote{E, G, M अङ्गुलं परिमाणं for अङ्गुलपरिमाणम्}परिमाणं योज्यम्~॥

\begin{quote}
{\vy मण्डले मण्डले देयाः कुशाः शास्त्रप्रचोदिताः~।\\
विन्यसेच्च पदं तेषु कर्ता नित्यमिति स्थितिः~॥}
\end{quote}

मिताक्षरायां मदनरत्ने च \textendash\ शान्त्यर्थं जुहुयादग्रौ घृतमष्टोत्तरं शतम्~। अयं च होमो\renewcommand{\thefootnote}{3}\footnote{D, F अर्थवहोमो (?) for होमः.}ग्नये पावकाय\renewcommand{\thefootnote}{4}\footnote{B, D, F omit अग्नये before पावकाय.} स्वाहेति मन्त्रेण कार्य इति विज्ञानेश्वरः~॥ नारदः (२. २८८\textendash\ २८९ )

\begin{quote}
{\vy जात्यैव लोहकारो यः कुशलश्चाग्निकर्मणि~।\\
दृष्टप्रयोगश्चान्यत्र तेनायोग्नौ प्रतापयेत्~॥

अग्निवर्णमयःपिण्डं सस्फुलिङ्गं सुरक्षितम्\renewcommand{\thefootnote}{5}\footnote{नारद reads सुरक्तितम् for सुरक्षितम् ; अपरार्क reads सुयन्त्रितम्; मिता० सुरक्षितम्.}~।}
\end{quote}

पितामहः

\begin{quote}
{\vy अस्रिहीनं समं कृत्वा अष्टाङ्गुलमयोमयम्\renewcommand{\thefootnote}{6}\footnote{C adds by mistake five lines after अष्टाङ्गुलमयोमयं, which refer to शरक्षेप {\qt क्षत्रियस्तद्वृत्तिब्राह्मणो वानायसाग्रान् \ldots शरग्राही द्रुततरं तोरणं गत्वा.}}~।\\
पिण्डं तु \renewcommand{\thefootnote}{7}\footnote{B, D,F वह्नौ for अग्नौ .K पातयेत् for तापयेत्.}तापयेदग्नौ पञ्चाशत्पलिकं समम्~॥}
\end{quote}

\newpage
%७१ दिव्येषु अग्निदिव्यविधिः [§७ 

कालिकापुराणे\renewcommand{\thefootnote}{1}\footnote{B ०पुराणे तु ; D,F, पुराणेपिः.}

\begin{quote}
{\vy शतार्धपलिकं वृत्तं द्वादशाङ्गुलमायतम्~।\\
\renewcommand{\thefootnote}{2}\footnote{C, D, F लोह० for लौह०.}लौहमग्निमयं ध्मातं देयं राज्ञाभि\renewcommand{\thefootnote}{3}\footnote{B,D,F. अतिशापिते for अभिशापिने\textendash }शापिने~॥}
\end{quote}

शङ्खलिखितौ तु पिण्डस्य षोडशपलत्वमाहतुः\textendash\ अथवा सप्ताश्चत्थपत्रान्तरितं षोडशपल\renewcommand{\thefootnote}{4}\footnote{B ०वर्णमयं पिण्ड०; D,F ः अग्निमयं पिण्ड०.}मग्निवर्णं पिण्डमञ्जलिनादाय \textendash\ इत्यादि~। एतच्चाशक्तौ~। तृतीयतापे तप्यन्तम् ( नारद् ४. २९० )\textendash\ इति नारदोक्तेस्त्रिवारं पिण्डः\renewcommand{\thefootnote}{5}\footnote{D,F पिण्डं संतर्प्य for पिण्डः संताप्यः} संताप्यः~। तत्र संताप्य जले क्षिप्त्वा\renewcommand{\thefootnote}{6}\footnote{F क्षिप्त्वा तु पुनः.} पुनः संताप्य जले क्षिप्त्वा पुनस्ताप्यमाने प्राड्विवाको धर्मावाहनादिशिरःपत्रारोपणान्तं कुर्यात्~॥ तत्र वह्निपूजायां\renewcommand{\thefootnote}{7}\footnote{B, D,F . धर्मूजायां for वह्निपूजायाम्} विशेषमाह पितामहः 

\begin{quote}
{\vy तत्र पूजां हुताशस्य कारयेन्मनुजाधिपः\\
रक्तचन्दनधूपाभ्यां रक्तपुष्पैस्तथैव च~॥}
\end{quote}

हारीतः 

\begin{quote}
{\vy प्राङ्मुखस्तु ततस्तिष्ठेत्प्रसारितकराङ्मुलिः~।\\
आर्द्रवासाः शुचिश्चैव शिरस्यारोप्य पत्रकम्~॥}
\end{quote}

शोध्य इति शेषः~। पितामहः\renewcommand{\thefootnote}{8}\footnote{C, G, H, N omit पितामह\ldots शुचिः स्मृतिच० and अपरार्क read पश्चिमे for प्रथमे.}

\begin{quote}
{\vy प्रथमे मण्डले तिष्ठेत्प्राग्मुखः प्राञ्जलिः शुचिः~॥}
\end{quote}

\newpage
% §७] दिव्येषु अग्निदिव्यविधिः ७२ 

नारदः ( ४. ३०१ ) 

\begin{quote}
{\vy हस्तक्षतेषु सर्वेषु कुर्याद्धंस\renewcommand{\thefootnote}{1}\footnote{नारद reads काकपदानि.}पदानि तु~।\\
तान्येव पुनरालक्षेद्धस्तौ बिन्दुविचित्रितौ~॥}
\end{quote}

याज्ञवल्क्यः (२. १०३) 

\begin{quote}
{\vy करौ विमृदितव्रीहेरङ्कयित्वा ततो न्यसेत्~।\\
सप्ताश्वत्थस्य पत्राणि तावत्सूत्रेण वेष्टयेत्~॥}
\end{quote}

तावदिति क्रियाविशेषणम्~। तेन सप्तवारं वेष्टयेदिति विज्ञानेश्वरः तावतां\renewcommand{\thefootnote}{2}\footnote{D,F तावत्सूत्राणां for तावतां सूत्राणाम्\textendash } सूत्राणां समाहारस्तावत्सूत्रं तेन सप्त\renewcommand{\thefootnote}{3}\footnote{B, C, D, F. G, K read तावत्सूत्रेण bofore\textendash\ सप्तसूत्र्या.}सूत्र्या सकृद्वेष्टयेदित्यर्थं इति मदनः~॥ पितामहः

\begin{quote}
{\vy सप्त पिप्पलपत्राणि अक्षतान्सुमनो दधि~।\\
हस्तयोर्निक्षिपेत्तत्र सूत्रेणावेष्टनं तथा~॥}
\end{quote}

अत्र प्राड्विवाकस्य पिण्डस्थाग्र्यभिमन्त्रा प्रयोगे वक्ष्यन्ते~॥ याज्ञवल्क्यः २.१०४\textendash\ १०५ ) 

\begin{quote}
{\vy त्वमग्ने सर्वभूतानामन्तश्चरसि पावक\renewcommand{\thefootnote}{4}\footnote{C पावकः for पावक.}~।\\
साक्षिवत्पुण्यपापेभ्यो ब्रूहि सत्यं कवे मम~॥

तस्येत्युक्तवतो लौहं पञ्चाशत्पलिकं\renewcommand{\thefootnote}{5}\footnote{D,F ०पलिके समे for ०पलिकं समम्.}समम्\\
अग्निवर्णं न्यसेत्पिण्डं हस्तयोरुभयोरपि~॥}
\end{quote}

पितामहः 

\begin{quote}
{\vy ततस्तं समुपादाय राजा धर्मपरायणः~।\\
संदशेन\renewcommand{\thefootnote}{6}\footnote{C,K. संशयेन for संदंशेन; F सन्दृशेन (?).} नियुक्तो वा हस्तयोस्तस्य निक्षिपेत्~।}
\end{quote}

\newpage
% ७३ दिव्येषु अग्निदिव्यविधिः [ §७

नारदः 

\begin{quote}
{\vy हस्ताभ्यां तं समादाय प्राड्वाकसमीरितः~।\\
स्थित्वैकस्मिंस्ततोन्यानि व्रजेत्सप्त त्वजिह्मगः~॥}
\end{quote}

पितामहः 

\begin{quote}
{\vy त्वरमाणो न गच्छेत्तु \renewcommand{\thefootnote}{1}\footnote{G स्वस्थैर्गच्छेत्}स्वस्थो गच्छेच्छनैः शनैः~।\\
न मण्डलान्य\renewcommand{\thefootnote}{2}\footnote{B,D,F मण्डलमतिक्रामेत;, C,K मण्डलं व्यतिक्रामेत्\textendash }तिक्रामेन्नान्तरा स्थापयेत्पदम्~॥

अष्टमं मण्डलं गत्वा\renewcommand{\thefootnote}{3}\footnote{कृत्वा for गत्वा.} नव\renewcommand{\thefootnote}{4}\footnote{C नवमं for नवमे.}मे निक्षिपेद्बुधः~॥} \renewcommand{\thefootnote}{5}\footnote{B,D,F omit इति after बुधः.}इति~।
\end{quote}

पिण्डस्तु सतृणे\renewcommand{\thefootnote}{6}\footnote{G सतृणानवममण्डले} नवममण्डले निक्षेप्यः

\begin{quote}
{\vy मण्डलानि तथा\renewcommand{\thefootnote}{7}\footnote{Creads मण्डलानि ततस्तद्धस्तयोः प्रास्येत् for मण्डलानि तथा \& c.} सप्त षोडशाङ्गुलमानतः~।\\
तावदन्तरतो गच्छेद्गत्वा नवतृणे क्षिपेत्~॥} इति
\end{quote}

कालिकापुराणात्~॥ पितामहः

\begin{quote}
{\vy ततस्तद्धस्तयोः\renewcommand{\thefootnote}{8}\footnote{B,D,F omit स्तयोः प्रास्वे\ldots तस्य वि ( निर्दिशेत्)} प्रास्येद्व्रीहीन्वा यदि वा यवान्~।\\
निर्विशङ्केन तेषां तु हस्ताभ्यां मर्दने कृते~॥

निर्विकारे दिनस्यान्ते शुद्धिं तस्य विनिर्दिशेत्~॥}
\end{quote}

कात्यायनः

\begin{quote}
{\vy प्रस्खलत्यभियुक्तश्चेत्स्थानादन्यत्र दह्यते~।\\
न तद्दग्धं विदुर्देवास्तस्य भूयोपि दापयेत्~॥}
\end{quote}

\newpage
\fancyhead[CE,CO]{अग्निदिव्यप्रयोग}
%७§] अग्निदिव्यप्रयोग ७४ 

\noindent
याज्ञवल्क्यः (२. १०७ ) 

\begin{quote}
{\vy अन्तरा \renewcommand{\thefootnote}{1}\footnote{C पातिते पिण्डे}पतिते पिण्डे संदेहे वा पुनर्हरेत्~॥}
\end{quote}

अथ प्रयोगः~॥ पूर्वद्युर्भूशुद्धिं विधाय परेद्युर्नव मण्डलानि कार्याणि~। तत्राद्यं षोडशाङ्गुलं विधायाग्रे द्वात्रिंशदङ्गुलं भूभागं द्विधा विभज्य द्वितीयभागे गन्तृपदप्रमाणं द्वितीयं मण्डलं कार्यम्~। अवशिष्टमन्तरालं भवति~। एवमेव तृतीयाद्यष्टमपर्यन्तमन्तरालसहितं निर्वर्त्याग्रे षोडशाङ्गुलमन्तरा\renewcommand{\thefootnote}{2}\footnote{पर्यन्तमन्तरालं मुक्त्वा नवम०}लं मुक्त्वा नवमम\renewcommand{\thefootnote}{3}\footnote{C, K नवमपरिमिता ;F ०परिमिताङ्गुलानां षट्पञ्चाशदधिकम्}परिमिताङ्गुलप्रमाणं कार्यम्~। तथा चाष्टानां मण्डलानामन्तरालं च मिलित्वाङ्गुलानां षट्पञ्चाशदधिकं शतद्वयं संपद्यते~॥

\begin{quote}
{\vy तिर्यग्यवोदराण्यष्टावूर्ध्वाग्रा व्रीहयस्त्रयः~।\\
प्रमाणमङ्गुलस्योक्तं वितस्ति\renewcommand{\thefootnote}{4}\footnote{B,D,F वितस्तिर्द्वादशाङ्गुलम् for ्०ङ्गुलः.}र्द्वादशाङ्गुलः~॥

हस्तो वितस्तिद्वितयं दण्डो हस्तचतुष्टयम्~।\\
तत्सहस्रद्वयं क्रोशो योजनं तच्चतुष्टयम्~॥}
\end{quote}

वितस्त्याद्युत्तरत्रो\renewcommand{\thefootnote}{5}\footnote{D,F उपयोज्यते for उपयोक्ष्यते~।}पयोक्ष्यते~॥ ततः पश्चिमादिषु नवसु मण्डलेष्वधिदेवताः क्रमेणाग्निं वरुणं वायुं यममिन्द्रं कुबेरं सोमं सवितारं सर्वदेवताश्च संपूज्य मण्डलभूभागाद्दक्षिणप्रदेशे लौकिकमग्निमुपसमाधायाग्नये पावकाय स्वाहेत्याज्येनाष्टोत्तरशतं जुहुयात्प्राड्विवाकः शान्त्यर्थम्~। ततस्तस्मिन्नग्नावस्रिरहितं वृत्तं श्लक्ष्णमष्टाङ्गुलायामं पञ्चाशत्पलसंमितमयःपिण्डं प्रक्षिप्य तस्मिंस्ताप्यमाने धर्मावाहनादि हवनान्तं धटविध्युक्तमनुष्ठानकाण्डं कृत्वा तृतीये\renewcommand{\thefootnote}{6}\footnote{D,F तृतीयतापे for तृतीये तापे.} तापे वर्तमानेयः\textendash\ पिण्डस्थमग्निमभिमन्त्रयेदेभिर्मन्त्रैः प्राड्विवाकः

\newpage
% ७५ अग्निदिव्यप्रयोगः [§७ 

\begin{quote}
{\vy त्वमग्ने वेदाश्चत्वारस्त्वं च यज्ञेषु हूयसे~।\\
त्वं मुखं सर्वदेवानां त्वं मुखं ब्रह्मवादिनाम्~॥

जठरस्थो हि भूतानां ततो वेत्सि शुभाशुभम्~।\\
पापं पुनासि वै यस्मात्तस्मात्पावक उच्य\renewcommand{\thefootnote}{1}\footnote{C, D, G, K उच्यते for उच्यसे}से~॥

पापेषु दर्शयात्मानमर्चिष्मान्भव पावक~।\\
अथवा शुद्धभावेषु \renewcommand{\thefootnote}{2}\footnote{मिता० and अपरार्क read शीतः for शुद्धः , and वीर०reads शान्तः.}शुद्धो भव हुताशन~॥

त्वमग्ने सर्वभूतानामन्तश्चरसि साक्षिवत्~।\\
त्वमेव देव जानीषे न विदुर्यानि मानवाः~॥

व्यवहाराभिशस्तोयं मानुषः शुद्धिमिच्छति~।\\
तदेनं संशयादस्माद्धर्मतस्त्रातुमर्हसि\renewcommand{\thefootnote}{3}\footnote{A, M omit इति after अर्हसि.}} इति~। ( last two are विष्णुध. सूः ११. ११\textendash\ १२)
\end{quote}

लोहशुद्ध्यर्थं सुतप्तं लोहपिण्डमुदके निक्षिप्य पुनः संताप्योदके निक्षिप्य पुनः संतापनं \renewcommand{\thefootnote}{4}\footnote{C, D,F ,K तृतीयस्तापः.}तृतीयतापः~। एवमभिमन्त्रितं सुतप्तमग्निवर्णमयःपिण्डं संदंशेन\renewcommand{\thefootnote}{5}\footnote{F संदशनेन for संदंशेन.} गृहीत्वा प्राड्विवाक उपोषितस्य स्नातस्यार्द्रवाससः शिरोबद्धप्रतिज्ञापत्रस्य पश्चिमे मण्डले तिष्ठतो दिव्यकर्तुः \renewcommand{\thefootnote}{6}\footnote{B परतःfor पुरतः.}पुरत आनीय

\begin{quote}
{\vy त्वमग्ने सर्वभूतानामन्तश्चरसि पावक~।\\
साक्षिवत्पुण्यपापेभ्यो ब्रूहि सत्यं कवे मम~॥} इति (याज्ञवल्क्य २. १०४)
\end{quote}

\begin{sloppypar}
दिव्यकर्त्राभिमन्त्रितं तस्य कृतसंस्कारयोर्हस्तयोर्निक्षिपेत्~। कृतसंस्कारयोर्व्रीहीन्मर्दयित्वाञ्जलीकृतयोस्तिलकालकव्रणकिणादिस्थानेष्वलक्तकर\textendash
\end{sloppypar}

\newpage
% ७§ ] अग्निदिव्यप्रयोगः ७६

\noindent
सादिनाङ्कयित्वा सप्त समान्यश्वत्थस्य पत्राणि अलाभेर्कपत्राणि सप्त शमीपत्राणि सप्त वा\renewcommand{\thefootnote}{1}\footnote{B, C omit वा after सप्त.} दूर्वापत्राणि अक्षतान्दध्यक्ताक्षतान्पुष्पाणि च विन्यस्य सप्तभिः शुक्लसूत्रैः सप्तकृत्वो वेष्टितयोस्ततो दिव्यकर्ता द्वितीयादीन्यष्ट\renewcommand{\thefootnote}{2}\footnote{B, C, F, G, K ०ष्टमान्तानि मण्डलेष्वे पदम्.}मान्तानि मण्डलानि तेष्वेव पदं निक्षिपन् सप्त पदानि शनैर्गत्वाञ्जलिस्थमयःपिण्डं नव\renewcommand{\thefootnote}{3}\footnote{C,K सतृणे नवमे निक्षिपेत्.}मे निक्षिपेत्~। ततः पुनः कराभ्यां व्रीहीन् संघृष्यादग्धहस्तश्चेच्छुद्धो भवति~॥ \renewcommand{\thefootnote}{4}\footnote{A, D, M omit इत्य\ldots विधिः.}इत्यग्निदिव्यविधिः~॥

\begin{quote}
\textbf{\Large ॥~अथ जलविधिः~॥}
\end{quote}

पितामहः

\begin{quote}
{\vy तोयस्यातः प्रवक्ष्यामि विधिं धर्म्यं सनातनम्~।\\
\renewcommand{\thefootnote}{5}\footnote{B ( by oversight ) reads मण्डलं धूपदीपाभ्यां शरान्पूजयेदित्यन्वयः दिव्यतत्त्व reads मण्डपं for मण्डलम्.}मण्डलं धूपदीपाभ्यां कारयेत विचक्षणः~॥

शरान्संपूजयेद्भक्त्या वैणवं च धनुस्तथा~।\\
मङ्गलैः पुष्पधूपैश्च ततः कर्म समाचरेत्~॥}
\end{quote}

धूपदीपाभ्यां शरान्संपूजयेदित्यन्वयः~। पूजा च मण्ड\renewcommand{\thefootnote}{6}\footnote{D, F मण्डलं for मण्डले.}ले कार्या~॥ धनुःपरिमाणा\renewcommand{\thefootnote}{7}\footnote{B, C,D. F, G. K धनुःपरिमाणाद्याह.}न्याह नारदः (४. ३०७ ) 

\begin{quote}
{\vy क्रूरं धनुः सप्तशतं मध्यमं षट्शतं स्मृतम्~।\\
मन्दं पञ्च्चशतं ज्ञेयमेष ज्ञेयो धनुर्विधिः~॥

मध्यमेन तु चापेन प्रक्षिपेच्च शरत्रयम्~।\\
हस्तानां तु शते सार्धे लक्ष्यं कृत्वा विचक्षणः~॥}
\end{quote}

\newpage
% ७७ जलदिव्यविधिः [ §७ 

\noindent
सप्तशतं सप्ताधिकशताङ्गुलम्~। एवं षट्शतं पञ्चशतं च~॥ कात्यायनः

\begin{quote}
{\vy शरांश्चानायसाग्रांश्च प्रकुर्वीत विशुद्धये~।\\
वेणुकाण्डमयांश्चैव क्षेप्ना तु सुदृढं क्षिपेत्~॥}
\end{quote}

नारदः 

\begin{quote}
{\vy गत्वा तु तज्जलस्थानं तटे तोरण\renewcommand{\thefootnote}{1}\footnote{B, C, D, F, G, H, K तोरणमण्डिते (D, F ०मण्डितैः ).}मुच्छ्रितम्~।\\
कुर्वीत कर्णमात्रं तु भूमिभागे समे शुचौ~॥

गन्धमाल्यैः सुरभिभिर्मधुक्षीरघृतादिभिः~।\\
वरु\renewcommand{\thefootnote}{2}\footnote{C,G,H सतोरणाय कुर्वीत for वरुणाय प्रकुवीत.}णाय प्रकुर्वीत पूजामादौ समाहितः~॥

ब्राह्मणः क्षत्रियो वैश्यो रागद्वेषविवर्जितः~।\\
नाभिमात्रे जले स्थाप्यः पुरुषः स्थाणुव\renewcommand{\thefootnote}{3}\footnote{B,D स्थाणुवद्वनी नारद reads स्तम्भवत्.}द्वली~॥}
\end{quote}

पितामहः 

\begin{quote}
{\vy स्थापयेत्प्रथमं तोये स्तम्भवत्पुरुषं नृपः~।\\
आगतं प्राड्मुखं कृत्वा तोयम\renewcommand{\thefootnote}{4}\footnote{C ०मध्येनुकारणम्; K मध्यैनुकारिणम्.}ध्ये तु कारिणम्~॥

ततस्त्वावाहयेद्देवान्सलिलं चानुमन्त्रयेत्~॥}
\end{quote}

देवान्धर्मादीन्~। धर्मावाहनादि शिरःपत्रारोपणान्तं कुर्यादित्याद्यनुमन्त्रणमन्त्राः प्रयोगे द्रष्टव्याः~। व्यासः ( $=$याज्ञवल्क्य २. १०८) 

\begin{quote}
{\vy सत्येन माभिरक्ष त्वं वरुणेत्यभिशाप्य कम्~।\\
नाभिदघ्नोदकस्थस्य गृहीत्वोरू जलं विशेत्~॥}
\end{quote}

कमुदकमभिशाप्याभिमन्त्र्येत्यर्थः~। बृहस्पतिः 

\begin{quote}
{\vy अप्सु प्रवेश्य पुरुषं प्रक्षिपेत्सायकत्रयम्~॥}
\end{quote}

\newpage
%७§ ] जलदिव्यविधिः ७८ 

पितामहः 

\begin{quote}
{\vy }
\end{quote}
क्षेप्ता तु क्षत्रियः कार्यस्तद्वृत्तिर्ब्राह्मणोपि वा~।\\
अक्रूरहृदयः शान्तः सोपवासस्तथा शुचिः~॥

कात्यायनः 

\begin{quote}
{\vy क्षिप्तेषु मज्जनं कार्य गमनं\renewcommand{\thefootnote}{1}\footnote{B, D, F मज्जनं सप्तकालिकम्.} समकालिकम्~॥}
\end{quote}

मज्जन\renewcommand{\thefootnote}{2}\footnote{मज्जनसमकालिकालिकमित्यर्थः.}समकालिकमित्यर्थः~। नारदपितामहौ (नारद ४. ३०९\textendash\ ३१२)

\begin{quote}
{\vy शरप्रक्षेपणस्थानाद्युवा जवसमन्वितः~।\\
गच्छेत्परमया शक्त्या यत्रासौ मध्यमः शरः~॥

मध्यमं शरमादाय पुरुषोन्यस्तथाविधः~।\\
प्रत्यागच्छेत्तु वेगेन यतः स पुरुषो गतः~॥

आगत\renewcommand{\thefootnote}{3}\footnote{F अगतस्तु.}स्तु शरग्राही न पश्यति \renewcommand{\thefootnote}{4}\footnote{नारद reads यदा जले~। अन्तर्जलं यदा सम्यक् तदा शुद्धिम्.}तथा जले~।\\
अन्तर्जलगतं सम्यक्तदा शुद्धं विनिर्दिशेत्~॥

अन्यथा न विशुद्धः स्यादेकाङ्गस्यापि दर्शनात्~।\\
स्थानाद्वान्यत्र गमनाद्यस्मिन्पूर्वं निवेशितः~॥}
\end{quote}

एकाङ्गस्येति कर्णपरम्~। तथा च कात्यायनः

\begin{quote}
{\vy शिरोमात्रं तु दृश्येत न कर्णौ ना\renewcommand{\thefootnote}{5}\footnote{D,F न च नासिका.}पि नासिका~।\\
अप्सु प्रवेशने यस्य शुद्धं तमपि निर्दिशेत्~॥}
\end{quote}

पितामहः 

\begin{quote}
{\vy शरस्य पतनं ग्राह्यं सर्पणं \renewcommand{\thefootnote}{6}\footnote{B,D,F सर्पणं परिवर्जयन्.}परिवर्जयेत्~॥}
\end{quote}

\end{document}