\documentclass[11pt, openany]{book}
\usepackage[text={4.85in,7.85in}, centering, includefoot]{geometry}
\usepackage[table, x11names]{xcolor}
\usepackage{fontspec,realscripts}
\usepackage{polyglossia}
\usepackage{tikz}
\setdefaultlanguage{sanskrit}
\setotherlanguage{english}
\setmainfont[Scale=1]{Shobhika}
\newfontfamily\s[Script=Devanagari]{Shobhika}
\newfontfamily\regular{Linux Libertine O}
\newfontfamily\en[Language=English, Script=Latin]{Linux Libertine O}
\newfontfamily\knu[Script=Devanagari, Scale=1.1, Color=purple]{Shobhika-Bold}
\newfontfamily\knui[Script=Devanagari, Scale=0.8, Color=purple]{Shobhika-Bold}
\newfontfamily\qt[Script=Devanagari, Scale=1, Color=violet]{Shobhika-Regular}
\newcommand{\devanagarinumeral}[1]{%
	\devanagaridigits{\number \csname c@#1\endcsname}} % for devanagari page numbers
%\usepackage[Devanagari, Latin]{ucharclasses}
%\setTransitionTo{Devanagari}{\s}
%\setTransitionFrom{Devanagari}{\regular}
\XeTeXgenerateactualtext=1 % for searchable pdf
\usepackage{enumerate}
\pagestyle{plain}
\usepackage{fancyhdr}
\pagestyle{fancy}
\renewcommand{\headrulewidth}{0pt}
\usepackage{afterpage}
\usepackage{multirow}
\usepackage{multicol}
\usepackage{wrapfig}
\usepackage{vwcol}
\usepackage{microtype}
\usepackage{amsmath,amsthm, amsfonts,amssymb}
\usepackage{mathtools}% <-- new package for rcases
\usepackage{graphicx}
\usepackage{longtable}
\usepackage{setspace}
\usepackage{footnote}
\usepackage{perpage}
\MakePerPage{footnote}
%\usepackage[para]{footmisc}
%\usepackage{dblfnote}
\usepackage{xspace}
\usepackage{array}
\usepackage{emptypage}
\usepackage{hyperref}% Package for hyperlinks
\hypersetup{colorlinks,
citecolor=black,
filecolor=black,
linkcolor=blue,
urlcolor=black}

\newcommand\blfootnote[1]{%
 \begingroup
 \renewcommand\thefootnote{}\footnote{#1}%
 \addtocounter{footnote}{-1}%
 \endgroup
}

\setlength{\parskip}{0.6em}

\begin{document}
\thispagestyle{empty}
\indent

\vspace{2.5cm}
\begin{center}
\includegraphics[width=0.3\linewidth]{latex/a.JPG}

\rule{0.7\linewidth}{0.5pt}

\vspace{6mm}
{\huge GAEKWAD'S}
\vspace{4mm}
{\huge ORIENTAL SERIES}
\rule{0.2\linewidth}{0.5pt}
{\large No. 154}

\rule{0.7\linewidth}{0.5pt}
\end{center}

\newpage
\thispagestyle{empty}
\noindent
\underline{Gaekwad's Oriental Series}
Published under the Authority
of the Maharaja Sayajirao
University of Baroda, Baroda.

\vspace{2cm}
\begin{flushright}
General Editor
B. J. Sandesara,
M.A., Ph.D.
Director, Oriental Institute
\end{flushright}

\vspace{2.5cm}
\noindent
\textbf{No. 154}

\vspace{2.5cm}
\begin{center}
प्रशस्तपादभाष्यम्~।

\vspace{3mm}
\textbf{\Huge किरणावली~।}
\end{center} 

\newpage
\thispagestyle{empty}
\begin{center}
{\large PRAŚASTAPĀDABHĀṢYAM}

\vspace{2mm}
{\large With the Commentary}

\vspace{2mm}
\textbf{\huge KIRAṆĀVALĪ}

\vspace{2mm}
{\large OF}

\vspace{2mm}
{\large UDAYANĀCĀRYA}

\vspace{2cm}
Edited by

\textbf{Jitendra S. Jetly},

\hspace{2cm} M.A., Ph.D., Nyāyācārya

Director, Institute of Indology,

Dwarka (Saurashtra)

\vspace{2cm}
\includegraphics[width=0.3\linewidth]{latex/a.JPG}

\vspace{2cm}
\textbf{\Large Oriental Institute,}

\textbf{\Large Baroda.}

\textbf{\Large 1971}
\end{center}

\newpage
\thispagestyle{empty}
\noindent
First Edition
Copies 500
1971

\begin{center}
Published with the Financial Aid of the University Grants Commission
and the Gujarat State

\vspace{2cm}
\textbf{Price Rs. 30/ \textendash\ }
\end{center}

\vspace{2cm}
\noindent
\textit{\en Copies can be had from:\textendash}

\begin{center}
\textit{\textbf{\en The Manager,}}

\textbf{THE UNIVERSITY PUBLICATIONS SALES UNIT,}

\textbf{M. S. University of Baroda Press (Sadhana Press),}

\textbf{Near Palace Gate, Palace Road,}

\textbf{Baroda.}
\end{center}

\footnotetext{Printed by Shri R. J. Patel, Manager, The Maharajia Sayajirao University of Baroda Press (Sadhana Press), Near Palace Gate, Palace Road, Baroda and published by Dr. B. J. Sandesara, Director, Oriental Institute, Maharaja Sayajirao University of Baroda, Baroda \textendash\ 2, September 1971.}

\newpage
\thispagestyle{empty}
\begin{center}
{\Large FOREWORD}
\end{center}

Old commentaries on Praśastapāda's Bhāṣya on the Vaiśeṣikasūtras of Kaṇāda \textendash\ like the Vyomavatī of Vyoma Śivācārya, the Kiraṇāvalī of Udayanācāryā and the Nyāyakandalī of Śrīdharācārya have always wielded great influence among the students of the Vaiśeṣika school of Indian philosophy and are considered as landmarks in the history of the system. I am glad to put before the world of Sanskrit scholarship a fresh critical edition of the Kiraṇāvalī, prepared by Dr. Jitendra S. Jetly, M.A. Ph.D., Nyāyācārya, who is eminently qualified to edit the learned texts of the Nyāya \textendash\ Vaiśeṣika system, being very well up in traditional learning as well as modern methods.

It is a tradition among the Sanskrit Pandits to the purport that Udayana had written his Kiraṇāvalī upto the Guṇaprakaraṇa. But the edition of the Kiraṇāvalī which was published about fifty years ago by M. M.Vindhyeshvariprasad Dvivedi and Dhundiraj Shastri on the basis of nine manuscripts read upto some portions of the Buddhiprakaraṇa. There was no evidence to state definitely whether the above \textendash\ mentioned tradition was correct.

The present edition of the Kiraṇāvalī is mainly based on a rare manuscript from the Jñāna \textendash\ Bhaṇḍāra at Jaisalmer which, orthographically, appears to belong to about the 13th century. To our great pleasure and surprise, the Jaisalmer manuscript is complete, i.e. upto the Guṇaprakaraṇa, and proves the above \textendash\ mentioned tradition to be correct. In fact, this should be considered a veritable literary discovery.

Dr. Jetly has tried to make the edition as useful as possible, and I trust that it will be welcomed by scholars, as no edition of the Kiraṇāvalī was available in the market for a long time. Dr. Jetly is busy preparing a critical edition of the Nyāyakandalī also and we hope to publish the same in the G.O.S. in near future.

I take this opportunity to thank the University Grants Commission and the Government of Gujarat for jointly giving full financial aid towards the publication of this volume.

\noindent
Oriental Institute, \hspace{5.5cm} B. J. SANDESARA
Baroda. \hspace{8.5cm} Director
Dated 8th June, 1971

\newpage
\thispagestyle{empty}
\begin{center}
{\Large CONTENTS}
\end{center}

\begin{tabular}{m{1em} m{20em} m{6em}}
& & Page\\ [1.5ex]
I & Introduction & I \textendash xx \\ [1.5ex]
२ & भूमिका (संस्कृत) & xxi \textendash\ xxx\\ [1.5ex]
३ & विषयानुक्रमः & १ \textendash\ ७\\ [1.5ex]
४ & अकारादिविषयानुक्रमः & ८ \textendash\ १४ \\ [1.5ex]
५ & प्रशस्तपादभाष्यस्य किरणावली & १ \textendash\ २७५ \\ [1.5ex]
६ & परिशिष्टम् \textendash\ १ & २७६ \textendash\ २८३\\  [1.5ex]
७ & परिशिष्टम् \textendash\ २ & २८४ \textendash\ ३०६\\
\end{tabular}

\newpage
\thispagestyle{empty}
\begin{center}
{\Large INTRODUCTION}
\end{center}

\begin{quote}
{\qt यच्चिति  \textendash\ कणिकासङ्गो जगच्चितिविधायकः~।\\
तद्वन्दे परमं ब्रह्म सच्चिदानन्दलक्षणम्~॥}
\end{quote}

\noindent
\textbf{Editing of the Text :}

In the year 1950 \textendash\ 51 when I was permitted by the authorities of Sheth B. J. Institute of Learning and Research, Ahmedabad to go Jaisalmer for assisting पू. मुनिश्री पुण्यविजयजी in his rearrangement work of the MSS. libraries of Jaisalmer, I had a good fortune of coming across a number of MSS. which were unknown to the world of orientalists. The present edition of the किरणावली of उदयनाचार्य is one of the outcome of the work done in those days.

At that time I came across a MS. of the किरणावली. When the said MS. was examined and compared with the printed text of the किरणावली, to my surprise I found that the MS. contained some more portion than the printed text did. The MS. was complete upto the end of the गुणप्रकरण. It was known to the world of orientalists that उदयनाचार्य had written the किरणावली, a commentary on the प्रशस्तपादभाष्य of प्रशस्तपादाचार्य upto the end of the गुणप्रकरण but unfortunately no MS. was available containing even that much matter. Hence the किरणावली which was edited by Late म. म. पं. विन्ध्येश्वरीप्रसाद द्विवेदी and पं. ढुण्ढिराज though based on eight to nine MSS. was upto the incomplete portion of the topic बुद्धि of the गुणप्रकरण. It ended with न शब्दोऽर्यप्रत्यायक इति स्ववचनविरोधी~।\renewcommand{\thefootnote}{1}\footnote{Vide \textendash\ The किरणावली, p. 340, edited by late म. म. पं विन्ध्येश्वरीप्रसाद and ढुण्ढिराजशास्त्री and published in the Benares Sanskrit Series as Nos. 15, 50, 155 156, 157, in the year 1919.}

In the beginning it was thought to publish the new discovered portion of the किरणावली in some oriental journal, but as the printed text which was first edited and published in the year 1919 was out of print and was not available in the market it was found advisable to re \textendash\ edit the whole text once again and publish. My

\afterpage{\fancyhead[CE,CO]{\thepage}}
\cfoot{}
\newpage
%%%%%%%%%%%%%%%%%%%%%%%%%%%%%%%%%%%%%%%%%%%%%%%%%%%%%
\renewcommand{\thepage}{\roman{page}}
\setcounter{page}{2}

\noindent
friend Dr. B. J. Sandesara, the Director of the Oriental Institute, Baroda inspired me to re \textendash\ edit the work and accordingly the किरणावली is re \textendash\ edited.

In the same Jaisalmer MSS. Library a Palm \textendash\ leaf MS. of the प्रशस्तपादभाष्य was also discovered. This MS. was not complete but with the help of that MS. and other MS. of the देवशापाडा 's जैनज्ञानभण्डार MS. library which was made available then it was thought to re \textendash\ edit the प्रशस्तपादभाष्य also. Of course, the printed text of the भाष्य with the (1) न्यायकन्दली 0f श्रीधर भट्ट (2) with the किरणावली of उदयनाचार्य and the (3) प्रशस्तपादभाष्य of प्रशस्तपादाचार्य and the वैशेषिकसूत्र of कणाद with the उपस्कार of शङ्करमिश्र were all available at that time. Thus, not only the present edition of the किरणावली but the प्रशस्तपादभाष्य is also once again re \textendash\ edited. I must make it clear that while editing the प्रशस्तपादभाष्य with the किरणावली I have tried to include all the useful foot \textendash\ notes of the first edition of the किरणावली. As a suppliment to the work of उदयनाचार्य a small treatise viz. the लक्षणावली which was also attached to the first edition is attached here also as an appendix No. I. I have also tried to give the वैशेषिकसूत्र पाठ with its comparative readings of the different editions particularly that of Mithila Research Institute and of G.O.S. No. I36. It is given in this edition as an appendix No. 2.

In editing the किरणावली help of the printed text is taken but with that I have also taken the help of one more MS ot the किरणावली which was made available to me from देवशा 's पाडा MSS. Library, Ahmedabad. Of course, this MS. also contained the same portion as it is found in the printed text of the किरणावली. The MS. of the किरणावली of the Jaisalmer MSS. library measure 6.5" x 4.5" while that of देवशा 's पाडा measures 10" x 3.5". In the foot \textendash\ note, the different MSS. and the text are indicated by the following signs. भा or भु.भा. printed प्रशस्तपादभाष्य. क is equal to the same with the न्यायकन्दली of श्रीधरभट्ट. कि is equal to the printed text of the किरणावली which was first edited and published in the year 1919.

Text of the MS. of देवशा 's पाडाज्ञानभण्डार = दे and the text of the MS, of the Jaisalmer Bhandara = जे.

\newpage
\noindent
\textbf{The brief historical account of the न्याय and वैशेषिक system.}

In the realm of Indian philosophy both these systems go together because both the systems accept elements of each other without any reservation. If there is any difference it is in the number of प्रमाणs. The न्याय system accepts four प्रमाणs viz. प्रत्यक्ष, अनुमान, उपमान and शब्द while the वैशेषिक system does only two प्रमाणs viz. प्रत्यक्ष and अनुमान. It includes उपमान and शब्द in अनुमान. But the whole of the logic of the न्याय system is acceptable to the वैशेषिक system. On the other hand all the elements of the वैशेषिक systems are acceptable to the न्याय system. वात्स्यायन the author of the न्यायभाष्य clearly states this in his भाष्य on the न्यायसूत्र of गौतम alias अक्षपाद.\renewcommand{\thefootnote}{1}\footnote{अस्त्यन्यदपि द्रव्यगुणकर्मसामान्यविशेषसमवायः प्रमेयम्~। तद्भेदेन चापरिसङ्ख्येयम्~। अस्य तु तत्त्वज्ञानादपवर्गो मिथ्याज्ञानात्संसार इत्यत उपदिष्टं विशेषेणेति~। न्यायसूत्र \textendash\ वात्स्यायन भाष्य १ \textendash\ १ \textendash\ ९.

From very ancient time न्याय also came with वैशेषिक as a sister system probleming the atomic theory and many other things in common  \textendash\ A. B. Keith, Indian Logic and Atomism, p. 3.3.3.} Dr. D. R. Bhandarkar considers both the systems as Śaivaits i.e. of मद्देश्वर sect. His views are based on the statement of हरिभद्रसूरि the author of the षड्दर्शनसमुच्चय Who states नैयायिकs as Śaivaits and वैशेषिकs as पाशुपत\renewcommand{\thefootnote}{2}\footnote{Vide the तर्कभाषा edited by Dr. D. R. Bhandarkar, Bombay Sanskrit and Prakrit Series, No. XXXIV, intro.p.p. ii to vi.}. Dr. Bhandarkar also thinks that गौतम the founder of the न्याय system and कणाद the founder of the वैशेषिक system were the pupils of the said teacher. To support his thesis he quotes the following verses from the वायुपुराण:\textendash\

\begin{quote}
{\knui सप्तविंशतितमे प्राप्ते परिवर्ते समागते~।\\
जातूकर्ण्यो यदा व्यासो भविष्यति तपोधनः~॥~२०१~॥

तदाऽहं सम्भविष्यामि सोमशर्मा द्विजोत्तमः~।\\
प्रभासतीर्थमासाद्य योगात्मा लोकविश्रुतः~॥~२०२~॥

अत्रापि मम ते पुत्रा भविष्यन्ति तपोधनाः~।\\
अक्षपादः कणादश्च उलूको वत्स एव च~॥~२०३~॥\renewcommand{\thefootnote}{3}\footnote{Ibid intro., P. VII.}}
\end{quote}

\noindent
But how far the authorities of the different पुराणs are acceptable as historical facts is also a problem which requires serious consideration. However, there is no doubt that both these systems

\newpage
\noindent
were inter \textendash\ related from their very inseption.\renewcommand{\thefootnote}{1}\footnote{{\en \textit{cf.}}

\begin{quote}
{\qt न सामयिकत्वाच्छब्दार्थसम्प्रत्ययस्य~। न्या. सू 2 \textendash\ 1 \textendash\ 55\\
सामयिकत्वाच्छब्दार्थसम्प्रत्ययस्य~। वै. सू. 7 \textendash\ 2 \textendash\ 20}
\end{quote}

वात्स्यायन in his न्यायभाष्य on the न्यायसूत्र 2 \textendash\ 1 \textendash\ 34 quotes the tenets of the वैशेषिक system as follows:\textendash

\begin{quote}
{\qt यद्यवयवी नास्ति सर्वस्य ग्रहणं नोपपद्यते~। किं तत्सर्वम् ?}
\end{quote}

\noindent
द्रव्यगुणकर्मसामान्यविशेषसमत्वायाः~। Similarly वै.सू 3 \textendash\ 1 \textendash\ I6 and 4 \textendash\ 1 \textendash\ 6 indirectly referred o in the न्यायभाष्य of the न्या. सू. 3 \textendash\ 1 \textendash\ 33 and 3 \textendash\ 1 \textendash\ 67.} राजशेखरसूरि following his ancient predecessor हरिभद्रसूरि gives the brief पौराणिक history of the न्याय system as follows:

\begin{quote}
{\knui अथ योगमतं ब्रूमः शैवमित्यपराभिधम्~।\\
ते दण्डधारिणः प्रौढकौपीनपरिधायिनः~॥~८४~॥

कम्बलिकाप्रावरणा जटापटलशालिनः~।\\
भस्मोद्धूलनकर्तारो नीरसाहारसेविनः~॥~८५~॥

दोर्मूले तुम्बुकभृतः प्रायेण वनवासिनः~।\\
आतिथ्यकर्मनिरताः कन्दमूलफलाशनाः~॥~८६~॥

सस्त्रीका अथ निस्त्रिका निःस्त्रीकास्तेषु चोत्तमाः~।\\
पञ्चाग्निसाधनपराः प्राणलिङ्गधराः करे~॥~८७~॥

विधायदन्तपवनं प्रक्षाल्याङ्घ्रिकराननम्~।\\
स्पृशन्ति भस्मनाङ्गं त्रिस्त्रिः शिवध्यानतत्पराः~॥~८८~॥

यजमानो वन्दमानो वक्ति तेषां कृताञ्जलिः~।\\
ॐ नमः शिवायेत्येवं शिवाय नम इत्यसौ~॥~८९~॥

तेषां च शङ्करो देवः सृष्टिसंहारकारकः~।\\
तस्यावतारा सारा ये तेऽष्टादश तदर्चिताः~॥~९०~॥

तेषां नामान्यथ ब्रूमो नकुलीशोऽथ कौशिकः~।\\
गार्ग्यो मैत्र्यः कौरुषश्च ईशानः षष्ठ उच्यते~॥~९१~॥

सप्तमः पारगार्ग्यस्तु कपिलाण्डमनुष्यकौ~।\\
अपरकुशिकोऽत्रिश्च पिङ्गलाक्षोऽथ पुष्पकः~॥~९२~॥

बृहदाचार्योऽगस्तिश्च सन्तानः षोडशः स्मृत~।\\
राशीकरः सप्तदशो विद्यागुरुरथापरः~॥~९३~॥

 एतेऽष्टादश तीर्थेशास्तैः सेव्यन्ते पदे पदे~।\\
पूजनं प्रणिधानं च तेषां ज्ञेयं तदागमात्~॥~९४~॥}
\end{quote}

\newpage
\begin{quote}
{\knui अक्षपादो गुरुस्तेषां तेन तेह्याक्षपादिकाः~।\\
उत्तमां संयमावस्थां प्राप्ता नग्ना भ्रमन्ति ते~॥~९५~॥\renewcommand{\thefootnote}{1}\footnote{Vide षड्दर्शनसमुच्चय by राजशेखरसूरि pp. 8 \textendash\ 9. यशोविजयग्रन्थमाला (17). In these verses, it should be noted that the term योग means here न्याय therefore यौगाः means नैयायिकाः because they are more concerned with ${}^\circ$युक्ति 's or arguments.}}
\end{quote}

\vspace{-3mm}
In the same texts regarding the वैशेषिक system the author states briefiy as follows :\textendash\

\vspace{-3mm}
\begin{quote}
{\knui अथ वैशेषिकं ब्रूमः पाशुपतान्यनामकम्~।\\
लिङ्गादि यौगवत्तेषां ते ते तीर्थकरा अपि~॥\renewcommand{\thefootnote}{2}\footnote{Ibid p. 11.}}
\end{quote}

\vspace{-3mm}
Regarding the founder of the school the author of the same षड्दर्शनसमुच्चय states that :\textendash\

\vspace{-3mm}
\begin{quote}
{\knui शिवेनोलूकरूपेण कणादस्य मुनेः पुरः~।\\
मतमेतत् प्रकथितं तत औलूक्यमुच्यते~॥~३०~॥

अक्षपादेन ऋषिणा रचितत्वात्तु यौगिकम्~।\\
आक्षपादमितिख्यातं प्रायस्तुल्यं मतद्वयम्~॥~३१~॥\renewcommand{\thefootnote}{3}\footnote{Ibid p. 12.}}
\end{quote}

\vspace{-3mm}
\noindent
Here it should be however noted that राजशेखरसूरि is dependent for his above contentions on ancient पुराणs. But as we have stated that the authority of पुराणs regarding their historical contents have not become acceptable to all the oriental scholars.

As far as the न्याय system is concerned the date of its founder गौतम or अक्षपाद is not yet certainly fixed. Some scholars do not accept the identity of गौतम and अक्षपाद. Of course, पुराणs state गौतम as the author of न्यायशास्त्र.\renewcommand{\thefootnote}{4}\footnote{गौतमेन तथा न्यायम् Vide पद्मपुराण, उत्तरखण्ड 'अ ' 263. गौतमः स्वेन तर्केण etc. स्कन्दपुराण कालिका खण्ड अ. 96.} विश्वनाथ the author of the न्यायवृत्ति accepts गौतम as the author of the present न्यायसूत्र.\renewcommand{\thefootnote}{5}\footnote{एषामुनिप्रवरगौतमसूत्रवृतिः Vide beginning of the न्यायसूत्रवृत्ति.} While वात्स्यायन the author of the न्यायभाष्य, उद्योतकर the author of the न्यायवार्तिक, वाचस्पतिमिश्र the author of the न्यायवार्तिकतात्पर्यटीका and जयन्तभट्ट the author of the न्यायमञ्जरी accept अक्षपाद as the author of the present न्यायसूत्र.\renewcommand{\thefootnote}{6}\footnote{.\begin{quote}
{\qt योऽक्षपादमृषिं न्यायः प्रत्यभाद्वदतां वरम्~।\\
तस्य वात्स्यायन इदं भाष्यजातमवर्तयत्~॥} ending of the न्या. भा.~।
\end{quote}} In

\newpage
\noindent
\blfootnote{यदक्षपादः प्रवरो मुनीनां, शमाय शास्त्र जगतां जगाद~। मङ्गल verse of न्या. वा. अथ भगवता अक्षपादेन निःश्रयसहेतौ शास्त्रे प्रणीते~। न्या. ता. टी. आरभ्भ~। अक्षपादप्रणीतो हि विततो न्यायपादपः~॥ न्या. मं.~।}the same context, we get the dame of मेधातिथि. In the V act of the प्रतिमानाटक of भास it is stated that भोः काश्यपगोत्रोऽस्मि~। साङ्गोपाङ्गं वेदमधीये, मानवीयं धर्मशास्त्रं माहेश्वरं योगशास्त्रं, बार्हस्पत्यमर्थशास्त्रं मेधातिथेर्न्यायशास्त्रं प्राचेतसं श्राद्धकल्पं च~। Dr. Dasgupta thinks that गौतम learnt न्यायशास्त्र from मेधातिथि. He states that गौतम was not a historical personality, the real author of the न्यायसूत्र is अक्षपाद\renewcommand{\thefootnote}{1}\footnote{Vide History of Indian Philosophy by Dasagupta, Vol. II, pp. 393 \textendash\ 94.}. But म. म. डॉ. सतीशचन्द्र विद्याभूषण accepts मेधातिथि गौतम as a historical personality. According to him मेधातिथि गौतम is the founder of आन्वीक्षिकी, while अक्षपाद गौतम is altogether a different person who is the author of the न्यायसूत्र.\renewcommand{\thefootnote}{2}\footnote{Vide History of Indian Logic (Ancient).} आचार्य विश्वेश्वर also thinks अक्षपाद and गौतम as different persons. According to him, मेधातिथि गौतम is the author of the न्यादर्शन.\renewcommand{\thefootnote}{3}\footnote{Vide Intro. P.p. 25 \textendash\ 27 हिन्दी तर्कभाषा by आचार्य विश्वेश्वर.} On the other side श्रीफणीभूषणतर्कवागीश thinks that मेधातिथि अक्षपाद and गौतम are identical names of the one and the same person. He thinks that this गौतम is none else but the famous गौतम of the रामायण the husband of अहल्या. According to him, he is the author of the न्यायसूत्र. He depends upon a verse of स्कन्दपुराण.\renewcommand{\thefootnote}{4}\footnote{.\begin{quote}
{\qt अक्षपादो महायोगी गौतमाख्योऽभवन्मुनिः~।\\
गोदावरी समानेता अहल्यायाः पतिः प्रभुः~॥} स्कं. पु. महे. खं. २ कु. खं. अ. ५५.५
\end{quote}}

\noindent
All these references however do not lead to any definite conclusion regarding the अक्षपाद, गौतम or मेधातिथि. There are other such conjectures also but by such conjectures the date of the author of the present न्यायसूत्र can be adjusted somewhere between fifth and fourth century B.C. The भाष्य on the न्या. सू. is the work of वात्स्यायन. हेमचन्द्रसूरि in his अभिधानचिन्तामणि considers पक्षिलस्वामी, वात्स्यायन, कौटिल्य and चाणक्य as identical personalities.\renewcommand{\thefootnote}{5}\footnote{.\begin{quote}
{\qt वात्स्यायनो मल्लनागः कौटिल्यश्चणकात्मजः~॥~५१७~॥

द्रामिलः पक्षिलस्वामी विष्णुगुप्तोऽङ्गुलश्च सः~॥~५१८~॥} अभि. चि. मर्त्यकाण्ड
\end{quote}} In the त्रिकाण्डशेष the author पुरुषोत्तमदेव also

\newpage
\noindent
refers these names\renewcommand{\thefootnote}{1}\footnote{\vspace{-3mm} \begin{quote}
{\qt विष्णुगुप्तस्तु कौण्डिन्यश्चाणक्यो द्रामिलोंशुलः~।\\
वात्स्यायनो मल्लनाग पक्षिलस्वामिनावपि~॥} त्रिकाण्ड \textendash\ शेष कोष \textendash\ ब्रह्मवर्ग~।
\end{quote}}. However, there are no definite references to decide that वात्स्यायन the author of the न्यायभाष्य and चाणक्य, कौटिल्य and others are identical persons. He may be put somewhere in the 2nd century A.D. or so. On the न्यायभाष्य of वात्स्यायन, उद्योतकर alias भारद्वाज wrote the न्यायवार्तिक where he has vehemently refuted the views of the famous logicians like दिङ्नाग. He was also known as पाशुपताचार्य. He in the मङ्गलश्लोक of his वार्तिक clearly states his purpose of writing the न्यायवार्तिक\renewcommand{\thefootnote}{2}\footnote{\vspace{-3mm} \begin{quote}
{\qt यदक्षपादः प्रवरो मुनीनां शमाय शास्त्रं जगतां जगाद~।\\
कुतार्किकाज्ञाननिवृत्तिहेतुः करिष्यते तस्य मया निबन्धः~॥} न्या. वा.
\end{quote}}. As the Buddhist logician दिङ्नाग refuted the views of the न्यायभाव्य, उद्योतकर in his न्यायवार्तिक refuted दिङ्नाग. Max Mueller puts दिङ्नाग in the 6th century A.D. So the date of उद्योतकर may be put after that. His वार्तिक was refuted by धर्मकीर्ति another Buddhist logician who flourished in the 7th century A.D. His views were refuted by वाचस्पतिमिश्र in his न्यायवार्तिकतात्पर्यटीका and also by उदयनाचार्य in his ताप्तर्यपरिशुद्धि १ detailed commentary on the तात्पर्यटीका of वाचस्पतिमिश्र. Thus on the न्यायसूत्र we get good chain of works. On all the three works viz. भाष्य, वार्तिक and तात्पर्यटीका one अनिरुद्ध wrote short commentary viz. विवरणपज्जिका. Following him श्रीकण्ठ tried to write a commentary on all the four works of न्याय but unfortunately he could not complete it but a Jain scholar अभयतिलकसूरि did this डork of श्रीकण्ठ and wrote a big commentary on all the four न्याय works viz. भाष्य, वार्तिक, तात्पर्यटीका and तात्पर्यपरिशुद्धि. He flourished in the I4th century A.D. and the name of his work is न्यायालङ्कारसूत्रवृत्ति alias न्यायपञ्चप्रस्थान. There are also other works on the न्याय system like the तात्पर्यपरिशुद्धिप्रकाश of वर्धमानोपाध्याय (I5th century A.D.) the न्यायसूत्रोद्धार of वाचस्पतिमिश्र (द्वितीय) 15th century A.D. the न्यायसूत्रवृत्ति of विश्वनाथ न्यायपञ्चानन (I7th century A.D.) the न्यायसंक्षेप of गोविन्द खन्ना (I7th century A.D.) the न्यायसूत्रविवरण 0f राधामोहन गोस्वामी (19th century A.D.) and the गौतमसूत्रसंदीप of श्री कृष्णकान्त वागीश (19th century A.D.).

Unfortunately this is not the case with the वैशेषिकसूत्र though according to some scholars the वैशेषिक system is older than the न्याय

\newpage
\noindent
system.\renewcommand{\thefootnote}{1}\footnote{Vide introduction p. XXXV of the तर्कसंङ्ग्रह edited by Athalye and Bodas, and published in the Bombay Sanskrit Series No. LV.} Still however, the वैशेषिकसूत्र did not get that much importance as the न्यायसूत्र got. The latest discoveries fortunately show that the system was studied by the scholars of all the philosophical schools of India whether orthodox or heterodox. In the orthodox system it is acclaimed that काणादं पाणिनीयं च सर्वशास्त्रोपकारकम्~।

\noindent
According to the name of the propounder of the वैशेषिक system it is known as काणाद or औलूक्य. On the basis of the name of the गोत्र of कणाद or उलूक it is also known as काश्यपीयदर्शन. Like the history of गौतम alias अक्षपाद the history of कणाद or उलूक is also fully filled up with पौराणिक legends, so no definite date can be ascertained to his work viz. वैशेषिकसूत्र. The reason why the school is named वैशेषिक is explained by different scholars in different ways. गुणरत्नसूरि in his commentary on the षड्दर्शनसमुच्चय of हरिभद्रसूरि states that नित्यद्रव्यवृत्तयोऽन्त्या [ विशेषाः ]~। विशेषा एव वैशेषिकं, विनयादिभ्यः स्वार्थं इकच्~। तद् वैशेषिकं विदन्त्यधीयते वा 'तद्वेत्यधीते' इत्यणि वैशेषिकाः~। तेषामिदं वैशेषिकम्\renewcommand{\thefootnote}{2}\footnote{Vide षड्दर्शनसमुच्चय of हरिभद्रसूरि with the वृत्ति of गुणरत्न, p.23, also of न्यायावतारटिप्पण, p. 9, edited by Dr. P. L. Vaidya.}~। उदयनाचार्य also in his किरणावली states that विशेषो व्यवच्छेदः तत्त्वनिश्चयः, तेन व्यवहरतीत्यर्थः while दुर्वेकमिश्र in his धर्मोत्तरप्रदीप states that द्रव्यगुणकर्मसामान्यविशेषसमवायात्मकैः पदार्थविशेषैः व्यवहरन्तीति वैशेषिकाः~। रूढेश्चाभ्युपगतकणादशास्त्रा एवोच्यन्ते~। अथवा षट्पदार्थप्रतिपादेकतया विशेष्यते तदन्यस्माच्छास्त्रादिति विशेषः काणादं शास्त्रं विवक्षितम्~। तद्विदन्त्यधीयते वा इति वैशेषिकाः\renewcommand{\thefootnote}{3}\footnote{Vide धर्मोत्तरप्रदीप p. 240.}~।

Wi gives quite different tradition from c \textendash\ I \textendash\ Tsan's commentary the शतशास्त्र of देव in Chinese\renewcommand{\thefootnote}{4}\footnote{Wi \textendash\ Vaiśeṣika Philosophy Intro. p. 4.} " वैशेषिक the name of the सूत्र means superior or excellent and distinguished (or different). The origin of the name is in the fact that the system is distinguished from and superior to the साङ्ख्य ". Some scholars opine that the Śāstra in its initial state had its affliation to " an " Anti \textendash\ Vedic logic and epistemology of the Pre \textendash\ Buddhist " वैशेषिक ".\renewcommand{\thefootnote}{5}\footnote{A Primer of Indian Logic, Madras, I95I, intro. p. I0.} However the available literature does not support this notion. On the contrary वैदिकसंहिताs, ब्राह्मणs and उपनिषद्s evince a strong urge to know

\newpage
\noindent
the properties of entities and often record the result of investigation of the seers which bear close resemblance to the वैशेषिक system. The later वैशेषिकs try to see the seed of the system in उपनिषद्s like श्वेताश्वतर and others.\renewcommand{\thefootnote}{1}\footnote{cf. संबाहुभ्यां धमति संपतत्रैः \textendash\ श्वेताश्च. III \textendash\ 3 and also cf. षष्ठेन परमाणुरूप प्रधानाधिष्ठेयत्वम्~। ते हि गतिशीलत्वात् तत्रव्यपदेशाः पतन्तीति. The कुसुमाञ्जलि of उदयनाचार्य, V. 3.} This is also the case with other systems too. We also see that the वैशेषिक terms like अणु, परमाणु, वैशेषिकगुण, निःश्रेयस and समवाय are used in the old texts of the महाभारत also.\renewcommand{\thefootnote}{2}\footnote{For the term अणु vide महाभारत V. 16 \textendash\ I2 VIII. 8 \textendash\ 16 गीताप्रेस, गोरखपुर (आवृति). For the term वैशेषिक Vide Ibid VII. 5.15 and XII. 47.71. For the term निःश्रेयस vide Ibid V, 25.12; 33.15; 63.9.95.4 and 124.24 and others.}

Inspite of the olderness of the वैशेषिक System no भाष्य like the न्यायभाष्य on the न्यायसूत्र is available on the वैशेषिकसूत्र. Of course, the recent discoveries have brought before us the two old commentaries on the वैशेषिकसूत्र The one is edited by Prof. A. L. Thakur and published by Mithila Research Institute, Darbhanga, in the year 1957. The commentary though old the name of the commentator is not defnitely known. वादीन्द्र may be the supposed commentator but authentic proofs are yet to be found. Another old commentary on the वैशेषिकसूत्र is by चन्द्रानन्द and is edited by the learned जैनमुनि श्री जम्बूविजयजी and published by the Gaekwad Oriental Series in the year 1961. Both these commentaries are older than the available commentary viz. उपस्कार by शङ्करमिश्र.\renewcommand{\thefootnote}{3}\footnote{Vide वैशेषिकदर्शन edited by Prof. A. L. Thakur. Intro. p. 8 and also the वैशेषिकसूत्र edited by मुनिश्रीजम्बूविजयजी. Sanskrit Intro. p.12 \textendash\ 13.} The editor of the वैशेषिकसूत्र with the वृत्ति of चन्द्रानन्द gives the following table of the commentaries on the वैशेषिकसूत्र on the basis of his another edited important work द्वादशारनयचक्र.\renewcommand{\thefootnote}{4}\footnote{Vide Sanskrit Intro. p. 5 \textendash\ 11 of the वैशेषिकसूत्र with the वृत्ति of चन्द्रानन्द.} According to that following were commentaries.

\begin{center}
\includegraphics[width=0.8\linewidth]{latex/b.JPG}
\end{center}

\newpage
There was also the भाष्य of आत्रेय on the वैशेषिकसूत्र.\renewcommand{\thefootnote}{1}\footnote{Ibid English Intro. p. 12 \textendash\ 13.} Moreover, a भाष्य on the वैशेषिकसूत्र by रावण is referred to by मुरारि in his अनर्घराघव by पद्मनाभमिश्र in his किरणावली भास्कर, by गोविन्दप्रभु in his रत्नप्रभा a commentary on the शाङ्करभाष्य of the ब्रह्मसूत्र and by अनुभूतिस्वरूपाचार्य in his प्रकटार्थविवरण.\renewcommand{\thefootnote}{2}\footnote{Ibid appendix VI. p. 150. Foot \textendash\ note No. I.} मुनिश्रीजम्बूविजयजी on the strength of a reference from the अनर्घराघव of मुरारि concludes that a scholar viz. रावण (not the famous king of Laṅkā as referred to by मुरारि). was the author of कटन्दी a commentary on the वैशेषिकसूत्र.\renewcommand{\thefootnote}{3}\footnote{Ibid Sanskrit Intro. pp. 7.} Besides this the table referred to in this introduction shows that there was a commentary viz. वाक्य on the वैशेषिकसूत्र. On the वाक्य there was a commentary भाष्य and on the भाष्य there was commentary by प्रशस्तमति. Whether प्रशस्तमति and प्रशस्तपाद are identical or different persons requires thoughtful consideration. AIl these वैशेषिक works referred to in the table are not available in any form at present. MSS. or Ms. of any of the वैशेषिक work referred to in their introduction by Prof. A. L. Thakur and मुनिश्रीजम्बूविजयजी are yet to be discovered. Instead of these works as stated above a commentary on the वैशेषिकसूत्र by some unknown author (perhaps वादीन्द्र) and another by चन्द्रानन्द are discovered and edited by Prof. A. L. Thakur and मुनिश्रीजम्बूविजयजी respectively. These two works are not referred to by शङ्करमिश्र in his उपस्कार, so it seems that they were not easily available in the days of even शङ्करमिश्र. The सूत्रपाठ of these two editions also differ than that accepted by the author of the उपस्कार. The problem of the वैशेषिकसूत्र is well discussed in detail by Prof. A. L. Thakur in the introduction of वैशेषिकदर्शन edited by him.\renewcommand{\thefootnote}{4}\footnote{Vide intro. pp. 9 \textendash\ 15 वैशेषिकदर्शन with the commentary of an unknown author edited by Prof. A. L. Thakur.} It does not require any more repeatation here. There he has also pointed out how the पदार्थधर्मसङ्ग्रह alias प्रशस्तपादभाष्य of प्रशस्तपाद proved to be the worst enemy of the वैशेषिकसूत्र as far as its study was concerned. Due to more popularity of प्रशस्तपादभाष्य the learned scholars wrote commentaries on this भाष्य instead of वैशेषिकसूत्र. However, by this time now

\newpage
\noindent
three commentaries of वैशेषिकसूत्र are made available (1) उपस्कार of शङ्करमिश्र, (2) a commentary by वादीन्द्र or unknown author and (3) a commentary by चन्द्रानन्द. Later on sub \textendash\ commentaries on the उपस्कार are (1) the वृत्ति by जयनारायणतर्कपञ्चानन and (2) the उपस्कारपरिष्कार by पञ्चाननतर्करत्न. Both these Commentaries show marked inflluence of the Neo \textendash\ logical system. चन्द्रकान्त तर्कालङ्कार gives an Advaitic interpretation in his भाष्य of वैशेषिकसूत्र while the भारद्वाजवृत्तिभाष्य of गङ्गाधर वैद्य shows the infiuence of साङ्ख्य, आयुर्वेद and other systems. Thus the traditional interpretation of the वैशषिकसूत्र is partly vitiated. In the interpretation of the वैशेषिकसूत्र the व्याख्या of an unknown author and the वृत्ति of चन्द्रानन्द are of inmense help. Their interpretations are often supported by ancient sources including the Jaina and Buddhist texts. चन्द्रानन्द shows his close acquaintance with the पदार्थधर्मसङ्ग्रह of प्रशस्तपाद and is singularly free from any non \textendash\ Vaśieṣika bias.\renewcommand{\thefootnote}{1}\footnote{Vide Eng. intro. pp. 21 \textendash\ 22 of the वैशेषिकसूत्र with the वृत्ति of चन्द्रानन्द्र. Prof. A. L. Thakur also refers to the मानमनोहर a वैशेषिक treatise of वादी वागीश्वर in the introduction p. 21 of the edition of वैशेषिकदर्शन edited by him.}

With all this available material the study of the वैशेषिक system was made popular with the help of प्रशस्तपादभाष्य. Though it is the digest of the वैशेषिकसूत्र for the study of the system it proved to be the worst enemy of the सूत्र text. Four commentaries were written on this भाष्य by different learned scholars, (1) the व्योभवती by व्योमशिवाचार्य, (2) the न्यायलीलावती.\renewcommand{\thefootnote}{2}\footnote{The available न्यायलीलावती by वल्लभ is quite different work than this लीलावती.} by वत्साचार्य, (3) the किरणावली by उदयनाचार्य and (4) the न्यायकन्दली by श्रीधरभट्ट. Out of these commentaries the लीलावती of वत्साचार्य is yet to be discovered. It is not available even in the Ms. form in any of the MSS. libraries of India and abroad. From the available three commentaries the व्योमवती of व्योमशिवाचार्य seems to be the oldest one. व्योमशिव's views are referred to by the author of the न्यायकन्दली and by the author of the किरणावली by the name आचार्य. व्योमशिव thinks that वैशेषिकs are also agreeable in accepting शब्द as an independent प्रमाण.

\newpage
The system was studied not only by the scholars of orthodox schools but it was studied by the scholars of the heterodox schools too. The वैशेषिक views are referred to in the old Buddhist works like the लङ्कावतारसूत्र, the सूत्रालङ्कार which is attributed to अश्वघोष and in the रत्नावली of नागार्जुन.\renewcommand{\thefootnote}{1}\footnote{Vide Eng. Intro. p. 9 of the वैशेषिकसूत्र edited by मुनि श्रीजम्बूविजयजी.} In his एकश्लोकशास्त्र, नागार्जुन refers to the system as the system of उलूक as Kimura informs.\renewcommand{\thefootnote}{2}\footnote{Ibid Eng. Intro. p. 9.} In the तत्त्वोपप्लवसिंह the only available authentic work of चार्वाक system जयराशि the author of the work refutes the वैशेषिक views\renewcommand{\thefootnote}{3}\footnote{Ibid Eng. Intro. p. 9.}. जैनs go to that extent that one रोहगुक्त or रोहगुप्त the pupil of सिरिगुत्त of श्रीगुप्त a जैनाचार्य was the founder of the वैशेषिक school. According to them उलूक is another name of that रोहगुत्त. Prof. A. L. Thakur has discussed this problem in his introduction of the वैशेषिकसूत्र edited by मुनिश्रीजम्बूविजयजी\renewcommand{\thefootnote}{4}\footnote{Ibid Eng. Intro. pp. 6 \textendash\ 9.}. Whether उलूक or कणाद the founder of the वैशेषिक system is identical with this रोहगुत्त requires thoughtful consideration unless historically it is proved so. In the प्रमाणसमुच्चय of दिङ्नाग the वैशेषिक tenets are examined in detail\renewcommand{\thefootnote}{5}\footnote{Ibid Appendices 6 to 7.}. However, जैनs continued their interest in the study of the न्याय and the वैशेषिक system to such an extent that जैनाचार्यs began to write honest and faithful commentaries on the different न्याय and वैशेषिक works e.g. नरचन्द्रसूरि (13th century A.D.) and राजशेखरसूरि (14th century A.D.) \textendash\ both wrote commentaries on the न्यायकन्दली of श्रीधर known as टिप्पण and पञ्जिका respectively. अभयतिलकसूरि (14th century A.D.) wrote न्यायालङ्कारटिप्पण alias पञ्चप्रस्थान on all the four works viz. the न्यायभाष्य the वार्तिक the तात्पर्यटीका and the तात्पर्यपरिशुद्धि. जिनवर्धन (15th century A.D.) wrote a commentary viz. जिनवर्धनी on the सप्तपदार्थी of शिवादित्य (10th century A D.) गुणरत्नगणि (17th century A.D.) wrote a detailed commentary viz. तर्कतरङ्गिणी on the प्रकाश of गोवर्धन a commentary on the तर्कभाषा of केशवमिश्र. शुभविजयगणि (17th century A.D.) pupil of हीरविजयसूरि wrote the वार्तिक on the तर्कभाषा of केशवमिश्र. क्षमाकल्याण wrote a commentary on the दीपिका of the तर्कसङ्ग्रह

\newpage
\noindent
both of अन्नंभट्ट\renewcommand{\thefootnote}{1}\footnote{जिनवर्धनी on the सप्तपदार्थी of शिवादित्य and a commentary on the दीपिका of अन्नंभट्ट by क्षमाकल्याण are edited by me and published by L. D. Institute of Indology Ahmedabad and Rajasthan Puratattva Series, Jaipur respectively, while तर्कतरङ्गिणी is edited by Dr. Vasant G. Parikh as his Ph.D. work and avaits its publication.}. So it is not surprising now that Jaina scholars discuss in detail the tenets of the वैशेषिक schools in all their philosophical works. In the वैशेषिक works the study of the न्यायकन्दली of श्रीधरभट्ट the inhabitant of राढ region of Bengal gained much popularity. Good many MSS. of this work are available in different जैनभण्डारs. Similarly small treatise Iike सप्तपदार्थी, तर्कभाषा and तर्कसङ्ग्रह were also popular works of their study. The MSS. of different न्याय and वैशेषिक works found in the different जैनभण्डारs are the proofs of the same.\renewcommand{\thefootnote}{2}\footnote{Vide जिनतरत्नकोश of Prof. H. D. Velankar and other MSS. catalogues of different जैनभण्डारs published and unpublished.}

\noindent
\textbf{उदयनाचार्य the author of the किरणावली}

Before some years there was a controversy among scholars regarding the place of उदयनाचार्य. Some thought that he was of south, others thought that he was of Bengal and still others thought that he was of Mithila.\renewcommand{\thefootnote}{3}\footnote{Vide intro. of the किरणावली pp. 26 \textendash\ 27 edited by म. म. विन्ध्येश्चरीप्रसाद and ढुण्ढिराजशास्त्री.} But all these views do not hold much water. By this time it is a proved fact that he was a native of the village मङ्गरौनी situated at 21 miles north of दरभङ्गा situated on the eastern bank of the river कनका. Thus he was of मिथिला region of the present Bihar. Regarding his date there is no much dispute, because he himself mentions his date in his लक्षणावली and writes that he composed this work at the end of 906 शक year.\renewcommand{\thefootnote}{4}\footnote{Vide colophon of the लक्षणावली.
\begin{quote}
{\qt तर्काम्बराङ्कप्रमितेष्वतीतेषु शकान्ततः~।\\
वर्षेषूदयनश्चक्रे सुबोधां लक्षावलीम्~॥}
\end{quote}
Also vide Intro. pp. 1 \textendash\ 2 of लक्षणावली edited by Late Pt. शशिनाथ ज्ञा and published by Mithila Research Institute in the year 1963.}

\newpage
\begin{center}
\textbf{\large HIS WORKS}
\end{center}

About his independent works known to the oriental scholars are: (1) The न्याय \textendash\ कुसुमाञ्जलि proving the existence of God, (2) the आत्मतत्त्वविवेक alias the बौद्धधिक्कार refuting बौद्ध tenets regarding आत्मा in detail, (3) the लक्षणमाला a small treatise on the वैशेषिक elements, (4) The लक्षणावली a primer of again वैशेषिक tenets and (5) the न्यायपरिशिष्ट alias प्रबोधसिद्धि  \textendash\ an independent exposition of the 1st chapter of the न्यायसूत्र of गौतम. Other two works are the commentaries on न्याय and वैशेषिक works. The तात्पर्यपरिशुद्धि is a detailed commentary on the तात्पर्यटीका of वाचस्पतिमिश्र. It is not yet fully published. MSS. are there and awaits publication. The किरणावली is another work a commentary on the प्रशस्तपादभाष्य of प्रशस्तपादाचार्य and is now re \textendash\ edited and published. It is said that श्रीहर्ष refuted the views of उदयन in his खण्डनखण्डखाद्य to take revenge of his father whom उदयन had defeated in an open debate. We do not know how far this legend is historically true because it reflects on the proved date of उदयन.\renewcommand{\thefootnote}{1}\footnote{For detail vide intro. 26 and footnote No. 1 and 2 of the first edition of the किरणावली.}

\noindent
\textbf{Commentaries and sub \textendash\ commentaries on the किरणावली}

उपाध्याय वर्धमान the son of the great savant उपाध्याय गङ्गेश the famous author of the चिन्तामणि and the father of नव्यन्याय had written commentaries on almost all the works of उदयन. उपाध्याय वर्धमान names his commentaries as प्रकाश. So he has written the प्रकाश on the किरणावली also.\renewcommand{\thefootnote}{2}\footnote{This commentary measures about 6000 श्लोकप्रमाण. It is in two parts द्रव्यप्रकाश and गुणप्रकाश. It begins as follows: \textendash
\begin{quote}
{\qt मिलन्मन्दाकिनीमल्लीदामां मूर्ध्नि पुरद्विषः~।\\
विश्वबीजाङ्कुरप्रख्यां वैधवीं तां कलां नुमः~॥}
\end{quote}

कर्तव्यविघ्नविघातकं रविनमस्कारं निबध्नाति विद्येति~। यदिति सामान्यतोऽपि कर्तृनिर्देशो विद्याविद्ययोः सन्ध्यारजनीभ्यां निरूपणात् रविरुदेता लभ्यते \textendash\ and so on. The end reads in the Ist part as follows :

समुच्चयभ्रमं निराकरोति क्वचिदिति~। सुषुप्त्यनन्तरं केवलादृष्टजन्या मनःक्रियेत्यर्थः~। आशुभं चास्त्वित्यत्रेति शद्वस्यार्थमाह इति शब्द इति~। अत्र वर्तते इति शेषः~।} On the प्रकाश of उपाध्याय वर्धमान there is a commentary viz.

\newpage
\noindent
\blfootnote{Contd. Foot \textendash\ note No. 2 from page No. 27
\begin{quote}
{\qt यस्तर्कतन्त्रशतपत्रसहस्ररश्मि\\
गङ्गेश्वरः सुकविकैरवकाननेषु~।\\
तस्यात्मजोऽतिगहनां किरणावलीं तां\\
प्राकाशयत् कृतिमुदे बुध \textendash\ वर्धमानः~॥}
\end{quote}
The 2nd part begins as follows :
\begin{quote}
{\qt एकत्रसंन्निपतितासितकण्ठबिम्बमन्यत्र चन्द्रकिरणाहितशुभ्रभावम्~।\\
सन्ध्याञ्जलिमहमनङ्गरिपोर्नमामि, हस्तप्रविष्टमिव पुण्यतमं प्रयागम्~॥}
\end{quote}
नतिवन्नुतेरपि प्रारीप्सितविघ्नविघातकत्वात् तामादौ कृतां निबध्नाति तुष्टेरिति~। etc.

The end reads as follows :

कारणाभावादिति~। कारणत्वं कार्यनिरूप्यमिति कार्यस्य प्राक् सत्त्वं सिद्धमित्यर्थः~। and the verse which is at the end of the 1st part is once again repeated here. Ibid. Intro. pp. 2 \textendash\ 3, footnote No. 1.}
प्रकाशिका\renewcommand{\thefootnote}{1}\footnote{This commentary measures about 5500 श्लोकप्रमाण. It is only on the 1st part of the प्रकाश. It begins as follows :
\begin{quote}
{\qt कैशोरं कलयन्तं मायाकायं पुरातनं पुरुषम्~।\\
नन्दालिन्दनिकेतं निगमरहस्यं नमस्यामि~॥

यः कैशोरे विश्वविख्यातकर्मा\\
धर्माचार्यः श्रीमहादेवशर्म्मा~।\\
तत्सोदर्यो वर्धमानस्य सूक्तौ\\
भावं मेघः सम्यगाविष्करोति~॥}
\end{quote}

मूर्ध्नि पुरद्विष इति पूर्वान्वयि~। तेनान्यदपि बीजं जलसम्बधेनाङ्कुरतीति ध्वनिर्लभ्यते~। विद्याविद्ययोरिति \textendash\ नन्वत्र व्यारव्याने विपरीतो रूप्यरूपकभाव इति चेदत्र विभक्तिविपरिणामेन तथैवान्वय इति मिश्राः~।
The end reads as follows:

यद्यपि यत्र तृतीय क्षण एव द्रव्यनाशस्तत्र न परत्वापरत्वोत्पत्तिर्द्वितीयक्षणे कालादिसंयोगस्यासमवायिकारणस्योत्पत्तेः, तृतीयक्षणे च द्रव्यस्यैव नाशादिति व्यभिचारो भवत्येव; तथापि स्वरूपयोग्यतामात्रमिह साध्यम्~। तत्सिद्धे च नित्यत्वसमानाधिकरणायास्तस्याः फलोपधानव्याप्यतया परत्वादिति भावः~। शेषं सुबोधम्~।
\begin{quote}
{\qt विंशाब्दे जयदेवपण्डितकवेस्तर्काब्धिपारं गतः\\
श्रीमानेष भगीरथः समजनि श्रीचन्द्रपत्यात्मजः~।\\
श्रीधीरातनयेन तेन रचिता श्रीमन्महेशाग्रज \textendash\ \\
श्रीदामोदरपूर्वजेन जयतादाचन्द्रमेषा कृतिः~॥

हंहो गिरीश करुणामय मानसोऽपि, किं मां मुहुः क्षिपसि दुःखमये शरीरे~।\\
सत्कर्म तादृंगिति चेन्ननु चन्द्रचूड, तत्कर्म कारयसि ननु हतचेतसं माम्~॥}
\end{quote}
\noindent
Ibid page 3, foot note No. 1.} by भगीरथ ठक्कुर alias मेधठक्कुर of Mithila. He was a
pupil of

\newpage
\noindent
पक्षधरमिश्र great neologician of मिथिला and a contemporary of वासुदेव सार्वमौम. It seems that भगीरथ ठक्कुर has commented upon almost all the works of उपाध्याय वर्धमान who was the native of the village करिजन in the दरभङ्गा district of मिथिला. Another commentary on this work of उपाध्याय वर्धमान is of रघुनाथ शिरोमणि of Bengal. The name of this commentary is गुणप्रकाशविवृति alias गुणदीधिति.\renewcommand{\thefootnote}{1}\footnote{The work is about 1500 श्लोकप्रमाण. It is on the गुण chapter. The 1st part on the द्रव्य chapter is however not available. It begins as follows: \textendash
 \begin{quote}
{\qt ॐ नमः सर्वभूतानि विष्टभ्य परितिष्ठते~।\\
अखण्डानन्दबोधाय पूर्णाय परमात्मने~॥}
 \end{quote}
विनापि धर्मिनिरूपणं धर्मनिरूपणं भवत्येव, तद्धर्मत्वेन निरूपणं~। तन्निरूपणाधीनमिति चेत्,
तद्धर्मितया तन्निरूपणाधीनमेव~। निरूपितानि च गुणतद्विशेषतया द्रव्याणीत्यत आह यद्वेति~॥
etc. As the ending page is broken the end is not available. Ibid. p.4 foot note No.1.} He was one of the main four pupils of the famous वासुदेव सार्वभौम. Other pupils and collagues of रघुनाथ शिरोमणि were (1) गौराङ्गदेव, (2) रघुनन्दन and (3) कृष्णानन्द. गौराङ्गदेव as known as चैतन्यप्रभु and was considered to be the incarnation of विष्णु by his devotees.\renewcommand{\thefootnote}{2}\footnote{\begin{quote}
{\qt शाके मुनिव्योमयुगेन्दुगण्ये\\
पुण्ये तिथौ फाल्गुनपौर्णमास्याम्~।\\
त्रैलोक्यभाग्योदयपुण्यकीर्ति \textendash\ \\
देवः शचीनन्दन आविरासीत्~॥}
\end{quote}
\noindent
This work was obtained by the editor of the 1st edition of the किरणावली from व्रजनाथ शिरोमणि indicating the date of गौराङ्गदेव Vide ibid Intro. p.32. Foot Note No.1.} रघुनन्दन was a scholar in the धर्मशास्त्र while कृष्णानन्द was a scholar in the मन्त्रशास्त्र. रघुनाथशिरोमणि the famous neologician is more known as दीधितिकार by the name of his commentary दीधिति on several न्याय Works like the आत्मतत्त्वविवेक and others. मथुरानाथ तर्कवागीश of Bengal was the pupil of रघुनाथशिरोमणि as believed by some scholars. He has commented upon almost all the works of रघुनाथशिरोमणि and also on the आत्मतत्त्वविवेक of उदयनाचार्य and तत्त्वचिन्तामणि of उपाध्याय गङ्गेश. His commentaries are known as माथुरी by his own name in the realm of neologicians. Thus his commentary on the

\newpage
\noindent
गुणदीधिति of रघुनाथ शिरोमणि is known as गुणदीधितिमाथुरी.\renewcommand{\thefootnote}{1}\footnote{Vide Ibid Intro. p.4 Foot Note No.2. The work measures about 1000 श्लोकप्रमाण. It begins as follows : \textendash
\begin{quote}
{\qt कुञ्चिताधरपुटेन पूरयन्, वंशिकां प्रचलदङ्गलिपङ्क्तिः~।\\
मोहयन्नखिलवामलोचनाः, पातुकोऽपि नवनीरदच्छविः~॥

श्रीमता मथुरानाथतर्कवागीशधीमता~।\\
गुणप्रकाशविवृतेर्भावो व्याख्यायते मया~॥}
\end{quote}
निर्विघ्नं प्रारिप्सितग्रन्थपरिसमाप्तिकामनया कृतं भगवन्नमस्काररूपं मङ्गलं शिष्यशिक्षायै आदौ निबध्नाति ॐ नम इति~। अनुमानदीधितिरहस्ये प्रपञ्चितमेतत्~।

The end is not available because the editing base is broken. मथुरानाथ has also commented upo the गुणकिरणावलीं and its प्रकाश. His work is know as रहस्य.} Thus on the किरणावली we get the chain of commentaries and sub \textendash\ commentaries. (1) The commentary प्रकाश of उपाध्याय वर्धमान, (2) On it the द्रव्यप्रकाशिका of भगीरथठक्कुर, (3) One another commentary on the प्रकाश is of रघुनाथ शिरोमणि known as the गुणप्रकाशविवृति alias गुणदीधिति, (4) On it there is गुणप्रकाशविवृतिरहस्य alias गुणदीधितिमाथुरी by मदुरानाथतर्कवागीश, (5) One more commentary on the work of रघुनाथशिरोमणि by गुणविवृतिभावप्रकाशिका by रुद्रभट्टाचार्य. The work is known as गुणप्रकाशविवृतिपरीक्षा,\renewcommand{\thefootnote}{2}\footnote{Vide Ibid Intro. p.4, Foot Note No.3. The work is of about 4000 श्लोकप्रमाण. It begins as follows: \textendash

\begin{quote}
{\qt भवजलनिधौ भीमावर्ते जनस्य निमज्जतो\\
निरूपमकरालम्बो लम्बोदरप्रणयोत्सुकः~।

परमकरुणासिन्धुर्बन्धुर्वरस्य नतिप्रियो\\
हरतु दुरितं प्रारब्धेऽस्मिन् प्रतीपमुमापतिः~।

अभ्यामृशन् मृदुमृणाललताभ्रमेण\\
हस्तेन बाहुलतिकां बहुशः शिवायाः~।

आलोलकण्ठगतषण्मुखदन्ततालो\\
बालो विभावयतु भावुकमेकदन्तः~॥

विद्यानिवास \textendash\ पुत्रस्य न्यायवाचस्पतेरियम्~।\\
कृतिः कृतिधियां भूरि परितोषाय जायताम्~॥

गुणप्रकाशविवृतेरियं भावप्रकाशिका~।\\
इतः स्यादपि मन्दानां सफलोऽत्र परिश्रमः~॥}
\end{quote}

तुष्ट्यतुष्ट्योः सुखदुःखयोरीश्वरेऽसम्भवादाह प्रकाशे तुष्टिस्तुतिरिति~।

It ends as follows: \textendash

तस्मात्समवायिकारणमात्रवृत्तिरेव विभागस्तथा~। न च साक्षात्परम्परासाधारणसमवायित्वं प्रयोजकं तस्य दुर्वचत्वाद् गुरुत्वाच्च~। इति श्रीमहामहोपाध्यायश्रीविद्यानिवासभट्टाचार्यात्मजश्रीरूद्रभद्टाचार्यविरचिता गुणप्रकाशविवृतिपरीक्षा समाप्ता~।} (6) Still another

\newpage
\noindent
commentary is by रामकृष्ण\renewcommand{\thefootnote}{1}\footnote{Ibid Intro. p.5, Foot Note No.1 
 
 The MS. of this work is broken at the end so the ending sentence and the measure of the work are not available. It begins as follows : \textendash
\begin{quote}
{\qt वाणि प्रसीद करुणामयि ते नतोऽस्मि त्वं येन देवि सुतवत्यसि पुत्रिणीषु~।\\
येनोदधारि कुनिबन्धतमोऽन्धकूपे, भग्नाक्षपादकणभक्षमतं निरीक्ष्य~॥

यन्मूलमेव सुकृतानि तयोः कृतानि व्यासादयः सदसि नित्यमुदाहरन्ति~।\\
तस्याशयं गुणविवेचनमाकलय्य, ब्रूते शिरोमणिगुरोरिह रामकृष्णः~॥}
निरूपणं शाब्दबोधानुकूलब्यापारः and so on.
\end{quote}} and (7) Another also by जयराम भट्टचार्य.\renewcommand{\thefootnote}{2}\footnote{Ibid Intro. p. 5, Foot Note No. 2.

 The ending portion of this work is also broken so the end and the measure of the work both are not available. It begins as follows : \textendash
 \begin{quote}
{\qt निराकारं नराकारं जगदाकारमद्वयम्~।\\
गोपीदृगञ्जनं वन्दे निरञ्जनमहो महः~॥

धीरः श्रीजयरामोऽसौ युक्तियुक्ताभिवन्दिताम्~।\\
गुणदीधितिगूढार्थां विवृणोति सरसस्वतीम्~॥}
 \end{quote}
\noindent
" प्रकाशे एकत्र अञ्जलिमूलस्थे अन्यत्रतद्ग्रहस्ते हस्तारूण्ये न स्वाभाविकजलशुक्लताभिभवाद् " and so on.} There is also one more independent commentary on the किरणावली is the किरणावलीभास्कर by the son of बलभद्रमिश्र.\renewcommand{\thefootnote}{3}\footnote{Ibid Intro. p. 6, Foot Note No. I.

This commentary is of 2000श्लोकप्रमाण. It begins as follows : \textendash
\begin{quote}
{\qt उपदिष्टा गुरुचरणैरस्पृष्टा वर्धमानेन~।\\
किरणावल्यामर्थास्तन्यन्ते पद्मनामेन~॥

विलसद्वर्धमानापि तिरोहितदिवाकरा~।\\
सक्रलार्थप्रकाशाय न क्षमा किरणावली~॥

बलभद्रमुखाम्भोजवचनादुदयाचलात्~।\\
उदितो भास्करस्तस्मादादरेण निषेव्यताम्~॥}
\end{quote}
The end reads a follows:\textendash
\begin{quote}
{\qt यस्तर्कदुस्तरतरार्णवकर्णधारो~।\\
वेदान्तवर्त्मनिरताध्वगसार्थवाहः~।\\
श्रीपद्मनाभरचितेन दिवाकरेण\\
तुष्टोऽमुनास्तु सुकृती बलभद्रमिश्रः~॥}
\end{quote}}

\newpage
Thus we have got a chain of works on the किरणावली of उदयनाचार्य. This shows its usefulness as well as popularity. Same is the case with the other works of उदयनाचार्य. We get प्रकाश of उपाध्याय वर्धमान on the न्यायकुसुमाञ्जलि and the आत्मतत्वविवेक of उदयनाचार्य. However, we do not get any commentary of वर्धमान on the small treatise viz. लक्षणावली. There is a commentary in the MS form of the लक्षणावली by शेषशार्ङ्गधराचार्य. It is not yet published. Another commentary by केशवभट्ट on the लक्षणावली is edited by शशिनाथ झा and published by Mithila Institute Darbhanga in the year 1963. There are good many commentaries on the न्यायकुसुमाञ्जलि,. Its popularity has urged आचार्य विश्वेश्वर to translate it into Hindi. The लक्षणावली seems to be the Ist work tending somewhat towards neologic. It is a small treatise giving essence and definition of वैशेषिक elemeans. We have here added it in this edition as an appendix No. I. As this is a वैशेषिक work it was thought worthwhile to give वैशेषिक सूत्रपाठ with all available latest readings. We have edited the सूत्रपाठ accordingly and added in this edition as an appendix No.2. The सूत्रपाठ bears the readings of the सूत्रपाठ of the Mithila edition as well as that of G.O.S. edition. उपस्कार is accepted as the base of the सूत्रपाठ.

\noindent
\textbf{Acknowledgment}

In preparing this edition I am very much indebted to पू. मुनिश्री पुण्यविजयजी for allowing me to use a valuable single MS. of the किरणावली of Jaisalmer MSS. library. I am also much indebted to पू. मुनिश्रीजम्बूविजयजी who gave me considerable help and guided me in preparing the presnt edition as well as for using his valuable edition of the वैशेषिकदर्शन. I must acknowledge with thanks and due regards the suggestions given by different scholars and friends particularly great savant प्रज्ञाचक्षु पं. सुखलालजी, पं. दलसुखभाई मालवणिया and प्राध्या, श्री. अनन्तलाल ठाकुर. Prof. L. Alsdorf, my teacher Prof. R.C.Parikh of B.J.Research Institute, my late teacher पं. मयाशङ्गरशर्मा and my learned teacher in न्याय पं. आशुतोष भट्टाचार्य, for encouraging me in preparing this edition I feel highly thankful to Dr. B. J. Sandesara, the Director of Oriental Institute, Baroda, Dr. Umakant P. Shah, Deputy Director and Head of the Ramayana Department, of the Oriental Institute, Baroda and Prof. J. P. Thaker Curator of the

\newpage
\noindent
MSS Department, Oriental Institute, Baroda. I should not forget here the Manager of the University Press Shri Ramanbhai Patel and his colleagues for their co \textendash\ operation in printing of the edition. With all this help if a few printing mistakes or any other mistakes are found by scholars in this edition, the responsibility is mine for which learned friends will excuse me.

\begin{quote}
{\qt गच्छतः स्खलनं क्वापि भवत्येव प्रमादतः~।\\
हसन्ति दुर्जनास्तत्र समादधति सज्जनाः~॥}
\end{quote}

प्रियन्तां गुरवः~।

\begin{center}
\rule{0.2\linewidth}{0.5pt}
\end{center}

\newpage
\thispagestyle{empty}
\begin{center}
{\Large ॥~भूमिका~॥}
\end{center}

\begin{quote}
{\qt यच्चितिकणिकासङ्गो जगच्चितिविधायकः~।\\
तद्वन्दे परमं ब्रह्म सच्चिदानन्दलक्षणम्~॥}
\end{quote}

ख्रिस्तीये १९५० तमे वर्षेऽहं " शेठ भो. जे. अध्ययन  \textendash\ संशोधन \textendash\ विद्याभवनाध्यक्षैर्जैंसलभेरनगरे तत्रभवद्भिर्मुनिभिः पुण्यविजयैः क्रियमाणस्य हस्तलिखितपुस्तकभाण्डागारस्य संशोधनकार्यस्य यत्किञ्चित्साहाय्यार्थं प्रेषित आसम्~। तदानीं न्यायवैशेषिकविषयाणि हस्तलिखितपुस्तकान्यवलोकयतो म {\knu उदयनाचार्यरचितकिरणावल्या} एकं हस्तलिखितपुस्तकं दृष्टिपथं समायातम्~। प्रथममुद्रितकिरणावल्या सह तुलनया ज्ञातं यदिदं हस्तलिखितपुस्तकं मुद्रितपुस्तकादधिकभागं भजते~। अधिकपरीक्षणेन विज्ञातं यदस्मिन् हस्तलिखितपुस्तके किरणावल्या गुणप्रकरणं सम्पूर्णं वर्तते~। अद्यावधि न केनाप्येतावद्भागपर्यन्तं हस्तलिखितपुस्तकं प्राप्तमत एव मुद्रितपुस्तकं यद्यप्यष्टादर्शपुस्तकाधारितं तथाप्यपूर्णमेव~। प्रथमं तत् पण्डितवर्यैः {\knu म. म. विन्ध्येश्वरीप्रसादमहोदयैः ढुण्ढिराजशास्त्रिभिश्च} मिलित्वा संशोध्य सम्पादितं प्रकाशितं च १९१९ तमे ख्रिस्तीयवर्षे~। अस्मिन्प्रथमप्रकाशिते किरणावलीग्रन्थे {\knu "न शब्दोऽर्थप्रत्यायक इति स्ववचनविरोधी"} एत्येतत्पर्यन्त एव भागः~।\renewcommand{\thefootnote}{१}\footnote{द्रष्टव्या प्रथमप्रकाशितकिरणावल्याः पृ. ३४०~। इयमेव किरणावली स्व. म. म. विन्ध्येश्वरीप्रसादद्विवेदीमहोदयैः ढुण्ढिराजशास्रिभिश्च मिलित्वा सम्पादिता, ख्रिस्तीये १९१९ तमे च वर्षे बनारससंस्कृत सिरीज इति ग्रन्थमालायां प्राकाश्यं नीता~।}

उपलब्धो नवीनो भाग एव सम्पादनं कृत्वा कस्मिंश्चित्संशोधनसामयिके प्रकाशनीय इति तदानीं विमृष्टं, किन्तु मुद्रितकिरणावल्या अपि तदानीमेव दुष्प्रापणीयत्वात् सम्पूर्णकिरणावल्याः पुनः सम्पादनं विधेयमिति विचारितम्~। समर्थितश्चायं विचारो विदुषा मित्रेण {\knu साण्डेसरा} इत्युपाह्वेन {\knu भोगीलालमहीदयेन}~। तदानीमेव तत्र जैसलमेरभण्डारे {\knu प्रशस्तप्रादभाष्यस्या}प्येमकमसम्पूर्णं ताडपत्रीयं हस्तलिखितपुस्तकं समायातं दृष्टिगोचरम्~। तत्सहायेन तथैव अहमदाबादनगरे देवशापाडास्थितस्य हस्तलिखितपुस्तकभाण्डागारस्य प्रशस्तपादभाष्यकिरणावल्याश्च हस्तलिखितपुस्तकसाहाय्येन प्रशस्तपादभाष्यं किरणावली च द्वेऽपि सम्पादिते~। प्रकाशनं चास्य न्यग्रोधपुरीयप्राच्यविद्यामन्दिराधिकारिभिः सानुग्रहं स्वीकृतम्~। प्रथममुद्रितकिरणावलीपुस्तके {\knu उदयानाचार्य}विरचिता लक्षणावली समाविष्टाऽऽसीदतोऽत्रापि सा परिशिष्ट १ कृत्वा समाविष्टा~। तथैव तत्र वैशेषिकसूत्रपाठोऽपि समाविष्ट आसीत्~। अतोऽत्रापि शङ्करमिश्रकृतोपस्कारस्वीकृतवैशेषिकसूत्रपाठं प्रधानीकृत्य

\newpage
\noindent
तुलनात्मकः सूत्रपाठः परिशिष्ट २ इति कृत्वा समाविष्टः~। परिशिष्टद्वयं विदुषामुपयोगीति मत्वाऽत्र समाविष्टम्~।

\vspace{-3mm}
\begin{center}
{\knu न्याय \textendash\ वैशेषिकपरम्परयोः सङ्क्षिप्तमितिवृतम्~।}
\end{center}

भारतीयदर्शनपरम्परासु न्यायवैशेषिकदर्शने समानतन्त्रत्वं भजेते इति सुविदितमेव विदुषाम्~। समानतन्त्रत्वादिमे द्वेऽपि परम्परे परस्परं पदार्थानङ्गीकुर्वाते~। प्रमाणसङ्ख्यां विहाय परस्परपदार्थाङ्गीकारे नैतयोर्मतवैमत्यम्~। प्रत्यक्षानुमानं द्वे एव प्रमाणे इति वैशेषिकाणां मतम्~। "प्रत्यक्षानुमानोपमानशब्दाः प्रमाणानि" इति चत्वारि प्रमाणानि नैयायिकैस्स्वीकृतानि~। एवं न्यायदर्शने प्रतिपादिता तर्कपद्धतिर्वैशेषिकैर्विना विरोधमङ्गीकृता~। तथैव वैशेषिकाभिमतपदार्थानां स्वीकारो नैयायिकैरपि कृतः~। इत्थं प्राचीनकालादेवेमे परम्परे सङ्गच्छेते~। स्वीकृतमेतच्च न्यायभाष्यकृता वात्स्यायनमुनिना~।\renewcommand{\thefootnote}{२}\footnote{द्रष्टव्याऽत्रैवास्मदाङ्ग्लभाषाप्रस्तावना पृ. ३ पा. टि. १.}

परम्पराद्वयमिदं शैवमिति {\knu स्व. डॉ. भाण्डारकरमहोदयस्य} मतम्~। आधारश्चास्मिन् विषये {\knu षड्दर्शनसमुच्चय}कर्ता {\knu हरिभद्रसूरिः~। षड्दर्शनसमुच्चये हरिभद्रसूरि}र्नैयायिकान् शैवान् वैशेषिकांश्च पाशुपतान् वर्णयामास~।\renewcommand{\thefootnote}{३}\footnote{द्रष्टव्या डॉ. भाण्डारकरमहोदयसम्पादिततर्कभाषा प्रस्तावना पृ. २ \textendash\ ७.} वायुपुराणमाश्रित्य डॉ. भाण्डारकरमहोदयो {\knu गौतमर्षिं न्यायशा}स्त्रस्य कणादर्षिं च {\knu वैशेषिकशा}स्त्रस्य प्रस्थापकत्वेन मन्यते~।\renewcommand{\thefootnote}{४}\footnote{द्रष्टव्या तदेव. \textendash\ पृ. ७.} किन्तु पुराणानामाधार इतिहासविषये कियान् प्रामाणिक इत्यत्र विदुषां वैमत्यमत एतादृशा आधारास्सदैव चिन्तनीयाः~।

इदं तु निश्चितप्रचं यत्प्राचीनसमयादेव द्वे अपीमे शास्त्रे परस्परं सम्बद्धे बभूवतुः~।\renewcommand{\thefootnote}{५}\footnote{द्रष्टव्याऽत्रेवास्मदाङ्ग्लभाषाप्रस्तावना पृ. ४ पा. टि. १.} {\knu हरिभद्रानुसारी राजशेखर}सूरिरपि स्वकीये षड्दर्शनसमुच्चये\renewcommand{\thefootnote}{६}\footnote{षड्दर्शनसमुच्चय पृ. ८. ९. अत्रैवास्मदाङ्ग्लभाषाप्रस्तावनायामुद्धृताः श्लोकाश्च, प्रस्तावना पृ. ४.} न्यायवैशेषिकयोरितिवृत्तं वर्णयन् वैशेषिकान् पाशुपतानाचाष्टे\renewcommand{\thefootnote}{७}\footnote{द्रष्टव्यं तदेव पृ. ११.}~। अनयोः शास्त्रयोः प्रस्थापकविषयेऽपि स्पष्टमेव तस्य षड्दर्शनसमुच्चये निर्दिष्टं यदुलूकरूपेण शिवेनैतत्मतं कणादस्य मुनेः पुरः प्रकथितमितीदं मतमौलूक्यमुच्यते~। एवमेवाक्षपादेन प्रस्थापितं न्यायमतं युक्त्यनुसारीति {\knu यौगिकमाक्षपादम्} इत्युच्यते~। द्वयमपि मतं प्रायस्तुल्यमित्यपि करथितवान्~।\renewcommand{\thefootnote}{८}\footnote{द्रष्टव्यं तदेव पृ. १२~।} इदं राजशेखरसूरिवर्णितं वृत्तमपि पौराणिकमिति विद्वद्भिश्चिन्तनीयम्~।

\newpage
न्यायशास्त्रप्रस्थापकस्य {\knu गौतमर्षे}र्निश्चितः कालोऽद्यावधि न निर्णीतः~। केचिद्विदांसस्तु {\knu गौतमाक्षपादौ} द्वौ भिन्नावेव बभूवतुरिति मन्यन्ते~। पुराणैस्तु गौतम एव न्यायशास्त्रप्रवक्ताऽङ्गीक्रियते~।\renewcommand{\thefootnote}{९}\footnote{गौतमेन तथा न्यायम् पद्म. पु. उ. खण्ड अ. २६३; गौतमः स्वेन तर्केण इत्यादि स्कन्द पु. कालिका ख. अ. ७६~।} {\knu न्यायवृत्तिकारो विश्वनाथो गौतमं} न्यायसूत्राणां कर्तारं वर्णयति~।\renewcommand{\thefootnote}{१०}\footnote{एषा मुनिप्रवरगोतमसूत्रवृत्तिः \textendash\ न्यायसूत्रवत्तेर्मङ्गलम्~।} भाष्यकारो {\knu वात्स्यायनमुनिर्वार्तिककार उद्योतकरः, तात्पर्यटीकाकृद्वाचस्पतिमिश्रः}, न्यायमञ्जरीकर्ता जयन्तभट्टश्च ऋषि{\knu मक्षपादं} न्यायसूत्राणां कर्तारं ज्ञापयन्ति~।\renewcommand{\thefootnote}{११}\footnote{द्रष्टव्यमत्रैवाङ्ग्लभाषाप्रस्तावना पृ. ५. पा. टि ६.} महाकवि{\knu भास}विरचिते {\knu प्रतिमा}नाटकस्य पञ्चमाङ्के " भोः काश्यपगोत्रोऽस्मि~। साङ्गोपाङ्गं वेदमधीये, मानवीयं धर्मशास्त्रं, माहेश्वरं योगशास्त्रं, बार्हस्पत्यमर्थशास्त्रं {\knu मेधातिथेर्न्यायशास्त्रं} प्राचेतसं श्राद्धकल्पं च " इत्युल्लेखेन मेघातिथिं न्यायशास्त्रकर्तारमनुमापयति~। डॉ. दासगुप्तामहोदयः स्वकीये भारतीयदर्शनेतिहासे ' गौतमेन मेघातिथेर्न्यायशास्त्रमधीतमिति निरूपयति~। तन्मतेन गौतमो नैतिहासिकः पुरुषः किन्तु न्यायसूत्राणामवितथः कर्ताऽऽक्षपाद एव ~।\renewcommand{\thefootnote}{१२ \textendash\ १३}\footnote{तदेव पृ. ६. पा. टि. २.} किन्तु स्व. म. म. शतीशचन्द्रविद्याभूषणमहोदयो {\knu मेधातिथिंगौतम}मैतिहासिकपुरुषत्वेन मन्यते~। मेधातिथिर्गौतम एवान्वीक्षिक्या प्रस्थापकस्तन्महोदयस्य मतेन भिन्न {\knu एवाक्षपादो गौतमः}$^{\hbox{१३}}$~। आचार्यविश्वेश्वरमहोदयो गौतमाक्षपादौ भिन्नौ मन्यते~। स तु मेधातिथिगौतमं न्यायदर्शनकर्तारं मन्यते~।\renewcommand{\thefootnote}{१४}\footnote{हिन्दी तर्कभाषा \textendash\ आचार्य विश्वेश्वरसम्पादिता प्रस्तावना पृ.२५ \textendash\ २७.} {\knu फणीभूषणतर्कवागीशमहोदयो मेधातिथिगौतममक्षपादं} चाभिन्नपुरुषं गणयति\renewcommand{\thefootnote}{१५}\footnote{स्कन्द पु. महेश्वरखण्ड कुमार ख. अ. ५५. श्लो. ५.}~। तन्मतेन रामायणे प्रसिद्धोऽहल्यापतिर्गौतम एव गौतमो न्यायसूत्रकर्ता~। तन्महोदयमतम् \textendash

\begin{quote}
{\qt अक्षपादो महायोगी गौतमाख्यो महामुनिः~।\\
गोदावरी समानेता अहल्यायाः पतिः प्रभुः~॥}
\end{quote}

\noindent
इति स्कन्दपुराणोक्तावाश्रितम्~।

सर्वाण्यप्येतानि विदुषां मतानि न नो निश्चितनिर्णयं प्रति नयन्ति~। अन्येषां विदुषामप्यनेकान्यनुमानानि सन्ति किन्तु न तान्यस्मिन् विषये स्पष्टतया निर्णायकानि~। अतो गौतमोऽक्षपादो मेधातिथिश्चाभिन्ना भिन्ना वा, कश्च तेषां निश्चितः काल इति विषये नाद्यापि कस्यचिद्विदुषो प्रामाणिको निर्णयः~। तथाप्येतानि प्रमाणान्याश्रित्य यथाकथंचित् ख्रिस्तीयवर्षारम्भात्पञ्चशताब्दीपूर्वमस्य न्यायसूत्रकर्तुः समयोऽनुमातुं शक्येत~। एवं न्यायसूत्रभाष्यकृ\textendash

\newpage
\begin{sloppypar}
\noindent
द्वात्स्यायनस्य समयोऽपि न पूर्णतया निश्चितः~। जैनाचार्यो {\knu हेमचन्द्रः} स्वीयेऽ{\knu भिधानचिन्तामणौ} वात्स्यायनस्य पक्षिलस्वामी, कौटिल्यः, चाणक्य, इत्यादीन्यनेकानि नामानि गणयति~।\renewcommand{\thefootnote}{१६}\footnote{अभि. चि. मर्त्यकाण्ड श्लो. ५१७  \textendash\ १८ अत्रैव आङ्ग्लभाषाप्रस्तावनायां पृ ६. पा. टि. ५} पुरुषोत्तममहोदयः त्रिकाण्डशेषेऽप्येतानि नामानि परिगणयति~।\renewcommand{\thefootnote}{१७}\footnote{त्रिकाण्डशेषकोषे \textendash\ ब्रह्मवर्गः; अत्रैवाङ्ग्लभाषाप्रस्तावनायां पृ. ७. पा टि. १} पुस्तकद्वयेऽपि विष्णुगुप्तः, द्रामिलः, अङ्गुलोंऽशुलो वा इत्यन्यानि नामान्यपि परिगणितानि~। किन्त्वेतानि सर्वाणि भाष्यकर्तुर्वात्स्यायनस्यैवाभिधानानि वा एते सर्वे भिन्नाः पुरुषा वा इति विषये निर्णायकं प्रमाणं नोपलभ्यते~। इतिहासविदो भाष्यकर्तुर्वात्स्यायनस्य कालं ख्रिस्तीयाब्दस्य द्वितीयशताब्दीमनुमन्यन्ते~। न्यायभाष्यस्य व्याख्यारूपं वार्तिकं पाशुपताचार्येण भारद्वाजापरनामधेयेनोद्योतकरेण व्यधायि~। तेन स्ववार्तिके भाष्यकारमतखण्डनकर्तुर्बौद्धतार्किकस्य\renewcommand{\thefootnote}{१८}\footnote{न्यायवार्तिक \textendash\ मङ्गल श्लोकः~।} {\knu दिङ्गनागस्य} मतं खण्डितमिति {\knu वार्तिककार उद्योतकरो} दिङ्नागात्पश्चाद्भावीति निश्चितमेव~। दिङ्नागस्य समयः ख्रिस्तीया षष्ठी शताब्दीति गण्यते~। अतो वार्तिककारस्तत्पश्चाद्भावीति स्थितम्~। वार्तिककारस्य मतानि बौद्धतार्किकेण {\knu धर्मकीर्तिना} खण्डितानि~। तन्मतानि {\knu तात्पर्यटीकायां} विदुषा {\knu वाचस्पतिमिश्रेण} तथैव {\knu तात्पर्यपरिशुद्धौ} तत्कर्त्रा {\knu उदयनाचार्येण} च खण्डितानि~।
\end{sloppypar}

\begin{sloppypar}
इत्थं न्यायसूत्रेषु न्यायभाष्यादारभ्यानेकाः सुन्दरा व्याख्योपव्याख्याश्चोपलभ्यन्ते~। अन्यैर्विद्वद्भिरपि न्यायशास्त्रविषयका ग्रन्था रचिताः~। तद्यथा {\knu जयन्तभट्ट}रचिता {\knu न्यायमञ्जरी}, उदयनाचार्यविरचिता {\knu प्रबोधसिद्धिर्न्यायपरिशिष्टे}त्यपरनामधेया, विदुषाऽ{\knu निरुद्ध}विरचिता न्यायभाष्य \textendash\ वार्तिक \textendash\ तात्पर्यटीकासु टिप्पणस्वरूपा {\knu विवरणपञ्चिका}~। अयमनिरुद्धः साङ्ख्यसूत्रटीकाकर्तुरनिरुद्धाद्भिन्नः~। अनेन तात्पर्यपरिशुद्धिः स्वटिप्पणे विवरणपञ्चिकायां न स्पृष्टाऽतोऽनुमीयतेऽयमुदयनाचार्यात्पूर्वभावीति~। तमनुसरता विदुषा {\knu श्रीकण्ठेन} भाष्य \textendash\ वार्तिकतात्पर्यटीका \textendash\ तात्पर्यपरिशुद्विषु चतुर्षु ग्रन्थेषु व्याख्याऽऽरब्धा किन्तु सा न समाप्तिं प्राप्ता~। श्रीकण्ठस्यापूर्णां रचनां दृष्ट्वा सूरिणाऽ{\knu भयतिलकेन} पूर्वोक्तेषु चतुर्षु न्यायमहाग्रन्थेषु पञ्चममिदं व्याख्यानमिति कृत्वा {\knu न्यायपञ्चप्रस्थानं न्यायालङ्कारटिप्पण}मित्यपरनामधेयं टिप्पणं विदधे~। {\knu उपाध्यायवर्धमानेन} तात्पर्यपरिशुद्धौ {\knu परिशुद्धिप्रकाश}व्याख्याऽलेखि~। {\knu द्वितीयवाचस्पतिमिश्रेण न्यायसूत्रोद्धार} इति ग्रन्थो रचितः~। {\knu विश्वनाथपञ्चाननेन न्यायसूत्रवृत्ती रचि}ता~। गोविन्दखन्ना इत्यनेन विदुषा {\knu न्यायसङ्क्षेपो} विदधे, राधामोहनगोस्वामिना च {\knu न्यायसूत्रविवरणं} कृतम्~। तथैव विदुषा कृष्णकान्तवागीशेन {\knu गौतमसूत्रसंदीपो} रचितः~। अन्या अपि टीकोपटीका न्यायसूत्रमुद्दिश्य सञ्जाताः~।
\end{sloppypar}

\newpage
केषांचिद्विदुषां मतेन यद्यपि न्यायपरम्परातो वैशेषिकपरम्परा प्राचीनतरा\renewcommand{\thefootnote}{१९}\footnote{आथल्येबोडाससम्पादित तर्कसंग्रह प्रस्तावना पृ. ३५.} तथापि न्यायसूत्रसदृशी व्याख्योपव्याख्यापरम्परा वैशेषिकसूत्राणां नोपलभ्यते~। दर्शनान्तरेषूपलभ्धैर्वैशेषिकमतसन्दर्भैरेतप्तु ज्ञायते यदिदं शास्त्रं सर्वेषामपि नास्तिकानामास्तिकानां च दार्शनिकानां सुपरिचितं स्वधीतं चासीत्~। आस्तिकपरम्परायां तु "{\knu काणादं पाणिनी}यं च सर्वशास्त्रोपकारकम्" इति प्रसिद्धैवोक्तिः~। इदं च शास्त्रं स्थापकस्य {\knu कणाद}र्षेर्नाम्ना {\knu काणादमौलूक्यं} वाऽभिधीयते~। ऋषेर्गोत्रनाम्ना {\knu काश्यपी}यमित्यप्युच्यते~। कथङ्कारमस्य शास्त्रस्य {\knu वैशेषिक}मित्यभिधानमित्यत्रानेकानि भिन्नानि मतानि~।

{\knu हरिभद्रसूरेः षड्दर्शनसमुच्चयस्य} व्याख्याकृता गुणरत्नसूरिणा\renewcommand{\thefootnote}{२०}\footnote{हरिभद्रसूरेः षड्दर्शनसमुच्चये गुणरत्नसूरिटीका पृ. २३.} व्युत्पद्यते "नित्यद्रव्य [ वृत्तयोऽन्त्याविशेषाः] विशेषा एव वैशेषिकं विनयादिभ्यः स्वार्थे इकच्~। तद्वैशेषिकं विदन्त्यधीयते वा तद्वेत्ति अधीते इत्यपि वैशेषिकाः तेषामिदं {\knu वैशेषिकम्}~। उदयानाचार्यस्तु किरणावल्यां" विशेषो व्यवच्छेदस्तत्त्वनिर्णयः तेन व्यवहरतीति "वैशेषिक इत्यस्यार्थं प्रख्यापयति~। दुर्वेकमिश्रो धर्मोतरप्रदीपे" द्रव्यगुणकर्मसामान्यविशेषसमवायात्मकैः पदार्थविशेषैर्व्यवहरन्तीति वैशेषिकाः~। रूढेश्चाभ्युपगतकणादशास्त्रा एवोच्यन्ते~। अथवा षट्पदार्थ प्रतिपादकतया विशेष्यते तदन्यस्माच्छास्त्रादिति विशेषः {\knu काणादं} शास्त्रं विवक्षितम्~। तद्विदन्त्यधीयते वेति वैशेषिकाः~।\renewcommand{\thefootnote}{२१}\footnote{धर्मोतरप्रदीप : \textendash\ पृ. २४०.} मतमिदमनुसृतं वि महोदयेन चीनप्रदेशविदुषा~।\renewcommand{\thefootnote}{२२}\footnote{'व' प्रणीतं वैशेषिकदर्शनं प्रस्तावना पृ. ४.} केचिद्विद्वांसोऽस्य शास्त्रस्य श्रुतिविरुद्धतर्कप्रतिपादकत्वं बौद्धमतप्रभावित्वं च प्रतिपादयन्ति;\renewcommand{\thefootnote}{२३}\footnote{द्रष्ठव्यं " प्राइमर ऑफ इण्डिअन लोजिक " इति नाम पुस्तकम् प्रस्तावना पृ. १०.} किन्तु तेषामिदं मन्तव्यं वितथं दृश्यते वैदिकसंहितानां ब्राह्मणग्रन्थानामुपनिषदां चावलोकनेन~। वस्तुतस्तूपनिषस्तु श्वेताश्वतरसदृश्य उपनिषदो वैशेषिकपरम्परापरा दृश्यन्ते~।\renewcommand{\thefootnote}{२४}\footnote{'तु' श्चेताश्वतर \textendash\ ३. ३; अत्रेवाङ्ग्लभाषाप्रस्तावना पृ. ५. पा. टि. १} {\knu महाभारते}ऽपि वैशेषिकपारिभाषिकशब्दानामणुपरमाणुवैशेषिकगुणादीनामनेकेषु स्थलेषूपयोगो दृश्यते~।\renewcommand{\thefootnote}{२५}\footnote{वै. द. अनन्तलालशर्म्मण आङ्ग्लभाषाप्रस्तावना पृ. ८.}

इत्थं वैशेषिकपरम्परायाः प्राचीनत्वेऽपि वैशेषिकसूत्राणां न्यायभाष्यसदृशं प्राचीनं प्रामाणिकं च भाष्यं नोपलभ्यते~। भाग्यवशात् साम्प्रतं संशोधनेन किञ्चित्प्राचीनं टीकाद्वयं समुपलब्धम्~। तयोरेका व्याख्या विदुषा सुहृदा {\knu ठाकुर इत्युपाह्वेनानन्तलाल शर्म्मणा} सम्पादिता प्रकाशिता च {\knu दरभङ्गास्थमिथिलासंशोधनमन्दिरेण} ख्रिस्तीय १९५७ 

\newpage
\noindent
तमेऽब्दे, अज्ञातकर्तुकेयं व्याख्या~। द्वितीया तु {\knu चन्द्रानन्दवृत्ति}स्तत्रभवता विदुषा {\knu मुनिना जम्बूविजयमहोदये}न सम्पादिता प्रकाशिता च {\knu गायकवाडप्राच्यग्रन्थमालायां} ख्रिस्तीय १९६१ तमे वर्षे~। इदं टीकाद्व्यं {\knu शङ्करमिश्रस्योपस्कारात्} प्राचीनतरमिति निश्चितम्~।\renewcommand{\thefootnote}{२६}\footnote{चन्द्रानन्दवृत्तिसहित वै. सू. प्रस्तावना: पृ. १२  \textendash\ १३.} द्वितीयटीकायाश्चन्द्रानन्दवृत्तेः प्रस्तावनायां विद्वन्मुनिना स्वसम्पादित{\knu द्वादशारनयचक्र}माश्रित्याधोनिर्दिष्टा वैशेषिकटीकापरम्परा प्रदर्शिता~।\renewcommand{\thefootnote}{२७}\footnote{तत्रैव प्रस्तावना. पृ. ५  \textendash\ ११.} तद्यथा

\begin{center}
\includegraphics[width=0.75\linewidth]{latex/c.JPG}
\end{center}

अन्यच्च वैशेषिकसूत्रेष्वात्रेयप्रणीतं भाष्यमप्यासीत्~।\renewcommand{\thefootnote}{२८}\footnote{तदेव आङ्ग्लभाषाप्रस्तावना पृ. १२  \textendash\ १३.} महाकविर्मुरारिः स्वकीयेऽनर्घराघवनाटके वैशेषिकभाष्यकर्तारं रावणनामानं विद्वासं निर्दिशति~। अयं रावणो लङ्केशो रावण इति मुरारेर्भ्रान्तिरेव~। वैशेषिकसूत्रभाष्यकृद्रावणस्तु भिन्न एव~। पद्मनाभमिश्रेणापि किरणावलीभास्करे भाष्यकृद्रावणस्य नाम निर्दिष्टम्~। इत्थं ब्रह्मसूत्रशाङ्करभाष्यटीकायां रत्नप्रभायां गोविन्दप्रभुणा, प्रकटार्थविवरणे चानुभूतिस्वरूपाचार्येणापि तन्निर्दिष्टम्~।\renewcommand{\thefootnote}{२९}\footnote{तदेव परिशिष्ट. ६ पृ. १५० पा टि  \textendash\ १.} कटन्दीटीकायाः कर्ताऽयमेव रावण इति मुरारिवचसाऽनुमीयते~।\renewcommand{\thefootnote}{३०}\footnote{तदेव संस्कृतप्रस्तावना पृ. ७} प्रदर्शितकोष्टकानुसारं वैशेषिकसूत्राणां वाक्यनाम्नी व्याख्याऽऽसीत्~। वाक्यस्योपरि भाष्यं तस्य च प्रशस्तमतिना टीकेति व्याख्यात्रयमासीत्~। प्रशस्तपादः प्रशस्तमतिश्च भिन्नावेकस्य पुरुषस्य वाऽभिधानं द्वयमिति चिन्तनीयमेतत्~। एता सर्वा अपि कृतयो न केनापि स्वरूपेणोपलभ्यन्ते~। पूर्वग्रन्थेष्वनिर्दिष्टं यट्टीकाद्वयं प्राप्तं सम्पादितं च तत्तूक्तमेव~। इदं सम्पादितं टीकाद्वयं शङ्करमिश्रोऽपि स्वकीयोपरकारे न निर्दिशति~। अतोऽनुमीयते यत्प्राचीनतरमपीदं टीकाद्वयं तदानीं शङ्करमिश्रस्य दृष्टिपथं नायातम्~। वैशेषिकसूत्राणां समयस्य विषये विस्तरार्थं जिज्ञासुभिर्विदुषोऽनन्तलालशर्मणः प्रस्तावनाऽवश्यं द्रष्टव्या~।\renewcommand{\thefootnote}{३१}\footnote{द्रष्टव्या वै. द. अनन्तलाल ठाकुर सम्पादितम्, आङ्ग्ललभाषा. प्रस्ता. पृ.९  \textendash\ १५} प्रशस्तभाष्येण वैशेषिकसूत्राणामध्ययनमधरीकृतमित्यपि तत्र कथितमेव~।

\newpage
प्रशस्तपादभाष्यस्याध्ययनमियत्प्रसृतं येन विद्वद्भिरतस्यैव टीकोपटीका रचिता न वैशेषिकसूत्राणाम्~। अधुना तु संशोधनेन यट्टीकाद्वयं प्राप्तं तत्तु दर्शितमेव~। तृतीयव्याख्या या प्रथिता वर्तते सा {\knu शङ्करमिश्रस्योपस्कारः~।} अस्मिन्नुपस्कारेऽपि टीकाद्वयमुपलभ्यते~। {\knu (१) जयनारायणतर्कपञ्चाननकृता वृत्तिः, (२) पञ्चाननतर्करत्नकृत उपस्कारपरिष्कारश्च~।} इदं टीकाद्वयं नव्यन्यायप्रभावयुतम्~। {\knu चन्द्रकान्ततर्कालङ्कार}कृते वैशेषिकसूत्रभाष्ये तु वेदान्तस्य प्रभावो दृश्यते~। {\knu गङ्गाधरवैद्य}कृतभारद्वाजवृत्तिभाष्ये वैशेषिकसूत्रभाष्ये साङ्ख्यायुर्वेदादिशास्त्राणां प्रभावो दृष्टिगोचरो भवति~। अतो वैशेषिकसूत्राणां प्राचीनपरम्पराया ज्ञानार्थं संशोधनादुपलब्धं टीकाद्वयमभ्यसनीयम्~। तत्र चन्द्रानन्दवृत्तिस्तु प्रशस्तपादकृतस्य षदार्थसङ्ग्रहस्य परिचयं निर्दिशति~। चन्द्रानन्दवृत्तावज्ञातकर्तृकटीकायां च न वैशेषिकेतरस्यान्यस्य प्रभावो रचयित्रोः पूर्वग्रहो वा दृश्यते~।\renewcommand{\thefootnote}{३२}\footnote{जिज्ञासुभिर्द्वावपि ग्रन्थौ दर्शनीयौ~।}

अधुनैतावत्यः कृतयः समुपलभ्यन्ते किन्तु यत्तु कथितं प्रशस्तपादभाष्येण वैशेषिकसूत्राणामध्ययनमेतावदधरीकृतमासीद्यद् वैशेषिकपरम्पराविद्धिः पण्डितैर्वैशेषिकपरम्पराध्ययनार्थं प्रशस्तपादभाष्यस्य टीका रचिताः~। एवं वैशेषिकविचारधाराया अध्ययनं प्रशस्तपादद्वारैवाधिकं प्रसृतम्~। प्रशस्तपादभाष्यं तु न वैशेषिकसूत्राणां न्यायभाष्यसदृशं भाष्यं किन्तु स्वतन्त्रैव वैशेषिककृतिर्यत्र सर्वेऽपि वैशेषिकराद्धान्ता व्यवस्थितरूपेण प्रतिपादिताः~। अस्याः कृतेर्लोकप्रियत्वात् अस्यैव प्राचीनाश्चतस्रष्टीकाः प्रथिताः~। (१) {\knu व्योमशिवाचार्यस्य व्योमवती}, (२) {\knu श्रीवत्साचार्य}स्य लीलावती,\renewcommand{\thefootnote}{३३}\footnote{टीकेयं वल्लभप्रणीताया न्यायलीलावत्या भिन्ना~।} (३) {\knu उदयनाचा}र्यस्य किरणावली, (४) {\knu श्रीधरभट्ट}कृता {\knu न्यायकन्दली} च~। आसु टीकासु {\knu लीलावती} केनापि स्वरूपेण नोपलभ्यते~। उदयनाचार्येण किरणावल्यां श्रीधरभट्टेन च न्यायकन्दल्यां तत्र तत्राचार्यमतस्योल्लेखेन तेषां मतानां च व्योमवत्यां दर्शनेनानुमीयते यद् व्योमवत्युपलब्धासु टीकासु प्राचीनतमेति~। तेषां मतेन वैशेषिकाः शब्दप्रमाणमप्यङ्गीकुर्वते~।

वैशेषिकस्याध्ययनं न केवलं वैदिकदर्शनानुयायिभिरेव कृतमपि तु खण्डनप्रयोजनेन श्रुतिविरोधिभिरितरदर्शनानुयायिभिरपि गभीरतया कृतम्~। बौद्धग्रन्थेषु {\knu लङ्कावतारसूत्रेऽश्वघोषस्य सूत्रालङ्कारे, नागार्जुनस्य रत्नावल्यां च} वैशेषिकमतखण्डनं दृश्यते~।\renewcommand{\thefootnote}{३४}\footnote{आलङ्ग्लभावाप्रस्तावना पृ. २, वै. सू. सं. जम्बूविजयमुनिः~।} एवमेव नागार्जुनः स्वकीय {\knu एकश्लोकशास्त्रे उलूकशास्त्रनाम्ना} वैशेषिकं निर्दिशति~।\renewcommand{\thefootnote}{३५}\footnote{तदेव पृ. ९} लोकायतिकानां {\knu जयराशि}कृते तत्त्वोपप्ल्वसिंहेऽपि वैशेषिकमतं खण्डितम्~।\renewcommand{\thefootnote}{३६}\footnote{तदेव पृ. २} जैनास्तु न केवलं खण्डन\textendash

\newpage
\noindent
मुद्दिश्यैव न्यायवैशेषिकशास्त्राध्ययनं विदधते किन्तु तैः न्यायवैशेषिककृतिषु प्रामाणिक्यो व्याख्या अपि रचिता~। जैनानां मतेन कस्यचिद् {\knu श्रीगुप्ताचार्य}स्य शिष्येणोलूकापरनामधेयेन रोहगुप्तेन शास्त्रमिदं प्रस्थापितमिति~। विस्तरार्थं जिज्ञासुभिर्जम्बूविजयसम्पादितवैशेषिकसूरत्रस्याङ्ग्लभाषाप्रस्तावनाऽवलोकनीया~।\renewcommand{\thefootnote}{३७}\footnote{तदेव पृ. ६ \textendash\ ९} अत्र तु विस्तरभयात्पुनरुक्तिभयाच्च न चर्च्यते~। {\knu प्रमाणसमुच्चये दिङ्नागेनापि} वैशेषिकमतं प्रस्थाप्य खण्डितम्~।\renewcommand{\thefootnote}{३८}\footnote{तदेव परिशिष्ट ७}

यथा च जैनाचार्यैर्न्यायवैशेषिकग्रन्थेषु याष्टीका रचितास्तासां नामान्येव विदुषां परिचयार्थं दीयन्ते~। (१) {\knu नरचन्द्रसूरि}णा {\knu न्यायकन्दलीटिप्पण}कम्, (२) {\knu राजशेखरसूरिणा न्यायकन्दलीपञ्जिका} च~। (३) {\knu अभयतिलकेन न्यायालङ्कारटिप्पणं पञ्चप्रस्थानाभिधेयमत्रैव} पूर्वं वर्णितम्~। (४) जिनवर्धनेन शिवादित्यस्य {\knu सप्तपदार्थ्याष्टीका जिनवर्धनी}~। (५) {\knu गुणरत्नगणिना गोवर्धनस्य तर्कभाषाप्रकाशस्य तर्कतरङ्गिणी~।} शुभविजयगणिना {\knu तर्कभाषावार्तिकम्} (७) {\knu क्षमाकल्याणेन तर्कसङ्ग्रहदीपिका}याः फक्किका~।\renewcommand{\thefootnote}{३९}\footnote{दीपिकायाः फक्किका सप्तपदार्थ्या जिनवर्धनी च मत्सम्पादिते प्रकाशिते च क्रमशः राजस्थान \textendash\ पुरातन \textendash\ ग्रन्थमालायां \textendash\ ला. द. भारतीयसंस्कृतिप्रचलितग्रन्थमालायां च~।} एवमन्या अपि जैनाचार्यप्रणीताष्टीकास्सन्ति~।\renewcommand{\thefootnote}{४०}\footnote{द्रष्टव्य \textendash\ जिनरत्नकोषः जैनभण्डारेषु स्थिता हस्तलिखितपुस्तकानां सूचयोऽन्याश्च हस्तलिखित पुस्तकसूचयः~।}

{\knu उदयनाचार्यस्य सङ्क्षिप्तमितिवृत्तम्~।}

पुरा उदयनाचार्यस्य निवासस्थानविषयेऽनेके विवादा आसन्~।\renewcommand{\thefootnote}{४१}\footnote{किरणावल्याः प्रथमावृत्ति प्रस्तावना, पृ. २६ \textendash\ २७.} इदानीं तु निश्चतमेव तस्य स्थानं {\knu "मङ्गरौनि"} इति ग्रामः~। ग्रामोऽयं मिथिलाप्रदेशान्तर्गतदरभङ्गामण्डले विद्यते~। तस्य कालविषये नास्ति सर्वेषां विदुषां विवादः तेन {\knu 'लक्षणावली}पुष्पिकायां स्वयमेवास्याः कृते रचनाकालो दर्शितः~।\renewcommand{\thefootnote}{४२}\footnote{तर्काम्बराङ्कप्रमितेष्वतीतेषु शकान्ततः~। वर्षेषूदयनश्चके सुबोधां लक्षणावलीम् कैश्चित्पुष्पिकेयं कृत्रिमाऽपि मन्यते~।} 'तस्य कृतयस्तु व्याख्या स्वरूपा अन्येषां ग्रन्थानां (१) तात्पर्यपरिशुद्धिः तात्पर्यटीकाया व्याख्या, (२) किरणावली प्रशस्तपादभाष्यटीका च~। स्वतन्त्रा कृतयस्तु, (१) न्यायकुसुमाञ्जलिः, (२) आत्मतत्त्वविवेकः, (३) लक्षणावली, (४) लक्षणमाला, (५) {\knu प्रबोधसिध्य}परनामधेयं {\knu न्यायपरिशिष्टं} च~। तात्पर्यशुद्धिं विहाय सर्वा अपि इमा कृतयः सकृत्प्रकाशिता एव~। तात्पर्यपरिशुद्धिर्यद्यपि हस्तलिखितपुस्तकरूपेण सम्पूर्णा तथापि नेयं सम्पूर्णतयाऽद्यावधि प्राकाश्यं नीता~। श्रूयते यन्महाकविना

\newpage
\noindent
{\knu श्री हर्षेण} स्वकीये खण्डनखण्डखाद्ये उदयनाचार्यस्य मतं खण्डितं यत उदयनाचार्येण श्रीहर्षस्य पिता शास्त्रार्थे राजसभायां पराजितः~। वृत्तान्तस्यास्य सम्पूर्णा प्रामाणिकताऽद्यावधि न स्थापिता~।\renewcommand{\thefootnote}{४३}\footnote{प्रथमसम्पादितकिरणावली प्रस्तावना, पृ. २६ पा. टि. १.}

\noindent
{\knu किरणावल्याष्टीकोपटीकाश्च~।}

नव्यन्यायप्रस्थापकस्य महाविदुषो {\knu गङ्गेशोपाध्यायस्या}त्मजः पितृसदृश एव मेधावी पण्डितश्च वर्षमानोपाध्यायः प्राय उदयनाचार्यस्य सर्वासु कृतिषु {\knu प्रकाश}नाम्नीं व्याख्यां विदधौ~। इत्थं तेन {\knu किरणावल्याः प्रकाशो}ऽपि रचितः~।\renewcommand{\thefootnote}{४४}\footnote{तदेव प्रस्तावना पृ. २  \textendash\ ३ पा. टि. १ अत्रैवाङ्ग्लभाषाप्रस्तावना पृ.} {\knu मेघठक्कुर} इत्यपरनामधेयेन {\knu भगीरथठक्कुरे}ण विदुषाऽस्य {\knu वर्धमानकृतप्रकाशस्य प्रकाशिका}\renewcommand{\thefootnote}{४५}\footnote{तदेव प्रस्तावना पृ. ३ पा. टि. १ आङ्ग्लभाषाप्रस्ता, पृ. १५ पा. टि. १. १४ पा. टि. २.} नाम्नी टीका किरणावलीप्रकाशस्य विदधे~। अयं {\knu भगीरथठक्कुरो वासुदेवसार्वभौम}कालिकस्य पक्षधरमिश्रस्य शिष्यः~। तेन वर्धमानोपाध्यायस्य सर्वसु कृतिषु टीका व्यधायि~। वर्धमानोपाध्यायः मिथिलाप्रदेशस्य दरभङ्गामण्डलस्य {\knu करिजन} इति ग्रामस्य निवास बभूव~। बङ्गदेशीयेन विदुषा {\knu रघुनाथशिरोमणिना}ऽपि {\knu गुणदीधिति}रित्यपरनामधेया\renewcommand{\thefootnote}{४६}\footnote{तदेव पृ. ४ पा. टि. १ आङ्ग्लभाषाप्रस्तावना पृ. १६ पा. टि. १.} {\knu गुणप्रकाश}विवृतिनाम्नी टीका वर्धमानकृतप्रकाशस्य विदधे~। {\knu रघुनाथशिरोमणिः} प्रसिद्धतार्किकविदुषो तत्र भवतो {\knu वासुदेवसार्वभौमस्य} शिष्यः~। रघुनाथस्य सतीर्थ्या अन्ये (१) {\knu गौराङ्गदेवश्चैतन्यमहाप्रभुरिति}\renewcommand{\thefootnote}{४७}\footnote{तदेव पृ. ३२ पा. टि. १ आङ्ग्लभाषाप्रस्तावना पृ. १६ पा. टि. २.} नाम्ना बङ्गेषु ख्यातः~। (२) {\knu रघुनन्दनो} धर्मशास्तरधूरीणः, (३) {\knu कृष्णानन्दो} मन्त्रशास्त्रविद्वांश्च बभूवुः~। नव्यन्यायपण्डितग्रामणी {\knu रघुनाथशिरोमणिस्तु} नैयायिकेषु {\knu दीधितिकार} इति नाम्नाऽपि ख्यातः~। {\knu तेनोदयनाचार्यस्यात्मतत्त्वविवेके}ऽपि दीधिती रचिता~।

रघुनाथशिरोमणिशिष्येण बङ्गदेशनिवासिना {\knu मथुरानाथतर्कवागीशेन} शिरोमणिग्रन्थेषु {\knu माथुरीति} स्वनाम्ना प्रसिद्धाष्टीका विदधिरे~। तेनापि {\knu गङ्गेशोपाध्यायस्य तत्त्वचिन्तामणा}वुदयनाचार्यस्यात्मतत्त्वमिवेके च माथुरी व्यधायि~। किरणावल्या गुणदीधितौ {\knu गुणदीधितिमाथुरी} विद्यते~।\renewcommand{\thefootnote}{४८}\footnote{ततदेव पृ. ४ पा. टि. २ आङ्ग्लभाषाप्रस्तावना पृ. १७ पा. दि. १.} इत्थं किरणावल्याष्टीकोपटीकासङ्क्षेपे (१) वर्धमानस्य प्रकाशः, (२) प्रकाशस्य प्रकाशिका भगीरथ \textendash\ ठक्कुररचिता, (३) प्रकाशिकाया गुणप्रकरणे रघुनाथशिरोमणेर्गुणदीधितिः, (४) तस्याश्च मथुरानाथकृता गुणदीधितिमाथुरीति परम्परा~। दीधितेरन्याष्टीका {\knu रुद्रभट्टाचार्य \textendash\ }

\newpage
\noindent
रचिता {\knu गुणविवृतिभावप्रकाशिका} विद्यते~। तस्या {\knu 'गुणप्रकाशविवृतिपरीक्षा} इत्यप्यपरमभिधानम्~।\renewcommand{\thefootnote}{४९}\footnote{तदेव पृ. ४. पा. टि. ३ आङ्ग्लभाषाप्रस्तावना पृ. १७ पा. टि. २.} दीधितेस्तृतीया टीका {\knu रामकृष्ण}विरचिताऽपि वर्तते~।\renewcommand{\thefootnote}{५०}\footnote{तदेव पृ. ५. पा. दि. १~~~~~~~~~"~~~~~~~~~पृ. १८ पा. दि. १.} चतुर्थी टीका {\knu जयरामभट्टाचार्येण} विदधे\renewcommand{\thefootnote}{५१}\footnote{तदेव पृ. ५. पा. दि. १~~~~~~~~~"~~~~~~~~~पृ. १८ पा. टि. १.}~। {\knu बलभद्रमिश्रात्मजेन पद्मनाभेन किरणावलीभास्कर}\renewcommand{\thefootnote}{५२}\footnote{तदेव पृ. ६. पा. टि. १~~~~~~~~~"~~~~~~~~~पृ. १८ पा. टि. ३.} इति नाम्ना विख्याता स्वतन्त्रा टीकाऽरच्यत~।

एताः सर्वाष्टीकोपटीकाः किरणावल्या लोकप्रियतामुपयोगितां च निर्दिशन्ति~। एवमेवोदयनाचार्यस्यान्यासु कृतिष्वपि विद्वद्भिर्व्याख्या रचिताः~। एत्सर्वं न केवलमुदयनाचार्यस्य वैदुष्यमपि तु तस्य कृतीनां विद्वत्प्रियत्वमुपयोगितां च शंसति~। इदानीमपि न्यायकुसुमाञ्जलिसदृशीनां कृतीनां लोकप्रियत्वाद् हिन्दीभाषायामनुवादा उपलभ्यन्ते~।

सम्पादनेऽस्मिन् यैर्विद्वद्भिर्मित्रैश्च साहाय्यं दत्तं तेषामृणाङ्गीकरणस्माभिराङ्ग्लभाषाप्रस्तावनायां कृतमित्यत्र नाभ्यस्यते~। यद्यपि सावधानेनेदं पुस्तकं संशोधितं तथापि ये केचन दोषास्युस्ते विद्वांसः क्षमादृष्ट्या द्रक्ष्यन्ति~।

\begin{quote}
{\qt गच्छतः स्खलनं क्वापि भवत्येव प्रमादतः~।\\
हसन्ति दुर्जनास्तत्र समादधति सज्जनाः~॥}
\end{quote}

प्रियन्तां गुरवः~।

\begin{center}
\rule{0.2\linewidth}{0.5pt}
\end{center}

\newpage
\thispagestyle{empty}
\begin{center}
\textbf{\Large किरणावलीसहितप्रशस्तपादभाष्यस्य विषयानुक्रमः}
\end{center}

\noindent
\begin{tabular}{m{28em} m{2em}}
विषयः & पृष्ठाङ्काः\\
 & \\
मङ्गलम् & १\\
पदार्थोद्देशः & ४\\
अपवर्गचर्चा & ५\\
द्रव्योद्देशः & १०\\
तमसो द्रव्यत्वचर्चा & ११\\
गुणोद्देशः & १४\\
कर्मोद्देशः & १५\\
सामान्योद्देशः & "\\
विशेषोद्देशः & १७\\
समवायोद्देशः & १८\\
पदार्थानां साधर्म्यनिरूपणम् & "\\
साधर्म्यनिरूपणे द्रव्यादीनां पञ्चानां साधर्म्यनिरूपणम् & १९\\
~~~~~~"~~~~~~~~गुणादीनां~~~~"~~~~~~~~"~~~~~~ & "\\
~~~~~~"~~~~~~~~द्रव्यादीनां त्रयाणां~~~~~~"~~~~~~ & २०\\
~~~~~~"~~~~~~~~सामान्यादीनां~"~~~~~~~~" & २१\\
~~~~~~"~~~~~~~~सर्वेषां द्रव्याणां साधर्म्यनिरूपणम् & २२\\
पृथिवीनिरूपणम् & २८\\
पृथिवीनिरूपणे तस्या गुणनिरूपणम् & ३१\\
~~~~~~~"~~~~~~~~"~~~भेदनिरूपणम् & ३४\\
~~~~~~~"~~~~~~~~"~~~शरीरनिरूपणम् & ३८\\
~~~~~~~"~~~~~~~~"~~~इन्द्रियनिरूपणम् & ४१\\
~~~~~~~"~~~~~~~~"~~~विषयनिरूपणम् & ४२\\
जलनिरूपणम् & ४५\\
जलनिरूपणे तस्य गुणनिरूपणम् & "\\
~~~~~~~"~~~~~"~~~भेदनिरूपणम् & ४७\\
~~~~~~~"~~~~~"~~~शरीरनिरूपणम् & ४८\\
~~~~~~~"~~~~~"~~~इन्द्रियनिरूपणम् & "\\
~~~~~~~"~~~~~"~~~विषयनिरूपणम् &४९\\
तेजोनिरूपणम् & "\\
तेजोनिरूपणे तस्य गुणनिरूपणम् & "\\
\end{tabular}

\afterpage{\fancyhead[CE,CO]{\thepage}}
\cfoot{}
\newpage
%%%%%%%%%%%%%%%%%%%%%%%%%%%%%%%%%%%%%%%%%%%%%%%%%%%%%
\renewcommand{\thepage}{\devanagarinumeral{page}}
\setcounter{page}{2}

\noindent
\begin{tabular}{m{28em} m{2em}}
\textbf{विषयः} & \textbf{पृष्ठाङ्काः}\\
 & \\
 तेजोनिरूपणे तस्य भेदनिरूपणम् & ५०\\
~~~~~~"~~~~~~"~~~शरीरनिरूपणम् & "\\
~~~~~~"~~~~~~"~~~इन्द्रियनिरूपणम् & "\\
~~~~~~"~~~~~~"~~~विषयनिरूपणम् & ५१\\
वायुनिरूपणम् & ५३\\
वायुनिरूपणे तस्य गुणनिरूपणम् & "\\
~~~~~~"~~~~~~"~~~भेदनिरूपणम् & ५४\\
~~~~~~"~~~~~~"~~~शरीरनिरूपणम् & ५५\\
~~~~~~"~~~~~~"~~~इन्द्रियनिरूपणम् & "\\
~~~~~~"~~~~~~"~~~विषयनिरूपणम् & ५६\\
चतुर्णां महाभूतानां सृष्टिसंहारविधिः & ६०\\
ईश्वरसिद्धिः & ६५\\
आकाशकालदिशामेकैकत्वकथनम् & ७०\\
आकाशनिरूपणम् & "\\
आकाशस्य गुणनिरूपणम् & "\\
शब्दस्याकाशगुणनिरूपणम् & ७१\\
आकाशस्यैकत्वादिनिरूपणम् & ७४\\
आकाशेन्द्रियनिरूपणम् & ७५\\
कालनिरूपणम् & ७६\\
कालस्य क्षणादीनां व्यवहारहेतुत्वकथनम् & ७८\\
कालस्य गुणनिरूपणम् & ७९\\
दिङ्निरूपणम् & ८१\\
दिशो गुणकथनम् & ८३\\
दिशः प्राच्यादिव्यवहारहेतुत्वकथनम् & "\\
आत्मनिरूपणम् & ८४\\
शरीरेन्द्रियमनसामात्मत्वनिराकरणम् & ८६\\
आत्मातिरिक्तत्वनिरूपणम & ८८\\
सुखादिभिर्गुणैरात्मानुमानम् & ९०\\
आत्मविषये बौद्धमतखण्डनम् & ९३\\
आत्मगुणनिरूपणम् & ९७\\
मनोनिरूपणम् & १००\\
मनोगुणनिरूपणम् & १०१\\
गुणपदार्थनिरूपणम् & १०४\\
गुणानां साधर्म्यनिरूपणम् & "\\
\end{tabular}

\newpage
\noindent
\begin{tabular}{m{28em} m{2em}}
\textbf{विषयः} & \textbf{पृष्ठाङ्काः}\\
 &\\ 
 मूर्तगुणनिरूपणम् & १०६\\
अमूर्तगुणनिरूपणम् & "\\
मूर्तामूर्तगुणनिरूपणम् & १०६\\
अनेकाश्रितगुणनिरूपणम् & "\\
एकाश्रितगुणनिरूपणम् & "\\
वैशेषिकगुणनिरूपणम् & "\\
सामान्यगुणनिरूपणम् & १०७\\
बाह्यैकैकेन्द्रियग्राह्यगुणनिरूपणम् & "\\
द्वीन्द्रियग्राह्यगुणनिरूपणम् & "\\
अन्तःकरणग्राह्यगुणनिरूपणम् & "\\
अतीन्द्रियगुणनिरूपणम् & "\\
कारणगुणपूर्वकगुणनिरूपणम् & "\\
अकारणगुणपूर्वकगुणनिरूपणम् & १०८\\
पाकजगुणनिरूपणम् & "\\
कर्मजगुणनिरूपणम् & "\\
बुद्ध्यपेक्षगुणनिरूपणम् & "\\
समानजात्यारम्भकगुणनिरूपणम् & १०९\\
असमानजात्यारम्भकगुणनिरूपणम् & "\\
समानासमानजात्यारम्भकगुणनिरूपणम् & "\\
स्वाश्रयसमवेतारम्भकगुणनिरूपणम् & "\\
परत्रारम्भकगुणनिरूपणम् & ११०\\
उभयत्रारम्भकगुणनिरूपणम् & "\\
क्रियाहेतुगुणनिरूपणम् & "\\
असमवायिकारणगुणनिरूपणम् & १११\\
निमित्तकारणगुणनिरूपणम् & "\\
उभयकारणगुणनिरूपणम् & "\\
अकारणगुणनिरूपणम् & ११२\\
प्रदेशवृत्तिगुणनिरूपणम् & "\\
आश्रयव्यापिगुणनिरूपणम् & "\\
यावद्द्रव्यभाविगुणनिरूपणम् & "\\
अयावद्द्रव्यभाविगुणनिरूपणम् & ११३\\
रूपादीनां रूपादिसञ्ज्ञाकारणकथनम् & "\\
रूपनिरूपणम् & ११४
\end{tabular}

\newpage
\noindent
\begin{tabular}{m{28em} m{2em}}
\textbf{विषयः} & \textbf{पृष्ठाङ्काः}\\
 &\\ 
 रसनिरूपणम् & ११६\\
गन्धनिरूपणम् & ११७\\
स्पर्शनिरूपणम् & ११७\\
पाकजोत्पत्तिविधानम् & ११८\\
सङ्ख्यानिरूपणम् & १२१\\
सङ्ख्यानिरूपणे द्वित्वोत्पत्तिविनाशनिरूपणम् & १२५\\
द्वित्वोत्पत्तिविनाशविषये सहानवस्थानलक्षणविरोधदोषप्रदर्शनम् & १३३\\
परिमाणनिरूपणम् & १३६\\
पृथक्त्वनिरूपणम् & १४१\\
संयोगनिरूपणम् & १४३\\
अजसंयोगखण्डनम् & १४८\\
युतसिद्धिनिरूपणम् & १४९\\
संयोगविनाशकारणकथनम् & १५०\\
विभागनिरूपणम् & "\\
विभागजविभागनिरूपणम् & १५३\\
परत्वापरत्वनिरूपणम् & १६३\\
दिक्कृतपरत्वापरत्वनिरूपणम् & १६४\\
कालकृतपरत्वापरत्वनिरूपणम् & १६५\\
परत्वापरत्वविनाशहेतुनिरूपणम् & १६६\\
बुद्धिनिरूपणम् & १७०\\
बुद्धिनिरूपणे अविद्योद्देशः & १७१\\
अविद्यानिरूपणे संशयनिरूपणम् & १७२\\
~~~~~~~"~~~~~~~विपर्ययनिरूपणम् & १७४\\
प्रत्यक्षविषयविपर्ययनिरूपणम् & "\\
अनुमानविषयविपर्ययनिरूपणम् & १७६\\
अनध्यवसायनिरूपणम् & १७८\\
प्रत्यक्षविषयानध्यवसायनिरूपणम् & "\\
अनुमानविषयानध्यवसायनिरूपणम् & १७९\\
स्वप्ननिरूपणम् & "\\
स्वप्नान्तिकनिरूपणम् & १८२\\
विद्योद्देशः & १८३\\
प्रत्यक्षनिरूपणम् & "\\
द्रव्यप्रत्यक्षहेतुनिरूपणम् & १८४\\
रूपरसगन्धस्पर्शग्राहकप्रत्यक्षनिरूपणम् & १८६
\end{tabular}

\newpage
\noindent
\begin{tabular}{m{28em} m{2em}}
\textbf{विषयः} & \textbf{पृष्ठाङ्काः}\\
 &\\ 
 शब्दोपलब्धिहेतु प्रत्यक्षनिरूपणम् & १८६\\
सङ्ख्यापरिमाणादीनां प्रत्यक्षहेतुनिरूपणम् & १८७\\
आत्मविशेषगुणानां प्रत्यक्षोपलब्धिहेतुनिरूपणम् & १८८\\
भावद्रव्यादीनां प्रत्यक्षहेतुनिरूपणम् & "\\
योगिप्रत्यक्षनिरूपणम् & १८९\\
निर्विकल्पप्रत्यक्षनिरूपणम् & १९१\\
सविकल्पकप्रत्यक्षनिरूपणम् & १९२\\
अनुमाननिरूपणम् & १९३\\
लिङ्गलक्षणम् & "\\
स्वार्थानुमानग्रहणम् & १९७\\
व्याप्तिनिरूपणम् & १९८\\
बौद्धाभिमतव्याप्तिनिरासः & १९९\\
व्याप्तिग्रहणप्रमाणनिरूपणम् & २०१\\
अनुमानभेदाः & २०३\\
दृष्टानुमाननिरूपणम् & "\\
सामान्यतोदृष्टानुमाननिरूपणम् & २०४\\
अनुमाने शब्दप्रमाणान्तर्भावः & २०६\\
चेष्टाया अनुमानेऽन्तर्भावः & २१३\\
उपमानस्यानुमानेऽन्तर्भावः & २१४\\
अर्थापत्तेरनुमानान्तर्भावः & २१६\\
सम्भवस्यानुमानेऽन्तर्भावः & २१८\\
अभावस्यानुमानेऽन्तर्भावः & "\\
ऐतिह्यस्यानुमानेऽन्तर्भावः & २२१\\
परार्थानुमाननिरूपणम् & "\\
परार्थानुमानावयवनिरूपणम् & २२४\\
प्रतिज्ञानिरूपणम् & "\\
प्रतिज्ञाभासनिरूपणम् & २२५\\
हेतुनिरूपणम् & २२८\\
हेत्वाभासनिरूपणम् & "\\
असिद्धहेत्वाभासनिरूपणम् & २२९\\
विरुद्धहेत्वाभासनिरूपणम् & २३०\\
सन्दिग्धहेत्वाभासनिरूपणम् & "\\
अनध्यवसितनिरूपणम् & २३४\\
निदर्शननिरूपणम् & २३५
\end{tabular}

\newpage
\noindent
\begin{tabular}{m{28em} m{2em}}
\textbf{विषयः} & \textbf{पृष्ठाङ्काः}\\
 & \\
निदर्शनाभासनिरूपणम् & २३६\\
अनुसन्धाननिरूपणम् & २३८\\
प्रत्याम्नायनिरूपणम् & २३९\\
परार्थानुमाने पञ्चावयवावश्यकतानिरूपणम् & २४१\\
अवधारणज्ञाननिरूपणम् & २४२\\
स्मृतिनिरूपणम् & २४३\\
प्रातिभज्ञाननिरूपणम् & २४५\\
सिद्धदर्शनस्य ज्ञानान्तरत्वप्रतिषेधः & २४६\\
सुखनिरूपणम् & २४७\\
दुःखनिरूपणम् & २४८\\
इच्छानिरूपणम् & २४९\\
द्वेषनिरूपणम् & २५०\\
प्रयत्ननिरूपणम् & २५१\\
गुरूत्वनिरूपणम् & २५३\\
द्रवत्वनिरूपणम् & २५४\\
स्नेहनिरूपणम् & २५६\\
संस्कारनिरूपणम् & २५७
\end{tabular}

\begin{center}
\textbf{(अत्र किरणावली समाप्यते)}
\textbf{\Large (अथ अवशिष्टप्रशस्तपादभाष्यमात्रस्य विषयानुक्रमः)}
\end{center}

\noindent
\begin{tabular}{m{28em} m{2em}}
संस्कारनिरूपणे वेगनिरूपणम् & २५९\\
~~~~~~~"~~~~~~~भावनानिरूपणम् & "\\
~~~~~~~"~~~~~~~स्थितिस्थापकनिरूपणम् & "\\
धर्मनिरूपणम् & "\\
अधर्मनिरूपणम् & २६१\\
मोक्षनिरूपणम् & "\\
शब्दनिरूपणम् & २६२\\
कर्मनिरूपणम् & "\\
उत्क्षेपणनिरूपणम् & २६३\\
अवक्षेपणनिरूपणम् & "\\
आकुञ्चननिरूपणम् & "\\
प्रसारणनिरूपणम् & "\\
गमननिरूपणम् & "\\
कर्म्मणां जातिपञ्चत्वस्थापनम् & "
\end{tabular}

\newpage
\noindent
\begin{tabular}{m{28em} m{2em}}
\textbf{विषयः} & \textbf{पृष्ठाङ्काः}\\
 & \\
 सत्प्रत्ययकर्मनिरूपणम् & २६६\\
पाणिमुक्तेषु गमनविधिः & २६७\\
यन्त्रमुक्तेषु~~~~~" & "\\
अप्रत्ययकर्मनिरूपणम् & २६८\\
सामान्यनिरूपणम् & २७१\\
परसामान्यनिरूपणम् & "\\
अपरसामान्यनिरूपणम् & "\\
विशेषनिरूपणम् & २७२ \\
समवायनिरूपणम् & २७३\\
परिशिष्ट \textendash\ १ लक्षणावली (मूलभाग) & २७६\\
परिशिष्ट \textendash\ २ वैशेषिकसूत्रपाठः & २८१\\
\end{tabular}

\vspace{3cm}
\begin{center}
\rule{0.2\linewidth}{0.5pt}
\end{center}

\newpage
\thispagestyle{empty}
\begin{center}
\textbf{\Large किरणावलीसहितप्रशस्तपादभाष्यस्य अकारादिक्रमानुसारेण विषयानुक्रमः~।}
\end{center}

\noindent
\begin{tabular}{m{28em} m{2em}}
\textbf{विषयः} & \textbf{पृष्ठाङ्काः}\\
 & \\
अकारणगुणनिरूपणम् & ११२\\
अकारणगुणपूर्वकगुणनिरूपणम् & १०८\\
अजसंयोगखण्डनम् & १४४\\
अतीन्द्रियगुणनिरूपणम् & १०७\\
अधर्मनिरूपणम् & २६१\\
अनध्यवसायनिरूपणम् & १७८\\
अनध्यवसितनिरूपणम् & २३४\\
अनुमाननिरूपणम् & १९३\\
अनुमानभेदाः & २०७\\
अनुमानविषयविपर्ययनिरूपणम् & १७६\\
अनुमानविषयानध्यवसायनिरूपणम् & १७९\\
अनुमाने शब्दप्रमाणान्तर्भावः & २०६\\
अनुसन्धाननिरूपणम् & २३८\\
अनेकाश्रितगुणनिरूपणम् & १०६\\
अन्तःकरणग्राह्यगुणनिरूपणम् & १०७\\
अपरसामान्यनिरूपणम् & २७१\\
अपवर्गचर्चा & ५\\
अप्रत्ययर्कर्मनिरूपणम् & २६८\\
अभावस्यानुमानेऽन्तर्भावः & २१८\\
अमूर्तगुणनिरूपणम् & १०६\\
अयावद्द्रव्यभाविगुणनिरूपणम् & ११३\\
अर्थापत्तेरनुमानान्तर्भावः & २१६\\
अवक्षेपणनिरूपणम् & २६३\\
अवधारणज्ञाननिरूपणम् & २४२\\
अविद्यानिरूपणे विपर्ययनिरूपणम् & १७४\\
~~~~~~~"~~~~~~~संशयनिरूपणम् & १७२\\
असमवायिकारणगुणनिरूपणम् & १११\\
असमानजात्यारम्भकगुणनिरूपणम् & १०९\\
असिद्धहेत्वाभासनिरूपणम् & २२९\\
आकाशकालदिशामेकैकत्वकथनम् & ७०
\end{tabular}

\newpage
\noindent
\begin{tabular}{m{28em} m{2em}}
\textbf{विषयः} & \textbf{पृष्ठाङ्काः}\\
 & \\
 आकाशनिरूपणम् & ७०\\
आकाशस्य गुणनिरूपणम् & "\\
आकाशेन्द्रियनिरूपणम् & ७५\\
आकाशस्यैकत्वादिनिरूपणम् & ७४\\
आकुञ्चननिरूपणम् & २६३\\
आत्मगुणनिरूपणम् & ९७\\
आत्मनिरूपणम् & ८४\\
आत्मविशेषगुणानां प्रत्यक्षोपलब्धिहेतुनिरूपणम् & १८८\\
आत्मविषये बौद्धमतखण्डनम् & ९३\\
आत्मातिरिक्तत्वनिरूपणम् & ८८\\
आश्रयव्यापिगुणनिरूपणम् & ११२\\
इच्छानिरूपणम् & २८९\\
ईश्वरसिद्धिः & ६५\\
उत्क्षेपणनिरूपणम् & २६३\\
उपमानस्यानुमानेऽन्तर्भावः & २१४\\
उभयकारणगुणनिरूपणम् & १११\\
उभयत्रारम्भकगुणनिरूपणम् & ११०\\
एकाश्रितगुणनिरूपणम् & १०६\\
ऐतिह्यस्यानुमानेऽन्तर्भावः & २२१\\
कर्म्मजगुणनिरूपणम् & १०८\\
कर्म्मणां जातिपञ्चत्वस्थापनम् & २६३\\
कर्म्मनिरूपणम् & २६२\\
कर्म्मोद्देशः & १५\\
कारणगुणपूर्वकगुणनिरूपणम् & १०७\\
कालकृतपरत्वापरत्वनिरुपणम् & १६५\\
कालनिरूपणम् & ७६\\
कालस्य क्षणादीनां व्यवहारहेतुत्वकथनम् & ७८\\
कालस्य गुणनिरूपणम् & ७९\\
क्रियाहेतुगुणनिरूपणम् & ११०\\
गन्धनिरूपणम् & ११७\\
गमननिरूपणम् & २६३\\
गुणपदार्थनिरूपणम् & १०४\\
गुणानां साधर्म्यनिरूपणम् & "\\
गुणोद्देशः & १४
\end{tabular}

\newpage
\noindent
\begin{tabular}{m{28em} m{2em}}
\textbf{विषयः} & \textbf{पृष्ठाङ्काः}\\
 & \\
 गुरूत्वनिरूपणम् & २५३\\
चतुर्णां महाभूतानां सृष्टिसंहारविधिः & ६०\\
चेष्टाया अनुमानेऽन्तर्भावः & २०३\\
जलनिरूपणम् & ४५\\
जलनिरूपणे तस्येन्द्रियनिरूपणम् & ४८\\
जलनिरूपणे तस्य गुणनिरूपणम् & ४५\\
जलनिरूपणे तस्य भेदनिरूपणम् & ४७\\
जलनिरूपणे तस्य विषयनिरूपणम् & ४९\\
जलनिरूपणे तस्य शरीरनिरूपणम् & ४८\\
तमसो द्रव्यत्वचर्चा & ११\\
तेजोनिरूपणम् & ४९\\
तेजोनिरूपणे तस्य गुणनिरूपणम् & "\\
तेजोनिरूपणे तस्य भेदनिरूपणम् & ५०\\
तेजोनिरूपणे तस्य शरीरनिरूपणम् & "\\
तेजोनिरूपणे तस्य विषयनिरूपणम् & ५१\\
तेजोनिरूपणे तस्येन्द्रियनिरूपणम् & ५०\\
दिक्कृतपरत्वापरत्वनिरूपणम् & १६४\\
दिङ्निरूपणम् & ८१\\
दिशो गुणकथनम् & ८३\\
दिशः प्राच्यादिव्यवहारहेतुत्वकथनम् & "\\
दुःखनिरूपणम् & २४८\\
दृष्टानुमाननिरूपणम् & २०३\\
द्वित्वोत्पत्तिविनाशविषये सहानवस्थानलक्षणविरोधदोषप्रदर्शनम् & १३३\\
द्वीन्द्रियग्राह्यगुणनिरूपणम् & १०७\\
द्वेशनिरूपणम् & २५०\\
द्रवत्वनिरूपणम् & २५४\\
द्रव्यप्रत्यक्षहेतुनिरूपणम् & १८४\\
द्रव्योद्देशः & १०\\
धर्मनिरूपणम् & २५९\\
निदर्शननिरूपणम् & २३५\\
निदर्शनाभासनिरूपणम् & २३६\\
निमित्तकारणगुणनिरूपणम् & १११\\
निर्विकल्पकप्रत्यक्षनिरूपणम् & १९१\\
पदार्थानां साधर्म्यनिरूपणम् & १८
\end{tabular}

\newpage
\noindent
\begin{tabular}{m{28em} m{2em}}
\textbf{विषयः} & \textbf{पृष्ठाङ्काः}\\
 &\\ 
 पदार्थोद्देशः & ४\\
परत्रारम्भकगुणनिरूपणम् &११०\\
परत्वापरत्वनिरूपणम् &१६३\\
परत्वापरत्वविनाशहेतुनिरूपणम् &१६६\\
परसामान्यनिरूपणम् &२७१\\
परार्थानुमाननिरूपणम् &२२१\\
परार्थानुमानावयवनिरूपणम् &१३६\\
परार्थानुमाने पञ्चावयवावश्यकतानिरूपणम् &१०८\\
परिमाणनिरूपणम् &११८\\
पाकजगुणनिरूपणम् &२६७\\
पाकजोत्पत्तिविधानम् &१४१\\
पाणिमुक्तेषु गमनविधिः &२८\\
पृथक्त्वनिरूपणम् &४१\\
पृथिवीनिरूपणम् &२८\\
पृथिवीनिरूपणे तस्या इन्द्रियनिरूपणम् &४१\\
पृथिवीनिरूपणे तस्या गुणनिरूपणम् &२८\\
पृथिवीनिरूपणे तस्या भेदनिरूपणम् &३४\\
पृथिवीनिरूपणे तस्या विषयनिरूपणम् &४२\\
पृथिवीनिरूपणे तस्याः शरीरनिरूपणम् &३८\\
प्रतिज्ञानिरूपणम् &२२४\\
प्रतिज्ञाभासनिरूपणम् &२२५\\
प्रत्यक्षनिरूपणम् &१८३\\
प्रत्यक्षविषयविपर्ययनिरूपणम् &१७४\\
प्रत्यक्षविषयानध्यवसायनिरूपणम् &१७८\\
प्रत्याम्नायनिरूपणम् &२३९\\
प्रदेशवृत्तिगुणनिरूपणम् &११२\\
प्रयत्ननिरूपणम् &२५१\\
प्रसारणनिरूपणम् &२६३\\
प्रातिभज्ञाननिरूपणम् &२४६\\
बाह्यैकैकेन्द्रियग्राह्यगुणनिरूपणम् &१०७\\
बुद्धिनिरूपणम् &१७०\\
बुद्धिनिरूपणे अविद्योद्देशः &१७१\\
बुद्ध्यपेक्षगुणनिरूपणम् &१०८\\
बौद्धाभिमतव्याप्तिनिरासः &१९९\\
भावद्रव्यत्वादीनां प्रत्यक्षहेतुनिरूपणम् &१८८\\
मङ्गलम् &१
\end{tabular}

\newpage
\noindent
\begin{tabular}{m{28em} m{2em}}
\textbf{विषयः} & \textbf{पृष्ठाङ्काः}\\
 & \\
 मनोगुणनिरूपणम् &१०१\\
मनोनिरूपणम् &१००\\
मूर्तगुणनिरूपणम् &१०६\\
मूर्तामूर्तगुणनिरूपणम् &"\\
मोक्षनिरूपणम् &२६१\\
यन्त्रमुक्तेषु गमनविधिः &२६७\\
यावद्द्रव्यभाविगुणनिरूपणम् &११२\\
युतसिद्धिनिरूपणम् &१४९\\
योगिप्रत्यक्षनिरूपणम् &१८९\\
रसनिरूपणम् &११६\\
रूपनिरूपणम् &११४\\
रूपरसगन्धस्पर्शग्राहकप्रत्यक्षनिरूपणम् &१८६\\
रूपादीनां रूपादिसञ्ज्ञाकारणकथनम् &११३\\
लिङ्गलक्षणम् &१९३\\
वायुनिरूपणम् &५२\\
वायुनिरूपणे तस्य गुणनिरूपणम् &"\\
वायुनिरूपणे तस्य भेदनिरूपणम् &५४\\
वायुनिरूपणे तस्य विषयनिरूपणम् &५६\\
वायुनिरूपणे तस्य शरीरनिरूपणम् &५५\\
वायुनिरूपणे तस्येन्द्रियनिरूपणम् &"\\
विद्योद्देशः &१८३\\
विभागनिरूपणम् &१५०\\
विभागजविभागनिरूपणम् &१५३\\
विरूद्धहेत्वाभासनिरूपणम् &२३०\\
विशेषनिरूपणम् &२७२\\
विशेषोद्देशः &१७\\
वैशेषिकगुणनिरूपणम् &१०६\\
व्याप्तिग्रहणप्रमाणनिरूपणम् &२०१\\
व्याप्तिनिरूपणम् &१९८\\
शब्दनिरूपणम् &२६२\\
शब्दस्याकाशगुणनिरूपणम् &७१\\
शब्दोपलब्धिहेतुप्रत्यक्षनिरूपणम् &१८६\\
शरीरेन्द्रियमनसामात्मत्वनिराकरणम् &८६\\
सङ्ख्यानिरूपणम् &१२४\\
सङ्ख्यानिरूपणे द्वित्वोत्पत्तिविनाशनिरूपणम् &१२५\\
सङ्ख्यापरिमाणादीनां प्रत्यक्षहेतुनिरूपणम् १८७
\end{tabular}

\newpage
\noindent
\begin{tabular}{m{28em} m{2em}}
\textbf{विषयः} & \textbf{पृष्ठाङ्काः}\\
 &\\ 
 सत्प्रत्ययासत्प्रत्ययकर्मनिरूपणम् &२६६\\
सन्दिग्धहेत्वाभासनिरूपणम् &२३०\\
समवायनिरूपणम् &२७३\\
समवायोद्देशः &१८\\
समानजात्यारम्भकगुणनिरूपणम् &१०९\\
समानासमानजात्यारम्भकगुणनिरूपणम् &"\\
सम्भवस्यानुमानेऽन्तर्भावः &२१८\\
सविकल्पकप्रत्यक्षनिरूपणम् &१९२\\
साधर्म्यनिरूपणे गुणादीनां पञ्चानां साधर्म्यनिरूपणम् &१९\\
साधर्म्यनिरूपणे द्रव्यादीनां त्रयाणां साधर्म्यनिरूपणम् &२०\\
साधर्म्यनिरूपणे द्रव्यादीनां पञ्चानां साधर्म्यनिरूपणम् &१९\\
साधर्म्यनिरूपणे सर्वेषां द्रव्याणां साधर्म्यनिरूपणम् &२२\\
साधर्म्यनिरूपणे सामान्यादीनां त्रयाणां साधर्म्यनिरूपणम् &२१\\
सामान्यगुणनिरूपणम् &१०७\\
सामान्यतोदृष्टानुमाननिरूपणम् &२०४\\
सामान्यनिरूपणम् &२७१\\
सामान्योद्देशः &१५\\
सिद्धदर्शनस्य ज्ञानान्तरप्रतिषेधः &२४६\\
सुखनिरूपणम् &२४७\\
सुखादिभिर्गुणैरात्मानुमानम् &९०\\
संयोगनिरूपणम् &१४३\\
संयोगविनाशकारणकथनम् &१५०\\
संस्कारनिरूपणम् (अत्र किरणावली समाप्यते) &२५७\\
संस्कारनिरूपणे भावनानिरूपणम् &२५९\\
संस्कारनिरूपणे वेगनिरूपणम् &"\\
संस्कारनिरूपणे स्थितिस्थापकनिरूपणम् &"\\
स्नेहनिरूपणम् &२५६\\
स्पर्शनिरूपणम् &११७\\
स्मृतिनिरूपणम् &२४२\\
स्वप्ननिरूपणम् &१७९\\
स्वप्नान्तिकनिरूपणम् &१८२\\
स्वार्थानुमानग्रहणम् &१९७\\
स्वाश्रयसमवेतारम्भकगुणनिरूपणम् &१०९\\
हेतुनिरूपणम् &२२३\\
हेत्वाभासनिरूपणम् &"
\end{tabular}

\vspace{-7mm}
\begin{center}
\rule{0.2\linewidth}{0.5pt}
\end{center}

\newpage
\thispagestyle{empty}
\begin{center}
\textbf{(१) ॥~प्रशस्तपादभाष्यम्~॥}
\end{center}
\begin{quote}
{\qt प्रणम्य हेतुमीश्वरं मुनिं कणादमन्वतः~।\\
पदार्थधर्मसङ्ग्रहः प्रवक्ष्यते महोदयः~॥}
\end{quote}
\begin{center}
\textbf{\LARGE प्रशस्तपादभाष्यस्य किरणावली~॥}

\textbf{\small श्रीगणेशाय नमः~॥}
\end{center}

\begin{quote}
{\knu (१) \renewcommand{\thefootnote}{१}\footnote{कर्तव्यविघ्नविघातकं रविनमस्कारं निबध्नाति \textendash\ विद्येति~। यदिति सामान्यतोऽपि कर्तृनिर्देशे विद्याऽविद्ययोः सन्ध्यारजनीभ्यां निरूपणाद् रविरुदेता लभ्यते~। विद्येव या पूर्वसन्ध्या तदुदयोद्रेकादुत्पत्त्याधिक्यादविद्येव या रजनी तत्क्षये सति यदुदेति \textendash\ उदयगिरिशिखरमधिरोहति~। सन्ध्या च न रात्रेर्भागविशेषः~। निरस्तैतद्वीपवर्तिरविरश्मिजालस्य कालविशेषस्य रात्रित्वात्~। सन्ध्यायां चात्र द्वीपे कतिपयतत्सत्त्वात्~। अत एव रात्रिसन्ध्ययोर्धर्मशास्त्रे पृथगभिधानम्~।

*यद्वा ल्यब्लोपे पञ्चमी, सप्तमी च निमित्ततायाम्; तथा च विद्याहेतुर्या सन्ध्या तदुदयोद्रेकं प्राप्याविद्याहेतुर्या रजनी तत्क्षयार्थं यदुदेति तस्मै विश्वतो \textendash\ विश्वस्मात् त्विट् दीप्तिर्यस्य तथाभूताय कस्मैचिद्विशिष्याशक्यनिर्वचनतत्तद्गुणगरिम्ने नमः~।

यद्वा एतच्छास्त्रसाध्यमोक्षहेतुतत्त्वज्ञानविषयाय आत्मन एवायं नमस्कारः~। विद्या \textendash\ आत्मसाक्षात्कारः, सैव तत्त्वप्रकाशकत्वात्सन्ध्या, तदुदयोद्रेकाद् दृढतरसंस्कारजननादविद्या \textendash\ आत्मनि मिथ्याज्ञानम्; सैव तज्ज्ञानविरोधित्वाद् रागजनेकत्वाद्वा रजनी, तस्याः क्षये सति यदात्मकस्वरुपमुदेति मोक्षरूपप्रयोजनभाग्भवति तस्मै कस्मैचित्सर्वोकृष्टाय विश्वतस्त्विषे योगजधर्मसाचिव्याद्विश्वविषयज्ञानाय नमः~। कि. प्र. व.}
विद्यासंध्योदयोद्रेकादविद्यारजनीक्षये~|

     यदुदेति नमस्तस्मै कस्मैचिद्विश्वतस्त्विषे~|| १ ||
     
\renewcommand{\thefootnote}{२}\footnote{एतच्छास्त्रप्रतिपाद्यपदार्थोद्देशं कुर्वन्नेव नमस्कारवद् रूपान्तरेणापि भगवदुपासनमिष्टसाधनमिति दर्शयति यत इति~। उत्पतिज्ञप्तिहेतुत्वं नित्यानित्ययोर्यथायोगम्~। अत्र द्रव्याणां बहुत्वेऽप्यात्मनः प्राधान्यख्यापनाय द्रव्यमित्येकवचनम्~। तद्विषयाणां श्रवणादिप्रतिपत्तीनां बहुत्वं गुणा इति बहुवचनेन व्यज्यते~। परा व्यापिका~। अपरा व्याप्या~। ' वा ' शब्दः समुच्चये~। \rule{0.4\linewidth}{0.5pt}}यतो द्रव्यं गुणाः कर्म तथा जातिः परापरा~।\\
विशेषाः समवायो वा तमीश्वरमुपास्महे~॥~२~॥}
\end{quote}

\blfootnote{*ननु निरस्तरविकत्वमेककालस्य रात्रित्वं लाघवात्, अतः सन्ध्यापि रात्रिरेव~। धर्मशास्त्रे पृथगुपादानं गोबलीवर्दन्यायेन दोषातिशयप्रतिपादनार्थमित्यस्वरसादाह यद्वेति~। कि. प्र. व्याख्याय्यां द्रव्यप्रकाशिकायां भगीरथठक्कुरः~।}

\afterpage{\fancyhead[CE,CO]{\thepage}}
\cfoot{}
\newpage
%%%%%%%%%%%%%%%%%%%%%%%%%%%%%%%%%%%%%%%%%%%%%%%%%%%%%
\renewcommand{\thepage}{\devanagarinumeral{page}}
\setcounter{page}{2}

\blfootnote{यद्वा द्रव्यं हिरण्यादि, गुणाः शास्त्रादयः कर्म हिताहितहेतुव्यापारः, जातिः परा  \textendash\ उत्कृष्टा ब्राह्मणत्वादिः अपरा  \textendash\ अधमा चाण्डालत्वादिः, जातेरनिमित्तकत्वेपि तज्जातीयशरीरोत्पत्तिस्तद्धेतुका~। विशेषा अतिशयरूपाः~। समवायो मेलकः~। सदसदभ्यां सह इत्येतत्सर्वं यत ईश्वराद्भवति स उपास्यः इति युक्तमिति~। कि. प्र. व.}

\begin{quote}
{\knu अर्थानां प्रविवेचनाय जगतामन्तस्तमःशान्तये\\
सन्मार्गस्य विलोकनाय गतये लोकस्य यागार्थिनः~।\\
\renewcommand{\thefootnote}{१}\footnote{तत्तदिति~। ते ते तामसाः कुतर्काभ्यासजनिततमोगुणप्रधाना नास्तिका नक्तंचराश्च भूताः प्राणिनः, तेषां भीतये भीत्यै~। प्राचीननिबन्धाश्च सौगताद्युत्थापितकुहेतुसन्तमसाच्छादितास्तत्त्वज्ञानाय न पर्याप्ता इति तन्निरासाय~। कि प्र व.~।}तत्तत्तामसभूतभीतय इमां विद्यावतां प्रीतये\\
\renewcommand{\thefootnote}{२}\footnote{व्यातेने व्यातनोत्~। आर्शसायां भूतवच्चेत्याशंसायां लिडिति मतमयुक्तम्, तत्र भूतसामान्यप्रत्ययस्यैवातिदेशात् भूतविशेषविहितयोर्लङ्लिटोरनतिदेशात् तिङन्तप्रतिरुपकोऽयं निपातः~। चक्रे सुबन्धुः सुजनैकबन्धुरितिवदिति केचित्~। वस्तुतः णलुत्तमो वेति ज्ञापकात्कर्त्रपरोक्षेपि लिट् साधुः~। ग्रन्थकरणरभसवशेन चित्तविक्षेपो वा ऋजुवस्तुनाऽयमाचार्यस्य श्लोकः किन्त्वन्यस्येति समादधुः~। कि. प्र. व.~।}व्यातेने किरणावलीमुदयनः सत्तर्कतेजोमयीम्~॥~३~॥

अतिविरसमसारं मानवार्ताविहीनं\\
प्रविततबहुवेलं$^1$ प्रक्रियाजालदुःखम्~।\\
उदधिसममतन्त्रं तन्त्रमेतद्वदन्ति\\
प्रखलजडधियो ये तेऽनुकम्प्यन्त एते~॥~४~॥}
\end{quote}

शास्त्रारम्भे2 सदाचारपरिप्राप्ततया कायवाङ्मनोभिः कृतं परापरगुरुनमस्कारं शिष्यान् शिक्षयितुमादौ 3निबध्नाति \textendash\ {\knu प्रणम्ये}ति~। कर्तव्यापेक्षया ${}^4$पूर्वकालभावित्वात् प्रणामस्य क्त्वानिर्देशः~। भक्तिश्रद्धातिशयलक्षणः \renewcommand{\thefootnote}{३}\footnote{ननु नत्वेत्यनेनैव नमस्कारनिबन्धनात् ' प्र ' शब्दो व्यर्थ इत्यत आह भक्तीति~। आराध्यत्वेन ज्ञानं भक्तिः~। आराधना च गौरवितप्रतीतिहेतुः क्रिया~। वेदबोधितफलावश्यंभावनिश्चयः श्रद्धा~। यद्वा भक्तिश्रद्धे ज्ञानत्वव्याप्यजातिविशेषौ~। कि. प्र. व. \rule{0.4\linewidth}{0.5pt}}प्रकर्षः 'प्र' शब्देन द्योत्यते~। तथा भूता हि

\blfootnote{*ननु प्रीतिजनकतावच्छेदकं च रूपं प्रतीतं न वा ? आद्ये तन्निश्चय एव प्रवर्तकः तदेव च प्रकर्ष उपजीव्यत्वात्~। अन्त्ये कथमुक्तरूपनिश्चयोऽपीत्यरूचेराह \textendash\ यद्वेति भक्तिश्रद्धेति भावप्रधानो निर्देशः~। तेनादृष्टविशेषोपगृहीतमनःप्रयोज्यौ नमस्कारहेतुज्ञाननिष्ठजातिविशेषौ भक्तित्वश्रद्धात्वे क्रियातौल्येऽपि फलसत्त्वासत्त्वाभ्यां कल्प्येते इति भावः~। कि. प्र. व्या. भ

1 वेल मु. कि.~। 2 सदाचारपरम्परा  \textendash\ क~। 3 न बध्नाति  \textendash\ जे~। 4 कालत्वात्  \textendash\ जे; क}

\newpage
\begin{sloppypar}
\noindent
परमेश्वरनतिर्मङ्गलमावहति~। कृतमङ्गलेन चारब्धं कर्म निर्विघ्नं${}^1$परिसमाप्यते \renewcommand{\thefootnote}{१}\footnote{प्रचयश्च प्रारिप्सितग्रन्थस्य गुरुणा शिष्याय दानस्याविच्छेदः~। कि. प्र. व.}प्रचीयते च~। \renewcommand{\thefootnote}{२}\footnote{आगमेति~। दोषाय यो व्यभिचारो व्याप्यव्यभिचारः, विनापि मङ्गलं समीहितसिद्धिरित्येवं रूपः; स इह नास्ति, कुतः; आगममूलत्वात्~। अस्यार्थस्य कार्यकारणभावस्य तथा चागमात् कार्यकारणभावेऽवधृते यत्रापाततो नमस्कारादर्शनेऽपि निर्विघ्नं समाप्तिर्दृश्यते तत्रापि तथैव लिङ्गेन जन्मान्तरीयो नमस्कारोऽनुमीयत इत्यर्थः~। कि. प्र. व. \rule{0.4\linewidth}{0.5pt}}आगममूलत्वाच्चास्यार्थस्य व्यभिचारो न दोषाय; तस्य ${}^2$कर्तृकर्मसाधवैगुण्य3हेतुकत्वात् साद्गुण्येऽपि विघ्नहेतूनां बलीयस्त्वात्~। न चैवं ${}^4$सति किमनेनेति वाच्यम्, प्रचितस्यास्यैव बलवत्तरविघ्नवारणेऽपि कारणत्वात्~। नहि घनविमुक्तमेकस्तृणस्तम्बो ${}^5$निवारयितुमसमर्थ इति तदर्थं नोपादीयते, सजातीयप्रचयसम्बलितस्यैव$^6$ विघ्नवारणे शक्तत्वात्~। न च विघ्नहेतुसद्भावनिश्चयाभावात् तद्वारणे कारणमनुपादेयम्; यतस्तत्सन्देहेऽपि तदुपादानस्य न्याय्यत्वात्~। अन्यथाऽनुपस्थितपरिपन्थिभिः पार्थिवैर्द्विरदयूथपतयो नाद्रियेरन्निति~।
\end{sloppypar}

{\knu ईश्वर}मित्यनेनैव लब्धे जगद्धेतुत्वे {\knu हेतु}मिति पुनर्विशेषणोपादानं प्रमाणसूचनाय~। कार्यं हि हेतुना विनाऽऽत्मानमनाप्नुवद्धेतुमत्तया कर्तारमाक्षिपति~। ईश्वरपदसन्निधिप्रयुक्तो वा हेतुशब्दो विशिष्ट एव श्रेयःसमधिगमनिमित्ते प्रवर्तते~। प्रस्तुतशास्त्रहेतुत्वाद्वा हेतुमित्याह~। स्मर्यते हि यत् {\knu कणादो} मुनिर्महेश्वरनियोगप्रसादावधिगम्य प्रणीतवांस्तेन तं {\knu हेतुं प्रणम्य सङ्ग्रहः प्रवक्ष्य}ते इत्यर्थः~। {\knu अतः} ईश्वरप्रणामादनु पश्चात् {\knu कणाद}नामानं मुनिं प्रणम्येत्यनुषज्यते~। यद्यपि गुरुतमगुरुतरगुरु$^7$क्रमेणैव प्रणामः क्रियते इति शिष्टाचारादेव लभ्यते, तथापि शिष्यशिक्षायै क्रमो$^8$ निबद्धोऽन्विति~। तथा च मुनिप्रणतेः पश्चाद्भावे दर्शिते सन्निधिसिद्धमवधित्वमीश्वरप्रणामस्येत्यत इति मन्दप्रयोजनमित्यपि न वाच्यम्; श्रुतिप्राप्तेऽर्थे$^9$ प्रकरणादीनामनवकाशात्~। अथवा यतः शुश्रूषवः श्रेयोऽर्थिनः श्रवणादिपटवोऽनसूयकाश्चान्तेवासिन$^10$ उपसेदुरतो वक्ष्यत इत्यनेन सम्बद्ध्यते; अन्यथाऽरण्यरुदितं स्यादित्यपि शिष्यशिक्षायै~। एवं हि शिक्षिते शिष्या अपि तथा कुर्यस्तथा ${}^11$चाविच्छिन्नसम्प्रदायवीर्यवत्तरं$^12$ शास्त्रं स्यात्; येन विद्यैवाह \textendash

\blfootnote{1 समाप्यते  \textendash\ जे~। 2 कर्मकर्तृ,  \textendash\ जे क~। 3 हेतुत्वात् \textendash\ जे~। 4 'सति' जे; क पुस्तकयोर्नास्ति~। 5 न वारयितुं  \textendash\  ${}^\circ$जे, वारयितुं न  \textendash\ क~। 6 तस्य शक्तत्वात्  \textendash\ जे; क~। 7 प्रक्रमेणैव \textendash\ जे~। 8 क्रमोऽपि बद्धो \textendash\ जे~। 9 प्राप्ते प्रकरणादीनां \textendash\  ${}^\circ$जे~। 10 असूयवतश्च \textendash\ जे~। 11 चानवच्छिन्न  \textendash\ क~। 12 वीर्यवच्च  \textendash\ जे; वीर्यवत्तरं च  \textendash\ क~।}

\newpage
\begin{quote}
{\qt $\rightarrow$ '${}^1$विद्या ह वै ब्राह्मणमाजगाम गोपाय मा शेवधिष्टेऽहमस्मि ' $\leftarrow$\\
असूकायाऽनृजवे जडाय$^2$ न मां ब्रूया अवीर्यवती तथा स्याम्'~॥}
\end{quote}

\noindent
इति एतेन सौत्रमप्यतः$^3$ पदं$^4$ व्यारव्यातं स्यात्~।

\hangindent=2cm {\knu (२) द्रव्यगुणकर्मसामान्यविशेषसमवायानां ${}^5$पदार्थानां ${}^6$साधर्म्यवैधर्म्य \textendash\ तत्त्वज्ञानं निःश्रेयसहेतुः~। तच्चेश्वरचोदनाभिव्यक्ताद्धर्मादेव~।}

{\knu पदार्थधर्म}सङ्ग्रह इति \textendash\ पदार्था द्रव्यादयः, तेषां धर्माः साधर्म्यवैधर्म्यरूपास्त एव परस्परं विशेषणीभूतास्तेऽनेन सङ्गृह्यन्ते~। शास्त्रे नाना स्थानेषु वितता एकत्र सङ्कलय्य कथ्यन्त इति सङ्ग्रहः~। स प्रकृष्टो वक्ष्यते; ${}^7$प्रकरणशुद्धेः सङ्ग्रहपदेनैव दर्शितत्वात्~। वैशद्यं लघुत्वं ${}^8$कृत्स्नत्वं च प्रकर्षः ${}^9$सूत्रेषु वैशद्याभावात्, भाष्यस्य च विस्तरत्वात् प्रकरणादीनां${}^10$चैकदेशत्वात्~। एतेनाभिधेयं ${}^11$दर्शितम्~। न च तत्प्रतीतावपि प्रेक्षावान् प्रयोजनं विना प्रवर्तते इति ${}^12$तदाह {\knu महोदय} इति~। महानुदय$^13$ उद्गम उद्बोधो ज्ञानमिति यावत्; सोऽस्माद् भवतीति महोदयः सङ्ग्रह उक्तः~।

ततः किं न ह्ययं पुरुषार्थः~। के$^14$ पदार्थाः के च तेषां धर्मा इत्यत आह {\knu द्रव्ये}ति~। ${}^15$अत्र के पदार्था इत्यपेक्षायां पदार्था ${}^16$द्रव्यादयः~। के धर्मा ${}^17$इत्यपेक्षायां साधर्म्यवैधर्म्यरूपाः \textendash\ अनुवृत्तव्यावृत्तरूपा इत्यर्थः~। तेषामुद्बोधः कथं पुरुषार्थ इत्यत्र {\knu तत्त्वज्ञानं निःश्रेयस}हेतुरिति~। तत्त्वमनारोपितं रूपम्, तच्च साधर्म्यधर्म्याभ्यामेवं विविच्यते~। साक्षादपि हि दृश्यमाना अत्यन्ता असङ्कीर्णाः स्थाण्वादयो वक्रकोटरादिभिः पुरुषादिभ्यो विविच्यन्ते ${}^18$नत्वन्यथा किं पुनरतीन्द्रिया मिथो मिश्रीभूता${}^19$अत्यन्तसङ्कीर्णाः कालाकाशादयः ${}^20$शरीरेन्द्रियात्मादयो वेति~। एतेन पदार्था एव प्रधानतयोद्दिष्टा वेदितव्याः~। अभावस्तु स्वरूपवानपि पृथङ्नोद्दिष्टः; प्रतियोगिनिरूपणाधीननिरूपणत्वात्, न तु

\blfootnote{1 $\rightarrow$ $\leftarrow$ एतच्चिह्मान्तर्गतः पाठः जे क पुस्तकयोर्नास्ति~। 2 ऽदान्ताय \textendash\ जे~। 3 सौत्रमतः  \textendash\ जे~। 4 अथातो धर्मं व्याख्यास्यामः वै. सू. १. १. १.~। 5 ' षण्णां पदार्थानां ' \textendash\ इति कं; व्यो; पुस्तकयोः~। 6 साधर्म्यवैधर्म्याभ्यां तत्त्वज्ञानं  \textendash\ कि दे; साधर्म्यतत्त्वज्ञान  \textendash\ जे~। 7 प्रकरणशुद्धः पदेन  \textendash\ क~। 8 कृत्स्नता  \textendash\ जे. क~। 9 सूत्रे  \textendash\ जे. क~। 10 देशत्वादिति  \textendash\ जे~। 11 दर्शितं भवति  \textendash\ क~। 12 तत्राह  \textendash\ जे; तमाह  \textendash\ क~। 13 महानुदय उद्बोधस्तत्त्वज्ञामिति  \textendash\ कं~। 14 के ते पदार्थाः  \textendash\ क~। 15 ' अत्र ' \textendash\ ' क ' पुस्तके नास्ति~। 16 द्रव्यादयः षट्  \textendash\ मु. कि.~। 17 इत्यत्र  \textendash\ मु. कि.~। 18 नान्यथा \textendash\ मु. कि.~। 19 ' अत्यन्तसङ्कीर्णाः ' इति जे. क पुस्तकेयोर्नास्ति~। 20 शरीरात्मादयो  \textendash\ जे; क~।}

\newpage
\noindent
तुच्छत्वात्~। उत्पत्तिविनाशचिन्तायां प्रागभावप्रध्वंसाभावयोर्वैधर्म्ये चेतरेतराभावात्यन्ताभावयोस्तत्र ${}^1$तत्र दर्शयिष्यमाणत्वादिति~। तेन द्रव्यादीनां साधर्म्यवैधर्म्याभ्यां तत्त्वं प्रतिपादयन् सङ्ग्रहो निःश्रेयसं साधयति यतोऽतः प्रेक्षावतामुपादेय इति तात्पर्यम्~।

निःश्रेयसं पुनर्दुःखनिवृत्तिरात्यन्तिकी~। अत्र च वादिनामविवाद एव, नह्यपवृक्तस्य दुःखं$^2$ प्रत्यापद्यत इति कश्चिदभ्युपैति~। केवलमात्माऽपि दुःखहेतुत्वान्निवर्तयितव्यः शरीरादिवदिति ये वदन्ति तेषां यद्यसौ नास्ति किं विनिर्तयितव्यम् ? अत्यन्ताऽसतो नित्यनिवृत्तत्वात्~। अथास्ति, तथापि किं निवर्तनीयम् ? नित्यत्वेन तन्निवृत्तेरशक्यत्वात्~। अथ ज्ञानस्वभाव एवासौ निवर्तनीय इति मतम्; अनुमतमेतत्~। दग्धेन्धनानलवदुपशमो मोक्ष इति वक्ष्यमाणत्वात्~। तस्मादतिरिक्ते आत्मनि प्रमाणं वक्तव्यमित्यवशिष्यते तद्वक्ष्यामः~।

{\knu साङ्ख्याना}मपि दुःखनिवृत्तिरपवर्ग इत्यत्र न विप्रतिपत्तिः~। \renewcommand{\thefootnote}{१}\footnote{प्रकृत्याश्रयमिति \textendash\ यद्यपि भावाष्टकसम्पन्नतया महत एव दुःखमुपेयते साङ्ख्यैः तथापि तन्मते कार्यकारणयोरभेदात् प्रकृत्याश्रयं दुःखमुक्तम्~। एतस्थ विवादस्य मोक्षाविषयत्वेऽपि यस्य दुःखनि वृत्तिस्तस्य मुक्तिरिति प्रकृतेरेव मोक्षो नात्मन इनि मोक्षविषयत्वमस्यास्त्येवेति भावः~। कि. प्र. व~॥}प्रकृत्याश्रयं तु$^3$ दुःखं न पुरुषाश्रयमिति ${}^4$विवादस्तन्मतमग्रे निराकरिष्यामः~। ये त्वनुपप्लवां चित्तसन्ततिमनन्तामपवर्गमाहुस्तेऽप्युपप्लवस्य दुःखमयत्वात् तन्निवृत्तिमेवेच्छन्ति~। न च चित्तसन्ततेरनन्तत्वं प्रामाणिकम्; निमित्तस्य शरीरादेरपाये नैमित्तिकस्य चित्तस्योत्पादयितुमशक्यत्वात्~। उपप्लवावस्थायां तन्निमितमिति चेत्, न, अनुपप्लवस्यापि तत्साध्यत्वात्~। नहि शरीरनिरपेक्षा तत्सिद्धिः सम्भवति; योगाभ्याससाध्यत्वात्तस्य~। अन्यथाऽन्योन्याश्रयप्रसङ्गात्~। शरीरादिनिवृत्तावनुपप्लवः चित्तस्य$^5$, अनुपप्लवे च तस्मिन् शरीरादिनिवृत्तिरिति~। अथ शरीरादिकमपि चित्तविलसितमात्रं न तु ${}^6$वास्तवमित्यभिप्रायः\renewcommand{\thefootnote}{२}\footnote{प्रवृत्तिविज्ञानोपादानमालयविज्ञानसन्तानः पूर्वपूर्वतज्जातीयज्ञानोपादेयः स एवैको मोक्षोऽनुवर्तते इति मतमुत्थाप्य निराकरोति ये त्विति~। उपप्लवः संसारः~। उपप्लवेति \textendash\ अनुपप्लवावस्थायां शरीरं विना पूर्वपूर्वालयविज्ञानेनैवोत्तरोत्तरं तदुत्पाद्यते इत्यर्थः~। अनुपप्लवस्यापीति \textendash\ उपप्लवध्वंसो वा अनुपप्लवः~। अविद्यमान उपप्लवो यत्रेति चित्तविशेषो वा; उभयत्रापि शरीरं कारणमित्यर्थः~। कि. प्र. व.~।}; तत्र वक्ष्यते~।

\renewcommand{\thefootnote}{३}\footnote{वेदान्तिनाभिति \textendash\ वस्तुतो ब्रह्माद्वैतसाक्षात्कारादविद्यानिवृत्तौ विज्ञानसुखात्मकः केवलमात्माऽपवर्गे वर्तत इत्येकदण्डिमतमप्ययुक्तम्~। स्वप्रकाशसुखात्मकस्य ब्रह्मणो नित्यतया मुक्तसंसारिणोऽ \textendash\ \rule{0.4\linewidth}{0.5pt}}वेदान्तिनामपि अविद्यायां निवृत्तायां केवलमात्मैवापवर्गे वर्तत इति मते न नो

\blfootnote{1 निर्दिष्य$^\circ$ \textendash\ क~। 2 ' दुःखेमुत्पद्यते ' इति ' प्रकाशसम्मतः ' पाठः~। 3 ' तु ' ' मु. कि.' पुस्तके नास्ति~। 4  ${}^\circ$स्तमेनमग्रे  \textendash\ जे~। 5 नुपप्लुते  \textendash\ मु. कि.~। 6 बाह्य  \textendash\ जे~।}

\newpage
\blfootnote{विशेषापातात् पुरुषप्रयत्नं विनापि तत्सत्त्वेन तस्यापुरुषार्थत्वात्~। अविद्यानिवृत्तिमात्रस्य प्रयत्नसाध्यत्वेऽप्यपुरुषार्थत्वात्~। क्रि. प्र. व.~॥}
\noindent
विवादः ${}^1$ न\renewcommand{\thefootnote}{१}\footnote{न पुत्र इति \textendash\ एषा श्रुतिः पुत्रस्यात्मसम्बन्धेन प्रियत्वं बोधयन्त्यात्मनः स्वाभाविकप्रियत्वं बोधयति~। यदन्वये तु यत्प्रतीयते यद्विरहे तु यन्न प्रतीयते तत्तस्य स्वाभाविकमिति लोकसिद्धव्युत्पत्तेः~। आनन्दं ब्रह्मणो रूपमित्येकवाक्याच्चात्मनः प्रियात्मकानन्दरूपत्वं प्रतीयते~। न चात्मने इति तादर्थ्ये चतुर्थ्यनुपपत्तिः, ' यूपाय दारु ' इत्यत्र दारुस्वरूपयूपार्थत्ववदुपपत्तेरित्यर्थ~। कि प्र. व.~।} पुत्राय प्रियो भवति, आत्मने वै पुत्रः प्रियो$^2$ भवति इत्यादि श्रौतोपपत्तिबलात् सर्वस्यात्मौ$^3$पाधिकप्रियत्वं स्वभावतश्चात्मैव ${}^4$प्रियो इति पुनरवशिष्यते तत्रापि वक्ष्यते~। \renewcommand{\thefootnote}{२}\footnote{दुःखसाधनशरीरविनाशे नित्यनिरतिशयसुखाभिव्यक्तिर्मुक्तिरिति भाट्टं मतं निराकरोति तोतातितास्त्विति~। ईश्वरे शरीरं विना नित्यमपि न ज्ञानमित्यङ्गीकृत्य मुक्तस्य भोगायतनं विना भोग इत्यभ्युपगमात् त्रपा~। कारणं विनापि कार्यमिति विरोधः~। शरीरं विनापि ज्ञानसत्त्वेनानभिमतेश्वरसिद्धिप्रसक्त्या भयमित्यर्थः~। कि प्र. वः~।}तोतातितास्त्वकार्यमपि ईश्वरज्ञानं शरीरमन्तरेणानिच्छन्तः कार्यमेव सुखज्ञानमपवर्गेऽस्तीति वदन्तः त्रपा, विरोधो, भयमिति त्रयमपि ${}^5$त्यक्तवन्तः~। एतेन पारतन्त्र्यं बन्धः, स्वातन्त्र्यं च मुक्तिरित्यप्यपास्तम् ${}^6$~। न हि पारतन्त्र्यं स्वरूपतो हेयमपि तु दुःखहेतुतया स्वातन्त्र्यमपि यदि दुःखतत्साधननिवृत्तिस्तदा ओमित्युच्यते~। \renewcommand{\thefootnote}{३}\footnote{ऐश्वर्यमिति \textendash\ अणिमाद्यष्टसिद्धिरित्यर्थः कि प्र. व.~।}ऐश्वर्यं चेत् कार्यतया तदपि साधनपरतन्त्रं क्षयि चेति दुःखाकरत्वाद्धेयमेवेति~। तस्मादनिष्टनिवृत्तिरात्यन्तिकी निःश्रेयसमिति~।

${}^7$नन्वपुरुषार्थोऽयं ${}^8$सुखस्यापि\renewcommand{\thefootnote}{४}\footnote{सुखस्यापीति \textendash\ तुल्यायव्ययत्वाद्दुःखाभावोऽपि न पुरुषार्थ इत्यर्थः~। बहुतरेति \textendash\ ननु तथाप्यावश्यकत्वेन दुःखस्यैव हेयत्वं सुखस्य निरुपाधीच्छाविषयत्वात्; अन्यथा दुःखाननुविद्धतया तस्य काम्यत्वेन स्वतः पुरुषार्थत्वविरोधः ? मैवम्; सुखमनुद्दिश्यापि दुःखभीरूणां दुःखहानार्थं प्रवृत्तिदर्शनेन दुःखाभावस्यैव स्वतः पुरुषार्थत्वात्~। न हि दुःखाभावदशायां सुखमस्तीत्युद्दिश्य दुःखाभावार्थं प्रवर्तन्ते; वैपरीत्यस्यापि सुखत्वेन सुखस्यापुरूषार्थत्वापत्तेः~। अतो दुःखाभावदशायां सुखं नास्तीति ज्ञानं न दुःखभावार्थिनः प्रवृत्तिप्रतिबन्धकम्~। दुःखभीरुणां सुखालिप्सुनां मोक्षेऽधिकारादिति भावः~। कि. प्र. व.~॥} हानेरिति चेन्न~। बहुतरदुःखानुविद्धतया सुखस्यापि प्रेक्षावद्धेयत्वात्; मधुविषसम्पृक्तान्नभोजनजन्यसुखवत्~। तथापि दुःखोच्छितिरपुरुषार्थ\renewcommand{\thefootnote}{५}\footnote{अपुरुषार्थत्वं कृत्यसाध्यत्वम्~। कि. प्र. व.~॥ \rule{0.4\linewidth}{0.5pt}};

\blfootnote{1 'न वाऽरे पुत्राणां कामाय पुत्राः प्रिया भवन्ति, आत्मनस्तु कामाय पुत्राः प्रिया भवन्ति'  \textendash\ इत्यधिकं ३ पु.~। 2 सर्वं प्रियं  \textendash\ पा. ३ पु~। 3 त्मौपाधिक  \textendash\ कि~। 4 प्रियो भवतीति  \textendash\ मु. कि; जे~। 5 त्यक्तवन्तश्च \textendash\ क~। 6 गतार्थम्  \textendash\ जे~। 7 तथाप्यपुरुषार्थोऽयं  \textendash\ जे~। 8 सुखस्यापि पुरुषस्यापि  \textendash\ ते~।}

\newpage
\noindent
अनागतस्य निवर्तयितुमशक्यत्वात्, वर्तमानस्य च पुरुषप्रयत्नमन्तरेणैव विरोधिगुणान्तरोपनिपातनिवर्तनीयत्वादतीतस्याप्यतीत्वादेवेति चेत्, न, हेतूच्छेदे पुरुषव्यापारात् प्रायश्चित्तवत्~। तथा हि मिथ्याज्ञानं \renewcommand{\thefootnote}{१}\footnote{सवासनमिति \textendash\ वासना \textendash\ तज्जन्यः संस्कारः~। तच्चेति \textendash\ सवासनं मिथ्याज्ञानम्~। कि. प्र. व.}सवासनम् इह संसारमूलकारणम्; तच्च तत्त्वज्ञानेन विरोधिना निवर्त्यते~। तन्निवृत्तौ रागाद्यपाये प्रवृत्तेरपायाज्जन्माद्यपायः, तथा च दुःखसन्तानोच्छेदः~। तच्च तत्त्वज्ञानं पुरुषप्रयत्नसाध्यमिति~। 

किं पुनरत्र प्रमाणम् ? ' दुःखसन्ततिरत्यन्तमुच्छिद्यते सन्ततित्वात्, प्रदीपसन्ततिवद् ' इत्याचार्याः~। पार्थिवपरमाणुगतरूपादिसन्ताने नानैकान्तिकमिदमिति चेत्; न; सर्वात्मगतदुःखसन्ततेः$^1$ पक्षीकरणे फलतस्तस्यापि पक्षेऽन्तर्भावात्~। न हि सर्वमुक्तिपक्षेसर्वोत्पत्तिमन्निमित्तस्यादृष्टस्याभावात् तदुत्पत्तौ बीजमस्ति~। न च सर्वभोक्तृणामपवृक्तौ$^2$ ${}^3$तदुत्पत्तेः प्रयोजनमस्ति~। न हि बीजप्रयोजनाभ्यां विना कस्यचिदुत्पत्तिरस्ति~। सर्वमुक्तिरित्येव नेष्यत इति चेत्, तर्हि य ${}^4$एवात्र नापवृज्यते तस्यैव ${}^5$दुःखसन्तानेनानैकान्तिकमिदं ${}^6$किमुदाहरणान्तरगवेषणया ? एवमस्तु, न चोदाहरणमादरणीयमिति चेत्, न; असिद्धेः~। सिद्धौ वा संसार्येकस्वभावा एव केचिदात्मान इति स्थिते ' अहमेव यदि तथा स्यां तदा मम विपरीतप्रयोजनं पारिव्राजकम् ' इति शङ्कया न कश्चित्तदर्थं ब्रह्मचर्यादिदुःखमनुभवेत्~। अथ यदि ${}^7$सर्वदुःखसन्ततिनिवृत्तिर्भविष्यति तर्हि इयता कालेन किं नाभूत् ? एकैकस्मिन् कल्पे यद्यैकैकोऽप्यपवृज्येत ${}^8$तथाप्युच्छिन्नः संसारः स्यात्~। कल्पानामनन्तत्वादनन्ता$^9$ एव ह्यपवृक्ताः, सत्यम्, न तु सर्वे, संसारस्य प्रत्यक्षसिद्धत्वात्~। नन्वेतदेव न स्यादित्युच्यते इति चेत्, न; कालनियमे प्रमाणाभावात्~। न च सर्वोत्पत्तिमन्निमित्तादृष्टानुत्पत्तौ सर्वमुक्ते${}^10$रनुपपत्तिः, अपवर्गस्य भोगतत्साधनेतरत्वात्~। नह्यदृष्टनिवृत्तिरप्यदृष्टान्तरसाध्या,$^11$ एकस्याप्यपवर्गप्रसङ्गादिति$^12$~।

स्यादेतत् \renewcommand{\thefootnote}{२}\footnote{आदिमतीति \textendash\ तथा च तद्व्यतिरेकमादाय केवलव्यतिरेकिणा सत्प्रतिपक्षत्वमादिमत्त्वं \rule{0.4\linewidth}{0.5pt}}आदिमती प्रदीपसन्ततिर्निवर्तते, दुःखसंन्ततिस्तु अनादिरियमनुवर्तिष्यत इति चेत्, न; मूलोच्छेदानुवृत्त्योः प्रयोजकत्वात्~। मूलोच्छेदाद्धि सन्ततेरुच्छेदः,

\blfootnote{ 1 सन्ततिपक्षीकरणे  \textendash\ मु. कि~। 2 मप्रवृत्तौ  \textendash\ जे~। 3 तदुत्पत्ते  \textendash\ जे~। 4 एव  \textendash\ मु. कि.~। 5 दुःखसन्ताने  \textendash\ क~। 6 मुदाहरणगवेषणया  \textendash\ मु. \textendash\ कि. 7 सर्वमुक्तिर्भविष्यति  \textendash\ पा. ४. पु.~। 8 उच्छेदः संसारस्य स्यात्  \textendash\ क~। 9 सत्यमनन्ता एव ह्यपवृक्ताः न तु सर्वे सम्प्रति  \textendash\ मु. कि.~। 10 दृष्टानुत्पत्तौ सर्वमुक्तेरनुत्पत्तिः  \textendash\ क. मु. कि.~। 11 अदृष्टसाध्या  \textendash\ क.~। 12 प्रसङ्गात्  \textendash\ क~।}

\newpage
\blfootnote{चोपाधिरित्यर्थः~। अनादिरिति  \textendash\ स्वाश्रयध्वंसव्याप्यप्रागभावप्रतियोगिमात्रवृत्तिजातिमत्त्वम*नादित्वम्~॥ कि. प्र. व.}
\noindent
मूलनुवृत्तौ चानुवृत्तिः~। अन्यथात्वादिमत्त्वाविशेषे कालानियमो न स्यात्~। काचित्प्रदीपसन्ततिः प्रहरमनुवर्तते, काचिदहोरात्रमित्याद्यनियमो हि तैलादिमूलोच्छे$^1$दानियमप्रयुक्त इति~।

'\renewcommand{\thefootnote}{१}\footnote{स्थापनायां विपक्षे बाधकमाह \textendash\ अशरीरमिति~। वावसन्तमिति यङ्लुकि, तेन संसारावस्थायां क्षणमात्रमशरीरतया नान्यथासिद्धिः~। यद्वा वा एवार्थे, तेनाशरीरमेव वसन्तमित्यर्थः~। वावेति सम्बोधनम्~। तेनाशरीरमेव सन्तं वर्तमानम्, वर्तमानकालस्य क्षणादधिकत्वे नोक्तदोषः~। कि. प्र. व.~।}अशरीरं वा वसन्तं प्रियाप्रिये न स्पृशतः' ${}^2$इत्याद्यागमाच्चायमर्थोऽध्यवसेयः~।

स्यादेतत्, तत्त्वज्ञानं हि विरोधितया समूलं मिथ्याज्ञानमुन्मूल्यन्निःश्रेयसहेतुः~। न चोपपत्त्या ${}^3$शब्देन तर्केण वा जनितमिदं ${}^4$परोक्षमपरोक्षं मिथ्याज्ञानं निवर्तयितुमुत्सहते, \renewcommand{\thefootnote}{२}\footnote{दिङ्मोहेति \textendash\ सूर्योदयेन प्राचीनमनुमायापि मुह्यन्तीत्यर्थः~। कि. प्र. व.~।}दिङ्मोहादौ तथानुपलब्धेः~। ${}^5$अतोऽपरोक्षमव्युत्थायि\renewcommand{\thefootnote}{३}\footnote{अव्युत्थायीति \textendash\ व्युथातुं भ्रमितुं शीलमस्येति व्युत्थायि भ्रान्तं न तथेत्यर्थः~। बलवत्तरम् बहुतरसंस्काराधायकं च शरीरादावात्मधीः प्रत्यक्षापरोक्षेण शब्दानुमानजन्येन ज्ञानेन निवर्तयितुमशक्येति तन्निवर्तनक्षमा शरीरभिन्नात्मधीरध्यक्षा स्यात् साधनशास्त्रादित्यर्थः~। तत्त्वज्ञानमिति शरीरादिभिन्नात्मभावनातः साक्षात्काररूपमित्यर्थः~। कि. प्र. व.~। \rule{0.4\linewidth}{0.5pt}} बलवत्तरं तत्त्वज्ञानं तन्निवर्तनसमर्थम्~। ${}^6$तच्च कुतो ${}^7$भविष्यतीत्यत आह {\knu तच्चेति} \textendash\ ईश्वर$^8$चोदना उपदेशो वेद इति यावत्~। तेनाभिव्यक्तात् \textendash\ प्रतिपादिताद्धर्मात्$^9$~। अयमर्थः \textendash\ शास्त्रेण पदार्थान् विविच्य श्रुतिस्मृतीतिहासपुराणोपदिष्टयोगविधिना दीर्घकालादरनैरन्तर्यसेवितान्निवृत्तिलक्षणाद्धर्मादेव तत्त्वज्ञानमुत्पद्यते यतोऽपवृज्यते~। नह्युपपत्त्या विना विवेकः~। न च${}^10$विवेचनाद्विना उपदेशमात्रेणाश्रद्धामलक्षालनम्~। $\rightarrow$ ${}^11$न च तेन विना $\leftarrow$ शङ्काशू$^12$कत्यागः~। न च तमन्तरेण निवर्तको धर्मः~। न च तेन विना दृढभूमिविभ्रमसमुन्मूलसमर्थश्च तत्त्वसाक्षात्कार इति~।

\blfootnote{*नन्वनादित्वं प्रागभावाप्रतियोगित्वं दुःखसन्ततावसिद्धमत आह अनादित्वमिति~। येन रूपेणानादित्वं तदिह स्वपदेन विवक्षिते, तदाश्रयस्य यो ध्वंसस्तद्व्याप्यो यः प्रागभावस्तत् प्रतियोगिमात्रवृत्तिजातिमत्त्वमित्यर्थः~॥ कि. प्र. व्या भ.

1 च्छेदादिनियम  \textendash\ क~। 2 इत्यागमा  \textendash\ मु. कि.~। 3 शब्देन वा जनित  \textendash\ मु. कि., क.~। 4 परोक्षं तत्त्वज्ञानं  \textendash\ जे~। 5 ततो  \textendash\ पा. ३. पु.~। 6 तत्कृतो पा. ३  \textendash\ ६~। 7 भवति  \textendash\ क.~। 8 नोदना  \textendash\ मु. कि.~। 9 धर्मादेवेत्यर्थः  \textendash\ क.~। 10 विवेकादिना  \textendash\ क.~। 11 $\rightarrow$ $\leftarrow$ एतच्चिह्नान्तर्गतः पाठः जे पुस्तके नास्ति~। 12 शूकस्याभ्यायोगः  \textendash\ जे~।}

\newpage
\renewcommand{\thefootnote}{१}\footnote{ननु तत्त्वज्ञानवत्कर्मापि श्रुत्या बोधितमिति तत्कुतो नोद्दिष्टमित्यत आह \textendash\ एतेनेति \textendash\ सत्त्व आत्मा तस्य शुद्धिस्तत्त्वज्ञानोत्पत्तिप्रतिबन्धकदुरितनिवृत्तिः तद्द्वारा परम्परया कर्म मोक्षकारणम्~। तत्त्वज्ञानं तु सन्निपत्य कर्मापेक्षया सन्निधायोपकारकम्, कर्मानन्तरं तदुत्पत्तेरिति~। कि. प्र. व~॥}एतेन सत्त्वशुद्धिद्वारेणारादुपकारकं कर्म, सन्निपत्त्योपकारकं च ${}^1$तत्त्वज्ञानमिति मन्तव्यम्~। न तु तुल्यकक्षतया तत्समुच्चयः, नापि ज्ञानेन धर्मो जन्यते विहितत्वादिति धर्मस्यैव प्राधान्यम्~। दृष्टद्वारेणोपपत्तावदृष्टकल्पनानवकाशात्~। अन्यथा भेषजादिष्वपि तथा कल्प्येत~। उपपत्तिविरुद्धश्च $\rightarrow$ ${}^2$ज्ञानकर्मसमुच्चयः $\leftarrow$ काम्यनिषिद्धयोस्त्यागादेव समुच्चयानुपपत्तेः~। नापि ${}^3$असङ्कल्पितफलकाम्यकर्मसमुच्चयः, चतुर्थाश्रमविधिविरोधात्~। यावन्नित्यनैमित्तिककर्मसमुच्चयस्यापि तत एवानुपपत्तेः~। यत्याश्रमविहितेन कर्मणा ${}^4$ज्ञानसमुच्चय इत्यपि नास्ति, तदभावेपि ${}^5$गृहस्थस्यापि ज्ञाने सति मुक्तेः~। यतः स्मरन्ति \textendash\

\begin{quote}
{\qt 'कर्मणैव हि संसिद्धिमास्थिता जनकादय'} इति\\
{\qt 'न्यायागतधनस्तत्त्वज्ञाननिष्ठोऽतिथिप्रियः~।\\
श्राद्धकृत् सत्यवादी च गृहस्थोऽपि ${}^6$विमुच्यते~॥} इति च~।
\end{quote}
 
न च साध्यस्यावैचित्र्ये साधनवैचित्र्यमुपपद्यते~। न च स्वर्गवदपवर्गेऽपि प्रकारभेदः सम्भवति~। तस्मात् तत्त्वज्ञानमेव निःश्रेयससाधनम्~। कर्माणि त्वनुत्पन्नतत्त्वज्ञानस्य तत्त्वज्ञानार्थिनस्तत् ${}^7$प्रतिबन्धकाधर्मतिरोधानद्वारेण प्रायश्चित्तवदुपयुज्यन्ते~। उत्पन्नतत्त्वज्ञानस्य त्वन्तरालब्धवृष्टेः प्रारब्धकारीरीपरिसमाप्तिवत् प्रारब्धाश्रमधर्मसमापनं लोकसङ्ग्रहार्थमिति युक्तमुत्पश्यामः~। \renewcommand{\thefootnote}{२}\footnote{ननु यतोऽभ्युदयनिःश्रेयससिद्धिः स धर्मं इति सूत्रे धर्मस्य मोक्षहेतुताबोधकत्वविरोध इत्यत आह \textendash\ एतेनेति~। सूत्रमप्यभ्युदयमात्रसाधकधर्मपरतयैव व्याख्येयमित्यर्थः~। अभ्युदयोऽत्र तत्त्वज्ञानम्~। तद्वचनादिति तेनेश्वरेण वचनात्प्रणयनादाम्नायस्य प्रामाण्यमित्यर्थः~। अन्यथेति यद्येकोऽभ्युदयसाधकोऽन्यश्च निःश्रेयससाधक इत्यर्थः~। न चैकस्येव धर्मस्योभयसाधकतायां नोक्तदोषः, उक्तयुक्तेस्तत्वज्ञानोत्पादनेनैवान्यथोपपत्तौ धर्मस्य निःश्रेयसहेतुत्वे मानाभावादिति भावः~। किः प्र. व~॥ \rule{0.4\linewidth}{0.5pt}}एतेन 'अथातो धर्मं व्याख्यास्यामः' 'यतोऽभ्युदयनिःश्रेयससिद्धिः स धर्मः', ${}^8$तद्वचनादाम्नायप्रामाण्यमिति त्रिसूत्री व्याख्याता~। अन्यथा व्याख्याने हि यतोऽभ्युदयेति प्रत्येकसमुदायाभ्यामुभयत्राप्यव्यापकं स्यात्~। ${}^9$यतोऽभ्युदयसिद्धिः स धर्म इत्येतावतैव लक्षणे सिद्धे पारम्पर्येण निःश्रेयसेऽप्यस्य हेतुत्वं प्रतिपादयितुं निःश्रेयसङ्ग्रहणमिति~॥

\blfootnote{1 च  \textendash\ ज्ञानमिति \textendash\ मु. कि 2 $\rightarrow$ $\leftarrow$ एतचिह्नान्तर्गतः पाठः जे पुस्तके नास्ति~। 3 असङ्कल्पितफलकर्मसमुच्चये चतुर्थाश्रमविरोधात्  \textendash\ पा ३. पु; चतुर्थाश्रमविरोधात्  \textendash\ मु. कि.~। 4 ज्ञानस्य  \textendash\ जे. क.~। 5 गृहस्थस्य ज्ञाने मु. कि; क~। 6 मुच्यते \textendash\ जे. क~। 7 ${}^\circ$काधर्मापनयद्वारेण  \textendash\ पा ३. ६~। 8 ${}^\circ$दाम्नायसिच्चैः प्रामाण्य$^\circ$\textendash\ क~। 9 तद्यतो पा. २ पु~।}

\newpage
\indent
\hangindent=2cm {\knu (३) अथ के द्रव्यादयः पदार्थाः किञ्च तेषां ${}^1$साधर्म्यं वैधर्म्यं चेति~॥}

\renewcommand{\thefootnote}{१}\footnote{एवमिति~। अभिधेयं द्रव्यादि; प्रयोजनं निःश्रेयसम्; सम्बन्धः तत्त्वज्ञाननिःश्रेयसयोर्हेतुहेतुमद्भावः~। कि. प्र. व~॥}एवं प्रतिपन्नप्रयोजनाभिधेयसम्बन्धो जिज्ञासुः पृच्छति \textendash\ अथेति~। ${}^2$अथ कानि द्रव्याणि कियन्ति च; के च गुणाः कियन्तश्च; कानि कर्माणि \renewcommand{\thefootnote}{२}\footnote{अग्रे विभागस्यापि वचनातद्विषयां जिज्ञासामाह कियन्तीति~। विशेषाणामानन्त्यात् समवायस्यैकत्वात् तत्र न विभागे जिज्ञासा~। कि. प्र. व~॥}कियन्ति च; किं सामान्यं कतिविधं च; के विशेषाः कः समवाय इत्यर्थः~। किं च तेषामिति \textendash\ सामान्यतो विशेषतश्च पदार्थानां द्रव्याणां गुणानां कर्मणामित्यादि ${}^3$नेयम्~। ${}^4$चकारौ\renewcommand{\thefootnote}{३}\footnote{'अथ के द्रव्यादयः पदार्थाः किं च तेषां साधर्म्यं चे' ति पाठो वर्धमानोपाध्यायसम्मतः~॥ चकारादिति \textendash\ किं चेति चकारः प्रश्नसमुच्चये वैधर्म्यं चेति चकारः साधर्म्यसमुच्चय इत्यर्थ इति व्याख्यानात्~॥} समुच्चये~। \renewcommand{\thefootnote}{४}\footnote{ननु द्रव्यादीनां लक्षणात्मकस्य साधर्म्यादेर्लक्षणं किन्न पृष्टमित्यत आह \textendash\ साधर्म्येति~। एष्वेव \textendash\ द्रव्यादिष्वेवेत्यर्थः~। तथा चं द्रव्यादेः साधर्म्यादेश्च मिथो विशेषणविशेष्यभावाल्लक्ष्यलक्षणभाव इत्यर्थः~। यद्वा ननु द्रव्यादीनां लक्षणप्रश्नः कुतो नेत्यतआह \textendash\ साधर्म्येति~। एष्वेव \textendash\ लक्षणेष्वेव~। तथा च साधर्म्यप्रश्नेन तत्प्रश्न इत्यथः~॥ कि. प्र. व.\\ \rule{0.4\linewidth}{0.5pt}}साधर्म्यवैधर्म्ययोरेतेष्वेवान्तर्भूतत्वात् पृथग्लक्षणार्थमपि न प्रश्नः~॥

\hangindent=2cm {\knu (४) तत्र द्रव्याणि पृथिव्यप्तेजोवाय्वाकाशकालदिग्मनां$^5$सीति सामान्यविशेष$^6$ सञ्ज्ञयोक्तानि ${}^7$नवैव~। ${}^8$तद्व्यतिरेकेणान्यस्य सञ्ज्ञानभिधानात्~॥}

तत्रेति~। तत्र \textendash\ तेषु द्रव्यादिषु वक्तव्येषु द्रव्याणि ${}^9$पृथिव्यादीनि~। यद्यपि विभागस्य न्यूनाधिकरसङ्ख्याव्यवच्छेदपरत्वादेव नवत्वं लब्धं तथापि स्फुटार्थं नवग्रहणम्~। एवकारश्च${}^10$विप्रतिपत्तिनिराकरणार्थः~। सामान्यसञ्ज्ञा द्रव्यमिति~। विशेषसञ्ज्ञा पृथिवीत्यादिका~। तथोक्तानि सूत्रकृतेति शेषः, अवगताप्तभावस्य तस्योक्तेरागमत्वात्~। अनवगताप्तभावस्यापि लोकप्रसिद्धार्थानुवादकत्वात्~। लोके च तावतामेव सामान्यतो

\blfootnote{1 साधर्म्यं चेति  \textendash\ कि~। 2 कानि २. ३. पु०~। 3 ज्ञेयम् पा. १ पु.~। 4 चकारौ मिथः समुच्चये  \textendash\ मु. कि~। 5 मनांसि \textendash\ कि; मनांसि च दे~। 6 सञ्जोक्तानि \textendash\ कि. व्यो. जे~। 7 नवैवेति \textendash\ कं~। 8 तद्व्यतिरेकेण सञ्ज्ञान्तरानभिधानात् \textendash\ व्यो, कि, दे, तद्व्यतिरेकेण सञ्ज्ञानभिधानात् \textendash\ जे~। 9 'पृथिव्यादीनि' इति जे पुस्तके नास्ति~। 10 परविप्रतिपत्ति पा. २ पु.~।}

\newpage
\noindent
विशेषतश्च व्यवहारात्~। किं पुनरत्र प्रतिषिद्ध्यते नवैवेति ? नह्यनवगतस्य प्रतिषेधः सम्भवति इति, उच्यते; द्रव्यस्य सतो नवबाह्यत्वम्, नव बाह्यस्य ${}^1$सतो द्रव्यत्वं वा; तथा च प्रतिपन्नस्यैव प्रतिपन्ने प्रतिषेध इति न किञ्चिद्दुष्यति~। अतः परं न शङ्का न चोत्तरम्~। \renewcommand{\thefootnote}{१}\footnote{तथाहीति~। इदं द्रव्यमिति सूत्रपाठे एभ्यो नवभ्योऽधिकं तमो द्रव्यमिति योजना~। इदमेभ्य इति \textendash\ इदं द्रव्यं सुवर्णमेभ्यः पृथिव्यादिभ्योऽधिकं स्यादित्यर्थः~। कि. प्र. व.~॥ \rule{0.4\linewidth}{0.5pt}}तथा हि इदं द्रव्यमेभ्योऽधिकं स्यादिति ${}^2$वाऽऽशङ्केत इदमेभ्योऽधिकं द्रव्यं स्यादिति वाऽऽशङ्केत~। ${}^3$प्रथमे ${}^4$आधिक्यं निराकरिष्यामो यथा सुवर्णस्य~। ${}^5$द्वितीये द्रव्यत्वं निराकरिष्यामो यथा तमसः~। अतः परं न शङ्का ${}^6$न चोत्तरम्; धर्मिण एव बुद्ध्यनारोहात्~। यदि कथंञ्चिद् बुद्धिमारोक्ष्यते तदाऽस्माभिरप्युक्तेष्वेवान्तर्भावयिष्यते~। अनन्तर्भावे वा द्रव्यत्वं तस्य निराकरिष्यत इत्यभिप्रायवानाह \textendash\ {\knu 'तदव्यतिरेकेण सञ्ज्ञान्तरानभिधानात्'} इति सूत्रकृतेति शेषः~। लोकेनेति वा~।

स्यादेतत्~। अन्धकारस्तावदनुभवसिद्धतया दुरपह्नवः~। न च सामान्यविशेषसमवायेष्वन्यतमत्तमः; तेषां व्यञ्जकवैचित्र्येऽपि व्यक्त्याश्रयसम्बन्धिनामुपलम्भमन्तरेणानुपलम्भनियमात्~। उपलभ्भे वा तत्त्वव्याघातात्~। न$^7$ च कर्म, संयोगविभागयोरकारणत्वात्~। नह्यन्धकारेण किञ्चित् कुतश्चिद् विभज्य केनचित् संयोज्यते~। अतथाभूतस्य च तल्लक्षणानुपपत्तेरतत्त्वात्~। न गुणः; द्रव्यासमवायात्~। द्रव्यासमवेतं ह्यसमवेतमेव स्यादद्रव्यसमवेतं वाः उभयथापि गुणत्वव्याघातः~। सामान्यवतः स्वतन्त्रस्य द्रव्यत्वापत्तेः~। निःसामान्यस्य ${}^8$गुणलक्षणव्याघातात्~। गुणकर्मणोर्निर्गुणतया गुणस्य तत्र समवायविरोधात्~। द्रव्यासमवाय एवास्य कथम् ? इति चेत्; इत्थम् : \textendash\ न दिक्कालमनसामयं तेषां विशेषगुणविरहात्~। सामान्यगुणस्य चाश्रयसहोपलम्भ$^9$ नियमे तदप्रत्यक्षतायामप्रत्यक्षत्वप्रसङ्गात्~। नात्मनो बाह्यकरणप्रत्यक्षत्वादिदन्तास्पदत्वाच्च~। नापि नभोनभस्वतोः, चाक्षुषत्वात्~। चाक्षुषता हि गुणानां रूपिद्रव्यसमवायेने व्याप्ता~। तच्च रूपित्वं गगनपवनाभ्यां व्यावर्तमानं चाक्षुषगुणसम्बन्धमपि व्यावर्तयति~। न${}^10$तेजसः, तत्प्रतीतौ तद्विरोधित्वात्, शैत्यवत्~। गुणिनः स्वगुणप्रतीतिपरिपन्थित्वे गुणस्य नित्यमनुपलम्भप्रसङ्गात्~। सत्याश्रये तेनैव प्रतिबन्धात्~। ${}^11$स्वाश्रयेऽसति गुणिनि गुणस्यासत्वात्; तत्सहचरितगुणान्तरानुपलब्धेश्च~। न ताव\textendash

\blfootnote{1 सतो ' जे ' पुस्तके नास्ति~। 2 वा इदमेभ्यो  \textendash\ मु. कि~। 3 द्वितीये तु  \textendash\ जे~। 4 अधिक्ये  \textendash\ जे~। 5 द्वितीये  \textendash\ जे~। 6 ' न चोत्तरम् ' \textendash\ जे पुस्तके नास्ति~। 7 न कर्म इति प्रकाशसम्मतः पाठः~। 8 गुणत्व  \textendash\ क~। 9 नियमेन  \textendash\ मु. कि~। 10 तेजसः तद्विरोधित्वात्  \textendash\ मु. कि~। 11 असति गुणिनिमु. कि; स्वाश्रये असति गुणस्यासत्त्वात्  \textendash\ जे, क~।}

\newpage
\noindent
च्छाया तेजसो रूपमेव; तद्रूपस्य शुक्लभास्वरत्वनियमात्~। ${}^1$न चेन्द्रनीलप्रभावादाश्रयोपाधेरतथाभूतमिदमाभातीति साम्प्रतम्; शैलभूतलस्फटिकपद्मरागाद्याश्रयरूपाननुविधानात्~। तस्माद् गुणान्तरमेवेदं तेजस इति वाच्यम्~। तथा च ${}^2$तद्ग्रहे तदग्रहः तद्विरह एव, तद्  \textendash\ ग्रहणमिति विपरीतम्; इति महत्यनुपपत्तिः~।$^3$ $\rightarrow$ नापि पाथःपृथिव्योः, आलोकनिरपेक्षचक्षुर्ग्राह्यत्वात् $\leftarrow$~। पार्थिवमेवेदमारोपितं रूपमित्यपि न समीचीनम्; बाह्यालोकसहकारिविरहे चक्षुषस्तदारोपेऽप्यसामर्थ्यात्~। तदेव हि ${}^4$धर्म्यन्तरे ${}^5$समारोप्येत पित्तपीतिमवत्~। तत्रैव ${}^6$वा नियतदेशेऽनियतदेशत्वम्~। \renewcommand{\thefootnote}{१}\footnote{नेदीयसीति \textendash\ गोलकसन्निकृष्टाणुनि यथाधिकदेशत्वमारोप्यत इत्यर्थः~। तत्प्रथनं रूपसाक्षात्कार इत्यर्थः~। कि. प्र. व.~।}नेदीयस्यणीयस्यपि महत्त्ववत्~। उभयथाऽपि ${}^7$तत्प्रथन मन्तरेणानुपपत्तिरेव, एकत्रारोप्यत्वादन्यत्रारोपविषयत्वात्तस्यैव~। न चालोकमन्तरेण रूपग्रहणे चक्षुषः सामर्थ्यमित्युक्तम्~। न चारोप्यारोपविषयाप्रथने भ्रान्तिसम्भवः~। न चोभयोरन्यतरस्मिन्नव्यापृतस्यैव चक्षुषो भ्रान्तिजनकत्वम्~। न चायमचाक्षुषः प्रत्ययः, तदनुविधान \textendash\ स्यान्यथसिद्धत्वात्~। स्वप्नविभ्रमवन्मानस एवायं न चाक्षुष ${}^8$इत्येत्तदपि नाशङ्कनीयम्; निमीलितनयनस्य ${}^9$गेहेऽस्त्यन्धकारो न वेति सन्देहानुपपत्तेः~। तस्मात् क्रियावत्त्वाद्${}^10$गुणवत्वाच्च द्रव्यमेतत्~। क्रियावत्त्वादेव ${}^11$नाकाशाद्यात्मकम्~। रूपवत्त्वादेव न मनोवायू~। ${}^12$स्पर्शरहितत्वाच्च न पृथिवी, जलं, तेजो वा~। इति दशमं ${}^13$द्रव्यं प्राप्तम्; तत् कथं नवैवेति ?

न, वस्तुतोऽस्य क्रियावत्त्वे रूपवत्त्वे वा, अचाक्षुषत्वप्रसङ्गात्~। आलोकसहकारिण एव चक्षुषस्तत्र सामर्थ्याव$^14$धारणादित्युक्तम्~। न \renewcommand{\thefootnote}{२}\footnote{अद्रव्यमिति \textendash\ न विद्यते द्रव्यं समवायिकारणतयासम्बन्धि यस्य निरवयवमित्यर्थः~। रूपवत इति \textendash\ अस्य च चाक्षुषत्वादित्यर्थः~। अनित्यत्वाच्चेत्यपि द्रष्टव्यम्~। स्पर्शरहितेति \textendash\ स्पर्शवत्तयाऽऽरम्भकत्वेनानारब्धस्यास्पर्शवत्त्वानियमादित्यर्थः~। न चेति \textendash\ साम्प्रतमित्यग्रेतनेन सम्बन्धः~। स्पर्शरहितस्यापि तमसः पुरुषार्थत्वहेतुत्वेनारब्धत्वसम्भवादप्रयोजकत्वम्~। मनस्तु नारम्भकम्, तदारब्धस्य शरीरेन्द्रियहेतुत्वाभावेन वैयर्थ्यात्~। साधनावच्छिन्नसाध्यव्यापकस्य नीरूपत्वस्योपाधित्वाच्च~। अन्यथा रूपवत्त्वेनारम्भकत्वे नीरूपस्यानारब्धत्वाद्वायुरप्यनुद्भूतरूपा पृथिवी स्यात्~। अत एव तमःपरमाणुर्न द्रव्यारम्भकः स्पर्शशून्यत्वान्मनोवदित्यपास्तम्, सिद्ध्यसिद्धिव्याघाताच्च~। तस्माद् भोजकादृष्टाधीनमुद्भवत्त्वमिति तदभावान्न तमसि स्पर्शोद्भव इति भावः~। कि. प्र. व~॥ \rule{0.4\linewidth}{0.5pt}}चेदमद्रव्यं रूपिद्रव्यम्; रूपवतो मूर्तिना\textendash

\blfootnote{1 न चेदं  \textendash\ क~। 2 तदग्रहणे तद्विरह एव  \textendash\ जे~। 3 $\rightarrow$ $\leftarrow$ एतच्चिह्नान्तर्गतः पाठः ' जे ' पुस्तके नास्ति~। 4 धर्मान्तरे \textendash\ क~। 5 समारोप्यते  \textendash\ मु. कि~। 6 तत्रैव च. मु. कि~। 7 ग्रहण  \textendash\ क~। 8 इति  \textendash\ मु. कि~। 9 गेहेऽन्धकारो  \textendash\ जे~। 10 गुणसम्बन्धाच्च  \textendash\ जे; रूपवत्त्वाच्च  \textendash\ क~। 11 क्रिया  \textendash\ वत्त्वान्नाकाशाद्यत्मकम् पा. २. पु. नाकाशात्मकम्  \textendash\ क~। 12 स्पर्शविरहत्वाच्च  \textendash\ पा. १. पु~। 13 द्रव्यमिति प्राप्ते  \textendash\ मु. कि~। 14 ${}^\circ$वधारणेत्यूक्तम्  \textendash\ पा. १. पु~।}

\newpage
\noindent
न्तरीयकत्वेन निरवयवस्यः परमाणुतयाऽतीन्द्रियत्वापत्तेः~। नाप्यनेकद्रव्यं द्रव्यम्; स्पर्शरहितत्वद्रव्यत्वेनानारब्धत्वान्मनोवत्~। न च रूपवत्तया स्पर्शोऽप्यनुमास्यते; तद्रहितस्यापि पुरुषार्थहेतुत्वादारब्धं वा स्यात्~। प्रारब्धस्य चानुभवसिद्धत्वात्, मनसोऽनुपलभ्यमानधर्मस्य स्वयमनुपलभ्यमानस्य च वैयर्थ्यादेवारम्भानुपपत्तिरिति साम्प्रतम्~। रूपवत्त्वस्य प्रागेवाप्रत्यक्षत्वप्रसङ्गेनापास्तत्वात्~। ${}^1$प्रत्यक्षत्वस्य चानुभवसिद्धत्वाद् इत्येतत्स$^2$र्वमभिसन्धाय भगवान् मुनिराह ' द्रव्यगुणकर्मनिष्पत्तिवैधर्म्याद्भाभावस्तम' इति~। सोऽपि कथमालोकमन्तरेण प्रतियोगिस्मरणाधिकरणग्रह$^3$विरहेऽपि विधिमुखेन च चाक्षुषः ? इति चेत्, न; यद्ग्रहे हि यदपेक्षं चक्षुः, तदभावग्रहेऽपि तदपेक्षते~। एवं हि तदितरसामग्रीसाकल्यं स्यात्~।$^4$ $\rightarrow$ यदि चालोकग्रहे आलोकान्तरमपेक्षते$\leftarrow$ तदाऽऽलोकाभावेप्यालोकापेक्षा ${}^5$स्यात्~। न त्वेतदस्ति; प्रत्युत विरोध एव~। तस्मिन् ${}^6$सति हि तदभाव एव न स्यात्, किं तदपेक्षेण चक्षुषा गृह्येत ? दिवा च प्रतियोगिनः प्रभा$^7$मण्डलस्य ग्रहण एव प्रदेशान्तरे ${}^8$तदभावग्रह इति न किञ्चिदनुपपन्नम्~। अन्यत्रापि न रात्रिमप्रतिसन्धायान्धकारग्रहः~। रात्रिज्ञानं च न दिवसमप्रतिसन्धाय~। निरस्तैतद्वीपवर्त्तिरविरश्मिजालः कालविशेषो ह्यत्र रात्रिरित्युच्यते~। गिरिदरीविवरवर्त्तिनस्तु यदि योगिनः, न ते तिमिरावलोकिनः; तिमिरदर्शिनश्चेत्, नूनं स्मृतालोका इति~।

अधिकरणमपि दृष्टमनुमितं स्मृतं वा, ' इहेदानीमन्धकारः ' इति प्रत्ययात्~। विधिमुखस्तु प्रत्ययोऽसिद्धः~। न हि नञोऽप्रयोग एव विधिः; प्रलयविनाशावसानादिषु व्यभिचारात्~। नञर्थान्तर्भावेन वाक्यार्थे पदप्रयोग इति तु समं समाधानम्, अन्यगाभिनिवेशात्~। गतेः का गतिः ? इति चेत्; भ्रान्तिः, स्वाभाविक्याङ्गतावावरकद्रव्यानुविधानानुपपत्तेः~। प्रभातुल्यत्वे तेजःप्रभाश्रयेषु रत्नविशेषेषु छाया दिवसे न स्यात्~। छाययैव तदभिभवे बहुलतमे तमसि तेषामालोको न स्यात्~। आलोकान्तरेण वा तदभिभवे ${}^9$छायाया अपि उद्भवो न स्यात्~। तस्माद् आवरकद्रव्ये गच्छति यत्र यत्र तेजसोऽसन्निधिः, तत्र तत्र छायाग्रहणादन्यदेशतानिबन्धनो गतिभ्रम इति~।

कथं भावधर्माध्यारोपोऽभाव इति चेत्, न किञ्चिदेतत्~। सारूप्यतत्त्वाग्रहाविह, तन्निबन्धनम्, न त्वन्यत्~। दृष्टश्च दुःखाभावे${}^10$सुखाध्यारोपः, यथा ${}^11$दुःखापगमे 'सुखिनः

\blfootnote{1 तस्य  \textendash\ जे~। 2 सर्वमनुसन्धाय  \textendash\ क~। 3 विरहे विधिमुखेन मु. कि~। 4 $\rightarrow$ $\leftarrow$ एतच्चिह्नान्तर्गतः पाठः मु. कि इत्यत्र नास्ति~। 5 स्यात् यद्यालोके तदपेक्षा स्यात् नत्वेतदस्ति; मु. कि~। 6 सति तद्भाव मु. कि.~। 7 मण्डलग्रहण मु. कि~। 8 तद्ग्रह  \textendash\ क~। १ छायाया उद्भवो  \textendash\ मु. कि~। 10 सुखत्व  \textendash\ क~। 11 भारापगमे  \textendash\ पा. २. पु; भारावतारे  \textendash\ मु. कि~।}

\newpage
\begin{sloppypar}
\noindent
संवृत्ताः स्मः' इति; संयोगाभावे ${}^1$विभागाभिमान इत्यादि~। एतेन नीलिमाध्यारोपो व्याख्यातः~। शुक्लभास्वरविरोधित्वसारूप्येण तदारोपोपपत्तेः~। न चैवं ${}^2$रक्तत्वाद्यारोपप्रसङ्गोऽपि, आरोपे सति निमित्तानुसरणात्~। न तु निमितमस्तीत्यारोपः; अदृष्टादिकं चात्र नियामकमध्यवसेयम्~। स्मर्यमाणं चैतद्रूपमारोप्यते, रजतत्ववन्न गृह्यमाणम्, अतो न सहकार्यपेक्षाचोद्यमाशङ्कनीयम्, धर्मिणि निरपेक्षत्वात्~। यद्येवमारोपितं रूपं न तमो भाभावस्तु ${}^3$तम इति विनिगमनायां को हेतुरिति चेत्, उच्यते; एषा तावदनुभवस्थितिः 'तमो नीलम्' न तु 'नीलिमा तमः' इति~। न चारोपितेन वास्तवेन ${}^4$वा नीलिम्ना तमोबुद्धिव्यपदेशौ समानार्थौ, सहप्रयोगानुपपत्तेः~। नीलीद्रव्योपरक्तेषु ${}^5$वस्त्रचर्मादिषु तमोबुद्धिव्यपदेशप्रसङ्गाच्च~। अवश्यम्भावी च भाभावानुभवो निरालम्बनस्य भ्रमस्यानुपपत्तेः~। न च तमःप्रत्ययो बाध्यते~। नीलप्रत्ययस्तु बाध्यत 'इह' इति प्रत्ययवत्~। तस्माद्यत्र गुणक्रियारोपः तदन्धकारो न तु नीलमेति सुष्ठूक्तं नवैवेति~॥
\end{sloppypar}

\hangindent=2cm {\knu (५) ${}^6$गुणाः रूपरसगन्धस्पर्शसङ्ख्यापरिमाणपृथक्त्वसंयोगविभागपरत्वापरत्वबुद्धिसुखदुःखेच्छाद्वेष7प्रयत्नाश्चेति कण्ठोक्ताः सप्तदश~। ' च ' शब्दसमुच्चिताश्च गुरुत्वद्रवत्वस्नेहसंस्कारादृष्टशब्दाः इत्येव' ${}^8$चतुर्विंशति \textendash\ र्गुणाः~॥}

५ गुणान् विभजते \textendash\ गुणा इति~। रूपादयः सप्तदश \renewcommand{\thefootnote}{१}\footnote{कण्ठोक्ता इति साधारणस्वब्देनोक्ता इत्यर्थः~। कर्त्रपेक्षायमाह सूत्रकारेणेति~। अभ्युपगमेति \textendash\ साक्षादसूत्रितत्वेऽपि समानतन्त्राभिहितत्वेनाभ्युपगभ्यमानत्वादित्यर्थः~। कि प्र. व~॥ \rule{0.4\linewidth}{0.5pt}}कण्ठोक्ता सूत्रकारेण~। अभ्युपगमसिद्धान्तन्याये9नान्येऽपि सप्त सिद्धगुणभावाः, तत्र तेषां व्युत्पादनात्;${}^10$अनभ्युपगमे व्युत्पादनविरोधात्~। तथा च विभागसूत्रं न्यूनम्~। "रूपरसगन्धस्पर्शाः, सङ्ख्याः, परिमाणानि, पृथकत्वम्, संयोगविभागौ, परत्वापरत्वे, बुद्धयः, सुखदुःखे, इच्छाद्वैषौ प्रयत्नाश्च गुणाः"~। (वै. सू. १ \textendash\ १ \textendash\ ६) इति हि तत्~। अत् आह \textendash\ 'च शब्दसमुच्चिताश्च सप्त ' इति~। ${}^11$अदृष्टशब्देन धर्माधर्मयोः, संक्षेपेणाभिधानम्~। न त्वदृष्टत्वं नाम सामान्यमस्ति; कार्यकारणलक्षणानां तद्व्यवस्थापकानामभावात्~। तेन 'गुरुत्व \textendash\ द्रवत्व \textendash\ स्नेह \textendash\ संस्कार \textendash\ धर्मा \textendash\ धर्म \textendash\ शब्दाः' इत्युक्तं भवति~। एवं कण्ठोक्त्या समुच्चयेन चैकतया चतु\textendash

\blfootnote{1 विभागमात्राभिमान  \textendash\ जे; विभागत्वाभिमान  \textendash\ क~। 2 तदत्वाद्यारोपः  \textendash\ जे, क~। 3 तद्  \textendash\ जे~। 4 'वा' 'क' पुस्तके नास्ति~। 5 वस्त्रधर्मा \textendash\ क~। 6 गुणाश्च  \textendash\ कं~। 7 प्रयत्नाश्च  \textendash\ जे~। 8 चतु  \textendash\ र्विंशतिगुणाः  \textendash\ कि~। 9 ऽन्ये सप्त  \textendash\ क~। 10 अनभ्युपगमे तेषां  \textendash\ क~। 11 ननु अदृष्टशब्देन  \textendash\ क~।}

\newpage
\noindent
र्विंशतिर्गुणा व्यवहर्तव्याः, तथाविधबुद्धिविषयतया सारूप्येण, न तु सङ्ख्यायोगेन~। यथा चैतत् ${}^1$तथा वक्ष्यामः~।

\hangindent=2cm {\knu (६) ${}^2$उत्क्षेपणापक्षेपणाकुञ्चनप्रसारणगमनानि पञ्चैव कर्माणि~। गमनग्रहणाद् भ्रमणरेचनस्यन्दनोर्ध्वज्वलन$^3$ ${}^4$तिर्यक्पवननमनोन्नमनादयो ${}^5$गमनविशेषा न जात्यन्तराणि~॥}

६ कर्माणि विभजते \textendash\ उत्क्षेपणेति \textendash\ अत्रापि 'पञ्चैव' इति स्पष्टार्थम्; विभागवचनादेव पञ्चत्वसिद्धेः~। आधिक्यमाशङ्क्याह {\knu 'गमनग्रहणाद्'} इति~। कर्मपदार्थे चैतद्व्युत्पादनीयम्~।

\hangindent=2cm {\knu (७) सामान्यं द्विविधं परमपरं ${}^6$च, ${}^7$अनुवृत्तिप्रत्ययकारणम्~। तत्र परं सत्ता, महाविषयत्वात्~। सा चानुवृत्तेरेव हेतुत्वात् सामान्यमेव~। द्रव्यत्याद्यपरम्, अल्पविषयत्वात्~। तच्च व्यावृत्तेरपि हेतुत्वात् सामान्यं सह विशेषाख्यामपि लभते~॥}

७ सामान्यं विभजते \textendash\ {\knu सामान्यमिति}~। \renewcommand{\thefootnote}{१}\footnote{समानानां भाव उपाधिरपीत्यत उक्तं स्वाभाविक इति~। सोऽपि यदि स्वभावजन्यस्तर्ह्यसिद्धिः~। स्वभावाश्रितश्चोपाधिरपीत्यत उक्तम् 'अनागन्तुक' इति~। साक्षात्समवेत इत्यर्थः~। नित्यमिति एकमिति स्वरूपाभिधानमात्रं न तु लक्षणार्थमित्येके~। एकं लक्षणमिति योज्यम्~। लक्षणान्तरं चासमवायित्वे सत्यनैकसमवेतत्वमित्यन्ये~। अनेकवृत्तित्वम् \textendash\ अनेकाधारत्वम्, तच्चाभावसमवाययोरप्यस्तीत्यत उक्तम् एकम् \textendash\ असहायम्~। अभावसमवाययोश्च प्रतियोगिसम्बन्धिनौ सहायौ इत्यपरे~। अनेकवृत्तित्वं च स्वाश्रयान्योन्याभावसामानाधिकरण्यम्~। असमाविष्टजात्योर्व्यावर्तनार्थमाह \textendash\ समावेशे सतीति~। सामानाधिकरण्ये सतीत्यर्थः~। यदा सामान्यं समाविष्टमसमाविष्टं चेत्येको विभागः~। समाविष्टमपि परमपरं चेति विभक्तविभाग इत्यसमाविष्टजात्यपेक्षया समुच्चयार्थश्चकार इत्यर्थः~॥ कि. प्र. व~। \rule{0.4\linewidth}{0.5pt}}समानानां भावः स्वाभाविकोऽनागन्तुको ${}^8$बहूनां धर्मः सामान्यमित्यर्थः~। तथा च धर्मिणां बहुत्वे धर्मस्य चानागन्तुकत्वे विवक्षिते 'नित्यमेकमनेकवृत्ति सामान्यम्' इति लक्षणं सूचितम्~। तद् द्विविधम्; ${}^9$द्वैविध्यं दर्शयति \textendash\ {\knu परमपरं चेति}~। एकव्यक्तिसमावेशे सतीति चकारार्थः~। नैकव्यक्तिकं सामान्यमस्तीति आकाशादौ वक्ष्यते~। नान्यूनाननिरिक्तव्यक्तिकमिति बुद्धिरुपलब्धिर्ज्ञानमित्यादि\textendash

\blfootnote{1 तथा गुणे  \textendash\ मु. कि~। 2 वक्षेपण  \textendash\ पा. ६~। 3  \textendash\ 4 तिर्यक्पवनादयो  \textendash\ जे, पतन  \textendash\ कि~। 5 गमनविशेषा एव न  \textendash\ जे;~। गमनविशेषा एव न तु  \textendash\ कि~। 6 चेति  \textendash\ कि; जे; दे~। 7 नच्चानुवृत्तिप्रत्ययकारणम्  \textendash\ कि; दे~। 8 ऽनागन्तुको धर्मः  \textendash\ क~। १ द्विविधे  \textendash\ क~।}

\newpage
\noindent
प$^1$र्यवस्थितौ~। न मिथो व्यभिचारीति निष्क्रमणत्वप्रवेशनत्वादिजातिसङ्करापत्तौ~। न \renewcommand{\thefootnote}{१}\footnote{न सामान्यादीति  \textendash\ सामान्यादित्रिके जात्यनुपपत्तौ यथासङ्ख्यमनवस्थानादिहेतुत्रयम्~। लक्षणेति  \textendash\ विशेषस्य सामान्यवत्त्वे सामान्यवत्तद्भिन्नत्वे सति समवेतत्वलक्षणव्याघात इत्यर्थः~। असम्बन्धादिति  \textendash\ समवायस्य समवायन्तराभावादित्यर्थः~। कि. प्र. व~। \rule{0.4\linewidth}{0.5pt}}सीमान्यादिव्यक्तिकम्; अनवस्थानाल्लक्षणव्याघातादम्बन्धाच्चेति$^2$~। तस्मात्परस्परपरिहारस्थितिविरुद्धम्~। अविरुद्धं तु परापरभावस्थितीति नियमः~। {\knu परं} व्यापकम्, {\knu अपरं} व्याप्यमित्यर्थः~। प्रमाणं सूचयति \textendash\ {\knu अनुवृत्तिप्रत्ययकारणमिति}~। यदि सामान्यं न स्याद् भिन्नेष्वेष्वनुगताकारः प्रत्ययो न स्यात्~। द्रव्यगुणकर्मणामपि सामान्यद्वारेणैवानुवृत्तिप्रत्ययहेतुत्वात्~। परमुदाहरति \textendash\ {\knu तत्र$^3$ परं सत्ते}ति~। सत्तासामान्यं परमिति व्यवहर्तव्यम्~। कुतः ? महाविषयत्वात्~। द्रव्यत्वादिभ्योऽधिकविषयत्वात्~। एवमन्यत्रापि ${}^4$यदल्पविषयापेक्षयाधिकविषयं तत्तदपेक्षया परमिति व्यवहर्तव्यं यथा सत्तेत्यर्थः~। सा च सत्ता सामान्यमेव न तु द्रव्यत्वादिवद्विशेषोऽपि; कुतः ? ${}^5$अनुवृत्तेरेवेति~।

ननु सामान्यादिभ्यो व्यावर्तमानापि सत्ता यदि स्वाश्रयं ततो न व्यावर्तयेत् तर्हि द्वव्यत्वादिकमपि न व्यावर्तयेदविशेषात् ? न; सत्ताया व्यक्तिस्वरूपमात्रव्यङ्ग्यतया ${}^6$व्यक्त्यैकनियमाभावात्~। बाधकात्तु सामान्यादौ ${}^7$तस्यास्त्यागः~। सामान्यान्तरस्य हि संस्थान$^8$गुणविशेषकार्यकारणादिव्यङ्ग्यतया तेषां च नियतत्वान्न सर्वत्राभिव्यक्तिः, तर्हि वस्तुस्वरूपमेव सत्ताऽस्तु~। न च गोत्वाद्यभावेऽपि यदि ${}^9$गौ र्गौरिति प्रत्ययानुवृत्तिः स्वरूपतः स्यात्, तदाऽश्वादावपि स्यादितिवत् यदि सत्तया विनापि 'सत्सद्' इति प्रत्ययानुवृत्तिः स्वरूपतः स्यात्, सर्वत्र स्यादित्यनिष्टापत्तिः इति वाच्यम्; तदनुवृत्तेस्तदभावेऽपीष्टत्वादिति~। न; प्रत्ययानुवृत्तेर्निमित्तमन्तरेणानुपपत्तेः~। न च विशेषा एव तन्निमित्तम्; लक्षणमात्रं वा;${}^10$सामान्योच्छेदप्रसङ्गात्~। नहि ${}^11$विशेषान् लक्षणं वा विहाय क्वचित्सामान्याभिव्यक्तिरस्ति~। कथं तर्हि सामान्यादौ ' सत्सद् ' इति प्रत्ययः ? ${}^12$सत्तयैकार्थसमवायात्; गुणादिषु ${}^13$सङ्ख्यादिप्रत्ययवत्~। अभावेऽपि तर्हि स्यात्; इति चेत्, न; तस्य सद्विरुद्धतयैव प्रतीतिरिति~।

द्रव्यत्वाद्यपरम्; सत्तापेक्षयाऽल्पविषयत्वात्~। {\knu तच्चे}ति \textendash\ चस्तु अर्थः;~। अपिः

\blfootnote{1 पर्यायस्थितौ  \textendash\ मु. कि; 2 सम्बन्धाच्चेति साधर्म्धम्  \textendash\ क~। 3 तत्रेति~। परं सत्ता  \textendash\ क 4 यद्यदपेक्षया  \textendash\ मु. कि~। 5 अनुतृत्तेरेव हेतुत्वात्  \textendash\ मु. कि~। 6 व्यञ्जक  \textendash\ क~। 7 तत्त्यागः  \textendash\ मुः कि; क~। 8 गुणकार्य$^\circ$\textendash\ क~। 9 गौरितिप्रत्यया$^\circ$\textendash\ क~। 10 सामान्यमात्रोच्छेद \textendash\ क~। 11 विशेषानु  \textendash\ जे~। 12 सत्तैकार्थ  \textendash\ मु. कि~। 13 सङ्ख्याप्रत्ययवत्  \textendash\ मु. कि~।}

\newpage
\noindent
समुच्चये~। 'अनुवृत्तेर्हेतुत्वाद्' इति हेतुमनुकर्षति, सत्ताया$^1$मन्त्येषु विशेषेषु चैकैकनिमित्तवशादेकैका सञ्ज्ञा; इह तु निमित्तद्वयसमावेशात्सञ्ज्ञाद्वयसमावेश इत्यर्थः~। \renewcommand{\thefootnote}{१}\footnote{एकद्रव्या इति \textendash\ एकमात्रद्रव्याश्रया इत्यर्थः~। स्वरूपेति \textendash\ स्वरूपेणैव सन्तो न तु सत्तायोगेन~। निःसामान्यत्वे सत्येकद्रव्यमात्रवृत्तित्वमिति लक्षणार्थः~। एवं चेति \textendash\ विशेषपदसङ्केतग्रहोऽपि तत एवेति द्रष्टव्यम्~। कि. प्र. व~। \rule{0.4\linewidth}{0.5pt}}एतद्व्युत्पादनप्रयोजनं साधर्म्यादौ भविष्यतीति~॥

\begin{sloppypar}
\hangindent=2cm {\knu (८) नित्यद्रव्यवृत्तयोऽ$^2$न्त्या विशेषाः~। ते$^3$ च खल्वत्यन्त$^4$व्यावृत्तिहेतुत्वाद्विशेषा ${}^5$एव~॥}
\end{sloppypar}

विशेषान् आह \textendash\ नित्येति~। {\knu 'विशेषाः'} इति$^6$ बहुवचनेनानन्त्यं ${}^7$विवक्षितम्~। के ते ? {\knu अन्त्याः } \textendash\ अन्ते \textendash\ भवन्ति सन्तीति यावत्~। येभ्योऽपरे विशेषा न सन्तीत्यर्थः~। सामान्यरूपेभ्यो हि विशेषेभ्योऽपरे गुणादयो विशेषाः सन्ति; एभ्यस्तु नापरे, किन्त्वेतेष्वेव वैशिष्ट्यं समाप्यते~। क्व ते वर्तन्त इत्यत उक्तम् \textendash\ {\knu नित्येति}~। अयमर्थः \textendash\ अनित्यद्रव्येषु तावदाश्रयादिभिरेव विशिष्टबुद्धिरुपपन्नेति$^8$ ततोऽधिकेषु विशेषेषु प्रमाणाभावः~। नित्येषु तु द्रव्येष्वाश्रयरहितेषु समानजातीयेषु समानगुणकर्मसु च भवितव्यं व्यावर्तकेन$^9$ धर्मेण, व्यावृत्तत्वात्~। न चैवं गुणादिष्वपि${}^10$तत्कल्पनाऽवकाशः, आश्रयविशेषेणैव तद्व्यावृत्त्युपपत्तेरिति प्रमाणसूचनम्~। तथा च वक्ष्यते~।

${}^11$ननु तथापि सामान्यान्येव कानिचित्तथा भविष्यन्ति, गुणा वा; किं पदार्थान्तरकल्पनया ? इत्यत आह \textendash\ {\knu ते चेति}~। चस्त्वर्थः~। अयमर्थः \textendash\ ते पुनर्यद्ये$^12$कैकवृत्तयः कथं सामान्यरूपाः ? अनेकव्यक्ति$^13$वृत्तित्वे च कथमत्यन्तव्यावृत्तबुद्धिहेतवः ? गुणा अपि भवन्तः सामान्यवन्तः स्युः, तथाप्यत्यन्तव्यावृत्ति$^14$बुद्धिहेतुत्वं व्याहन्येत अतो निःसामान्याः, तथा च गुणत्वव्याघातः~। तस्यादन्त्यव्यपदेशादत्यन्तव्यावृत्तबुद्धेरेव हेतुत्वाद् विशेषा एव विशेषा नान्यत्रान्तर्भूता इति~। एतेन ए१कद्रव्याः ${}^15$एकस्वरूपाः स्वरूपसन्त इति सूचितं भव$^16$ति~। एवं च निःसामान्यत्वेपि विशेषोऽयं विशेषोऽयमित्यनुगतव्यवहार उपाधेर्लक्षणं च उपाधिरध्यवसेय इति~॥

\blfootnote{1 ${}^\circ$मन्त्येषु चैकैक  \textendash\ क~। 2 ह्यन्त्या  \textendash\ कि~। 3 ते खलु  \textendash\ कं~। 4 व्यावृत्तिबुद्धिहेतुत्वात्  \textendash\ व्यो; (५८) कि; ते च खल्वितरव्यावृत्तिहेतुत्वाद्विशेषा एव  \textendash\ पा. ८. पु~। 5 एव विशेषाः  \textendash\ कि~। 6 इति च  \textendash\ जे; विशेषा इति 'क' पुस्तके नास्ति~। 7 वक्ष्यति जे; लक्षयति \textendash\ क~। 6 अधिकविशेषेषु  \textendash\ क~। केनविद्धर्मेण  \textendash\ क~। 10 तत्कल्पनानवकाशः  \textendash\ जे~। 11 ननु सामान्या$^\circ$\textendash\ क~। 12 कैकव्यक्तिवृत्तयः  \textendash\ क~। 13 वृत्तित्वेन  \textendash\ क~। 14 व्यावृत्तहेतवः \textendash\ क~। 15 एकद्रव्याः स्वरूप  \textendash\ कि. क~। 16 भवतीति  \textendash\ मु. कि~। ३}

\newpage
\hangindent=2cm {\knu (९) अयुतसिद्धानामाधार्याधारभूतानां ${}^1$यः सम्बन्धः 'इह' \textendash\ प्रत्ययहेतुः स समवायः~॥}

समवायस्यैकत्वाद्विभागो नास्तीति लक्षणमाह \textendash\ {\knu अयुतसिद्धाना}मिति~। \renewcommand{\thefootnote}{१}\footnote{ननु चायुतसिद्धौ यदि युतौ न सिद्धौ तदा कयोः सम्बन्धः ? धर्मिणोरेवाभावात्~। अथायुतसिद्धौ तथापि कयोः सम्बन्धः सम्बन्धिनोरपृथग्भूतत्वात् पृथग्भूतयोरेव सम्बन्धादित्यत आह \textendash\ अयुता इति~। अन्योन्यपरिहारेण पृथगाश्रयाऽनाश्रिता इत्यर्थः~। एतदेव स्पष्टयति प्राप्ता एवेति~। अनेन अप्राप्ति \textendash\ र्निषिध्यते~। सा च प्राप्तिप्रागभावः~। तथा च तदप्रतियोगी सम्बन्ध इत्यर्थः~। तेन विशेषणतारूपोऽपिनित्यः सम्बन्धो निरस्तः~। कि. प्र. व~॥}अयुताः \textendash\ प्राप्ताश्च ते सिद्धाश्चेत्ययुतसिद्धाः, प्राप्ता एव सन्ति नं ${}^2$वियुक्ता इति यावत्~। तेषां सम्बन्धः प्राप्तिलक्षणः समवायः~। तेन संयोगो व्यवच्छिन्नः, तस्याप्राप्तिपूर्वकत्वात्~। ${}^3$तथा च नित्या प्राप्तिः समवाय इति लक्षणं सूचितं ${}^4$भवति~। \renewcommand{\thefootnote}{२}\footnote{अजेति \textendash\ विभुनोर्मिथः संयोगो यास्त्यज इति न तत्रातिव्याप्तिरित्यर्थः~। प्राप्तिपदेनेति  \textendash\ न च नित्यपदेनैव तन्निरासः तस्येश्वरेच्छारूपतया नित्यत्वात्~। यद्यपि प्राप्तित्वं संयोगसमवायान्यतरत्वमित्यन्योन्याश्रयः तथापि जातिशून्यत्वे सति सामान्यत्वविशेषत्वशून्यभावत्वं तल्लक्षणमनेनोपलक्षितम्~। कि. प्र. व~॥ \rule{0.4\linewidth}{0.5pt}}अजसंयोगाभावो वक्ष्यते, समवायस्य नित्यत्वं च~। प्राप्तिपदेनैव वाच्यवाचकभावादिलक्षणः सम्बन्धो न प्रसज्यते~। एतदेव स्पष्टयति \textendash\ {\knu आधार्याधारभूताना}मिति~। स्वभावत आधार्याधाराणां न त्वागन्तुकेन धर्मेणेत्यर्थः~। प्रमाणमाह \textendash\ {\knu इहप्रत्ययहेतु}रिति~। 'इह तन्तुषु पट' 'इह पटे शुक्लत्वम्' 'इह गवि गोत्वम्' इत्यादयः ${}^5$प्रत्ययाः सम्बन्धमन्तरेणानुपपद्यमानास्तं व्यवस्थापयन्तीत्यर्थः~॥

{\knu (१०) एवं धर्मैर्विना धर्मिणामुद्देशः कृतः~॥}

अथाऽन्येपि शक्तिसङ्ख्यासादृश्यादयः पदार्थाः किमिति नोद्दिष्टा इत्यत आह \textendash\ एवमिति~। ${}^6$उक्तेन क्रमेण धर्मिणामुद्देशः कृतो धर्मैर्विना धर्मा एव परं नोद्दिष्टाः, शक्त्यादीनामेष्वेवान्तर्भावात्~। तथा च वक्ष्यामः~। यद्यपि सामान्यविशेषसमवायानां लक्षणमप्युक्तं तथापि तस्येहाव्युत्पादनादनुक्तकल्पतयोद्देशः कृत इत्याह~।

\hangindent=2cm {\knu (११) षण्णामपि ${}^7$पदार्थानामस्तित्वाभिधेयत्वज्ञेयत्वम्$^8$; आश्रितत्वं चान्यत्र नित्यद्रव्येभ्यः~॥}

\blfootnote{1 इह इति प्रत्ययहेतुर्यः  \textendash\ दे~। 2 अप्राप्ता  \textendash\ क~। 3 तेन  \textendash\ पा. १. पु ३~। 4 'भवति' 'क' पुस्तके नास्ति~। 5 'प्रत्ययाः' ' क 'पुस्तके नास्ति~। 6 एवमुक्तेन  \textendash\ ~। 7 पदार्थानां साधर्म्य  \textendash\ मस्तित्व$^\circ$\textendash\ कि; जे~। 8 ज्ञेयत्वानि  \textendash\ क. कि. दे~।}

\newpage
\begin{sloppypar}
${}^1$यद्यपि धर्मा अपि षइभ्यो नातिरिच्यन्ते तथापि त एव ${}^2$परस्परसङ्गतांमापन्नाः परस्परविवेकायोपयोक्ष्यन्त इति पृथगुच्यन्त इत्यभिप्रायवानाह \textendash\ {\knu षण्णा}मपीति~। अपिरभिव्याप्तौ~। {\knu अस्ति}त्वम् \textendash\ विधिमुखप्रत्ययविषयत्वम्; प्रतियोग्यनपेक्षनिरू$^3$पणत्वमितियावत्~। {\knu अभिधेयत्व}म् \textendash\ अभिधानयोग्यता~। शब्देन सङ्गतिलक्षणः सम्बन्धः~। {\knu ज्ञेय}त्वम् \textendash\ ज्ञानयोग्यता; ${}^4$ज्ञाप्यज्ञापकलक्षणः सम्बन्धः~। नन्वेतद् द्वयमभावेऽप्यस्ति, इति चेत्; अस्तु; न हि तदपेक्षया वैधर्म्यमिदं$^5$ विवक्षितम्, अपि तु षडपेक्षया साधर्म्यम्~। {\knu आश्रितत्वम्} \textendash\ ${}^6$आश्रयता स्वाभाविकी, सा च नित्यद्रव्येषु नास्तीत्यत आह \textendash\ {\knu अन्य}त्रेति~। नित्यद्रव्याणि विहायेदं ${}^7$साधर्म्यमुक्तमित्यर्थः~। ननु समवायेऽप्येतन्नास्ति, ${}^8$इति चेत्, न; समवायस्य समवायान्तराभावेऽपि स्वभावत एवाधारसन्निकृष्टत्वात्~। तथा च वक्ष्यामः~। ${}^9$चकारान्मूर्तं विहाय \renewcommand{\thefootnote}{१}\footnote{निष्क्रियत्वमिति \textendash\ न च निष्क्रियविनष्टे मूर्ते क्रियात्यन्ताभाववत्त्वमित्यतिव्यापकम्, कर्मवद्वृत्तिद्रव्यत्वव्याप्यजातिशून्यभावत्वस्येन्द्रियवृत्तिद्रव्यत्वसाक्षाद्व्याप्यजातिशून्यभावत्वस्य वा विवक्षितत्वात्~। कि. प्र. व~॥}निष्क्रियत्वम्~॥
\end{sloppypar}

{\knu (१२) द्रव्यादीनां${}^10$पञ्चानामपि समवायित्वमनेकत्वं च~॥}

अथ समवायाद्वैधर्म्यमितरेषां साधर्म्यमाह \textendash\ द्रव्यादीनामिति~। \renewcommand{\thefootnote}{२}\footnote{समवायित्वमिति \textendash\ यद्यपि समवायित्वं यदि समवेतत्वं तदा नित्यद्रव्याव्याप्तिः~। समवायाश्रयत्वं च सामान्याद्यव्यापकं तयोः समवेतधर्माभावात्~। तथापि समवेतवृत्तिपदार्थविभाजकोपाधिमत्त्वं विवक्षितम्~। कि. प्र. व~॥}समवायित्वं समवायलक्षणः सम्बन्धः~। अनेकत्वं \renewcommand{\thefootnote}{३}\footnote{स्वरूपेति \textendash\ यद्यपि स्वरूपभेदः प्रामाणिकत्वमितरनिष्ठान्योन्याभावप्रतियोगित्वं वा द्वयमपि समवाये गतम्~। तथापि स्वाश्रयान्योन्याभावसमानाधिकरणविभाजकोपाधिमत्त्वे तात्पर्यम्~। कि. प्र. व~॥ \rule{0.4\linewidth}{0.5pt}}स्वरूपभेदः चकारात् समवायादन्यत्वम्~॥

{\knu (१३) गुणादीनां पञ्चानामपि निर्गुणत्वनिष्क्रियत्वे~॥}

द्रव्यं विहाय पञ्चानामाह \textendash\ गुणादीनामिति~। निर्गुणत्वं गुणाभावविशिष्टत्वम्~। निष्क्रियत्वं क्रियाया असमवायः~॥

\blfootnote{1 यद्यपि च  \textendash\ जे; क~। 2 परस्परमङ्गता  \textendash\ मु. कि~। 3 निरूपणीयमिति  \textendash\ पा. ३. पु; निरूपणमिति  \textendash\ क~। 4 ज्ञाप्यज्ञापकसम्बन्धः  \textendash\ जे; सम्बन्धलक्षणः  \textendash\ मु. कि; किन्तु अशुद्धतया प्रकरणसङ्गत्या च ${}^\circ$लक्षणः सम्बन्धः, इति स्थापितम्~। 5 इह  \textendash\ क~। 6 आधेयता  \textendash\ जे~। 7 साधर्म्यमित्यर्थः  \textendash\ क~। 8 'इति चेत्' 'क' पुस्तके नास्ति~। 9 मूर्तत्वं  \textendash\ क~। 10 पञ्चानां  \textendash\ क~।}

\newpage
\hangindent=2cm {\knu (१४) द्रव्यादीनां ${}^1$त्रयाणामपि सत्तासम्बन्धः, सामान्यविशेषवत्त्वम्; स्वसमयार्थशब्दाभिधेयत्वं, धर्माधर्मकर्तृत्वं च~। कार्यत्वानित्यत्वे कारणवतामेव~। कारणत्वं चान्यत्र ${}^2$पारिमाण्डल्यादिभ्यः~। द्रव्याश्रितत्वं चान्यत्र नित्यद्रव्येभ्यः~।}

{\knu द्रव्यादीनां त्रयाणा}मिति \textendash\ सत्तासम्बन्धः समवायलक्षणः~। सामान्यविशेषा द्रव्यादयस्तद्वत्त्वम्~। निरूपपदेनार्थशब्देन द्रव्यादयस्त्रय एवाभिधीयन्ते नापरे~। एष एव स्वसमयो {\knu वैशेषिका}णां स्वशास्त्रे ${}^3$लाघवाय~। यथा ध्यानधारणासमाधित्रयमेकत्र 'संयम' इति योगानुशासने~। द्रव्यादित्रयं तु प्रत्येकसमुदायाभ्यामिति विशेषः~। तदिदमुक्तं {\knu स्वसम}येति~। {\knu धर्माधर्मकर्तृत्वं चे}ति \textendash\ साधर्म्यद्वयसूचनाय चकारः~। धर्मकर्तृत्वमधर्मं विहाय, अधर्मकर्तृत्वं च धर्मं विहाय~। ननु जात्यादीनां कथं नैतत् ? उच्यते; ${}^4$द्रव्यादीनां विहितनिषिद्धभावनाविष्टानां हि तद्धेतुत्वं न स्वरूपतः~। न च ${}^5$भावनावेशो स्वरूपतो द्रव्यादिकमनन्तर्भाव्य सम्भवति, नित्यत्वेनाव्यापारत्वात्; ${}^6$अनित्यधर्मायोगेनाव्यापारित्वात्~। न च ज्ञानमात्रेण तेषां ${}^7$तदुपयोगोऽभियोगवदनिषेधात्~। तस्मात् स्वाश्रयावच्छेदकमात्रेणैवोपयुज्यन्त इति~॥

कार्यत्वम् \textendash\ अभूत्वा भावित्वम्~। अनित्यत्वं च भूत्वा भावित्वमिति हि विवक्षितम्; \renewcommand{\thefootnote}{१}\footnote{अभियोगो  \textendash\ विधिः~। 'नियोगवद्' इति पाठे नियोगो विधिरित्यर्थः~। कि प्र. व~॥}अन्यथा 'कारणवतामेव' इति नियमो न स्यात्~। तदनपेक्षान् विहायेत्येवकारार्थः~।

कारणत्वं च \renewcommand{\thefootnote}{२}\footnote{ननु ज्ञानादिकं प्रति सामान्यादीनामपि कारणत्वमस्त्येवेत्यत आह \textendash\ ज्ञातृधर्मेति~। अन्यथेति \textendash\ यद्यपि योगिप्रत्यक्षस्य विषयाजन्यतया पारिमाण्डल्यमपि न ज्ञातृधर्मजनकं तथापि अतद्गुणसंविज्ञानबहुव्रीहिणाऽऽदिपदग्राह्याणामस्मदादिप्रत्यक्षज्ञानजनकत्वात् ते व्यवच्छेद्या द्रष्टव्याः~। योगिप्रत्यक्षमपि विषयजन्यमिति केचित्~। त्रयग्रहणं चेति \textendash\ अत्र त्रयाणामित्यनुवृत्तेः त्रयस्यैव तत्साधर्म्यम्, तच्च सामान्यादेरपि ज्ञानकारणत्वात् कारणत्वमात्रविवक्षायां नोपपद्येत~। न चाग्रे सामान्यादेरकारणत्वमेतद्वैधर्म्यं स्यादित्यर्थः~। कि प्र. व~॥ \rule{0.4\linewidth}{0.5pt}}ज्ञातृधर्मेतरकार्यापेक्षया; अन्यथा पारिमाण्डल्यादिव्यवच्छेदः; त्रयग्रहणं

\blfootnote{1 त्रयाणाम्  \textendash\ कि~। 2 पारिमाण्डिल्यादिभ्यः  \textendash\ कि. ५~। 3 व्यवहारलाघवाय  \textendash\ क~। 4 'द्रव्यादीनां' 'क' पुस्तके नास्ति~। 5 भावनादेशो \textendash\ जे; 'जात्यादिषु' 'क' पुस्तके~। 6 धर्मायोगित्वेन  \textendash\ जे~। 7 तेषामुपयोगो  \textendash\ मु. कि~।}

\newpage
\begin{sloppypar}
\noindent
च नोपपद्येत ${}^1$पारिमाण्डल्यम् \textendash\ परमाणुपरिमाणम्~। आदिग्रहणात् ${}^2$परममहत्त्वम्, अन्त्यावयविगतरूपरसगन्धस्पर्शपरिमाणानि, द्वित्वद्विपृथक्त्वपरत्वापरत्वानि, ${}^3$विनश्यदवस्थद्रव्यनिष्ठसंयोगविभागवेगकर्माणि, अन्त्यः शब्दः, चरमः संस्कारो, ज्ञानं च गृह्यते~॥
\end{sloppypar}

\begin{sloppypar}
द्रव्याश्रितत्वं च द्रव्यसमवायिकारणता~। एवं द्रव्यत्वादिसामान्यविशेषपदार्थयोरप्रसङ्गः~। तथाप्यव्यापकमतआह \textendash\ {\knu अन्यत्रे}ति~। नित्यद्रव्येभ्य इत्युपलक्षणं ${}^4$नित्यगुणेभ्य इत्यपि द्रष्टव्यम्~। नित्यद्रव्याणि नित्यगुणांश्च विहायेदं ${}^5$पदार्थत्रितयसाधर्म्यमित्यर्थः~। आद्यौ संयोगविभागौ नित्यधर्मं च विहाय गुणासमवायिकारणकता च इति चार्थः~॥
\end{sloppypar}

\hangindent=2cm {\knu (१५) सामान्यादीनाां ${}^6$त्रयाणामपि स्वात्मसत्त्वं, बुद्धिलक्षणत्वम्, अकार्यत्वम्, अकारणत्वम्, असामान्यविशेषवत्त्वं, नित्यत्वम्, अर्थशब्दानभिधेयत्वं ${}^7$चेति~॥}

{\knu सामान्यादीना}मिति \textendash\ स्वात्मसत्त्वं सत्ताविरहः~। बुद्धिलक्षणत्वम् \textendash\ बुद्धिमात्रममीषां लक्षणं प्रमाणम्, न तु द्रव्यादिवत्प्रमाणान्तरमस्तीत्यर्थः~। अनुवृत्तबुद्धिर्व्यावृत्तबुद्धिरिह इति बुद्धिरित्येव हि सामान्यादित्रये प्रमाणमिति~। अकार्यत्वम् \textendash\ अनादित्वम्~। कथम् ? इति चेत्; अन्यथा स्वरूपव्याघातात्~। सामान्यस्य हि कार्यत्वे व्यक्तिरेव ${}^8$समवायिकारणं स्यात्~। तथा च तदुत्पत्तिविनाशयोर्जात्युत्पत्तिविनाशे प्रतिव्यक्तिभिन्नं सत् सामान्यरूपतां जह्यात्~। अभेदे तु व्यक्तेः पूर्वमपि सत्त्वान्न तत्कारणकं स्यात्~। ${}^9$एवं पूर्वपूर्वतरपूर्वतमादिव्यक्तिभ्यो प्राक् सत्त्वाद् अनादित्वं सामान्यस्य~। अन्यथा स्वरूपव्याघात इति~। नित्यद्रव्याणां च कदाचिद्विशेषाभावे व्यावृत्तिरपि निवर्तेत तथा च द्रव्यसङ्करः स्यात्~। न च स्वभावसाङ्कर्ये पुनरसाङ्कर्यम्, स्वभावपरावृत्तिप्रसङ्गात्; ततः सर्वदैवासङ्कीर्णत्वात् सर्वदैव विशिष्टानीत्यनादय एव विशेषाः~। समवायोऽपि निःसमवायः कथं समवायिकारणं विना भवेत् ? \renewcommand{\thefootnote}{१}\footnote{भवन् वेति \textendash\ भावकार्यस्य मर्यादामतिक्रामेदित्यर्थः~। कि. प्र. व~॥ \rule{0.4\linewidth}{0.5pt}}भवन् वा कथं कार्यान्तरमर्यादामतिक्रामेत् ? कथं चोत्पन्नोऽपि विनश्येत् ? तथा च कथमुत्पद्येतापि ? भावस्याविनाशिनोऽनुत्पत्तेः~। समवायान्तराभ्युपगमे च कथमनवस्थां नापादयेत् ? कथं वा पश्चाद्युत्पद्यमानः संयोगलक्षणप्राप्तौ स्वभावं न जह्यात् ? अप्राप्तिपूर्विका\textendash

\blfootnote{1 बहुषु प्राचीनपुस्तकेषु 'पारिमाण्डिल्यम्' इति पाठः किन्तु शुद्धतृया 'पारिमाण्डल्यम्' इति स्थावितः 'नित्यं परिमण्डलम्' इति सूत्रम् परिमण्डलमेव परिमाण्डल्यमिति शङ्करमिश्राः~। कन्दल्यामन्यत्र च परिमाण्डल्यमिति पाठः~। 2 द्व्यणुकपरिमाणमित्यधिकं  \textendash\ १. ३. पु~। 3 विनश्यदवस्थद्रव्ये  \textendash\ मु. कि~। 4 गुणेषु  \textendash\ क~। 5 त्रयं  \textendash\ क~। 6 त्रयाणाम्  \textendash\ कि. दे~। 7 च  \textendash\ दे~। 8 समानाधिकरणं  \textendash\ क~। 9 एवं पूर्वतर  \textendash\ मु. कि~।}

\newpage
\noindent
प्राप्तिरिति हि तत् नित्यसम्बन्धिषु तत्रानभ्युपगमाच्चेति~। तस्मात्सुष्ठूक्तम् 'सामान्यादीनां त्रयाणामकार्यत्वम्' इति~।

अकारणत्व$^1$मात्मधर्मेतरकार्यापेक्षया~। असामान्यविशेषवत्त्वमपरसामान्यविरहः, स च सामान्येष्वनवस्थानात्, विशेषेष्वपि सामान्यसद्भावे गुणत्वापत्तौ, पुनः समानगुणेषु समवायस्यै$^2$कत्वात् समवायान्तरापेक्षायामनवस्थानाच्चेति~। नित्यत्वमनन्तत्वम्, तच्चाकार्यत्वात्~। अनित्यत्वं हि ${}^3$कार्यतया व्याप्तम्~। सा च सामान्यादिभ्यो व्यावर्तमाना स्वव्याप्यमनित्यत्वमुपादाय निवर्तते~। अकार्यमपि हि यद्याकाश$^4$परमाण्वादि सामान्यादि वा निवृत्तं स्यात्, पुनस्तन्न स्यात्, कारणाभावात्~। ततस्तदभावेऽपि निराश्रयं किञ्चिदपि न स्यादिति~।

\begin{sloppypar}
अर्थशब्दानभिधेयत्वम् \textendash\ स्वसमयार्थशब्दानभिधेयत्वम्~। चकारात् ${}^5$कारणानपेक्षत्वम्~। उपलक्षणं चैतत्~। एवमन्यदप्यूहनीयम्~। तद्यथा \textendash\ अनित्यधर्मत्वमविशेषाणामेव, नित्यत्वमक$^6$र्मकाणामेव, अयोगिप्रत्यक्षत्वं द्रव्यादीनां चतुर्णामेव, असमवायिकारणत्वं गुणकर्मणोरेव, असमवेतत्वं नित्यद्रव्यसमवाययोरेव~॥
\end{sloppypar}

\hangindent=2cm {\knu (१६) पृथिव्यादीनां नवानामपि द्रव्यत्वयोगः, ${}^7$स्वात्मन्यारस्भकत्वं ${}^8$गुणवत्त्वं, कार्यकारणविरोधित्वम्, अन्त्यविशेषवत्त्वम्, अनाश्रितत्वनित्यत्वे चान्यत्रावयविद्रव्येभ्यः~॥}

इदानीं द्रव्याणामेव साधर्म्यं वैधर्म्यं चाह {\knu पृथिव्यादीना}मिति~। कियतामित्यत आह \textendash\ {\knu नवानामपी}ति~। अपिरभिव्याप्तौ~। द्रव्यत्वयोगो द्रव्यत्वसमवायः~। द्रव्यत्वमित्येतावति वक्तव्ये योगग्रहणमुपलक्षणनियमार्थम्~। अपरिच्छिन्नदेश$^9$त्वात्तु सामान्यसमवाययोः कथमत्रैवेदं नान्यत्रेति प्रत्ययनियम इति${}^10$केचिद्दर्शयन्ति तत्रेदमुत्तरम् 'अत्रैव द्रव्यत्वं वर्तते यतः' इत्युच्यते~। अयमेव हि द्रव्यत्वस्य स्वभावो यदेतत्समवायमभिव्य$^11$ञ्जयेत्, एताभिर्व्यक्तिभिः सह न रूपादिभिः~। गुणत्वं च रूपादिव्यक्तीरादाय न पृथिव्यादिव्यक्तीरित्यादि वक्ष्यते~। द्रव्यत्वमेव नास्ति गोत्वादिवदनुपलब्धेः, इति चेत्,$^12$ न; ${}^13$कार्याश्रयतोपलक्षणेन साधर्म्येणाभिव्यक्तस्य सामान्यस्य सास्नादिसंस्थानाभिव्यक्तगोत्वादिवत् प्रतीतेः~। अन्यथा

\blfootnote{त्वनात्मधर्मपेक्षया  \textendash\ क~। 2 ${}^\circ$कवाच्च  \textendash\ जे~। 3 कार्यतया व्याप्तम्  \textendash\ मु कि~। 4 परमात्मादि \textendash\ पा. २. पु~। 5 कारणानपेक्षत्वमिति  \textendash\ जे~। ${}^6$ मकर्मणामेव \textendash\ पु. कि. क~। 7 स्वात्मनारम्भकत्वम्  \textendash\ जे~। 8 गुणवत्त्वं च  \textendash\ दे~। 9 देशत्वात्सामान्य  \textendash\ मु कि  \textendash\ ~। 10 केचिच्चोदयन्ति तत्रेदमुत्तरं तत्रैव  \textendash\ पा. १ पु~। 11 व्यञ्जयन्ति पा. २ पु~। 12 केचित्  \textendash\ जे; क~। 13 अकार्याश्रय  \textendash\ क~।}

\newpage
\noindent
कार्याश्रयत्वमपि सामान्यानियतं न वस्त्वेव न स्यात्~। कारणत्वं हि सामान्येन नियम्यते, ${}^1$कार्यत्वं च~। तच्च स्वाभाविकमबाधनात् ${}^2$साधनाच्चौपाधिकमिति विशेषः~॥

\begin{quote}
{\qt \renewcommand{\thefootnote}{१}\footnote{जातौ बाधकमेव किम् इत्यत आह \textendash\ व्यक्तेरभेद इति~। अभिन्नव्यक्तिकान्यजात्या सह अन्यूनानतिरिक्तव्यक्तिका च परस्परात्यन्ताभावसमानाधिकरणवत्त्वे सति परस्परसमानाधिकरणा~। अनवस्थादिपराहता च जातिर्न भवतीत्यर्थः~। तथा हि 'आकाशत्वं न जातिः, एकव्यक्तिमात्रवृत्तित्वात्, एतद्घटत्ववत्'~। अन्यथा जातिलक्षणव्याघातात्~। "बुद्धित्वम् ज्ञानपदवृत्तिनिमित्तं" न ज्ञानत्वभिन्ना जातिः, ज्ञानभिन्नावृत्तित्वे सति सकलज्ञानवृत्तित्वात्, विषयित्ववत्'~। निष्क्रमणत्वप्रवेशनत्वे न जाती, परस्परात्यन्ताभावसमानाधिकरणत्वे सति परस्परसमानाधिकरणत्वात्; भूतत्वमूर्तत्ववत्~। 'सामान्यं यदि द्रव्यकर्मभिन्नं सज्जातिमत् स्याद्, गुणः स्यात्' इति सामान्यरूपाव्यवस्थैवाऽनवस्थाविशेषः~। 'यदि द्रव्यकर्मान्यत्वे सति जातिमान् स्यात्' गुणः स्यात् तथा च व्यावृत्तधीहेतुर्न स्यात्~। समवायो यदि प्राप्तित्वे सति समवायवान् स्यात् संयोगः स्यात्~। न च प्राप्तित्वात् समवायत्त्वं साध्यम्, साधनावच्छिन्नसाध्यव्यापकस्य संयोगत्वस्योपाधित्वाद्' इति~। परमते समवायनानात्वमभ्युपेत्योक्तम्~। अस्माकं व्यक्तेरभेद एव तत्रापि बाधक इति क्रमेणापादनमिति भावः~। कि. प्र. व~॥}व्यक्तेरभेदस्तुल्यत्वं सङ्करोऽथानवस्थितिः~।\\
रूपहानिरसम्बन्धो जातिबाधकसङ्ग्रहः~॥}
\end{quote}

\begin{sloppypar}
व्यञ्जकधर्मानुपादाय जातिनिराकरणे गोत्वादिकमपि न स्यात्~। न हि सास्नासम्बन्धाद्यनवभासने$^3$ कस्यचिद् गौरिति प्रत्ययानुवृत्तिरप्यस्ति; तस्मादस्ति द्रव्यत्वम्~। {\knu स्वात्मन्यारम्भकत्वम्} \textendash\ स्वसमवेतकार्यकारित्वं$^4$ समवायिकारणत्वमित्यर्थः~। {\knu गुणवत्त्वम्} \textendash\ गुणसमवायः, तदेतद्द्वयं निमित्तव्यवस्थापकम्~। द्रव्यत्वं तु द्रव्यव्यवहारनिमित्तमि$^5$त्यवधेयम्~। \renewcommand{\thefootnote}{२}\footnote{कार्यकारणेति \textendash\ यथा शब्दो गुणः कार्येण शब्देन नाश्यते, कर्म चोत्तरसंयोगेन तथाऽन्त्यः शब्द उपान्त्येन, नैवं द्रव्यजातीयम्; तद्धि क्वचित् समवायिकारणनाशेन, क्वचिदसमवायिकारणनाशेन नश्यतीत्यर्थः~। कि. प्र. व~॥ \rule{0.4\linewidth}{0.5pt}}कार्यकारणाविरोधित्वम् \textendash\ कार्यकारणयोरन्यतरेणापि द्रव्यजातीयं न विरुध्यत इत्यर्थः~। अन्त्यविशेषवत्त्वम् \textendash\ एतद् द्रव्यजातीयस्यैव सम्भवतीत्यर्थः~। अथवाऽन्यत्रावयविद्रव्येभ्य इति भविष्यति~॥ अनाश्रितत्वम् \textendash\ आधारैकस्व$^6$भावता~। नित्यत्वम् द्रव्यत्वे सतीति बोध्यम्~। चकाराद् विपर्ययेण विशेषरहितद्रव्यत्वम्, आश्रितद्रव्यत्वम्, अनित्यद्रव्यत्वं चान्यत्र निरवयवद्रव्थेभ्य इति~।
\end{sloppypar}

\hangindent=2cm {\knu (१७) पृथिव्युदकज्वलनपवनात्ममनसामनेकत्वापरजातिमत्त्वे~।}

\blfootnote{1 कार्यत्वं भूतत्वं  \textendash\ जे~। 2 बाधनं त्वौपाधिकमिति विशेषः  \textendash\ क~। 3 भासेनैकस्य कस्यचित्  \textendash\ की 4 कार्यत्वम्  \textendash\ क~। 5 त्यवसेयम्  \textendash\ पा. २. पु~। 6 स्वभावत्वम्  \textendash\ पा. इ. पु~।}

\newpage
\noindent
{\knu पृथिवीत्यादि~। अनेकत्वम्} \textendash\ ${}^1$बहुत्वसङ्ख्या~। अपराजातिः पृथिवीत्वादिका; तद्वत्ता तत्समवायः~। संस्कारवत्ता चेति द्रष्टव्यम्~॥

\hangindent=2cm {\knu (१८) क्षितिजलज्योतिरनिलमनसां क्रियावत्त्वमूर्तत्वपरत्वापरत्ववेगवत्त्वानि~॥}

{\knu क्षिती}त्यादि~। \renewcommand{\thefootnote}{१}\footnote{धात्वर्थः क्रियाऽतिव्यापिका इत्यन्यथा व्याचष्टे स्पन्द इति~। स्पन्दवद्वृत्तिद्रव्यत्वव्याप्यजातिमत्त्वमित्यर्थः~। एवं परत्वादावपि नोत्पन्नविनष्टा व्याप्तिः~। कि. प्र. व~।}क्रिया \textendash\ स्पन्दः तद्वत्ता~। {\knu मूर्तत्वम्} \textendash\ असर्वगतपरिमाणयोगः~। {\knu परत्वापरत्वे} गुणविशेषौ वक्ष्येते~। {\knu वेगः} \textendash\ संस्कारविशेष, तद्वत्ता~॥

\hangindent=2cm {\knu (१९) आकाशकालदिगात्मनां सर्वगतत्वं, परममहत्त्वं, सर्वसंयोगिसमानदेशत्वं ${}^2$च~।}

{\knu आकाशादीना}मिति~। {\knu सर्वगतत्वम्} \textendash\ पूर्वोक्तेषु सर्वेषु मूर्तेषु गतत्वं सम्बन्धः~। परमहत्त्वम् \textendash\ प्रकर्षकाष्ठा$^3$प्राप्तमहत्परिमाणयोगः~। ननु इयत्तैव परिमाणमुच्यत इति चेत्; न; सङ्ख्यागुरूत्वयोरपि तथाभावप्रसङ्गात्~। हस्तवितस्त्यादिपरिकल$^4$नाभावे~। परमाणुषु$^5$ तदभावप्रसङ्गाच्च~। तदभावेऽपि परमसूक्ष्मत्वात् परिमिति एव परमाणुः, इति चेत्; न; तदभावेऽपि परमहत्त्वात् परिमितमेवाकाशादि, इति न कश्चिद्विशेषः~। तस्माद्धस्तवितस्त्यादिप्रकर्षनिकर्षवान्, इतरेभ्यो व्यावृत्तः, परस्परमनुवृत्तश्च गुणविशेषः प्रत्यक्षसिद्धो दुरपह्नवः~। तस्य यथा निकर्षकाष्ठया $\rightarrow$ ${}^6$परमाणुत्वं तथा प्रकर्षकाष्ठया $\leftarrow$ परमहत्त्वमपीति~। {\knu सर्वसंयोगिसमानदेशत्वम्} \textendash\ सर्वेषां संयोगिनां मूर्तानां संयोगवृत्त्या समाना आकाशादयो देशाः, तेषां भावः, तत्त्वम्~। पूर्वमाकाशाद्येव सर्वमूर्तेषु वर्तत इत्युक्तम्~। सम्प्रति त एवाकाशादिषु वर्तन्त इत्यपौनरूक्त्यम्~। अथवा पूर्वं मूर्तसंयोगा एव आकाशादिषु वर्तन्त इत्युक्तम्~। सम्प्रति मूर्ता एवनभःप्रभृतिषु वर्तन्त इत्यपौनरूक्त्यम्~। चकारात् क्रियापरत्वापरत्ववेगविरहः~॥

\hangindent=2cm {\knu (२०) पृथिव्यादीनां ${}^7$पञ्चानामपि भूतत्वेन्द्रिपप्रकृतित्वबाह्यैकेन्द्रियग्राह्यविशेषगुणवत्त्वानि~। चतुर्णां द्रव्यारम्भ \textendash\ }

\blfootnote{1 बहुत्वसमवाया  \textendash\ क~। 2 चेति  \textendash\ कि, दे~। 3 ग्राह्य  \textendash\ जे~। 4 कल्पनाभावे  \textendash\ मु. कि; 'तदभाव इि चेन्न' इत्यधिकम्  \textendash\ १. पु~। 5 त्वस्याप्यभावेऽपि प्रसङ्गात्  \textendash\ मु . कि; अत्र 'अपि' इति स्थाने 'अति' इति भाव्यम्  \textendash\ सं~। 6 $\rightarrow$ $\leftarrow$ एतच्चिह्नान्तर्गतः पाठः \textendash\ 'क' पुस्तके नास्ति~। 7 पञ्चानामपि भूतत्वेन्द्रियप्रकृतित्वं \textendash\ दे~।} 

\newpage
\indent
\hangindent=2cm {\knu \textendash\ कत्वस्पर्शवत्त्वे; त्रयाणां प्रत्यक्षत्वरूपवत्त्वद्र1वत्वानि द्वयोर्गुरुत्वं रसवत्त्वं ${}^2$च~।}

{\knu पृथिव्यादीना}मिति \textendash\ भूतत्वम् \renewcommand{\thefootnote}{१}\footnote{औपाधिकमिति \textendash\ संस्कारत्वान्यबहिरिन्द्रियग्राह्यगुणत्वव्याप्यजातिमद्विशेषगुणवत्वम्, आत्मभिन्नविशेषगुणवत्त्वं वेत्यर्थः~॥~कि. प्र. व.~॥}औपाधिकं सामान्यम्~। अथ जातिरेव किं न स्यात् ? न स्यात्, ${}^3$व्यञ्जकाभावेन व्यक्तिनियमानुपपत्तेः~। बहिरिन्द्रिय$^4$ग्राह्य विशेषगुणवत्तैव व्यञ्जिका, इति चेत्, न; तदा मूर्तत्वमपि जातिरेव स्यात्; अवच्छिन्नपरिमाणस्य ${}^5$व्यञ्जकस्य सद्भावात्~। एवमस्तु, को दष इति चेत्, न; जातिसङ्करप्रसङ्गात्~। तथा हि मनसि मूर्तत्वं नभसि च भूतत्वं मिथः परिहारेण वर्तमानं पृथिव्यादौ सङ्कीर्यते~। अस्तु तर्हि पृथिवीत्वाद्यनेकनिबन्धनप्रवृत्तिरेव भूतशब्दः, अक्षादिवत्, इति चेत्; न; एकनिमित्तत्वे सम्भवति अनेकार्थत्वकल्पनानवकाशात्~। तस्माद्$^6$भोक्तव्यविषयाश्रयतया भूतानिसिद्धानि ${}^7$प्रसिद्धानि भूतानीत्यु्यन्ते~।

{\knu इन्द्रियप्रकृतित्वम्} \textendash\ इन्द्रियोपा$^8$दानत्वम्; तच्च नभसोऽवच्छिन्नाऽनवच्छिन्न \textendash\ भेदकल्पनयोपपादनीयम्~। अन्यथा 'भूतेभ्यः' इति पञ्चमी \renewcommand{\thefootnote}{२}\footnote{'घ्राणरसनचक्षुस्त्वक्श्रोत्राणीन्द्रियाणि भूतेभ्यः' इति पञ्चमी समानतन्त्रे गौतमसूत्र~॥ न्या. सू. १ \textendash\ १ \textendash\ १२.}समानतन्त्रे स्वतन्त्रे च मनस एतद्वैधर्म्यं न स्यादिति~। ' प्रकृति ' शब्दः स्वभावार्थो वा, तथा च मनोव्यवच्छेदाय सन्निद्धिसिद्धं बाह्यपदमनुसन्धेयम्~। {\knu बाह्यैकैकेन्द्रियग्राह्यविशेषगुणवत्वम्} \textendash\ बाह्येनाना\renewcommand{\thefootnote}{३}\footnote{अनात्मेति \textendash\ अनात्मगुणग्राहकत्वं मनस्यपि इत्यात्मगुणग्राहकेणेत्यर्थः~। कि. प्र. व~।\\ \rule{0.4\linewidth}{0.5pt}}त्मगुणग्राहिणा; एकैकेनेन \textendash\ प्रतिनियतेनेन्द्रियेण साक्षात्कारिज्ञानसाधनेन च ग्राह्या ग्रहणयोग्या ये गुणा गन्धादयः परस्परं विशेषा व्यवच्छेदहेतवः तद्वत्त्वम्~। अत्र विवक्षा \textendash\ भेदेन बाह्यैकैकेन्द्रियग्राह्यगुणवत्त्वं बाह्येन्द्रियग्राह्यविशेषगुणवत्त्वं वेति बोद्धव्यम्~।

{\knu चतुर्णां} पृथिव्यप्तेजोवायूनां {\knu द्रव्यारम्भकत्वम्} \textendash\ द्रव्यसमवायिकारणत्वम्~। स्पर्शवत्त्वं स्पर्शसमवायः~। उपलक्षणं चैतत्~। अवान्तराणुत्वमहत्त्वे, स्थितिस्थापकसंस्कार \textendash\ योगः, शरीरेन्द्रियारम्भकत्वं च इत्यपि द्रष्टव्यम्~।

\blfootnote{1 द्रवत्ववत्त्वानि  \textendash\ कि; रूपवत्त्ववत्वानि  \textendash\ दे~। 2 चेति  \textendash\ कि; दे~। 3 व्यञ्जकनियमावा  \textendash\ भावेन  \textendash\ क~। 4 ग्राह्यगुणवत्ता  \textendash\ क~। 5 व्यञ्जकसद्भावात्  \textendash\ पा र. पु~। 6 भोक्तव्याश्रयतया  \textendash\ क~। 7 नासिद्धानीति  \textendash\ जे, सिद्धानि भूतानि  \textendash\ कि~। 8 पादानकारणत्वम्  \textendash\ क~।\\ ४}

\newpage
{\knu त्रयाणां} पृथिव्यप्तेजसां {\knu प्रत्यक्षत्वम्} \textendash\ बाह्येन्द्रियग्राह्यत्वम्~। {\knu रूपवत्त्वम्} \textendash\ रूपसमवायः~। {\knu द्रवत्ववत्त्वम्} \textendash\ द्रवत्वसमवायः~। सम्भाव्यते चैतत् ॥ 

{\knu द्वयोः} पृथिव्युदकयोः {\knu गुरुत्वम्}, तत्कार्यं पतनं च~। {\knu रसवत्त्वं च} \textendash\ रसो माधुर्यादिस्तद्वत्त्वम्~। आलोकप्रकाश्यत्वमभास्वररूपत्वत्त्वं चेति चार्थः ॥

{\knu (२१) भूतात्मनां वैशेषिकगुणवत्त्वम्~॥}

{\knu भूतात्मनाम्} \textendash\ पृथिव्यादीनां पञ्चानामात्मनां च {\knu वैशेषिकगुणवत्त्वम्} \textendash\ स्वाश्रयव्यवच्छेदौपयिकावान्तरसामान्य िशेषवन्तो वैशेषिका गुणा रूपादयो बुद्ध्यादयश्च तद्वत्त्वम्~। उपलक्षणं चैतत् प्रत्यक्षगुणवत्त्वम् इत्यपि द्रष्ठव्यम्~॥

{\knu (२२) क्षित्युदकात्मनां चतुर्दशगुणवत्त्वम्~॥}

{\knu क्षित्युदकात्मनां चतुर्दशशुणवत्त्वम्} ${}^1$सङ्ख्यामात्रेण साधर्म्यमेतत्~। न तु गुणा विशेषतो विवक्षिताः सम्भावितं चैतत् साधर्म्यं न तु व्यापकं परमेश्वरात्मन्य \textendash\ सम्भवात्~। तानग्रे गणयिष्यति~।

{\knu (२३) आकाशात्मना क्षणिकैकदेशवृत्तिविशेषगुणवत्त्वम्~।}

{\knu आकाशात्मनां क्षणिकैकदेशवृत्तिविशेषणुणवत्त्वम्~।} क्षणिका आशुतरविनाशिनः, एकदेशवृत्तयोऽव्याप्यवृत्तयः\renewcommand{\thefootnote}{१}\footnote{अव्याप्येति \textendash\ समानाधिकरणात्यन्ताभावप्रतियोगिन इत्यर्थः~। कि. प्र. व~।\\ \rule{0.4\linewidth}{0.5pt}}, विशेषाय स्वाश्रयव्यवच्छेदाय गुणा विशेषगुणाः, तद्वत्त्वम्~। अत्रापि विवक्षाभेदाद् एकदेशवृत्तिविशेषगुणवत्त्वं क्षणिकविशेष \textendash\ गुणवत्त्वं वेति द्रष्टव्यम्~। आकाशे तादृशो गुणः शब्दः; आत्मनि बुद्ध्यादिः~। 

\hangindent=2cm {\knu (२४) दिक्कालयोः पञ्चगुणवत्त्वं सर्वोत्पत्तिमतां निमित्तकारण \textendash\ त्वंच~॥}

{\knu दिक्कालयोः पञ्चगुणवत्त्वम्} \textendash\ सङ्ख्या \textendash\ परिमाण \textendash\ पृथक्त्व \textendash\ संयोगविभागाः पञ्चैत्र गुणाः दिशि काले च~। {\knu सर्वोत्पत्तिमतां निमित्तकारणत्वं च~।} तत्समवेत \textendash\ द्वित्वद्विपृथक्त्वादिसंयोगविभागवर्जं सर्वाण्युत्पत्तिमन्ति गृह्यन्ते, तेषां निमित्तकारणं दिक्कालौ~। न हि देशकालानपेक्षं किञ्चिदुत्पद्यते~। तथा च व्यपदिश्यते 'इहेदानीं जातः' इति~। 'गेहे जातः' 'गोष्ठे जातः' इत्यनिमित्तमपि गेहादि व्यपदिश्यते, इति चेत्, न; 

\blfootnote{1 सङ्ख्यामाश्रित्य \textendash\ पा. २ पु~।}

\newpage
\noindent
तस्याप्यधिकरणतया निमित्तत्वात्~। न ह्यधिकरणमकारणमिति~। यद्येवं सम्प्रदानतयाऽदृष्ट \textendash\ द्वाराधिष्ठातृतया वाऽऽत्मनामपि सर्वोत्पत्तिनिमित्तकारणत्वमस्ति, तत्कथमनयोरेवोपन्यासः ? इति चेत्, सत्यम्; अधिकरणतया तु सर्वोत्पत्तिमन्निमित्तत्वं विवक्षितम्~। यथा हि दिक्कालो  \textendash\ पाध्यधिकरणा ${}^1$सर्वस्योत्पत्तिः, नैवमात्मोपाध्यधिकरणेति~। परत्वापरत्वानुमेयत्वं च इति चार्थः~॥

{\knu (२५) क्षितितेजसोर्नैमित्तिकद्रवत्वयो गः~॥}

क्षितितेजसोर्नैमित्तिकद्रवत्वयोगः~। अग्निसंयोगान्निमित्तादुत्पद्यते यद् द्रवत्वं तत्समवायः सुवर्णादौ तेजसि, घृतादौ पार्थिवे~॥

{\knu (२६) ${}^2$एवं सर्वत्र साधर्म्यं विपर्ययाद्वैधर्म्यं च वाच्यमिति~॥}

एवं सर्वत्र विपर्ययात् साधर्म्यं वैधर्म्यं च वाच्यमिति \textendash\ एवमनेन न्यायेन सर्वत्र साधर्म्यं यत्तदेव विपर्ययाद् व्यावृत्तेर्वैधर्म्यम्, वैधर्म्यं यत्तदेव विपर्ययादनुवृत्तेः साधर्म्यम्~। अथवा \renewcommand{\thefootnote}{१}\footnote{साधर्म्यविपर्ययोऽप्यन्यस्मात्साधर्म्यमित्यात्र वैधर्म्यं चेत्यपि द्रष्टव्यम्~। अन्यस्मादितित्यब्लोपेपञ्चमी, अन्यं प्राप्य साधर्म्यमित्यर्थः~। वैधर्म्यविपर्ययोऽप्यन्यस्माद्वैधर्म्यं साधर्म्यं चेत्यपि द्रष्टव्यम्~। कि प्र. व~।}साधर्म्यविपर्ययोऽप्यन्यस्माद् ${}^3$वैधर्म्यम्, वैधर्म्यविपर्ययोऽप्यन्यस्मात्$^4$ ${}^5$साधर्म्यं च स्वयमूहित्वा वाच्यमध्यापकेन ग्राह्यं च शिष्यैरिति~।

तद्यथा 'गन्धवती पृथिवी' इति वैधर्म्यं वक्ष्यति, तद्विपर्ययाद् निर्गन्धत्वमबादीनां साधर्म्यमुक्तम्~। जलभूम्योः साधर्म्यं गुरुत्वं रसवत्त्यं च तद्विपर्ययादितरेभ्यो वैधर्म्यम्~। विपर्ययस्त्वितरेषां तेजःप्रभृतीनां साधर्म्यमगुरुत्वं नीरसत्वं च~। एवमात्मानं विहाय \renewcommand{\thefootnote}{२}\footnote{परार्थत्वम् \textendash\ भोगानधिकरणत्वमित्यर्थः~। अनन्तरिति \textendash\ आत्मग्राहकेन्द्रियान्यद्रव्यत्वमित्यर्थः~। कि. प्र. व~।}परार्थत्वमचेतनत्वं च~। मनो ${}^6$विहायानन्तःकरणत्वमनष्टगुणवत्त्वं च ~। अम्भो विहाय \renewcommand{\thefootnote}{३}\footnote{अक्लेदत्वमिति \textendash\ सांसिद्विकद्रवत्वविरहिद्रव्यत्वमित्यर्थः~। कि. प्र. व~।\\ \rule{0.4\linewidth}{0.5pt}}निःस्नेहत्वमक्लेदत्वं च~। तेजो विहायाऽनुष्णत्वमदाहकत्वम्~। जलज्योतिरनिलानामपाकजस्पर्श \textendash\ वत्त्वम्~। जलज्योतिषोरपाकजरूपवत्त्वम्~। दिक्कालमनसां वैशेषिकगुणविरहः~। वाय्वाका \textendash\ 

\blfootnote{1 सर्वोत्पत्तिः  \textendash\ कि~। 2 एवं सर्वत्र विपर्ययात् साधर्म्यं वैधर्म्यं च वाच्यमिति  \textendash\ कि~। 'द्रव्यासङ्करः' इत्यधिकं 'क' पुस्तके~। 3 साधर्म्यम्, \textendash\ कि; क. प्रकाशकारेणाप्ययं पाठः स्वीकृतस्तथापि 'वैधर्म्यं च' इत्यधिकं स्थापितम्~। 'जे' पुस्तके तु 'वैधर्म्यम्' इति समीचीनः पाठोऽस्ति, अतः स्वीकृतः  \textendash\ सं~। 4 प्यन्यस्य \textendash\ जे~। 5 वैधर्म्यम्  \textendash\ कि; क~। 6 ${}^\circ$नन्तःकरणत्वमनणुत्वं \textendash\ पा.~। पु; ${}^\circ$नन्तःकरणत्वमणुकरणत्वं  \textendash\ क~।}

\newpage
\noindent
शदिक्कालमनसामतीन्द्रियत्वम्~। \renewcommand{\thefootnote}{१}\footnote{आत्ममनसोरिति~। शरीरावच्छेदः शरीरसंयोगविशेषः~। वृत्तिलाभो भोगजनकत्वम्, तच्चासमवायिकारणत्त्वम्~। तेन भोगासमवायिकारणसंयौगाश्रयत्वमित्यर्थः~। शरीरतत्संयोगौ च निमित्तकारणे इति नातिव्याप्तिः~। यद्वा निरवयत्वे सति ज्ञानहेतुशरीरसंयोगवत्त्वमित्यर्थः~। कि. प्र. व~।}आत्ममनसोः शरीरावच्छेदेन वृत्तिलभः~। वाय्वादीनां नीरूपत्वम्~॥ आकाशादीनां स्पर्शशून्यत्वम्~। कालादीनामभूतत्वमित्यादि~।

तदेवमुक्तैर्द्रव्यत्वादिभिर्द्रव्यमि तरेभ्यो विवेचितं न तु \renewcommand{\thefootnote}{२}\footnote{मध्यवर्तिभिः \textendash\ एकदेशवृत्तिभिरित्यर्थः~। अन्त्यविशेषवत्त्वादिभिरित्यत्र 'आदि' पदेन मूर्तत्वादिग्रहणम्~। अव्याप्तेः भागासिद्धतया सर्वद्रव्याव्यावर्तकत्वाद् इत्यर्थः~। कि. प्र. व~।

 *यदि कार्यैकार्थप्रत्यासत्त्या शरीरात्मसंयोगोऽप्यसमवायिकारणं तदा शरीर एव तदव्याप्तम्, अतो लक्षणान्तरमाह \textendash\ यद्वेति~। प्राणाद्यतिव्याप्तिनिरासायाद्यविशेषणम्~। ज्ञानहेतुमनःसंयोगवति श्रोत्रेऽतिव्याप्तिरिति 'शरीर' पदम्~। शरीरेन्द्रियसंयोगस्य ज्ञानहेतुत्वे मानाभाव इति तद्विशेषणेनातिव्याप्तिनिरासः~। न च तस्याहेतुत्वे 'शरीरसंयुक्तं सह' इत्यग्रिममूलविरोधः; तत्तेन नियमभागाभिधानान्न तु हेतुत्वाभिधानात्~। यदि शरीरेन्द्रियमयोगोऽपि हेतुः, तदा श्रोत्रातिव्याप्तिनिरासाय श्रोत्रान्यत्वं विशेषणम्~। यद्वा ज्ञानहेतुत्वं सकलजन्यज्ञानहेतुत्वं विवक्षितमित्यदोषः~। 'शरीर' पदं चैवं सति व्यर्थमिति केचित्~। वयं तु उद्भूतरूपपरमाणुनिष्ठमहत्त्वाभावसाक्षात्कारप्रत्यासत्तिवटकचक्षुःसंयोगवति परमाणावतिव्याप्तिवारणाय 'शरीर' पदमिति ब्रूमः~। कि. प्र. व्या. भ~।\\ \rule{0.4\linewidth}{0.5pt}}मध्यवर्तिभिरन्त्यविशेष \textendash\ वत्त्वादिभिरव्याप्तेः~। नापि परस्परतोऽतिव्याप्तेः, किन्तु कियानपि विवेकः ${}^1$सिद्ध्यती \textendash\ त्युक्तम्~। न चैतावतैव कृतकृत्यत्वम्, अविवेकस्य तदवस्थत्वादिति~। तत्कुतो विवेकः स्यादित्यपेक्षायामाह \textendash\ 

{\knu (२६ अ) ${}^2$इहेदानीमेकैकशो वैधर्म्यमुच्यते~॥}

{\knu इहेदानी}मिति  \textendash\ इह प्रकरणे इदानीमवसरप्राप्तौ एकैकमित्यैतावतैव ${}^3$चरितार्थत्वे ' शस् ' प्रत्ययोपादानं वीप्सायां भूयस्त्वज्ञापनार्थम्~।

\hangindent=2cm {\knu (२७) पृथिवीत्वाभिसम्बन्धात् पृथिवी~। रूपरसगन्धस्पर्श \textendash\ सङ्ख्यापरिमाणपृथक्त्वसंयोगविभागपरत्वा परत्वगुरू \textendash\ त्वद्रवत्वसंस्कारवती~।}

{\knu पृथिवीत्वाभिसम्बन्धात् पृथिवीति~।} पृथिवीत्वं नाम सामान्यविशेषः, तेनाभिमतः सम्बन्धः समवायलक्षणः तस्मात्~। न ${}^4$त्वेकार्थसमवायसंयुक्तसमवायादेरित्यथः~।

\blfootnote{1 सम्भवति \textendash\ क~। 2 'इह' इत्यादिकः समग्रः पाठः 'जे' पुस्तके नास्ति~। 3 चरितार्थत्वात्  \textendash\ पा, २. पु; चरितार्थे  \textendash\ पा. ३. पु~। 4 त्वेकार्थसंयुक्तसमवायादित्यर्थः  \textendash\ क~।}

\newpage
ननु पृथिवीस्वरूपसिद्धौ किं लक्षणेन ? ${}^1$सिद्धे[द्धौ] साधनस्य वैयर्थ्यात्, तदसिद्धौ आश्रयासिद्धेः~। न; ${}^2$स्वरूपसिद्धावपीतरव्यवच्छेदस्य साध्यमानत्वात्~। तथाहि 'पृथिवी \textendash\ अबादिभ्यो भिद्यते पृथिवीत्वात्~। यत्पुनरितरेभ्यो न भिद्यते नासौ पृथिवी, यथाऽबादि~। न चेयं न पृथिवी, तस्मादितरेभ्यो न भिद्यते~। तथाप्यप्रसिद्धविशेषणः पक्षः, इतरव्यव \textendash\ च्छेदस्य क्वचिदप्यप्रसिद्धेः; सिद्धौ वा साधनवैयर्थ्यात्~। ${}^3$न; ${}^4$इतरव्यावृत्तेर्घटादावेव प्रत्यक्षसिद्धत्वात्~। किन्त्वापरमाणोरा च भूगोलकात् पृथिवीत्वनिमित्ताक्रान्ते व्याप्त्या ${}^5$व्यवच्छेदभेदव्यावृत्त्या व्यावृत्तिर्न सिद्धेति साध्यत इति न दोषः ~।

अथ किमेतल्लक्षणमिति ? उच्यते~। केवलव्यतिरेकिहेतुविशेष एव लक्षणम्~। तथा चाचार्याः 'समानासमानजातीयव्यवच्छेदो लक्षणार्थः' इति~। एवं तर्हि पृथिवीत्व \textendash\ निमित्ताक्रान्ते \renewcommand{\thefootnote}{१}\footnote{व्याप्त्येति अस्यैव व्याख्यानमवच्छेदेति~। अवच्छेदभेदो घटत्वपटत्वादिः~। तस्य व्यावृत्त्या विरहितत्वेन लक्षिता व्यावृत्तिरित्यर्थः~। कि. प्र. व~॥}व्याप्त्यावच्छेदभेदव्यावृत्त्या व्यावृत्तिः क्वचिन्न सिद्धा, इति पुनरप्यप्रसिद्ध \textendash\ विशेषणत्वं समायातम्; इति चेत्; न; पक्षसम्बन्धिनो विशेषणस्यं ${}^6$सर्वत्रान्वयिन्यपि ${}^7$अपूर्वस्यैव साध्यत्वात्~। प्रसिद्धो हि धर्मो धर्मिणमन्वेति न तु धर्म्यन्विततयैव प्रसिद्धः साध्यत इति~। तथा सति सिद्धेः साधनवैयर्थ्यादिति~। तथापि घटादौ चेद् इतरव्यावृत्तिः प्रत्यक्षसिद्धा ततस्तद्दृष्टान्तबलेनान्वयादेव परमाण्वादौ साध्यतां किं घटादिकमपि पक्षे निक्षिप्य व्यतिरेक आद्रियते ? इति चेत्, आस्तां तावदयं सुहृदुपदेशः, केवलव्यतिरेकिलक्षणं तावन्निर्व्यूढम्~।

\renewcommand{\thefootnote}{२}\footnote{व्यवहारसिद्धेर्वेति~। ननु पृथिवीव्यवहारः पृथिवीपदप्रयोगविषयत्वं वा, पृथिवीपदजन्यज्ञानविषयत्वं वा, पृथिवीत्वेन निमित्तेन पदवाच्यत्वं वा पृथिवीत्वेऽप्यस्ति इति, ततो न हेतुव्यावृत्तावसाधारण्यम्~। मैवम्, पृथिवीत्वेन निमित्तेन पृथिवीशब्दाभिधेयत्वस्य तद्व्यवहारार्थत्वात्~। कि. प्र. व~।\\ \rule{0.4\linewidth}{0.5pt}}व्यवहारसिद्धिर्वा लक्षणप्रयोजनम्~। तथाहि \textendash\ विवादाध्यासितं द्रव्यं पृथिवीति ${}^8$व्यवह्रियते, पृथिवीत्वात्~। यत्पुनः पृथिवीति न व्यवह्रियते न सा पृथिवी, यथा अबादि~। ${}^9$न चेयं न पृथिवी, तस्मान्न तथा व्यवह्रियत इति~। अत्रापि व्यवहारविषयस्य स्वरूपतः

\blfootnote{1 'सिद्धौ' इति समीचीनं भांति, प्रकाशकारस्यापि सम्मतः पाठोऽयम्  \textendash\ सं~। 2 स्वरूपसिद्धस्यापि व्यवच्छेदस्य साध्यत्वात्  \textendash\ क~। 3 'न'  \textendash\ कपुस्तके नास्ति~। 4 इतरव्यवच्छेदस्य  \textendash\ क~। ५ अवच्छेदभेदेन व्यावृत्ति 'न'  \textendash\ पा. ३. पु~। 6 सर्वत्राप्रसिद्धस्यैव  \textendash\ पा. १. पु~। 7 अप्रमितस्यैव  \textendash\ क~। 8 व्यवह्रियते लोकेन  \textendash\ कि; क~। 9 न चेयं पृथिवी तस्मात्तथा व्यवह्रियते इति  \textendash\ कि~।}

\newpage
\noindent
सिद्धेर्नाश्रयासिद्धिः~। पृथिवीव्यवहारस्य संमुग्धस्य ${}^1$सिद्धेर्नप्रसिद्धविशेषणत्वम्~। पक्षे व्याप्त्या निमित्तविशेषवतः$^2$ साध्यत्वान्नसाधनवैयर्थ्यम्, सर्वपृथिवीपक्षीकरणेन च नान्वयि \textendash\ त्वम्~। यद्यस्यान्वयव्यतिरेकावनुविधत्ते तत् तद्धेतुकम्, यथा घटादि मृदादिहेतुकम्, अनुविधत्ते च पृथिवीव्यवहारः पृथिवीत्वस्यान्वयव्यतिरेकौ इति चेत्, न, पृथिवीत्वनिमित्त \textendash\ कत्वे पृथिवीव्यवहारस्य साध्ये अन्वयाभावात्~। \renewcommand{\thefootnote}{१}\footnote{केवलमिति~। न हि यत्त्वं तत्त्वं चानुगतमिति व्यतिरेकव्याप्तिमुपजीव्य सर्वनाम्नाऽन्वयाभिधानमेव वक्रत्वार्थः~। कि. प्र. व~।}केवलं ${}^3$सर्वनाम्ना व्यवहारमात्रेण वक्रोऽयमन्वयः, स च व्यतिरेकान्नभिद्यते, विवक्षाभेदादिति~।

यत्पुनराह {\knu भूषणो} '$^4$लक्षणं चिह्नं लिङ्गमिति पर्याया' इति, तदसत्; व्यावृत्तौ व्यवहारे वा साध्येऽन्वयिनोऽनवकाशात्~। व्युत्पन्नस्य स्वयमेव व्यवहारात्~। अव्युत्पन्नस्य सपक्षपरिचयाभावात्~। कथं तर्हि शस्त्रे ' क्रियावद् द्रव्यम् ' इति ${}^5$द्रव्यलक्षणेषु पठ्यते ? द्रव्यस्यैवायमसाधारणो धर्मः समवायिकारणत्ववद् गुणवत्त्ववच्चेति प्रतिपादनार्थम्, न तु लक्षणत्वेन द्रव्यमात्रं पक्षीकृत्य, व्यावृत्तिसाधने भागासिद्धत्वात्~। मूर्त्तद्रव्यमात्रपक्षीकरणेऽ \textendash\ प्यासाधारणत्वात्~। तस्माद् वायुमनसोरप्रत्यक्षत्वेन द्रव्यत्वविप्रतिपत्तौ च क्रियावत्त्वेन तत्प्रसाध्य भागासिद्धिः ${}^6$परिहरणीयेति तस्य तार्त्पर्यम्~।

ये तु प्रमाणमेव सर्वस्य व्यवस्थापकं न तु लक्षणम्, \renewcommand{\thefootnote}{२}\footnote{तदपेक्षायामिति \textendash\ लक्षणेऽपि लक्षणापेक्षायामित्यर्थः~। कि. प्र. व~।}तदपेक्षायामनवस्थामाहुः; तेषां निन्दामि च पिबामि चेति ${}^7$न्यायः~। यतोऽव्याप्त्यतिव्याप्तिपरिहारेण तत्तदर्थव्यव \textendash\ स्थापकं तत्तद्व्यवहारव्यवस्थापकं च प्रमाणमुपाददते तदेव लक्षणम्~। अनुवादः स\renewcommand{\thefootnote}{३}\footnote{स इति \textendash\ 'लक्षणरूपोऽर्थोऽनुवादो न त्वपूर्वार्थप्रापकः, *मानान्तरप्रतीतार्थप्रमाकत्वाद्' इत्यर्थः~। कि. प्र. व~।

*प्रमितार्थप्रतिपादकत्वाद् इत्यपि पाठः पुस्तकान्तरे~।\\ \rule{0.4\linewidth}{0.5pt}} इति चेत्, अस्माकमप्यनुवाद एव~। न ह्यलौकिकमिह किञ्चिदुच्यते~। न चानवस्था, वैद्यकादौ रोगादिलक्षणवद् व्याकरणादौ शब्दादिलक्षणवच्च व्यवस्थोपपत्तेः~। ${}^8$तत्रापि हि संबन्ध \textendash\ व्यवहारमाश्रित्य लक्षणैरेव व्युत्पत्तिरिति~।

\blfootnote{1 नाप्यसिद्धि  \textendash\ क~। 2 विशेषणवतः  \textendash\ क~। 3 सर्वनामशब्देन पा ३ पु.; क~। 4 लक्षणं चिह्नं गमकं लिङ्गमिति  \textendash\ पा. ३ पु~। 5 द्रव्यलक्षणं  \textendash\ कि; द्रव्यलक्षणमुपपद्यते  \textendash\ पा. १. पु~। 6 प्रतीकरणीयेति  \textendash\ जे~। 7 न्यायापातः  \textendash\ पा १. पु~। 8 तत्रापि संमुग्ध  \textendash\ मु. कि~।}

\newpage
क्षित्युदकात्मनां चतुर्दशगुणवत्त्वं साधर्म्यमित्युक्तम्~। तत्र के ते चतुर्दश गुणाः पृथिव्यामित्यत आह \textendash\ ${}^1${\knu रूपेति~।} ${}^2$ननु गन्धसाहचर्याद्रूपादयोप्यसाधारणतामापन्ना ्तत्कथं$^3$ न लक्षणम् ? ${}^4$इति चेत्, न; केवलस्यैव गन्धस्यासाधारणत्वे विशेषणासामर्थ्यात्~। एतच्चानुपदमेव स्फुटयिष्यति~।

\hangindent=2cm {\knu (२८) एते च गुणविनिवेशाधिकारे रूपादयो ${}^5$गुणविशेषाः सिद्धाः~। 'चाक्षुष' वचनात्सप्तसङ्ख्यादयः~। पतनो \textendash\ पदेशाद् गुरूत्वम्~। 'अद्भिः सामान्य' वचनाद् द्रव्यत्वम्~। 'उत्तरकर्म' वचनात् संस्कारं इति~॥}

${}^6$एषु सूत्रकारसम्मतिमाह \textendash\ {\knu एतेे चे}ति~। गुणानां विनिवेशो द्रव्येषु समवायः~। स च प्रतिपाद्यत्वेनाधिक्रियतेऽस्मिन्निति गुणविनिवेशाधिकारः द्वितीयोध्यायः~। ${}^7$तथा च सूत्रम् \textendash\ {\knu 'तत्र$^8$रूप \textendash\ रस \textendash\ गन्धस्पर्शवती पृथिवी'} \textendash\ इति (वै. सू. २ \textendash\ १ \textendash\ १) सूत्रकारवचनाद्रूपादयः सिद्धाः~। ${}^9$सङ्ख्यादौ सूत्रकारसम्मतिमाह {\knu चाक्षुषवचनात् सप्त सङ्ख्यादय इति \textendash\ "सङ्ख्याः, परिमाणानि, पथक्त्वम्, संयोगविभागौ, परत्वापरत्वे कर्म च रूपद्रव्यसमवायाच्चाक्षुषाणि"} (वै. सूः ४ \textendash\ १ \textendash\ ११) इति सूत्रे सङ्ख्यादीनां चाक्षुषत्वे हेतुत्वेन रूपद्रव्य \textendash\ समवाय उक्तः~। न चासिद्धश्च हेतुत्वम्; अतस्तेऽपि रूपवत्यां पृथिव्यां सप्त सिद्धाः~। {\knu पतनोपदेशाद् गुरुत्व}मिति \textendash\ " {\knu${}^10$संयोगविभागप्रयत्नवेगाद्य भावे गुरुत्वात् पतनम्"} (वै. सूः ५ \textendash\ १ \textendash\ ७) इति पतने गुस्त्वस्य हेतो$^11$ रु  \textendash\ पदेशात् पतनवत्यां पृथिव्यां गुरुत्वमपि सिद्धम्~। नह्यसिद्धस्य हेतुत्वम्, नापि व्यधिकरणस्य कर्मासमवायिकारणत्वम् नोदनादिषु विपर्यासदर्शनात् इति~। {\knu अद्भिः सामान्यवचनाद् द्रवत्व}मिति \textendash\ {\knu "सर्पिर्जतुमधूच्छिष्टानां ${}^12$पार्थिवानामग्निसंयोगाद्} \textendash\ अद्भिः {\knu द्रवत्वम् सामान्यम्"} \textendash\ (वै. सू. २ \textendash\ १ \textendash\ ६) इति सूत्रे सर्पिरादीनां पार्थिवानां

\blfootnote{1 रूपेत्यादि  \textendash\ जे~। 2 न तु  \textendash\ क~। 3 ${}^\circ$पन्ना इति युक्तम्  \textendash\ क~। 4 'इति चेत् न' इति क पुस्तके नास्ति~। 5 विशेषगुणाः \textendash\ जे 'तथा च रूपरसगन्धवती पृथिवी' इत्यनेन विशेषगुणाश्चत्वा रोऽभिहिताः (व्यो. ५. १९४) गुणविशेषा इत्यनेकेषु पुस्तकेषूपलभ्यते इति स्वीकृतः \textendash\ सं~। 6 अत्र \textendash\ जे~। 7 'तथा च सूत्रम्' \textendash\ इति जे पुस्तके नास्ति~। 8 'तत्र' इति सूत्रे नास्ति~। 9 वाक्यमिदं कि; जे पुस्तकयोर्नास्ति~। 10 संयोगाभावे इति मूलसूत्रपाठः, संयोगाद्यभावे \textendash\ पा. १. पु; संयोगवेगप्रयत्नाभावे  \textendash\ क~। 11 रुद्देशात् \textendash\ कि~। 12 'पार्थिवानाम्' इति मूलसूत्रे नास्ति~।}

\newpage
\noindent
द्रवत्वमद्भिः समानो धर्म इत्युक्तम्, अतो द्रवत्वमपि सिद्धम्~। {\knu उक्तकर्मवचनात् संस्कार इति \textendash\ "नोदनादाद्यमिषोः कर्म, तत्कर्मकारिताच्च संस्कारा \textendash\ दुत्तरं तथोक्तरमुत्तरं च"} (वै. सू. ५ \textendash\ १ \textendash\ १७) इति सूत्रे इषुकर्मोपदेशेन पार्थिव \textendash\ द्रव्यस्योत्तरस्मिन् कर्मणि संस्कारः कारणतयोक्तः, अतः सोऽपि सिद्धः~। उपलक्षणं चैतत्, आद्यकर्मकार्यतयाप्युक्तस्तत्रैव सिद्ध इत्यपि द्रष्टव्यम्~।

\begin{sloppypar}
तदनेन सूत्रकारसम्मतिप्रदर्शनव्याजेन सर्वत्र ${}^1$प्रमाणमादर्शितम्, तथाहि \textendash\ रूपादिष्वेकादशसु प्रत्यक्षं \renewcommand{\thefootnote}{१}\footnote{पतनलिङ्गकमिति \textendash\ गन्धवत्यधः संयोगफलिकाक्रिया सा समवायिकारणिका, क्रियात्वात्, सम्प्रतिपन्नवद्, इति, न तेजस्यगुरुण्यपि तद्दर्शनाद् व्यभिचारः~। न चैवं रसेनार्थान्तरम्; रसोत्कर्षेण पतनोत्कर्षाभावात्~। कि. प्र. व~॥}पतनलिङ्गकमनुमानं गुरुत्वे~। अग्निसंयोगान्वयव्यतिरेकानु \textendash\ विधायितया नेदमिह क्षीरादिवत् संयुक्तसमवायेन नान्यस्येति प्रत्यक्षमेव सोपपत्तिकं द्रवत्वे निरन्तरो गतिसन्तान एव वेगव्यवहारहेतुरिति ${}^2${\knu मीमांस}कदुर्दुरूढादिप्रतिपत्तिव्युदाय \renewcommand{\thefootnote}{२}\footnote{कार्यलिङ्गकमिति \textendash\ नोदनाभिधानजन्यं द्वितीयं कर्मासमवायिकारणम्, कर्मत्वाद्, आद्यकर्मवत्~। न च कर्मणाऽर्थान्तरम्, कर्मवति कर्मान्तराऽनुत्पन्नेरित्यर्थः~। कि. प्र. व~।\\ \rule{0.4\linewidth}{0.5pt}}कार्यलिङ्गकमनुमानं संस्कार इति~।
\end{sloppypar}

\hangindent=2cm {\knu (२९) क्षितावेव गन्धः~। रूपमनेकप्रकारं$^3$ शुक्लादि~। रसः षड्विधो मधुरादिः~। गन्धो द्विविधः सुरभिरसुरभिश्च~। ${}^4$स्पर्शोऽनुष्णाशीतत्वे सति पाकजः~।}

एवं गुणवत्त्वेन द्रव्यत्वे व्यवस्थिते पृथिवीत्वव्यवस्थाहेतून् गुणविशेषानाकृष्य दर्शयति क्षितावेव गन्ध इति~। {\knu क्षितावेव गन्धो} वर्तते न द्रव्यान्तरे~। तेन गन्धसमवायः ${}^5$पृथिवीत्वव्यवस्थाहेतुरित्युक्तं भवति~। यत्तु 'सुगन्धि सलिलम्'; 'सुरभिः सभीरण', इति तत्पार्थिवकुसुमाद्यधिवासनिबन्धनम्~। तदन्वयव्यतिरेकानुविधानात् न तु स्वभावत इत्यव \textendash\ धेयम्~। ननु तथाप्यव्यापको गन्धः क्षितेः, वक्ष्यमाणेषु मणिवज्रादिष्वनुपलब्धेः; तत्कथं तद्व्यवस्थाहेतुः ? न, पाकेन पूर्वरूपनिवृत्तौ$^6$ रूपान्तरोत्पत्तौ च पाकजरू$^7$पस्यैव गन्धसह \textendash\ चरिततया तत्रापि तदनुमानात्~। अनुद्भवात्तु तदनुपलम्भः तत्र घ्राण इवेत्यभिप्रायः$^8$~।

\blfootnote{1 प्रमाणं दर्शितम् \textendash\ क 2 ' मीमांसक ' इति ' क ' पुस्तके नास्ति~। 3 प्रकारकं  \textendash\ कि~। 4 स्पर्शोऽस्या  \textendash\ दे~। 5 पृथिवीत्यवस्थाहेतु  \textendash\ पा. २. पु~। 6 परावृत्तौ  \textendash\ पा. ३. पु~। 7 स्य च  \textendash\ जे; क~। 8 त्यभिप्रायवान् आह  \textendash\ जे~।}

\newpage
{\knu रूपमनेकप्रकारं शुक्लादि~।} क्षितावेवेत्यनुवर्तते~। यद्यपि शुक्लादिरूपं ${}^1$समुच्चयेन न सम्भवति~। स हि जातिसमुच्चयो विरोधादेवानुपपन्नः~। व्यक्तिसमुच्चयोऽपि न व्याप्त्या, अनुपलम्भबाधितत्वात्~। नाव्याप्त्या, अव्याप्यवृत्तिजातीयताविरोधात्~। किमेतत् ${}^2$तर्हि चित्रं रूपमिति ? यथा ${}^3$शुक्लमिति शुक्लजातीयं तथा चित्रमपि चित्रजातीयं व्याप्यवृत्त्येव; प्रत्येकं च न ${}^4$क्षितिव्यवस्थाहेतुः भागासिद्धत्वात् ~। तथापि क्रमेण शुक्लाद्यने \textendash\ करूपसमवायलक्षणः समुच्चयः क्षितावेव~। किञ्च शुक्लशुक्लतरशुक्लतमाद्यवान्तरानन्तज तिमद्रूपं क्षितावेव नान्यत्र~। जलस्य हि शुक्लत्वेऽपि नावान्तरतारतम्यमस्ति, ${}^5$पार्थिवसंयोगात्तु तथा प्रतिभासः~। एवं तेजसोऽपि शुक्लभास्वरैकस्वभावत्वे$^6$ स्वाभाविकविशेषाभावः~।

ये पुनराहुः \textendash\ शुक्लत्वादिकमेव सामान्यं नास्ति कुतस्तदवान्तरतारतम्यमिति~। एकैका एव हि शुक्लारुणादिरूपव्यक्तयो नित्या अपि अनित्याभिर्द्रव्यव्यक्तिभिर्व्य यन्ते~। तारतम्यं त्वमूषामाश्रयमिश्रतया यथायथा हि धवले कृष्णद्रव्यानुप्रवेशस्तथा तथा तारतम्या \textendash\ वभास इति; ${}^7$तदयुक्तम्; तेषामाश्रयस्थितावपि पावकसंयोगात् पूर्वरूपनिवृत्तिरुत्तर \textendash\ रूपोत्पादश्च न स्यात्~। ${}^8$अनेकव्यक्तिसमवेतस्य निरतिशयस्य नित्यस्य सामान्यादन्यत्वं च नोपपद्यते~। अस्तु शुक्लत्वादिकं ${}^9$सामान्यमेव, इति चेत्, न; गोत्वादिना परापरभावानु \textendash\ पपत्तौ जातिसङ्करप्रसङ्गादिति~॥

{\knu रसः षड्विधो मधुरादि}रिति~। अस्यापि${}^10$वृत्तिर्मण्यादिषु गन्धवत् पाकज  \textendash\ रूपेणानुमानात्~। अत्राप्यवान्तरानेकसामान्यविशेषवान् रसः पृथिव्यामेवेति वाक्यार्थः~। अपामव्यक्तमधुर एव हि रसः स्वभावतः~। अवान्तरभेदास्तु नारिकेलादिजलगताः पार्थिव \textendash\ द्रव्योपाधिकाः~। तेन गन्धवती पृथिवी, अवान्तरानेकसामान्यविशेषवद्रूपवती अवान्तराऽनेक \textendash\ सामान्यविशेषवद्रसवती चेति लक्षणार्थः~।

स्यादेतत्~। यथा रूपवत्त्वाविशेषेऽपि त्रयाणामनेकविधरूपवती पृथिवी~। स्वच्छ  \textendash\ धवलरूपवज्जलम्~। शुक्लभास्वररूपवत्तेजः~। तथा गन्धाऽवान्तरप्रकारभेदमाश्रित्य ${}^11$द्रव्यभेदः स्याद् इत्यत आह \textendash\ 'गन्धो द्विविधः सुरभिरसुरभिश्चेति~। क्षितावेवेत्यनुवर्तते~।

\blfootnote{1 समुच्चयेन सम्भवति  \textendash\ क~। 2 तर्हि रूपमिति  \textendash\ क~। 3 शुभ्रमिति  \textendash\ जे~। 4 व्यवच्छेदहेतुः  \textendash\ पा. २, ३. पु, अवच्छेदहेतुः \textendash\ क~। 5 पार्थिवद्रव्यसंसर्गात्  \textendash\ क~। 6 स्वभावान्न  \textendash\ पा. ३. पु~। 7 यत्तदसत्  \textendash\ पा. ३ पु~। 8 अनन्त  \textendash\ जे~। शुक्लादिकं द्रव्यवृत्तिसामान्य  \textendash\ पा. १. पु., क~। 10 व्याप्तिर्मण्यादिषु  \textendash\ कि; क~। 11 द्रव्यभेदहेतुः  \textendash\ क~।}

\newpage
न द्रव्यान्तरे~। पाकजः खल्वयमेकस्मिन्नेव द्रव्ये कालभेदेन वर्तमानः प्रकारभेदवानपि न$^1$ द्रव्यान्तरव्यवस्थाहेतुरिति भावः~।

स्पर्शोऽस्याः पाकजो विलक्षण इत्याह \textendash\ {\knu स्पर्श} इति~। $\rightarrow$ ${}^2$यद्यपि नभस्वतोऽ \textendash\ नुष्णाशीतस्पर्शउक्तस्तथापि न पाकजः $\leftarrow$~। पाकजस्पर्शवती पृथिवीति लक्षणार्थः ~। अत्र हि स्पर्शवतीत्युच्यमाने$\rightarrow$ $^3$ज ादावतिव्याप्तिः स्यात्~। पाकजवतीत्युच्यमाने$\leftarrow$तेजसा पाकजद्रवत्व \textendash\ वतातिव्याप्तिः, अतः पाकजस्पर्शवती~। अत्र च गन्धरूपरसानां पाकजत्वं सदपि स्फुरत्वान्नोक्तम्; स्पर्शस्य तु तथात्वे विप्रतिपत्तिः~। न हि गन्धादिवद्विशेषतः स्पर्शः पाकजस्यान्वयव्यतिरेकावनुविधत्ते~। पार्थिवगुणत्वमात्रेण तत्परिकल्पने परिमाणादिष्वपि तथा$^4$ कल्पनाप्रसङ्गात्~। तदेतद् विप्रतिपत्तिबीजमपनिनीषन्नेवोक्तवान् पाकजः स्पर्श इति~। यदि हि पाकजो न स्यात् सङ्ख्यापरिमाणादिवदविशिष्टः स्यात्~। तथा चाशीविषवृश्चिककीटादि \textendash\ दंशेषु शूकशिम्बिवृश्चिकपत्रिकादिषु संस्पृष्टेषु दृष्टो दुःखविशेषो नोपपद्येत~। मणिमूलादि \textendash\ विशेषेषु$^5$ च स्पृष्टेषु तदुपशमो न स्यात्~। गवादिचाण्डालादिस्पर्शविधिनिषेधौ च न स्याताम्~। न च ${}^6$द्रव्यमात्रे तावुपपद्येते; दर्शनस्पर्शनघ्राणास्वादनभेदेन मातङ्गमदिरादिषु प्रायश्चितभेदानुपपत्तिप्रसङ्गात्~। तस्मात्पार्थिवस्पर्शोऽऽपि पाकजः पार्थिवविशेषगुणत्वात् गन्धवदि$^7$त्यवधेयम्~।

\begin{sloppypar}
ये तु मन्यन्ते पृथिव्यां स्पर्शो नास्त्येव, औष्ण्यशैत्ये विहाय स्पर्शान्तरस्यानुपलम्भ \textendash\ बाधितत्वात्~। शैत्यमेव हि तेजः स्पर्शेनाभिभूतमौष्ण्यं चोदकस्पर्शेनाभिभूतमनुष्णाशीतत्वे$^8$ नानुभूयत इति; तान् प्रत्याह \textendash\ {\knu अनुष्णाशीतत्वे सतीति~।} बलवत्सजातीयग्रहण  \textendash\ कृतमग्रहणं खल्वभिभवः~। तथा चाभिभावकमन्यतरदवश्यमुपलभ्येत~। न चैवं प्रकृते, किन्तु शैत्योष्णत्वानुपलम्भेऽपि तद्विलक्षणः स्पर्श उपलभ्यत एव~। ${}^9$तथाविधे तमसि शुष्क$^10$ \textendash\ पार्थिवद्रव्यस्येति भावः~।
\end{sloppypar}

\hangindent=2cm {\knu (३०) सा ${}^11$तु द्विविधा~। नित्या चानित्या च~। परमाणुलक्षणा नित्या~। कार्यलक्षणा त्वनित्या~। सा च स्थैर्याद्यवयव \textendash\ }

\blfootnote{1 द्रव्यभेद  \textendash\ पा. र. पु~। 2 $\rightarrow$ $\leftarrow$~। 3 $\rightarrow$ $\leftarrow$ $\rightarrow$ एतच्चिह्नार्गतौ पाठौ 'क' पुस्तके न स्तः~। 4 तथाप्रसङ्गात् \textendash\ क~। 5 विशेषेषु स्पृष्टेषु \textendash\ कि~। 6 द्रव्यमाश्रित्यैवोपपद्येते  \textendash\ पा. २. पु~। 7 ${}^\circ$त्यवधातव्यम्  \textendash\ पा. ३. पु~। 8 नाभिधीयते  \textendash\ क~। 9 तथाविधेऽप्यन्धे तमसि  \textendash\ पा. ३. पु~। 10 शुक्ल  \textendash\ जे; शुष्क पार्थिवस्येति  \textendash\ कि~। 11 च  \textendash\ कि~।}

\newpage
\indent
\hangindent=2cm {\knu सन्निवेशविशिष्टाऽपरजातिबहुत्वोपेता शयनासनाद्यनेकोपकारकरी च~। त्रिविधं चास्याः कार्यम्~। शरीरेन्द्रियविषयसञ्ज्ञकम्~॥}

सेयं\renewcommand{\thefootnote}{१}\footnote{नित्यानित्यभेदकथनस्यार्थन्तरत्वमपाकरोति \textendash\ सेयमिति~। अवयवानवस्थेति \textendash\ यद्यप्यनव \textendash\ स्थामात्रं बीजाङ्कुरसाधारणेन न दूषणं तथापि सर्वकार्यद्रव्यनाशात् प्रलयानन्तरं सृष्टिरिति व्यवस्थाविरह एवानवस्थेत्येके~। द्वणुकावयवस्यानेकद्रव्यारब्धत्वे महत्त्वं स्यादित्यन्ये~। एकस्येति \textendash\ सर्गादौ सर्वानित्यद्रव्यनिवृत्तौ द्वणुकनिराश्रयतापत्तिरित्यर्थः~। निःप्रमाणकेति \textendash\ नित्यपृथिव्या अनुपलम्भात् पृथिवीत्वस्य अनित्यपृथिवीवृत्तित्वेनैवोपलम्भादित्यर्थः~। कि. प्र. व~॥} पृथिवी यद्यनित्यैव स्यात्तदाऽवयवानवस्था स्यात् एकस्यापि कस्यचिदवयवस्य निवृत्तौ निराश्रयं च कार्यमापद्येत, अथ नित्यैव, निःप्रामाणिका तर्हि स्यात्~। कार्यं च गन्धवदपि द्रव्यान्तरमापद्येतेत्यत आह \textendash\ {\knu सा ${}^1$तु द्विविधा नित्या चानित्या चेति}~। चकारौ मिथः समुच्चयार्थौ नियमनिराकरणपरौ; तेन नोक्तदोषावकाश इत्यभिप्रायः~।

का पुनर्नित्या ? महतीहि मही महत्त्वादेव कार्यत्वादनित्या~। अणुपरिमाणा तु नोप \textendash\ लभ्यत एवेत्यत आह \textendash\ {\knu परमाणुलक्षणा नित्ये}ति~। लक्षणं स्वभावः~। कार्यमेवात्र प्रमाणं भविष्यतीत्यभिप्रायवता नेह प्रमाणान्तरमादर्शितम्~। तथाहि स्थूलकार्यस्य लोष्टादेरवयव \textendash\ क्रियाविभागादिन्यायेन विभज्यमानस्याल्पतरतमादिभावाद् यतो नाल्पीयस्तं परमाणुमाचक्ष्महे~। एष ह्यवयवावयविप्रसङ्गो न तावन्निरवधिरेव; \renewcommand{\thefootnote}{२}\footnote{अनन्तेति~। नत्वनन्तावयवारब्धत्वाविशेषेऽपि कश्चित्तादृग्बहुतरावयवारब्धः, कश्चित्तादृक् तदवयवापेक्षयाऽल्पावयवारब्धः इत्यवान्तरसङ्ख्याभेदादेव परिमाणभेदः स्यात्~। अत्राहुः \textendash\ सर्षपो यदि साक्षात्परम्परासाधारणमेवारम्भकन्यूनावयवारब्धः स्यात्, एतावत्परिमाणाधिकपरिमाणः स्यात्~। न ह्यवयवानां निबिडत्वेऽप्येतावानपकर्षः सम्भवति~। यद्वा मेरुपदं स्थूलपरम्, सर्षपपदं च द्वणुकपरम्, तेन यदि द्व्यणुकं सावयवारब्धं स्यात्, महत् स्याद्, इति महत्त्वापत्तिरेव तुल्यपरिमाणत्वापत्तिः~। यद्वा द्वणुकत्रसरेण्वोः सावयवारब्धत्वेन तुल्यपरिमाणापादने तात्पर्यम्~॥ कि. प्र. व~॥\\ \rule{0.4\linewidth}{0.5pt}}अनन्तावयवारब्धत्वाविशेषेण मेरुसर्षपादीनां परिमाणभेदानुपपत्तेः~। न च कारणसङ्ख्याया अविशेषेऽपि ${}^2$परिमाणप्रचयविशेषाद्विशेष इति युक्तम्; तयोरपि सङ्ख्याविशेषाभावेऽनुपपत्तेः~। नापि प्रलयावधिः, कस्यचिदन्त्यस्य निरवयवत्वेन ${}^3$प्रलयानुपपत्तेः~। द्रव्यस्यावयवविभागविनाशाभ्यां विना विनाशाभावात्~। नापि विभागा \textendash\ वधिः, तस्य विभज्यमानाश्रयतया तदभावेऽनुपपत्तेः~। न चैकाश्रय एव विभाग इति युक्तम्, किञ्चिद्धि कुतश्चिद् विभज्यते न तु तदेव तस्मात्~। तस्मान्निरवयवावधिरयमवयवावय िप्रसङ्ग इति विज्ञायते~।

\blfootnote{1 च  \textendash\ कि~। 2 परिमाणप्रचयाभ्यां  \textendash\ क~। 3 तदनुपपत्तेः  \textendash\ क~।}

\newpage
${}^1$तर्कितमेतन्न तु ${}^2$प्रमितमिति चेत्; अत्र कश्चिदाह \textendash\ किमत्र प्रमाणगवेषणया, त्रसरेणोः ${}^3$अप्रत्यक्षप्रमाणसिद्धत्वात्~। तस्य चावयवकल्पनायां ${}^4$प्रमाणभावेन निरवयवत्वादिति~। तदसत्; तस्य चाक्षुषद्रव्यत्वेन महत्त्वादनेक$^5$द्रव्यवत्त्वाच्चेति~। यदि हि ${}^6$द्रव्यस्य चाक्षुषता महत्त्वमन्तरेण स्यात्, दूरदूरतरादौ न महत्परिमाणप्रकर्षमनुविदध्यात्~। यदि च साऽनेकद्रव्यवत्तां नाद्रियेत मइत्त्वमपि नाद्रियेत~। तद्व्यावृत्तौ तस्यापि व्यावृत्तेस्तस्य तत्कारणत्वात्~। अन्यथा तत्प्रकर्षं महत्त्वप्रकर्षो नानुविदध्यात्~। तस्माद् विवादा \textendash\ ध्यासितस्त्रसरेणुर्महान् अनेकद्रव्यवांश्च, अस्मादादिचाक्षुषद्रव्यत्वाद् घटवद् इति~। 

अपर आह \textendash\ अणुपरिमाणतारतम्यं क्वचिद्विश्रान्तम्, परिमाणतारतम्यत्वात्, महत्प \textendash\ रिमाणतारतम्यवदिति~। तदप्यसत्; \renewcommand{\thefootnote}{१}\footnote{परस्परेति \textendash\ परमाणुसिद्धौ महत्त्वविरुद्धाणुपरमाणुसिद्धिः, तत्सिद्धौ परमाणुसिद्धिरित्यर्थः~। न च महत्त्वापकर्षं इत्थं व्यपदिश्यते तस्य त्रुटावेव विश्रामात्~। कि. प्र. व~॥}परस्पराश्रयदोषप्रसङ्गात्~। यदि नाऽसौ परमाणुः, सन्तु तर्हि तदवयवा एव कार्यानुमिताः परमाणवः; न हि तेऽपि महान्तः, रूपविशेषतां महतां चाक्षुषत्वप्रसङ्गात्~। तथा च ${}^7$तेषां च कार्यत्वे प्रमाणानवकॢप्तौ तदवयवाननुमानात्~। यद्यपि चाणुत्वान्निरवयवत्वाच्च तेषां महत्परिमाणारम्भे न कारणं महत्त्वप्रचयौ स्तः; तथापि बहुत्वसङ्ख्याऽस्तीति सर्वं समञ्जसम्~। न; उक्तोत्तरत्वात्~। अवश्यं हि त्रसरेणोरवयवैर \textendash\ नेकद्रव्यैर्भवितव्यम् $\rightarrow$ ${}^8$तस्य महत्त्वात्, कार्य $\leftarrow$महत्त्वं प्रति चावयवगतस्यानेकद्रव्यत्वस्यापि कारणत्वात्~। अन्यथा महत्त्वस्य तत्प्रकर्षानुविधानानुपपत्तेः~। तस्मात् त्रसरेणोरवयवाः सावयवाः महद्द्रव्यारम्भकत्वात् तन्तुवत्~। ब\renewcommand{\thefootnote}{२}\footnote{बहवश्चेति \textendash\ परिमाणारम्भकबहुत्ववन्तः परिमाणप्रचयाजन्यमहत्त्वपरिमाणाश्रयद्रव्यजनकत्वादित्यर्थः~॥ कि. प्र. व~।\\ \rule{0.4\linewidth}{0.5pt}}हवश्च ते, परिमाणप्रचयानुपपत्तौ महदारम्भ \textendash\ कत्वात्, तुल्यपरिमाणप्रचयतन्त्वपेक्षयाधिकसङ् य$^9$पटारम्भकतन्तुवत्~। ते च विभज्यन्ते, सावयवत्वात्, घटवत्~। विपक्षे बाधकमुक्तम्~। तदवयवानां त्ववयवकल्पनानवकाशः, प्रमाणाभावाद् बाधकाच्च~। द्रव्यारम्भकत्वमूर्तत्वादीनामनुकूलत काभावात्; अनवस्थाप्रसङ्गस्य च प्रतिकूलस्य स्फुटत्वादिति परमाणुसिद्धिः~। 

\blfootnote{1 तर्कितमेव  \textendash\ क~। 2 नत्वनुमितिमूले, न प्रमाणमिति च टिप्पण्याम्  \textendash\ पा. ३. पु~। ३ प्रत्यक्षसिद्धत्वात्  \textendash\ क~। 4 प्रमाणाभावान्निरवयवत्वमिति  \textendash\ पा. ३. पु~। 5 द्रवत्वात्  \textendash\ पा ३. ३~। 6 स्वस्य  \textendash\ क~। 7 तत्कार्यत्वे  \textendash\ क~। 8 $\rightarrow$ $\leftarrow$ एतच्चिह्नान्तर्गतः पाठः 'क' पुस्तके नास्ति~। 9 पटारम्भक इति कि; जे पुस्तकयोर्नास्ति~।}

\newpage
तस्माद्यत्कार्यद्रव्यं$^1$ तत्सावयवम्, इति समनियमसिद्धौ कार्यद्रव्यत्वेन सावयवत्वं प्रसाध्य यत एवावयवात् कार्यद्रव्यत्वं व्यावर्तते तं निरवयवं परमाणुमुपपादयिष्याम इत्यभिप्राय \textendash\ वानाह \textendash\ {\knu कार्यलक्षणा त्वि}ति~। 'तु' शब्दः ${}^2$परमाणोर्व्यवच्छिनत्ति~। किमत्र प्रमाणम्, अत आह \textendash\ {\knu सा चे}ति~। येयमुक्तरूपा दृश्यते सा प्रत्यक्षसिद्धेत्यर्थः~। स्थैर्ये स्थिरता \renewcommand{\thefootnote}{१}\footnote{चिरकालेति \textendash\ यद्यप्येतज्जलादावप्यस्ति तथापि निबिडसंयोगविशेषवद्वृत्तिद्रव्यत्वसाक्षाद्व्याप्यजातिमत्त्वमभिप्रेतम्~। विष्टम्भकत्वं स्वाभाविकगुरूत्वाश्रयद्रव्यान्तरगतिप्रतिबन्धकत्वम्~। व्यूहविरोधित्वं च स्वाभाविकगुरुत्ववत् पतनप्रतिबन्धकत्वम्~॥ कि. प्र. व~।}चिरकालावस्थायित्वमिति यावत्~। आदिग्रहणाद् विष्टम्भकत्वं जलादिव्यूहविरोधित्वं च~। अवयवसन्निवेशास्तत्तत्सामान्यविशेषाभ व्यञ्जकसंस्थानविशेषाः~। स्थैर्यादीनि चावयवसन्नि \textendash\ वेशाश्च तैर्विशिष्टा~। ${}^3$न ह्येतद् द्रव्यान्तरे सम्भवति, जलादीनां ${}^4$यत्किञ्चित्स्पर्शवद्वेगवद द्रव्योप \textendash\ निपातमात्रेणैव भङ्गुरत्वात्~। न च पार्थिववदम्भो द्रव्यान्तरं गच्छद्विष्टभ्नाति, पूर्वव्यूहं वा विरुणद्धि~। न च पार्थिवद्रव्यवद् द्रव्यान्तरं तत्तज्जात्यभिव्यञ्जकसंस्थानमेदवदिति~। यतस्तत्संस्थानविशेषवती अत एवापरजातिबहुत्वोपेता~। ननु सर्वस्य चेतनादन्यस्य परार्थत्वात् किमनया चेतनप्रयोजनं साध्यत इत्यत आह \textendash\ {\knu शयनेति~।} शयनं शय्या, आसनमवस्थितिः~। आदिग्रहणाद् ${}^5$गृहकरणकृषिकर्मादि~। चकारादभिधीताद्यनेकोपकारकरी चेति~॥

तदिदं ${}^6$पारार्थ्यमस्याः किं विषयतयैव ? न इत्युच्यते~। किन्तु त्रैविध्येनेत्याह \textendash\ त्रिविधं चेति~। शरीरमिन्द्रियं विषय इति सञ्ज्ञा यस्य तदिदं तथोक्तम् ~। एतेन न शरीरत्वमिन्द्रियत्वं विषयत्वं वा जातिरस्ति; प्रथमद्वितीययोः पृथिवीत्वादिना परापरभावानु \textendash\ पपत्तेः~। तृतीयेऽपि निःसामान्योरपि सामान्यभावयोर्विषयत्वात्~। \renewcommand{\thefootnote}{२}\footnote{यदवच्छिन्न इति \textendash\ 'पादे मे सुखम्', 'शिरसि मे वेदना', इत्याद्यनुभवबलादात्मभिन्नेऽपि सुखस्यावच्छेद्यावच्छेदकभावः सम्बन्ध इत्यर्थः~। न च मृतशरीरादावव्याप्तिः, आत्मविशेषगुणकारणमनः संयोगवदन्त्यावयविमात्रवृत्तिजातिमत्त्वस्य विवक्षितत्वात् तत्रापि मनुष्यत्वादिजातिसत्त्वात्~॥ कि. प्र. व~॥}तस्माद्यदवच्छिन्ने आत्मनि भोगस्तदन्त्यावयवि शरीरम्~। \renewcommand{\thefootnote}{३}\footnote{शरीरसंयुक्तमिति \textendash\ यद्यप्येतद्दिक्कालप्राणाद्यतिव्यापकं तथापि स्मृत्यजनकज्ञानकारणमनःसंयोगाश्रयत्वमिन्द्रियत्वम्~। त्वगिन्द्रियस्यस्य सर्वज्ञानाजनकत्वान्न तत्राव्याप्तिः~। शब्देतरोद्भूतविशेषगुणानाश्रयत्वे सति ज्ञानहेतुमनःसंयोगाश्रयत्वं वा~। प्रतीयमानतयेति \textendash\ न चाग्रे वायोर्विषयत्वाभिधाने विरोधः, साक्षात्क्रियमाणत्वस्याविवक्षितत्वात्~। वायवीयस्पर्शस्य विषयत्वात्तदाधारत्वेन वायोर्विषयत्वोपचाराद्वा~। शरीरादेश्च विषयत्वेऽप्यज्ञायमानस्यापि भोगसाधनत्वात् पृथगुपादानम्~। नायं विभागः किन्तु पृथिवीजातीयस्य येन रूपेण पारार्थ्यं तद्रूपोपदर्शनमित्यप्याहुः कि प्र. व~॥\\ \rule{0.4\linewidth}{0.5pt}}शरीरसंयुक्तमतीन्द्रियं साक्षात्प्रतीतिसाधनमिन्द्रियम्~।

\blfootnote{1 कार्य~। 2 परमाणुं  \textendash\ जे~। 3 न चैतद्  \textendash\ पा ३. १~। 4 यत्तदुपनिपातं  \textendash\ क~। 5 भ्रमणकृषिकर्मादि  \textendash\ कि; जे; क~। 6 परार्थत्वमस्याः पा. १. पु~।}

\newpage
\noindent
प्रतीयमानतया भोगसाधनं विषय इति~। लक्षणरूपोपाधिनिबन्धना एताः सञ्ज्ञा इति~। चस्त्वर्थो वाय्वादिभ्यो भिनत्ति~। वायोर्हि चतुर्विधं कार्यं नभःप्रभृतीनां चैकविधमेवेति~॥

\hangindent=2cm {\knu (३१) ${}^1$शरीरं द्विविधं, योनिजमयोनिजं च~। तत्रायोनिज$^2$ मनपेक्ष्य शुक्रशोणितं देवर्षीणां धर्मविशेषसहितेभ्योऽणुभ्यो जायते~। क्षुद्रजन्तूनां यातनाशरीराण्यधर्मविशेषसहितेभ्योऽणुभ्यो जायन्ते~। शुक्रशोणितसन्निपातजं योनिजम्~। तद्विविधं जरायुजमण्डजं च~। मानुषपशुमृगाणां जरायुजम्~। पक्षिसरीसृपाणामण्डजम्~॥}

अनेकावान्तरप्रकारवच्छरीरारम्भिका पृथिव्येव, न द्रव्यान्तरमित्यभिप्रायवांस्तद्भेदं \ दर्शयति \textendash\ शरीरं द्विविधं योनिजमयोनिजं चेति~। ननु योनिजवन्नायोनिजमपि शरीरमुपलभामहे इत्यत आह \textendash\ तत्रेति~। ${}^3$देवानामृषीणामपि श्रूयते हि 'ब्रह्मणो मानसा मन्वादय' इति~। ननु \renewcommand{\thefootnote}{१}\footnote{कारणमिति \textendash\ शरीरं प्रति शुक्रशोणितसन्निपातस्य कारणत्वात्तमतिपत्य शरीरं न स्यादित्यर्थः~। कि. प्र. व~।}कारणमतिपत्य$^4$ यदि कार्यं भवेत् कार्यमेव न स्यात्, स्याच्च मृदादि \textendash\ व्यतिरेकेण घटादिरिति~। न; शरीरत्वस्योष्मजकृमिमशकादिमिर्व्यभिच रात्, संस्थानविशेष \textendash\ वत्त्वस्य चासिद्धेः~। त्रिदशाः खल्वेतेऽनेकमन्वन्तरायुषोऽनिमिषचक्षु षो नभश्चारिणः कामगास्त्रिनयनचतुर्भुजादयः$^5$~। अस्मादादयस्तु दशदशाः शतवर्षपरमायुषः परिभ्रमच्चक्षुषो भूचराः प्रतिहतगतयो द्विभुजनयनाः किञ्चित् ${}^6$संस्थानसाम्येऽपि गोगवयातिवज्जादिभेदो \textendash\ पपत्तेः~। न\renewcommand{\thefootnote}{२}\footnote{न वा संस्थानविशेषे योनिः कारणम्, अपि तु कर्मविशेष एव इत्याह \textendash\ न चेति~। कि. प्र. व~॥ ऐ\\ \rule{0.4\linewidth}{0.5pt}} च संस्थानमपि योनिजत्वे नियामकं ${}^7$मनुष्यगोगवयगजतुरगादीनां तदभावेऽपि योनिजत्वात्~। तस्मात् सर्वेषां स्वकर्मनिबन्धनो भोगः; तच्च तन्नान्तरीकतया जन्मायुषी आक्षिपति~।

\blfootnote{1 तत्र शरीरं  \textendash\ दे~। 2 मनपेक्ष  \textendash\ कि; व्यो~। 3 देवानामृषीणामिति \textendash\ पा. १. पु; देवानामृषीणां चेति  \textendash\ ३. पु.~। 4 परित्यज्य  \textendash\ पा. १. २ पु~। 5 मुखादयः  \textendash\ क~। 6 अवयवसंस्थान  \textendash\ पा. २. पु~। 7 गोगजोरगादीनां  \textendash\ किः गोगजभुजगादीनां  \textendash\ क; मनुष्यगोगजभुजगादीनां  \textendash\ पा २. ३ पु~।}

\newpage
\renewcommand{\thefootnote}{१}\footnote{क्व तर्हि शरीरे योनिः कारणमित्यत आह \textendash\ तत्रेति~। अयोनिजदेवादिशरीरे चागमः प्रमाणम्~। योनिं विना च शरीरमित्यत्र चागमे 'योनि' पदं कारणमात्रपरं मनुष्यदेहपरं वा~। 'चैत्रः तदीययोनिजशरीरान्यपार्थिवशरीरवान्, शरीरित्वान्मैत्रवद्' इत्यनुमानमप्याहुः~॥ कि. प्र. व~॥}तत्र येन कर्मणा गर्मवासादिदुःखमपि भोजयितव्यमस्ति तेन योनिजमुत्पाद्यते तदन्येन त्वन्यथेति युक्तमुत्पश्यामः~। तदेतदाह \textendash\ {\knu धर्मविशेषसहितेभ्योऽणुभ्य} इति~। अणुभ्य इति मूलकारणनिर्देशोऽयम्, न त्वणुभिरेवाहत्य शरीरमारभ्यते, तदेतदग्रे व्यक्तीकरिष्यति {\knu द्व्यणुकादिप्रक्रमेणेति}~। क्षुद्रजन्तूनां मशकादीनां यातनाशरीराणि~। यातना नरकपीडा तस्यैव ${}^1$शरीराणीत्यत्र नारकिणामिति शेषः~। क्षुद्रजन्तूनां शरीराणीति संस्थानानि, संस्थानभेदविवक्षया ${}^2$भोक्तृभोगायतनभेदविवक्षया वा षष्ठी~। अधर्मविशेषः प्रकृष्टोऽधर्मः, तत्सहितेभ्यः~। न हि योनिजं शरीरमाकल्पमयःकुम्भीपाकक्रकचदारणादि \textendash\ दुःखसहनक्षमम्, नापि नरकयातनासु योनिस्पर्शसुखमानुषङ्गिकमपि श्रूयते~। न च ${}^3$नरकाण्येव न सन्ति, ${}^4$स्वर्गेऽप्यनाश्वासप्रसङ्गादिति~।

इह 'योनि' शब्देन यदि ${}^5$कारणमात्रमुच्यते, द्विविधत्वं नोपपद्यत इत्यत आह \textendash\ शुक्रेति~। सन्निपातः परस्परमेलनं तज्जं योनिजम्~। पशवो ग्राम्याश्चतुष्पदाः, आरण्या मृगाः~। गर्भवेष्टनचर्मपुटकं जरायुः~। पक्षिणो विहङ्गमाः~। सरिसृपाः परितः प्रसरण \textendash\ शीलाः सर्पकीटमत्स्यादयः~। यद्यपि चोद्भिदोऽपि वृक्षादयः शरीरभेदतयाऽत्रैव व्याख्यातुमुचिताः तथाप्यतिमन्दान्तःसञ्ज्ञतया ${}^6$लौकिकापेक्षया सुखदुःख$^7$भोक्त्रधिष्ठा \textendash\ नत्वममीषामविवक्षन् प्रायैण जङ्गमोपकारकतया तदधीनतया च विषयतां विवक्षन् तेष्वेवान्तर्भाव्य व्याख्यास्यन्ते~। कयाचिद्विवक्षया ह्यन्तर्भूतमपि पृथग्व्याख्यायते~। यथा \textendash\ ऽयथार्थत्वेन विपर्ययत्वे सति स्वप्नसंशयज्ञाने लोकप्रसिद्धिमाश्रित्यः कयाचिद्विवक्षया पृथग्भूतमप्यन्तर्भाव्य व्युत्पाद्यते, यथा प्रमितेरन्यत्वेनाऽविद्यापि, यथार्थत्वेन स्मृतिर्विद्यायाम्~। कयाचिद्विवक्षया सदपि किञ्चिद्रूपमनभिधाय रूपान्तरमुच्यते; यथाऽत्रैवरूपरसगन्धानां पाकज \textendash\ त्वमनुक्त्वा सौरभाद्युक्तम्~। शरीरेन्द्रिययोर्द्व्यणुकादिप्रक्रमम पेक्ष्ययोनिजत्वाद्युक्तमित्याचार्यश लीयम्~। कथं पुनरेतत् ? इत्थम्; 'वृक्षादयः \renewcommand{\thefootnote}{२}\footnote{प्रतिनियतेति \textendash\ भोगजनकमनःसंयोगवन्त इत्यर्थः~। आत्मविशेषगुणजनकमनःसंयोगो जीवनम्~। ज्ञानजनकमनःसंयोगाजनकस्यन्दनजन्यविभागजन्यज्ञानजनकमनःसंयोगध्वंसो मरणम्~। स्वप्नवहनाडीमनः \textendash\ संयोगजन्यं ज्ञानं स्वप्नः~। तदसंयुक्तमनःसंयोगजन्यज्ञानं जागरणम्~। रोगो धातुवैषम्यम्~। चिकित्सा भेषजप्रयोगः~। कि. प्र. व~। रोगचिकित्सा इति प्रकाशसम्मतः पाठः~।\\ \rule{0.4\linewidth}{0.5pt}}प्रतिनियतभोक्त्रधिष्ठिताः, जीवनमरणस्वप्न

\blfootnote{1 शरीराणीति  \textendash\ कि; यातनाशरीराणीत्यत्र  \textendash\ पा. ३. पु~। 2 मोक्ष  \textendash\ क~। 3 नरकाएव  \textendash\ मु. कि; जे~। 4 नरके  \textendash\ क~। 5 कारणत्वमात्रमुच्यते तद्द्वैविध्यं  \textendash\ पा. २. पु~। 6 लोकाप्रसिद्ध्या जेः लौकिकापेक्षसुखःदुःख  \textendash\ क~। 7 भोगाधिष्ठानत्व  \textendash\ पा. ३. पु~।}

\newpage
\begin{sloppypar}
\noindent
प्रजागरणरोगभेषजप्रयोगबीजस\renewcommand{\thefootnote}{१}\footnote{सजातीयानुबन्धः \textendash\ सजातीयस्नेहः, स च ज्ञानरूपतया यद्यपि न शरीरधर्मस्तथापि तेन तद्व्यञ्जकचेष्टोपलभ्यते~। कि. प्र. व~।}जातीयानुबन्धा ुकूलोपगमप्रतिकूलापगमादिभ्यः, प्रसिद्ध \textendash\ शरीरवत्'~। न चैते सन्दिग्धासिद्धाः, आध्यात्मिकवायुसम्बन्धात्, सोऽपि मूले निषिक्ता \textendash\ नामपां दोहदस्य च पार्थिवस्य धातोरभ्यादानात्~। तदपि वृद्धिभग्नक्षतसंरोहणाभ्यामिति~। अन्यथा कारणं विना कार्यानुत्पत्तिप्रसङ्गे सर्वमि$^1$दमामूलविशीर्णमापद्येतेति संक्षेपः~। \renewcommand{\thefootnote}{२}\footnote{ननु वृद्ध्यादेर्गिरिपाषाणेन व्यभिचारोऽभ्यादानहेतुवायोः प्राणत्वेमानाभावश्चेत्यत आह \textendash\ आगमश्चेति~।
\begin{quote}
{\qt नर्मदातीरसञ्जाताः सरलार्जुनपादपाः~।\\
नर्मदातोयसंस्पर्शात्ते यान्ति परमां गतिम्~॥\\
स्मशाने जायते वृक्षः ${}^3$काकगृध्रोपसेदितः~॥}
\end{quote}
इत्याद्यागम इत्यर्थः~। न च तस्याधिकानिष्टपरत्वम्, बाधकं विना मुख्यान्यपरत्वाभावादिति भावः~। कि. प्र. व~।}आगमश्चात्रार्थे बहुतरोऽनु$^2$सन्धेय इति~।
\end{sloppypar}

कथं पुनर्मानुषादिशरीरं पार्थिवम् ? \renewcommand{\thefootnote}{३}\footnote{क्लेदः क्षरणम्~। पाक औष्ण्यम्~। व्यूहः क्रिया~। अवकाशो मुखादावन्नादिगत्यप्रतिबन्धः~। कि. प्र. व~॥\\ \rule{0.4\linewidth}{0.5pt}}गन्धक्लेदपाकव्यूहावकाशदानेभ्यः पाञ्चभौतिक \textendash\ मिति हि प्रसिद्धिः; सत्यम्, सा पञ्चानां कारणत्वमात्रेण न तु समवायितया~। अथ तथैव किं न स्यात् ? इत्थम्; द्व्यणुकादिप्रक्रमेण तावदयमारम्भ इति वक्ष्यते~। तत्र यदि विजाती \textendash\ यपरमाणुभ्यामे$^4$कमारभ्येत, अगन्धमरसमित्याद्यापद्येत; एकस्य गन्धादेरनारम्भकत्वात्~। आरम्भकत्वे वा सततोत्पत्तिप्रसङ्गादपेक्षणीयाभावा~। अथानेकस्यापि ${}^5$कुत एतन्न प्रसज्यते, समवायिकारणस्यापेक्षणीयत्वात्~। नह्यनपेक्षितसमवायिकारणमारभत इति चेत्; तदेतदेकत्रापि तुल्यम्~। न च तन्त्वादिगतानां रूपादीनामनेकेषामेवारम्भकत्वदर्शनादन यत्रापि तथा कल्पयितुं युक्तम्; कारणानां सजातीयतया तत्र तथोपपत्तेः~। विजातीयानामारम्भकत्वे त्वन्यथापि भविष्यति~। विपक्षे बाधकाभावात्~। तेन 'उष्णं जलम्','सुरभिः समीरणः' इत्याद्युपपद्यते~। न; एकेन गुणेन गुणारम्भे जातिसङ्करप्रसङ्गात्~। तथा हि गन्धवत्त्वादेव तद्द्व्यणुकं पृथिवी \textendash\ जातीयम्, सांसिद्धिकद्रवत्वाच्छीतत्वाच्च जलं स्यात्~। एवं शुक्लभास्वररूपवत्त्वादुष्णत्वात्तेजः

\blfootnote{1 मालूनं \textendash\ कि~। 2 प्रतिसन्धेय इति \textendash\ पा. ३. पु~। 3 कङ्कगृध्रोपसेवितः इति वा पाठः~। 4 मेकजातीयक$^\circ$\textendash\ जे; विजातीयपरमाणुभ्य एकं द्रव्यमारभेत  \textendash\ क~। 5 कुतः एतत्प्रसज्यते  \textendash\ क~।}

\newpage
\noindent
स्यात्~। अपाकजानुष्णाशीतस्पर्शवत्त्वा वायुः स्यात्~। तदयं प्रमाणार्थः ~। 'गन्धवन्ति द्व्यणुकानि गन्धवद्भिरेव परमाणुभिरारभ्यन्ते, कार्यत्वे सति गन्धवत्त्वात्, पार्थिवपरमाणु \textendash\ द्वयारब्धद्व्यणुकवत्'~। 'विवादाध्यासिताः परमाणवः, गन्धवन्त्येव द्व्यणुकान्यारभन्ते, गन्धवत्त्वात्'~। 'ये पुनर्निर्गन्धान्यारभन्ते न ते गन्धवन्तः; यधा निर्गन्धजलाद्यारम्भकाः परमाणवः'~। न च न ते गन्धवन्तः तस्माद् गन्धवन्त्येव द्व्यणुकान्यारभन्ते ~। न च दृष्टान्तासिद्धिः अनियमारम्भे सजातीयाभ्यामप्यारम्भाभ्युपगमात्~। अन्यथा निर्गन्धस्य जलादेनीरसस्यानलादेनीरूपस्य च पवनस्यानुपलम्भप्रसङ्गाः~। एवं 'सांसिद्धिकद्रवत्ववन्ति द्व्यणुकानि तथाभूतैरेव परमाणुभिरारभ्यन्ते, कार्यत्वे सति सांसिद्धिकद्रवत्ववत्त्वात्, आप्यपरमाणुद्वयारब्धद्व्यणुकवत्~। जलपरमाणवः, शीतान्येव द्व्यणुकान्यारभन्ते, शीतत्वात्~। ये पुनरशीतान्यारभन्ते ते न शीताः, यथाऽशीततेजोद्व्यणुकारम्भका इत्यादि स्वयमूहनीयम्~॥

तस्मात् सजातीयानामेवारम्भकत्वं न विजातीयानामिति स्थितम्~। तथापि शरीरस्य कथं पार्थिवत्वनिश्चयः ? इत्थम्; क्लेदादीनां परावृत्तावपि शरीरस्य \renewcommand{\thefootnote}{१}\footnote{प्रत्यभिज्ञानादिति मूलम्~। न च भुक्ते परिमाणान्यत्वादन्यदेव तच्छरीरमिति तत्रापि प्रत्यभिज्ञानमनुपपन्नमिति~। वाच्यम्; जात्यभेदावगाहमानेनापि तदुपपत्तेः~। कि. प्र. व्या. भ~।}प्रत्यभिज्ञानात्~। गन्धस्य च यावद्द्रव्यभावित्वाद् घटवत्~। यथाहि घटादौ पार्थिवे क्लेदशोषद$^1$शायामपि गन्धोऽनुवर्तते शैत्यादयस्तु कुसुमादिवासितजलादिगन्धवन्निवर्तन्ते तथा शरीरमपीति; तद्वत्पार्थिवत्वमवधार्यत इति~॥

\hangindent=2cm {\knu (३२) इन्द्रियं ${}^2$गन्धव्यञ्जकं सर्वप्राणिनां जलाद्यनभिभूतैः पार्थिवावयवैरारब्धं घ्राणम्~॥}

इन्द्रियमाह {\knu इन्द्रियमि}ति \textendash\ गन्धव्यञ्जकं यदिन्द्रियं ${}^3$तत्पार्थिवं घ्राणमिति व्यवहर्तव्यम्, रूपाद्यव्यञ्जकत्वे सति ${}^4$गन्धस्य व्यञ्जकत्वादित्यर्थः~। \renewcommand{\thefootnote}{२}\footnote{पार्थिवे घ्राणे मानमाह \textendash\ गन्धोपलब्धेत्यादि~। * गन्धोपलब्धिः रसाद्युपलब्धिजनकशरीर संयोगिकरणजन्या, तदन्यसाक्षात्कारत्वात्, रूपप्रत्यक्षवत्~। न च दृष्टान्तासिद्धिः, 'अन्धकारस्थं $\dagger$ गोलकं घटसाक्षात्कारकारणघटसंयोगाधिकरणतेजोवृत्तिसाक्षात्कारकारणसंयोगाधिकरणम्, रूपवत्त्वे सति $\ddagger$ घटसाक्षात्कारकारणत्वाद्, घटवत्' इति संयोगिसंयोगिव्यावृत्तकरणसिद्धेः~। 'तत्पार्थिवम्, रसाद्यव्यज्जकत्वे सति गन्धव्यञ्जकद्रव्यत्वात्, चम्पकाद्यधिवासाद्यधिकरणविततद्रव्यवत्'~। कुङ्कुमगन्धाभिव्यञ्जककुक्कुटोच्चारो न दृष्टान्तः, तस्य साक्षात्कारविषयत्वेन रूपादिव्यञ्जकत्वात्~। अत एव नवशरावगन्धव्यञ्जकस्थले न\\ \rule{0.4\linewidth}{0.5pt}}गन्धोपलब्ध्या तावत्करण

\blfootnote{1 दशयोरपि  \textendash\ पा. ३. पु. क~। 2 सर्वप्राणिनां गन्धव्यञ्जकं  \textendash\ दे~। 3 तत्घ्राणं  \textendash\ क~। 4 गन्धस्यैव व्यञ्जकम्  \textendash\ क~।}

\newpage
\noindent
मात्रं सिद्धम्~। क्रियायाः ${}^1$करणमात्रनिष्पाद्यत्वनियमात्~। तच्च प्राप्यकारि जनकत्वे सति तदप्राप्तावजनकत्वात्~। द्रव्यं च संयोगाश्रयत्वात्, पार्थिवं च गन्धवत्त्वात्, गन्धवत्त्वं च रूपाद्यव्यञ्जकत्वे सति ${}^2$गन्धाभिव्यञ्जकत्वात्, कुङ्कुमगन्धाभिव्यञ्जक तैलवत् इति~। {\knu सर्वप्राणिना}मितिप्राणिभेदेनेन्द्रियप्र कृतिवैचित्र्यं निवारयति~। यदि ${}^3$गन्धाभिव्यक्तौ शक्तिरस्ति पृथिव्याः, किं न सर्वैव पृथिव्येवमित्यत आह \textendash\ {\knu जलाद्यनभिभूतैरिति}~। अत एव श्लेष्माद्यभिभवे सति गन्धो न गृह्यते इति भावः~। {\knu घ्राणमि}ति \textendash\ लोक \textendash\ प्रसिद्धमित्यर्थः~। 

\begin{sloppypar}
\hangindent=2cm {\knu (३३) विषयस्तु द्व्यणुकादि$^4$प्रक्रमेणारब्धस्त्रिविधो मृत्पाषाणस्थावरलक्षणः~। तत्र भूप्रदेशाः ${}^5$प्राकारेष्टिकादयः मृत्प्रकाराः~। पाषाणा उपलमणिवज्रादयः स्थावराः ${}^6$तृणौषधिवृक्षगुल्मलतावतानवनस्पतय इति~॥}
\end{sloppypar}

तृतीयं च पृथिवीकार्यमाह \textendash\ {\knu विषयस्त्वि}ति यद्यपि~। शरीरमपि विषयः, तथापि शरीरत्वेनैव तदसाधारणं पुरुषार्थसाधनमिति दर्शयितुं तदेव रूपं तस्य दार्शितम्~। वक्ष्यमाणे तु विषयत्वमेवासाधारणमतस्तदुच्यते~। ${}^7${\knu द्व्यणुकादिप्रक्रमेणेति} \textendash\ द्व्यणुकमादिर्यस्य प्रक्रमस्य तथोक्तस्तेनारब्धो निष्पादितः~। मृत्त्वं मृत्तिकात्वम्~। पाषाणत्वं प्रस्तरत्वम्~। स्थावरत्वम् उद्भित्वं सामान्यविशेषाः~। तत् मृत्तिकात्वं केषु वर्तत इत्यत आह \textendash\ {\knu तत्रे}ति~। ${}^8$मृद्विकारा मृत्तिकारूपा विकाराः~। के पाषाणा इत्यत आह \textendash\ {\knu उपले}ति~। उपलाः ग्रावाणः~। मणयो मणिक्यादयः~। वज्रं हीरकम्~। आदिग्रहणाद् गौरिकादयः~। {\knu स्थावरा}

\blfootnote{व्यभिचारः, तेन स्वीयरसस्याप्यभिव्यक्तेः~। यद्वा 'गन्धसाक्षात्कारासाधारणकारणतावच्छेदकद्रव्यत्वव्याप्य (व्याप्य)जातिमत्त्वाद्' इति हेतुः~। जलस्य गन्धव्यञ्जकत्वं जलत्वेनैव न तु तद्व्याप्यजातिन्तरेण~॥ कि. प्र. व~॥

* गन्धोपलब्धिरिति \textendash\ लौकिकगन्धसाक्षात्कार इत्यर्थः~। साध्ये 'आदि' पदं स्पर्शाव्यञ्जकत्वगर्भसाध्यान्तरसूचनाय, अन्यथा व्यर्थविशेषणतापत्तेः~। रूपाव्यञ्जकत्वगर्भं च न साध्यम्, तैलदृष्टान्तानुपपत्तेः~। परकीयविशेषणे च गौरवात्~। कि. प्र. व्या. भ~।

$\dagger$मानुषेति च गोलकविशेषणम्, अतो न मार्जारनयनस्थिततेजसाऽर्थान्तरम्~। कि. प्र. व्या. भ~।

$\ddagger$घटेति \textendash\ घटसाक्षात्कारकारणं घटसंयोगः, तदधिकरणं यत्तेजः, तद्वृत्तिर्यो घटसाक्षात्कारकारणं संयोगः, तदधिकरणमित्यर्थः~। कि. प्र. व्या. भ~।\\
\rule{0.4\linewidth}{0.5pt}

1 कारणनिष्पाद्यत्व  \textendash\ कि~। 2 गन्धस्य व्यञ्जक्त्वात्  \textendash\ कि~। 3 गन्धाभिव्यक्तिसामर्थ्य  \textendash\ मस्ति  \textendash\ पा. ३. पु; क~। 4 क्रमेण  \textendash\ कं~। 5 प्राकारेष्टकादयः  \textendash\ कं~। 6 वृक्षतृणौषधि$^\circ$\textendash\ कि; जे~। 7 प्रतीकमिदं 'क' पुस्तके नास्ति~। 8 मृतिकाकारा विकारा मृद्विकाराः  \textendash\ पा. ३ पु~।}

\newpage
\noindent
इति \textendash\ वृक्षाः पुष्पफलवन्तः स्कन्धशाखिनः~। तृणान्युलपादीनि~। ओषधयः फलपाकान्ताः ${}^1$यवादयः~। गुल्मा झाटाः$^2$~। लताः ${}^3$कुष्माडीप्रभृतयः~। अवतानाः केतक्यादयः~। वनस्पतयो विना पुष्पं फलिनः~। अत्र च शरीरग्रहणेनैव तदवयवा मांसास्थ्यादयस्तद्विकाराश्च क्षीरघृतादयः संगृहीताः~। इन्द्रियग्रहणेनैव तदवयवाः~। विषयग्रहणेनैव द्व्यणुकादयः~। पाषाणग्रहणेनैव हरितालादयः~। स्थावरग्रहणैनैव पत्रपुष्पादयः~। वृक्षग्रहणेतैव काष्ठादयः~। तृणग्रहणेनैव तालादयः~। ओषधिग्रहणेनैव तद्विकारास्तन्दुलतैलादय इति~।

ननु कोऽयं द्व्यणुकादिपक्रमः ? द्र्यणुक एव माना$^4$ भावात्~। त्रसरेणवस्ताव \textendash\ त्प्रत्यक्षसिद्धाः~। ते च ${}^5$चाक्षुषद्रव्यत्वादेव महान्तः, ${}^6$द्रव्यस्य चाक्षुषतां प्रति रूपविशेष \textendash\ वन्महत्त्वस्यापि कारणत्वात्~। अन्यथा दूरदूरतरादौ तत्प्रकर्षानुविधानानुपपत्तेः~। महत्त्वे सति रूपवत्त्वात् क$^7$र्मवत्त्वाच्च ते सावयवाः कार्याश्च~। अन्यथाऽव$^8$यवगतसङ्ख्याद्यनुपपत्तौ ${}^9$महत्त्वानुपपत्तेः~। नित्यत्वे प्रकृष्टकाष्ठाप्राप्ताल्पभावस्य महत्त्वविरोधात्~। न च त्रसरेण्वयवा अपि सावयवाः, तदवयवक्ल्पनायां प्रमाणाभावात्~। नहि त्रसरेणुवत् तेऽपि महन्तः, रूपविशेषवतां महतां चाक्षुषत्वप्रसङ्गात्~। यद्यपि च तेऽपि महनत्त्वव्यावृत्तेरणुपरिमाणाः, निरवयवत्त्वाच्च प्र$^10$चयवञश्चिताः, तथापि ${}^11$बहुत्वान्महान्ति कार्यद्रव्याण्यारप्स्यन्ते~। तथा च परमाणव एव त्रसरेण्ववयवा इति, तत्कुतो द्वणुकम् ?~।

\begin{sloppypar}
उच्यते; महतः ${}^12$कार्यद्रव्यस्य कार्यद्रव्यारभ्यत्वनियमात् द्व्यणुकसिद्धिः~। अन्यथा परमाणुभिरेव बहुत्वसङ्ख्योपेतैर्महद्द्रव्यारम्भे गोघटादेरपि ${}^13$तैरेवारम्भादान्तरालिकाल्पाल्प\renewcommand{\thefootnote}{१}\footnote{आन्तरालिकेति  \textendash\ घटे नष्टे कपालशर्कराचूर्णादिकार्यं नोपलभ्येतेत्यर्थः~। कि. प्र. व~॥\\ \rule{0.4\linewidth}{0.5pt}}  \textendash\ तरादि कार्योपलम्भविरोधात्~। ${}^14$तथा च द्वणुकवदनुपलक्षितरेखोपरे$^15$खादिर्गोघादिरपि स्यात्~। तथा च संस्थानभेदानुपलम्भे तदभिव्यङ्ग्यं घटत्वादिकमपि नोपलभ्येत ~। न चारभ्याप्यारभन्त इति तदुपलम्भोपपत्तिर्भूर्तानां ${}^16$समानदेशताऽविधानात्~। न्यूनाधिक \textendash\ देशतया समानदेशत्वेऽप्यविरोध इत्यपि नास्ति~। न्यूनदेशतया हि यत्र नास्ति तत्र 
\end{sloppypar}

\blfootnote{1 कदल्यादयश्च इत्यधिकम्  \textendash\ २. पु; कलमादयः  \textendash\ क~। 2 राटाः  \textendash\ कि; भाटाः  \textendash\ क~। 3 कुश्माण्डी  \textendash\ पा. ३. पु~। 4 प्रमाणाभावात्  \textendash\ पा. ३. पु; क~। 5 चाक्षुषद्रव्यतया  \textendash\ पा. ३. पु~। 6 रूपवद्द्रव्यस्य  \textendash\ पा. ३. पु~। 7 क्रियावत्त्वाच्च  \textendash\ क~। 8 सावयव  \textendash\ क~। 9 अवयविमहत्त्वानुपपत्तेः  \textendash\ पा. ३. पु~। 10 प्रशिथिलप्रचय  \textendash\ पा. ३. पु~। 11 बहुत्वसङ्ख्यास्तीति महत्कार्यद्रव्याणि  \textendash\ पा. ३. पु; बहुत्वान्महत्कार्यद्रव्याणि \textendash\ क~। 12 कार्यनियमाद्द्वणुकसिद्धिः  \textendash\ क~। 13 रम्भसभ्भवा$^\circ$ \textendash\ क~। 14 तथा हि  \textendash\ पा. ३. पु~। 15 रेखादिर्घटादिरपि  \textendash\ कि~। 16 समानदेशत्वानुपपत्तिः  \textendash\ कि~।}

\newpage
\noindent
अधिकदेशस्य स्थितिं मा विरौत्सीत्, स्वदेशो तु विरुन्ध्यादेव~। तत्राप्यविरोधे मूर्तत्वं जह्यात्~। अवयवरूपादिवत् संयोगिन्यपि वा देशे न्यूनाधिकदेशतया समानदेशत्वाविरोधः स्यात्~। न ${}^1$चोदञ्च\renewcommand{\thefootnote}{१}\footnote{उदञ्चनं * शरावः~। कि. प्र. व~। (??? ?)

*यदपि घटशरावोदञ्चनादिभेदेनेति चतुर्थटीकादर्शनाद् उदञ्चनशरावयोर्भेदस्तथापि 'शराव' पदेनात्र शराव उक्तः~। टीकायामपि सामान्यविशेषभेदेनैव पौनरुक्त्यम्~। यद्वा, उदञ्चनं शराव एव, टीकायां 'तूदञ्चन' पदेन ऊर्ध्वमञ्चनतमस्येति व्युत्पत्त्या ऊर्ध्वगमनशीलं वस्त्वन्तरमेवोक्तम्~। कि. प्र. व. भ~।}नावच्छिन्ने देशे घटादिः संयोगेन वर्तते~। कथं तर्हि चरमादितन्त्वप \textendash\ कर्षणेऽल्पतरतमादिपटोपलम्भः, इति चेत्; प्रतिबन्धकविगमेऽवस्थितेभ्यः संयोगेभ्यः खण्डपटोत्पत्तेः~। आद्यादितन्त्वपकर्षणे त्वयाप्येषैव रीतिर$^2$नुमन्तव्या ~। ${}^3$अन्यथा द्वितन्तु \textendash\ कादिपर्यन्तसमस्तकार्यविनाशे खण्डपटानुत्पत्तौ च तन्त्वतिरिक्तं न किञ्चिद् लभ्येत~। नन्वेष विचारो द्व्यणुकेष्वपि समानः, तैरपि हि यदि बहुभिः कार्यद्रव्यामारभ्यते, घटादयोऽप्यारभ्ये \textendash\ रन्; तथा चान्तरालिककार्यपरम्परानुपपत्तौ तदनुपलम्भप्रसङ्गः~। अ$^4$थ तैस्त्रसरेणुरेवारभ्यत इति नियमः~। परमाणुभिरपि स एवारभ्यत इति तुल्यम्~। ${}^5$हातव्योऽयं पक्षो विशेषो वा वक्तव्यः~।

\begin{sloppypar}
उच्यते~। यथा ह्यवयवावयविप्रसङ्गः परमाणुषु विश्राम्यति तथाऽयमवयवसङ्ख्या \textendash\ पकर्षः क्वचिद्विश्राम्येत्~। न च त्रित्वमपकर्षकाष्ठा द्वित्वैकत्वयोः सम्भवात्~। तत्रैकम् अनारम्भकम्, अवयवसंयोगानुपपत्तावसमवायिकारणाभावात्~। अनेकसङ्ख्याभावे वाऽवय \textendash\ विनोऽवयवापेक्षयाऽधिकपरिमाणानुपपत्तेः~। अवयवापेक्षया चावयविनोऽधिकपरिमाणत्व \textendash\ नियमात्~। एकस्य चावयवस्य विभागानुपपत्तौ तत्कार्यस्य विनाशासम्भवे नित्यत्वप्रसङ्गादिति~। तथा च द्वित्वमपकर्षपर्यन्तस्तदादिरयमारम्भ ~। तच्च द्वाभ्यां \textendash\ परमाणुभ्यामारब्धं कार्यं ${}^6$न महत् स्यात्~। तथात्वे तस्य कारणाभावात् तदपि च द्व्यणुकं यदि नारभेत्; यदि हि द्वित्वसङ्ख्या \textendash\ युक्तमेवारभेत~। उभयथाऽपि तदुत्पादनवैरर्थ्यं स्यात्~। तस्मात्तेन महद्द्रव्यमारभ्यमाणं बहुत्वसङ्ख्यायोगिनैवेति नियमः~। न च द्व्यणुकवत्परमाणवोऽपि बहुत्वसङ्ख्यामाश्रित्य महत्कार्य मरप्स्यन्त इति युक्तम्; \renewcommand{\thefootnote}{२}\footnote{अनियमारम्भस्येति \textendash\ त्रसरेणोः कस्यचिद्व्यणुकेन कस्यचिच्च परमाणुभिरारम्भे नियतहेतुत्वं न स्यादित्यर्थः~। नन्वणुपरिमाणाश्रयत्वेन कारणानुगमः स्यादित्यत आह \textendash\ कारणकारणस्येति~। त्र्यणुककारणद्व्यणुककारणपरमाणोः स्वकार्यद्व्यणुकस्य यत्कार्यजातीयं त्रसरेणुजातीयं तदारम्भकत्वं नास्तीत्यर्थः~। यद्वा, आरम्भकवादनिषेधाभिधायक एवायं ग्रन्थः~। कि. प्र. व~॥\\ \rule{0.4\linewidth}{0.5pt}}अनियमारम्भस्य निषिद्धत्वात्~। कारणकारणस्य च 
\end{sloppypar}

\blfootnote{1 चोदञ्चनादिच्छिन्ने  \textendash\ जेः क~। 2 रनुसरणीया  \textendash\ क~। 3 तथाहि  \textendash\ पा. २. पु; तदाहि  \textendash\ क~। 4 अपरैः  \textendash\ क~। 5 विशेषो वक्तव्यः  \textendash\ एतावानेन 'कं' पुस्तके~। 6 न स्यात्  \textendash\ कि; न महत्  \textendash\ क~।}

\newpage
\noindent
कार्यकार्यजातीयानारम्भकत्वात्~। न हि तन्तवः पटमारभन्ते, तत्कारणं चांशव इति~। \renewcommand{\thefootnote}{१}\footnote{तस्मादिति \textendash\ ननु साधनावच्छिन्नसाध्यव्यापकं महावयवारभ्यत्वमुपाधिः~। तथा 'त्रसरेणुर्नित्यद्रव्यारभ्यः, द्रव्यारभ्यत्वे सति महदनारभ्यत्वात्, परमाणुगन्धवद्', इति प्रतिरोधः~। अत्राहुः \textendash\ 'अस्ति तावन्महत्त्वात्मकं कार्यम्, तत्प्रति कारणत्वं कार्यद्रव्यत्वेनैव गृहीतम्~। न च महत्त्वविशेष एव तत्त्वेन जनकत्वम्, त्रसरेणौ व्यभिचाराद् इत्यकारणकार्योत्पत्त्या विपक्षबाधकेन साधनस्य साध्यव्याप्यतया तदव्यापकत्वेन, उपाधेः साध्याव्यापकत्वात्, द्व्यणुकारब्धत्रसरेणावेव साध्याव्यापकत्वात्~। यद्वा 'त्रसरेणुर्ननित्यद्रव्यसमवेतः, शब्दाजन्यत्वे सति अस्मदादिबहिरिन्द्रियग्राह्यत्वाद्, घटवत्'~। न च प्रत्यक्षसमवेतत्वमुपाधिः, अप्रत्यक्षव्यक्तिप्रत्यक्षगन्धादौ साध्याव्यापकत्वादित्यत्र तात्पर्यम्~। कि. प्र. व~॥\\ \rule{0.4\linewidth}{0.5pt}}तस्माद् विवादाध्यासितास्त्रसरेणवो न निरवयवद्रव्यारब्धाः सावयवद्रव्यारब्धा इति वा महत्त्वे सति कार्यत्वात् उभयवादिसिद्ध्यणुकारब्धत्रसरेणुवत् घटवद्वेति~। एवं तर्हि ' द्वणुकमपि न निरवयवद्रव्यारब्धं सावयवद्रव्यारब्धं वा, कार्यद्रव्यत्वात् $\rightarrow$ ${}^1$त्रसरेणुवद् घटवद्वेति स्यात्~। न स्यात् $\leftarrow$ अनवस्थाप्रतिहतत्वादिति द्व्यणुकादिप्रक्रमसिद्धिः~। ${}^2$पर्यवसिता पृथिवी ' इति~॥

\begin{sloppypar}
\hangindent=2cm {\knu (३४) अप्त्वाभिसम्बन्धादापः~। रूपरसस्पर्शद्रवत्वस्नेह सङ्ख्यापरिमाणपृथक्त्वसंयोगविभाग$^3$परत्वापरत्व गुरुत्व$^4$संस्कारवत्यः~॥}
\end{sloppypar}

अथेदानीमपो निरूपयति \textendash\ {\knu अप्त्वाभिसम्बन्धादिति~।} अप्त्वं नाम सामान्यविशेषः; तेनाभिस$^5$म्बन्धादिति पूर्ववद् व्याख्येयम्~। तासां द्रव्यत्वसिद्धये गुणानाह {\knu रूपरसेत्यादि~।}

\hangindent=2cm {\knu (३५) एते च ${}^6$पूर्ववत्सिद्धाः~।}

तत्र प्रमाणं सूचयति \textendash\ ${}^7$एते चेति~। पूर्ववदिति यथा पृथिव्यां सूत्रकारवचनाद् गन्धादयः सिद्धास्तथाऽप्सु रूपादयः सिद्धा इत्यर्थः~। तथा च सूत्रम् 'रूपरसस्पर्शवत्य आपो द्रवाः स्निग्धाश्च' (वै. सू. २ \textendash\ १ \textendash\ २) इति~। यथा च चाक्षुषवचनात् सप्त \textendash\ सङ्ख्यादयः पृथिव्यां तथा अप्स्वपि~। यथा पतनोपदेशाद् गुरूत्वं पृथिव्यां तथा तोयेऽपि~। यथा चोत्तरकर्मवचनात् पृथिव्यां संस्कारस्तथा पाथःस्वपीति~।

\hangindent=2cm {\knu (३६) शुक्लमधुरशीता एव रूपरसस्पर्शाः~। स्नेहोऽम्भस्यैव सांसिद्धिकं च$^8$ द्रवत्वम्~॥}

\blfootnote{1 $\rightarrow$ $\leftarrow$ एतञच्चिह्नान्तर्गतः पाठः 'क' पुस्तके नास्ति~। 2 इति पृथिवी समाप्ता  \textendash\ क~। 3 परत्वगुरूत्व  \textendash\ दे~। 4 संस्कारवत्यश्च  \textendash\ कि~। 5 सम्बन्धादित्यादि  \textendash\ पा. १ पु~। 6, 7 पूर्ववदेषां सिद्धिः  \textendash\ क~। 8 द्रवत्वं च  \textendash\ कि~।}

\newpage
एवं गुणवत्त्वेन द्रव्यत्वे सिद्धे ${}^1$पृथिव्यादिभ्यो व्यवच्छेदहेतोरप्त्वस्य व्यवस्थाहेतून् गुणविशेषानाह \textendash\ {\knu शुक्लमधुरशीता एव रूपरसस्पर्शा} इति~। सहस्रशोऽग्निसंयोगे \textendash\ नाप्यपरावर्तमानमभास्वरशुक्लमेव रूपमप्सु$^2$~। नहि पार्थिवपटस्फटिकादौ वर्तमानं शुक्लमपि रूपमग्निसंयोगान्न परावर्तते~। ${}^3$अपरावृत्तमपि वा ${}^4$तेजसि भास्वरम्~। तस्मादनेन विशेषेण रूपमप्त्वव्यवस्थाहेतुः~। ${}^5$एष च विशेषः पृथिवीरूपस्य पाकजत्वप्रदर्शनेन तेजोरूपस्य भास्वरत्वख्यापनेन सूचितः~। उक्तयुक्तेरेव जलादिरूपस्यापाकजत्वम्~। पार्थिवा हि सं$^6$सृष्ट \textendash\ भागास्तत्र क्वाथेन पच्यन्ते एतेन रसोऽपि व्याख्यातः~। नहि पार्थिवे क्षीरशर्करादौ वर्तमानमपि माधुर्यं पाकेन [ न ] व्यावर्तते~। इह तु सहस्रशः पच्यमानमपि माधुर्यमेव; ईषत्स्वाभावं तु तत्तदपि तथैव~। यत्पुनरम्बुसिन्धुभेदेन क्षारादिरपि रसः ${}^7$पाथसि उपलभ्यते स दशमूलकषायस्यैव पार्थिवद्रव्योपाधिकः~। कथमन्यथा तस्यैव घनपीतमुक्तस्य रसो मधुर एव~। एवमपां स्पर्शोऽपि शीत एव~। क्वचित्तु सन्नपि तेजोभिभवान्नोपलभ्यते, तदपगमे पुनरुपलब्धेरिति~। तदेभिर्विशेषै रूपरसस्पर्शा आप्त्वव्यवस्थापका इत्युक्तम्~।

स्नेहस्तु स्वरूपत एवेत्याह \textendash\ स्नेहोऽम्भस्यैव~। अत्र केचिदाहुः ' न स्नेहो जलधर्मः, तैलादिवज्जलेऽनुपलब्धेः; नापि पृथिव्या विशेषगुणः, सर्वपृथिव्यव्यापकत्वात्~। तस्मात् पार्थिवविशेषेषु घृततैलवसादिषूपलभ्यमानः सामान्यविशेषोऽयम्, गव्यमाहिषादिषु दधित्ववद्' इति~। तदसत्; स्निग्ध$^8$स्निग्धतरादिभेदेन सातिशयत्वात्~। न च गोमहिष्या \textendash\ दिषु गुणमनाश्रित्य तारतम्यमस्ति~। नहि इषन्मधुरादिवद् गौरतिशयेन गौरिति भवति~। तस्मादतिशयस्य गुणैकनिमित्तत्वात् सामान्यविशेषतया च स्नेहस्य सामान्येऽसम्भवाद् गुण एव स्नेहः~। किञ्च स्निग्धद्रव्यावयवपरम्पराया अपि स्निग्धत्वादापरमाणुं स्नेहः स्यात्~। न च परमाणुषु पृथिवीत्वादेरन्या अपरा जातिरस्ति; व्यवस्थापकाऽभावात्~। क्षीरत्वादिवद्भविष्यति इति चेत्; न; तस्यापि परमाणुष्वभावात्~। पाकजरूपरसस्पर्शविशेषव्यङ्ग्यं हि क्षीरत्वादि~। तथा च पाकान्तरेण तन्निवृत्तौ क्षीरत्वादेर्निवृत्तिप्रसङ्गात्~। न च सत्यां व्यक्तौ सामान्यनि \textendash\ वृत्तिरिष्यते न चाविवक्षितसत्ताकाः सम्भवमात्रेण ${}^9$तेऽस्याभिव्यञ्जकाः, सर्वपार्थिवपरमाणूनां क्षीरत्वापत्तौ तदारब्धसर्वकार्यस्य पृथिवीत्ववत् क्षीरत्वापत्तेः; द्व्यणुकादौ क्षीरत्ववद् भविष्य \textendash\ तीति चेत्, नः तदारम्भकेषु परमाणुषु तद्व्यवस्थाहेतोः क्षीरत्ववद् गुणविशेषस्याभावात्;

\blfootnote{1 पृथिव्यादिव्यवच्छेदहेतून् विशेषानाह  \textendash\ क~। 2 मप्सु व्यवस्थहेतुः  \textendash\ क~। 3 अपरावर्तमानमपि  \textendash\ पा. १. पु; क~। 4 तेजसो  \textendash\ पा. २. पु; क~। 5 एष एव  \textendash\ पा. २ पु~। 6 द्रव्यभागास्तत्र  \textendash\ पा. २. पु; दृष्टमागाः  \textendash\ क~। 7 पयसि \textendash\ पा. २ पु; पाथस  \textendash\ कि~। 8 स्निग्धतरस्निग्धतमादि  \textendash\ क~। तस्याभिव्यञ्जकाः  \textendash\ मु. कि~।}

\newpage
\begin{sloppypar}
\noindent
भावे वा स एव स्नेहः, न ह्यत्र ${}^1$रूपादयो व्यवस्थाहेतवः, ${}^2$नापि तद्विशेषाः, तेषां घृतत्वादिव्यवस्थापकत्वात्~। न च$^3$ व्यवस्थाकारणं कारणसामान्यमस्ति, तैलघृतवसानां बीजक्षीरमांसयोनित्वात्~। अत एव ${}^4$नारिकेलस्नेहस्यापि तैलत्वम्, रसादिसारूप्यादपि तु घृ$^5$तमिति व्यपदेशः~। तस्मात् परमाणुषु वर्तमानकारणगु$^6$णपूर्वक्रमेणान्त्यावयवि र्यन्तमायातो गुणविशेषः स्नेह इति युक्तम्~॥
\end{sloppypar}

न च घृतादिष्वेव स्नेहोपलब्धिः, ${}^7$सातिशयो हि स्नेहस्तत्रोपलभ्यते, गुरुत्वमिव सुवर्णे~। वस्तुतस्तु ${}^8$सर्वजलसाधारणी स्निग्धता, अन्यथा जलावसेकेन सक्तुसिक्तादीनां सङ्ग्रहो न स्यात्~। स हि स्नेहद्रवत्वकारितः संयोगविशेषो धारणाकर्षणहेतुः ~। न ${}^9$चासौ केवलद्रवत्वहेतुकः, काचकाञ्चनादिभिरति$^10$द्रवीभूतैरपि वालुकाद्यसङ्ग्रहात्~। नापि स्नेह \textendash\ ${}^11$मात्रकारितः, स्त्यानैर्घृतादिभिरप्यसङ्ग्रहात्~। तस्मात् सकलजलहेतुकः सङ्ग्रहो भवंस्तत्र द्रवत्वमिव स्नेहमपि ${}^12$सांसिद्धिकमवस्थापयति~। तथापि नैमित्तिकद्रवत्वमिव नैमित्तिकः पृथिव्यामयं भविष्यति, पार्थिवेष्वपि घृतादिषूपलब्धेः, इति चेत्; तदन्वयव्यतिरेकाननु \textendash\ विधानात्~। नहि यथा जतुघृतादिष्वद्रवेषु पावकसम्पर्काद् द्रवत्वमुत्पद्यते तद्विरोधिसलिला \textendash\ वसेकाच्च निवर्तते तथा स्नेहः~। तस्मात् पाकजतथाविधगन्धरसरूपादिमद्भिः परमाणुभिद्व्यणु \textendash\ कादिक्रमेण घृतादिद्रव्यमारभ्यते~। तत्र चोपष्टम्भकतया निमित्ततामापन्नाः पानीयावयवास्तेषां संयुक्तसमवायेन स्नेहस्तत्रोपलभ्यत इति स्थितम्~।

\begin{sloppypar}
{\knu सांसिद्धिकं च द्रवत्वम}म्भस्येवेत्यनुकर्षणार्थश्चकारः~। ${}^13$तैलघृतादिषु ${}^14$पार्थिवत्वसिद्धेः संयुक्तसमवायात्तदुपलब्धिः~। शैत्यस्नेहवतां तेषां पार्थिवत्वमेव कथम् ? इति चेत्, तैलस्य भौमानलेन्धनत्वाद् घृतवत्~। ${}^15$जतुक्षीरस्य च पिण्डीभावेऽपि प्रत्यभि \textendash\ ज्ञानाद् घृतवदेवेत्यादि स्वयमूहनीयम्~।
\end{sloppypar}

\hangindent=2cm {\knu (३७) ता$^16$स्तु पूर्ववद्द्विविधाः, नित्यानित्यभावात्~। तासां कार्यं च त्रिविधम्~। शरीरेन्द्रियविषयसञ्ज्ञकम्~॥}

\blfootnote{1 रूपरसगन्धादयो  \textendash\ पा. २. पु~। 2 न च  \textendash\ पा. २. पु; क~। 3 न च कारणसामान्यमस्ति  \textendash\ क~। 4 नारिकेलफलस्यापि \textendash\ पा. २. पु~। 5 घृतमिति व्यपदेशः \textendash\ कि~। 6 ${}^\circ$गुणक्रमेण  \textendash\ कि; गुणक्रमेण \textendash\ क~। 7 सातिशयः स्नेहः \textendash\ कि~। 8 सर्वजलधारिणि  \textendash\ कि; जलसाधारणी  \textendash\ क~। 9 न वासौ  \textendash\ कि; जे~। 10 रभिद्रवीर्भूतै  \textendash\ पा. ३. पु~। 11 स्नेहमात्रहेतुकः स्नेहवद्भिरभि पिण्डीभूतैर्घृतादिभिः \textendash\ पा. २. पु~। 12 तस्मादुभयमपि व्यवस्थापयति \textendash\ पा. २. पु~। 13 तैलक्षीरा  \textendash\ दिषु  \textendash\ कि; क~। 14 पार्थिवत्वसिद्धिः  \textendash\ कि~। 15 क्षीरस्य च  \textendash\ क~। 16 ताश्च  \textendash\ कं~।}

\newpage
{\knu तास्तु पूर्ववदेव द्विविधाः; नित्यानित्यभावात्~।} 'तु' शब्दोऽप्यर्थः ~। पृथिवी तावद्द्विविधा, आपोऽपि$^1$ तथैव~। पूर्ववदिति कार्यतया परमाणुतया चेति~। यथा पृथिवी नित्याऽनित्या च आपोऽपि तथेत्यर्थः~। 'ता' इत्यत्र ${}^2$विभक्तिविपरिणामेन {\knu तासां कार्यं त्रिविधम्, शरीरेन्द्रियविषयसञ्ज्ञकम्~।} तदेतन्निगदव्याख्यानं पृथिव्या \textendash\ मिवात्रापि तात्पर्यमूहनीयम्~।

\begin{sloppypar}
\hangindent=2cm {\knu (३८) ${}^3$तत्र शरीरमयोनिजमेव वरुणलोके, पार्थिवावयवोपष्टम्भा$^4$च्चोपभोगसमर्थम्~॥}
\end{sloppypar}

आपः ${}^5$क्वचिच्छरीरारम्भिकाः, द्रव्यारम्भकत्वात्, पृथिवीवत्~। द्रव्यस्य हि द्रव्यारम्भणयोग्यता शरीरेन्द्रियविषयभावेन व्याप्ता यदेतेष्वेकमप्यारभते तदन्यदपि, यन्नैकं नैतदन्यदपि, इत्यन्वयव्यतिरेकाभ्यामुपलम्भात्~। ततो यद्यापः शरीरं$^6$ नारभेरन्निन्द्रिय \textendash\ विषयावपि नारभेरन्~। न चैत्रम्; तस्मात् क्वचिच्छरीरमप्यारभन्ते, इति स्थिते कीदृक् शरीरमित्यपेक्षायमाह \textendash\ {\knu शरीरमयोनिजमेवेति~।} शुक्रशोणितयोर्नियमेन पार्थिवत्वात्~। पार्थिवेन च पाथसीयानामनारम्भात् तदयोनिजमेव~। नन्वनुपलब्धिबाधितमेतद् ? इत्यत आह \textendash\ {\knu वरुणलोके} इति~। व्यवहितविप्रकृष्टत्वान्नोपलभ्यते; न त्वभावात्~। आगमादु \textendash\ पलब्धेरिति भावः~। ननु ${}^7$स्वभावद्रवत्वादपां तदारब्धं कथमुपभोगायतनं तद् ? इत्यत आह \textendash\ {\knu पार्थिवेति}~। यथा कठिनत्वात् पृथिव्यास्तदारब्धमप्यवययोपष्टम्भा$^8$दु भोग \textendash\ समर्थम्~। तथाऽप्यत्वाद्द्रवमपि पार्थिवाद्यवयवोपष्टम्भात् तदप्युपभोगसमर्थम्~। विजाती \textendash\ यारम्भकत्वमात्रं हि प्रतिषिद्धं न तु परस्परानुग्रहोऽपीत्यर्थः~।

\hangindent=2cm {\knu (३९) इन्द्रियं ${}^9$रसव्यञ्जकं${}^10$विजात्यनभिभूतैर्जलावयवैरारब्धं रसनम्$^11$~॥}

{\knu इन्द्रियमि}ति  \textendash\ यद् रसोपलब्धिसाधनमिन्द्रियं तदाप्यम्~। ${}^12$अत्रापि रसो \textendash\ पलब्ध्या स्वरूपसिद्धिः~। आप्यत्वं च रूपाद्यव्यञ्जकत्वे सति रसाभिव्यज्जकत्वात्, सक्तु \textendash\ रसाभिव्य$^13$ञ्जकोदकवत्~। प्राणिभेदेनेन्द्रियप्रकृतिवैचित्र्यनिवारणाय {\knu सर्वप्राणिना}मिति~।

\blfootnote{1 द्विविधाः  \textendash\ पा. २. पु~। 2 परिणामेन  \textendash\ पा. १. पु~। 3 'तत्र' \textendash\ दे पुस्तके नास्ति~। 4 'च' 'कि' पुस्तके नास्ति~। 5 कुत्रचित्  \textendash\ कि; जे~। 6 न्नेन्द्रिय \textendash\ कि, क; जे  \textendash\ अशुद्धमिदम्  \textendash\ न्निन्द्रिय इति शुद्धमर्थदृष्ट्या~। 7 भङ्गुरस्वभावत्वाद्  \textendash\ कि; जे~। 8 पष्टाम्भाच्चोपभोग \textendash\ पा. ३. पु~। 9 रसोपलम्भकं \textendash\ कि; जे; सर्वप्राणिनां रसोपलम्भकं  \textendash\ दे~। 10 अन्त्यावयवानभिभूतैः  \textendash\ कि; व्यो. (२४६) जे. दे~। 12 रसनेन्द्रियम्  \textendash\ दे~। 13 तत्रापि  \textendash\ कि; जे~। 14 व्यञ्जकदन्तोकवत्  \textendash\ पा. ३. पु~।}

\newpage
\noindent
अन्या{\knu वयवानभिभूतै}रिति \textendash\ वाताद्यभिभवे जडजिह्वत्वादिति भावः~। किं नामधेयं तद् ? इत्यत आह \textendash\ {\knu रसन}मिति~।

{\knu (४०) विषयस्तु सरित्समुद्रहि$^1$मकरकादिरिति~॥}

तृतीयां विधामाह \textendash\ {\knu विषयस्त्विति}~। आप्यत्वे हिमकरकयोः शिलीभावोपपत्तिं द्रवत्वे वक्ष्यतीति~॥

\hangindent=2cm {\knu (४१) तेजस्त्वाभिसम्बन्धात्तेजः~। रूपस्पर्शसङ्ख्यापरिमाणपृथक्त्वसंयोगविभागपरत्वापरत्वद्रवत्वसंस्कारवत्~॥}

तेजसो लक्षणमाह \textendash\ {\knu तेजस्त्वाभिसम्बन्धा}दिति~। व्याख्यानं पूर्ववत् ~। तस्य द्रव्यत्वव्यवस्थितये गुणानाह \textendash\ रूपस्पर्शेत्यादि~। 

{\knu (४२) पूर्ववदेषां सिद्धिः~॥}

अत्र प्रमाणं सूचयति \textendash\ {\knu पूर्ववदेषां सिद्धि}रिति~। 'तेजो रूपस्पर्शवद्' (वै. सू. २ \textendash\ १ \textendash\ ३) इति सूत्रकारवचनाद् रूपस्पर्शौ~। चाक्षुषवचनात् सप्त सङ्ख्यादयः~। अद्भिः सामान्यवचनाद् द्रवत्वम्~। उत्तरकर्मवचनात् संस्कार इति~। पृथिवीवदेते सिद्धाः, प्रसिद्धा इत्यर्थः~।

{\knu (४३) ${}^2$शुक्लं भास्वरं च रूपम्~। ${}^3$उष्ण एव स्पर्शः~॥}

तेजस्त्वव्यवस्थापकान् विशेषानाह \textendash\ {\knu शुक्लं भास्वरं च रूप}मिति~। चत्वर्थः पृथिव्यादिरूपाद् व्यवच्छिन्नति~। भास्वरत्वं च सामान्यविशेषः~। स च रूपान्तरप्रकाशकत्वेन व्यज्यते~। तद्रूपं तेजस एव; तेन भास्व$^4$ररूपं तेज इति लक्षणार्थः~। शुक्लमिति स्वरूप  \textendash\ कथनम्~। पच्यमानेन्धनसन्निधिकृतस्तु लोहिताद्युपलम्भः~। तद्विरहे चन्द्रतारकादिमहसि शुक्लस्यैवोपलम्भादिति प्रतिपादनार्थम्~। न च तत्राप्यबिन्धनतया तोयरूपसङ्क्रान्त्या धावल्यप्रतीतिरिति वाच्यम्; अपां रूपेण विजातीयानभिभवात्~। नहि जलप्लाविता नीलादयः श्वेतन्त इति~। {\knu उष्ण एव स्पर्श} इति \textendash\ एवकारश्चन्द्रचामीकरचक्षुरादिषु अनुपलम्भहेतुकां विप्रतिपत्तिमपनेतुम्~। तेषां तैजसत्वेनोष्णत्वानुमानात्~। अनुद्भूतत्वेनानु \textendash\ पलम्भः, न त्वभावात्~।

\blfootnote{1 करकादिः  \textendash\ दे~। 2 तत्र शुक्लं  \textendash\ कं~। 3 'नैमित्तिकं द्रवत्वं च' इत्यधिकं कि. पुस्तके व्योमवत्यां च (२५५), उष्ण स्पर्शः  \textendash\ दे~। 4 भास्वरं रूपं तैजस  \textendash\ जे~।}

\newpage
\hangindent=2cm {\knu (४४) तदपि द्विविधमणुकार्यभावात्~। ${}^1$कार्यं च शरीरादित्रयम्~।}

{\knu तदपि द्विविध}मिति~। पूर्ववत्तात्पर्यम्~। कार्यभेदमाह \textendash\ {\knu शरीरादित्रय}मिति शरीरमादिर्यस्य त्रयस्य तत्तथा~।

\hangindent=2cm {\knu (४५) शरीरमयोनिजमादित्यलोके, पार्थिवावयवोपष्टम्भाच्चोपभोगसमर्थम्~॥}

न पार्थिववत्तैजसं शरीरमुपलभ्यत इत्यत आह \textendash\ {\knu शरीर}मिति~। तेजसोऽपि शरीरारम्भकत्वं तोयवत् साधनीयम्~। ज्वलदनलकल्पेन कस्तेनोपभोग इत्यत आह \textendash\ पार्थिवेति~।

\hangindent=2cm {\knu (४६) इन्द्रियं सर्वप्राणिनां रूपव्यञ्जकमन्यावयवानभिभूतैस्तेजोऽवयवैराब्धं चक्षुः~॥}

\begin{sloppypar}
इन्द्रियमाह \textendash\ {\knu इन्द्रिय}मिति यद्गन्धाद्यव्यञ्जकत्वे सति रूपस्य व्य्जकमिन्द्रियं तत् तैजसम्~। {\knu सर्वप्राणिनामिति} \textendash\ प्राणिभेदेनेन्द्रियप्रकृतेर्वैचित्र्य शङ्कानिवारणार्थम् ~। तत्सद्भावेरूपलब्धिः प्रमाणम्~। तैजसत्वे च ${}^2$रसाद्यव्यञ्जकत्वे सति ${}^3$रूपाभिव्यञ्जकत्वात्, प्रदीपवत्~।
\end{sloppypar}

इह केचिदाहुः, अप्राप्यकारि चक्षुः, अधिष्ठानासम्बद्धार्थ$^4$ग्राहित्वात्~। यत्पुनः प्राप्यकारि, न तदधिष्ठानासम्बद्धार्थग्राहि यथा रसनादि~। पृथुतरग्रहणाच्च ~। यदि च प्राप्यकारि, चक्षुः स्यात् स्वतोऽधिकपरिमाणं न गृह्णीयात्~। न खलु ${}^5$नखरञ्जनिका परशुच्छेद्यं छिनत्ति~। शा\renewcommand{\thefootnote}{१}\footnote{शाखेति \textendash\ गतिक्रमे प्राप्यग्रहणे हि क्रमिकं ग्रहणं स्यात्तथावाऽनुभवविरोध इत्यर्थः~। कि. भा. प~।\\ \rule{0.4\linewidth}{0.5pt}}खाचन्द्रमसोस्तुल्यकालग्रहणाच्च~। यदि हि गत्वा गृह्णीयात् निकटस्थमाशु प्राप्नुयात् द$^6$वीयस्तु चिरेणेति~। न तुल्यकालमुपलम्भयेत्~। अनुभवती तून्मीलन्नेव नयने शाखां शीतमयूखं चेति~। काचाभ्रपटलस्फटिकाद्यन्तरितोपलब्धेश् ~। यदि हि प्राप्य गृह्णीयात् प्रतिघातिना स्फटिकादिद्रव्येण विष्टम्भादप्राप्तं प्रस$^7$र्पत्तृणादिकं नाददीत~। तस्मादप्राप्यकारि; ततो न तैजसमिति~।

\blfootnote{1 कार्यं शरीरेन्द्रियविषयषञ्ज्ञकम् इत्यधिकं कि. पुस्तके; शरीरेन्द्रियविषयसञ्ज्ञकम्  \textendash\ दे~। 2 रसाद्य  \textendash\ व्यञ्जकत्वे  \textendash\ कि~। 3 रूपाभिव्यञ्जकम्  \textendash\ कि; जे~। 4 ग्राहकत्वात्  \textendash\ कि; जे~। 5 नखरः  \textendash\ क~। 6 दवीयांस्तु  \textendash\ पा. ३. पु~। 7 पसार्यतूणादिकं  \textendash\ पा, ३. पु~।}

\newpage
तदसत्; अधिष्ठानासम्बद्धार्थग्राहित्वस्य प्रदीपेनानैकान्तिकत्वात्~। पृथुतरग्रहणस्यापि पृथ्वग्रतया तद्वदेवोपपत्तेः~। तुल्यकालग्रहणं त्वसिद्धमेव~। तदभिमानस्य कालसन्निकर्षेणोपपत्तेः~। अचिन्त्यो हि तेजसो लाघवातिशयेन वेगातिशयो यत्प्राचीनाचल$^1$चूडावलम्बिन्येव भगवति मयूखमालिनि 'भवनोदरेषु आलोक' इत्यभिमानो लौकिकानाम्~।

\renewcommand{\thefootnote}{१}\footnote{चक्षुर्बाह्यालोकाभ्यामारब्धेन चक्षुषा तावदर्थसंसृष्टेन युगपत् तावदर्थग्रहणमिति केषाञ्चिन्मतमाह \textendash\ केचित्त्विति~। कि भा. प.~।}केचित्तु संसर्गिद्रव्यतया निःसरदेव नायनं तेजो बाह्यालोकेनैकतां गतं युगपदेव तावदर्थेन संसृष्टमिन्द्रियमुत्पादितवदिति शाखाचन्द्रमसोस्तुल्यकालग्रहणमुपपद्यत एवेति ${}^2$समाधानमाहुः~। तदसत्; पृष्ठभाग$^3$व्यवहितार्थोपलम्भप्रसङ्गात्~। इन्द्रियप्राप्तये ह्यार्जवस्थानमुपयुज्यत इति~। स्फटिकाद्यन्तरितोपलब्धिरपि प्रसादस्वभावतया स्फटिकादीनां तेजोगतेरप्रतिबन्धकतया प्रदीपप्रभावादेवोप$^4$पन्नेति~।

येषां त्वप्राप्यकारि चक्षुस्तेषामप्राप्तत्वाविशेषादव्यवहितवत्~। कुड्यादिव्यवहितमपि किं न गृह्णीयात् ? न हि व्यवधायकं प्राप्तिविघातादन्यत्र सत्तयैव कार्यविरोधि; तथात्वे तस्मिन् सति न क्वचित्कार्यं भवेत्, योग्यं योग्येन गृह्येत~। तत्त्वयोग्यत्वान्न गृह्यते, न तु व्यवहितत्वाद्, इत्यपि वार्तम्, स्थैर्ये स्वरूपयोग्यतास्तदानीमप्यपरावृत्तेः~। क्षणिकत्वेऽपि प्रत्यासीदतां सहकारिणामतिशयजनकत्वात्~। नह्यप्रत्यासीदन्तः सहकारिणि प्रत्ययाः समाः सहस्रेणापि भावानतिशाययन्ति~। प्रत्यासत्तिश्च बौद्धनये निरन्तरोत्पादः~। अस्माकं च द्रव्ययोः संयोग एव$^5$ स्यात्~। न च कृष्णसारस्यार्थेन निरन्तरोत्पादः~। नापि संयोगः; ततस्तदाश्रयस्यातीन्द्रियस्य गतिक्रमेण निरन्तरोत्पादेन वा संयोगेन वेत्यवशिष्यते तदेतदग्रे ${}^6$निरूपयिष्यत इति दिक्~।

कथमेतदेव तथा नान्यदूष्मादीत्यत आह \textendash\ {\knu अन्येति}~। अन्येत्युपलक्षणम्; समानजातीयेन बलवता दुर्बलमभिभूयते~। कथमन्यथा सैरेण तेजसा नयनमभिभूतं रूपग्रहणासमर्थं भवति~। चक्षुरिति \textendash\ लोकप्रसिद्धमित्यर्थः~।

\hangindent=2cm {\knu (४७) विषयसञ्ज्ञकं चतुर्विधम् भौमं दिव्यमुदर्यमाकरजं च~। तत्र भौमं काष्ठेन्धनप्रभवमूर्ध्वज्वलनस्वभावं पचनदहन$^7$स्वेदनादिसरमर्थं च दिव्यमबिन्धनं ${}^8$सौरविद्यु\textendash}

\blfootnote{1 चूला  \textendash\ जे~। 2. समादधुः  \textendash\ कि. जे~। 3 व्यवस्थितार्थोपलम्भ  \textendash\ पा. ३ पु~। 4 वोपपत्तिरिति पा. १. पु; उपपत्तेरिति  \textendash\ कं~। 5 इति स्यात् जे~। 6 निराकरिव्यतीत्येषा दिक्  \textendash\ पा. ३. पु~। 7 पचनदहनस्वेदनादि  \textendash\ कं~। 8 सौरं  \textendash\ दे~।}

\newpage
\indent
\hangindent=2cm {\knu दादि~। भुक्तस्याहारस्य र$^1$सादिपरिणामार्थमुदर्यम्~। आकरजं सुवर्णादि~। तत्र संयुक्तसमवायाद्रसाद्युपलब्धिरिति~॥}

विषयविभागमाह \textendash\ {\knu विषयसञ्ज्ञ}मिति~। विषय इति सञ्ज्ञा यस्य, तच्चतुर्विधम्~। भौमं भूमौ भवम्~। दिव्यं दिवि भवम्~। उदर्यमुदरे भवम्~। आकरजं खनिराकरस्तस्माज्जातम्~। काष्ठेनोद्भिद्विकारान् पार्थिवानुपलक्ष्यति, तदेवेन्धनं दीपनं यस्य तत्तथोक्तम्~। ${}^2${\knu ऊर्ध्वज्वलनस्वभावकम्} \textendash\ ऊर्ध्वज्वलनमेव स्वभावो यस्य तत्तथा~। जातिनिब$^3$न्धनो धर्मः स्वभाव इत्युच्यते~। परार्थतामाह \textendash\ {\knu पचनेति}~। पाको विक्लेदनं रूपपरावृत्तिर्वा~। स्वेदनं वातश्लेष्मणोरपगमः~। 'आदि' ग्रहणाच्छीतदुःखोपशभो दाहदुःखोत्पादनं च~। {\knu दिव्यमबिन्धनं} \textendash\ आप एवेन्धनं यस्यं तत्~। किं ${}^4$तदित्यत आह \textendash\ सौरम्~। सूरस्येदं सौरम्~। विद्युदचिरप्रभा~। आदिग्रहणीद्वडवानल इत्येतदपि दिव्यम्~। {\knu भुक्तस्याहारस्येति} \textendash\ आह्रियत इत्याहार ओदनादिः, तस्य रसादिपरिणामनिमित्तमुदर्यम्~। भुक्त एवाहारोऽस्य पार्थिवाप्यरूपमिन्धनं कार्यो रसः~। आदिग्रहणान्मलधातुस्तद्विकारास्तएव परिणामास्तेषां निमित्तम्~। {\knu आकरजं सुवर्णादि} \textendash\ आदिग्रहणाद् रजतताम्रकांस्याद्यष्टकम्~।

\begin{sloppypar}
कथं पुनरेतत् तैजसम् ? यद्यपि गुरूत्वादयः संयुक्तसमवायेनाप्युपपद्यन्ते तथापि नैमित्तिकद्रवत्वादेव पार्थिवत्वमस्य सिद्ध्यति घृतादिवत्~। न च घृतादिद्रवत्ववदेवात्यन्ताग्निसंयोगादेत$^5$द्द्रवत्वं सुवर्णस्य क्षीयेत तथा ${}^6$तददर्शनात्~। विशेषविरोधस्य चादूषणत्वात्~। तथा च वक्ष्यामः~। न चाक्षीयमाणेन द्रवत्वेन सत्प्रतिपक्षत्वम्, असाधारणत्वात्~। किं च योपि तैजसमिच्छति सुवर्णं तेनाऽप्यत्र पार्थिवो भाग उपष्टम्भक एष्टव्यः~। न चाऽसौ द्र$^7$वत्वदशायामद्रव एवास्ते~। तथा चाक्षीयमाणद्रवत्वं तत्राप्यस्तीति विपक्षैकदेशवृत्तित्वाद्विरुद्धो हेतुः~। तस्मात् परिभावनीयमेतत्सुरिभिः~।
\end{sloppypar}

उच्यते~। रूपवत्तया तावदेतत्त्रितयान्तर्भूतमिति स्थितम्~। तत्र सांसिद्धिकद्रवत्वशैत्ययोरभावात् स्नेहाविनाभूतद्रव्यान्तरसङ्ग्रहानुपपत्तेश्च न जलम्~। न च नैमित्तिकद्रवत्व\textendash

\blfootnote{I रसादिभावेन परिणामसमर्थम् \textendash\ कि; रसादिपरिणामकारणमौरदर्य \textendash\ जे~। 2 'ऊर्ध्वज्वलनस्वभावकम्' इत्यन्येषु पुस्तकेषु नास्ति केवलं 'जे' पुस्तके एव \textendash\ तथाप्यावश्यकमिति स्थापितम्~। 3 निबद्धो \textendash\ पा. ३; १. पु~। 4 तदत आह \textendash\ पा. ३. पु~। 5 द्रवत्वं क्षीयेत \textendash\ कि~। 6 तथा प्रदर्शनात् \textendash\ कि. क~। 7 तद्द्रवदशाया \textendash\ पा. ३. पु~।}

\newpage
\noindent
वत्त्वात् पार्थिवम्, विपक्षे बाधकाभावात्~। नहि निमित्ताद् द्रव्यान्तरे द्र$^1$वत्वोपपत्तिं किञ्चिद्विरूणद्धि न चोर्ध्वज्वलनतिर्यक्पवनवज्जातिनियतमेतत्, एकदेशवृत्तित्वात्~। 'न च तद्विशेष'नियतं सर्पिरादीनां भिन्नजातीयत्वात्~। तस्मात्तत्तद्भोगहेतुनियतादृष्टसहकार्यग्निसंयोगनिबन्धनं द्रवत्वमुत्पद्यमानं कम्पवन्न पार्थिवत्वादिव्यवस्थाहेतुः ${}^2$किन्तु पार्थिवं रूपमग्निसंयोगादसति विरोध्यन्तरे परावर्तते, सति तु नेति नियमः~। दह्यमानेषु घटादिषु दर्शनेन जलसम्बन्धाच्च तेष्वेवादर्शनेन कार्यकारणभातव्यवस्थितेः~। तदिदं सुवर्णादि निरन्तरं ध्मायमानपि न पूर्वरूपं जहाति; न च रूपान्तरमापद्यते~। तेन जलपार्थिवत्वमवधार्यते~। तेनैव च द्रव्यान्तरेण प्रतिबद्धत्वादुपष्टम्भकोऽपि पार्थिवभागः सदृशरूपे एवानुवर्तते~। यत्तु पुटपाकादिना रक्तसारता दृश्यते तन्मिश्रिभूताभिभावकतत्तद्द्रव्योपगमादिति युक्तमुत्पश्यामः~।

तदयं प्रयोगः \textendash\ 'सुवर्णादि न पार्थिवम्; अत्यन्तानलसंयोगेऽप्यपरावर्त्तमानरूपवत्त्वात् जलवत्~। तैजसं चैतत् पार्थिवाप्याभ्यामन्यत्वे सति रूपवत्त्वात्; सूर्यालोकवत्~। $\rightarrow$ ए$^3$तेन पारदादि व्याख्यातम् $\leftarrow$~। आप्यादन्यत्वस्य द्रव्यान्तरसङ्ग्रहभावादेव सिद्धेः~। सांसिद्धिकद्रवत्वस्याग्निसंयोगाज्जायमानत्वेन बाधितत्वात्~। अबिन्धनतेजोवद् वैधर्म्योपपत्तिः~। तैजसत्त्वे आगमसत्त्वादिति~।

नन्वभास्वररूपत्वान्न तैजसमित्यपि स्यात्~। न स्यात्, असिद्धेः~। न हि मध्याह्ने प्रदीपो बहिर्भास्वरो नोपलभ्यत इति न भास्वरः~। उपष्टम्भकद्रव्यरूपाभिभवेनैवानुपलम्भोपपत्तेः~। अथ नयनरश्मिवदनुद्भवेनैवानुपलम्भः किं न स्यात्, न स्यात्, तद्वत्सुवर्णादिद्रव्यस्याप्यनुपलम्भप्रसङ्गात्~। नह्यनुद्भूतरूपं द्रव्यमुपलभ्यते~। ग्रीष्मोष्मणोऽपि निशि दर्शनप्रसङ्गात्~। अभिभवेऽपि मध्यन्दिनोल्काप्रकाशानुपलम्भवदनुपलम्भप्रसङ्गस्तुल्य इति चेत्; न; नहि रूपस्याभिभवेन द्रव्यं नोपलभ्यते~। महारजनरूपाभिभूते धवलिम्नि पटानुपलम्भप्रसङ्गात्~। तस्मादभिभूतत्वाद्रूप$^4$मेव तस्य न दृश्यते~। स तु माषराशिप्रविष्टमशीवन्महाप्रकाशसमाहारान्नेक्ष्यत इति सर्वमवदातम्~। कथं तर्हि सुवर्णादौ रसाद्युपलब्धिरित्यत आह \textendash\ तत्रेति~॥

\begin{sloppypar}
\hangindent=2cm {\knu (४८) वायुत्वाभिसम्बन्धाद् वायुः~। स्प$^5$ र्शसङ्कख्यापरिमाणपृथक्त्वसंयोगविभागपरत्वापरत्व$^6$ संस्कारवान्~॥}
\end{sloppypar}

\blfootnote{I द्रवत्वोत्पत्तिः \textendash\ जे~। 2 किं च कि. क~। 3 $\rightarrow$ $\leftarrow$ एतच्चिह्नान्तर्गतः पाठः जे पुस्तके नास्ति~। 4 मेव न दृश्यते किं~। 5 तस्य गुणाः स्पर्श \textendash\ जे~। 6 संस्कारस्तद्वान् \textendash\ जे~।}

\newpage
${}^1$प्रतिनियतेन्द्रियग्राह्यगुणाश्रयतया पृथिव्यादीनां सुप्रसिद्धत्वाद् गन्धादिह्वासक्रमेण च व्याख्यानात् तेजोऽनन्तरं वायुलक्षणमाह \textendash\ {\knu वायुत्वाभिस$^2$म्बन्धादिति}~। लक्षणार्थः पूर्ववत्~। तस्य द्रव्यत्वसिद्धयर्थं गुणानाह \textendash\ {\knu स्पर्शेति}~।

\hangindent=2cm {\knu (४९)स्पर्शोऽस्याऽनुष्णाशीतत्वे सत्यपाकजः, गुणविनिवेशात् सि$^3$द्धः~। अरूपिष्वाचाक्षुवचनात् सप्त सङ्ख्यादयः~। तृणकर्मवचनात् संस्कारः~॥}

एवं गुणवत्तया द्रव्यत्वे सिद्धे वायुत्वव्य४वस्थाहेतुविशेषमाह \textendash\ {\knu स्पर्शोऽस्ये}ति~। अस्य वायोरनुष्णः स्पर्श इति तेजोव्यावृत्तिः, अशीत इति अद्भ्यः, अपाकज इति पृथिव्याः~। अ$^5$त्राऽनुष्णाशीतत्वमस्य प्ररत्यक्षसिद्धम्~। अपाकजत्वं त्वपार्थिवस्पर्शत्वात्तोयस्पर्शवत्~। पाकजत्वं हि विशेषान्तरव्याप्तमुपलब्धम्~। अन्यथा सर्वगुणानां पाकजत्वप्रसङ्गात्~। तदितो व्यावर्तमानं स्वव्याप्यमुपादाय तद्विरहिणि पार्थिवे विश्राम्यतीति प्रतिबन्धसिद्धिः~। सोऽस्ति वायावित्यत्र प्रमाणं सूचयति \textendash\ {\knu गुणविनिवेशेति}~। तथा च सूत्रम् \textendash\ '$^6$वायुः स्पर्शवान्' (वै. रू. २ \textendash\ १ \textendash\ ४) इति सङ्ख्यादिषु सूत्रकारसम्मतिमाह \textendash\ {\knu अरूपिष्वि}ति~। रूपरहितेषु र्क्तमानानामेषामचाक्षुषत्वं वदता सूत्रकारेण रूपरहिते वायौ सद्भावः स्वीकृतः~। प्रमाणं तु सङ्ख्यापरिमाणपृथक्त्वसंयोगविभागवत्तायां वायोर्द्रव्या$^7$रम्भकत्वम्; सङ्ख्यापरिमाणपृथक्त्वसंयोगैर्विना द्रव्यारम्भानुपपत्तेः~। विभागेत विना संयोगविनाशानुपपत्तौ आरब्धद्रव्यस्य विनाशानुपपत्तेः~। परत्वापरत्वाभ्यां विना ${}^8$संयुक्तसंयोगाल्पीयस्त्वभूयस्त्वासम्भवे मूर्त्तत्वानुपपत्तेरिति~। संस्कारे सू$^9$त्रकारानुमतिमाह तृणेति~। 'तृणे कर्म वायुसंयोगात्' (वै. सूः ५ \textendash\ १ \textendash\ १४) इति वचनादित्यर्थः~। तथा च प्रयोगः 'एतद्रूप्वद्द्रव्याभिघातमन्तरेण तृणे स्पर्शवद्वेगवद्द्रव्याभिघातजम्, विशिष्टकर्मत्वात्; नदीपुराभिहतकाशादिवन$^10$कर्मवद्' इति~। तदेतत् कम्पलिङ्ग इत्यत्र स्फुटीभविष्यति~। एतेन 'अरूपस्पर्शवान् वायुः' इति लक्षणमूहेत्~। तथा च सति 'अशीतानुष्णापाकजस्पर्शवान् वायुः' इति सिद्ध्यति~। 

{\knu (५०) स चाऽयं द्विविधोऽणुकार्यभावात्~।}

\blfootnote{I प्रतिनियतेन्द्रियग्राह्य \textendash\ पा. १. पु~। 2 सम्बन्धाद्वायुः \textendash\ जे~। 3 सिद्ध्येः \textendash\ कि; जे~। 4 व्यवस्थाविशेषमाह \textendash\ कि; जे~। 5 तत्रा$^\circ$\textendash\ पा. ३. पु~। 6 'स्पर्शवान् वायुः' इति सूत्रे पाठः~। 7 द्रव्यत्वमेव \textendash\ पा. १. पु~। 8 विना संयोगाष्मीयस्त्वम् \textendash\ मु. कि; क~। 9 सूत्रकारसम्मति \textendash\ पा. ३. पु : I0 कम्यवदिति \textendash\ कि; क~।}

\newpage
सोऽयं यद्यनित्य एव, अवयवानवस्था स्यात्~। कस्यचिन्नाशेऽनाश्रयं च कार्यमापद्येत~। अथ नित्य एव तर्हि नि$^1$ष्प्रामाणिकः स्यात्, परमाणुस्पर्शस्यातीन्द्रियत्वात्~। स्थूलानां शब्दधृतिकम्पानां च परमसूक्ष्मद्रव्याभिघाताद्यहेतुकत्वादित्यत आह \textendash\ {\knu स चायमिति}~। द्विविधो नित्यश्चानित्यश्चेति~। यथासङ्ख्यमणुकार्यभावादिति~। 

\hangindent=2cm {\knu (५१) तत्र कार्यलक्षणश्चतुर्विधः \textendash\ शरीरमिन्द्रियं विषयः प्राण इति~॥}

तत्र तयोरणुकार्ययोर्मध्ये कार्यलक्षणश्चतुर्विधः~। चतस्रो विधामाह \textendash\ {\knu शरीरमि}ति~। 

\hangindent=2cm {\knu (५२) तत्रायोनिजमेव शरीरं मरुतां लोके पार्थिवावयवोपष्टम्भाच्चोपभोगसमर्थम्

तत्र तेषु मध्ये शरीरमयोनिजमेव न तु पार्थिववद्योनिजमपि~। वायोर्द्रव्यारम्भकत्वेन तेजोवच्छरीराम्भकत्वे स्थिते क्व तदित्यपेक्षायामाह \textendash\ {\knu मरुतां लोक} इति~। ननु तस्य तात्वाद्यभावे वाग्व्यापाराभावात्, करचरणाद्यभावे ${}^2$चाहारविहरणाद्यभावात्, संस्थानविशेषाभावे चेन्द्रियाश्रयत्वानुपपत्तिः, कथं ${}^3$भोगायतनमित्यत आह \textendash\ पार्थिवेति~। उपलक्षणं चैतद्, भूतान्तरानुग्रहस्यापि सम्भवात्~। 

\hangindent=2cm {\knu (५३) इन्द्रियं सर्वप्राणिनां स्पर्शोपलम्भकं पृथिव्याद्यनभिभूतैर्वाय्ववयवैरारब्धं ${}^4$शरीरव्यापि त्वगिन्द्रियम्~॥}

{\knu इन्द्रियमिति} \textendash\ रूपादिषु मध्ये स्पर्शस्यैवोपलम्भकं यदिन्द्रियं तद्वायवीयम्~। {\knu सर्वप्राणिनामिति} \textendash\ पूर्ववदत्रापि स्पर्शोपलब्द्ध्या करणमात्रे सिद्धे, संयोगाश्रयतया च द्रव्यत्वे, रूपादिषु मध्ये स्पर्शस्यैवाभिव्यञ्जकत्वाद्वायवीयत्वम्~। अङ्गसङ्गिसलिलशैत्याभिव्यञ्जकव्यजनपवनवद् इत्यूहनीयम्~। कस्यचिदेव वायोरिन्द्रियारम्भकत्वं न सर्वस्येति नियमे हेतुमाह\extendash\ {\knu पृथिव्यादीति~।} अत एव कु$^5$ष्ठाद्यपहतायां त्वचि न स्पर्शोपलम्भ इत्यभिसन्धिः~। तस्याधिष्ठानमाह \textendash\ {\knu सर्वशरीरव्यापीति}~। सर्वशरीरव्याप्तस्य स्पर्शोपलम्भादि$^6$त्यवधातव्यम्~। तदवसानं शरीरं नखलोमकेशदशनानां संयोगिद्रव्यमात्रत्वं ${}^7$त्वगिन्द्रियानाश्रयत्वादित्यभिप्रायः~। लोक$^8$प्रसिद्धां तस्य सञ्ज्ञामाह \textendash\ {\knu त्वगिति}~। ननु वृद्धिक्षय\textendash

\blfootnote{I निष्प्रमाणकः \textendash\ जे~। 2 चाहरण \textendash\ पा. ३. पु.~। 3 भोगायतनत्व \textendash\ पा, ३. पु~। 4 सर्वव्यापि \textendash\ दे~। 5 कुष्ठापहतायां \textendash\ कि; क~। कुष्ठापहृतायां \textendash\ पा. ३. पु~। 6 त्यवदातव्यम्  \textendash\ पा. ३. पु~। 7 अतीन्द्रियानाश्रयत्त्वाद् \textendash\ जे; संयोगित्वमात्रमिन्द्रियानाश्रयत्वाद् \textendash\ पा. ३. पु~। 8 लोकप्रसिद्धां सञ्ज्ञामाह \textendash\ कि, क~।}

\newpage
\noindent
वद्देह$^1$सहजावरणमात्रं त्वगित्युच्यते~। तच्च पार्थिवं प्रत्यक्षसिद्धं नेन्द्रियं न च वायवीयमित्यत आह \textendash\ {\knu इन्द्रियमिति}~। न त्वगेव मुख्यतो वायवीयमिन्द्रियं त्वगित्युच्यते किन्तु लक्षणया तत्रस्थमिन्द्रियं त्वगिति व्यवद्वियते लौकैरित्यर्थः~।

\hangindent=2cm {\knu (५४) विषयस्तूपलभ्यमानस्पर्शाधिष्ठानभूतः स्पर्शशब्दधृतिकम्पलिङ्गस्तिर्यग्गमनस्वभावो मेघादिप्रेरणधारणादिसमर्थः~॥}

{\knu विषयस्त्विति} \textendash\ 'तु' शब्देनो$^2$पलभ्यमानात् पृथिव्यादित्रयाद् व्यावर्तयति~। उ$^3$पलभ्यमानस्य स्पर्शस्याधिष्ठानभूतो न तूपलभ्यते इत्यर्थः~। नन्वनुपपन्नमेतत्, स्पार्शनप्रत्यक्षत्वाद् वायोः त्वगिन्द्रियव्यापारानन्तरं वायुर्वातीति प्रतीतेः~। न च स्पर्शमात्रमेव तत्र प्रतीयते वायुस्त्वनुमीयत इति युक्तम्; शीतो वायुरुष्णो वायुरिति तदीयस्पर्शाभिभवेऽपि वायुप्रत्ययात्~। न च शीतोष्णद्रव्योपनायकत्वेनानुमितस्य पवनस्य तथा प्रतीतिरिति युक्तम्; बाधकाभावात्~। ${}^4$अस्त्यचाक्षुषत्वं बाधकम्, द्रव्यस्य हि स्पार्शनत्वं चाक्षुषत्वेन व्याप्तम्~। तदितो व्यावर्तमानम् स्पार्शनत्वमपि व्यावर्तयतीति चेत्; न; विपक्षे बाधकाभावात्~। न हि चक्षुरूपाभ्यां त्वचः स्पर्शस्य च किञ्चिदुपकर्तव्यम्, येन मिथोऽपेक्षा स्यात्~। तथात्वे वा जात्यन्धेन घटादयोऽपि त्वचा नोपलभ्येरन्~। रू$^5$पाभावेन द्रव्यवत् स्पर्शोऽपि वा नोपलभ्येत~। 'उष्णं जलम्' शीतलं शिलातलमित्यत उष्णशीतस्पर्शाधिष्ठानयोस्तेजः पयसोश्चक्षुषानुपलम्भेऽपि स्पर्शम्योपलब्धेः~। द्रव्यस्य चक्षुषाऽनुपलभ्यमानस्य स्पर्शनेन क्वचिदनुपलम्भान्नैवमिति चेत्; न; अपार्थिवे सर्वत्र तर्हि त्वचा स्प$^7$र्शमात्रोपलब्धिरस्तु, द्रव्योपलब्धिस्तु चक्षुषैव क्वचिदनुमानेन वा भविष्यति~। वायुवत् प्रत्यभिज्ञानमपि तद्वदेवोपपत्स्यते~। अथ भ्रान्तौ यथा तथाऽस्तु; प्रमाणचिन्तायां तु' उपलभ्यमानः स्पर्शः आश्रयसहित एवोपलभ्यतेः अनुभवस्य दुरपह्नवत्त्वात्~। अतस्तोयतेजसोः त्वचोपलम्भ इष्यते; गुणयोग्यतामाश्रित्य त्वक्चक्षुषोर्द्रव्यावगाहनादिति; यदि वायावपि समानमेतत्~। तथा च प्रयोगः; वायुः प्रत्यक्षः, उपलभ्यमानस्पर्शाधिष्ठानत्वाद् घटवदिति~।

अत्रोच्यते~। द्रव्यग्रहणयोग्यतान्तर्भूतयोग्यताकानि खलु सङ्ख्यापरिमाणपृथक्त्वसंयोगविभागपरत्वापरत्वानि कर्माणि एव ह्यमीषां द्वीन्द्रियग्राह्यता ${}^8$निर्वहति~। तद्यदि त्वचा

\blfootnote{I द्रव्य \textendash\ जे~। 2 उपलभ्यत्वात् \textendash\ कि; जे~। 3 उपलभ्यमानस्पर्शाधिष्ठानभूतो \textendash\ कि. क~। 4 अस्ति चाक्षुषत्वं \textendash\ जे~। 5 रूपापायेन \textendash\ पा. ३. पु.~। 6 सर्वत्रत्वचा \textendash\ कि. क~। 7 स्पर्शोपलब्धिरस्तु \textendash\ कि; जे,~। 8 भवति \textendash\ कि; जे~।}

\newpage
\noindent
वायुरुपलभ्येत; सङ्ख्यादयोप्युपलभ्येरन् न चैतद् द्रव्यमपि नोपलभ्येत, तुल्योपलम्भनयोग्यत्वात्~। न च त्वचा वायोरेकत्वादिका सङ्ख्या वा, हस्तस्तवितस्त्यादि परिमाणं वा, पृथिव्यादिभ्यः सजातीयतो वा, पृथक्तवं संयोगविभागौ वा, परत्वापरत्वे च कर्म चोपलभ्यते~। न तुल्योपलम्भनयोग्यतैवासिद्धा, दूरात् सङ्ख्यादीनामनुपलम्भेऽपि शा$^1$लताल \textendash\ तमालादीनामुपलम्भादिति चेत्; न; दूरादिदोषेण सङ्ख्यादयो नोपल$^2$म्भिषत, तथापि योग्यताया अनिवृत्तेर्दोषनिवृत्तावु$^3$भलम्भात्~। तस्माद् द्रव्योपलम्भसहचरोपलम्भयोग्य$^4$ताकानां सङ्ख्यादीनामनुपलम्भादवगच्छामो न स्पार्शनः पवन इति~। तदेतत्तस्याप्रत्यक्षस्यापि नानात्वं संमूर्च्छनेनानुमीयत इत्यत्र स्फुटीभविष्यति~। अत एवानुपलभ्यमानसङ्ख्यादिके तोयतेजसी अपि त्वचा नोपलभ्येते इति विनिश्चयः~। तथा चोपलभ्यमानस्पर्शधिष्ठानत्वमप्यनैकान्तिकमिति सिद्धम्~।

कुतस्तर्हि प्रत्येतव्य इत्यत आह \textendash\ {\knu स्पर्शेति}~। स्पर्शः, शब्दो, धृतिः, कम्पश्च लिङ्गानि यस्य स तथोक्तः~। तथाहि योऽयं रूपवद्द्रव्यासमवेतः स्पर्शः, स क्वचिदाश्रितः, स्पर्शत्वात्, पृथिवीस्पर्शवत्~। न चाऽयं ज$^5$लानलयोः, अनुष्णाशीतत्वात्~। न च पार्थिवः, रूपेणासमानाधिकरणत्वात्~। यस्तु पार्थिवो नासावेवं यथा घटादिस्पर्शः~। न च रूपेणासमानाधिकरणत्वमसिद्धम्, योग्यानुपलब्धेः~। न चायोग्यरूपमिह भविष्यतीति ${}^6$वाच्यम् अभिभावकरूपान्तरस्यानुपलब्धेरभिभवानुपपत्तेः~। पाकजत्वेन पार्थिवरूपस्पर्शयोः तुल्ययोगक्षेमत्वेन स्पर्शस्योद्भवे रूपस्यानुद्भवायोगात्~। अन्यथा पृथिव्यां रूपोद्भवेऽपि स्पर्शनुद्भवप्रसङ्गात्~। किञ्च अनुपलभ्यमानरूपस्य स्पर्शाधिष्ठानस्य पृथिवीत्वे गुरुत्वमपि स्यात्~। न च पवनापूरितस्य चर्मपुटकादेरपूरणदशातोधिकमवनमनमुन्नीयमानस्य धूमापूरितेनेदमनैकान्तिकमिति चेत्; न; अलं धूमदशायां तत्रापि गरिमविशेषहेतोरवनतिविशेषस्य प्रतीतेः~। तस्माद्विवादाध्यासितं द्रव्यं न पृथिवी, रूपरहितत्वात्, आकाशवत्; गुरुत्वरहितत्वात् तेजोवद्; इति स्थिते$^7$ स्पर्शाश्रयतया चतुर्थ द्रव्यं सिद्ध्यति इति~। एवमसति रूपवद्द्रव्यभिघाते योऽयं पर्णादिषु शब्दसन्तानः, स्पर्शवद्वेगवदूद्रा$^10$व्यसंयोगप्रभवः, अविभज्यमानावयवद्रव्यसम्बन्धित्वे सति शब्दसन्तानत्वात्,
दण्डाभिहतभेरीशब्दसन्तानवदिति~। 

\blfootnote{I चलत् शाल$^\circ$\textendash\ क~। 2 नोप \textendash\ कि; जे~। 3 वुपलभ्यते \textendash\ पा. ३. पु~। 4 योग्यतानां \textendash\ जे~। 5 जलाद्याश्रयः \textendash\ कि; जे~। 6 युक्तम् \textendash\ कि; जे~। 7 सिद्धे \textendash\ पो. ३. पु~। 8 इति \textendash\ कि; पुस्तके नास्ति~। 9 मसति द्रव्याभिघाते \textendash\ क~। 10 द्रव्याभिघातजः \textendash\ पा. ३. पु; वेगवत्संयोगजन्यः \textendash\ क~।\\ ८}

\newpage
\noindent
एवं गुरुणोद्रव्यस्यायतनं धृतिः~। ततोऽप्यनुमीयते वायुः~। तथा च1 प्रयोगः \textendash\ 'नभसि तृणतूलास्तनयित्नुविमानादीनां धृतिः स्पर्शवद्द्रव्यसंयोगहेतुका, अस्मदाद्यनधिष्ठितद्रव्यधृतित्वात्, नौकादिधृतिवत्~। कम्पस्य लिङ्गत्वमुक्तम्~। ${}^1$एवं चतुर्णामपि कार्यत्वान्निस्तरङ्गः प्रतिबन्धः~। 

{\knu तिर्यग्गमनस्वभावकः} \textendash\ तिर्यग्गमनं स्वभावो यस्य स तथोक्तः~। अदृष्टवदात्मसंयोगनिबन्धनमस्य तिर्यग्गमनम्~। तेन कः पुरुषार्थ इत्यत आह \textendash\ {\knu मेघादीति~।} आदिग्रहणाद्वैहायसानां विमानादीनां भौमादीनां च यानपात्रपासुपटलादीनां जलानलयोश्च परिग्रहः~। धारणादीत्यादिशब्दात् स्पर्शाभिव्य$^2$क्तिदीपनिर्वापणादि परिग्रहः, तेषु समर्थः~।

\begin{sloppypar}
\hangindent=2cm {\knu (५५) तस्याप्रत्यक्षस्यापि नानात्वं सम्मूर्च्छनेनानुमीयते~। सम्मूर्च्छनं पुनः ${}^3$समानजवयोर्वाय्वोर्विरुद्धढिक्क्रिययोः सन्निपातः~। सोऽपि ${}^4$सावयविनोर्वाय्वोरूर्ध्वगमनेनानुमीयते; तदपि तृणादिगमनेन~॥ }
\end{sloppypar}

{\knu तस्याप्रत्यक्षस्यापीति} \textendash\ यद्यपि परमाणुगुणानामतीन्द्रियत्वात् स्पर्शस्य प्रत्यक्ष सिद्धत्वादर्थादवयविवृत्तित्वे सिद्धे, अवयविनश्च स्थूलस्य द्व्य$^5$णुकादिक्रमेणरम्भात् पवनजातीयस्य नानात्वं धर्मिग्राहकप्रमाणादेव सिद्धम्~। शक्यते च स्पर्शवत्तयाऽनुमातुम्~। तथापि शिष्यव्युत्पादनाय ${}^6$बोधाभासनिराकरणाय च नानात्वानुमानप्रपञ्चमाह \textendash\ तस्येति~। 

अथ किमिदं स$^7$म्मूर्च्छनं नामेत्यत आह \textendash\ सम्मूर्छुनमिति~। समानजवयोस्तुल्यवेगयोः, विरुद्धदिक्क्रिययोर्विरुद्धयोर्दिशोः क्रियेययोस्तौ विरुद्धदिक्क्रियौ तयोः सन्निपातः, परस्परमेलनं संयोगविशेष इति यावत्~। वाय्वोरिति तु प्रकृतित्वेन स्पर्शवतोरिति तु विवक्षितम्~। नन्वतीन्द्रियस्य नानात्ववद् वेगसंयोगावप्यतीन्द्रियौ इत्यत आह \textendash\ {\knu सोऽपीति~।} सन्निपातोऽपीत्यर्थः; ऊर्ध्वगमनेनानुमीयते~। 

नन्वतीन्द्रियस्य गमनमप्यतीन्द्रियं तत्कथं तेन सन्निपातोऽनुमीयत इत्यत आह \textendash\ {\knu अनुमितेनेति~।} अथ केन लिङ्गेन तदनुमेयमित्यत आह तृणादिगमनेति~। अत्राप्यूर्ध्वेत्य\textendash

\blfootnote{I तथा हि \textendash\ जे~। Ia एतेषां \textendash\ जे~। 2 दीपनयोः परिग्रहः \textendash\ पा. ३. पु; क~। 3 समानजवयोर्विरुद्धदिक्क्रिययोर्वाय्वोः \textendash\ कि~। 4 तृणादिगमनेनानुमितेन सावयविनोरूर्ध्वगमनेनानुमीयते \textendash\ कि. दे; सावयविनोरूर्ध्वगमनेनानुमीयते तृणादीनां चोर्ध्वगमनेन \textendash\ जे; सावयविनोर्रर्ध्वगमनेनानुमीयते तदपि तृणादीनामूर्ध्वगमनेन व्यो. (२०६); तृणादिगमनेन \textendash\ कं~। 5 द्व्यणुकादिप्रकमेण \textendash\ पा. ३. पु~। 6 चोद्याभास \textendash\ पा. ३. पुः क~। 7 मूर्च्छनमित्यत \textendash\ पा. ३. पु~।}

\newpage
\noindent
नुषञ्जनीयम्~। ते$^1$नानेन तृणादीनामूर्ध्वगमनेन कार्येण प्रत्यक्षसिद्धेन कारणमूर्ध्वगमनं पवनस्यानुमेयम्~। न चैतद् ${}^2$द्व्णुकादिरूपस्य त्रसरेणुरूपस्य वा वायोस्तृणादिन्यूर्ध्वं ${}^3$नेतुं सामर्थ्यमित्यत आह \textendash\ सावयविनोरिति~। अवयवानामवयवित्वेन तृणादीप्रेरणौपयिकं महत्त्वमभिप्रैति~।

एवमियता प्रबन्धेन सङ्ख्यादीनां कर्मपर्यन्तानां वायावतीन्द्रियत्वमुक्तम्~। तिर्यग्गामनस्वभावस्य कथमूर्ध्वगमनमित्यत्रोपपत्तिदर्शिता~। तथा च वायुतोऽप्यधिकमूर्ध्वगमनशीलं द्रव्यान्तरमिति नाशङ्कनीयमिति सूचितम्~। निम्नाभिसर्पणस्वभावानामप्यपां यथा द्रव्यान्तरसम्मूर्च्छनादूर्ध्वगमनं तंथैव पवनस्याप्युध्वगमनोपपत्तिरिति~। अत्रैवं प्रयोगः '$^4$प्रतिहन्याद् वायोः प्रतिहन्ता वायुरन्यः, तत्प्रतिहन्तृत्वात्; देवदत्ताद् यज्ञदत्तवत्'~। अभेदे तु प्रतिघातानुपपत्तिः; संयोगस्य द्विनिष्ठत्वात्~। प्रतिहन्यते चायं वायुः, 'तिर्यग्गमनस्वभावस्य प्रयत्नाद्यसम्भवे सति ऊर्ध्वगमनवत्त्वात्, परस्पराहतनदीपयःपुरवत्'~। अन्यथा निमित्तान्तरासम्भवे कारणं विना कार्योत्पत्तिप्रसङ्गः~। 'ऊर्ध्वगतिमानयम्, तृणादीनामूर्ध्वगमनासमवायिकारणसंयोगाश्रयत्वात्' 'तृणाद्यूर्ध्वगमनासमवायिकारणसंयोगाश्रयज्वलदनलवत्'~। \renewcommand{\thefootnote}{१}\footnote{पूर्वक इति~। कारणं विना कार्योत्पत्त्य इति सम्भव इत्यर्थः~। कि प्र. व.~।\\ \rule{0.4\linewidth}{0.5pt}}पूर्वक एव तर्कः तृणादीनामूर्ध्वगमनम्, स्पर्शवद्वेगवद्द्रव्यसंयोगजन्यम्, अनूर्ध्वगमनशीलस्य प्रयत्नासम्भवे सत्यूर्व्वगमनत्वात्, 'ज्वलत्पावकप्रेरित ${}^5$तूलाद्यूर्ध्वगमनवद् इति नानुपपत्तिरिति~। 

\hangindent=2cm {\knu (५६) प्राणोऽन्तःशरीरे रसमलधातूनां प्रेरणादिहेतुरेकः सन् क्रियाभेदादपानादिसञ्ज्ञां लभते~॥}

चतुर्थी विधामाह \textendash\ {\knu प्राणोन्तः शरीर} इति~। शरीराभ्यन्तरसञ्चारी वायुः ${}^7$प्राण इत्युच्यत इत्यर्थः~। तत्किमयं स्थानभेदादेव विषयादिभ्यो भिद्यत इत्यत आह \textendash\ {\knu रसेति}~। अशीतपीतस्याहारस्य सारो विकारो रसः~। मलाः मूत्रपुरीषस्वेददूषिकादयः~। धातवो रुधिरादयः शुक्रान्ताः~। उपलक्षणं चैतत्~। दोषौ च पित्तकफौ~। उदर्यश्चाग्निस्तेषां प्रेरणमितस्ततो नयनम्~। 'आदि 'ग्रहणाद्धारणं विकारश्चैतेषां हेतुः~। एतेन विषयाद$^8$ साधारणक्रियाभेदाद्भेदो दर्शितः~। ${}^9$मूर्तयोः समानदेशत्वविरोधभिया 'शरीराः पञ्च वायवः'

\blfootnote{I तेन तृणादीनाम् \textendash\ कि.~। 2 द्व्यणुकरूपस्य वा वायो$^\circ$\textendash\ क;~। 3 नेतुमुत्सहत इत्यत \textendash\ आह \textendash\ पा. ३. पु~। 4 प्रतिघाताद्० कि; जे~। 5 तृणा \textendash\  ${}^\circ$क~। 6 दपानादिपञ्चसञ्ज्ञां \textendash\ जै; प्रणापानादिपञ्चसञ्ज्ञां \textendash\ कि~। 7 प्राण इत्यर्थः \textendash\ कि.~। 8  ${}^\circ$साधारणार्थक्रियाभेदो \textendash\ क~। 9 मूर्तत्वसमानदेशत्वयोः \textendash\ जे~।}

\newpage
\noindent
इत्यागमप्रसिद्धिमन्यथा समर्थयति \textendash\ {\knu एकः सन् क्रियाभेदाद्} इति~। एक एव प्राणस्तेषां मलादीनां कार्यमेदात् पञ्च सञ्ज्ञाः लभते~। न तु सञ्ज्ञाभेदात् सञ्ज्ञी भिद्यत इत्यर्थः~। मुखनासिकाभ्यां निष्क्रमणप्रवेशनात् प्राणः~। तेषामेव मलादीनामधोनयनादपानः~। ${}^1$समं नयनात् समानः~। ऊर्ध्वं नयनादुदानः~। नाडीमुखेषु वितननाद् व्यान इति~।

\begin{sloppypar}
\hangindent=2cm {\knu (५७) च$^2$तुर्णां महाभूतानां सृष्टिसंहारविधिरुच्यते~। ब्राह्मेण मानेन ${}^3$वर्षशतान्ते वर्तमानस्य ब्रह्मणोऽपवर्गकाले ${}^4$संसारे खिन्नानां ${}^5$प्राणिनां निशि विश्रामार्थं सकलभुवनपतेर्महेश्वरस्य सञ्जिहीर्षासमकालं शरीरेन्द्रियमहाभूतोपनिबन्धकानां सर्वात्मग$^6$तानामदृष्टानां वृत्तिनिरोधे सति महेश्वरेच्छाऽऽ$^7$त्माणुसंयोगजकर्मभ्यः शरीरेन्द्रियकारणाणुविभागेभ्यस्तत्संयोगनिवृत्तौ तेषामा \textendash\ ${}^8$परमाण्वन्तो विनाशः~। तथा पृथिव्युदकज्वलनपवनानामपि महाभूतानामनेनैव क्रमेणोत्तरस्मि$^9$न्नुत्तरस्मिन् सति पू$^10$र्वस्य पूर्वस्य विनाशः~। ततः ${}^11$प्रविभक्ताः परमाणवोऽवतिष्ठन्ते,धर्माधर्मसंस्कारानु$^12$विधाश्चात्मानस्तावन्तमेव कालम्~॥}
\end{sloppypar}

तेऽमी पृथिव्यादयः शरीरेन्द्रियविषयानारभमाणाः किम् अविच्छेदेन ${}^13$आरभन्ते आहो्स्विद्$^14$च्छेदोप्यस्ति ? यदापि विच्छेदेनारभन्ते तदापि किमत एव किं वाऽदृष्टमपेक्षन्ते ? यदाप्यदृष्टमपेक्षन्ते तदापि किमनधिष्ठिताः किं वा केनाप्यधिष्ठिताः इति ? कश्चात्र विशेषः ? अविच्छेदे 'तद्वचना$^15$दाम्नायस्य प्रामाण्यम्' (वै. सू १ \textendash\ १ \textendash\ ३) इति व्याकुप्येत; लोकसन्तत्यविच्छेदे वेदसम्प्रदायस्याप्यविच्छेदात्~। विच्छेदे वा ${}^16$आद्यव्यवहारे व्युत्वत्त्यनुपपत्तौ सर्वलोकव्यवहारविलोपप्रसङ्गः, मूलाभावात्~। एत एवेति पक्षे वैचित्र्यानुपपत्तिः, तेषां साधारण्यात्~। अदृष्टापेक्षायां ${}^17$सर्वादृष्टवृत्तिनिरोधे कार्यानुत्पत्तिप्रसङ्गः,

\blfootnote{I समान्तान्नयनात् \textendash\ पा. ३. पु~। 2 इहेदानीं चतुर्णा \textendash\ मु. भा~। 3 वर्षशतस्यान्ते \textendash\ कि~। 4 संसार \textendash\ कं; संसारेऽति \textendash\ दे~। 5 सर्वप्राणिना \textendash\ मु. भा~। 6 गतादृष्टाना \textendash\ दे~। 7 ऽऽत्माणुसंयोगकर्मभ्यः \textendash\ दे~। 8 मापरमाणु तो \textendash\ दे~। 9 न्नुतरस्मिंश्च \textendash\ कि. जे~। I0 पूर्वपूर्वस्य \textendash\ कि~। II प्रविभक्ता एव \textendash\ दे~। 12 संस्कारानुविधा आत्मानः \textendash\ कं~। I3 'आरभन्ते' कि, जे पुस्तकयोर्नास्तित~। I4 विच्छेदेनेति \textendash\ पा. ३. पु~। I5 आम्नायप्रामाण्यम् \textendash\ पा. ३. पु~। I6 आद्यव्यवहारानुपपत्तौ \textendash\ पा. ३. पु~। I7 सर्ववृत्तिनिरोधे \textendash\ क~।}

\newpage
अदृष्टानां प्रथमं वृत्तिलाभे कारणाभावात्~। अनधिष्ठितानामेव प्रवृत्तावचैतन्यानुपपत्तिः प्रवृत्त्यनुपपत्तिर्व, अचेतनप्रवृत्तेः पराधिष्ठानकार्यत्वनियमात्~। अधिष्ठातृत्वं च नास्मदादीनाम्, उपादानाद्यनभिज्ञत्वात्~। नेतरस्य, प्रयोजनाभावादिति समाकुलितान्तेवासिन समाश्वासयन्नाह \textendash\ {\knu चतुर्णा}मिति~। सृष्टिसंहारयोः सर्गप्रत्ययो$^1$र्विधिः प्रकारः~। द्व्यणुकादिक्रमेण सृष्टिः, क्रियाविभागादिन्यायेनापरमाण्वन्तः प्रलय इति कथ्यते~। सृष्टिसंहारयोरनिदं प्रथमतां प्रतिपादयितुं सृष्टिक्रममुपक्रम्य ${}^2$संहारप्रक्रमेण प्रतिपादयति \textendash\ ब्राह्मेण मानेनेति~।

\begin{quote}
{\qt क्षणद्वयं लवः प्रोक्तो निमेषस्तु लवद्वयम्~।\\
${}^3$अष्टादश ${}^4$निमेषास्तु काष्ठास्त्रिंशत्तु ताः कलाः~॥

त्रिंशत्कलो मुहूर्त्तः स्यात् त्रिंशद् रात्र्यहनी च ते~।\\
अहोरात्राः पञ्चदश, पक्षो मासस्तु तावुभौ~॥

ऋतुर्मासद्वयं प्रोक्तमयनं तु ऋतुत्रयम्~।\\
अयनद्वितीयं वर्षो मानुषोऽयमुदाहृतः~॥

एष दैवस्त्वहोरात्रस्तैः पक्षादि च पूर्ववत्~।\\
देववर्षसहस्राणि द्वादशैव चतुर्युगम्~॥

चतुर्युगसहस्रं तु ब्रह्मणो दिनमुच्यते~।\\
रात्रिश्चैतावती तस्य ताभ्यां पक्षादिकल्पना~॥}
\end{quote}

तथा च वर्षशतान्ते वर्तमोनस्य ब्रह्मणोऽपवर्गकाले मोक्षकाले ये ह्यसङ्कल्पितफलकर्मकर्तारः साकारोपास$^5$नापरिवासितचेतसो यतयोऽतस्ते हिरण्यगर्भपदवीमनुप्राप्यापवृज्यन्त इत्यागमात्~। यदा त्वीश्वर एव कार्यवशाद् गृहीतदिव्यदेहो ब्रह्याद्यवस्थामापद्यत इति पक्षः, तदा ब्रह्मणोऽपवर्गकाले दे$^6$हावसर्जनकाले इति व्याख्येयम्; तस्यात्यन्तदुःखाभाववतो नित्यमुक्तत्वादिति~।

संसारे जन्ममरणप्रबन्धान्धकारे खिन्नानां नानाविधैर्दुःखैर्निःसहीभूतानां निशीव निशि प्रलये विश्रामार्था कियत्कालं दुःखोपशमार्थ सकलभुवनपतेः कालाग्निरुद्रादिपरमेश्वरपर्यन्तानां भुवनानां पत्युः स्वामिनो यथेष्टं विनियोक्तुरिति यावत्~। अत एव ${}^7$महेश्वरस्य~। अन्ये हीश्वरा जगदेकदेशपतय इन्द्रवरुणयमादयः~। स पुनः कृत्स्नस्यैव जगतः~। अतो

\blfootnote{I विधः \textendash\ जै~। 2 संहारक्रमं व्युत्पादयति \textendash\ पा. ३. पु~। 3 पञ्चदश \textendash\ जे~। 4 निमेषाः स्युः \textendash\ जे~। 5 पासनावासित$^\circ$\textendash\ 'क'~। 6 देहापवर्गकाले \textendash\ कि, के~। 7 महेश्वरः \textendash\ कि. क~।}

\newpage
\begin{sloppypar}
\noindent
महानीश्वरः, तस्य सञ्जिहीर्षासमकालं संहरणेच्छासमनन्तरकालम्~। यद्यपि भगवत इच्छा 'अस्ति' इत्येवमाकारा एकैव, तथाप्युपाधिभेदात् सञ्जिहीर्षा चि$^1$कीर्षेत्यादिरुच्यते~। उत्पत्तिरपि तस्या उपाध्युत्पत्तिरेव~। स च कालविशेषो वर्षशतान्तादिशब्दवाच्यः~।
\end{sloppypar}

{\knu शरीरेन्द्रियमहाभूतोपनिबन्धकानाम्} \textendash\ उत्पत्तिस्थितिहेतूनाम्~। परमाणुभ्योऽधिकपरिमाणतयोपचाराद् द्व्यणुकमपि महाभूतमुक्तम्~। {\knu सर्वात्मगतानामदृष्टानाम्} \textendash\ सर्वेष्वात्मसु समवेतानामदृष्टानाम्~। सर्गहेतूतामिति शेषः~। {\knu वृत्तिनिरोधेसति} \textendash\ प्रलयहेतुनाऽदृष्टेन प्रतिबन्धे सति~। {\knu महेश्वरेच्छात्माणुसंयोगजकर्मभ्य} इति \textendash\ महेश्वरेच्छया सहिता ये आत्माणुसंयोगाः \textendash\ आत्मपरमाणुसंयोगास्तेभ्यो जातानि यानि कर्माणि तेभ्यः~। शरीरेन्द्रियकारणाणुविभागास्तेभ्यस्तत्संयोगनिवृत्तौ शरीरेन्द्रियमूलपरमाणुसंयोगनिवृत्तौ तेषां शरीरेन्द्रियाणामापरमाण्वन्तो विनाशः~। यद्यपि परमाण्वन्त इति वक्तुमुचितं तथाप्यन्तशब्दस्यावयववचनत्वभ्रमनिरासेन परमाणूनामवधित्वमाविष्कर्तुमापरमाण्वन्त इत्युक्तम्~। तेनैतदुक्तं भवति; तावदयं विनाशो यावत्परमाणवोऽवसानमिति; न तु यथा दशान्तः पट इति~। तथा तेनैव प्रकारेण महेश्वरेछादिना पृथिव्युदकज्वलनपवनानामपि महाभूतानामनेतैव क्रमेण क्रियाविभागादिनोत्तरस्मिन् सति पूर्वस्य पूर्वस्य विनाशः~। जलादौ सति पार्थिवस्य महाभूतस्य, ज्वलनादौ सति ${}^2$जलादेः~। अयं च क्रमविशेषनिश्चय आगमादनुसर्तव्यः~।

एवमापरमाण्वन्तो विनाश इति वचनादेव परमाणूनामेवावस्थाने प्रतिपादिते साङ्ख्यादिमतनिराकरणाय पुनस्तदवस्थानमाह \textendash\ तत इति~। तदेत्यर्थः~। प्रविभक्ताः कार्यद्रव्यरहिताः प्रचयाख्यसंयोगरहिताश्चेत्यर्थः~। तत्किमेत एवावतिष्ठन्ते ? न इत्युच्यते~। किं तर्हि ? धर्मश्चाधर्मश्च संस्कारश्च भावनाख्यः, तैरनुविद्धाः सम्बद्धाश्चात्मानः~। उपलक्षणं चैतत्, अन्योऽपि नित्यवर्गः~। अनित्येष्वपि पाकजा रूपादयो गुणाः कालावच्छेदोपाधितया वर्तमानानि च महाभूतसंक्षोभप्रभववेगजानि च कर्माणि सन्तन्यमानान्यवतिष्ठन्ते~। अन्यथा कालावच्छेदानुपपत्तौ पुनः सर्गानुपपत्तेः तदिदमुक्तं {\knu 'तावन्तमेव कालम्'} इति~। ब्राह्रेण मानेन वर्षाशतान्तं यावदित्यर्थः~।

\hangindent=2cm {\knu (५८) ततः पुनः प्राणिनां भोगभूतये ${}^3$महेश्वरस्य सिसृक्षान्तरं सर्वात्मगतवृत्तिलब्धादृष्टापेक्षेभ्यस्तत्संयोगेभ्यः}

\blfootnote{I चिकीर्षेत्युच्यते \textendash\ कि~। 2 जलस्येत्यादि \textendash\ 'क'~। 3 महेश्वरसिसृक्षा \textendash\ कं~।}

\newpage
\indent
\hangindent=2cm {\knu पवनपरमाणुषु कर्मोत्पत्तौ तेषां परस्परसंयोगेभ्यो द्वयणुकादिप्रक्रमेण महान् वायुः समुत्पन्नो नभसि दोधूयमानस्तिष्ठति~। तदनन्तरं ${}^1$तस्मिन्नेव वायावाप्येभ्यः परमाणुभ्यस्तेनैव क्रमेण महान् सलिलनिधिरुत्पन्नः पोप्लूयमानस्तिष्ठति~। तदनन्तरं तस्मिन्नेव ${}^2$जलनिधौ पार्थिवेभ्यः परमाणुभ्यो महापृथिवी ${}^3$समु त्पन्ना संहताऽवतिष्ठते~। ${}^4$तदनन्तरं तस्मिन्नेव महोदधौ तैजसेभ्यो ${}^5$णुभ्यो द्व्यणुकादिप्रक्रमेणोत्पन्नो महाँस्ते जोराशिर्देदीप्यमानस्तिष्ठति~॥}

एवं संहारक्रमं प्रतिपाद्य सृष्टिक्रमं प्रतिपादयन्नाह \textendash\ {\knu ततः} तदनन्तरम्, पुनरिति अभ्यासमाह~। तेन सर्गप्रलययोरनादित्वमभिव्यनक्ति~। अथ किमर्थं पुनर्मगवान् ${}^7$त्रक्ष्यतीत्यत आह \textendash\ {\knu प्राणिनां भोगभूतय} इति~। प्राणिनां प्राणसम्बन्धयोग्यानां संसारिणामिति यावत्~। भोगस्य ${}^8$स्वसमवेतसुखदुःखानुभवस्य भूतये निष्पत्तये; महेश्वरस्य सिसृक्षायाः सर्जनेच्छाया अनन्तरं सर्वेष्वात्मसु संसारिषु गताः समवेताश्च ते वृत्तिलब्धाः ल\renewcommand{\thefootnote}{१}\footnote{पूर्वप्रयोगानुरोधेनाह \textendash\ लब्धवृत्तय इति~। भाष्यं त्वार्षमिति भावः~। कि. भा. ए~।\\ \rule{0.4\linewidth}{0.5pt}}ब्धवृत्तयश्च ये अदृष्टविशेषाः; तदपेक्षेभ्यस्तत् संयोगेभ्य आत्माणुसंयोगेभ्यः पवनपरमाणुषु कर्मोत्पत्तौ सत्याम्, तेषां परमाणूनां परस्परसंयोगेभ्यो द्वयणुकमादिर्यस्य प्रक्रमस्य तेन महान् वायुः समुत्पन्नो नभस्याकाशे दोधूयमानो भृशं कम्पमानः, द्रव्यान्तरानुत्पत्तेरप्रतिहतवेगत्वात् पवनजातिनियतत्वाच्च तिर्यग्गमनस्वभावत्वमस्य~।

{\knu तदनन्तरं} पवनोत्पत्तेरनन्तरं तस्मिन्नेव पवने स्पर्शवद्वेगवत्तया गुरुत्वप्रतिबन्धकत्वादाधारभूत इत्यर्थः~। आप्येभ्यः परमाणुभ्यस्तेनैव द्व्यणुकादिक्रमेण महान् सलिलनिधिरुत्पन्नो भृशं प्लव्मानस्तिष्ठति पवनवेगात्~। तदनन्तरं सलिलनिधेरुत्पत्तेरनन्तरं तस्मिन्नेव

\blfootnote{I तस्मिन्नेवाप्येभ्यः \textendash\ कं. दे; तस्मिन् पार्थिवेभ्योऽणुभ्यः \textendash\ जे~। 2 जलधौ \textendash\ कं; महोदधौ \textendash\ कि. टी; तस्मिन्सलिलनिधौ व्यो (३००)~। 3 महापृथिवी संहतां \textendash\ कं~। 4 ततः \textendash\ जे~। 5 परमाणुभ्यो \textendash\ व्यो. (३००)~। 6 'तेजोराशिः नचिदभिभूतत्वात्' इति. मु. कन्दल्यां दृश्यते, अयमेव पाठः भाष्ये प्रमादेनागतो भाति~। कन्दलीकारेण तु 'देदीप्यमान' इस्यस्य स्पष्टीकरणार्थ व्याख्यानं कृतम्~। पाठोऽयमन्यासु टीकासु क्वचिदपि भाष्यप्रतीकरूपेण नोपात्तः~। 7 सृजतीत्त आह \textendash\ पा. ३. पु~। 8 स्वसमवेतस्य साक्षात् सुखःदुःखानुभवस्य निष्पत्तये \textendash\ पा. ३. पु~।}

\newpage
\noindent
महोदधौ पार्थिवेभ्यः परमाणुभ्यस्तेनैव द्व्यणुकादिक्रमेण महापृथिवी \textendash\ महती प्राथम्यात् सम्पूर्णतया संहता निबिडावयवाः अविशीर्णतया~। तदनन्तरं पृथिव्युत्पत्तेरनन्तरं तेजःपरमाणुभ्यः $\rightarrow$ ${}^1$तेनैव द्व्णुकादिक्रमेण महांस्तेजोराशिः समुत्पन्नस्तिष्ठति $\leftarrow$~। अनभिभवादतिशयेन देदीप्यमानः~।

\begin{sloppypar}
\hangindent=2cm {\knu (५२) एवं समुत्पन्नेषु चतुर्षु महाभूतेषु महेश्वरस्याभिध्यानमात्रात् तैजसेभ्योऽ$^2$णुभ्यः पार्थिवपरमाणुसहितेभ्यो महदः$^3$ण्डमारभ्यते~। तस्मिंश्चतुर्वदनकमलं सर्वलोकपितामहं ब्रह्माणं ${}^4$सकलभुवनसहितमुत्पाद्य प्रजासर्गे वि$^5$नियुङ्क्ते~। स च महेश्वरेण विनियुक्तो ब्रह्माऽतिशयज्ञानवैराग्यैश्वर्यसम्पन्नः प्राणिनां कर्मविपाकं विदित्वा कर्मानुरूपज्ञानभोगायुषः सुतान् प्रजापतीन् मानसान् मनुदेवर्षिपितृगणान् मुखबाहूरुपादतश्चतुरो वर्णान्, ${}^6$अन्यानि चोच्चावचानि ${}^7$भूतानि सृष्ट्वा आशयानुरुपैर्धर्मज्ञानवैराग्यैश्वर्यैः ${}^8$संयोजयति इति~॥}
\end{sloppypar}

\begin{sloppypar}
एवमनेन क्रमेणाभिध्यानं सङ्कल्पस्तन्मात्रात्~। न तु कुलालादिव$^9$त्कायव्यापारादित्यर्थः~। तैजसानां परमाणूनां पार्थिवाणुसाहित्येनाण्डस्य हिरण्मयत्वं विवक्षितम्~। तस्मिन्नण्डे चत्वारि वदनानि कमलानीव यस्य, तं सर्वेषां लोकानां पितामहं शरीरिणामाद्यं ब्रह्माणं सकलभुवनसहितं सकलैरधस्तनैर्भूतलादिभिरूपरितनैर्भूभुवादिलोकैः सहितमुत्पाद्य प्रजासर्गे विनियुङ्क्ते 'त्वमेव कुरु' इति~।
\end{sloppypar}

{\knu स चेति} \textendash\ 'च 'शब्दः समुच्चये, द्विकर्तृकेषु कार्येषु महेश्वरस्यापि कर्तृत्वात्~। न च महेश्वरे कर्तरि किं ब्रह्मणेति वाच्यम्, शरीरान्वयव्यतिरेकानुविधायिकार्येण तस्याकृष्यमाणत्वात्~। अतिशयेनज्ञानवैराग्यैश्चर्यैः सम्पन्नः~। यद्यज्ञः स्यात् समर्थोऽपि न कुर्यात्~। यदि विरा$^10$गी न भवेत्, न कर्माण्यनुरुन्ध्यात्~। तथा च प्राणिनां कृतहानमकृताभ्यागमश्च स्यात्~। यद्यनीश्वरो भवेत् जानतोऽप्यस्यापेक्षितं न सिद्ध्येत्~। यतस्तु त्रितयसम्पन्नः, अतः प्राणिनां यानि कर्माणि तेषां विविधं पाकं विदित्वा~। {\knu 'सुतान्'}

\blfootnote{I $\rightarrow$ $\leftarrow$ एतच्चिह्वान्तर्गतः पाठः जे पुस्तके नास्ति~। 2 परमाणुभ्यो \textendash\ व्यो. (३००)~। 3 मुत्पद्यते \textendash\ कि~। 4 सर्व \textendash\ दे~। 5 नियुङ्क्ते \textendash\ दे ~। 6 अन्यान्यप्योच्चा$^\circ$\textendash\ दे~। 7 भूतानि \textendash\ कं~। 8 संयोजयति \textendash\ जे~। 9 कार्य \textendash\ पा, ३. पु~। I0 विरागवान् \textendash\ पा. ३. पु; रागवान् \textendash\ जे~।}

\newpage
\noindent
इति प्रत्येकमभिसम्बद्ध्यते {\knu 'मानसान्'} इति चायोनिजान्~। {\knu प्रजापतीन्} \textendash\ दक्षादीन्, मनून् स्वायम्भुवप्रभृतीन्~। देवान् आदित्यादीन् ऋषीन् गोतमभरद्वाजादीन्~। पितृन् क$^1$व्यबालादीन्~। गणान् कूष्माण्डादीन्; 'गण' शब्दः प्रत्येकं वाऽभिसम्बद्ध्यते~। अत्र प्रजापतयस्तत्तत् प्रतिनियतप्रजास्रष्टारः~। मनवो राजधर्मप्रवर्तयितारः~। देवा हविर्भोक्तारः~। ऋषयः प्रथमं वेदाध्येतारः~। पितरः स्व$^2$धाभागाधिकारिणः~। गणा बलिभागिनः~।

{\knu मुखेति} \textendash\ मुखाद् ब्राह्मणम्, बाहुभ्यां राजन्यम्, ऊरुभ्यां वैश्यम्, पद्भ्यां शूद्रम्~। अन्यान्यपि तिर्यङ्नारकिप्रभृतीनि~। {\knu उच्चावचानि} \textendash\ उत्कृष्टापकृष्टानीति~। {\knu आशयानुरूपै}रिति \textendash\ फलज्ञाननपर्यन्तमाशेरत इत्याशयाः कर्म ज्ञानवासनाः, तदनुरूपैस्तत्सदृशैर्धर्मज्ञानवैराग्यैश्वर्यैरित्युपलक्षणम्~। अधर्मज्ञानावैराग्यानैश्चर्यैस्तत्फलैश्च सुखादिभिः सम्यग्योजयति~। यदा त्वीश्वर एव कार्यवशाद् ब्रह्मादिशरीमुपादत्तं इति पक्षस्तदा नियोगः कायवाद्मनोव्यापारसिद्धये तत्प्रेरणं प्रयत्नवदात्मसंयोगलक्षणम्~। मुखेत्यादिना ब्राह्मणत्वाद्यभिव्यञ्जकव्यक्तिविशेषनिमित्ततददृष्टोपनिबद्धभूतभेदप्रतिपादनम्~। तदेवमियता प्रबन्धेन प्रलयविच्छितादृष्टापेक्षैः परार्थप्रवृत्तेन परमेश्वरेणाधिष्ठितैः पृथिव्यादिभिर्विश्वमारभ्यत इति वदता सर्वे पूर्वाक्षेपाः परिहृता बोद्धव्याः~। तथाहि विच्छेदे तद्वचनादाम्नायस्य प्रामाण्यमप्रत्यूह मूहेत~। मानसमन्वादिसृष्ट्या चाद्यव्यवहारमुपपादयेत्~। अदृष्टापेक्षया वैचित्र्यं समर्थयेत्~। परार्थप्रवृत्त्या च ${}^3$निष्प्रयोजनशङ्कामुत्सारयेत्~। परमेश्वराधिष्ठानेनाचैतन्यप्रवृत्तिचैतन्यप्रवृत्तिनियमौ च ${}^4$समञ्जसयेदिति~।

स्यादेतत् सर्वमेतदीश्वरसद्भावसिद्धौ सम्भवेत्; तत्सिद्धावेव किं प्रमाणम् ? इति चेत्; तद् बहुत्वेऽपि किञ्चिद् उच्यते~। 'शरीरानपेक्षोत्पत्तिकम्, बुद्धिमत्पूर्वकम्, कारणवत्त्वात्; यत्कारणवत् तदूबुद्धिमत्पूर्वकम्, यथा रथः~। तथा चैतत्, त5स्मात्तद् बुद्धिमत्पूर्वकमिति~। नन्वनुपलब्धपूर्वकोटिकेषु पर्वतादिषु कारणवत्त्वस्याप्रतीतेर्भागासिद्धमिति चेत्, न; द्रव्येषु सावयवत्वेन सह दृष्टान्ते घटदौ दर्शनात्, तद्विपरीतस्य पक्षे विवक्षितत्वात्~। विरुद्धमिदमिति चेत्; न; साहचर्यमात्रेणाविरोधात्~। कार्यत्वेनानित्यत्वानुमाने रूपसाहचर्यवदरूपस्याविनाभावाभ्युपगमे त्वसर्वज्ञत्वाद्युपेतकर्तृकत्वमुपस्थापयति~। हेतौ कस्तद्विपरीतमभिप्रेयात्, यतो विरोधश्चोद्येत~। असर्वज्ञस्य शरीरिणः प्रमाणनिरस्तत्वात् त्वमेव तद्विपरीतमभिप्रैषीति चेत्; नन्वेत्रमसर्वज्ञेन शरीरिणा हेतोरव्यापनात् कुतस्तद्विपरीतेन

\blfootnote{I वाहनादीन् \textendash\ क्रिं~। 2 स्वधाधिकारिणः \textendash\ कि; जे~। 3 निष्प्रयोजनत्वात् शङ्का पा. ३. पु निष्प्रयोजकत्वशङ्का 'क'~। 4 सम्पादयेत् \textendash\ पा. ३. पु~। 5 तस्माद् बुद्धिमत् पूर्वकमिति कि; क~।\\ ९}

\newpage
\noindent
विरोधः, तथाविधस्यानुपलब्धेरिति चेत् सुतरां तर्हि विरोधसिद्धेरसम्भवो जात्यन्धेन रूपस्पर्शयोरिव परस्परधर्मिपरिहारेणाप्रतीतेः~। प्रतीतौ वा एकत्र धर्मिणि सहोपलम्भेन विरोधस्य बाधितत्वात्~। दर्शनस्पर्शनाभ्यां वायावनुपलम्भेऽपि पृथिव्यादौ रूपसहचरस्पर्शवत्~।

\begin{sloppypar}
स्यादेतत्; न हेतुसाध्यविशेषयोर्विरोधं विशेषविरोधमाचक्ष्महे~। किं नाम व्याप्तिपक्षधर्मताभ्यामानीयमानयोर्विशेषयोरेव विरोधः; तथा हि, व्याप्तिस्ताव$^1$द्दर्शनबलप्रवृत्ता यथादृष्टं विशेष$^2$मुगस्थापयति, नालौकिकमशरीरित्वादि~। पक्षधर्मताऽपि शरीरिणा किञ्चिज्ञेनानित्यज्ञानादिमता नोपपद्यत इति तद्विपरीतमुपस्थापयति~। न चानयोर्विरोधे किञ्चित्सिध्यति~। सोऽयं विशेषविरोध इति~। तदपि नास्ति; स्वरूपेणानयोरविशेधात्~। ${}^3$परस्परानपेक्षाभ्यां विशेषानुपस्थापनात्~। न हि यो धूमवान् स वह्निमानिति केवलया व्याप्त्या किश्चिदुपनीयते; तथा सति पक्षधर्गतावैयथ्यत्~। नाप्ययं धूमवानित्यनया पक्षधर्मतया नियमानपेक्षया किञ्चिदुपस्थाप्यते, तथा सति व्याप्तिवैयर्थ्यादिति~। किन्तु परम्परसहकारितया~। तथा च निरपेक्षतादशायां विशेषानुपस्थापनादेव विरोधासिद्धिः~। ${}^4$सापेक्षतादशायां सहोपलम्भादेवेति~। नन्वलौकिकं विशेषमुपस्थापयति पक्षधर्मता~। न$^5$ चासौ लौकिकं ${}^6$व्या\renewcommand{\thefootnote}{१}\footnote{व्याप्त्यनुस्मरणविषयीकृतमित्यर्थः~। कि. प्र. व~।}प्त्यानुस्मारितं सहत इति ब्रूमः~। ततः किम् ? स\renewcommand{\thefootnote}{२}\footnote{स इति~। लौकिको विशेष इत्यर्थः~। कि. प्र. व~। } न सिद्धयेदिति चेत्; को नु तं साधयितुमुद्यतः~। \renewcommand{\thefootnote}{३}\footnote{अनेनैवेति \textendash\ लौकिकतिशेषविरोधेनालौकिकोऽपि विशेषोऽशरीरित्वादिर्न सिद्ध्येदित्यर्थः~। कि. प्र. व~।\\ \rule{0.4\linewidth}{0.5pt}}अनेनैव विरोधेनान्योऽपि न सिद्ध्येदिति चेत्; तदिदमायातं द्व्यणुकगतं कार्यत्वं कार्यसमवायिकारणत्वं न सहत इत्यकार्यमपि न ${}^7$सिद्ध्यतीति~।
\end{sloppypar}

स्यादेतत्; साध्यांशवि$^8$शेषयोविरोधं विशेष$^9$विरोधमाचक्ष्महे~। तथा हि नित्यत्व माकाशादौ ज्ञानत्वपरिहारेण सिद्धम्, ज्ञानत्वं च नित्यत्वपरिहारेणास्मदादिज्ञाने; तथा कर्तुः शरीरिवं कुविन्दादेः, अशरीरस्याकर्तृत्वं मुक्तस्य~। तदनयोर्ज्ञानत्वनित्यत्वयोः कर्तृत्वा\textendash

\blfootnote{I दर्शनवशात् प्रवृत्ता \textendash\ पा ३. ५~। 2 मुपनयति \textendash\ पा. 3. पु~। 3 परस्परानपेक्षायां \textendash\ कि~। 4 धूमवानित्यतया पञ्चघर्मतया विशेषत्योपस्थापितत्वात् क्व विरोध \textendash\ इत्यधिकं पा. ३. पु; सापेक्षदशायां तूभयापेक्षतयार्थसमर्थकत्वादेव्र न विरोधः \textendash\ पा. २. ५. सापेक्षतादशायां सहोषलम्भादेः \textendash\ जे~। 5 सर्वलौकिकं~। 6 व्याप्त्यनुप्त्मारितमिति प्रकाशसम्मतः पाठः~। 7 भविष्यतीति \textendash\ कि~। 8 विशेषयोरिव \textendash\ क~। 9 विशेष्यभाव \textendash\ क~।}

\newpage
\noindent
शरीरित्वयोश्चैकधर्मिसंसर्गो विरुध्यत इति~। एतदपि नास्ति; उप\renewcommand{\thefootnote}{१}\footnote{उपसंहारस्थानस्येति \textendash\ ननु न व्याप्तिपक्षधर्मतोपस्थाप्यमानेन विरोधं क्रमः, किन्तु ज्ञानमनित्यमिति व्याप्तिग्राहकमानेन~। एतेन व्याप्त्योर्न विरोधः~। कार्य सकर्तृकमिति व्याप्तिपक्षधर्मतासाचिव्यात् बलीयसी, कर्ता शरीर्येवेत्यत्र व्यातौ तदभावान्न सा तथेति तुल्यबलत्वाभावादिति निरस्तम्~। कि. प्र. व. }संहारस्थानस्यैकस्य धर्मिणोऽप्रतीतेः~। नह्युष्णत्वशीतत्वे पय पावकयोरेव विरुद्धे किन्त्वेकत्र पयसि तेजसि वा पृथिव्यां वा, एकत्रोपसंह्रियमाणे~। तदेतेऽपि नियत्वज्ञानत्वे धर्मिभेदेनाविरुद्धेऽपि, अस्मदादिबुद्धौ गगने चोपसंहाराद्विरुद्धे इति विवक्षितमायुष्मतः~। अशरीरित्व्कर्तृत्वे वा कुलालादौ ${}^1$मुक्तात्मादौ चोपसंहाराविरुद्धे इत्यभिमतं चिरजीविनो न नः किञ्चिदनिष्टमापद्यते~। न हि नित्यत्वमूर्तत्वे नभसि घटादौ वा विरुद्धे इति परमाणुसाधनस्य किञ्चिदपि हीयते~। अथ्येश्वरात्मनि तज्ज्ञाने चोपसंहारादनयोर्विरोधोऽभिवातुमभिजषितः स तर्हि मूर्तत्वनित्यत्वयोः परमाणाविव धर्मिसाधकेनैव बाधित इति बुद्ध्यस्व; इति नास्ति विरोधगन्धोऽपि~। \renewcommand{\thefootnote}{२}\footnote{विशेषविरुद्धेति \textendash\ यत्र साध्यसाधनविशेषणयोर्विरोध इत्यर्थः~। कि. प्र. व.}विशेष$^2$विरुद्धोदाहरणं तु 'चन्दनदहनवानयम्, सुरभिधूनवत्त्वाद्' इति द्रष्टव्यम्~।

स्यादेतत्; व्या\renewcommand{\thefootnote}{३}\footnote{व्यापकेति \textendash\ कर्तृव्यापकशरीरस्य याऽनुपलब्धिस्ततो बाध इत्यर्थः~। कि. प्र. व.\\ \rule{0.4\linewidth}{0.5pt}}पकानुपलब्धिबाध एव विशेषविरोध इत्युच्यते~। तथा हि; कर्तृत्वव्यापकं शरीरादि; तच्चातो व्यावर्तमानं कर्तारमपि व्यावर्तयति~। तृणादिविकारकारित्वेन हिमस्य वह्विमत्त्वसाधने तद्व्यापकमौष्ण्यमिव व्यावर्तमानमग्निमत्त्वमिति चेत्; नैतदपि साधीयः~। कस्य हि किमत्र व्यापकम् ? किं कर्तुः शरीरित्वादिकम् ? आहोरिस्वित्कार्यस्य शरीरादिमत्कर्तृपूर्वकत्वम् ? प्रथमे नेश्वरः कर्ता, शरीरादिरहितत्वादिति प्रमाणार्थः स्यात्~। तथा चाश्रयासिद्धिः, धर्मिग्राहकप्रमाणबाधो वा; 'सावयवाः परमाणवो मूर्तत्वाद्' इतिवत्~। द्वितीयत्तु नियमः प्रत्यग्रजायमानाङ्कुरादावेव शरीरिणः कर्तुरनुपलम्भेन निरस्तः~। दृष्टान्तस्तु सर्वथै$^3$वानुपपन्नः~। तृणादिविकारो हि यदि रूपादिगरावृत्तिमात्रहेतुः स नूनमौष्ण्यापेक्षेण तेजसा कर्तव्यः~। तादृशे च पाकेऽनिमित्तं हिममिति न किञ्चिदनिष्टमापद्यते~। नहि सौरस्य तेजसस्रैलोक्यपाकहेतोर्हिमादपगमः ${}^4$क्षमते~। अथ विकारो भस्मादिरूपो वितक्षितः सोऽसिद्ध एव~। हिमहतेषु तृणादिषु क्व विरोधो बाधो वा ? अथ रूपादिपरावृत्तिमात्रेणैवाग्निः साध्यते तदशक्यम्~। तस्य दर्शनस्पर्शनग्राह्यस्य योग्यानुपलम्भबाधितत्वात्, अतादृशस्य

\blfootnote{I शुकादौ \textendash\ कि; क~। 2 विशेषविरोधोदाहरणं कि; क~। 3 ${}^\circ$वासङ्गतः \textendash\ जे~। 4 क्रियते \textendash\ पा. २. ६~।}

\newpage
\begin{sloppypar}
\noindent
तेजोमात्रस्य निवृत्तेरशक्यत्वादनिष्टत्वाच्च~। तस्माद् व्यापकानुपलब्ध्याऽनुमानं बाध्यत इति रिक्तं वचः~। सत्प्रतिपक्षस्तु कथञ्चित् स्यात्; 'अग्निमद्धिमं तृणादिविकारहेतुत्वात्'~। निरग्निकं हिमं तथाविधौष्ण्यरहितत्वादिति~। अस्तु तर्हि तदेव प्रकृते~। शरीरानपेक्षोत्पत्तिक्म् अकर्तृपूर्वकम्, शरीरानपेक्षोत्पत्तिकृत्वात्, गगनवद्' इति~। नैतदेवम्, शरीरानपेक्षाया उत्पत्तेरसाधरणत्वात् सककर्तृकाद् \renewcommand{\thefootnote}{१}\footnote{आकाशादेरिति \textendash\ तत्रोत्पत्तिरेव न कुत्र शरीरानपेक्षेत्यर्थः ननु शरीरापेक्षोत्पत्तिरहितत्वं हेतुः स चाकाशादावस्त्येवेत्यत आह \textendash\ शरीरादिनुत्पदश्चेति~। कि प्र. व~।}घटादेरिवाकर्तृकादाकाशादेरपि व्यावृत्तत्वात्; उत्पत्तिव्यावृत्तिमात्रस्य चासिद्धेः~। शरीरादनुत्पत्तेश्चासमर्थविशेषणत्वात्~। सपक्षे गगनादौ अनुत्पत्तिमात्रस्यैव व्याप्यत्वनिश्चयात्, निश्चितव्याप्तेश्च पक्षोपनयनात्, अन्यथाऽतिप्रसङ्गात् ${}^1$इति~।
\end{sloppypar}

स्यादेतत्; यथा ज्ञानस्य नित्यत्वे शरीरापेक्षा नास्ति, ज्ञानगतकार्यत्वप्रयुक्तत्वाच्छरीरसम्बन्धस्य~। तथा प्रयत्नस्य नि2त्यत्वे ज्ञानापेक्षाऽपि न स्यात्, प्रयत्नगतकार्यत्वप्रयुक्तत्वात् ज्ञानसम्बन्धस्येति~। तर्काणारिशुद्धिरस्तु दूषणमिति चेत्, तदप्यसारम्~। ईश्वरमधिकृत्य तर्काणामाश्रयासिद्ध्यसिद्धिभ्यामनवसरदुःस्थत्वात्~। 'यदीश्वरः कर्ता स्यात्, शरीरी स्यात्, अनित्यज्ञानवांश्च भवेत्'~। 'यदीश्वरस्य प्रयत्नः स्यादनित्यः स्यात्'~। 'यदीश्वरप्रयत्नो नित्यः स्यात्, ज्ञानानपेक्षः स्यात्'~। 'यदीश्वरस्य बुद्धिः स्यात्, असर्वविषया भवेद्' इत्यादयो हि ते क्षित्यादिकमधिकृत्य विपर्ययापर्यवसानात्~। यदि विवादाध्यासितं बुद्धिमद्धेतुकमभविष्यत्, अनित्यप्रयत्नहेतुकतामापत्स्येत~। शरीरजन्यं वाऽभविष्यत्, असर्वज्ञकार्य च इत्यादयो हि ते त\renewcommand{\thefootnote}{२}\footnote{तद्विपर्ययांश्चेति \textendash\ 'क्षित्यादिकं न बुद्धिमद्धेतुकम्' 'अनित्यप्रयत्नाजन्यत्वात् ' इत्यादौ व्यर्थविशेषणत्वमजन्यत्वस्यैव साध्यव्याप्यत्वादित्यर्थः~। न चेति \textendash\ यदि प्रयत्नेन बुद्धि बुद्धिनिवृत्त्या वा प्रयत्ननिवृत्तिं साधयेत् तदा ज्ञानसम्बन्धे प्रयत्नगतं कार्यत्वमुपाधिःस्यात्~। न चैवमपि तु यथा धूमेनैकदैवार्द्रेन्धनप्रभववह्निमत्त्वं साध्यते तथा कार्यत्वेनैकदैव प्रयत्नबुद्धिसाधनमित्यर्थः~। कि. प्र. व~।\\ \rule{0.4\linewidth}{0.5pt}}द्विर्ययाश्च असमर्थविशेषणत्वेन प्रागेव ${}^3$निरस्ताः~। न च प्रयत्नेन बुद्धिरिह साध्यते, बुद्ध्या वा प्रयत्नः इतरव्यावृत्त्या वाऽन्यतरव्यावृत्तिः, येनैतत्सम्बन्धौपाधिकत्वं ${}^4$चोद्यमवकाशमासादयेत्~। किन्तु धूमवत्त्वेनार्द्रेन्धनप्रभवदहनवत्त्वमिव कार्यत्वेन परिदृष्टसामर्थ्यकारकचक्रप्रयोक्तृमत्त्वम्~। तथा च ${}^5$साध्यांशयोर्ज्ञानप्रयत्नयोर्दहनार्द्रेन्धनयोरिव सोपाधिरपि सम्बन्धो हेतु न दूषयति, तस्य साध्यसम्बन्धे निरूपाधिकत्वादिति~।

\blfootnote{I ' इति ' कि, क. पुस्तकयोर्ना स्ति~। 2 ' नित्यत्वे ' क पुस्तके नास्ति~। 3 दूषिता इति \textendash\ जे~। 4 देश्य \textendash\ क~। 5 व्यापकयोः \textendash\ जे~।}

\newpage
ननु कारणविशेषा\renewcommand{\thefootnote}{१}\footnote{अवश्यमिति \textendash\ कार्ये विशेषस्याकस्मिकत्वापत्तेरित्यर्थः~। कि. प्र. व~।}दवर्श्यं कार्ये विशेषेण भवितव्यम्~। अस्ति च केषुचिच्छरीरादिकारणं ततोऽपि विशेषः स्यात्~। अवश्यं चैतदभ्युपगन्तव्यम्, अन्यथा शरीरिकर्तृत्वानुमानं न स्यात्~। स च ${}^1$विशेषो यद् दृष्टेरक्रियादर्शिनोऽपि कृतबुद्धिरूत्पद्यते~। तदित्यदूरविप्रकर्षेणोच्यते~। ततः स एव कर्तारं प्रयोजयति~। कार्यत्वसामान्यं तु तत्प्रयुक्तां व्याप्तिमुपजीवतीति, न; विशेषस्य विशेषं प्रत्येव प्रयोजक्रत्वे सामान्यं प्रत्यनुपाधित्वात्~। अन्यथा घरादित्ववान्तरविशेषेषु सत्सु यदृष्टेरित्याद्युपलक्षितोऽपि विशेषस्तु तप्रयुक्तां व्याप्तिमुपजीवतीत्यप्रयोजकः स्यादिति यत्किश्चिदेतत्~। त\renewcommand{\thefootnote}{२}\footnote{तर्कस्त्विति~। विवादपदं कर्तृजन्यं यदि न स्यात् जन्यं न स्याद् व्योमवदित्यर्थः~। कि. प्र. व~।}र्कस्तु यदि कार्यं कर्तारमतिपतेत्, तत् कारणान्त$^2$रमप्यतिपतेत्~। तथा च ${}^3$कार्यत्वादेव हीयत इति संक्षेपः~। वि\renewcommand{\thefootnote}{३}\footnote{ननु शरीरजन्यत्वमुपाधिरस्तु~। न च पक्षमात्रव्यावर्तकविशेषणकत्वात् साधनविशेषितत्वात् साधनतुल्ययोगक्षेमत्वाच्च तत्रोपाधिः~। प्रकृते तादृशानुपाधिताबीजस्य साध्यव्यापकताग्राहकमानाभावस्य भावात्~। शरीरसहकृतस्यैव कर्तुः कार्यजनकतया कार्यकारणभावस्य साध्यव्यापकाग्राहकत्वात्~। अत एव बाधोन्नीतः पक्षेतरवह्विमत्त्वे आर्द्रेन्धनप्रभववह्निमत्त्वेन जलस्य गन्धवत्त्वे पृथिवीत्वमुपाधिर्भवति, तेषां साध्यव्यापकताग्राहकतामानसत्त्वादित्यत आह \textendash\ विस्तरस्त्विति~। शरीरजन्यं नोपाधिः हस्तादिजन्यकार्ये साध्याव्यापकत्वात्~। साक्षात्प्रयत्नाधिष्ठेयजन्यत्वस्य च साधनव्यापक्त्वात्~। अदृष्टपरमाण्वादीनामीश्वरेण साक्षात्प्रयत्नाधिष्ठानाद् इत्याद्यन्यत्र विस्तर इत्यर्थः~। कि. प्र. व~॥\\ \rule{0.4\linewidth}{0.5pt}}तरस्तु न्यायकुसुमाञ्जलावात्मतत्त्वविवेके चाध्यवसेय इति~।

\begin{sloppypar}
अस्तु तावदेवम्, विशेषस्य तु कुतः सिद्धिः ? यथा ह्यप्रतीतेन विरेधादि प्रत्येतुमशक्यम्, तथा तस्य साधनमप्यशक्यम्~। लब्धरूपं हि क्वचित् किश्चिद्विधीयत इति चेत्, न; साधनसामान्यसम्बन्धबलेन साध्यसामान्यविधावुपक्रान्ते तत्सहकारिपक्षधर्मता$^4$बलेनाप्रतीते ${}^5$च विशेषे पर्यवसानात्~। सामान्यसङ्गतिबलप्रवृत्तानं पदानामाकाङ्क्षादिसहकारिलब्धेऽलौकिकवाक्यार्थ इति प्रमाणवृत्तम्~। अन्यथा न किञ्चित् क्वचिदप्रतीतं प्रतीयेतेति~।
\end{sloppypar}

अत्र कश्चिदाह; कृतं तर्हि केवलव्यतिरेकिणा अन्वयिन एवाष्टद्रव्यातिरेकादेरपि सिद्धेः~। तदसत्; आकाङ्क्षानुपपत्तिभ्यां विशेषव्यवस्थापनात्~। शाब्दो ह्यर्थो यावता विना नान्त्रयमुपैति तावन्तमनालम्ब्य प्रतीतेरपर्यवसानमत्राकाङ्क्षेत्युच्यते~। तावानवधिरन्वयि\textendash

\blfootnote{ विशेषोपदृष्टे \textendash\ कि~। 2 मप्यतिक्रामेत \textendash\ पा. ४. पु. क~। 3 कार्यत्वं \textendash\ क~। 4 ${}^\circ$वशादेवाप्रतीते \textendash\ जे~। 5 'च' कि. पुस्तके नास्ति~।}

\newpage
\noindent
नस्तत्प्रतीतौ पर्यवसितायां क्वचिद्विशेषे प्रमाणान्तरबाधोऽनुपपत्तिश्च~। तावती सीमा व्यतिरेकिणः~। तदत्र नगादिद्वयणुकर्पर्यन्तपक्षीकरणे तत्कारणपरमाण्वादिविषयं ज्ञानं यदि न स्यात् कः शब्दार्थो 'नगसागरादि सकर्तृकम्' इति ? परिदृष्टमामर्थ्यकारकचक्रप्रयोक्तुः कर्तृशब्दवाच्यत्वात्~। एवं सर्गादिकार्यपक्षीकरणे तत्काले यदि न शरीराद्यनपेक्षं ज्ञानं कोऽर्थः 'सर्गादि कार्यं बुद्धिमत्पूर्वकम्' इति ? सर्गसन्निहितो हि शरीरेन्द्रियविषयग्रामविरहोपलक्षितः कालः सर्गादिरित्युच्यते~। तस्माद्विशेषप्रतीत्या सामान्यप्रतीत्यपर्यवसानादिह पक्षधर्मतासहायात् कार्यत्वादेव नित्यसर्वज्ञकर्तृत्वसिद्धिः~। 'इच्छादयः क्वचिदाश्रिताः कार्यत्वाद् गुणत्वाद् वा' इत्यत्र त्वष्टद्रव्यातिरिक्तद्रव्यपारतन्त्र्यमप्रतीत्य शब्दार्थो न प्रतीयत इति नास्ति~। न ह्याश्रयादिशब्दाः ${}^1$क्षित्यादिव्यतिरिक्तार्थाः क्षित्यादिविरुद्धार्था वा, येन तत्प्रतिक्षेपनियमेनेच्छादिपदार्थानामन्वयः~। प्रमाणान्तरबाध$^2$पर्यालोचनादनुपपत्तिरित्यन्यदेतत्~॥

\hangindent=2cm {\knu (६०) आकाशकालदिशामेकैकस्वादपरजात्यभावे पारिभाषिक्यस्तिस्रो सञ्ज्ञा भवन्ति~। ${}^3$आकाशः कालो दिगिति~॥}

आकाशनिरूपण$^4$प्रसङ्गेन कालदिशोरुपन्यासः~। {\knu आकाशकालदिशामिति} \textendash\ पारिभाषिक्यो निमित्तमन्तरेण शृङ्गग्राहिकतया तिस्रः सञ्ज्ञा भवन्ति, 'आकाशं कालो दिगिति~। कुतो हेतोरित्यत आह \textendash\ {\knu अपरजात्यभावे सतीति}~। अपरजातीनामाकाशत्वादीनामभावादित्यर्थः~। तदेव कुत इत्याह \textendash\ एकैकत्वाद् व्यक्तिमेदाभावात्~। 'नित्यमे$^5$कमनेकव्यक्तिवृत्तिसामान्यम्' इति लक्षणव्याघातेन स्वरूपव्याघातप्रसङ्गाद् इति भावः~। तथापि पारिभाषिकत्वमनुपपन्नम्, अतीन्द्रियत्वेन स्वरूपनिर्देशानुपपत्तेः~। यः शब्दाश्रयस्तदाकाशम्, इत्येवं तु शब्दाश्रयत्वमेवोपाधिः स्यात्~। न; शब्दाश्रयत्वस्योपलक्षणतया तदस्थत्वात् 'अयमसौ देवदत्तः' इत्यत्र इदन्तावत्~। अन्यथा शब्द गुणमाकाशम्' इति सह प्रयोगो न स्यात्~। यद्येवमाकाशादिशब्देभ्यः परस्य त्वतलदेर्निरर्थकत्वमापद्येत इति चेत्; न; आकाशस्य स्वरूपमितिवदौपचारिकत्वाद् व्यावृत्तिपरत्वाद् वेति~।

\hangindent=2cm {\knu (६१) त$^6$त्राकाशस्य गुणाः शब्दसङ्ख्यापरिमाणपृथक्त्वसंयोगविभागाः~॥}

\blfootnote{I क्षित्याद्यतिरिक्तार्था \textendash\ कि. जे~। 2 पर्यालोचनयाऽनुपपत्ति \textendash\ पा. ४. पु~। 3 आकाशं \textendash\ कि; व्यो~। 4 प्रसङ्गात् \textendash\ पा. ४. पु~। 5 ${}^\circ$मनेकवृत्ति \textendash\ पा. ३. पु~। 6 तत्राकाशगुणाः \textendash\ कि; तस्यगुणाः \textendash\ व्यो. (३२१)}

\newpage
तस्य ${}^1$स्वरूपस्थितये ${}^2$द्रव्यत्वव्यवस्थितये च गुणानाह \textendash\ {\knu तत्रेति}~। तास्वाकाशकालदिक्षुमध्ये तत्राकाशस्य शब्देन स्वरूपं सङ्ख्यादिना च द्रव्यत्वं व्यवस्थाप्यते~। अरूपतया ${}^3$चक्षुषस्तत्राप्रवृत्तेः तस्य रूपयोग्यतामुपादायैव द्रव्यग्राहकत्वात्~। अन्यथाऽऽत्मनोऽपि चाक्षुषत्वप्रसङ्गात्~। कथं तर्हि इह पक्षी नेह पक्षीति प्रत्ययं ? इति चेत्, आलोकमण्डलमाश्रित्येति ब्रूमः~। तथापि 'इहालोको नेहालोक' इति कथमिति चेत्; तदवयवानाश्रित्येति य$^4$त्किश्चिदेतत्~।

\hangindent=2cm {\knu (६२) श$^5$ब्दः प्रत्यक्षत्वे सति अकारणगुणपूर्वकत्वाद्, अयावद्द्रव्यभावित्वाद्, आश्रयादन्यत्रोपलब्धेश्च न ${}^6$स्पर्शवद्विशेषगुणः~। बाह्ययेन्द्रियप्रत्यक्षत्वाद्, आत्मान्तरग्राह्यत्वाद्, आत्मन्यसमवायाद्, अहङ्कारेण विभक्तग्रहणाच्च नात्मगुणः~॥}

कथं पुनः शब्दस्तत्स्वरूपव्यवस्थापक इत्यत आह \textendash\ {\knu शब्द} इति~। 'शब्दो न स्पर्शवद्विशेषगुणः, प्रत्यक्षत्वे सति अकारणगुणपूर्वकत्वात्'~। योऽपि स्पर्शवद्विशेषगुणं शब्दमिच्छति सोऽपि ${}^7$स्थूलवीणाशङ्खाद्याश्रयमिच्छेत् ${}^8$न तु परमाण्वाद्याश्रयम्; अप्रत्यक्षत्वप्रसङ्गात्; परमाणुगुणानामतीन्द्रियत्वात्~। तथा च स्थूलाश्रयस्य सामान्यवतः कार्यत्वेन सामान्यादित्रयव्यवच्छेदे स्पर्शरहिततया च ${}^9$कार्यद्रव्यत्वव्यावृत्तौ प्रतिनियतेन्द्रियग्राह्यतया च सामान्यगुणत्वकर्मत्वानुपपत्तौ विशेषगुणत्वमुभयवादिसिद्धम्~। स्पर्शवद्द्रव्यं प्रति शब्दस्य तत्प्रतिषेधः प्रतिज्ञायते${}^10$शब्दो {\knu न स्पर्शवद्विशेषगुणं} इति~। अत्र हेतुः 'अकारणगुणपूर्वकत्वाद् ' इति~। स्पर्शवतां शङ्खादीनां यानि समवायिकारणानि तेषां ये गुणास्तदनपेक्षत्वात्~। 'ये पुनः स्पर्शवद्विशेषगुणा न ते तदनपेक्षाः, यथा रूपादयः' इति केवलव्यतिरेकी~। अत्र च पिठरपाकानभ्युपगमात् न तद्विशेषगुणैर्विरोधः, परमाणुगुणैस्तु स्यात्~। तन्निषेधार्थमाह \textendash\ {\knu प्रत्यक्षत्वे सतीति}~। हेत्वन्तरमाह \textendash\ 'अयावद्द्रव्यभावित्वाद्' इति~। सत्स्वेव ${}^11$शङ्खादिषु तन्निवृत्तैः~। ये पुनस्तेषां विशेषगुणाः, न ते तेषु सत्सु 

\blfootnote{I द्रव्यत्वव्यवस्थितये स्वरूपस्थितये \textendash\ कि; क~। 2 द्रव्यत्वसिद्धये \textendash\ पा. ४. पु~। 3 अचाक्षुषस्तत्राप्रतीतेः \textendash\ पा. ४. पु~। 4 किश्चिदिदम् \textendash\ पा. ४. पु; ब्रूमः यत्किञ्चिदेतत् \textendash\ क~। 5 तत्र शब्दः \textendash\ कि~। 6 स्पर्शवतां \textendash\ दे~। 7 वीणाद्याश्रय \textendash\ जे~। 8 न च \textendash\ पा. ४. पु~। 9 बहिरिन्द्रियप्रथमद्रव्यत्वव्यावृत्तौ \textendash\ पा. ४. पु; द्रव्यत्वव्यावृत्तौ \textendash\ क~। I0 'शब्दो' इति कि; जे पुस्तक्योर्नास्ति~। II वंशशङ्खादिषु तन्निवृत्तेः \textendash\ कि; सत्स्वेवं शङ्खादिषु तन्निवृत्तेः \textendash\ क; शङ्खादिषु निवृत्तेः \textendash\ पा. ४; पु; जे~।}

\newpage
\noindent
निवर्तन्ते, यथा रूपादयः~। अत्रापि पूर्ववत् 'प्रत्यक्षत्वे सति' इत्यनुषञ्जनीयम्, अन्यथा परमाणुगुणैरेव विरोधः स्यात्~। सुखादीनामात्मगुणत्वसिद्धावन्वयोऽपि द्रष्टव्यः~। अनयोस्तर्कं सूचयति \textendash\ {\knu आश्रायदन्यत्रोपलब्धेरिति~।} आश्रयाभिमताच्छङ्खादेरन्यत्र कर्णशष्कुस्यवच्छिन्ने नभस्युपलब्धेः~। यदि हि शङ्खाद्याश्रयः शब्दः स्यात्, तत् पिधाने नोपलभ्येत, कर्णविवरपिधान इव सर्वदैवाप्राप्तेः~। अप्राप्तस्य च ग्रहणे सर्वत्र सर्वदैव सर्वैः सर्वशब्दोपलम्भप्रपङ्गः~। न च शङ्खश्रवसोः प्राप्तिरस्ति~। न चाश्रयापेक्षा एव गुणाः परेण प्राप्यन्ते~। तस्माच्छङ्खादीनि निमित्तान्यपहाय द्रव्यान्तरे वर्तमानो वीचीतरङ्गन्यायेन ${}^1$कदम्बमुकलन्यायेन वा ${}^2$कर्णशष्कुल्यवच्छिन्नमाकाशादिदेशं ${}^3$प्रत्यासन्न उपलभ्यते; अन्यथा त्वनुपलब्धिप्रसङ्गः~।

ननु वायुगुणोऽपि भविष्यतीति~। अन्यथाऽपि मेर्यादिषु ताडितेषु यावद्वेगमप्रतिष्ठमानः पवनो निमित्ती$^4$भविष्यतीत्यभ्युपगमः~। वदन्ति च केचिद् वायुरापद्यते शब्दतामिति~। नः श्रोत्रस्य प्रतिनियतार्थग्राहकत्वेन बहिरिन्द्रियत्वात्~। बहिरिन्द्रियस्य च प्रतिनियतग्रह्यजातीयविशेषगुणवत्त्वनियमात्~। अन्यथा सर्वस्य सर्वार्थत्वेऽन्धबधिराद्यभावप्रसङ्गात्~। एकस्य सर्वार्थत्वे तद्व्यावृत्तौ गुणान्तरानुपलम्भप्रसङ्गात्~। तथा च शब्दस्य वायवीयत्वे श्रोत्रस्यापि वायवीयत्वप्रसङ्गः~। ततः किमिति चेत्; एकस्य स्पर्शशब्दग्राहकत्वेऽनेकार्थतया बाह्यत्व5व्याघातः~। बाह्यत्वे चानेकार्थत्वाऽनुपपत्तिः~। वायवीयत्वाविशेषेऽपि त्वक्श्रोत्रयोर्व्यक्तिभेदात् प्रतिनियतार्थत्वाविरोध इति चेत्; न; प्रतिशरीरं घ्राणादीनां व्यतिभेदेऽपि विषयवैचित्र्यादर्शनात्~। अदृष्टवैचित्र्यादेवमपि स्यात् को विरोधः, इति चेत्; तर्हि पार्थिवैकप्रकृतीनामेवादृष्टवैचित्र्यादुत्पादकवैचित्र्यं विषयव्यवस्थायै भविष्यतीति कृतं रसनादीनां प्रकृत्यन्तरकल्पनया~। बाह्यानुसारेण तत्र तथेति चेत् समः समाधिः~। तस्माद्यथोक्तहेतुभ्यां शब्दो न स्पर्शवद्विशेषगुण इति~।

आत्मगुणो भविष्यतीत्यत आह \textendash\ {\knu नात्मगुणः}~। कुतः, इत्यत आह \textendash\ {\knu बाह्येन्द्रियप्रत्यक्षत्वात्}~। मनसोऽन्यदिन्द्रियं बह्येन्द्रियम्, तत्प्रत्यक्षत्वाद् गन्धादिवदित्यन्वयः~। बाह्येन्द्रियत्वं श्रोत्रस्य प्रतियतार्थग्राह्यत्वात् सिद्धम्~। आत्मगुणानां बाह्वेन्द्रियेण ग्रहीतुमयोग्यत्वात्~। अन्यथा बुद्ध्यादीनामपि तथात्वप्रसङ्गः~। हेत्वन्तर\textendash

\blfootnote{I पाठोऽयं कि. पुस्तके नास्ति~। 2 कर्णशष्कुलीमन्तं \textendash\ क~। 3 आपन्न \textendash\ पा. ४. पु., आसन्न क~। ४ भविष्यतीत्युपरमः \textendash\ पा. ४६ पु~। 5 व्याघातप्रसङ्गात् \textendash\ पा. ४. पु; बाह्यत्वानुपपत्तिः \textendash\ क~। 6 प्रतिनियतविषयत्वाच्चक्षुर्वत् \textendash\ कि. जे; प्रतिनियतेन्द्रियत्वाच्चक्षुर्वत् \textendash\ क~।}

\newpage
\noindent
माह \textendash\ {\knu आत्मान्तरग्राह्ययत्वादिति}~। यत्रस्थः शब्दस्ततोऽन्य आत्मा आत्मान्तरम्, तद्ग्राह्यत्वादित्यर्थः~। ${}^1$वीणाद्याश्रयतया हि शब्दः प्रतीयते तत्सन्तानन्या$^2$येन श्रोत्रस्थो वा~। तस्मादन्यश्चास्य ग्राहकः~। ये त्वात्मगुणा गृह्यन्ते ${}^3$ते च ग्राहकस्था एव गृह्यन्ते; यथा सुखादयः~। गृह्यमाणस्तु शब्दो न ग्राहकस्थ इति~। तदेव कुत इति चेदत आह \textendash\ आत्मन्यसमवायात् $\rightarrow$ ग्रा$^4$हकाऽसमवायात् $\leftarrow$~। इदमप्यसिद्धमत आह \textendash\ अहङ्कारेण विभक्तस्य व्यधिकरणस्य ग्रहणात्~। वीणा वाद्यते, वेणुः पूर्वते, न त्वहं वाद्ये, अहं पूर्वे वेति कस्यचित्प्रत्ययः~। अहङ्कारेण व्यधिकरणस्याप्यात्मसमवायेन गन्धादीनामपि तथात्वप्रसङ्गः~। 

\begin{sloppypar}
अ$^5$न्ये तु 'आत्मान्तरग्राह्यत्वाद्' इत्यनेकप्रतिपत्तृसताधारणत्वाद् इति हेत्वन्तरं वर्णयन्ति; स तु सन्दिग्धासिद्धः~। सन्तानानुमानेन मूलप्रत्यभिज्ञानान्नैष दोष इति चेत्, तर्हि सुखादीनामप्यनुमानेनानेकप्रतिपत्तृसाधारणप्रत्यभिज्ञानविषयत्वादनैकान्तिको हेतुः स्यात्~। आत्मसमवेतस्यैव वीचीतङ्गन्यायेन मूलप्रत्यभिज्ञानोपपत्तौ विपक्षे बाधकाभावाच्च~। तस्माद्यथाव्याख्यातमेवास्तु~।
\end{sloppypar}

{\knu न दिक्कालमनसाम्}, विशेषगुण इति शेषः~। तत्र हेतुः {\knu श्रोत्रग्राह्यत्वाद्} इति~। अत्र श्रोत्रेति स्वरूपकथनमात्रम्, ग्राह्यत्वात् प्रत्यक्षत्वाद् इत्येव हेतुः, गन्धादिवदिति~। आत्मगुणत्वे शब्दस्य मनसः श्रोत्रस्य साङ्कर्यप्रसङ्गः~। दिक्कालमनोगुणत्वे च शब्दस्य विशेषगुणत्वव्याघातः~। हेत्वन्तरमाह \textendash\ {\knu वैशेषिक$^6$गुणभावाच्चेति}~। दिक्कालमनसामिति वाक्यशेषः~। प्रयोगस्त्वयम् 'शब्दो न दिक्कालमनसां गुणः, विशेषगुणत्वाद्, गन्धादिवद्' इति~। तदूगुणत्वे शब्दस्य तेषां तत्त्वव्याघातः~। धर्मिग्राहकेण प्रमाणेन तेषां विशेषगुणप्रतिक्षेपादिति~।

ननु यदि नाम शब्दः पृथिव्यादीनामष्टानां गुणो न भवति मा भूत्, आकाशं तु नवममस्तीति किं प्रमाणम् ? अत आह \textendash\ {\knu पारिशेष्याद् गुणो भूत्वा आकाशस्याधिगमे लिङ्गमिति}~। अयमर्थः शब्दस्तावद् द्रव्याश्रितो गुणत्वाद्रूपवदिति सामान्यतो द्रव्यत्वसिद्धौ शब्दाश्रयत्वेनाकाशसञ्ज्ञायां विनिवेशितायां तदाकाशं पृथिव्यादिभ्यो

\blfootnote{I वीणाद्याश्रयो \textendash\ कि.~। 2 द्वारेण \textendash\ कि~। 3 ते च आत्मस्था ग्राहकस्था \textendash\ पा० ४. पु.~। 4 ग्राहकेऽसमवायात् \textendash\ पा. ४. पु. क, $\rightarrow$ $\leftarrow$ 'जे' पुस्तके एतच्चिह्वान्तर्गतः पाठो नास्ति~। 5 इदं मतं न्यायकदल्यां दृश्यते~। 6 वैशेषिकगुणभावात् \textendash\ पा. ३. पु, विशेषगुणभावाच्च \textendash\ कि, विशेषगुणाभावाच्च \textendash\ जे~।\\ १०}

\newpage
\noindent
${}^1$भिद्यते शब्दाश्रयत्वात्~। 'यः पुनर्न पृथिव्यादिभ्यो भिद्यते नासौ शब्दाश्रयो यथा पृथिव्यादय एव~। न च नायं शब्दाश्रयः, तस्मादितरेभ्यो भिद्यते~। यथा च पृथिव्यादिषु शब्दो न समवैति तथोक्तमधस्ताद् इति निरवद्यम्~। सोऽयं परिशेषो व्यतिरेकः, तदिदमुक्तं {\knu गुणो भूत्वेति}~।

\hangindent=2cm {\knu (६३) शब्दलिङ्गाविशेषादेकत्वं सिद्धम्~। तदनुविधानादे$^2$कपृथक्त्वम्~। विभववचनात्परममहत्परिमाणम्~। शब्द ${}^3$कारणवचनात् संयोगविभागाविति~।}

\begin{sloppypar}
एवमाकाशस्य स्वरूपे सिद्धे शब्दगुणत्वे च गुणान्तराण्यस्यैवाह \textendash\ {\knu शब्दलिङ्गाविशेषादिति}~। यद्यपि द्रव्यत्वादेव सामान्यतः सङ्ख्यायां सिद्धायामेकत्वमपि सिद्धम्~। नह्येकत्वमन्तरेणासमवायिकारण$^4$निरपेक्षं द्वित्वादिकमपि सिध्यति~। भवितव्यं च तत्र द्वित्वादिना, पृथिव्यादीनां नवानां पञ्चानां त्रयाणामि$^5$र्त्यादेर्व्यवहारस्य बाधकं विना सङ्ख्यादि$^6$निमित्तकत्वात्~। अन्यथा पृथिव्यादिष्वपि सा न स्यात्~। तथापि शब्दाश्रयस्य द्रव्यस्यानेकत्वव्यवच्छेदेन ${}^7$नियतमेकत्वं साधयितुमयमारम्भः~। तच्च वैभवे सति शब्दलिङ्गमविशिष्टं सत् साधयति~। सर्वेषामेव शब्दानां तदे$^8$काश्रयतयैवोपपत्तौ आश्रयान्तरकल्पनायां कल्यनागौरवप्रसङ्गात्~। श्रोत्रव्यवस्थाप्यदृव्ष्ट्यवस्थयैवोपपद्यत इति वक्ष्यमाणत्वादिति~। {\knu दनुविधानात् पृथक्त्वमिति~।} एतदपि पूर्ववत्~। द्रव्यत्वेन सामान्यतः परिमाणसिद्धौ सत्यामाह \textendash\ {\knu विभववचनादिति~।} 'विभवान्महानाकाशस्तथा चात्मा (वै. सू. ७ \textendash\ १ \textendash\ २२) इति सूत्रकारवचनात्~। विभवो विभुत्वं र्सर्वैर्मूर्तिमद्भिः सह संयोगः; तस्मात् परममहत्परिमाणम्~। ${}^9$परिमितस्य सर्वसंयोगानुपपत्तेः~। 'परममहद्' इति भावप्रधानो निर्देशः~। अथवा परमहतो द्रव्यस्य परिमाणं${}^10$षष्ठीतत्पुरुषः~। सामानाधिकरण्ये हि परममृहापरिमाणमिति स्यात्~। अथ विभुत्वस्य कुतः सिद्धिः ? स्पर्शशून्यत्वे सति प्रत्यक्षगुणाश्रयत्वाद् आत्मवत्~। अन्यथा परमाणुत्वे प्रत्यभ्नगुणाश्रयत्वविरोधात्, अवयवित्वे स्पर्शशूत्यत्वशिरोधात्~। शब्दकरणवचनात् संयोगविभागौ~। 'संयोगाद्विभागा$^11$च्छब्दाद्वा
\end{sloppypar}

\blfootnote{I व्यावर्त्यते \textendash\ कि~। 2 पृथक्त्वम् \textendash\ कि, जे; दे~। 3 कारणत्व \textendash\ मु. भा~। 4 कारणानपेक्षं \textendash\ पा. ४. पु~। 5 मित्यादि \textendash\ पा. ४. पु~। 6 निमित्तत्वात् \textendash\ पा. ४. पु~। 7 नियमेनैकत्वं \textendash\ पा. ४. पु~। 8 अभिन्नाश्रय \textendash\ पा. ४. पु~। 9 अपरिमितस्य संयोगानुपपत्तेः \textendash\ पा. ४. पु~। 10 षष्ठीसमासः \textendash\ कि~। II शब्दाच्च \textendash\ सू. पा. शब्दाच्च तस्य निष्पत्तिरिति \textendash\ पा. ४. पु~।}

\newpage
\noindent
शब्दनिष्पत्तिः' (वै. सू. २ \textendash\ २ \textendash\ ३१) इति सूत्रकारवचनात् संयोगविभागौ~। यद्यपि द्रत्यत्वेनैव संयोगविभागयोः सिद्धिः तथापि विपक्षे बाधकं वक्तव्यमिति तदुच्यते~। यदि संयोगविभागौ नभसि न स्याताम्, असमवायिकारणभावाच्छब्दो ${}^1$नोत्पद्येत~।

\hangindent=2cm {\knu (६४) अतो गुणवत्त्वादनाश्रितत्वाच्च द्रव्यम्~। समानासमानजातीयभावाच्च नित्यम्~॥}

यद्यपि धर्मिग्राहकादेव प्रमाणाद् द्रव्यत्वमस्यसिद्धम्, तथापि ${}^2$सङ्ख्यादिना गुणेनापि तत्साधयितुं शक्यमिति तदप्याह \textendash\ अत इति~। गुणवतोप्यद्रव्यत्वे तल्लक्षणत्याघातादिति~। अनाश्रितत्वाद् इति~। ${}^3$आश्रयैकस्वभावत्वादित्यर्थः, अन्यथा समवायेन व्यभिचारः स्यादिति~। आश्रयैकस्वभावस्याप्यद्रव्यत्वे गुणत्वाद्यापत्तौ स्वरूपव्याघातादिति~। यद्यप्येकत्वादेव नित्यत्वाकार्यत्वेसिद्धे; अनित्यत्वे कार्यत्वे च ततः पूर्वं पश्चाजायमानाः शब्दा आकाशान्तरे वर्त्तेरन्; तस्मादेकत्वादेव पूर्वापरकोटिविकलं नभः, तथापि प्रपञ्चार्थमाह \textendash\ समानेति~। तथा हि नास्यारम्भकं नभ एव तेनैव तस्यानारम्भात्~। नापि नभोऽन्तरम्, असत्त्वात्~। नापि विजातीयं पृथिव्यादि कारणगुणपूर्वक्रमेण, कार्ये रूपादिमत्त्वप्रसङ्गात्~। नाप्यात्मदिक्कालमनांसि स्पर्शशून्यतया तेषामनारम्भकत्वाद्, इति शब्दाश्रयतया परार्थत्वमस्योक्तम्~।

{\knu (६५) ${}^4$सर्वप्राणिनां शब्दोपलब्धौ निमित्तं श्रोत्रभावेन~॥}

सम्प्रति तदुपलम्भकतया परार्थतामाह \textendash\ {\knu सूर्वप्राणिनामिति~।} तर्हि सर्वेषां सर्वशब्दोपलब्धिः स्यात्, तस्य सर्वसाधारणत्वाद् व्यापकत्वाच्च इत्यत आह \textendash\ {\knu श्रोत्रभावेनेति~।}

\hangindent=2cm {\knu (६६) श्रोत्रं पुनः ${}^5$श्रवणविवरसञ्ज्ञको नभोदेशः शब्दनिमित्तोपभोगप्रापकधर्माधर्मोपनिबद्धः~। तस्य ${}^6$च नित्यत्वे सत्युपनिबन्धवै ${}^7$कल्याद् बाधिर्यम्~॥}

ननु किमिदं श्रोत्रम् ? नहि पृथिव्यादेर्घ्राणादिवदाकाशस्यापि किञ्चित्कार्यमस्तीत्यत आह \textendash\ {\knu श्रोत्रं पुनरिति}~। अत्र 'पुनः' शब्देन पृथिव्यादिकार्येभ्यो घ्राणादिभ्यो व्यवच्छितत्ति~। श्रवणे कर्णशष्कुल्यौ तयोर्विवरं तत्सञ्ज्ञको नभोदेशः~। अव्याप्यवृत्ति संयोगोपचारात् देशव्यपदेशः~। ननु$^8$ स बधिरस्याप्यस्तीत्यत आह \textendash\ {\knu शब्देति}~। शब्देन निमित्तेन

\blfootnote{I नोपपद्येत \textendash\ पा. ४. पु~। 2 सङ्ख्यादिगुणत्वेनापि \textendash\ क~। 3 आश्रयैकस्वभावात् \textendash\ पा. ४. पु~। 4 सर्वत्राणिनां च दे, कं, कि~। 5 श्रवणको \textendash\ दे~। 6 तस्य नित्यत्वे \textendash\ जे~। 7 वैकलत्ये \textendash\ दे~। 8 ननु बधिरस्यास्ती$^\circ$\textendash\ कि;~।}

\newpage
\noindent
विषयतया य उपभोगः सुखदुःखानुभवः तत् प्रापकाभ्यां तत्सम्पादकाभ्यां धर्माधर्माभ्यामुपनिबद्धो विशिष्ट इत्यर्थः~। एवं सति च विशेष्यस्य सद्भावेऽपि विशेषणांशस्य वैकल्यात् कस्यचिद् बाधिर्यम्~। यथा शरीरसद्भावेऽप्युपभोगहेत्वदृष्टवैकल्यात् सुखज्ञानाद्यनुपपत्तिरित्यत आह \textendash\ {\knu तस्य चेति}~। उपलक्षणं चैतत्, कर्णमलादिना ${}^1$शुषिरव्याधातादपि बाधिर्यं भवति; अन्यथा मेषजादिप्रयोगवैयर्थ्यापत्ति$^2$रिति~॥

\hangindent=2cm {\knu (६७) कालः परापरव्यतिकरयौग$^3$पद्यायौरगपद्यचिरक्षिप्रप्रत्य यलि$^4$ङ्गः~। तेषां विषयेषु पूर्वप्रत्ययविलक्षणानामुत्पत्ता वन्यनिमि$^5$त्ताभावाद् यदत्र निमित्तं स ${}^6$कालः~।}

काललक्षणमाह \textendash\ {\knu काल} इति~। अत्र परापरेति भावप्रधानो निर्देशः~। परत्वापरत्वयोर्व्यतिकरः परापरव्यतिकरः~। दिक्कृताभ्यां परत्वापरत्वाभ्यां विपरीते परत्वापरत्वे इत्यर्थः~। ${}^7$परापरव्यतिकरश्च यौगपद्यादिप्रत्ययाश्चेति समासः, ते लिङ्गानि यस्य स तथोक्तः~। परत्वापरत्वयोरूत्पत्तिर्गुणे वक्ष्यते~। ते युवस्थविरयोर्वर्तमाने कालमनुमापयतः~। कथम् ? 'बहुतरतपनपरिस्पन्दान्तरितजन्मनि हि स्थविरे युवानवधिं कृत्वा परत्वमुत्पद्यमानमसमवायिकारणवत्' कार्यत्वाद् इति स्थिते च रूपादीनां च व्यभिचारे तपनपरिस्पन्दानां च वैयधिकरण्ये तदवच्छिन्नद्रव्यसंयोगोऽसमवायिकारणमिति विज्ञायते~। न ${}^8$चाविभुनो द्रव्यस्य संयोगस्तथा भवितुमर्हति व्यभिचाराद् इति तद्द्रव्यं विभु स्यात्~। न चात्माकाशौ तथा भवितुमर्हतो विशेषगुणवत्वात् पृथिव्यादिवदि$^9$त्याचार्याः~।

अत्रैवमालोचनीयम्; पृथिव्यादीनां संयोगस्यातद्धेतुत्वे विशेषगुणवत्वमप्रयोजकं सूर्या$^10$दिगतिप्राप्तिविरहस्योपाधेः सत्त्वात्~। अन्यथा मनसस्तथाभावप्रसङ्गात्~। विशेषगुणविरहिणस्तस्याकारणत्वेऽप्राप्तेरन्यस्य कारणस्याभावात्~। तस्माद्विशेषगुणवत्त्वाविशेषे पृथिव्यादीनाम्व्यापकत्वात् परत्वाद्यकारणत्वं व्यापकत्वाच्चात्माकाशयोस्तत्कारणत्वं भविष्यतीति को विरोधः ? विप्रकृष्टबुद्धिमपेक्षयोत्पद्यमानं विप्रकृष्टबुद्धिं निमित्तीकरोति~। अल्पतरसूरसञ्चरणापेक्षया बहुतरपनपरिस्पन्दान्तरितत्वं विप्रकर्षः~। न चैष स्थविर$^11$शरीरे स्वरूपतः सम्भवति,

\blfootnote{I शुषिरव्याघातात्तैरेव \textendash\ पा. ४. पु~। 2 रिति खं समाप्तमिति \textendash\ ४. पु~। 3 यौगपद्यचिर... जे~। 4 लिङ्गम् \textendash\ मु. भाः जे~। 5 निमित्तासम्भवद् \textendash\ व्यो (३४४) कि~। 6 काल इति \textendash\ दे~। 7 परत्वापरत्व \textendash\ पा. ४. पु~। 8 न चाविभुद्रव्यसंयोग पा. ४ पु~। 9 आचार्या व्योमशिवाचार्या इति किरणावलीप्रकाशे वर्धमानोपाध्यायाः~। 10 गतिविरहस्योपाधेः सत्त्वात् \textendash\ पा. ४. पु~। II शरीरस्य स्वरूपतः \textendash\ पा. ४. पु, क~।}

\newpage
\noindent
तपनसमवेतेन कर्मणाऽस्य सम्बन्धाभावात्~। अतो यद्द्वारेण संयुक्तसंयुक्तसमवायलक्षणया प्रत्यासत्त्या आदित्यगतिः स्थविरमुपावर्तते तद्द्रव्यमस्ति~। न च तत् पृथिव्यादि तथा भवितुमर्हति, अप्राप्तेर्व्यभिचाराच्च~। अतः सकलमूरत्त्द्रव्यसंयुक्तेन तेन भवितव्यम्~। $\rightarrow$ $^1$न चाकाशात्मानौ तथा भवतः, विशेषगुणवत्त्वात् पृथिव्यादिवदित्याचार्याः $\leftarrow$~। न$^2$न्वप्रयोजकमेतत्, अप्राप्त्युपाधिकत्वात्~। $\rightarrow$ अ$^3$न्यथा मनसस्तथाभावप्रसङ्गात्; विशेषगुणरहितस्य तस्याकारणत्वेऽप्राप्तेरन्यस्य कारणस्याभावात्~। तस्माद्विशेषगुणत्त्वाविशेषेऽपि पृथिव्यादीनामव्यापकत्वात् परत्वाद्यकारणत्वम्; व्यापतत्वाच्चात्माकाशयोस्तत्कारणत्वं भविष्यतीति को विरोधः $\leftarrow$~। न चाकाशस्य स्वप्रत्यासत्तिमात्रेण संयुक्तसमवायिनं धर्ममन्यत्र सङ्क्रामयितुमसमर्थत्वात्~। तथात्वे चैकत्र भेर्यामभिहतायां सर्वभेरीषु शब्दोत्पत्तिप्रसङ्गात् सूर्यक्रियावद् भेरीसंयोगस्यापि संयुक्तसंयुक्तसमवायाविशेषात्, कालस्य तु ${}^4$तत्स्वभावत्वाद् अयमदोषः~। असिद्धिदशायां प्रसङ्गस्याश्रयासिद्धेः, सिद्धिदशायां च बाधितत्वादात्मनोऽपि द्रव्यान्तरधर्मेषु द्रव्यान्तरावच्छेदाय स्वप्रत्यासत्त्यतिरिक्त्सन्निकर्षापेक्ष5त्वात्~। अन्यथा {\knu वाराणसी}स्थितेन नीलेन {\knu पाटलिपुत्र}स्थितस्य स्फटिकमणेरूपरस्नप्रसङ्गात्~। कालस्य तु विप्रकृष्टक्रियोपसङ्क्रमण$^6$क्ततयैव सिद्धेर्नाऽयं प्रसङ्ग ${}^7$इति सर्वमवदातम्~।

तदयं प्रमाणार्थः~। ' स्थविरादौ परत्वं तपनपरिस्पन्दप्रकर्षबुद्धिजन्यम्, तदनुविधायित्वात् पटकुविन्दवत्'~। 'ते च तपनपरिस्पन्दाः स्थविरशरीरप्रत्यासत्तिमपेक्षन्ते, स्वभावतोऽप्रत्यासन्नत्वे सति तदवच्छेदकत्वात् घटावच्छेदकमहारजनरागवत्'~। 'सा च प्रत्यासत्तिर्द्रव्यकृता, प्रकारान्तरासम्भवे सति ${}^8$प्रत्यासत्तित्वदेव $\rightarrow$ प$^9$टे महारजनप्रत्यासत्तिवत्'~। अन्यथा प्रत्यासत्त्यभावे सन्निकर्षविप्रकर्षयोरसम्भवादपेक्षणीग्रानुपपत्तौ परत्वाद्यनुत्पतिप्रसङ्गेन तद्व्यवहार$^10$विलोपप्रसङ्गः $\leftarrow$~। 'तच्च द्रव्यमितरेभ्यो भिद्यते, विशिष्टपरत्वाद्य$^11$नुमेयत्वात्'~। 'यत्पुनर्नेतरेभ्यो भिद्यते ततन्नैवं यथा ${}^12$पृथिव्यादि'~। एतेनापरत्वं लिङ्गतया व्याख्यातम्~।

यौगपद्यमेककालता, अयौगपद्यमनेकलालता, चिरत्वं दीर्घकालता क्षिप्रत्वमल्पकालता~। अत्र 'काल' शब्देन कालोपाधयः सूर्यगत्यादय उच्यन्ते, तेषां प्रत्ययैः कालोऽनुमीयते~।

\blfootnote{I, 3 $\rightarrow$ $\leftarrow$ एतचिह्वान्तर्गतौ पाठौ आवृत्ताविव दृश्येते सर्वेष्वपि आदर्श पुस्तकेषु वर्तमानत्वादत्र स्थापितौ~। 2 न चाप्रयोजकभिति पाठः सूचितः~। 4 तत्स्वभावात् \textendash\ पा. ४.~। 5 पेक्षितत्त्वात् \textendash\ कि.~। 6 सङ्क्रमेण तथैव सिद्धेर्नायं \textendash\ पा. २. पु~। 7 इति सर्वमवदातव्यम् \textendash\ पा. ४. कि~। 8 प्रत्यासन्नत्वात् \textendash\ पा. ४. पु~। 9 $\rightarrow$ $\leftarrow$ एतच्चिह्वान्तर्गतः पाठः ४. पु. नास्ति~। I0 रलोपप्रसङ्गः कि~। II ${}^\circ$नुमेयत्वाच्च पा. ४. पु~। I2 पृथिव्यादिरिति \textendash\ जे~।}

\newpage
ननु प्रत्ययाः कारणतया विषयतया वा कालं व्यवस्थापयेयुः ? न प्रथमः आत्मेन्द्रियादीनां तत्कारणत्वात्~। न द्वितीयः, तस्यातीन्द्रियतया ऐन्द्रियकप्रत्ययविषयत्वविरोधात्; अत आह \textendash\ {\knu तेषा}मिति~। विषयेषु घटादिषु युगपद्भवन्ति, युगपद्गच्छन्ति, युगपत्कुर्वन्ति, युगपद्वर्तन्ते इत्यादीनां पूर्वप्रत्ययविलक्षणानां द्रव्यगुणादिप्रत्ययविलक्षणानां द्रव्यादिप्रत्यय$^1$व्यावृत्तावप्यनुवृत्तेः, तदनुवृत्तावपि तेषां व्यावृत्तेर्विलक्षणत्वेन विनिश्चितानामुत्पत्तावन्यस्य निमित्तस्यासम्भवात् कालमन्तरेणेति शेषः~।

\begin{sloppypar}
एते युगपज्जायन्ते कुर्वन्त्यवतिष्ठन्ते विनश्यन्ते इत्यादीनामेकस्मिन् काले एकस्यां सूर्यगतावेकसूर्यगत्यवच्छिन्ना इत्यर्थः~। न चाप्राप्ताः सूर्यगतयो विषयान्तरमवच्छेत्तुमुत्सहन्ते~। न च स्वरूपेण तासां प्राप्तिरस्ति; तस्माद् विशिष्टप्र$^2$त्ययाद्यनुपपत्त्या विशेषणप्रापकं द्रव्यमुपकल्प्यते~। अन्यथा निमित्तमन्तरेण ${}^3$नैमित्तिकस्यापि प्रत्ययस्यासम्भवप्रसङ्गः~। वाच्यं न च सम्बन्धस्य प्रतीतौ विशिष्टप्रत्ययो विशिष्टप्रत्ययेनैव च तत्प्रतीतावितरेतराश्रयत्वमिति~। नहि सम्बन्धोऽत्र विशेषणमपि तु क्रिया, सा च प्रतीतैवेति ${}^4$सर्वं सुस्थम्~। विशेषणसम्बन्धो हि विशेष्यप्रतीतौ सत्तयैवोपयुज्यते, न तु स्वज्ञानेन~। अन्यथा समवायस्यातीन्द्रिथत्वाद् गुणकर्मसामान्यविशिष्ट$^5$प्रतीति{\knu र्वैशेषिक}नये दत्तजलाञ्जलिः स्यात्~। परेषामप्यहं शरीरीत्यादेरदृष्टस्य नियामकस्याप्रतीतौ विशिष्टप्रत्ययो न स्यात्~। तस्माद्विशिष्टप्रत्ययान्यथानुपपत्त्या विशेषणसम्बबन्धः प्रतीयते, न तु सम्बन्धाप्रतीतौ विशिष्टप्रतीतिरेव न सिध्यतीति$^7$ स्थितिः~।
\end{sloppypar}

\begin{sloppypar}
\hangindent=2cm {\knu (६८) सर्वका$^8$र्याणां चोत्पत्तिस्थितिविनाशहेतुस्तद्व्यपदेशात्~। क्षणलवनिमेषकाष्ठाकलामुहूर्तयामाहोरात्रार्धमास$^9$ मासर्त्वयनसंवत्सरयुगकल्पमन्वन्तरप्रलयमहाप्रलयव्य$^10$ वहारहेतुश्च~।}
\end{sloppypar}

यत्तत्साधर्म्यमुक्तं सर्वोत्पत्तिमतां निमित्तकारणत्वमिति तदाह \textendash\ {\knu सर्वकार्याणामिति}~। ऊत्पत्तिरात्मलाभः; स्थितिराविनाशात् क्रमवत्सहकारिसम्बन्धः विनाशः प्रध्वंसः; एतेषां हेतुः~। अत्र प्रमाणमाह \textendash\ {\knu तद्व्यपदेशादिति}~। तेन कालेनोत्पत्त्यादीनां व्यपदेशात् 'अद्योत्पन्नः श्वः परश्वो वा' इति, 'इदानीमत्रास्ति अद्य नष्टः श्वः परश्वो वा'

\blfootnote{I द्रव्य व्यावृत्ता \textendash\ पा० ४. पु.~। 2 प्रत्ययानुपपत्या~। विशिष्टप्रत्ययान्यथानुपपत्या 'क'~। 3 नैमित्तिकस्य व्यवहारस्या \textendash\ नुपपत्तेः 'क'~। 4 प्रतीतेनवेति सुष्ठ \textendash\ मु. कि~। 5 प्रतिपत्तिर्वैशेषिकनये \textendash\ पा. ४ पु~। 6 विशेषसम्बन्धः मु. किः जे~। 7 नियमः \textendash\ पा. ४ पु~। 8 कार्याणामुत्पति \textendash\ दे~। 9 मासायन \textendash\ दे~। 10 व्यवहार \textendash\ हेतुः \textendash\ कि~।}

\newpage
\noindent
इति व्यपदेश्यते यथा चैतत् तथोक्तम्~। ${}^1$व्यवहारहेतुमाह \textendash\ {\knu क्षणेति}~। उत्पन्नं द्रव्यं यावदगुणमुत्पद्यते, अन्त्यतन्तुसंयोगे यावन्न पटः, उत्पन्ने कर्मणि यावन्नं विभागस्तावत्कालः क्षणः~। अथवा सामग्री कार्यरहिता क्षण इति संक्षेपः~। अक्षिपक्ष्मकर्मैकं निमेष इत्यादि गणितशास्नानुपारेण ग्राह्यम्~। 

\hangindent=2cm {\knu (६९) तस्य गुणाः सङ्ख्यापरिमाणपृथक्त्व$^2$संयोगविभागाः~॥}

यद्यपि धर्मिग्राहकादेत प्रमाणात् सङ्ख्यादयः सिद्धास्तथापि आकाशादिन्यवच्छेदार्थ विशे$^3$षगुणाविरहमाविष्कर्तुं सामान्यगुणानाह \textendash\ {\knu तस्य गुणा} इति~।

\hangindent=2cm {\knu (७०) काललिङ्गाविशेषादेकत्वं सिद्धम्~। तदनुविधानात् पृथक्त्वम्~। कारणे काल इति वचनात्परममहत्परिमाणम्~। कारणपरत्वादिति वचनात् संयोगः~। तद्विनाशकत्वाद्विभाग इति~।}

\begin{sloppypar}
अत्राप्येकत्वादिसमस्त$^4$सङ्ख्यायोगे सिद्धे विशिष्टपरत्वाद्यनुमेयस्यानेक$^5$ त्वव्यवच्छेदेन निय$^6$ तमेकत्वं विवक्षन्नाह \textendash\ {\knu काललिङ्गाविशेषादिति}~। व्यापकतयैकस्यैव सर्वत्रं सूर्यादिगत्युपनायकतया सन्निकृष्टविप्रकृष्टबुद्धयुत्पादकद्वारेण विश्ववर्त्तिपरत्वापरत्वोत्पत्तिनिमित्तस्योपपत्तेरनेकत्वकल्पनायां कल्पनागौरवप्रसङ्गादित्यर्थः~। {\knu तदनुविधानात् पृथक्त्व}मित्येतदपि पूर्ववत् {\knu कारणे काल इति वचनात्} \textendash\ कारणे कालाख्या (वै. सू. २.२.९.) इति सूत्रकारवचनात् परममहत्परिमाणम् विशिष्टपरत्वाद्यसाधारणकारणद्रव्ये अविभाोर्विश्ववर्तिमूर्तगतपरत्वाद्युत्पत्तौ सूर्यादिगत्युपाध्युपनयनशक्तेः परममहत्त्वाभावे च ${}^7$वैभवानुपपत्तेः परममहत्त्वं कालस्य सिद्ध्यतीत्यर्थः~।
\end{sloppypar}

{\knu कारणपरत्वादिति वचनात् संयोग} इति \textendash\ 'कारणपरत्वात् कारणापरत्वाच्च परत्वापरत्वे' (वै. सू. ७. २. २२) इति सूत्रकारवचनात् इत्यर्थः~। कारणं परत्वादेः कालः, तस्य परत्वापरत्वे तदुत्पादकौ मूर्तसंयोगावुपचारात्~। {\knu तद्विनाशकत्वाद्विभागः} \textendash\ तस्य संयोगस्य उत्पन्नस्य सत्याश्रये विनाशको विभागस्तद्विनाशादेव सिद्ध इत्यर्थः~। एतेन धर्मिग्राहकप्रमाणस्य द्रव्यत्वस्य वा सङ्ख्यादिपञ्चकसाधकस्य सहायास्वर्का

\blfootnote{I व्यवहारान्तरे हेतुत्वमाह \textendash\ क~। 2 संयोगविभागः पूर्ववदेते सिद्धाः \textendash\ दे~। 3 विशेषगुणमाविष्कर्तुं \textendash\ पा. ४. पु~। 4 सङ्ख्यायोगित्वे सति \textendash\ पा. ४. पु~। 5 ${}^\circ$नुमेयस्य सतोऽनेकत्व \textendash\ पा. ४. पु~। 6 नियमेनैकत्वं \textendash\ पा. ४. पु.~। 7 विभवानुपपत्तेः \textendash\ पा. ३. पु~।}

\newpage
दर्शिताः~। तथाह्येकत्वाभावे द्वित्वाद्यसिद्धिरसमवादिकारणाभावात्, पृथक्तवाभावे एकत्वानुपपत्तिर्व्यापकनिवृत्तेः द्विपृथकत्वासिद्धिश्च~। परममहत्परिमाणाभावे वैभवानुपपत्तिः, संगोगाद्यसिद्धौ परत्वाद्यसिद्धिरसमवायिनिमित्तयोरभावात्~। विभागाभावे कार्यस्य संयोगस्य नित्यत्वप्रसङ्गः~।

{\knu (७१) तस्याकाशवद् द्रव्यत्वनित्यत्वे सिद्धे~॥}

{\knu तस्याकाशवदिति} \textendash\ यथाऽऽकाशस्य धर्मिग्राहकादेव द्रव्यत्वं ${}^1$सिद्धे गुणवत्वादपि सिद्ध्यति तथा कालस्यापि~। यथाऽऽकाशस्यैकत्वादेव नित्यत्वं सिद्धं समानासमानजातीयकारणाभावादकार्यतयापि सिद्धयति तथा कालस्यापीत्यर्थः~।

\hangindent=2cm {\knu (७२)काललिङ्गाविशेषादञ्जसै$^2$कत्वेऽपि~। ${}^3$सर्वकार्याणां प्रारम्भक्रियाभिनिर्वृत्तिस्थितिनिरोधोपाधिभेदाद् मणिवत्पा$^4$चकद्वानानात्वोपचार इति~॥}

स्यादेतत्; यद्येकः कालः, कथमतीतानागतव$^5$र्तमानमेदेन व्यवह्रियते ? नहि क्षणलवादिवद् ${}^6$वर्षादिवद्वा अतीतानागतादयोऽप्युपाधिभेदाः ${}^7$सन्ति, किन्तु त एव भावाः; भावे वा त एव सन्तु ${}^8$किमन्तर्गडुना कालेनेत्यत आह \textendash\ काललिङ्गेति~। अञ्जसा मुख्यया वृत्त्या आरम्भ इति \renewcommand{\thefootnote}{१}\footnote{सन्निकृष्टमिति \textendash\ यद्यपि प्रागभावमात्रस्य भविष्यत्ताव्यवहारेतुत्त्वम्, तथाऽपि प्रतियोगिव्याप्तधर्मिग्रहं विना प्रागभावो न गृह्यत इति दर्शनार्थं सन्निकर्षोऽप्युक्तः~। कि. प्र. व.~।\\ \rule{0.4\linewidth}{0.5pt}}सन्निकृष्टं प्रागभावमाह \textendash\ क्रियेति व्यापारः~। ${}^9$अभिनिर्वृत्तिरिति फलसिद्धिः~। निरोध इति विनाशः~। तेन प्रत्येकमनागतादित्रिकं धर्मधर्मिभेदात् त्रिविधमुक्तं भवति~। तथाहि 'भविष्यति पटः' इति पटस्य धर्मिणः प्रागभावास्तिता भविष्यता, तदुपलक्षितः कालो भविष्यन्~। 'पटं भावयिष्यति कुविन्दः' इत्यत्र धर्मिणो वर्तमानत्वेऽपि पटोत्पत्त्यनुगुणव्यापारप्रागभाववत्ता कुविन्दस्य भविष्यत्ता, तदुपलक्षितः कालोऽपि भविष्यन्~। एवमुत्पद्यते पटः स्मारयति पूर्वानुभवोऽनुभूतमर्थमनुभावयति पूर्वकर्म स्वानुरूपं फलभिति धार्मीणोऽभावेऽपि व्यापारसत्ता वर्तमानता'${}^10$तदुपलक्षितः कालोऽपि वर्तमानः~। 'अस्त्यात्मा', 'विद्यते व्योम' 'वर्तते पटः' इत्यादौ तु धर्मिसत्तयैव ${}^11$वर्त्तमानता, तदुपलक्षितः कालोऽपि

\blfootnote{I सिद्धमिति नास्ति \textendash\ ४. पु~। 2 कत्वे सति \textendash\ जे~। 3 सर्वकार्याणामारम्भ \textendash\ कि~। 4 पाचकवद्नानात्वोपचारः \textendash\ पाठोऽयं लेखक दोषात् 'जे' पुस्तके वायुप्रकरणे प्रविष्टः तत्रापि 'वा' नास्ति~। 5 वर्तमानतया \textendash\ पा. ४. पु.~। 6 'वर्षादिवद्वा' इति 'क' पुस्तके नास्ति~। 7 स्युः \textendash\ पा. ४. पु.~। 8 कृतमन्तर्गडुना \textendash\ क~। 9 अभिनिर्वृत्तिरपि \textendash\ पा. ३. पु.~। I0 तदुपहितः \textendash\ जे~। II विद्यमानता~।}

\newpage
\noindent
वर्तमानः~। 'एवमभूत्पटः,' 'अकार्षीत्पटम्' इत्यादौ धर्म्यवस्थानेऽपि फलसिद्धौ तदुत्पादानुगुणव्यापारप्रध्वंसस्थितिरतीतत्वम्, तदुपलक्षितः कालोऽप्यतीत उच्यते~। 'नष्टो घटः' इत्यादौ धर्मिस्वरूपनिवृत्तिरेवातीतत्त्वम्, तदुपलक्षितः कालोऽप्यतीत इति न; चैष\renewcommand{\thefootnote}{१}\footnote{ न चेति~। यदेह मध्याह्रस्तदोत्तरकुरुष्वर्द्धरात्रमिति वर्तमानयोः सामानाधिकरण्यमसति कालेऽनुपपन्नमिति तद्घटकः कालो मन्तव्य इत्यर्थः~। बौद्ध इति \textendash\ विशिष्टधीजनकत्वसम्बन्धः~। तदर्थसेवेति \textendash\ सैव सम्बन्धं विना नेति तदर्थं कालमपेक्ष्यत इत्यर्थः~॥ कि. प्र. व~।} व्यवहार उपाधिमात्रेण शक्यते; यदा भारतवर्षेषु मध्याह्नस्तदोत्तरेषु कुरुष्वर्द्धारात्रः' इत्यादौ सम्बन्धाभावात्~। बौद्धः सम्बन्धो भविष्यतीति चेत्; सत्यम्; तदर्थमेव निमित्तमनुसराम इत्यलं विस्तरेण~।

स एकोऽप्युपाधिभेदादनेकव्यवहारमातनोतीत्यत्र दृष्टान्तमाह \textendash\ {\knu मणिवदिति}~। यथा स्फटिकादिरेकरूपोऽपि ${}^1$उपाधिभेदान्नीलः पीत इत्यादि व्यवह्रियते तथा कालोऽपीत्यर्थः~। ननु किं मणौ पीतादिव्यवहारवत् कालेऽनागतादिव्यवहारो भ्रान्तः, इत्यत आह \textendash\ {\knu पाचकवद्वेति}~। न चौपाधिक इत्येतावतैव भ्रान्तः~। भाभूत् पाचकादिव्यवहारोऽभ्रान्तः, किन्त्वतस्मिंस्तदिति ग्रहः~। न चैवं प्रकृत ${}^2$इत्यर्थः~।

\hangindent=2cm {\knu (७३) दिक् पूर्वापरादिप्रत्ययलिङ्गा~। मूर्तद्रव्यमवधिं कृत्वा मूर्तेष्वेव द्रव्येष्वेतस्मादिदं पूर्वेण, दक्षिणेन, पश्चिमेन, उत्तरेण, पूर्वदक्षिणेन, ${}^3$दक्षिणापरेण, अ$^4$परोत्तरेण, उत्तरपूर्वेण, चाधस्तादुपरिष्टाच्चेति दश प्रत्यया यतो भवन्ति सा दिगिति; अन्यनिमित्तासम्भवात्~॥}

गु\renewcommand{\thefootnote}{२}\footnote{ननु गुणे कालकृतपरत्वापरत्वविजातीये परत्वापरत्वेऽपि दिशो लिङ्गे तत्कुतो नोक्ते इत्यत आह गुण इति~। कि. प्र. व~।}णे परत्वापरत्वलिङ्गं दिशः स्फुटी \textendash\ भविष्यतीत्यभिप्रायवांस्तदुपेक्ष्य पूर्वापरादिप्रत्ययलिङ्गतामाह \textendash\ दिगिति~। एतदेव स्फुटयति मूर्तद्रव्येति~। अमूर्त्तद्रव्यस्यावधित्वं नास्ति व्यापकत्वाद् अत उक्तं मूर्तद्रव्यमवधिं कृत्वेति~। व्यापकत्वादेव तेषां पूर्वापरादि प्रत्ययविषयत्वं नास्तीन्यत आह मूर्तेष्वेवेति~। पूर्वेण दक्षिणेनेत्यादौ तृ\renewcommand{\thefootnote}{३}\footnote{ननु पूर्वेणेति तृतीया न युक्ता कर्तृकरणयोस्तदर्थयोरभावादित्यत आह तृतीयाविधानमिति~। कि. प्र. व~।\\ \rule{0.4\linewidth}{0.5pt}}तीया\textendash

\blfootnote{I नीलपीताद्यु \textendash\ क~। 2 इति समाप्तः कालः \textendash\ इत्यथिकं ४. पु~। 3 दक्षिणपश्चिमेन \textendash\ दे~। 4 पश्चिमोत्तरेण \textendash\ दे~।\\ ११}

\newpage
\begin{sloppypar}
\noindent
विधाने 'प्रकृत्यादिभ्य उपसङ्ख्यान'मिति तृतीया पूर्वं दक्षिणमित्याद्यर्थे~। ननु ${}^1$प्रत्ययान्निमित्तं सिद्ध्यति, दि$^2$शस्तु कुतः सिद्धिरत आह \textendash\ {\knu अन्यनिमित्तासम्भवात्}~। अन्येषां ${}^3$गुणादीनां व्यभिचारात् प्र\renewcommand{\thefootnote}{१}\footnote{ प्रथमचरमेति \textendash\ यदपेक्षया सूर्योदयाचलसंयोगसन्निहिता दिक् सा प्राची~। यदपेक्षया सूर्यास्ताचलसंयोगसन्निहिता सा प्रतीची~। सन्निधिस्तु सूर्यसंयुक्तसंयोगात्पीयस्त्वमिति तत्संयोगघटकतया दिक् सिद्ध्यतीत्यर्थः~। तेऽपिति \textendash\ आदित्यसंयोगा अपि द्रव्यान्तरे घटादावनुपसङ्क्रान्ता असम्बद्धा इत्यर्थः~। कि. प्र. व~।\\ \rule{0.4\linewidth}{0.5pt}}थमचरमादित्यसंयोगो ह्यत्र निमित्तम्~। न च तेऽपि द्रव्यान्तरवर्तिनो द्रव्यान्तरेऽनु$^4$पसङ्क्रान्ता विशिष्ट$^5$प्रत्ययानुत्पादयितुमीशते; ततो भवितव्यं परमहता द्रव्यान्तरेण~। न चाव्यापकस$^6$म्बन्धमाश्रित्य सूर्यसंयोगा विशिष्टप्रत्ययानुत्पादयन्ति~। न चाकाशात्मानौ तथा भवितुरमर्हतः, पूर्वकादेव न्यायात्~। कालो भविष्यतीति चेत्, न, कालस्य क्रियामात्रोपाधिनिबन्धनव्यवहारहेतुत्वात्~। अन्यथा जगति वर्तमानैकत्ववत् प्राच्यादेरेकत्वप्रसङ्गात्~। अस्माकं ${}^7$प्राच्याऽन्येषां प्रतीचित्वात्~। न चैवं कालेऽपि; कालोऽप्यन्यधर्मोपाधिकं व्यवहारमन्यत्र कुर्वाणः प्रमातृनियममपेक्षते~। नन्वपेक्षत एव, योह्यत्र दिवसस्तस्यान्यत्र रात्रित्वात्~। सत्यम्, दिक्व्यवहारमन्तर्भाव्या, न तु वर्तमानादिस्तथा, वर्तमानस्य सर्वत्र वर्तमानत्वात्~।
\end{sloppypar}

ननु प्राच्याः सर्वान् प्रति प्राचीत्वं तस्माद्विलक्षणेन व्यवहारेण च विलक्षणमेव निमित्तमाकर्षणीयम्~। उपाधिरेव तथास्तु, इति चेत्, न; सूर्यसंयोगस्य साधारणत्वात्~। कथं तर्हि व्यवहारवैलक्षण्यं दिग्वादिनापि समर्थनीयमीति चेत्, न; उपनायकस्वभाववैलक्षण्येनैव तदुपपत्तेः~। दिशो ह्ययं स्वभावभेदो यत्प्रमातृभेदमाश्रित्य प8क्षधर्मोगनायकत्वं न त्वन्यस्येति~। कालस्यैवं स्वभावमेदः किं न कल्प्यत इति चेत्, न; क्रियां प्रत्यन्यथादर्शनात्~। अपि च अन्यधर्ममात्रमन्यत्रोप$^9$दध्यादिति कालस्य स्वभावः क्रियामात्रं वा ? आद्ये सर्वधर्मान् सर्वत्रोपसङ्क्रामयेदिति अतिप्रसङ्गः द्वितीये संयोगो$^10$पधायकमन्यत् कल्पनीयम्; अन्यथा प्राच्यादिव्यवहारानुपपत्तेः~। न च तत्क्रियासंयोगयोरेकजातीयत्वमस्ति, यतोऽन्यन्नोपदधातीत्युपपद्यते तस्मात् क्रियोपसङ्क्रान्तिरिव रविसंयोगोपसङ्क्रान्तिरपि नियतस्वभावं द्रव्यमुपस्थापयतीति सा$^11$धूक्तम् \textendash\ अन्यनिमित्तासम्भवादिति~। तत्सिद्धमेतद्; 'दिग्, इतरेभ्यो

\blfootnote{I प्रत्ययेन \textendash\ पा. ३. पु~। 2 विशेषस्य \textendash\ क~। 3 द्रव्यगुणादीनां \textendash\ पा. ४. पु 'क'~। 4 द्रव्यान्तरमसङ्क्रान्ता \textendash\ पा. ४. पु~। 5 विशिष्टप्रत्ययानुत्पादयन्ति \textendash\ कि~। 6 संयोग 'क'~। 7 प्राच्यैव क~। 8 पर \textendash\ क~। 9 पदर्शयेदिति \textendash\ क~। I0 ${}^\circ$गोपनायक \textendash\ क~। II साधु युक्तम् \textendash\ पा. ४. पु~।}

\newpage
\noindent
भिद्यते, उपाध्युपनयनद्वारेण पूर्वापरादिप्रत्ययहेतुत्वात्'; यत्रैवं न तदिरेभ्यो भिद्यते यथा ${}^1$पृथिव्यादि~।

\hangindent=2cm {\knu (७४) तस्यास्तु गुणाः सङ्ख्यापरिमाणपृथक्त्वसंयोगविभागाः कालवदेते सिद्धाः~।}

तद्गुणानाह \textendash\ {\knu तस्यास्त्विति}~। अत्रापि सङ्ख्यादीनां धर्मिग्राहकप्रमाणात् सिद्धौ द्रव्यत्वाद्वा विशे2षगुणव्यवच्छेदे तात्पर्यम्~। ${}^3$तर्कान् सूचयति \textendash\ {\knu कालवदेते सिद्धा} इति~। यथा काललिङ्गाविशेषादेकत्वं तथा दिग्लिङ्गाविशेषादेकत्वं दिशि~। यथा तदनुविधानात् पृथक्त्वं काले तथा दिश्यपि, यथा 'कारणे कालः' (वै. सू. ७ \textendash\ १ \textendash\ २५) इति वचनात् परममहत्त्वं तथा दिश्यपि~। यथा 'कारणपरत्वाद्' (वै. सू. ७ \textendash\ २ \textendash\ २२) इति वचनात् काले संयोगस्तथा दिश्यपि~। यथा तत्र तद्विनाशकत्वाद्विभागः तथा दिश्यपीत्यर्थः~।

\hangindent=2cm {\knu (७५) ${}^5$दिग्लिङ्गाविशेषादञ्जसैकत्वेऽपि दिशः प6रमर्षिभिः श्रुतिस्मृतिलोकसंव्यवहारार्थं मेरुं प्रदक्षिणमावतैमानस्य भगवतः सवितुर्ये ${}^7$संयोगविभागा लोकपाल परिगृहीतदिक्प्रदेशानामन्वर्थाः प्राच्यादिभेदेन दशविधाः सञ्ज्ञाः कृताः~। ${}^8$ततो भक्त्या दशदिशः सिद्धाः~। तासामेव दे9वतापरिग्रहवशात् पुनर्दश सञ्ज्ञा भवन्ति~। मा10हेन्द्री, वैश्वानरी, या11म्या, नैर्ऋती, वारूणी, वा12यव्या, चा13न्द्रमसी, ऐशानी ब्राह्मी नागी चेति~॥}

यद्यैकेव दिक् कथं 'दशदिशः' इति प्रसिद्धिरित्यत आह \textendash\ {\knu दिग्लिङ्गेति}~। अञ्जसा मुख्यया वृत्त्येति परमर्षिभिः सर्गादिप्रभवैर्मरीच्यादिभिर्दशविधाः सञ्ज्ञाः कृताः~। किमर्थम् ? श्रुतिस्मृतिलोकसंव्यवहारार्थम्~। '14प्राचीनप्रवणे वैश्वदेवेन 

\blfootnote{I पृथिवीवादि \textendash\ मु. कि~। 2 विशेषव्यवहारोच्छेदे \textendash\ पा. ४. पु~। 3 तर्ककर्ता सूचयति \textendash\ कि~। 4 'दिशि' कि, जे पुस्तकयोर्नास्ति~। 5 दिग्लिङ्गादञ्जसैक्त्वेऽपि \textendash\ दे~। 6 परममहर्षिभिः \textendash\ मु भा~। 7 संयोगविशेषा \textendash\ मु. भा; विभागस्तेषां \textendash\ दे~। विशिष्टसंयोगास्तद्वशाच्चप्राच्यादिव्यवहारः \textendash\ व्यो~। 8 अतो कः कि. व्यो (३५९)~। 9 परिग्रहात् \textendash\ मु. भा~। I0 तद्यथा महेन्द्री \textendash\ जे; यथा माहेन्द्री दे~। II यामी \textendash\ ५ पु~। I2 वायवी \textendash\ ५ पु~। I3 कौबेरी \textendash\ मु. भा; दे~। I4 प्राचीप्रवणेन \textendash\ कि~।}

\newpage
\noindent
यजेत,' न प्रतीची शिराः शयीत इति श्रौतो व्यवहारः~। '1आयुष्मान् प्राङ्मुखो भुङ्क्ते इत्यादि स्मार्तः~। 'दक्षिणेन याहि' इत्यादिर्लौकिकस्तदर्थम्~। ताः किं पारिभाषिक्यः ? न; इत्याह \textendash\ {\knu अन्वर्थाः} अनुगतार्थाः~। केषाम् ? भगवतः सवितुर्ये संयोगविशेषास्तेषाम्~। किं भूतस्य ? सवितुर्मेरुं प्रदक्षिणमावर्तमानस्य परितो भ्रमतः~। पुनः केषामन्वर्था इत्यत आह \textendash\ लोकपालपरिगृहीतानां दिक्प्रदेशानाम्~। ${}^2$लोकापालाः इन्द्रादयः~। कथं ताः सवितृसंयोगविशेषार्थानुगमेन दशविधाः सञ्ज्ञा इत्यत आह \textendash\ {\knu प्राच्यादिभेदेनेति}~। तथा हि \renewcommand{\thefootnote}{१}\footnote{प्रथमस्यामिति \textendash\ ननु प्राथम्यं रविसंयोगस्य विभुना अविभुना वा ? नाद्यः; सामान्यतोऽसम्भवात्~। न च तद्दिने प्रथमः संयोगः, रात्रौ तदव्यवहारापत्तेः~। नान्त्यः; प्रतीच्यामपि प्राचीव्यवहारापत्तेः~। अत्राहुः; सूर्योदयसन्निहिता दिक् प्राची~। तद्व्यवहिता दिक् प्रतीची, प्राच्याभिमुखस्थितपुरुष वामदेशावच्छिन्ना दिगुदीची~। तादृशपुरुषदक्षिणप्रदेशावच्छिन्ना दिग्दक्षिणा~। वामत्वदक्षिणे च शरीरावयववृत्तिजातिविशेषौ~। गुरुत्वासमवायिकारणक्रियाजन्यसंयोगाश्रया दिग् अधः~। अदृष्टवदात्मसंयोगजाग्निक्रियाजन्य संयोगाश्रया दिग् उर्ध्वमिति संक्षेपः *~। कि. प्र. व~॥

*सूर्यति \textendash\ सूर्योदयाचलसंयुक्तसंयोगाल्पीयस्त्वं सन्निधनं तद्भूयस्त्वं व्यवधानमत्रैति द्रष्टव्यम्~। वामावेति \textendash\ न च करत्वादिजातिभिः सङ्करभयेन वामत्वदक्षिणत्वयोर्नत्वे कथमनुगम इति वाच्यम्; दक्षिणत्वाभावkUटसमानाधिकरणशरीरावयववृत्तिजातित्वेन नानावामत्वानामनुगमात्~। एवं वामत्वाभावकूटमादाय नानादक्षिणत्वानुगमनात् केचित्तु तदुत्तरत्वं तदवच्छिन्नत्वेन व्यवह्रियत इत्याहुः~।\\ \rule{0.4\linewidth}{0.5pt}}प्रथममस्यामञ्चति सवितेति प्राची, अवागस्यामञ्चतीत्यवाची, प्रतीपमस्यामञ्चतीति प्रतीची, उदगस्यामञ्चतीत्युदीची~। एवं प्रागवाची, अवाक्प्रतीची, प्रत्युदीची, उदक्प्राची सूर्यापिक्षया; यतः पृथिवी साधस्तात् पृथिव्यपेक्षया; यतो नक्षत्राणि सोर्ध्वा ततो भक्त्योपचारेण दश दिशः सिद्धा इत्येकत्वाविरोध इत्यर्थः ? ननु लोकपालपरिग्रहनिबन्धनादिभिर्दशभिः सह भक्त्याऽपि विंशतिर्दिशः स्युरित्यत आह \textendash\ तासामेवेति~। या चास्माकं प्राची सैव माहेन्द्री~। निमित्तमात्रं भिद्यते न तु नैमित्तिकमित्यर्थः; उपलक्षणं चैतत्, पूर्वापरादिसञ्ज्ञा अपि तत्समानार्था न तु भिन्नार्था इति भक्तयापि दशैव दिश इ1ति~।

\hangindent=2cm {\knu (७६) आत्मत्वाभिसम्बन्धादात्मा~। तस्य सौक्षम्यादप्रत्य क्ष4त्वे सति करणैः शब्दाद्युपलब्ध्यनुमितैः श्रोत्रादिभिः समधिगमः क्रियते~। वा5स्यादीनामिव करणानां कर्तप्रयोज्यत्वदर्शनात् शब्दादिषु प्रसिद्ध्या च प्रसाधकोऽनुमीयते~॥}

\blfootnote{I आयुष्मान् \textendash\ क; आयुष्यं \textendash\ कि~। 2 दिक्पाला कि.~। 3 इति समाप्ता दिशः \textendash\ इत्यधिकम् \textendash\ ४. पु~। 4 ${}^\circ$क्षत्वेकरणैः \textendash\ दे; क्षत्वेऽपिकरणैः पा. ५. पु~। 5 वास्यादीनां \textendash\ कं~।}

\newpage
उद्देशक्रमानुसारेण दिग्लक्षणानन्तरमात्मलक्षणमाह \textendash\ {\knu आत्मत्वेति}~। आत्मत्वं नाम सामान्यविशेषः, तदभिसम्बन्धादात्मा इतरेभ्यो भिद्यते इति पूर्ववद् व्याख्येथम्~।

ननु शरीराद्यतिरिक्तमात्मानं नोपलभामहे~। 'गौरोऽहम्', 'स्थूलोऽहम्' 'ब्राह्मणोऽहम्', 'मानुषोहम्', 'गच्छाम्यहम्', 'जातोऽहम्', 'पितुः पुत्रः', 'पुत्रस्य पिता', 'ज्येष्ठस्य कनीयान्', 'कनीयसो ज्यायान्', इति गुणकर्मसामान्यविशेषजन्माद्यधिकरणमात्मानं चक्षुरादिनैव सर्वः प्रत्येति~। न त्वरूपं विभुमचलमनन्तमजमपूर्वमपरमात्मानं कश्चित् स्वप्नेऽप्यनुभवतीत्याश्रयासिद्धमात्मत्वम्~। यथा प्रसिद्धाश्रयं च पृथिव्यादिना गतार्थम्, तथा 'अहं जाने यते भुञ्जे' इति ज्ञानाद्यधिकरणतया मनसाऽऽत्मानुभूयत एव~। अहङ्कारस्य शरीरादिविषयत्वं चाग्रे निषेत्स्यते इत्यभिप्रायः~।

तथाप्यात्मत्त्रं स्वरूपासिद्धम्, परात्मनोऽप्रतीतेः, व्यक्तिभेदासिद्धौ सामान्यविशेषासिद्धेरित्यत आह \textendash\ {\knu तस्येति}~। सौक्ष्म्यं बाह्येन्द्रियग्रहणयोग्यताविरहः; तस्मादप्रत्यक्षत्वे तस्य करणैः श्रोत्रादिभिः समधिगमः क्रियते~। ननु करणान्यप्यप्रत्यक्षाणीत्यत आह \textendash\ {\knu शब्दाद्युपलब्ध्यनुमितैः}~। 'शब्दाद्युपलब्धयः करणसाध्याः, क्रियात्वात्, छिदादिवद्' इत्यनुमितैः~। कथं पुनः ${}^1$करणेरात्मा समधिगम्यत इत्यत आह {\knu वास्यादीनामि}वेति2 अनेन व्याप्तिर्दर्शिता~। प्रयोगस्त्वेवम् 'श्रोत्रादीनि कर्तृव्यापार्याणि, करणत्वाद्, वासीवत्~। करणत्वं चामीषां धर्मिग्राहकप्रमाणसिद्धम्~।

ननु कस्यचित्करणस्य वास्यादेः पाण्यादिकरणप्रयोज्यत्वदर्शनेऽपि यथा न सर्वं करणं करणप्रयोज्यमनवस्थादर्शनात्, तथा कस्यचित् क3रणस्य कर्तृप्रयोज्यत्वेऽपि न सर्वं तथा भविष्यतीति को विरोधः ? उच्यते; कारकेषु समानजातीयकारकापेक्षानियमो नेष्यते, अनवस्थाप्रसङ्गात्, विजातीयकारकापेक्षानियमस्तु इष्यत एव, अन्यथा लक्षणव्याघातात्~। चक्षुरादीनां च कर्त्रनपेक्षत्वे तत्साध्यक्रियाप्रतिसन्धानानुपपत्तिप्रसङ्गात्~। तस्माद्योऽसौ श्रोत्रादीतामेकोऽधिष्ठाता असावात्मेति~। प्रमाणान्तरमाह \textendash\ {\knu शब्दादिष्विति}~। प्रसिद्ध्या ज्ञानेन प्रसाधकः ज्ञाताऽनुमीयते~। तथाहि '4शब्दज्ञानं क्वचिदाश्रितम्, कार्यत्वाद्,

\blfootnote{अदृष्टेति \textendash\ न च क्रियामात्रस्यादृष्टवदात्मसंयोगजन्यतया वातादिजन्याग्निक्रियामादाय तिर्थगादावतिव्याप्तिरिति वाच्यम्~। जन्येत्यन्तेन स्वाभाविकत्वाभिधानात्~। तज्जन्यत्वेन तदसमवादिकारणत्वस्योक्तत्वाद्वा~। अग्निपदं तु वायुक्रियामादाय तिर्यगतिव्याप्तिवारणाय~। ननु प्राच्यादेः सूर्येण निर्वचने रात्रौ नदव्यवहारापत्तिरित्यत आह \textendash\ संक्षेव इति~। कि. प्र. व्या. भ.~॥\\ \rule{0.4\linewidth}{0.5pt}

I and 3 कारण \textendash\ कि. जे~। 2 पूर्व \textendash\ क~। 4 ज्ञानं 'क'~।}

\newpage
\noindent
ग1न्धवत्'~। कार्यत्वमेषामभूत्वा भावित्वलक्षणं प्रत्यक्षसिद्धम्, कार्यस्याप्यनाश्रितत्वे समवायिकारणव्यावृत्तावितरयोरपि व्यावृत्तेरकारणत्वप्रसङ्गः~।

\begin{sloppypar}
\hangindent=2cm {\knu (७७)न शरीरेन्द्रियमनसामज्ञत्वात्~। न शरीरस्य चैतन्यं घ2टादिवद्भूतकार्यत्वान्मृते चासम्भवात्~। नेन्द्रियाणां करणत्वात्, उपहतेषु विषयासान्निध्ये चानुस्मृतिदर्शनात्~। नापि मनसः, करणान्तरानपेक्षित्वे युगपदालोचनस्मृतिप्रसङ्गात् स्वयंकरणभावाच्च~। परिशेषादात्मकार्यत्वादात्मा समधिगम्यते~॥}
\end{sloppypar}

तथापि शरीरेन्द्रियमनसामन्यतमस्मिन्नाश्रितं भविष्यतीत्यत आह \textendash\ {\knu न शरीरेन्द्रियमनसाम्}, चैत3न्यमग्रे वक्ष्यतीति~। कुतः ? अ4ज्ञत्वव्याप्ततयैव तेषां सिद्धत्वादित्यर्थः~। एतदेव दर्शयति '{\knu न शरीरस्य च चैतन्यम्}, शरीरं न चैतन्याश्रयमित्यर्थः~। '{\knu घटादिवद्भूतकार्थत्वात्}' इति भूतत्यात्कार्यत्वाच्चेत्यर्थः~। अन्यथा असमर्थविशेषणो हेतुः स्यात्~। ननु चैतन्याश्रयत्वे कार्यत्वस्य भूतत्वस्य वा को विरोध इत्यत आह \textendash\ {\knu मृते चासम्भवादिति}~। यदि हि शरीरविशेषगुणश्चैतन्यं भवेद् रूपादिवद्यावद्द्रव्यभावि भवेत्~। चकाराद्यावत् पृथिव्यादिव्यक्तिवृत्ति भवेत्, पृथिव्यादिविशेषगुणस्य यावदद्रव्यभावित्वनियमात्~। पृथिव्यादिचतुष्टयविशेषगुणस्य च यावत्पृथिव्यादिव्यक्तिभावित्वात्~। अत्राप्यनियमे को विरोध इति चेत्, ${}^5$न, शरीरस्य कार्यतया उत्पत्तेः प्रागनुभवाभावेन संस्काराभावात्~। स्मृत्यनुसम्धानानुपपत्तिरित्येत6त्तर्कप्रदर्शनार्थमेव कार्यत्वस्योपादानमित्यवधातव्यम्~।

{\knu नेन्द्रियाणामिति} \textendash\ चैतन्यमित्यनुवर्तते~। तत्कुतः ? करणभावात् कुठारवदिति शेषः~। तर्कमाह \textendash\ {\knu उपहत्वेष्विति}~। अनुभवानुविधायितया तावदनुभवकार्या स्मृतिः, सोऽपि य7द्यन्तरा किञ्चिदतिशयं नादध्यात्, चिरध्वस्तो न स्मृतिं जनयेत्~। अनुत्पन्ननिरन्वयध्वस्तयोरविशेषात्, अतिशयमप्यादधानः स्वसमानाश्रयमेवादधीत, अन्यत्राधानेऽतिप्रसङ्गात्~। तस्मात् स्मृतिसंस्कारानुभवाः समानाश्रया इति नियमः~। तथा चेन्द्रियेषु चैतन्याश्रयेषु स्वीक्रियमाणेषु तेषामुपघातेऽन्धादिभिः पूर्वानुभूतानां रूपादीनामस्मरणप्रसङ्गात्

\blfootnote{I रूपवत् 'क'~। 2 घटवत् \textendash\ जे~। 3 चैतन्यमित्यग्रे भविष्यति 'क'~। 4 अज्ञत्वात् अज्ञत्वव्याप्ततग्रैव \textendash\ क~। 5 'न' कि. पुस्तके नास्ति~। 6 कर्तृप्रदर्शनार्थ \textendash\ 'क'~। 7 सोऽपि अन्तरा किञ्चिद्यदि विशेषं 'क'~।}

\newpage
\noindent
घटादीनां विषयाणां कुठारादीनां च करणानामचैतन्ये जीवनविरहो हेतुः~। तत्राप्याध्यात्मिकवायुविरहः, तथाऽत्रापि ${}^1$बाह्यधातूपग्रहाभावः~। अत्रार्थं तर्कं सूचयति \textendash\ {\knu विषयासान्निध्ये चेति}~। यदि हि विषयाश्चेतयेरंस्तेषामसन्निधानेऽनु $\leftarrow$स्मृतिर्नदध्यात्~। असन्निधिश्च नाशहेतुको द्रष्टव्यः~। न ह्यसत्यनुभवितरिस्मरणमुपपद्यते कारणाभावादिति भावः~। अनुभवस्य पश्चाद्भावेन स्मृतिरेवानुस्मृतिरुच्यते~।

अस्तु तर्हि नित्यत्वात् सर्वविषयत्वादभूतत्वाच्च मनसश्चैतन्यमत आह \textendash\ {\knu नापि मनस} इति~। अयम2भिसन्धिः, ज्ञानाश्रयतया मनोऽनुमीयमानं कर्तृ स्यात् न तु तत्करणम्, तथा च युगपदनेकविषयसन्निधौ अनेकचक्षुरादिकरणाधिष्ठानेन युगपदनेकान्यालोचनान्यनेकाश्च स्मृतीरारभेत~। न हि करणवत् कतृरयं धर्मो ${}^3$यदेकदैकामेव क्रियामभिनिर्वर्तयते~। तथात्वे वा व्यासङ्गो न स्यात्~। करणान्तरापेक्षायां च सञ्ज्ञाभेदमात्रम्~। अथ करणतयैव तदनुमाने ज्ञानाश्रयतया तदनुमास्यत इत्यत आह \textendash\ स्वयं करणभावाच्च~। सिद्धे हि चैतन्याश्रये कर्तरि करणं मनोऽनुमीयते, तथा च धर्मिग्राहकप्रमाणबाध इति भावः~।

दिक्कालाकाशानां च प्रसक्तिरेव नास्ति, तेषां बुद्ध्याश्रयत्वप्रतिक्षेपेणैव शब्दादिभिरुपस्थापनात्~। अतः पारिशेष्यात् प्रसक्तप्रतिषेधेऽन्यत्राप्रसङ्गात् शिष्यमाणतयाऽऽत्मकार्यत्वात् पृथिव्याद्यसमवायित्वे सति कार्यत्वादित्यर्थः~। तदयं प्रयोगः, 'ज्ञानं पृथि4व्याद्यतिरिक्तद्रव्याश्रितम्, तदनाश्रितत्वे सति कार्यत्वात्'; 'यत् पुनः पृथिव्याद्यतिरिक्तद्रव्याश्रितं न भवति तत्तदनाश्रितत्वे सति कार्यमपि न भवति यथा ग5न्धादिः', इति केवलव्यतिरेकीति~।

\hangindent=2cm {\knu (७८) शरीरसमवाथिनीभ्यां हिताहितप्राग्निपरिहारयोग्याभ्यां प्रवृत्तिनिवृत्तिभ्यां रथकर्मणा सा6रथिवत् प्रयत्नवान् विग्रहस्याधिष्ठाताऽनुमीयते~। प्राणादिभिश्च7~। कथम् ? शरीरपरिगृहीते वायौ विकृतकर्मदर्शनात् भस्त्राध्मापयितेव, निमेषोन्मेषकर्मणा नियतेन दा8रुयन्त्रप्रयोक्तेव, देहस्य वृद्धिक्षतभग्नसंरोहणादिनिमित्तत्वाद् गृह्पतिरिव, अभिमतविषयग्राहकसम्बन्धनिमित्तेन मनः कर्मणा गृहकोणेषु पेलकप्रेरक इव दारकः, नयन \textendash\ } 

\blfootnote{I तत्रापि तृप्त्यादेरभावः तत्रापि पुष्टेरभावः इत्यधिकं 'क' पुस्तके~। 2 मभिप्रायः 'क'~। 3 यदेकामेव \textendash\ कि. जे~। 4 पृथिव्याद्यष्टकातिरिक्तं \textendash\ क~। 5 शब्दादिः 'क'~। 6 सारथिरिव \textendash\ दे~। 7 श्चेति मु. भा~। 8 दारुयन्त्रेण \textendash\ दे~।}

\newpage
\indent
\hangindent=2cm {\knu विषयालोचनानन्तरं रसानुस्मृतिक्रमेण रसनविक्रियादर्शनादनेकगवाक्षान्तर्गतप्रेक्षकवदु1भयदर्शी ${}^2$कश्चिदेको विज्ञायते~॥}

स्यादेतत्; ${}^3$आत्मत्वहेतुसिद्ध्यर्थं परात्मप्रसाधनार्थं चायमारम्भः~। यथा च परात्मा परोक्षस्तथा ${}^4$तदूबुद्धिरपि परोक्षैव~। तथा च किं केन साध्यत इत्याशङ्क्याह \textendash\ {\knu शरीरसमवायिनीभ्या}मिति~। विग्रहस्याधिष्ठाताऽनुमीयते~। अधिष्ठातेत्यस्य विवरणं प्रयत्नवानिति संयोगीति शेषः~। लिङ्गमाह \textendash\ {\knu प्रवृत्तिनिवृत्तिभ्यामिति}~। प्रवृत्त्या निवृत्त्या चेत्यर्थः~। अनयोः प्रत्येकमेव लिङ्गत्वात् क्रियामात्रस्य चानैकान्तिकत्वमित्याशङ्क्याह \textendash\ {\knu हिताहितप्राप्तिपरिहारयोग्याभ्यामिति}~। हितं सुखसाधनम्, दुःखसाधनमहितम्~। तयोर्यथासङ्ख्यं प्राप्तिपरिहारौ, तद्योग्याभ्यामिति~। योग्यग्रहणं क्वचिदन्तरायसम्भवेन प्राप्तिपरिहारयोरसिद्धावपि चैष्टात्वस्य प्रयोजकत्वमादर्शयितुम्~। ${}^5$एवंभूते च प्रवृत्तिनिवृत्ती प्रायेण शरीरतदवयवसमवायिन्याविति शरीरसमवायिनीभ्यामिति स्वरूपमुक्तम्~। यथा च लतादीनामपि शरीरित्वं तथोक्तमधस्ताद् वक्ष्यते च~। अत्र दृष्टान्तो {\knu रथकर्मणा सारथिवदिति}~। ${}^6$अत्र प्रयोगः, 'विवादाध्यासितं शरीरं प्रयत्नवदधिष्ठितम्, हितप्राप्तियोग्यक्रियावत्त्वात्, अहितपरिहारयोग्यक्रियावत्वाद्वा रथवत्' विपक्षे प्रयत्नवद्व्यतिरेकेण प्रवृत्त्यनुपपतिप्रसङ्गो बाधकम्, तत्कार्यत्वादिति~।

{\knu प्राणादिभिश्चे}ति \textendash\ अनेन 'प्राणापाननिमेषोन्मेषजीवनमनोगतीन्द्रियान्तरविकाराः सुखदुःखेच्छाद्वेषप्रयत्नाश्चात्मनो लिङ्गानि' (वै. सू. ३ \textendash\ २ \textendash\ ४) इति सूत्र स्मारयति~। {\knu कथमि}ति \textendash\ केन प्रकारेणात्र व्याप्तिर्ग्राह्येत्यर्थः~। उत्तरं शरीरेति~। लिङ्गत्वमभिव्यनक्ति \textendash\ {\knu विकृते}ति~। शरीरपरिगृहीतः, शरीराभ्यन्तरचारी मनःसहच२ इति यावत्~। तस्मिन् विकृतं स्वभावसिद्धात् ${}^7$तिर्यक्पवनादन्यादृशं कर्म ऊर्ध्वगतिलक्षणम्; तस्य दर्शनादिति व्याप्तिग्राहकं प्रमाणमुक्तम्~। {\knu भस्त्राध्मापयितेति} दृष्टान्तः~। प्रयोगस्त्वेवम् 'जीवच्छरीरं भोक्तृप्रयत्नवदधिष्ठितम्, ${}^8$सम्मूर्च्छनाभावे सति विकृतक्रियपवनाश्रयत्वाद् भस्त्रादिवद्' इति~। अथवा 'प्रा9णाख्यो वायुर्भोक्तृप्रयत्नवत्प्रेरितः, असति मूर्च्छने विकृतक्रियावायुत्वात्, भस्त्रावायुवत्'~।

\blfootnote{I दुपदर्शी \textendash\ दे~। 2 पाठोऽयं 'जे' दे पुस्तकयोर्नारित~। 3 आत्महेतु \textendash\ क~। 4 परबुद्धि \textendash\ क~। 5 एवं प्रवृत्तिनिवृत्ती \textendash\ कि~। 6 अस्य \textendash\ जे~। 7 तिर्यग्गमना$^\circ$\textendash\ कि. क~। 8 सम्मूर्च्छनाद्यभावे \textendash\ क~। 9 पार्थिवो \textendash\ जे; शारीरो \textendash\ पा. ४. पु. क~।}

\newpage
लिङ्गान्तरं विवृणोति \textendash\ {\knu निमेषेति}~। अक्षिपक्ष्मणोः संयोगहेतुः कर्म निमेषः, विभागहेतुरुन्मेषः, तेनापि प्रयत्नवाननुमीयते~। नन्वेतद्वायुविकारजनितेन नयनस्पन्देनानैकान्तिकमत आह \textendash\ {\knu नियतेनेति}~। ${}^1$दृष्टनियमेन यादृङ्निमेषोन्मेषकर्म प्रयत्नवदन्वयव्यतिरेकानुतिधायि दृष्टं दृष्टान्तेन तादृशेनेत्यर्थः~। अत्र किमिदं तादृशत्वम् ? इच्छापूर्वकत्वमिति चेत्; न, तस्या अप्यसिद्धतया साध्यत्वात्~। हिताहितप्राप्तिपरिहारार्थत्वस्य विवक्षितत्वात्, इति चेत्, न, प्रवृत्तिनिवृत्तिभ्यामेव गतार्थत्वात्; सत्यम्,, प्रपञ्चार्थमयमुपन्यासः~। अत्रापि प्रयोगः, 'जीवच्छरीरं भोक्तुप्रयत्नवदधिष्ठितम्, विशिष्टनिमेषोन्मेषक्रियावत्त्वात् दारुयन्त्रवत्'~। 'विशिष्टं वा निमेषोन्मेषकर्म प्रयत्नपूर्वकम्, विशिष्टनिमेषोन्मेषकर्मत्वात् दारुयन्त्रनिमेषौन्मेषकर्मवत्~।

जीवनलिङ्गकमनुमानमाह \textendash\ {\knu देहस्येति}~। वृद्धिरवयवोपचयः, भग्नक्षतयोर्विघटितविश्लिष्टयोरस्थिच2र्ममांसभागयोः~। संरोहणं पुनः सङ्घटनम्, तयोर्निमित्ततयाऽधिष्ठाताऽनुमीयते; 'शरीरावयववृद्धिक्षतभग्नसंरोहणे प्रयत्नवन्निमित्तके, वृद्धित्वात् संरोहणत्वाच्च, गृहकुड्यवृद्धिवत्तत्सरोहणवच्चेति'~। सूत्रे च 'जीवन'शब्दो जीवनकार्ये वृद्धिसंरोहणे लक्षयति~। मुख्यतया हि स्वकर्मोपार्जितशरीरावच्छिन्नेनात्मना मनः संयोगो जीवनमिति~। क्वचित्संरोहणादि इत्यादिपदं श्रूयते, तेनाहारोपादानमलोत्सर्गौ गृह्येते~। तथाहि 'जीवच्छरीरं भोक्तृप्रयत्नवदधिष्ठितम्, अभिमतानभिमतोपादानोत्सर्गवत्त्वात्, यन्त्रधारागृहवदिति'~। न चैतद् वृक्षादिगतवृद्धयादिनाऽनैकान्तिकम्, तत्रापि भोक्तृवि4शेषाधिष्ठानस्येष्टत्वात्~।

लिङ्गान्तरमाह \textendash\ {\knu अभिमतेति}~। अभिमतो विषयो रूपादिः, तद्ग्राहकं करणं चक्षुरादि, तेन संयोगः सम्बन्धः, तस्य निमित्तं कारणं यन्मनः कर्म, तेनाऽऽत्मानुमीयते~। गृहकोणेषु व्यवस्थितो गृहकोणव्यवस्थितः; तमेव पेलकमुद्विश्य, पेलकस्य जतुगुलकस्य प्रेरको दारको बालक इवेति~। अयमाशयः, प्रयत्नज्ञानायौगपद्यान्मनस्तावत् प्रतिशरीरमेकमणु गतिमच्चेति सिद्धम्; तत्पक्षयित्वेदमुच्यते~। 'मनः प्रयत्नवदधिष्ठितम्, गुरुत्पद्रवत्वादिकारणान्तराभावे सति गतिमत्त्वाद्, दारकप्रेरितपेलकवदिति'~। अदृष्टोपग्रहादेव तत्र क्रिया स्याद् इति चेत्, ${}^5$न, क्वचिदेवमपि~। यद्यपि चैवमध्यात्मसिद्धिस्तथाऽपि 'अभिमतेत्यादिना इच्छानुविधायिनः कर्मणो लिङ्गत्वेन विवक्षितत्वात् प्रयत्नवदधिष्ठातृसिद्धिः~। यद्यपि च ज्ञातमिच्छया विषयीक्रियते, इच्छाविषयश्च प्रयत्नस्य विषयः स्यात्, मनस्त्वतीन्द्रियं सद्ज्ञातं

\blfootnote{I 'दृष्ट' इति 'क' पुस्तके नास्ति~। 2 चर्म \textendash\ कि~। 3 वृक्षादिना \textendash\ क~। 4 विशेषनिष्ठ \textendash\ क~। 5 'न' कि पुस्तके नास्ति~।\\१२}

\newpage
\noindent
कथमेवं भविष्यतीति; तथापि स्पर्शनेन्द्रियेण रसमलधातुदोषवहनाडीवत् मनोवहानां नाडीनामुपलम्भात् तद्गोचराविच्छाप्रयत्नावदूरविप्रकर्षेण तदपि ${}^1$व्याप्नुत इत्युच्यते अदृष्टस्य नियामकत्वाच्चेति~। तत्रैव तदा तत्प्रत्यक्षम्~। न हि जलाद्यभ्यवहरणहेतवो मलाद्युत्सर्गनहेतवो वा नाड्यश्चक्षुषोपलभ्यन्ते~। न वा तद्विषयेच्छा; न च प्रयत्नः; अनाद्यभ्यासवासनावशादृष्टविशेषनियतं तु स्पार्शनप्रत्यक्षमेव तत्र कारणमेषितव्यम्, तथा प्रकृतेऽपीति~।

इन्द्रियान्तरविकारं विवृणोति \textendash\ {\knu नयनेति}~। नयनस्य चक्षुषो विषयश्चिरबिल्वा2दिरूपः, तस्यालोचनं ग्रहणम्, तदनन्तरं रसस्यानुस्मृतिः, तदनन्तरं व्याप्त्यनुस्मृतिः तदनन्तरं तथाविधरसानुमानम्, इत्यनेन क्रमेणेन्द्रियान्तरस्य रसनस्य विकारो दन्तोदकसम्प्लवलक्षणः, तस्य दर्शनादुभयदर्शी अनुमीयते~। उभाभ्यामुभयं द्रष्टुं शीलमस्येत्युभयदर्शी; 'अनेकगवाक्षान्तर्गतप्रेक्षकवद्' इति दृष्टान्तः~। प्रयोगस्तु 'नयनादीन्द्रियमचेतनम्, नियतविषयत्वात् गवाक्षवत्'~। यद्वा 'चेतनो नियतविषयेभ्यो भिद्यते, प्रतिसन्धातृत्वात् प्राक् प्र3त्यग्गवाक्षदर्शी वद्' इति~। अन्यथा अन्येन दृष्टस्यान्येनास्मरणात् प्रतिसन्धानानुपपत्ताविच्छानुपपत्तेर्विकारानुपपत्तिः कारणाभावादिति भावः~।


ननु भस्त्रादिषु सर्वत्र प्रयत्नवानधिष्ठाता शरीरी उपलब्धः, तथा च यथा सिद्धदृष्टान्तबलेन विग्रहस्यापि तथाभूत एवाधिष्ठातानुमातुमुचितः~। अनवस्थाभयान्नैवमिति चेत्, तर्हि विग्रह एव प्रयत्नवानस्तु, कृतं ततोऽप्यधिकेनाधिष्ठात्रेति; न, प्रतिसन्धातुस्तथाभावात्, शरीरस्य च भेदेन तथाभावानुपपत्तेः; भूतत्वकार्यत्वादिना च प्रागेव तस्य चैतन्यप्रतिषेधात्, प्रतिषेत्स्यमानत्वाच्चेति~।

\hangindent=2cm {\knu (७९) सुखदुःखेच्छाद्वेषप्रयत्नैश्च गुणैर्गुण्यनुमीयते~। ते च न शरीरेन्द्रियगुणाः, कस्मात् ? ${}^4$अहङ्कारेणैकवा5क्यताभावात् प्रदेशवृत्तित्वादयावद्द्रव्यभावित्वाद् बाह्येन्द्रियप्रत्यक्षत्वाच्च~। तथा 'अहम्' श6ब्देनापि पृथिव्यादिशब्दव्यतिरेकादिति~॥}

एवं कार्येण कारणमनुमाय गुणैर्गुणिनमनुमिनोति \textendash\ सुखेति~। तथाहि 'सुखादयो द्रव्याश्रिताः, गुणत्वाद् रूपवद्' इति प्रयोगः~। अनाश्रितत्वे द्रव्यादन्यत्राश्रितत्वे वा

\blfootnote{I प्राप्तुत \textendash\ पा. ३. पु~। 2 बिल्वरूपः \textendash\ कि~। 3 प्रत्येकगवाक्ष \textendash\ पा. ३. पु~। 4 अहङ्कारेण सह \textendash\ जे~। 5 वाक्यत्वाभावात् व्यो. (४०६)~। 6 शब्देन \textendash\ जे~।}

\newpage
\noindent
गुणत्वव्याघातात् सिद्धं च गुणत्वमेषाम्; 'कार्यत्वे सत्यव्यापकत्वे सति वा एकेन्द्रियह्यत्वाद्रूपवत् शब्दच्च~। कार्यस्य द्रव्यत्वे निरवयवत्वामूर्तत्वव्याघातात्, कर्मत्वे संयोगविभागानपेक्षकारणत्वप्रसङ्गात्, सामान्यादित्वे स्वलक्षणव्याघाताद्' इत्यादयस्तर्काः ऊहनीयाः~।

ननु भवत्वमीषां गुण1त्वाद्द्रव्याश्रितत्वम्, तथापि शरीराश्रितत्वमिन्द्रियाश्रितत्वं वा भविष्यतीत्याह \textendash\ ते चेति~। {\knu 'कस्माद्'} इति शिष्याकाङ्कायाम्~। ${}^2$हेतुमाह \textendash\ {\knu अहङ्कारेणैकवाक्यताभावादिति}~। सामानाधिकरण्येना3वबोधादित्यर्थः~। योऽहं सुखसाधनं स्रक्चन्दनादिकमुपलभ्य तदुपादित्सुः प्रयतमानस्तदुपात्तवान् सोऽहं सुखी, योऽहं दुःखसाधनमहिकण्टकादिकमुपलभ्यापि न परिहृतवान् सोऽहं दुःखी~। योऽहं सुखं दुःखं वानुभूतवानस्मि सोऽहं त4त्साधनमिदानीमुपलभ्याऽऽदातुमिच्छामि द्वेष्मि, वा इत्यनेन क्रमेण प्रतिसन्धीयमानाः खल्विमे गुणाः दुःखसुखादयोऽ5नुमीयन्ते~। इदं च प्रतिसन्धीयमानत्वममीषामनेकाश्रयतया विरुध्यते, चैत्रमैत्रादिसुखादिषु प्रतिसन्धानादर्शनात्~। तथा च शरीरेन्द्रियाणामपि प्रत्यहं परिणतिभेदेन भिन्नत्वात् तद्गुणानाममीषामप्रतिसन्धानप्रसङ्गः; भिन्नाश्रयाणामपि च प्रतिसन्धाने चैत्रमैत्रादिगुणानामपि प्रतिसन्धानापत्तिः~।

ननु शरीरगुणा अपि प्रतिसन्धीयन्त एव, तद्यथा 'योऽहं गौरः स्थूलो ह्रस्व आसं सोऽहमिदानीं श्यामः कृशो दीर्घः इति~। न, तस्य भ्रान्तत्वात्~। प्रकृतमपि प्रतिसन्धानं भ्रान्तमेव भविष्यति किं नश्छिन्नमिति चेत्; न, स्मर्तुरभ्रान्तेः अन्येन दृष्टस्य चान्येनास्मरणात्~। अन्यथा अतिप्रसङ्गात्~। तस्मात् प्रतिसन्धातुर्देहादिभ्योऽन्यत्वे सिद्धे तदधिकरणत्वे च सुखादीनामनुभूयमाने प्रयोगः~। 'सुखादग्रो न शरीरेन्द्रियगुणाः, अहङ्कारेण समानाधिकरणत्वात् स्मृतिवद्' इति~। 

स्यादेतत्, प्रतिसन्धानबलेन स्मृतिबुद्ध्योर्देहाद्यनधिकरणत्वं सिद्ध्यतु, देहादीनां भेदेन प्रतिसन्धातुं स्मर्तुं चासमर्थत्वात्~। सुखादयस्तु गौरत्वादिवद् देहधर्मा एव श्रान्तिवशादहङ्कारेण समानाधिकरणतया प्रतिसन्धास्यन्त इति चेत्; न, ज्ञानेनभिन्नाधिकरणानां कार्यकारणभावनियमानुपपत्तेः~। नह्यन्यो जानात्यन्य इच्छति अन्यः प्रयतत इति सम्भवति, तथात्वे च चैत्रे द्रष्टरि मैत्रे चषितरि जैत्रो यतते, तस्मा6दैकाधिकरण्येनैवामीषा कार्यकारणभावनियमो नान्यथेति~। 

\blfootnote{I द्रव्याश्रितत्वाद् गुणत्वम् \textendash\ क~। 2 हेतूनाह \textendash\ पा ३. पु~। 3 नावितेधादित्यर \textendash\ जे; का~। 4 'तत्साधनमहिकण्टकाद्युपलभ्यापि न परिहृतवान्' इदानीम् \textendash\ जे~। 5 ऽनुभूयन्ते \textendash\ जे, क~। 6 तस्मादन्याधि \textendash\ क~।}

\newpage
\begin{sloppypar}
हैत्वन्तरमाह \textendash\ प्रदेशवृत्तित्वादिति~। देहाद्यव्यापकत्वादित्यर्थः~। प्रयोगस्तु 'सु1खादयो न स्पर्शविद्विशेषगुणाः, तदव्यापकत्वात्' 'ये पुनस्तद्विशेषगुणास्ते तद्व्यापकाः, यथा रूपादयः' इति व्यतिरेकः~। 'तत्संयोगवद्' इत्यन्वयः~। 
\end{sloppypar}

हेत्वन्तरमाह \textendash\ {\knu अयावद्द्रव्यभावित्वादिति}~। सत्येव स्पर्शवति द्रव्ये निवर्तमानत्वदित्यर्थः~। ये पुनः स्प2र्शवद्विशेषगुणास्ते यावद्द्रव्यभाविनो भवन्ति, यथा रूपादय इति व्यतिरेकः~। संयोगवदिति वा पूर्ववदन्वयः~। अनयोस्तर्कः, 'यदि सुखादयः स्पर्शवद्विशेषगुणा भवेयुः, न नियतबुद्धिकार्या भवेयुः, निथामकाभावात्'; विषयविषयीभावस्य नियामकत्वे ${}^3$चैत्रस्य बुद्ध्या मैत्रविषयिण्या मैत्रसमवेतस्य सुखादेरूत्पादनप्रसङ्गादिति~।

स्यादेतत्; तथापि ताभ्यां हेतुभ्यामिन्द्रियगुणत्वममीषां न निषिद्धम्, शब्देन व्यभिचारात्~। न हि शब्दो ${}^4$यावद्द्रव्यभावी, नापि श्रोत्रं व्याप्य वर्तते, निमित्ताभिमुख्येन तत्सन्निधिमपेक्ष्योत्पत्तेः~। $\rightarrow$ या5वति गगने पवनसम्बन्धः, तावत्येव शब्दो जन्यते नान्यत्र $\leftarrow$~। अन्यथा 'प्राच्यां शब्दः' 'प्रतीच्यां शब्दः' इति देशावच्छेदेन प्रत्यक्षा6नुपपत्तिरित्याशङ्क्य हेत्वन्तरमाह \textendash\ {\knu बाह्येन्द्रियाप्रत्यक्षत्वादिति}~। मानसप्रत्यक्षत्वादित्यर्थः~। 'न शरीरेन्द्रियगुणा इच्छादयः, मानसप्रत्यक्षत्वाद बुद्धिवत्' ज्ञानवैयधिकरण्येऽमीषामप्रत्यक्षत्वप्रसङ्गः~। ${}^7$तथापि प्रशत्यक्षत्वे चैत्रेण मैत्रसमवेतानाममीषा ${}^9$मिच्छादीनामुपलम्भापत्तिः गुणानाममीषां चक्षुरादिग्राह्यत्वेनान्धादिभिरनुपलम्भप्रसङ्गः~। मनोग्राह्यत्वे च तद्धर्माणामन्धादिभिरपि रूपाद्युपलम्भापत्तिः; इन्द्रियसामर्थ्यनियमात् रूपरसादिवद् विषयनियमो भविष्यतीति चेत्; न; मनसः सर्वविषयत्वात्, नियतबाह्यविषयत्वे च मनस्त्वव्याघातादिति~। 'अहं युवा, कामी, यते, सुखी, दुःखी वा', 'स्पर्शनसुखो वायुः', 'कर्णे कटु रटति', 'मनो मे दुःखितम्', 'मनसि मे हर्षः', 'पादे मे सुखम्', 'शिरसि मे वेदना' इत्यादिप्रत्ययात्~। शरीरेन्द्रियगुणत्वममीषां शङ्कास्पदमिति तदेव प्रतिषिद्धम्, अन्यगुणत्वे तु शङ्कैव नास्तीत्युपेक्षितान्~।

सम्प्रति पूर्वहेतूनां चेष्टादीनां 'रथकर्मणा सारथिवद्' इत्यादिदृष्टान्तेषु विशेषविरोधा10भासपरिहाराय 'प्रत्यक्षसिद्धं पृथिव्याद्यतिरिक्तं स्वात्मानमेव दर्शयिष्यामः'

\blfootnote{I बुद्धयादयो \textendash\ कि~। 2 पुनस्तद्विशेषगुणा \textendash\ पा. ३. पु~। 3 चैत्रमैत्रविषयिण्या \textendash\ क~। 4 यावद्द्रव्यं भवति \textendash\ पा. ३; पा. ४. पु~। 5 $\rightarrow$ $\leftarrow$ एतच्चिह्वान्तर्गतः पाठो 'जे' पुस्तके नास्ति~। 6 प्रत्ययानुपपत्तेः \textendash\ पा. ३. पु~। 7 तथात्वे \textendash\ पा. ३. पु~। 8 चाक्षुषत्वे \textendash\ जे~। 9 सुखादीनां 'क' I0 विरोधपरिहाराय \textendash\ पा. ३. पु~।}

\newpage
\noindent
इत्य1भिसन्धाय प्रत्यक्षमाह \textendash\ {\knu तथाहं शब्देनेति~।} 'अहं'शब्देन प्रत्यक्षमुपलक्षयति~। यथा ${}^2$परात्मानुऽमीयते तथा प्रत्यक्षतोप्यहं प्रत्ययेनात्मा विषयीक्रियत इत्यर्थः~। नह्ययमहं प्रत्ययो लिङ्गजः शब्दजो वा, तदनुसन्धानमन्तरेणोपजायमानत्वात्~। ननु च पृथित्यादिविषय एवायमहं प्रत्ययो भविष्यतीत्याशङ्क्य तर्कमाह \textendash\ {\knu पृथिव्यादिकाब्दव्यतिरेकादिति} पृथिव्यादिप्रत्ययवैयधिकरण्यादित्यर्थः~। "यदि ह्ययमहं प्रत्ययः पृथिव्यादौ भवेत्, 'अहं पृथिवी' 'अहमापः' इत्याद्याकारो भवेत् द्रव्यप्रत्ययवत्" न चैवम्~।

स्यादेतद् सामान्यतो माभूत् विशेषतस्तुभविष्यति तथा कश्चित्पार्थिवो गोघटादिर्नतु सर्वः, तथा कश्चित् पार्थिवो 'अहं मैत्रेयः' इति को विरोधः ? नः अहङ्कारस्य प्रतिसन्धातृविषयत्वात्, शरीरस्य च भेदेनाप्रतिस4न्धातृत्वादित्युक्तत्वात्~। तन्मूलाः परमाणवः स्थिरत्वात् प्रतिसन्धातृत्वेनाहङ्कारविषया भविष्यन्तीति चेत्, न; तेषामतीन्द्रियत्वात्~। अपि च शरीरगोचरत्वे तस्य बाह्येन्द्रियजन्यत्वमापद्येत~। तथा च चक्षुरादिव्यापारोपरमे अहमिति प्रत्ययो न स्यात्~। मानसत्वे वाऽस्य शरीरस्य तद्गुणानां च गौरत्वादीनामपि मानसत्वप्रसङ्गात्~। न हि निर्गुणं द्रव्यमुपलभ्यते, तथा चान्धादीनां रूपादिवदहं प्रत्ययानुपपत्तिरिति~। तस्मात्परमार्थतः सामान्यतो विशेषतच्च सकलपृथिव्यादिप्रत्ययव्यतिरेकादहंप्रत्ययस्तदतिरिक्तं द्रव्यमालम्भते~। तथा चेष्टादीनां हेतूनामात्मकार्यत्वे सिद्धे, परत्रापि ततस्तत्सिद्धेः~। रथवदित्यादौ मच्छरीरवदित्यादिविपरिणामेनोदाहरणस्य निरुपद्रवत्वादिति सर्वमवदातम्~।

तदेत{\knu द्वैनाशिकैरु}पहस्यते~। बुद्धेर्हि कार्यस्य यद्युपादानकारणमात्मा~। विवक्षितः, स एव नो बुद्ध्यन्तरम्~। अथ सुखादीनां गुणानामाश्रयो द्रव्यम्, तदपि प्रक्रियामात्रम्, क्षणिकत्वसिद्धावस्य {\knu षट्पदार्थी}काव्यस्य बाललालनार्थत्वात्~।

तथा हि 'यत् सत् तत् क्षणिकम्, यथा घटः' संश्च विवादाध्यासितः शब्दादिरिति~। सत्त्वं तावद्यत्किञ्चिदस्तु, ${}^5$प्रकृते तावदर्थक्रियाकारित्वं {\knu बौद्धस्य} विवक्षितम्, तच्च भावानां प्रत्यक्षसिद्धे व्याप्तं च क्रमयौगपद्याभ्याम्~। न हि क्रमाक्रमाभ्यामन्यः प्रकारः शङ्कितुमपि शक्यते, व्याघातस्योत्कटत्वात्~। न क्रम इति निषेधादेव क्रमाभ्युपगमात्, नाक्रम इति निषेधादेवाक्रमोपगमात्~। ${}^6$अनभ्युपगमे च प्रति7क्षेपो व्याहन्यत इति~। तौ च क्रमाक्रमौ स्थिरे न सम्भवन्तौ अर्थक्रियामपि ततो व्यावर्त्तयतः~। वर्तमानार्थक्रियाकरणकाले

\blfootnote{I दृष्टान्तयिष्याम इति प्रतिसन्धाय \textendash\ पा. १. पु~। 2 परमात्मा \textendash\ पा. १. पु~। 3 इह \textendash\ कि~। 4 सन्धातृत्वमुक्तक्रमेण तन्मूलपरमाणवः क~। 5 प्रस्तुते सत्त्वं 'क'~। 6 अभ्युपगमे \textendash\ जे~। 7 प्रतिषेधो \textendash\ क~।}

\newpage
\noindent
हि अतीतानागतयोरप्यर्थक्रिययोः समर्थत्वे तयोरपि ${}^1$करणत्वप्रसङ्गात्~। असमर्थत्वे तु पूर्वापरकालयोरप्यकरणापत्तेः~। समर्थोऽप्यपेक्षणीयासत्निधर्नकरोति, तत्सन्निधेस्तु करोतीति चेत्; अथ किमर्थं सहकारिणामपेक्षा ? किं स्वरूपलाभार्थम् ? उतोपकारार्थम् ? अथ कार्यार्थं वा ? न प्रथमः, स्वरूपस्य स्वकारणाधीनस्य ${}^2$नित्यस्य वाऽनित्यस्य पूर्वसिद्धत्वात्~। न द्वितीयः, स्वयं सामर्थ्ये वा तस्यानुपयोगात्~। सामर्थ्ये हि स्वयमेव कार्यं कुर्यात् किं तेन ? असामर्थ्ये तूपकारसहस्रेणापि न कुर्यात् तथापि किं तेन ? अत एव न तृतीयः, स्वयं सामर्थ्ये उपकारिवत् सहकारिणामप्यनुपयोगादिति~। अनेकाधीनस्वभावतया कार्यमेवापेक्षत इति चेत्; न, तस्यास्वतन्त्रत्वात्~। स्वातन्त्र्ये वा कार्यत्वव्याघातात्~। तद्धि तत्साकल्येऽपि स्वातन्त्र्यादेव न भवेदिति~।

अथ केयमपेक्षा नाम ? किं तैः सहकरोतित्यन्वयः पर्यवसन्नः स्वभावभेदः ? अथ तैर्विना न करोतीति व्यतिरेकनिष्ठं स्वरूपमेव ? आहोश्चित् तैरुपकृतः करोतीत्युपकारभेदः ? न प्रथमः, स्वभावस्य तादवस्थ्यात्; नह्यस्य सहकारिव्यावृत्तौ स्वभावव्यावृत्तिः~। ननु यत एव सहकारिव्यावृत्तावस्य स्वभावो न व्यावर्तते, अत एव तैर्विना न करोति, कुर्वाणो हि तैः सहैव करोतीति स्वभावं जह्यात्~। स तर्हि स्वभावभेदः सहकारिसाहित्ये सति कार्यकारणनियतः सहकारिणो न जह्यात्; प्रत्युत पलायमानानपि गले ${}^3$पाशेन बध्वाऽकृष्याऽनयेत्, अन्यथा स्वभावहानिप्रसङ्गात्~। अत एव न द्वितीयः, अकर्तृस्वभावापरावृत्तेः~। अथ तद्विरहादकर्तृस्वभावः स्यात्, कालान्तरेऽपि स्वभावहेतुवशादुपसर्पतोऽपि सहकारिणः ${}^4$पराणुद्य न कुर्यात्~। तद्विरहाकर्तृशीलः खल्वयमिति~। अथ सहकारिषु सत्सु कर्तृस्वभावः, तद्विरहे त्वकर्तृरूप इत्यभिप्रायः, विरूद्धधर्माध्यासस्तर्हि स्वभावं भिन्द्यात्; न चायं कालमेदेन परिहर्तव्यः, धर्मिणोऽनतिरेकात्; अतिरेके चास्यौदासीन्यापत्तेः~। तथाहि सामथ्यासामथ्यभ्यामन्यो धर्मी स्वरूपेण न किञ्चिदेव स्यादिति~। उपकारपक्षस्तु अनेक5मुख्यनवस्थादुःस्थतया नोपन्यासस्यापि योग्यः~। तथाहि उपकारेऽपि कर्तव्ये सहकार्यान्तरापेक्षायामुपकारपरम्परापात इत्येका, उपकारेणाप्युकर्तव्यमित्यु6पकारपरम्परापात इत्यपरा; कर्तव्यस्य चोपकारस्य धर्म्यमेदे धर्म्यव क्रियेत, भिन्ने नु धर्मिणः किमायातम् ? नह्यस्मिन् जाते नष्टे वाऽन्यस्य स्वरूपं किञ्चिद् भवति, अतिप्रसङ्गात्~। अथ तेनापि ${}^7$तस्य किञ्चिदुपकारान्तरमाधेयमिति तृतीयाऽ8नवस्था~।

\blfootnote{I करणप्रसङ्गात् \textendash\ कि~। 2 अन्त्यस्य वा \textendash\ क~। 3 पाशादिनाऽऽकृष्य \textendash\ क, बद्ध्वाऽनयेत् कि~। 4 परमाणुवत् \textendash\ क्~। 5 मुखा \textendash\ क~। 6 तत्परम्परा \textendash\ क~। 7, 8 तस्य...अनवस्था इति पदद्वयं 'क' पुस्तके नास्ति~।}

\newpage
स्यादेतत्; क्षणिकोऽपि भावः किं निरपेक्षः सापेक्षो वा ? न प्रथमः; केवलाद्व1त्सरसहस्रेणापि कार्यस्यानुत्पत्तेः~। न द्वितीयः; अनुपकारापेक्षायामतिप्रसङ्गात्~। उ2पकारापेक्षायां च पूर्ववदनेकमुखानवस्थापातात्; तथा च पुञ्जात् पुञ्जोत्पत्तौ बीजमपि पवनपाथतेजोङ्कुरान् करोतीति पक्षे किं येनैव स्वभावेनाङ्कुरं क3रोति तेनैव पवनादीनन्येव वा ? न प्रथमः; कार्याणाममेदप्रसङ्गात्~। न द्वितीयः एकस्यापि क्षणिकस्य स्वभावभे4दापत्तेरिति~।

तदिदमशिक्षितोत्त्रासनम्~। साहित्यकरणलक्षणायां ह्यपेक्षायां क्षणिकस्य सापेक्षत्वादेव समर्थस्य सहकार्यव्यभिचारात्~। अन्यथा तु निरपेक्षत्वमेव, तैर्विनाऽस्याऽसत्वात्, सहकार्यनुपकार्यत्वाच्च स्थिरस्वभावस्तैर्विनाप्यस्तीति समर्थः किं न कुर्यादिति विशेषः~। एकस्यैवानेककार्यकारिस्वभावस्य स्वकारणादुत्पत्तेः~। कार्यभेदेन स्वभावमेदापादनस्यापि दूरनिरसनात्~। न च स्थैर्येऽप्येष न्यायः सम्भवति तदुत्पत्तेरनन्तरमेव कार्यकारणप्रसङ्गात्; तंदैव तस्य तत्स्वभावत्वादिति~। एतेनैकजातीयतैकदेशप्रसङ्गावपि निरस्तौ वेदितव्यौ~।

तस्माच्च क्रमयौगपद्ययोव्यपिकरयोर्व्यावृत्तेरक्षणिकाद् व्यावर्तमानाऽर्थक्रिया क्षणिके विश्राम्यतीति प्रतिबन्धसिद्धिः~। अन्वयेन वा यद्विरूद्धधर्माध्यस्तं तन्नाना, यथा शीतोष्णे; कालभेदेन विरूद्धधर्माध्यस्तश्च विवादास्पदीभूतो भावः~। न चायमसिद्धः, सामर्थ्यासामर्थ्ययोः प्रसङ्गतद्विपर्ययसिद्धत्वात्~। तथाहि यद्यदा यज्जननसमर्थं तत् तदा तत्करोत्येव, यथा सामग्री स्वकार्यम्~। समर्थश्चाऽयं भावः; वर्तमानार्थक्रियाकरणकालेऽतीताऽनागतयोरप्यर्थक्रिययोरतिप्रसङ्गः~। स्वभावहेतुः सामर्थ्यमात्रानुविधायित्वात् अर्थक्रियाकारित्वस्य~। यद्यदा यन्न करोति तत्तदा तत्र न समर्थम्; यथा शिलाशकलमङ्कुरे~। न चैष करोति, वर्तमानार्थक्रियाकरणकालेऽतीताऽनागते अप्यर्थक्रिये इति व्यापकाऽनुपलब्धिः प्रसङ्गविपर्यय इति~॥

अत्र संङ्क्षेपतः किञ्चिदुच्यते~। सत्त्वं तावदसिद्धव्याप्तिकतया ${}^5$व्याप्यत्वासिद्धम्; असिद्धप्रभेदोऽस्माकम्~। सन्दिग्धविपक्षव्यावृत्तिकतया सन्धिग्धानैकान्तिकप्रभेदो बौद्धस्य~। नहि कुर्वदूपत्वाक्षेपकरणधर्मत्वोत्पत्त्यनन्तरकारित्वादिशब्दवाच्यादन्यादृशं सामर्थ्यमुपादाय प्रसङ्गाद्यवकाशः~। तथाभूतं च सन्दिह्यते, न ह्येवंभूतत्वे भावस्य प्रत्यक्षमनुमानं वा सम्भवति~। सहकारिषु सत्सु करोति तैर्विना न करोतीत्ययमस्य स्वभावो दर्शनानतिक्रमेणाभ्युपगम्यते, अविरोधादुभयवादिसिद्धत्वाच्च~। ${}^6$त्वयापि ह्येतदवश्यमभ्युपगमनीयम्~। क्ष7गणिकोऽपि हि

\blfootnote{I वर्षशतेनाऽपि \textendash\ क~। 2 पाठोऽयं मु. कि. पुस्तके नास्ति~। 3 जनयति \textendash\ क~। 4 भेदोपपत्तेः \textendash\ क~। 5 व्याप्यत्वासिद्धप्रमेदोऽयम् \textendash\ क~। 6 त्वया ह्येत \textendash\ कि~। 7 क्षणिकोऽपि भाव \textendash\ कि~। क्षणिकत्वेऽपि \textendash\ क~।}

\newpage
\noindent
भावः कुर्वद्रूपः सहकारिषु सत्स्वेव सहकारिभिः सहैव करोतीति नियतस्वभावः, अन्यथा सहकारिणामसहकारित्वप्रसङ्गात्~। ततोऽन्वयनियमे तैर्विना न करोतीति व्यतिरेकोऽप्यवर्जनीयः~। अन्वयव्यतिरेकयोरन्यतरनियमस्यान्तरसिद्धिनान्तरीयकत्वादिति बौद्ध \textendash\ स्थितेः~। तथा च तैर्विना कृतस्य करणमापाद्यमानमापादकमेव व्याहन्तीति~।

नन्वत्रोक्तं तैः सहैव करोतीति स्वभावाभ्युपगमे सहकारिणो न जह्यात्; प्रत्युत पलायमानमपि गले पाशेन बद्धावाऽऽकृष्यानयेदिति सत्यमुक्तम्~। आत्मानमवीक्षमाणेन तत्प्रलपितम्, तवापि व्यतिरेकनियमे सहकारिणः स्वकारणादापततोऽपि निवार्य न कुर्यात्~। तैर्विना न करोतीत्ययमस्य स्वभावः~। अथ तैः सहैव करोतीत्ययमप्यस्य हि स्वभावः, एवं तर्हि विरोधः स्यात्~।

स्यादेतत् स्वकारणाधीनसन्निधयः सहकारिणस्तेषामुपनयापनययोरयमनीश्वर एव, यद्युपनिबन्धनेनोच्यते~। यदि सहकरिणो स्युः कुर्यादिति न2 तु ते तदानीं न सन्तीति~। तैर्विना तु समर्थो नास्त्येवभाव इति चेत्; नन्वस्माकमप्येवमस्तु~। स्वकारणसन्निधीनां सहकारिणामुपनयापनययोरयमनीश्वर, एव, यद्युपनिबन्धनेन तूच्यते यदि सहकारिणः स्युः कुर्यात्, न तु ते तदानीं स3न्तीति~। तैः सहैव करोति तैर्विना न करोतीत्येवायमस्य स्वभाव इत्यावयोः समोऽभ्युपगमः~। किन्त्वयं विशेषः~। तैः सहैव करोतीति योऽस्य स्वभावः स पश्चादस्ति नास्ति वा ? प्रथमे पश्चादपि कुर्यात्~। द्वितीये त्वस्मन्मतापात इति~।

अथ तवापि मते तैः सहैव कुर्वतोऽस्य तैर्विना न करोतीति यः स्वभावः स यद्यस्ति तदा नु कुर्यादेव~। नास्ति चेत्, व्यतिरेकव्यावृत्तावन्यव्यावृत्तिरिति स्वरूपहानिप्रसङ्गः~। स्वरूपमस्त्येव, तैर्विना भावमात्रं तु नास्तीति चेत्, ममाप्युत्तरकालं स्वरूपमस्ति, तत्साहित्यमात्रं तु नास्तीति समः समाधिः~। तस्मात् सहकार्यन्वयव्यतिरेकोपलक्षितमपेक्षितं च कार्येण स्वरूपमेव भावानां तादृशम्; तदेव सामर्थ्यपदार्थः; तदेव सहकार्यन्वयमपेक्ष्य ${}^4$कुर्वदित्युच्यते, तद्वयतिरेकमपेक्ष्य न कु5र्वदिति युक्तमुत्पश्यामः~॥

अथवाऽस्तु ${}^6$समर्थोऽऽसमर्थश्च भावस्तथापि क्षणिकपक्ष इव कालभेदेनाविरोधादसिद्धो विरुद्धधर्माध्यासः~। कथम् ? यदा हि क्षणिको भावः स्वकालस्थ एव एकां क्चिदर्थक्रियां करोतीति तदा तामेवार्थक्रियां कालान्तरेऽपि किं न करोतीति परिभावय~। असत्त्वादिति

\blfootnote{ I सत्स्वेव सहैव \textendash\ कि; कुर्वद्रूपः सहकारिभिः सहैव \textendash\ क~। 2 पाठोऽयं कि; जे पुस्तकयोर्नास्ति~। 3 सन्तीति स्यात् \textendash\ कि~। 4, 5 कुर्याद् \textendash\ क~। 6 समर्थः स्वभावः \textendash\ क~।}

\newpage
\noindent
चैत्; किमत्रासत्त्वेन ? नहि सत्त्वप्रयुक्तं ${}^1$करणम्, तथा सति देशान्तरेऽपि कुर्यात्~। तस्मात् तत्र काले देशे वाऽस्तु, मा भूत् यदि समर्थस्तदा कुर्यादिव~। अत एव स्वप्नावसानिकं विज्ञानं प्रहराद्यन्तरितमपि प्रसुप्तस्य जागराद्यज्ञानमुत्पादयतीति {\knu प्रज्ञाकरः}~। असार्मथ्यादेव न करोतीति चेत्, तर्हि तामेवार्थक्रियां प्रति समर्थोऽसमर्थश्चेत्यापतितम्~। को दोषः कालभेदादिति चेत्, स्थिरेऽप्येवमदोषोऽस्तु~। असत्त्वमेवासामर्थ्यमिति चेत्, न; उक्तोत्तरत्वात्~। सामर्थ्ये हि स्वकालस्थ एवान्यदा तामेवार्थक्रियां कुर्वाणो गीर्वाणशापेनापि नापहस्तयितुं शक्यते; किं पुनस्तवामुना मर्मपीडाजनितविक्वविक्रोशमात्रेण ? असामर्थ्ये तदापि न कुर्यात्~। किं च यस्य मते सत्त्वमधिकं नास्ति किन्तु स एव भावः, तदा कथं न विरूद्धधर्माध्यासः ? एकस्य विधिनिषेधस्वभावानुपपत्तेः~। 

काऽनुपपत्तिः ? साम्प्रतं यदयमस्ति कालान्तरे नास्तीति चेत्; सामर्थ्यासामर्थ्ययोरप्येवमस्तु~। प्रथमं समर्थस्यासमर्थस्य वा पश्चात्स्वभावविपर्ययः कुत आगत इति चेत्; प्रथमं सतोऽसतो वा पश्चात् स्वभावविपर्ययः कुत इति च तुल्योऽनुयोगः~। नाऽयं स्वभावविपर्ययः किन्तु स्वदेशकालस्थ एवान्यदेशकाल्योरसद्व्यवहारकारी स्वकारणादेवायात इति चेत्; ममापि तामर्थक्रियां प्रति तदानीं समर्थ एवान्यदा न समर्थ इति स्वकारणादेवायात इति तुल्यम्~। तस्मात् साऽर्थक्रिया तेन तदैव कर्तव्या नान्यदेति स्वभावनियमोऽप्युभयोर्दर्शनयोः समानः~। अन्यदा तु न कर्तव्यैवेति क्षणिकपक्षे, अन्यदैव कर्तव्यैवेति स्थिरपक्ष इति विशेषः~। न चान्यदाकरणतदातत्करणयोः कश्चिद्विरोधः, देशभेदेनार्थक्रियाभेदनियमवदुपपत्तेः~। तस्माद्वेशकालविषयमेदेन करणाकरणे सामर्थ्यासामर्थ्य च विरूद्धे एव न भवत इत्यन्यो विरूद्धधर्मोऽनुसर्तन्यः~। स यत्र नास्ति तत्र क्षणवदनवधेरपि गगनादेरभेदोऽध्यवसातव्य इति रहस्यम्~। विस्तरस्तु {\knu आत्मतत्त्वविवेके न्यायकुसुमाञ्जलौ} चानलसधियां सुलभ इति~॥

एवं स्थैर्यसिद्धौ न बुद्धिर्बुद्धेपरुपादानम्, गुणत्वात्~। न च गुणगुणिनोरभेदो विरूद्धधर्माध्यासादिति संक्षेपतः सर्वं सुस्थम्~॥

\hangindent=2cm {\knu (८०) तस्य गुणा बुद्धिसुखदुःखेच्छाद्वेषप्रयत्नधर्माधर्मसंस्कारसङ्ख्यापरिमाणपृथक्त्वसंयोगविभागाः~। आत्मलिङ्गाधिकारे बुद्ध्यादयः प्रयत्नान्ताः सिद्धाः~। धर्मा\textendash}

\blfootnote{I कारणम् \textendash\ क~। 2 विरुद्ध \textendash\ क~। \\ I3}

\newpage
\indent
\hangindent=2cm {\knu धर्मावात्मान्तरगुणानाम1कारणत्ववचनात्~। संस्कारः स्मृ2त्युत्पत्तौ कारणवचनात्~। व्यवस्थावचनात् सङ्ख्या~। पृथक्त्वमप्यत एव~। ${}^3$तथा चात्मेति वचनात् परममहत्परिमाणम्~। सन्निकर्षजत्वात् सुखादीनां संयोगः~। ${}^4$तद्विनाशकत्वाद्विभागः इति~॥}

स्यादेतत् किं सुखादय एवात्मनो गुणाः, ${}^5$यदेतैरेव गुणैर्गुण्यनुमातव्यः ? न त्वेवम्, बुद्ध्यदृष्टादीनामगुणत्वेऽन्यगुणत्वे वा बहु विप्लवेत~। सुखादीनामपि चात्मगुणत्वं न सिद्ध्येत्; संयोगाभावे तेषां तद्गुणानां कादाचित्कत्वानुपपत्तेरित्याशङ्क्याह \textendash\ {\knu तस्य गुणा} इति~। तेषां प्रत्यक्षसिद्धत्वेऽपि गौरवात् {\knu सूत्रकारानुम}तिमाह {\knu आत्मलिङ्गाधिकार} इति~। प्राणादिसूत्रे बुद्ध्यादयः प्रयत्नान्ताः सिद्धाः~। यद्यपि बुद्धिस्तत्र कण्ठरवेण नास्ति तथापि सुखादय एव स्वकारणतया तामाक्षिपन्ति~। न ह्यन्यसमवेतया इष्टादिबुद्ध्याऽन्यस्य सुखादिकमुत्पाद्यते; तथात्वे देवदत्ते चन्दनस्पर्शमनुभवति यज्ञदत्तस्य सुखमुत्पद्येत~। इन्द्रियान्तरविकाराच्चोभयदर्शित्वं विवक्षित्वा बुद्धिगुणत्वमात्मनो दार्शितमिति भावः~।

{\knu धर्माधर्मावात्मान्तरगुणानामकारणवचनादिति} \textendash\ विहितनिषिद्धाचरणेन धर्माधर्मौ जायेते~। चिरध्वस्तस्य कर्मणः कालान्तरभाविफलहेतुत्वानुपपत्तेस्तौ कुत्र जायेतामित्यपेक्षायां सूत्रकारेणोक्तम् 'आत्मान्तरगुणानात्मान्तरगुणेष्वकारणत्वाद्' (वै. सू. ६ \textendash\ १ \textendash\ ५) इति~। अस्यार्थः \textendash\ अन्यसमवेताभ्यां धर्माधर्माभ्यामन्यस्य सुखदुःखानुपपत्तेः~। उत्पत्तौ च कृतहानाकृताभ्यागमप्रसङ्गात्~। न चान्यसमवेतत्वेऽपि तत्कृतत्वेन नियमो भविष्यतीति वाच्यम्, ऋत्विगादिष्वरपि प्रसङ्गात्~। दृष्टप्रयोजनापेक्षित्वेऽपि तेषां कर्तृत्वात्~। न हि दृष्टान्तापेक्षया तत्कर्तृत्वं कर्मणो निवर्तते~। तदयं प्रमाणार्थः \textendash\ 'देवदत्तविशेषगुणप्रेरितभूतकार्यास्तदुपगृहीताश्च शरीरादयः, कार्यत्वे सति तद्भोगसाधनत्वाद् ${}^6$गृहवद्' इति~। जातेष्टिपितृयज्ञाभ्यामपि पुत्रपितृगामिफलसिद्धिः तदाश्रयापूर्वजननद्वारेणैवेत्यव्यभिचारः~। अन्यकृतेन कर्मणाऽन्यत्रापूर्वोत्पत्तिरिति शास्त्रसिद्धमृत्विगादिवदिति तदेव धर्माधर्मावात्मगुणौ मुनेरनुमतौ~।

\blfootnote{I ${}^\circ$मकारणवचनात् \textendash\ जे; दे~। 2 'स्मृत्युत्पत्तौ कारणवचनात्' इति दे पुस्तके नारित~। 3 तथात्मवचनात् चात्मेति \textendash\ त्वचात्मदे; वचनात् व्यो (४१०) 4 तद्विनाशाद् \textendash\ जे~। 5 एभिरेव गुणी \textendash\ क~। 6 गृहादिवद् \textendash\ जे~।}

\newpage
संस्कारे तदनुमतिमाह \textendash\ {\knu संस्कारः स्मृत्युत्त्पत्ताविति}~। अनेन 'आत्ममनसोः ${}^1$संयोगात् संस्काराच्च स्मृतिः' (वै. सूः ९ \textendash\ २ \textendash\ ६) इति सूत्रं स्मारयति~। न चान्यसंस्कारेणान्यत्रस्मृतिः, देवदत्तेनानुभूते यज्ञदत्तस्य स्मरणप्रसङ्गात्~। अतोऽनुभवसमानाश्रयः संस्कारः सिद्ध इत्यर्थः~।

{\knu व्यवस्थावचनात् सङ्ख्ये}ति '2नानात्मानो व्यवस्थातः' (वै. सू. ३ \textendash\ २ \textendash\ ३०) इति सूत्रं दर्शितम्~। द्रव्यत्वेन सामान्यतः सङ्ख्यायां सिद्धायां किं पृथिव्यादिवद् बहुत्वमुताकाशादिवदेकत्वमेवेति नानात्वे दर्शिते तन्नातरीयकतया बहुत्वसङ्ख्या दर्शितेत्यर्थः~।

अथ कोऽयं व्यवस्थानियमः ? प्रथमस्तावत् कश्चित् संसरति कश्चिदप3वृक्त इति~। यदि ह्येक एवाऽयमात्मा {\knu शुकः प्रह्लादो} वा मुक्त इति कोऽपरः संसरेत्~। न चेयता कालेन ${}^4$नापवृक्त एव कश्चित्, तथा सति नापवृज्येत च~। नह्यनादौ संसारे यन्न वृत्तं तद्वर्त्स्यतीति सम्भवति, न च बन्धमोक्षावेकस्य सम्भवतः, विरोधात्~।

अपरा च व्यवस्था, कश्चिदीश्वरः कश्चिदनीश्वर इति~। न चेश्वरो नाम जगति नास्ति, तत्सद्भावे यत्नस्य कृतत्वात्~। नाप्यस्मयादयः स्वात्मनि सार्वज्ञ्यं सर्वकर्तृत्वं वाऽवगच्छन्ति, उपदेशाद्यपेक्षत्वादिच्छाविधाताच्चेति~।

अथान्या व्यवस्था, कश्चित्सुखी कश्चिद्दुःखीत्यादि~। ननु 'पादे मे सुखम् पादान्तरे मे दुःखम्' 'शिरसि मे वेदना' इत्यादिवच्छारीरमेदेनैकस्यापि सुखदुःखे यदि स्याताम्, को दोषः ? न ब्रूमः एकस्य सुखदुःखे न सम्भवतः, किन्तु 'योऽहं देवदत्तः सुखी' 'सोऽहं यज्ञदत्तः दुःखी' इति प्रतिसन्धानं [ न ] स्याद् इति ब्रूमः~। अप्रतिसन्धानमपि शरीरभेदेनैवोपपत्स्यते, जन्मान्तरसुखादिवदिति चेत्, न, तत्र संस्कारस्यासत्त्वात्, अभिमवादनुद्धवाद्वा तदुपपत्तैः~। न चैवमिहापि स्यात्, यज्ञदत्तवद् देवदत्तेनाप्यप्रतिसन्धानप्रसङ्गात्~। अविद्यावशादात्मनोऽपि सुखादयो नात्मीयतया प्रतिसन्धीयन्त इति चेत्; न, यज्ञदत्तवद् देवदत्तेनाप्यात्मीयतया ${}^5$तत् प्रतिपत्तिप्रसङ्गात्~। परमविदुषेश्वरेणास्मदादिदुःखस्य स्वीयतया प्रतिसन्धानापत्तेश्च~। एतेन जीवात्मनामियं व्यवस्था न तु परब्रह्मण इति निरस्तम्~। यतो व्यवस्था तत एवात्मा तत्त्वतो भिन्न इति स्थितिः~। परब्रह्मणि चेश्वरादन्यस्मिन् प्रमाणाभावात्~। अनेकात्मत्वे ऐकात्म्यश्रुतिविरोध इत्यपि नास्ति; मुमुक्षुणा सर्वं विहा6य स्वात्मप्रतिष्ठेन भवितव्यम्; ${}^7$स च एक एवेति तात्पर्यात् इति~।

\blfootnote{I 'संयोगविशेषात् संस्काराच्चानुस्मृति' इति कि. पुस्तके सूत्रपाठः~। 2 'व्यवस्थातोनाना' इति सूत्रपाठः~। 3 कश्चिदपवृज्यते कश्चिन्नेति \textendash\ पा. ४. पु~। 4 नापवृक्त इत्येव \textendash\ जे~। 5 ${}^\circ$तयाऽप्रतिप्रसङ्गात् \textendash\ कि~। 6 विहायात्मप्रतिष्ठेन \textendash\ पा. ४. पु~। 7 न च \textendash\ कि~।}

\newpage
{\knu पृथक्त्वमप्यत एवैति} \textendash\ यतः सङ्ख्या तत्र सिद्धा, ततः पृथक्त्वमपि सङ्ख्यानुविधानात् तस्येत्यर्थः~। द्रव्यत्वेन परिमा1णयोगे सिद्धे सत्याह \textendash\ तथा चात्मेति वचनादिति~। 'वि2भुत्वान्महानाकाशस्तथा चात्मा (वै. सू. ७ \textendash\ १ \textendash\ २२) इति सूत्रं स्मारयति विभुत्वाद् यथाकाशो महांस्तथा तत एव वैभवादात्माऽपि महानित्यर्थः~। विभुत्वमेव तस्य कुत इति चेत्; न; तत्कार्यस् ज्ञानसुखादेः सर्वत्र दर्शनात्~। शरीरस्य गत्यैव तदुपपन्नमिति चेत्; न; अमूर्तत्वात्~। एतदपि कुत इति चेत्, परमाणुत्वेऽप्रत्यक्षत्वप्रसङ्गात्, प्रत्यक्षगुणानधिकरणत्वप्रसङ्गाच्च~। अवयवित्वेऽनित्यत्वप्रसङ्गात्, आदिमत्तया वीतराग3जन्मादर्शनानुपपत्तेश्चेति~।

एवं सूत्रकारानुमतिप्रदर्शनच्छलेन बुद्ध्यादिषु परिमाणान्तेषु प्रमाणमादर्शितम्~। सम्प्रति संयोगविभागवानात्मा द्रव्यत्वात् पृथिवीवदिति प्रमाणस्य तर्कमाह \textendash\ {\knu सन्निकर्षजत्वादिति~।} सहकार्यान्तरनिरपेक्षस्यात्मनः सुखादिकारणत्वे तेषां सततोत्पत्तिप्रसङ्गात्~। निमित्तमात्रापेक्षित्वेऽपि सर्वत्रोत्पत्तिप्रसङ्गात्~। तथा च शरीरावच्छेदस्य च संयोगादन्यस्यानुपपत्तेः~। एवं मनःसंयोगोऽसमवायिकारणत्वेन व्याख्यातः~। अदृष्टवदात्मसंयोगश्च ज्वलनपवनादिक्रियाकरणत्वेन बोद्धव्यः~। {\knu तद्विनाशकत्वाद्विभाग} इति \textendash\ यद्यात्मनि विभागो न स्यादुत्पन्नोऽपि संयोगो न विनश्येत, आश्रयनाशाभावादिति भावः~॥

\begin{sloppypar}
\hangindent=2cm {\knu (८१) म\renewcommand{\thefootnote}{4}\footnote{मनस्त्वाभिसम्बन्धान्मनः \textendash\ कि; दे~।}नस्त्वयोगान्मनः~। सत्यप्यात्मेन्द्रियार्थसान्निध्ये ज्ञा5नसुखादीनामभूत्वोत्पत्तिदर्शनात् करणान्तरमनुमीयते~। श्रोत्राद्यव्या6पारे ${}^7$स्मृत्युत्पत्तिदर्शनात् बाह्येन्द्रियैरगृहीतसुखादिग्राह्यान्तरभा8वात्~॥}
\end{sloppypar}

मनस्त्वाभिसम्बन्धान्मन इति \textendash\ मनस्त्वं नाम सामान्यविशेषः, तदभिसम्बन्धान्मन इतरेभ्यो भिद्यत इत्यर्थः~। ननु मन एवासि9द्धं निमित्तं कुतो मनस्त्वमित्यतः आह \textendash\ सत्यपिति~। यद्यस्मिन् सति कार्यं कदाचिद्भवति तत् तस्मादतिरिक्तापेक्षम्, यथा सत्यपि तुर्यादिसन्निकर्षे कदाचिद्भवति परस्तदतिरिक्तकुविन्दाद्यपेक्षः; सत्यपि चात्मेन्द्रियार्थसन्निकर्षे कदाचिद् ज्ञानसुखादिकं भवति, तस्मात् तदतिरिक्तापेक्षम्,${}^10$यच्च 

\blfootnote{1 परिमाणयोगे सत्याह \textendash\ कि~। 2 'विभवात्' इति सूत्रे पठः~। 3 जन्मापत्तेश्च \textendash\ क~। 4 मनस्त्वाभिसम्बन्धान्मनः \textendash\ कि; दे~। 5 ज्ञानसुखदुःखानाम् \textendash\ दे; जे, व्यो. (४२४)~। 6 व्यापारे च \textendash\ जे~। 7 स्मृतेरुत्पत्ति \textendash\ दे~। 8 भावाच्चान्तःकरणम् \textendash\ कं कि; भावाच्चेति \textendash\ दे~। 9 एवासिद्धं कुतो \textendash\ कि.~। 10 यश्च \textendash\ कि~।}

\newpage
\noindent
तदपेक्षणीयं तदेव मन इत्यर्थः~। अपेक्षणीयान्तराभावे सततं कार्योत्पादनप्रसङ्गात् व्यासङ्गेन चैतल्लक्षणीयम्~। तथा च क्वचिद्यौगपद्याभिमानः शतपत्रस्य पत्रशतव्यतिभेदवदाशुभावादूहनीयः~। अदृष्टमपेक्षणीयं भवतीति चेत्; न, तदितरकारणचक्रसाकल्ये तदसन्निधानानुपपत्तेः, तदसन्निधौ1तदितरसाकल्यानुपपत्तेरिति~।

लिङ्गान्तरमाह \textendash\ श्रोत्रादीति~। 'स्मृतिः शरीरावच्छिन्नात्मसंयुक्तद्रव्यसाध्या, कार्यत्वे सत्यात्मविशेषगुणत्वाद्, रूपादिज्ञानवत्', यत्तद्द्रव्यं तन्मनः~। श्रोत्राद्यन्यतमं तद्भविष्यतीति चेत्; अत उक्तं श्रोत्राद्यव्यापार इति~। अन्यथाऽसमवायिकारणाभावे तदनुत्पादप्रसङ्गात् विषयविप्रकर्षाच्च तद्व्यापारो परमो लक्षणीयः~।

लिङ्गान्तरमाह \textendash\ बाह्येति~। 'सुखाद्युपलब्धिरिन्द्रियसाध्या, जन्य साक्षात्कारत्वात् रूपाद्युपलब्धिवत्'~। 'श्रोत्रादिकमेव तत्साधनं भविष्यतीत्याशङ्क्योक्तं बाह्येन्द्रियैः श्रोत्रादिभिरगृहीतेति~। यदि बाह्येन्द्रियैः श्रोत्रादिभिः सुखादयो गृह्येरन्, अन्धबधिरादिभिर्नोपलब्धाः स्युरित्यर्थः~। सुखादीनां बुद्धेरन्यत्वं बुद्धेरस्ववेद्यत्वं च वक्ष्यते~।

\begin{sloppypar}
\hangindent=2cm {\knu (८२) तस्य गुणाः सङ्ख्यापरिमाणपृथक्त्वसंयोगविभागपरत्वापरत्व2संस्काराः~। प्रयत्नज्ञानायौगपद्यवचनात् प्रतिशरीरमेकत्वं सिद्धम्~। पृथक्त्वमप्यत एव~। तदभाववचनादगुणपरिमाणम्~। अपसर्पणोपसर्पणवचनात् संयोगविभागौ~। मूर्तत्वात्परत्वापरत्वे संस्कारश्च~। अस्पर्शवत्वाद् द्रव्यानारम्भकत्वम्~। क्रियावत्त्वान्मू3र्तत्वम्~। साधारणविग्रहवत्त्वप्रसङ्गाद4ज्ञम्~। क5रणभावात् परार्थम्~। गुणवत्त्वाद्द्रव्यम्~। प्रयत्नादृष्टपरिग्रहवशादाशुसश्चारि चेति~॥}
\end{sloppypar}

\vspace{-5mm}
\begin{center}
\textbf{॥~6इति द्रव्यपदार्थः समाप्तः~॥}
\end{center}

एवं सिद्धे मनसि व्यतिभेदसिद्धौ सत्यां नियतकार्यतया मनस्त्वमपि भेत्स्यतीत्यभिप्रायवांस्तस्य गुणान् स्मारयति त7स्येति~। यद्यपि द्रव्यत्वादेव सङ्ख्यादिपञ्चकं सिद्धे तथापि सूत्रकारानुमतिप्रदर्शनव्याजेन विशेषं विवक्षन्नाह \textendash\ प्रयत्नेति~। अणुत्वे हि मनसो बहुत्वमर्थसिद्धम्, अन्यकैस्मिन्नेकत्र वर्त्तमाने शरीरान्तरे तत्कार्यं ज्ञानादि न

\blfootnote{I प्रागपि \textendash\ क~। 2 संस्काराश्चेति \textendash\ दे~। 3 मूर्त्तम् \textendash\ जे~। 4 दज्ञत्वम् \textendash\ मु भाः व्योन ( ४२७ )~। 5 स्वयं करणभावात \textendash\ दे~। 6 इति प्रशस्तपादभाष्ये द्रव्यपदार्थः \textendash\ मु. भा; प्रशस्तकरीयभाष्ये \textendash\ पा. ५. पु श्रीप्रशस्तदेवाचार्यविरचिते वैशेषिकभाष्ये \textendash\ पा. ६. पु~। 7 तस्य गुणानाह \textendash\ तस्येति \textendash\ पा. ३. पु; जे~।}

\newpage
\noindent
स्यात्~। अतो बहुत्वे सत्येकस्मिन्नपि शरीरे तैर्बहुभिर्युगपद्ज्ञानोत्पत्तिप्रसङ्गस्तदवस्थः तद्वरमेकमेव विभु तदस्तु इत्याशङ्क्य सूत्रकृतोक्तम् 'प्रयत्नज्ञानायौग1पद्यादेकं मनः' (वै. सूः ३ \textendash\ २ \textendash\ ३) इति~। एकैकस्मिन् शरीरे एकैकमेव मनः~। प्रयत्नायौगपद्याद् ज्ञानायौगपद्यादणुत्ववत् प्रतिशरीरमेकत्वप्यनुमातव्यमित्यर्थः~।

{\knu पृथक्त्वमप्यत एव} \textendash\ सङ्ख्यानुविधानादेवेत्यर्थः~। तदभाववचनादणुपरिमाणम्, 'तदभावादणु मनः' (वै. सू. ७ \textendash\ १ \textendash\ २३) इति सूत्रकारवचनात्~। तच्छब्देन पूर्वोक्तं वैभवं परामृशति, तेन विभवाभावादित्यर्थः~। विभवाभावश्च ज्ञा2नाद्ययौगपद्यात्~। विभवे ह्येकेन्द्रियग्राह्या इव नानेन्द्रियग्राह्या अपि रूपादयो विषयाः सकृदेव ज्ञायेरन् कारणयौगपद्यात्~। एतेन स्पर्शशून्यत्वादयः परेषां विभुत्वे हेतवः कालात्ययापदिष्टा द्रष्टव्याः~।

यद्येवमणु मनः कथं तर्हि द्विस्त्रिश्चिन्नानां गृहगोधादिशरीराणामुत्पतनापसर्पणादि चेष्टाविशेषाः ? न ह्यविभुनैकेनानेकाधिष्ठानं सम्भवतीति चेत्; न; तत्क्षणमपूर्वोत्पन्नप्रयत्नवशात् कर्मोपगृहीतान्तःकरणान्तरवशाद्वेति ${}^3$न किञ्चिदेतत्~। 

{\knu अपसर्पणोपसर्पणवचनात् संयोगविभागाविति} \textendash\ 'अपसर्पणमुपसर्पणमशितपीतसंयोगाः कार्यान्तरसंयोगाश्चेदृत्यदृष्टकारितानि' (वै. सू ५ \textendash\ २ \textendash\ १७) इति सूत्रकारवचनात्~। अपसर्पणं पूर्वशरीरविभागहेतुर्मनःकर्म~। उपसर्पणम् उत्तरशरीयप्राप्तिहेतुर्मनःकर्म~। ततः संयोगविभागौ सिद्धौ~। अन्यथा त4दनुपपत्तौ प्रेत्यभावानुपपत्तिरिति भावः~।

स्यादेतत्; मनसोऽतीन्द्रियत्वात् परापरव्यवहारायोगात् प्रयत्नादृष्टोपग्रहाच्च क्रियोपपत्तौ परत्वापरत्ववे5गाभ्युपगमो निर्बीज इत्यत आह \textendash\ {\knu मूर्तत्वा}दिति~। मूर्तस्य तैःसह स्वभावतः सम्बन्धात् तेन तेऽनुमातव्याः~। न हि ${}^6$प्रात्यञ्चिक एव संव्यवहार इति नियमो धूमाद्यनुमिते वह्न्यादौ तद्व्यवहाराभावप्रसङ्गात्~। नाप्यदृष्टोपग्रहेश्च क्रियोत्पत्तौ वेगो न स्यात् ज्वलनपवनादौ तद्भावप्रसङ्गादिति भावः~।

स्यादेतत्; मूर्तत्वाद् द्रव्यारम्भकत्वं मनसः किं न स्थादित्यत आह \textendash\ {\knu अस्पर्शवत्त्वा}दिति~। न हि मूर्तत्वस्य द्रव्यारम्भकत्वेन सह स्वभाविकः सम्बन्धः, विपक्षे बाधकाभावात्~। स्पर्शवत्त्वं तु द्रव्यारम्भकत्वेनाऽप्रयोजकम्, द्रव्यानारम्मे स्पर्शवत्तायां

\blfootnote{I 'प्रयत्नयौगपद्याज्ज्ञानयौगपद्याच्चैकम्' इति सूत्रगाठः~। 2 ज्ञानायौगपद्यात् \textendash\ कि~। 3 यत्किञ्चिदेतत् \textendash\ कि. क~। 4 तदनुत्पत्तौ \textendash\ जे~। 5 वेगापगमौ \textendash\ मु. कि~। 6 प्रात्यक्षिक \textendash\ क~।}

\newpage
\noindent
प्रमाणाभावादनुपलभ्यत्वप्रस1ङ्गाच्च, तदीयकार्यान्तरस्याभावादिति~। न च मूर्तत्वेन स्पर्शोऽनुमातव्यः; तद्विरहेण वा अमूर्तत्वम्, मूर्तस्य स्पर्शविरहे विरोधाभावात्~। विभवाभावस्य च परममहत्त्वेन सह विरोधादित्याह 'क्रियावत्त्वान्मूर्त्तं मनः' इति प्रक्रमात् क्रियावत्त्वेनाविभुत्वमुपलक्षयति~।

स्यादेतत्; करणत्वेन मनसः पूर्वं ज्ञातृत्वं प्रतिषिद्धम्, तदनुपपन्नम्, ज्ञानकरणत्वज्ञानकर्तृत्वयोर्विरोधाभावात्~। प्रयोक्तृप्रयोज्ययोः कथमेकत्वमिति चेत्; तथापि भेदोऽस्तु~। कुतस्त्वज्ञत्वमत आह \textendash\ {\knu साधारणेति~।} तथा चैकाभिप्रायेण प्रवृत्तिनिवृत्ती न स्याताम्; प्रत्युताभिप्रायविरोधात् न प्रवृत्तिर्ननिवृत्तिश्च स्यादिति भावः~। यद्येवमज्ञं मनस्तर्हि स्वार्थं न गुणतः स्वरूपतश्च, न परार्थमतीन्द्रियत्वाद् अत आह \textendash\ {\knu करणभावादिति~।} तथापि मूर्तस्य विशेषगुणयोगः स्यात्; अन्यथा द्रव्यत्वमपि न स्यात्~। मूर्तत्वसमानाधिकरणस्य द्रव्यत्वस्य विशेषगुणवत्त्वव्याप्यत्वादत आह \textendash\ {\knu गुणवत्त्वादुद्रव्यम्~।} अवैभवान्मूर्तत्वेऽपि गुणतत्त्वाद् विशेषगुणरहितस्यापि द्रव्यत्वाविरोध इत्यर्थः~। तथापि गुरूत्वद्रवत्वस्थितिस्थापकाभावे कथं क्रियावत्त्वम् ? वेगस्योत्तरक्रियोत्तरकालीनत्वात्; नहि मूर्तत्वमात्रेण क्रियेत्यत आह \textendash\ {\knu प्रयत्नादृष्टपरिग्रहादिति~।} क्वचित्प्रयत्नपरिग्रहात् क्वचिददृष्टपरिग्रहादित्यर्थः~। इति शब्दो द्र2व्यसमाप्तौ~॥

\blfootnote{I प्रसङ्गात् \textendash\ कि~। 2 द्रव्यसमाप्तौ तदनन्तरं 'इति श्रीमदाचार्योदयनविरचितायां किंरणावल्यां द्रव्यपदार्थः समाप्तः' इत्यधिकं कि. क पुस्तकयोः; इत्युदयनाचार्यविरचितायां इति पाठान्तरं \textendash\ पा. ३ पुस्तकेऽस्याः पुष्पिकायाः अतःपरं पूर्वसम्पादकेन प्राप्ते तृतीयादर्शपुस्तके पद्यद्वयमधिकं दृश्यते~। तथा हि
\begin{quote}
{\qt * 'वन्दे शिवं शिवमिवोदयनं निदानमेकं गभीरनयनत्वविवैकसिन्धोः\\
दोषाकरादपि विविच्य कलां भजन्तमः[ न्तः ] कृताक्षतपदं सुमनः सहस्रैः~।'

शाकाब्दे रसपूर्णपञ्चदशके (१५०६) ह्यानन्दके वत्सरे\\
पञ्चम्यां रविवासरे च वि(व्य)लिखत् कृष्णे तपे शुक्रके (?)~।\\ 
ऋक्षे श्रावणिके विनायकसुतो दत्तात्रयः सत्वर \textendash\ \\
स्त्वाकूतोपकनामशोभनकरे शैवे गिरौ त्वोदये (?)~॥}
\end{quote}
'जे' पुस्तके तु "इति पण्डितश्री उदयनकृतौ किरणावत्यां निबन्धिका श्रीरहसदत्त सङ्गृहीतायां द्रव्यपदार्थः समाप्तः~। तदनन्तरं 'वन्दे शिवम्' इत्येकं पद्यमपि~।
\begin{center}
\rule{0.2\linewidth}{0.5pt}
\end{center}}

\newpage
\thispagestyle{empty}
\begin{center}
{\Large [ अथ प्रशस्तपादभाष्ये गुणपदार्थनिरूपणम् ]}
\end{center}

\hangindent=2cm {\knu (८३) रूपादीनां गुणानां सर्वेषां गुणत्वाभिसम्बन्धो द्रव्याश्रितत्वं नि\renewcommand{\thefootnote}{1}\footnote{निष्कियत्वमगुणत्व च \textendash\ कि; निष्कियत्वमगुणवत्त्वं \textendash\ जे; व्यो. (४३०); पा. ५, ६. पु~।}र्गुणत्वं निष्क्रियत्वम्~॥}

\begin{quote}
{\qt तुष्टेमोचियतो बद्धानतुष्टेर्बध्नतः पुनः~।\\
कारागारमिदं विश्वं यस्य नौमि तमीश्वरम्~॥ }
\end{quote}

साधर्म्यवैधम्यभ्यां निरूपितानि द्रव्याणि~। सम्प्रति ताभ्यामेव गुणान्निरूपयति {\knu रूपादीनामि}ति~। 'गुण' शब्देनाप्रधानमप्युच्यते, यथा 'वयमिह गुणीभूताः' इति~। न चाप्रधानमात्रस्य गुणत्वसमान्येन योग इत्याशङ्कामपनेतुमुक्तं {\knu रूपादीनामि}ति~। तथापि कियताम् ? अत उक्तम् {\knu सर्वेषामि}ति~। गुणत्वं नाम सामान्यविशेषः, तेनाभिमतः सम्बन्धः समवायलक्षणः~। तेन गुण इतरेभ्यो भिद्यते गुण इति वा व्यवहर्तव्यः गुणत्वाभिसम्बन्धात्~। यस्तु न तथा नासौ गुणत्वाभिसम्बन्धः यथा द्रव्यादिः इति लक्षणमुक्तं वेदितव्यम्~।

गुणत्वमसिद्धं गोत्वादिवदप्रतीतेर्व्यवस्थापकाभावाच्च~। तथाहि 'सामान्यवानगुणो गुणः' इति गुणव्यावृत्तिरितरेतराश्रयप्रसङ्गदूषिता, गुणरहितत्वे हि तत्सिद्धिः तत्सिद्धौ च तद्रहितत्वसिद्धिरिति~। 'सामान्यवान् स्पर्शरहितो द्रव्याश्रितो गुणः' 'सामान्यवान् कार्यानाश्रयो गुणः' इति चातिव्यापकं कर्मण्यपि गतत्वात्~। कर्मान्यत्वे सतीति तु प्रक्रियामात्रम्, \renewcommand{\thefootnote}{2}\footnote{अपेक्षित \textendash\ जे~।}अनपेक्षितत्यावृत्तेरतिप्रसञ्जकत्वात्, रूपान्यत्वे रसान्यत्वे सति चेत्यादेरपि सुवचत्वात्~। किमत्र व्यवस्थापकेन रत्नतत्त्वमिव गुणत्वमुपदेशापेक्षेण चक्षुरादिना प्रत्यक्षत एव प्रतीयत इति तु स्वशिष्य\renewcommand{\thefootnote}{3}\footnote{व्यामोहतम् \textendash\ कि~।}व्यामोहनमात्रम्, निमित्तमन्तरेणोपदेशस्यागममात्रत्वात्~। तस्माद्वरं \renewcommand{\thefootnote}{4}\footnote{भूषणो न्यायभूषणारख्यः न्यायसूत्रवृत्तिकारः तार्किकरक्षायामपि सुस्पष्टतया समुद्धृतोऽयम् \textendash\ कि. सं~।}{\knu भूषणः} कर्मापि गुणस्तल्लक्षणयोगादिति~। 

न; कर्मणश्चलनात्मकतया प्रतीयमानस्य तावद् रूपादिव्यावृत्तमुत्क्षेपणाद्यनुवृत्तमेकजातीयत्वमनुभवसिद्धम्, परस्परविरूद्धसंयोगविभागलक्षणकार्योत्पादकत्वं च तस्य व्यवस्थापकमप्यनुभवसिद्धमेव~। तद्विपरीतं च कार्यं रूपादीनां चतुर्विशतेरुपलभ्यते~। ततस्तैः कर्मविषरीतजातीयैर्मवितव्य्~। न ह्येकजातीयत्वे कार्यभेदनियम नोपपद्यते न चा

\newpage
\noindent
कर्मत्वेनैतद् भविष्यतीति वाच्यम्, अकर्मतया द्रव्यस्याप्यविरुद्धकार्यकारित्वप्रसङ्गात्~। तथा च रूपादिवत् नैकं द्रव्यं संयोगविभागावारमेत~। तस्माद्यथा विरुद्धकार्यकारित्वाविशेषेऽपि सापेक्षत्वानपेक्षत्वाभ्यां द्रव्यकर्मणोर्विजातीयत्वं, तथा सामान्यवत्त्वकार्यानाश्रयत्वाद्यविशेषेऽपि विरुद्धाविरुद्धकार्यत्वाभ्यां गुणकर्मणोरपि विजातीयत्वमित्यवदातम्~। सोऽयं 'सं\renewcommand{\thefootnote}{1}\footnote{एकद्रव्यमगुणं संयोगविभागेष्वनपेक्षकारणमिति कर्मलक्षणम्~। वै. सूत्रपाठे १ \textendash\ १ \textendash\ १७~। अयं तु भिन्न एव पाठः~।}योगविभागयोरनपेक्षं कारणं कर्म' (वै. सूः १ \textendash\ १ \textendash\ १७) इत्यस्य सूत्रस्यार्थः~। 

{\knu द्रव्याश्रितत्वमिति} \textendash\ 'अनाश्रितत्वनित्यत्वे चान्यत्रावयविद्रव्येभ्यः' (प्र. भा. १६) इति वदता अर्थादाश्रितत्वानित्यत्वे चान्यत्र नि\renewcommand{\thefootnote}{2}\footnote{नित्यद्रव्येभ्य \textendash\ कि~।}रवयवद्रव्येभ्य इत्युक्तम्~। {\knu 'नित्यद्रव्यबृत्तयोऽन्त्या विशेषा'} इति चोक्तम्~। वक्ष्यते च {\knu 'मूर्तद्रव्यवृत्तित्वम्'} इति च कर्मसु~। लक्ष्यते सामान्यमपि सामान्यरूपेण पदार्थत्रयवृत्ति~। समवायूस्त्ववृत्तिरेवेति~। गुणाम्तु गुणरूपेण सर्वत्र द्रव्य एव वर्त्तन्ते~। न न वर्त्तन्ते एव समवायवत्~। न क्वचिद् द्रव्ये वर्त्तन्ते कर्मवच्च~। न न वर्त्तन्तेऽपि द्रव्यवत्~। नाऽद्रव्येऽपि वर्त्तन्ते सामान्यवदिति~। तदेतेषां द्रव्याश्रितत्वमित्यस्यार्थः~।

तदेतद् द्रव्यलक्षणव्यावृत्त्या सिद्ध्यति~। न हि तस्मिन् सति गुणत्वाभिसम्बन्धो भवितुमर्हति जातिसङ्करप्रसङ्गात्~। नाप्यभिन्याप्त्या द्रव्याश्रितत्वम्, एकजात्युपग्रहाभावाद् इत्यभिप्रायवानाह \textendash\ {\knu 'निष्क्रियत्वमगुणवत्त्वं च'} इति~। चलनात्मिकाया; क्रियाया निष्क्रान्ता निष्क्रिया रूपादयः, तेषां भावस्तत्त्वम्~। मूर्तिविरहेणैतदवसेयम्~। क्रियायास्तदनुविधायित्वात्~। मूर्त्यभावश्च रूपादीनां समानदेशत्वादध्यवसेयः, मूर्तत्वसमानदेशत्वयोः सहजविरोधात्~। गुणा येषु वर्तन्ते ते गुणवन्तो, न गुणवन्तोऽगुणवन्तस्तेषां भावस्तत्त्वम्~। समानजातीय\renewcommand{\thefootnote}{3}\footnote{गुणसद्भाव \textendash\ कि~।}गुणाभावस्तावदनवस्थाप्रसङ्गात्~। रूपादौ रसादिवेगा\renewcommand{\thefootnote}{4}\footnote{रसादयो \textendash\ जे. पा. ६. पु~। १४}न्तानां सद्धावे मूर्तत्वप्रसङ्गात्~। बुद्ध्यादीनां तु प्रतिसन्धातृगुणत्वव्यवस्थितेरप्रसक्तिरेव~। शब्दस्य नभो नियमात्~। गुणेषु गुणयोगे च समवायिकारणत्वप्रभक्तौ द्रव्यत्वापत्तेर्गुणत्वव्याघात एव~। रूपावयवी च रूपेष्वेव वर्त्तत इति घटादेर्नीरूपत्वप्रसङ्ग इत्यादि~।

एवं निर्गुणत्वनिष्क्रियत्वे च 'रूपादयो गच्छन्ति' 'चतुर्विंशति गुणाः' 'महांश्छब्द ' इत्यादयो व्यवहाराः तदेकार्थसमवायादिना साधर्म्येण गौणाः समर्थनीयाः~। चकारात् समवायिकारणताविरहः~। यदा हि गुणकर्मणी गुणेषु न स्तस्तदा द्रव्ये को कथा ?

\newpage
\noindent
नहि निर्गुणं द्रव्यमाश्रयते, कारणगुणाभावे तस्यापि निर्गुणत्वप्रसङ्गात्~। न च निर्गुणे द्रव्ये किञ्चित् प्रमाणमस्तीति~॥

\hangindent=2cm {\knu (८४) रूपरसगन्धस्पर्शप\renewcommand{\thefootnote}{1}\footnote{परत्वद्रवत्व \textendash\ दे~।}रत्वापरत्वगुरुत्वद्रवत्वस्नेहवेगा मूर्तगुणाः~॥}

सम्प्रति गुणानामेवान्योन्यं वैधर्म्यमाह \textendash\ {\knu रूपेति}~। एते मूर्ति न व्यभिचरन्तीत्यर्थः~। वेगः संस्कारविशेषः~। उपलक्षणमेतत्, \renewcommand{\thefootnote}{2}\footnote{स्थितिस्थापक \textendash\ जे~।}स्थितिस्थापकोऽपि द्रष्टव्यः~॥

\hangindent=2cm {\knu (८५) बुद्धिसुखदुःखेच्छाद्वेषप्रयत्नधर्माधर्मभावनाशब्दा अमूर्तगुणाः~॥}

बुद्धित्यादि \textendash\ एते मूर्तिविरोधिन इत्यर्थः~। भावना संस्कारविशेषः~॥ 

\hangindent=2cm {\knu (८६) सङ्ख्यापरिमाणपृथक्त्वसंयोगविभागा उभयगुणाः~॥} 

{\knu सङ्ख्ये}ति \textendash\ द्रव्यमात्रस्य व्याप्यव्यापकभूता इत्यर्थः~।

\hangindent=2cm {\knu (८७) संयोगविभागद्वित्वद्विपृथक्त्वादयोऽनेकाश्रिताः~॥}

{\knu संयोगे}ति \textendash\ एका संयोगव्यक्तिर्द्वयोर्द्वव्यव्यक्त्योर्वर्तते~। तथा विभागद्वित्वद्विपृथक्त्वव्यक्तय एकैकाः~। आदिग्रहणात् त्रित्वत्रिपृथक्त्वादयः~। अनेकशब्दस्यैक्र्युदासवृत्तित्वात् त्रिचतुःपञ्चादिद्रव्यव्यक्तिष्वपि वर्त्तन्त इति~॥

{\knu (८८) शेषास्त्वेकैकद्रव्यवृत्तयः~॥}

{\knu शेषास्त्वि}ति \textendash\ रूपरसगन्धस्पर्शैकत्वैकपृथक्त्वपरिमाणपरत्वापरत्वगुरूत्वद्रवत्वस्नेहसंस्कारबुद्ध्यादिशब्दान्ता एकैकस्थामेव व्यक्तौ वर्तन्त इत्यर्थः~॥

\begin{sloppypar}
\hangindent=2cm {\knu (८९) रूपरसगन्धस्पर्शस्नेहसांसिद्धिकद्रवत्वबुद्धिसुखदुःखेच्छाद्वेषप्रयत्नधर्माधर्मभावनाशब्दा वैशेषिकगुणाः~॥}
\end{sloppypar}

{\knu रूपेति} विशेषा एव वैशेषिकं, तस्य गुणाः स्वाश्रयव्यवच्छेदोचितावान्तरसामान्यविशेषवन्त इत्यर्थः~। नैमित्तिकद्रवत्वं पृथिवीतेजसोः सामान्यगुण इत्यत उक्तं {\knu सांसिद्धिकेति~।} तथाहि रूपं भास्वरत्वादिना, रसो मधुरत्वादिना, गन्धो गन्धत्वेनैव, स्पर्शो उष्णत्वादिनाः स्नेहः स्नेहत्वेन; यथोक्तं द्रवत्वं ताद्रूप्येण, बुद्धयादयो बुद्धित्वादिना स्वाश्रयमितरेभ्यो व्यवच्छिन्दन्ति~॥

\newpage
\begin{sloppypar}
\hangindent=2cm {\knu (९०) सङ्ख्यापरिमाणपृथक्त्वसंयोगविभागपरत्वापरत्वगुरुत्वनैमित्तिकद्रवत्ववेगाः सामान्यगुणाः~॥}
\end{sloppypar}

{\knu सङ्ख्येति} \textendash\ वेगेति भावनातिरिक्तसंस्कारोपलक्षणम् $\rightarrow$ \renewcommand{\thefootnote}{1}\footnote{$\rightarrow$ $\leftarrow$ एतच्चिहान्तर्गतः पाठः 'जे' पुस्तके नास्ति~।}तेन स्थितिस्थापकोऽपि गृह्यते $\leftarrow$~। सामान्यं साधर्म्यं तद्रूपा गुणाः सामान्यगुणाः, न स्वाश्रयव्यवच्छेदाय प्रभवन्तीत्यर्थः~॥

\hangindent=2cm {\knu (९१) शब्दस्पर्शरूपरसगन्धा बाह्यैकैकेन्द्रियग्राह्या~॥}

{\knu शब्देति} \textendash\ बाह्येन चक्षुरादिना एकैकेनैवेन्द्रियेण ग्रहीतुं योग्या इत्यर्थः~॥

\hangindent=2cm {\knu (९२) स\renewcommand{\thefootnote}{2}\footnote{अयं पाठः सम्पूर्णतया 'दे' पुस्तके भ्रष्टः~।}ङ्ख्यापरिमाणपृथक्त्वसंयोगविभागपरत्वापरत्वद्रवत्वस्नेहवेगा द्रीन्द्रियग्राह्याः~॥}

{\knu सङ्ख्येति} \textendash\ बाह्येत्यनुवर्तते~। सङ्ख्यादयो वेगान्ताश्चक्षुस्पर्शनग्रहणयोग्या इत्यर्थः~॥

\hangindent=2cm {\knu (९३) बुद्धिसुखदुःखेच्छाद्वेषप्रयत्नास्त्वन्तःकरणग्राह्याः~॥}

{\knu बुद्धीति} \textendash\ बुद्ध्यादयः प्रयत्नान्ताः अन्तःकरणेनैव ग्रहीतुं योग्या इत्यर्थः~॥

\hangindent=2cm {\knu (९४) गुरुत्वधर्माधर्मभावना ह्यतीन्द्रियाः~॥}

{\knu ु\renewcommand{\thefootnote}{3}\footnote{अत्र मूलं पुनरुक्तमिवाभाति~।}रुत्वधर्माधर्मभावना ह्यतीन्द्रियाः} \textendash\ भावनेति वैगमात्रव्यवच्छेदाय~। तेन स्थितिस्थापकोऽपि गृह्यते~। इन्द्रियमतीत्य वर्त्तन्त इत्यतीन्द्रियाः; न केन चिदिन्द्रियेण गृह्यन्ते इत्यर्थः~। 'हि'शब्दः 'तु'अर्थः~।

\hangindent=2cm {\knu (९५) अपाकजरूपरसगन्धस्पर्शपरिमाणैकत्वैकपृथक्त्वगुरुत्वद्रवत्वस्नेहवेगाः कारणगुणपूर्वकाः~॥}

{\knu अपाकजेति} \textendash\ स्वाश्रयस्य यत्समवायिकारणं तस्य ये सजातीया गुणाः तत्पूर्वकाः~। तथाहि तन्त्वादिरूपादिभिः पटादि\renewcommand{\thefootnote}{4}\footnote{रूपाश्च \textendash\ कि~।}रूपादय आरभ्यन्ते~। वेगेन स्थितिस्थापकोऽप्युपलक्षणीयः~। पार्थिवपरमाणुरूपादयो नैमित्तिकं च द्रवत्वं नैवमित्याशङ्क्याह \textendash\ {\knu अपाकजेति~।} द्रवत्वमित्यत्राप्यपाकजेत्यनुषञ्जनीयम्~। कारणपदस्यानन्तरं मात्रपदमप्यध्याहार्यम्; अन्यथा संयोगविभागावपि क्वचिदेवमित्यतिव्याप्तिः स्यात्~। एष्वेवैतत्सम्भाव्यत इत्यन्ययोगव्यवच्छेदेनैतत्साधर्म्यं, तेन क्वापि कर्मजत्वेऽपि वेगस्य न दोषः~।

\newpage
\hangindent=2cm {\knu (९६) बुद्धिसुखदुःखेच्छाद्वेषप्रयत्नधर्माधर्मभा\renewcommand{\thefootnote}{1}\footnote{भावनास्त्वकारणगुणपूर्वकाः \textendash\ दे~।}वनाशब्दा अकारणगुणापूर्वकाः~॥}

{\knu बुद्धीति} \textendash\ बुद्ध्यादयः शब्दान्ता अकारणगुणपूर्वकाः~। एतेषामाश्रयस्य नित्यतया कारणाभावादिति भावः~। एते अकारणगुणपूर्वका एवेत्यर्थः~। तथा च संयोगविभाग वेगनैमित्तिद्रवत्वैरकारणगुणपूर्वकैरपि नातिव्याप्तिः~। नह्येवंजातीया अकारणगुणपूर्वका एव, क्वचित्कारणगुणपूर्वकाणामपि दर्शनादिति~॥

\begin{sloppypar}
\hangindent=2cm {\knu (९७) बुद्धिसुखदुःखेच्छाद्वेषप्रयत्नधर्माधर्मभावनाशब्द\renewcommand{\thefootnote}{2}\footnote{स्थूले \textendash\ जे~।}तुलपरिमाणोत्तरसंयोगनैमित्तिकद्रवत्वपरत्वापरत्व\renewcommand{\thefootnote}{3}\footnote{'परत्वापाकजाः' इति सर्वत्र मु. भाष्ये किन्तु पाठोऽयमर्थदृष्टयाऽशुद्धः अतः 'परत्वापरत्वपाकजाः' इति शुद्धः पाठः स्थापितः~।}पाकजाः संयोगजाः~।}
\end{sloppypar}

{\knu बुद्धीति} \textendash\ बुद्धयादयः पाकजान्ताः संयोगजाः, संयोगासमवायिकारणका इत्यर्थः~। बुद्ध्यादयो भावनान्ता आत्ममनःसंयोगजाः~। 'शब्द'शब्देन शब्दविशेषो ग्राह्यः, स च भेर्याकाशसंयोगजन्यः~। तुलेत्युपलक्षणम्, द्रव्यान्तराण्यपि समानसङ्ख्यापरिमाणैरवयवैरारब्धानि सातिशयपरिमाणानि ग्राह्याणि~। तत्परिमाणं प्रचयाख्यसंयोगजम्~। उत्तरः संयोगस्तन्तुतुर्यादिसंयोगात् पटतुर्यादिसंयोगः~। नैमत्तिकं द्रवत्वमग्निसंयोगात्~। परत्वापरत्वे दिक्कालपिण्डसंयोगात्~। पाकजाः पार्थिवपरमाणुरूपादयोऽग्निसंयोगादिति~॥

\hangindent=2cm {\knu (९८) संयोगविभागवेगाः कर्मजाः~॥}

{\knu \renewcommand{\thefootnote}{4, 5}\footnote{अत्र मूलभाष्यं पुनरुक्तमिवाभाति~।}संयोगविभागवेगाः कर्मजाः~।} संयोगविशेषविभागविशेषवेगविशेषा इत्यर्थः~। तत्र संयोगविभागावाद्यौ~। वेगोऽवेगवद्द्रव्यारब्धे वर्तमान इति~।

\hangindent=2cm {\knu (९९) शब्दोत्तरविभागौ विभागजौ~॥}

{\knu ${}^5$शब्दोत्तरविभागौ विभागजाविति~।} 'शब्द'शब्दः शब्दविशेषे वर्त्तते~।

\hangindent=2cm {\knu (१००) परत्वापरत्व\renewcommand{\thefootnote}{6}\footnote{द्वित्वपृथक्त्वादयो \textendash\ जे; व्यो (४३५)~।}द्वित्वद्विपृथक्त्वादयो बुद्ध्यपेक्षाः~॥}

{\knu परत्वेति} \textendash\ बुद्धयपेक्षाः \textendash\ अपेक्षाबुद्धिजन्या इत्यर्थः~। आदिग्रहणात् त्रित्वत्रिपृथक्त्वादयो ग्राह्याः, तदितरे च तद्विषरीता इत्यर्थात्~॥

\newpage
\hangindent=2cm {\knu (१०१) रूपरसगन्धानुष्णस्पर्शशब्दपरिमाणैकत्वैकफॄथक्त्वस्नेहाः समानजात्यारम्भकाः~॥}

{\knu रूपेति} \textendash\ रूपादयः स्नेहान्ताः समानजात्यारम्भका इति~। इह आरम्भकत्वमेसमवायिकारणत्वं यदि विवक्षितं तर्हि 'अनुष्ण'पदोपादानमनुपपन्नम्~। नह्युष्णस्पर्शो विजातीयासमवायिकारणम्~। अथ कारणत्वमात्रमारम्भकत्वं स्नेहोपादानमनुपपन्नम्, स हि विजातीयस्यापि सङ्ग्रहस्य कारणमिति वक्ष्यते~। तत्कथमेतत् ? इत्थमेतत्; कारणत्वमात्रे विवक्षिते द्वयस्यापि वर्जनम्~। असमवायिकारणत्वे विवक्षिते \renewcommand{\thefootnote}{1}\footnote{उष्णस्याप्यवर्जनम् \textendash\ जे~।}द्वयस्याप्यवर्जनम्~। समाना एका जातिर्येषां ते समानजातयः~। तदारम्भका रूपरसगन्धस्पर्शशब्दपरिमणैकत्वैकपृथकत्वस्नेहा असमवायिकारणतामाप्नुवन्तः समानजातीयेष्वेव एत एवोष्णस्पर्शस्नेहौ विहायात्मधर्मेतरकारणं भवन्तः समानजातीयोष्वेत्यर्थः~॥

\hangindent=2cm {\knu (१०२) सुखदुःखेच्छाद्वेषप्रयत्नाश्चासमानजात्यारम्भकाः~॥}

{\knu सुखेति} \textendash\ एते विजातीयमेवारभन्ते~। सुखमिच्छाम्, दुःखं द्वेषम्, तौ प्रयत्नम्, प्रयत्नः क्रियामिति~। 'तु'शब्दः पूर्वापरान् व्यवच्छिन्दन्नवधारणं द्योतयति~। पुत्रस्य सुखदुःखाभ्यां पितुः सुखदुःखे समानजातीये अपि भवत इति चेन्न, विषयमात्रत्वात्~। \renewcommand{\thefootnote}{2}\footnote{विषयतया \textendash\ जे~।}विषयत्वेऽपि \renewcommand{\thefootnote}{3}\footnote{विज्ञानद्वाराजै \textendash\ जे~।}ज्ञानद्वारा तत्कारणमिति चेत् न, भविष्यतोरपि पुत्रस्य सुखदुःखयोज्ञाने पितुः सुखदुःखाद्युत्पत्तिदर्शनात्~। न चानागतमपि साम्प्रतिकस्य कारणम्, पूर्वभावनियमस्यः तत्त्वादिति~। साक्षात् कारणत्वं वा विवक्षितम्~॥

\hangindent=2cm {\knu (१०३) संयोगविभागसङ्ख्यागुरुत्वद्रवत्वोष्णस्पर्शज्ञानधर्माधर्मसंस्काराः समानासमानजात्यारम्भकाः~॥}

{\knu संयोगेति} \textendash\ अत्र स्नेहोऽपि ग्राह्यः~। संयोगाद्विजातीयं द्रव्यादि, विभागाच्छदः सङ्ख्यातः परिमाणम्, गुरुत्वात्पतनम्, द्रवत्वात् स्यन्दनम्, स्नेहात् सङ्ग्रहः, \renewcommand{\thefootnote}{4}\footnote{औष्ण्यात् \textendash\ पा. ७. पु जे~।}उष्णस्पर्शात् पाकजाः, ज्ञानादिच्छा, धर्मात्सुखम्, अधर्माद् दुःखम्, संस्कारात् क्रियास्मृती इति विजातीयवर्गः~। सजातीया प्रसिद्धाः~। 

\hangindent=2cm {\knu (१०४) बुद्धिसुखदुःखेच्छाद्वेषभावनाशब्दाः स्वाश्रयसमवे \textendash\ तारम्भकाः~॥}

\newpage
{\knu \renewcommand{\thefootnote}{1}\footnote{मूलं पुनरुक्तमिवाभति~।}बुद्धिसुखदुःखेच्छाद्वेषभावनाशब्दाः स्वाश्रयसमवेतारम्भकाः~।} यद्यपि बुद्धिर्द्वित्वादिकमन्यत्राप्यारभत इति उभयत्रारम्भकेषु पठितुमुचिता तथापि स्व्राश्रयसमवेतविशेषगुणेष्विति द्रष्टव्यमित्यदोषः~।

\hangindent=2cm {\knu (१०५) रूपरसगन्धस्पर्शपरिमाणस्नेहप्रयत्नाः परत्रारम्भकाः~॥}

रूपरसगन्धस्पर्शपरिमाणस्नेहप्रयत्नाः परत्रारम्भका इति~। अवयवरूपादयः स्नेहान्ता अत्रयवविनि रूपादिकमारभन्ते~। प्रयत्नश्चात्मनि वर्तमानः शरीरे चेष्टाम्~। यद्यपि स्नेहः सङ्ग्रहं स्वाश्रयेऽप्यारभमाणः, प्रयत्नश्च विडितनिषिद्धप्रवृत्तिरूपः स्वाश्रये धर्माधर्मावुत्पादयन् संयोगादिषु पठितुमुचितः, तथापि परिमाणान्ताः परत्रैव, स्नेहान्ताः सजातीयं परत्रैव, प्रयत्नान्ता एकाश्रितमन्यत्रैव~। 'प्रयत्न'शब्दे\renewcommand{\thefootnote}{2}\footnote{न चात्र \textendash\ जे~।}नात्र प्रयत्नविशेषो विहितनिषिद्धविषयो ग्राह्यः इत्यदोषः~। 

\begin{sloppypar}
\hangindent=2cm {\knu (१०६) संयोगविभागसङ्ख्यैकपृथकत्वगुरुत्वद्रवत्ववेगधर्माधर्मास्तूभयत्रारम्भकाः~॥}
\end{sloppypar}

{\knu संयोगेति} \textendash\ संयोगो द्रव्यं स्वाश्रये, भेरीदण्डादिसंयोगो नभसि शब्दम्, वंशदलयोर्विभागो विभागं स्वाश्रये, शब्दमन्यत्र, सङ्ख्या कारणगता कार्ये सङ्ख्यां परिमाणं च, स्वाश्रये द्वित्वादिकम्~। एकपृथकृत्वं स्वाश्रये द्विपृथक्त्वादि अन्यत्रैकपृथक्त्वमेव~। एकः ग्रहणं द्विपृथक्त्वादीनामकारणत्वप्रतिपादनार्थम्~। गुरुत्वं स्वाश्रये पतनम्, अन्यत्र गुरुत्वमेव, द्रवत्वं स्वाश्रये स्यन्दनमन्यत्र द्रवत्वमेव, वेगः स्वाश्रये गमनमन्यत्र वेगवदवयवारब्धे द्रव्ये वेगमेव~। 'वेग'शब्दैनात्र स्थितिस्थापकोऽपि ग्राह्यः~। धर्मश्च स्वाश्रये सुखमन्यत्र \renewcommand{\thefootnote}{3}\footnote{पवनादौ \textendash\ कि; क.ज्वलनादिक्रियाम् पा. ७. पु~।}ज्वलनादौ क्रियामिति~। अधर्मः स्वाश्रये दुःखमन्यत्र पवनादौ क्रियामिति~। सामान्यतश्चात्र स्नेहः प्रयत्नश्च विहितनिषिद्धगोचर उपादेयः, संयोगविशेषेण सङ्ग्रहेण स्नेहकार्येण कारणस्य स्नेहस्य धर्माधर्माभ्यां च कार्याभ्यां विहितनिषिद्धगोचरस्य प्रयत्नस्य कारणस्योपलक्षणादिति~॥

\hangindent=2cm {\knu (१०७) गुरुत्वद्रवत्ववेगप्रयत्नधर्माधर्मसंयोगविशेषाः क्रिया \textendash\ हेतवः~॥}

\newpage
{\knu गुरुत्वे}ति \textendash\ गुरुत्वात् पतनम्, द्रवत्वात् स्यन्दनम्, वेगाद्\renewcommand{\thefootnote}{1}\footnote{गमनम् \textendash\ ज~। विशेषज्ञ \textendash\ मु. कि~।}भ्रमणम्,, प्रयत्नाच्चेष्टा, धर्माधर्माभ्यां भूकम्पादि, संयोगविशेषान्नोदनाभिघातलक्षणादिष्वादिकर्म~। अत्रापि वेगेत्युपलक्षणम्; स्थितिस्थापकोऽपि ग्राह्यः~।

\begin{sloppypar}
\hangindent=2cm {\knu (१०८) रूपरसगन्धानुष्णस्पर्शसङ्ख्यापरिमाणैकपृथकत्वस्नेहशब्दानाम\renewcommand{\thefootnote}{2}\footnote{मसमवायित्वम \textendash\ कि;  ${}^\circ$नां समवायित्वं \textendash\ दे~।}समवायथिकारणत्वम्~॥} 
\end{sloppypar}

रूपादिशब्दान्तानामसमवायित्वम् \textendash\ असमवायिकारणत्वम्~। तल्लक्षणमग्रे वक्ष्यते~। सावधारणं चैतत्, तेनोष्णस्पर्शस्य पाकजोत्पत्तिं प्रति निमित्तत्वमप्यस्तीति तद्व्यवच्छेदार्थमुक्तम् अनुष्णेति~। यद्यपि च रूपादीनां धर्माद्युत्पत्तौ निमित्तत्वमप्यस्ति 'अरुणयेत्यादि श्वेतं छात्रमित्यादिदर्शनात्, तथापि बुद्धिनिमित्तत्ववत् द्रव्यादिसाधारणत्वात् तस्य तदतिरिक्तं प्रतिं निमित्तत्वमिह व्यवच्छेद्यम्~। यद्यपि च रसस्य जीवनपुष्टिबलारोग्येषु निमित्तत्वमनुपदमेव वक्ष्यति तथापि विशे\renewcommand{\thefootnote}{3}\footnote{विशेषज्ञ \textendash\ मु. कि~।}षतः शास्र्रैकसमधिगम्यत्वात् अदृष्टनिमित्तत्ववत् तदुपेक्ष्य तदितरप्रमाणगम्यनिमित्तत्वमिह व्यवच्छेद्यम्~। यद्यपि च स्पर्शवेगावपेक्ष्य क्रियाऽभिघातमुत्पादयतीति स्पर्शस्याभिघाते निमित्तत्वमस्ति तथा स्नेहस्य सङ्ग्रहे, तथाप्येकाश्रितकार्यापेक्षया एतद्द्रष्टव्यमित्यदोषः~॥

\hangindent=2cm {\knu (१०९) बुद्धिसुखदुःखेच्छाद्वेषप्रयत्नधर्माधर्मभावनानां निमित्तकारणत्वम्~॥}

बुद्ध्यादीनां भावनान्तानां निमित्तत्वम्, निमित्तकारणत्वमेव~। यथा च प्रत्यासत्तिविशेषेपि नैषामसमवायिकारणत्वं तथा वक्ष्यते~॥ 

\hangindent=2cm {\knu (११०) संयोगविभागोष्णस्पर्शगुरुत्वद्रवत्ववेगानामुभयथाकारणत्वम्~॥}

संयोगादीनां वेगान्तानामुभयथा कारणत्वम्, असमवायितया निमित्ततया च~। तथा हि भेरीदण्डादिसंयोगः शब्दे निमित्तम्; भेर्याकाशसंयागोऽसमवायिकारणम्~। वंशदलविभागो निमित्तम्, वंशदलाकाशविभागोऽसमवायिकारणं शब्दस्य~। उष्णस्पर्शः पाकजे निमित्तम्, उष्णस्पर्शेऽसमवायी~। गुरुत्वमभिघातादिक्रियायां निमित्तं पतनगुरुत्वयोरसमवायि~। द्रवत्वं सङ्ग्रहादौ निमित्तम्, द्रवत्वस्यन्दनयोरसमवायि~। वेगो वेगगमनयोरसमवायी अभिघाते निमित्तमिति~॥

\newpage
{\knu (१११) परत्वापरत्वद्वि\renewcommand{\thefootnote}{1}\footnote{द्वित्वद्विपृथक्त्वादीनां \textendash\ कं, वैशेषिकनये द्वित्वस्य द्वणुकपरिमाणेऽसमवायिकारणत्वात् तथैव भाष्ये पूर्वं सङ्ख्याया असमवायिकारणत्वेनोक्तत्वाच्च 'द्वित्व' इति कं. दे पुस्तकयोः पाठोऽसमीचीनोऽतोन स्वीकृतः~। विस्तरस्तु व्योमवत्यां (४३९) द्रष्टव्यः~।}पृथक्त्वादीनामकारेणत्वम्~।}

परत्वादीनामकारणत्वम् \textendash\ न निमित्तत्वं नाप्यसमवायित्वमियर्थः~। आदिभग्रहणेनत्रित्वत्रिपृथक्त्वादीनां ग्रहणम्~॥

\hangindent=2cm {\knu (११२) संयोगविभागशब्दात्मविशेषगुणानां प्रदेशवृत्तित्वम्~॥}

संयोगादीनां प्रदेशवृत्तित्वम्~। अत्र प्रत्यक्षाणां युगपदुपलम्भानुपलम्भप्रमाणकमेतत्~। धर्माधर्मभावनानां तु व्यापककार्यविशेषगुणत्वेनानुमेयम्~। अन्यथाऽसमवायिकारणप्रादेशिकत्वानुरोधेन सुखादीनामपि प्रदेशनियमानुपपत्तिप्रसङ्गात्~। न च भावाभावावेकत्र विरुद्धौ, प्रकारभेदेनाविरोधात्~। अवयवान्तरदिगन्तरावच्छेदेन हि तावविरुद्धौ तथैव दर्शनात्~। न चैवं व्यवच्छेदकाश्रयावेव भावाभावौ; अविरोधार्थमुपाधिमेदो न तूपाधिमेद एव तदाश्रयः~। नहि कार्यभेदेन शक्त्यशक्ती कार्यभेदमेबाश्रयेते इति देशभेदेन वा शक्त्यशक्ती देशभेदमेवाश्रयेते इति~॥

{\knu (११३) शेषाणामाश्रयव्यापित्वम्~॥}

\begin{sloppypar}
\renewcommand{\thefootnote}{2}\footnote{भाष्यमत्र पुनरुक्तमिवाभाति~।}{\knu शेषाणामाश्रयव्यापित्वम्~।} पूर्वोक्तेभ्यः संयोगादिभ्यो येऽन्ये ते शे शेषाः, तेषां रूपरसगन्धस्पर्शसङ्ख्यापरिमाणपृथक्त्वगुरुत्वद्रवत्वस्नेहवेगस्थितिस्थापकपरत्वापरत्वानामित्यर्थः~॥
\end{sloppypar}

\begin{sloppypar}
{\knu (११४) अपाकजरूपरस\renewcommand{\thefootnote}{3}\footnote{गन्धानुष्णस्पर्श \textendash\ जे~।} गन्धस्पर्शपरिमाणैकपृ\renewcommand{\thefootnote}{4}\footnote{पृथक्त्वगुरुत्व \textendash\ जे~।} थक्त्वसांसिद्धिकद्रवत्वगुरुत्वस्नेहानां यावद्द्रव्यभावित्वम्~॥}
\end{sloppypar}

अ\renewcommand{\thefootnote}{5}\footnote{अपाकजानां \textendash\ कि; क~।}पाकजादीनां स्नेहान्तानां यावद्द्रव्यभावित्वम्, यावदाश्रयमेतेषां स्थितिरविनाशः~। आश्रयविनाशादेव परमेते विनश्यन्ति न त्वन्यथ्चेत्यथः~। पाकजानि रूपरसगन्धस्पर्शद्रवत्वानि सत्येवाश्रये विनश्यन्तीत्यत उक्तमपाकजेति~। स्थितिस्थापकोऽप्यत्र ग्राह्यः~॥

{\knu (११५) शेषाणामयावद्\renewcommand{\thefootnote}{6}\footnote{भावित्वमिति \textendash\ दे; भावित्वं चेति \textendash\ मु. भा~।}द्रव्यभावित्वम्~॥}

\newpage
{\knu शेषाणामिति~।} पाकजानां \renewcommand{\thefootnote}{1}\footnote{पार्थिवरूपादीना \textendash\ जे~।}पार्थिवपरमाणुरूपादीनां पाकजस्य च पृथिवीतेजसोर्द्रवत्वस्य बुद्धिजानां च परत्वापरत्वद्वितवद्विधृथक्कादीनां शब्दबुद्ध्यादीनां च व्यापकजन्यविशेषगुणानां संयोगविभागयोश्चायावद्द्रव्यभावित्वम्, सत्यपि द्रव्ये विनाश इत्यर्थः~॥

तदेतदुद्देशमात्रमाचार्येणोक्तम्~। एवमन्यदपि स्वयमूहनीयम्~। तद्यथा रूपाद्येकैकत्यागेनारूपत्वारसत्वादीनि~। सांसिद्धिकद्ववत्वरूपरसगन्धस्पर्शस्नेहानां मूर्तविशेषगुणत्वम्~। रूपरसगन्धस्नेहगुरुत्वद्रवत्वस्थितिस्थापकानां स्पर्शव्याप्यत्वम्~। स्नेहशैत्यसांसिद्धिकद्रवत्वानां जलमात्रवृत्तित्वम्~। औष्ण्यभास्वरत्वयोस्तेजोमात्रवृत्तित्वम्~। रसगुरुत्वयोजलावनिगुणत्वम्~। गन्धमाधुर्येतररसशुक्लेतररूपाणां पृथिवीमात्रवृत्तित्वम्~। बुद्ध्यादीनां भावनान्तानामात्मविशेषगुणत्वम्, रूपद्रवत्वयोः पृथिव्यप्तेजोमात्रवृ\renewcommand{\thefootnote}{2}\footnote{गुणत्वम् \textendash\ जे~।}त्तित्वम्~। $\rightarrow$ \renewcommand{\thefootnote}{3}\footnote{$\rightarrow$ $\leftarrow$ एतच्चिह्नान्तर्गतः पाठो 'जे' पुस्तके नास्ति~।}नैमित्तिकद्रवत्वस्य क्षितितेजोमात्र वृत्तित्वम् $\leftarrow$~। नैमित्तिकद्रवत्वपार्थिवपरमाणुरूपादीनां पाकजत्वम्~। सङ्ख्यापरिमाणप्रचयानामेव परिमाणकारणत्वम्~। संयोगविभागशब्दानामेव शब्दकारणत्वम्~। स्पर्शगुरूत्वप्रयत्नवेगानामेवाभिघातहेतुत्वम्~। एकत्वैकपृथक्त्वसंयोगविभागानामेवानेकाश्रितगुणारम्भकत्वम्~। परत्वापरत्वयोरेव दिक्कालपिण्डसंयोगजत्वम्~। परत्वापरत्वचरमज्ञानद्वित्वद्विपृथक्तवादीनामेव निमित्तविनाशविनाश्यत्वम्~। एतदन्येषामेव गुणान्तरविनाश्यत्वम्~। कार्यगुणानामेवाश्रयविनाशविनाश्यत्वम्~। अपार्थिवपरमाणुरूपरसस्पर्शसासिद्धिकद्रवत्वस्नेहगुरुत्वस्थितिस्थापकनित्यद्र्यैकत्वैकपृथक्तुवपरिमाणेश्वरबुद्धीच्छाप्रयत्नानां नित्यत्वाकार्यत्वे~। इतरेषां गुणानामनित्यत्वकार्यत्वे इत्यादि~॥

\hangindent=2cm {\knu (११६) रू\renewcommand{\thefootnote}{4}\footnote{रूपादीनां प्रत्येक \textendash\ दे~।}पादीनां सर्वेषां गुणानां प्रत्वेकमपरसामान्यसम्बन्धाद् रूपादिसञ्ज्ञा भवन्ति~॥}

सम्प्रति प्रत्येकं रूपादीनां वैधर्म्यं सङ्क्षेपार्थमेकग्रन्थेनाह \textendash\ रूपादीनामित्यादि~। सर्वेषां चतुर्विशतेरपीत्यर्थः~। रूपं रसो गन्ध इत्याधिकाः सञ्ज्ञाः, ताः किं परिभाषिक्य औपाधिकी वा ? न इत्याह \textendash\ अपरसामान्यसम्बन्धात्~। रूपत्वरसत्वगन्धत्वादिसमवायात् \textendash\ निमित्तादित्यर्थः~।

तदसिद्धम्, चक्षमत्ग्राह्यत्वादयुपाधिन्न्धत्वाद्रूपादिस्ज्ञानापिति चेत्; न अननुसंहितोपाधेरुपहितप्रत्ययायोगात्~। न च रूपत्वादिसामान्यविशेषविशिष्टरूपाद्युपलब्धिमन्तरेण परस्परव्यावृत्तेषु चक्षुरादिषु \renewcommand{\thefootnote}{5}\footnote{मानमस्ति \textendash\ कि; क~।}प्रमाणमस्ति, विशिष्टोपलब्धेरेव प्रमाणत्वे परस्परा\textendash

\newpage
\noindent
श्रयत्वम्~। चक्षुराद्यवगमे विशिष्टोपलब्धिः, विशिष्टोपलब्ध्या च चक्षुराद्युपलब्धिरिति~। न चाविशिष्टोपलब्धिमात्रोपनीतमविशिष्टमिन्द्रियमात्रमुपाधिर्विशिष्टव्यपदेशे, रसादिष्वपि रूपादिव्यवहारप्रसङ्गादिति~। तस्मात् रूपत्वादिसामान्यविशेषक्रोडीकृततत्तदर्थग्रहणनियमेन चक्षुरादीन्यप्युन्नेयानि~। तथा चोच्यते रसादिषु मध्ये चक्षुर्मात्रग्राह्यस्यैव व्यञ्जकत्वादित्यर्थः स्यात्~। तथा चासिद्धमसिद्धेन साधयतो महानैयायिकत्वमित्यलमनुकम्पनीयेष्वतिनिर्बन्धेन~। एतेन रसत्वादयो व्याख्याताः~॥

\hangindent=2cm {\knu (११७) तत्र रूपं चक्षुर्ग्राह्यम्~। पृथिव्युदकज्वलनवृत्ति, द्रव्याद्यु \textendash\  पलम्भकं नयनसहकारि शुक्लाद्यनेकप्रकारम्~॥}

{\knu तत्रे}ति \textendash\ गुणेषु मध्ये; गुणत्वे सति चक्षुषैव यद् ग्राह्यं तद्रूपमिति लक्षणार्थः~। गुणत्वे सतीति रूपत्वादिव्यवच्छेदाय~। एतेन प्रमाणमाप्युक्तम्~। आश्रयमाह \textendash\ {\knu पृथिव्युदकज्वलनवृत्तीति}~। 'ज्वलन'शब्देनेह तेजोमात्रं विवक्षितम्~। त्रिष्वेव वर्तते नान्यत्रेत्यर्थः~। विषयस्यार्थक्रियामाह \textendash\ द्रव्याद्युपलम्भकम्~। 'द्रव्य'शब्देनात्र बाह्यद्रव्यमभिमतम्~। तद्रूपसम्बन्धाद् गृह्यते न त्वन्यथा~। 'आदि'शब्देन गुणकर्मसामान्यानि~। तत्र गुणाः सङ्ख्यापरिमाणपृथक्त्वसंयोगविभागपरत्वापरत्वद्रवत्वस्नेहवेगाः~। कर्माणि च रूपिद्रव्यसमवायात्~। सामान्यानि तु कानिचित् रूपैकार्थसमवायात्, कानिचित् रूपैकार्थसमवेतसमवायात्, कानिचिद्रूवसमवायादिति सर्वत्र रूपस्य कारणत्वम्~। यद्यपि वैतन्महत्त्वस्याप्यस्ति तथापि विशेषगुणत्वे सतीति द्रष्टव्यमित्यदोषः~। कारणोपकारितामाह \textendash\ {\knu नयनसहकारीति}~। बाह्यालोकाप्यायितं हि चक्षुः स्वविषयग्रहणे क्षमम्~। स चोद्भूतरूपः चक्षुषः सहकारी न त्वन्यथेति रूपं नयनसहकारीत्युच्यते~। न तु स्वीयमेव रूपमस्य सहकारि सगुणानामिन्द्रियभावादिति न्यायेन तस्येन्द्रियशरीरत्वात्~। ${}^2$विधामाह \textendash\ {\knu शुक्लाद्यनेकप्रकारम्}~। शुक्लत्वादयोऽनेके प्रकारा अवान्तरविशेषा तस्य तत्तथोक्तम्~।

\hangindent=2cm {\knu (११८) सलिलादिपरमाणुषु नित्यं पार्थिवपरमाणुष्वग्निसंयोग \textendash\  विरोधि, स3र्वकार्थेषु का4रणगुणपूर्वकम्, आश्रय \textendash\ विनाशादेव विनश्यतीति~।}

यद्यपि नित्यत्वादिकं गुणान्तरसाधारणं तथापि पार्थिवपरमाणुषु पाकजत्ववैधर्म्यचिन्तायामुपयोक्ष्यत इत्यभिप्रायवान् कारणविरोधिनावभिधातुं नित्यं तावन्निर्धारयति \textendash\

\blfootnote{I एतेष्वेव \textendash\ जे~। 2 विभाग \textendash\ जे~। 3 सर्वकार्यद्रव्येषु \textendash\ मु. भा~। 4 कारणपूर्वकम् \textendash\ दे~।}

\newpage
\noindent
{\knu सलिलादिपरमाणुषु नित्यमिति}~। आदिशब्देन तेजः परमाणुपरिग्रहः~। अनित्यता हि कार्यतया व्याप्ता, सा च रूपादीनां कार्यद्रव्येषु विचित्रतया व्याप्ता, सा च सलिलादिपरमाणुरूपाद् व्यावर्तमाना स्वव्याप्त्या कार्यतामुपादाय विनाशितामपि निवर्तयतीति~। {\knu पार्थिवपरमाणुष्वग्निसंयोगविरोधि} \textendash\ अग्निसंयोगो विरोधी यस्य तत्तथा~। उपलक्षणं चैतत्, उत्पादकोप्यस्याग्निसंयोग एव~। {\knu सर्वकार्येषु कारणगुणपूर्वकम्} \textendash\ अवयविषु यद्रूपं तदवयवरूपपूर्वकमित्यर्थः~। {\knu आश्रयविनाशादेव विनश्यतीति} \textendash\ न तु पिठरपाकवादिनामिवाग्निसंयोगादपीत्येवकारार्थः~।

एतेन गुणगुणिनोरभेदवादो निरस्तः, फलतो विरुद्धधर्माध्यासस्य दर्शितत्वात्~। तथाहि रूपत्वनिमित्ता रूपसञ्ज्ञेति दर्शिते, पृथिवीत्वनिमित्ते च पृथिवीव्यवहारे एतयोस्तादात्म्ये पर्यायता स्यात्~। न च 'रूप'शब्दात् 'शुक्ल'शब्दाद् वा कश्चित् ${}^1$पटादिकं प्रत्येति ${}^2$पटादिशब्दाद्वा शुकादिकम्~। न चाभेदे सामान्यविशेषभावोऽप्युपपद्यते~। न च सामान्यविशेषभावेऽप्यभेदादेव सामान्यं विशेषश्चेति सम्भवति, विरोधात्~। अनुवृत्तो हि धर्मः सामान्यं व्यावृत्तो विशेषः; न च तदेवानुवृत्तं व्यावृत्तं चेति सम्भवति~। किञ्च रूपं चक्षुर्मात्रग्राह्यम्, तदेव चेद् द्रव्यं रूपवद्द्रव्यमप्यन्धेनं नोपलभ्येत घटादिद्रव्यवद्वा शुक्लादिरूपमप्यन्धेनोपलभ्येत~। यदि च पृथिव्येव रूपं न तोये तेजसि वा वर्त्तेत~। यदि वा पृथिवी, पयःपावकावपि पृथिव्येव स्यातां सामान्यविशेषभावे वा तादात्म्यं विरुध्येत~। न चैवं द्रव्यस्यैव सामान्यविशेषो रूपमस्तु कृतं गुणेनेति वाच्यम्, तस्य प्रकाराभावात्~। नहि गोत्वं प्रकारवत्~। रूपं तु शुक्लाद्यनेकप्रकारकम्~। न च गोत्वविशिष्टस्य शाबलेयादिभेदवत् रूपविशिष्टस्य शुक्लादिभेदो भविष्यतीति युक्तम्, तत्रापि शुक्लतरशुक्कतमादिप्रकारवत्त्वात्~। न च तारतम्यं सामान्यस्य सम्भवति~। नहि सम्भवति गोतरो गोतम इति~। किञ्च पिठरपाकोऽप्यस्तु, परमाणुपाको वा, शुक्लादिरूपं तावद्धर्मिणि सत्येव निवर्तते प्रवर्तते चेति सुप्रसिद्धम्~। न चैतत् सम्भवति तस्य नित्यैकरूपत्वात्; अन्यथा तत्त्वविरोधात्~।

एतेन रूपवदेव द्रव्यमुत्पद्यते रूपवदेव विनश्यतीति नास्त्यनयोः पूर्वापरभाव इति, नापि द्रव्यविनाशरूपविनाशयोरिति निरस्तम् एकत्र दर्शनेनान्यत्र तदनुमानात्~। तथा हि अवयविरूपं समवायिकारणवत्  कार्यत्वाद् घटवत्~। अन्यथा कारणमात्रानपेक्षतया कादाचित्कत्वानुपपत्तेः~। कार्यं च तत् कार्यगुणत्वात् पार्थिवरूपत्वाद्वा सामान्यवत्त्वे सति अवयव \textendash\

\blfootnote{I घटादिकं \textendash\ कि; क~। 2 घटशब्दाद्वा \textendash\ कि; क~।}

\newpage
\noindent
\textendash\ विधर्मत्वाद्वा सामान्यवत्त्वेऽसति अस्मदादिबाह्यकरणप्रत्यक्षाद्वा अस्मादादिप्रत्यक्षविशेषगुणत्वाद्वेति संयोगवत्, पार्थिवपरमाणुरूपवत्, कर्मवत्, घटवत् बुद्धिवद्वेति~। अन्यथाऽवयविनाशेप्यविनाशात् तदुपलम्भापत्तिः~। आश्रयस्य व्यञ्जकस्याभावादनुपलम्भे सत्याश्रये परमाणौ तन्निवृत्तेरनुपपत्तिरिति~। तस्मादवयविनो रूपस्य चोपादानोपादेयभावात् पूर्वापरभावनियम एवेति स्थितम्~। विनाशस्यापि सहेतुकत्वस्थितौ यावद्द्रव्यभावस्य \renewcommand{\thefootnote}{1}\footnote{I 'च' कि पुस्तके नास्ति~।}च कार्यरूपादीनामुपलम्भात् द्रव्याभावे चानुपलम्भनियमात् द्रव्यविनाशकार्यत्वमवसीयते~। यौगपद्याभिमानस्तु कालसन्निकर्षाच्चक्षुरुन्मेषरूपोपलब्धिवदिति सर्वमवदातम्~॥

\hangindent=2cm {\knu (११९) रसो रसनग्राह्यः~। पृथिव्युदकवृत्तिः, जीवनपुष्टिबलारोग्यनिमित्तम्, रसनसहकारी, भधुराम्ललवणतिक्तकटुकषायभेदभिन्नः~॥}

रसत्वाभिसम्बन्धाद्रस इति स्थिते सत्याह \textendash\ {\knu रसो रसनग्राह्य} इति~। रसनेनैव गृह्यते यो गुणः स रस इति \renewcommand{\thefootnote}{2}\footnote{प्रमाणलक्षणे \textendash\ कि~।}प्रमाणं लक्षणं च~। आश्रयमाह \textendash\ {\knu पृथिव्युदकवृत्तीरिति~।} यद्यपि गुरूत्वमप्येवं तथापि प्रत्यक्षत्वे सतीत्यदोषः~। अर्थक्रियामाह \textendash\ जीवनपुष्टिबलारोग्यनिमित्तम्~। जीवनं मनसः शरीरेऽवस्थानम्, पुष्टिः शरीरोपचयः, बलमायासक्षमता, आरोग्यं सुस्थता दोषधातुमलाग्नीनां समतेति यावत्~। उपलक्षणं चैतत्; मरणकृशत्वदौर्बल्य\renewcommand{\thefootnote}{3}\footnote{रोगरोगकारणान्यपि \textendash\ जे~।}रोगकारणान्यपि द्रष्टव्यानि~। तेषां निमित्तम्~। यद्यपि गन्धादयोऽप्येवम्भूतास्तथा संयोग\renewcommand{\thefootnote}{4}\footnote{संयोगभेदाश्च \textendash\ कि. क~।}स्पर्शभेदाश्च तथापि रसं प्रधानीकृत्य न तु स्वातन्त्र्येणेति विवक्षितम्~। बाह्यस्यार्थक्रियामाह \textendash\ {\knu रसनसहकारी~।} बाह्येनोदकरसेन स्तिमितस्य द्रव्यस्य रसो रसनेन गृह्यते न तु शुष्कस्य~। विभागमाह \textendash\ {\knu मधुरेति~।} मधुरादिभिस्तद्भेदैश्च मधुरतरमधुरतमादिभिर्भिन्नो नानेत्यर्थः~॥

{\knu (१२०) अस्यापि नित्यानित्यत्वनिष्पत्तयो रूपवत्~॥}

अस्यापीति \textendash\ नित्येति भावप्रधानो निर्देशः~। यथा रूपं सलिलपरमाणुषु नित्यं तथा रसोऽपि~। यथा रूपं पार्थिवपरमाणुष्वभिसंयोगविरोधि तदुत्पाद्यं च तथा रसोऽपि~। यथा रूपं कार्यद्रञ्येषु कारणगुणपूर्वकमाश्रयविनाशाद्विनश्यति तथा रसोऽपीत्यर्थः~। तथैव चैतत्सर्वमुपपादनीयमित्यर्थः~। एतेन कारणविरोधिनावुक्तौ वेदितव्यौ~॥

\newpage
\hangindent=2cm {\knu (१२१) गन्धो घ्राणग्राह्यः, पृथिवीवृत्तिः, घ्राणसहकारी सु\renewcommand{\thefootnote}{1}\footnote{I स द्विविधः सुरभिरसुरभिश्च \textendash\ जे~।}रभिरसुरभिश्च~॥}

गन्धत्वाभिसम्बन्धाद् गन्ध इति स्थिते सत्याह \textendash\ {\knu गन्धो घ्राणग्राह्यः~।} गुणेषु मध्ये यो घ्राणेनैव गृह्यते स गन्ध इत्यर्थः~। एतेन लक्षणप्रमाणे दर्शिते~। {\knu पृथिवीवृत्तिः} \textendash\ पृथिव्यामेव वर्तते नान्यत्रेत्याश्रय उक्तः~। {\knu घ्राणसहकारी}त्यर्थक्रिया~। घृतादिगन्धसहकृतेन घ्राणेन मृगमदादिगन्धग्रहणादेतदुन्नेयम्~। {\knu सुरभिसुरभि}श्चेत्यवान्तरभेदः~। असुरभिरिति \renewcommand{\thefootnote}{2}\footnote{सुरभिविपरीतं \textendash\ मु. कि; क~।}सौरभविपरीतं सामान्यविशेषमाह न तु तदभावमात्रमधर्मवत्~। {\knu अस्यापीति} \textendash\ अस्य नित्यत्वं नास्तीत्यतस्तद्विहायोत्पत्त्यादय इत्युक्तम्~। उत्पत्तिरादिर्येषाम्, उत्पत्तितद्धेतुविनाशतद्धेतूनां ते तथोक्ताः~। तेनोत्पत्तिर्द्वयी परमाणौ कार्ये च~। कारणमपि द्वयम्, अग्निसंयोगोऽवयवगुणश्चेति~। विनाशोऽपि द्विविधः परमाणौ कार्येच~। तत्कारणमपि द्विविधम्, अग्निसंयोगो द्रव्यविनाशश्चेति बहुवचनोपपत्तिः~। {\knu पूर्ववदिति} \textendash\ रसवदित्यर्थः~॥

\hangindent=2cm {\knu (१२२)स्पर्शस्त्वगिन्द्रियग्राह्यः~। क्षि\renewcommand{\thefootnote}{3}\footnote{पृथिव्युदक \textendash\ जे. व्यो, (४४४)~।}त्युदकज्वलनपवनवृत्तिः, त्वक्सहकारी रूपानुविधायी, शीतोष्णानुष्णाशीतभेदात्त्रिविधः~॥}

स्पर्शत्वाभिसम्बन्धात् स्पर्श इति स्थिते सत्याह \textendash\ {\knu स्पर्शस्त्वगिन्द्रियग्राह्यः~।} त्वगिन्द्रियेणैव गृह्यते यो गुणः सस्पर्श इत्यर्थः~। \renewcommand{\thefootnote}{4}\footnote{$\rightarrow$ $\leftarrow$ एतच्चिह्नान्तर्गतः \textendash\ पाठः 'जे' पुस्तके नास्ति~।}$\rightarrow$ आश्रयमाह $\leftarrow$ \textendash\ {\knu क्षित्युदकज्वलनपवनवृत्तिः~।} यद्यपि स्थितिस्थापकोऽप्येवं तथापि विशेषगुणत्वे सतीत्यदोषः~। {\knu त्वक्सहकारी} \textendash\ व्यजनादिवातस्पर्शसहकृतेन त्वगिन्द्रियेण उदकस्पर्शस्योपलम्भादेतदवसेयम्~। तदेतद्रूपादिचतुष्टयं विषयसंस्कारकतया नयनादिसहकारि~। {\knu रूपानुविधायी} \textendash\ रूपमनुविधातुं शीलमस्येति रूपानुविधायी~। यत्र रूपं तत्रावश्यं स्पर्शो भवति~। रूपव्यापक इति यावत्~। अथवा रूपमनुविधायि यस्य स तथोक्तः~। रूपं स्पर्शमनुविधत्ते स्पर्शपरतन्त्रं स्पर्शेन विना न भवतीत्यर्थः~। अत्रापि रूपव्यापकत्वं स्पर्शस्य लभ्यत इति फलतो न कश्चिद्विशेषः~। यद्यपि सङ्ख्यापरिमाणादयोऽपि रूपव्यापकास्तथापि विशेषगुणत्वे सतीत्यदोषः~। अवान्तरविभागमाह \textendash\ शीतेति~।

{\knu (१२३) अ\renewcommand{\thefootnote}{5}\footnote{तस्यापि \textendash\ जे~।}स्यापि नित्यानित्यत्वनिष्पतयः पूर्ववत्~॥}

\newpage
कारणविरोधिनावतिदेशेन दर्शयति {\knu अस्यापीति~।} अत्रापि नित्येति भावप्रधानो निर्देशः~। {\knu पूर्ववदिति} \textendash\ रसवदित्यर्थः~। गन्धे नित्यत्वाभावाद् यथा रसः सलिलपरमाणुषु नित्यस्तथास्पर्शोऽप्यपार्थिवपरमाणुषु~। यथा रसः पार्थिवपरमाणुष्वग्निसंथोगविरोधी तदुत्पाद्यश्च, अवयवविष्ववयवगुणपूर्वकः आश्रयविनाशादेव विनश्यति तथा स्पर्शोऽपीत्यर्थः~॥

\hangindent=2cm {\knu (१२४) पार्थिव परमाणुरूपादीनां पाकजोत्पत्तिविधानम्~॥}

अथेदानीं पाकजेषु परीक्षाशेषं वर्तयिष्यन्नाह \textendash\ {\knu पार्थिवेति~।} पृथिव्याः कार्यभूताया इमे कारणभूताया परमाणवः पार्थिवाः~। अथवा पृथिव्याः पृथिवीत्वस्याश्रयभूता इमे इति पार्थिवाः~। तेषां ये रूपादयो रूपरसगन्धस्पर्शाः, तेषां पाकजत्वेन रू\renewcommand{\thefootnote}{1}\footnote{I रूपेण इति \textendash\ कि. क. पुस्तकयोर्नास्ति~।}पेण यदु\renewcommand{\thefootnote}{2}\footnote{यदुत्पत्तिविधानं \textendash\ कि. क~।}पत्तर्विधानं प्रकारः स कथ्यते~। तत्र विप्रतिपत्तौ संशयः; केचिदाहुः कार्यकारणसमुदाये पच्यमाने तस्मिन्नवस्थित एव रूपादयो विनश्यन्ति जायन्ते चेति~। केचित्तु परमाणव एव स्वतन्त्राः पच्यन्ते, कार्यं तु पूर्वं निवर्तते, अपूर्वं प्रवर्तते~। तत्र कारणगुणपूर्वक्रमेण रूपादयो जायन्त इति~। प्रयोजनन्तु धर्मधर्मिणोः साधर्म्यवैधर्म्याभ्यां स्फुटं तत्त्वविवेकः, स च सम्य गात्मविवेके उपयोक्ष्यत इति~।

\hangindent=2cm {\knu (१२५) घटादेरामद्रव्यस्या\renewcommand{\thefootnote}{3}\footnote{ ${}^\circ$स्याग्निसंयुक्तस्य \textendash\ दे~।}ग्निना सम्बद्धस्याग्नयभिघातान्नोदनाद्वा तदारम्भकेष्व\renewcommand{\thefootnote}{4}\footnote{परमाणुषु \textendash\ दे~।}णुषु कर्माण्युत्पद्यन्ते~। तेभ्यो विभागाः, विभागेभ्यः संयोगविनाशाः, संयोगविना\renewcommand{\thefootnote}{5}\footnote{विनाशेभ्यः कार्यद्रव्यं \textendash\ दे~।}शेभ्यश्च कार्यद्रव्यं विनश्यति~॥}

तत्र परमाणव एव स्वतन्त्राः पच्यन्त इति पक्षमुपादाय व्युत्पादयति {\knu घटदेरित्यादि~।} आदिग्रहणेन सर्वपार्थिवोपदानम्~। {\knu आमेति} \textendash\ प्रकृतपाकापेक्षया न तु पार्थिवमपक्वं नामास्ति~। अग्निना सम्बद्धस्य संयुक्तस्य तदारम्भकेषु घटाद्यारम्भकेषु परम्परायाऽणुषु कर्माण्युत्पद्यतते~। {\knu आरम्भकेष्विति} कार्यविनाशादिप्रक्रियासिद्ध्यर्थम्~। कुतः ? {\knu अग्नयभिघातान्नोदनाद्वा} \textendash\ अग्निनैव नोदनाभिधातौ संयोगविशेषौ
कर्माधिकारे वक्ष्येते~। तेभ्यः \textendash\ कर्मभ्यो विभागाः, विभागेभ्यः
संयोगविनाशाः \textendash\ द्रव्यारम्भकसंयोगविनाशाः, तेभ्यः कार्यद्रव्यं विनश्यति
द्व्यणुकलक्षणम्~। जात्यभिप्रायेणैकवचनम्~। 

\newpage
अत्र नोदनाद्युत्पत्तौ कर्मोत्पत्तिः प्रमाणम्~। कर्मसु विभागाः, तेषु पूर्वसंयोगनिवृत्तिः, तत्र कार्यद्रव्यविनाश इत्यस्तु~। एषां कार्यकारणभावावधारणात् कारणं विना च कार्यानुत्पत्तेः~।

\hangindent=2cm {\knu (१२६) तस्मिन् विनष्टे स्व\renewcommand{\thefootnote}{1}\footnote{I 'स्वतन्त्रेषु' इति जे, व्यो, दे पुस्तकेषु नास्ति~।}तन्त्रेषु परमाणुष्वग्निसंथोगादौष्ण्यापेक्षाच्छ्यामादीनां विनाशः~। पुनरन्यस्मादग्निसंयोगादौष्ण्यापेक्षात् पाकजा जायन्ते~॥}

कार्यद्रव्यं \renewcommand{\thefootnote}{2}\footnote{${}^{\circ}$द्रव्यं विन$^\circ$ कि~।}तु विनश्यतीत्यत्र किं प्रमाणम् ? न हि सति द्रव्ये रूपादीनां विनाशोत्पत्ती नोपपद्येते परमाणुष्विव कार्यद्रव्येऽपि तयोरविरोधादित्यत आह {\knu तस्मिन्निति~।} न हि चन्द्रचक्षुपोस्तेजस्तथा पावकं यथा सौरमित्युक्त{\knu मौष्ण्यापेक्षादिति~।} पृथिव्याः श्यामं रूपमनादीति केचिदविचारितागम \textendash\ श्रद्धालवो लपन्ति, तान् प्रत्युक्तं {\knu पुनरि}ति~। तेन जन्मनोऽभ्यासं वदन् श्यामादीनामुत्पत्तिमत्त्वं \renewcommand{\thefootnote}{3}\footnote{वदति \textendash\ जे~।}दर्शयति, अन्यथा \renewcommand{\thefootnote}{4}\footnote{अत्र 'विनाशानुपपत्तेः' इत्यनन्तरं 'पाकजा इति रूपसगन्धस्पर्शाः संक्षेपेण उक्ताः" इत्यधिकः पाठः कि. क पुस्तकयोः; अत्रानेनं पाठेन न भाव्यमतो न स्थापितः \textendash\ सं~।}विनाशानुपपत्तेः~। 

यद्यप्येकस्योत्पादविनाशहेतुत्वमेकजातीयत्वेऽपि कार्यस्य व्यक्तिभेदेन मध्यमशब्दवत् कालभेदेन चोपान्त्यशब्दवदुपपद्यते, तथापि पाकजेषु नायं विधिः, कार्यविनाशविशिष्टं कालमपेक्ष्य रूपादिविनाशात्, रूपादिविनाशविशिष्टं च कालमपेक्ष्य रूपाद्युत्याददर्शनात्~। सति च कार्यावष्टम्भद्रव्यस्यैव प्रतिबन्धकत्वात्~। सति च रूपादौ रूपाद्यन्तरानुत्पत्तेः~। न चैतावन्तं कालमेकोऽम्निसंयोगोनुवर्ततितुमीष्टे, तेजसोऽविलम्बेन गमनशीलत्वात्, तेजोन्तरेणाभिहन्यमानत्वान्नुद्यमानत्वाच्च~। अपि चोत्पादकस्यैव विनाशकत्वेऽग्निसन्तानव्यावृत्तौ रूपादिसन्तान\renewcommand{\thefootnote}{5}\footnote{${}^{\circ}\hbox{व्यावृत्तौ}$ घटादेनीरूपत्वप्रसङ्गः~।}व्यावृत्तिप्रसङ्गः~। विनाशकस्यैवोत्पादकत्वेऽग्निसंयोगस्यैकत्वादविशेषेण पाकतारतम्यस्यानुपपत्तिः~। पुनस्तेनैव रूपादिविनाशात् पुनरुत्पादकस्य विनाशकत्वे पूर्वदोषननिवृत्तिः~। तस्मात् परमाणुषु रूपादीनां विनाशोत्पादौ नैकाग्निसंयोगकारणकौ रूपाद्युत्पत्तिविनाशत्वात् श्यामाद्युत्पत्तिविनाशवदिति~।

{\small तदिदमुक्त{\knu मन्यस्मादि}ति~। एतेन पाकेन पूर्वरूपादिपरावृत्तिरेवावयविनाशे प्रमाणम्~। अवयविरूपादिविनाशस्याश्रयविनाशहेतुत्वावधारणात्, कारणं च विना कार्यानुत्पत्तेः~। ननु सत्यप्यवयविन्यग्निसंयोगादपि रूपादिविनाशः स्यादित्युक्तम्; न; कतिपयावयवपाके कार्ये कारणरूपानुकारानुपपत्तेः~। नहि श्यामा अवयवाः, अवयवी च रक्त इति सम्भवति~। निःशेषःपाके तु आपरमाणुतेजःसम्बन्धे तेजसश्च \renewcommand{\thefootnote}{6}\footnote{I स्पर्शवत्तया \textendash\ कि. क~।}गत्वरतया वेगवत्तया च }

\newpage
\noindent
नोदनाभिघातयोरन्यतरः संयोगोऽवरयवविश्लेषहेतुक्रियाजनक इति कथं न कार्यद्रव्यविनाशः ? किञ्च रूपाद्यन्तरोत्पत्तिरप्यवयञ्यन्तरोपादे प्रमाणम्~। तद्रूपादेः कारणगुणपूर्वकत्वनियमात्~। अग्निसंयोगान्नष्टे रूपादौ तत्रावयविनि कारणगुणप्रक्रमेण रूपादिर्मविष्यतीति अदोष इति चेत्; न; यत्राग्निसंयोगो रूपादीनां नोत्पादकस्तत्र विनाशकोऽपि न स्यात् जलादिवदिति नियमात्~। अग्निसंयोगस्यैव तदुत्पादकत्वे कारणगुणपूर्वकत्वव्याघातात्~। व्याहन्यतामिति चेत्, न; आमदशायां श्यामादीनामनुत्पादप्रसङ्गात्~। $\rightarrow$ \renewcommand{\thefootnote}{1}\footnote{I $\rightarrow$ $\leftarrow$ एतच्चिह्नान्तर्गतः पाठः कि. क. पुस्तकयोर्नास्ति~। लेखकप्रमादात् पूर्वत्रायोग्यस्थल एव स्थापित आसीत् किन्त्वत्रानेन पठेन भाव्यम् \textendash\ सं~।}{\knu पाकजा} इति रूपरसगन्धस्पर्शाः संक्षेपेण उक्ताः $\leftarrow$~।

तस्मात् परमाणुषु पाकादेव रूपादीनां विनाशोत्पत्ती अवयविनि च कारणगुणेभ्य एवोत्पादः, आश्रयविनाशादेव विनाश इति परमुपपद्यते~। तदमी प्रयोगाः; पार्थिव्रावयववरूपादि, आश्रयविनाशादेव विनश्यति, अवयविरूपादित्वात् दग्धपटरूपादिवत्~। तत् कारणगुणेभ्य एवोत्पद्यते, \renewcommand{\thefootnote}{2}\footnote{अत एव तदेव \textendash\ कि~।}तत एव तद्वदेव~। घटादिपरमाणुष्वग्निसंयोगाः नियमेन द्रव्यारम्भकसंयोगप्रतिद्वन्द्विविभागजनकक्रियाहेतवः, अग्निसंयोगत्वात् \renewcommand{\thefootnote}{3}\footnote{मुषाग्नि \textendash\ संयोगवदिति \textendash\ मु. कि~।}तुषाग्निसंयोगवदिति~। 

{\knu (१२७) तदनन्तरं भोगिनामदृष्टापेक्षादात्माणुसंयोगादुत्पन्नपाकजेष्वणुषु कर्मोत्पत्तौ तेषां परस्पर\renewcommand{\thefootnote}{4}\footnote{संयोगेभ्यो \textendash\ दे~।}संयोगाद् द्व्यणुकादि\renewcommand{\thefootnote}{5}\footnote{प्रक्रमेण \textendash\ जे~।}क्रमेण कार्यद्रव्यमुत्पद्यते~। तत्र च कारणगु\renewcommand{\thefootnote}{6}\footnote{गुणप्रक्रमेण \textendash\ मु. भा; व्यो. (४४८)~।}णपूर्वप्रकमेण रूपाद्युत्पत्तिः~॥}

स्यादेतत्; यद्येवमापरमाण्वन्तो विनाशः स्वतन्त्राश्च परमाणवः पच्यन्ते, कुलालादिव्यापारमन्तरेण घटादीनां पुनरावृत्तौ लोकव्यवहारोच्छेदः~। व्यर्थश्चैषां प्राच्यः प्रयास इत्यत आह \textendash\ {\knu तदनन्तरमिति~।} रूपाद्यृत्पत्त्यन्तरं भोगिनामुदकाद्याहरणद्वारेण तत्साध्यसुखदुःखाद्यनुभवभागिनाम्~। {\knu अदृष्टापेक्षात्} \textendash\ धर्माधमपिक्षात्~। आत्माणुसंयोगादित्यत्र 'आत्म'ग्रहणं स्पष्टार्थम्; भोगनामित्यनेनैवात्मोपादानात्~। {\knu अणुष्विति} \textendash\ परमाणुष्वित्यर्थः~। कर्मणामुत्पत्तौ सत्याम्, {\knu तेषां} \textendash\ परमाणूनाम्, परस्परमन्योन्यम्~। {\knu संयोगा}दिति \textendash\ जात्यभिप्रायमेकवचनम्~। द्व्यणुकमादिर्यस्य प्रक्रमस्य~। तेन कार्यद्रव्यं घटादिलक्षणमुत्पद्यते~। तत्र च कारणगुणप्रक्रमेण रूपाद्युत्पत्तिः~। परमाणुरूपादेर्द्व्यणुकेषु रूपादिकम्, द्व्यणुकरूपादेस्त्र्यणुकेषु रूपादिकमित्यनेन क्रमेणेति~।

\newpage
अत्र च कारणचतुष्टयस्यापि भावाद्रूपादेव रूपमुत्पद्यते न स्पर्शत्, वायावपि तदुत्पत्तिप्रसङ्गात्~। रसादेस्तदुत्पत्तौ तदुद्भवाद्यनु\renewcommand{\thefootnote}{1}\footnote{I म विधानापत्तेः \textendash\ कि~।}विधानानुपपत्तेः~। एवं रूपाद्रसोत्पत्तौ तेजसि रसोत्पत्तिप्रसङ्गात्~। गन्धात्तदुत्पत्तौ तदनुकारापत्तेः, अवयवकदुत्वाद्यनुप्रसङ्गाच्चेत्यादि स्वयमूहनीयम्~। तदेवं रूपाद्यन्तरोत्पत्तिः स्थूल्द्रव्योत्पत्तौ प्रमाणम्, सा तदवयवपरम्परायाम्, सा द्व्यणुके; तदुत्पत्तिः परमाणुसंयोगे \renewcommand{\thefootnote}{2}\footnote{स च कि; क~।}परमाणुसंयोगश्च तत् क्रियायाम्, साऽदृष्टवदात्मसंयोगे रूपादिमत्तायां चेति द्रव्यारम्भानुगुणावयवक्रियाया रूपस्पर्शादिमत्युत्पत्तिनियमात् रूपादिविरहिणो द्रव्यस्य क्षणमात्रस्थितेः कर्मणश्चे कारणान्तरापेक्षत्वादिति~।

{\knu (१२८) न च \renewcommand{\thefootnote}{3}\footnote{कार्यद्रव्य एव रूपाद्युत्पत्तिर्विनाशो वा \textendash\ मु. भा~।}कार्यद्रव्ये रूपादिविनाश उत्पत्तिर्वा सम्भवति, सर्वावयवेष्वन्तर्बहिश्च वर्तमानस्याग्निना व्याप्त्यभावात्~। अणुप्रवेशादपि व्याप्तिर्नसम्भवति, कार्यद्रव्यविनाशादिति~॥}

अथ पूर्वं रूपादिनिवृत्तिरुत्तररूपान्तराद्युत्पत्तिश्च कार्यद्रव्य एव यदि स्यातां को दोष इत्यत्र तर्क\renewcommand{\thefootnote}{4}\footnote{तर्कशेषमाह \textendash\ पा, ७. पुः~।}विशेषमाह \textendash\ {\knu न चेति~।} किमिति न सम्भवतीत्यत आह \textendash\ {\knu सर्वावयवेष्विति~।} अवयविनः प्राप्तावप्यन्तर्वर्तिनामवयवानां बाह्यावयवावष्टम्भेनाग्निप्राप्तेरसम्भवात् पाको न स्यादित्यर्थः~। सान्तराण्येवावयविद्रव्याणि, कथमन्यथा मध्यस्थानामपां स्यन्दनप्रस्रवणे, तथा चाणुप्रवेशो न विरुद्ध्यत इत्यत आह \textendash\ {\knu अणुप्रवेशादपीति~।} कुतः ? {\knu कार्यद्रव्यविनाशादिति~।}

अयमाशयः~। ईदृशो हि तेजसो लाघवातिशयेन वेगातिशयः स्पर्शतिशयश्च यत् तज्जन्यं कर्म, कार्यद्रव्यं पूर्वव्यूहात् प्रच्यावयति~। तदवयवाँश्च व्यूहान्तरं प्रापयति~। अन्यथा सान्तरत्वेऽन्तरालेन प्रविशति पावके क्वथ्यमानाः क्षीरनीरादयो नोर्ध्वं \renewcommand{\thefootnote}{5}\footnote{तापयेरन् \textendash\ जे~। १६}ध्मापयेरन्~। मृदुसंयोगात्त्थेति चेत्; न; तन्दुलदीनामपि तथा दर्शनात्~। अतिदृढानामप्युपलमणिवज्रादीनाममिदग्धानां स्फुटनात्~। सातिशयोऽग्निसंयोगस्तथेति चेत्; कोऽतिशयार्थः ? बह्रभ्यास इति चेत्, न; प्रथमस्याकिश्चित्करत्वेऽभ्यासानुपयोगात्~। न ह्याशुविनाशिनां क्रमभाविनां स्वरूपतः समुच्चयः सम्भाति~। तस्माद्यथा शरीरादौ प्रत्यहमनुपलक्षणीयः कालान्तरे स्फुटीभूतो विशेषः प्रतीयते तथा घटादिपाकेऽपीति युक्तमुत्पश्यामः~।

\newpage
एतेन प्ररत्यभिज्ञानं सर्वावस्थादर्शनमुपरिनिहितमूर्त्तान्तरधारणमावरणसंस्थानसङ्ख्यापरिमाण\renewcommand{\thefootnote}{1}\footnote{I परिमाणरेखादि \textendash\ कि~।}रेखोपरेखादिचिह्नाद्यविपर्ययः पूर्वद्रव्याविनाशे प्रमाणमिति निरस्तम्~। सूचिव्यतिभेदेन विदलितत्रिचतुरत्रसरेणुधटादिवदुपपत्तेः~। तथापि न नश्यन्त्येव घटादय इति {\knu मीमांसक}दुर्दुरूढाः, तेऽनुकम्पनीयाः~। तथा हि परमाण्वपगमे द्व्यणुकमवश्यं नश्यतीति; अन्यथा कार्यनित्यताप्रसङ्गात्~। तन्नाशे तु त्रसरेणवोऽवश्यं विनश्यन्तिं~। तन्नाशे तत्कार्यमीत्यनेनैव क्रमेणान्त्यावयविपर्यन्तः प्रलयः, अन्यथा निराश्रयकार्यावस्थानप्रसङ्गात्~। कतिपयावयवनाशेप्यवस्थितावयवाश्रयं कार्यं स्यात्~। न स्यात्, तावदवयवव्याप्तियोग्यपरिमाणवन्तो \renewcommand{\thefootnote}{2}\footnote{अल्पेषु \textendash\ पा. ७. पु~।}न्यूनेष्वसंमितेः~। संवृतपटवत् परिमाणसङ्कोचो भविष्यतीति चेत्, न भवेत्; स हि न तयोर्भेदोऽमेदस्य द्रव्यधर्मत्वात्~। न विनाशः, परिमाणस्य यावदूद्रव्यभावित्वात्स्वाश्रयविनाशैकविनाश्यत्वाच्चेति~। स्वाश्रयावयवविक्लेषविनाशाभ्यां विनाश्यमपि भविष्यतीति चेत्, न, अवयवविश्लेषविनाशाभ्यामवयवी न विनश्यति तत् परिमाणं तु विनश्यतीति महती प्रत्याशा~। अवयविनः प्रत्यभिज्ञानादिति चेत्, क्षुद्रावयवविश्लेषं परिमाणं किं न प्रत्यमिज्ञायते ? इत्यलमतिपीडया~।

अथ द्व्यणुकनाशमारभ्य कतिभिः क्षणैः पुनरन्यद्व्यणुकमु्वव रूपादिमद्भवतीति शिष्यजिज्ञासायां शिष्यबुद्धिवैशद्याय प्रक्रिया~। तद्यथा नोदनादिक्रमेणं द्व्यणुकनाशः (१), नष्टे द्वयणुके परमाणावग्निसंयोगात् श्यामादीनां निवृत्तिः (२), निवृत्तेषु श्यामादिषु पुनरन्यस्मादग्निसंयोगाद् रक्तादीनामुत्पत्तिः (३), उत्पन्नेषु रक्तादिषु $\rightarrow$ उत्त\renewcommand{\thefootnote}{3}\footnote{3, 8, 9 $\rightarrow$ $\leftarrow$ एतच्चिह्नन्तर्गताः सर्वेऽपीमे पाठाः 'जे' पुस्तके न सन्ति~।}रसंयोगात् पूर्वक्रियानिवृत्तिः (४), ततो $\leftarrow$ ऽदृष्टवदात्माणुसंयोगात् परमाणौ \renewcommand{\thefootnote}{4}\footnote{द्रव्यारम्भणाय \textendash\ पा. ७. पु, जे~।}द्वयणुकोरम्भणाय क्रिया क्रियया पूर्वदेशात् विभागः (५), विभागेन च \renewcommand{\thefootnote}{5}\footnote{संयोगनिृतिस्तस्मिन्निवृत्ते \textendash\ पा, ७. पु., जे~।}पूर्वदेशसंयोगनाशः (६) तत्नाशे परमाण्वन्तरेण संयोगोत्पत्तिः (७), संयुक्ताभ्यां परमाणुभ्यां द्व्यणुकारम्भः (८), आरब्धे द्वयणुके \renewcommand{\thefootnote}{6}\footnote{कारेणरूपादिभ्यः \textendash\ किः~।}कारणगुणेभ्यः कार्यगुणानां रूपादीनामुत्पत्तिः (९) इति यथाक्रमं \renewcommand{\thefootnote}{7}\footnote{दश क्षणाः \textendash\ पा. ७. पु~।}नवक्षणाः~।

सेयं प्रक्रिया यदा क्रियातो विभागो $\rightarrow$ $\hbox{द्र}^{8}\hbox{व्यारम्भसंयोगप्रतिद्वन्द्यप्रतिद्वन्द्वी}$ चेति पक्षस्तदा $\leftarrow$~। यदा विभागात् पूर्वसंयोगनाशस्ततो द्रव्यविनाशविभागजविभागौ चेति पक्षस्तदा $\rightarrow$ ${}^{9}\hbox{दशमे}$ क्षणे~। तथा हि द्व्यणुकनाशविभागजविभागोत्पत्ती (१), पूर्वसंयोग\textendash

\newpage
\noindent
नाशश्यामादिनिवृत्ती (२), उत्तरसंयोगरक्ताद्यृत्पत्ती (३), विभागजविभागक्रिययोर्निवृत्तिः (४), द्रव्यारम्भणाय परमाणौ क्रिया (५), क्रियया विभागः (६), तेन पूर्वसंयोगनिवृत्तिः (७), परमाण्वन्तरसंयोगः (८), द्रव्योत्पत्तिः (९), गुणोत्पत्तिरिति (१०), $\leftarrow$~।

यदा तु द्रव्यविनाशोत्तरकालं विभागजविभागश्यामाविनिवृत्ती, ततः पूर्वसंयोगनिवृत्तिरक्ताद्युत्पत्ती, तत उत्तरसंयोगोत्पत्तिः, ततो विभागजविभागकर्मणोर्निवृत्तिः ततो द्रव्यारम्भणाय परमाणौ क्रियेति पक्षस्त\renewcommand{\thefootnote}{1}\footnote{I दशमे क्षणे \textendash\ जे~।}दैकादशे क्षणे~। तथा हि द्व्यणुकविनाशः (१); $\rightarrow$ \renewcommand{\thefootnote}{2}\footnote{2 3 4 5 7 9 इत्यत्र $\rightarrow$ $\leftarrow$ एतच्चिह्नान्तर्गताः पाठाः 'जे' पुस्तके न सन्ति~।}विभागजविभाग $\leftarrow$ श्यामादितिवृत्ती (२), $\rightarrow$ ${}^{3}\hbox{पूर्वसंयोगनिवृत्ति}$ $\leftarrow$ रक्ताद्युत्पत्ती (३), उत्तरसंयोगः (४), $\rightarrow$ $\hbox{वि}^{4}\hbox{भागजविभागकर्मणोर्निवृत्तिः}$ (५), द्रव्यारम्भणायं परमाणौ क्रिया $\leftarrow$ (६), क्रियया विभागः (७), संयोगविनाशः (८), परमाण्वन्तरसंयोगः (९), द्रव्योत्पत्तिः (१०), गुणोत्पत्तिरिति (११), कथम् ? द्रव्यविनाशे सति तद्विशिष्टं कालमपेक्ष्य विभागेनोत्तरकालं विभागजविभागजननात्~। विभागजविभागे सति $\rightarrow$ $\hbox{पूर्व}^{5}\hbox{संयोगनिवृत्तौ}$ उत्तरसंयोगजननात्~। तस्मिन् सति च पूर्वक्रियानिवृत्तिः, तन्निवृत्तौ च क्रियान्तरारम्भावकाशादिति~। कथं तर्हि पूर्वप्रक्रिया ? \renewcommand{\thefootnote}{6}\footnote{संयोगनिवृत्तिमपेक्ष्यैव \textendash\ पा. ८. पु; जे~।}कारणसंयोगनाशमपेक्ष्यैव~। विभागेनोत्तरविभागारम्भणाविति पक्षात् किमत्र तत्त्वमिति विभागे विवेचयिष्यामः~॥

सेयमेकस्मिन्नेव परमाणौ पूर्वद्व्यणुकविनाशोत्तरद्वयणुकोत्पादकक्रियाद्वयमधिकृत्य क्षणगणना~। यदा त्वेकत्र द्वयणुकविनाशिकाऽन्यत्र द्वयणुकान्तरोत्पादिका क्रिया तदा द्व्यणुकविनाशमारभ्य पश्चमे षष्ठे सप्तमेऽष्टमे $\rightarrow$ $\hbox{न}^{7}\hbox{वमे}$ वा क्षणे $\leftarrow$ गुणोत्पत्तिः~। कथम्? यदा द्रव्यारम्भकः संयोगो विनश्यति तदैव परमाण्वन्तरे कर्म (१), तत एकत्र क्रमेण द्रव्यविनाशश्यामादिविनाशरक्ताद्युत्मादाः (२), अन्यत्र विभागपूर्वसंयोगनाशोत्तरसंयोगाः (३), क्षणान्तरे द्रव्योत्पत्तिः (४), पञ्चमे क्षणे गुणोत्पत्ति\renewcommand{\thefootnote}{8}\footnote{${}^{\circ}\hbox{रपि \textendash\ जे~।}$}श्चेति~। एवं द्रव्यनाशसमकालं परमाण्वन्तरे क्रियाचिन्तनात् षष्ठे, श्यामादिनाशसमकालं क्रियाचिन्तायां सप्तमे, रक्ताद्युत्पत्तिसमकालं कर्मप्रक्रमेऽष्टमे क्षणे $\rightarrow$ $\hbox{रक्ता}^{9}\hbox{द्युत्पत्तेग्रिमक्षणे}$ परमाण्वन्तरे करर्मचिन्तायां नवमे क्षण $\leftarrow$ इति~। द्वित्रिचतुःक्षणा च प्रक्रिया न \renewcommand{\thefootnote}{10}\footnote{नास्त्येव \textendash\ पा. ७. पु~। पु~।}सम्भवत्येव~। उत्पन्ने रक्तादौ सगुणेनैव द्रव्येण द्रव्यान्तरारम्भणादिति संक्षेप~।

एवमेव पक्वापक्वाभ्यामारम्भश्चिन्तनीयः~। तदेवमववयविनोयावद्द्रव्यभाविना एव

\newpage
\noindent
विशेषगुणाः पाकजाः परमाणुष्वेवेति व्यवस्थिते सुखादयो न शरीरगुणाः अयावद्रव्यभावित्वात्, न परमाणुगुणाः प्रत्यक्षत्वादित्माधिकारोक्तं समाहितं वेदितव्यमिति~॥

{\knu (१२९) एकादिव्यवहारहेतुः सङ्ख्या~।}

अथोद्देशक्रमप्राप्तां सङ्ख्या निरूपयति\textendash\ {\knu एकादीति~।} व्यवह्रियतेऽनेनेति हानोपानादि क्रियते इति व्यवहारो विज्ञानम्~। एकं द्वे त्रिणीत्याकारम्, तस्य विषयतया हेतुः सङ्ख्येति प्रमाणमुक्तम्~। न ह्येकादि विज्ञानं निर्विषयं व्याघातात्~। नाप्यलीकविषयमबाधनात्~। नापि घटस्वरूपमात्रनिबन्धनं घटमन्तरेण घटान्तरेऽपि भावात्~। नापि घटत्वनिबन्धनं पटेऽपि भावात्~। न चारजतेऽपि रजतप्रत्ययो भवन्ननिमित्तान्तरमपेक्षते इतिवत् घटे भूत्वा पटादावपि भविष्यति~। न च निमित्तान्तरमपेक्षिष्यत इति साम्प्रतम् अभ्रान्तत्वात्~। सोऽयं घटादावेकप्रत्ययो घटपटत्वारिक्तविषयः तत्प्रत्ययनिमित्तमन्तरेण \renewcommand{\thefootnote}{1}\footnote{१ $\rightarrow$ $\leftarrow$ एतच्चिसान्तर्गतः पाठः 'जे' पुस्तके नास्ति~।}पटादावपि जायमानत्वात्, वस्त्रचर्मकम्बलेषु नीलप्रत्ययवत्~। न हि तद्व्यतिरेकेण भवति तद्विषयश्चेति व्याश्रातात्~। न हि तत्प्रत्ययनिमित्तमन्तरेण भवत्यभ्रान्तश्चेति सम्भवति, आकस्मिकत्वप्रसङ्गात्~।

एतेन स्वरूपाभेद एकत्वं स्वरूपभेदस्तु नानात्वं द्वित्वमिति \renewcommand{\thefootnote}{2}\footnote{२ भूषणो न्यायसूत्रवृत्तिकार इति प्रतिषादितमधस्तात्~।}भूषणः प्रत्याख्यातः~। स्वरूपाभेदो हि घटस्य घट एवोच्यते~। घटादिप्रत्ययस्य स्वरूपप्रत्ययस्य च घटाद्येकनिबन्धनत्वात् एकादिप्रत्ययस्य च घटपटादिसाधारणत्वात्, तथा च तन्निबन्धन \renewcommand{\thefootnote}{3}\footnote{३ एकादि व्यवहारः \textendash\ पा. ८. पु.~।}एकादिप्रत्ययः तं विहाय पटं नोपसङ्क्रामेत्~। एवं स्वरूपभेद एव यदि द्वित्वं तदा त्र्यादिष्वपि द्वित्वप्रत्ययः स्यात्; घटादावेव वा स्वरूपभेद इति पटादौ द्वित्वप्रत्ययो न स्यादिति~।

अथवा व्यवह्रियतेऽनेनेति व्यवहार इत्यभिधानमुच्यते, तेन एकादिशब्दप्रवृत्तिनिमित्ते सङ्ख्येत्युक्तं भवति~। तथा च सङ्ख्याग्राहकप्रत्यक्षानुग्राहकस्तर्को दर्शितः~। तथाहि यदि सङ्ख्या पदार्थो घटादिस्वरूपादधिको न स्यात् तदा घटमानयेत्युक्ते सत्तादाविवासन्दिग्धः किमेकमनेकं वेति न सन्दिह्येत~। एकमानम्रेत्युक्ते घटं पटं वा इति परावृत्य न पृच्छेत्~। समुच्चयासमुच्चयावेकत्वाऽनेकत्वे इति चेत्, न; एकत्वस्यैवासमुच्चयरूपत्वात्, द्वित्वादिसङ्ख्याया एव \renewcommand{\thefootnote}{4}\footnote{४ समुच्चयत्वात् \textendash\ कि; क~।}समुच्चीयमानत्वात्~। भिन्नानामभिन्नबुद्धयुपग्रहमात्रस्य विषयतः स्वरूपतश्च विशेषाभावे त्र्यादिव्यवहारविलोपप्रसङ्गात्~।

स्यादेतत्; त्वयाऽपि सङ्ख्याविशेषोत्पत्तयैऽपेक्षाबुद्धेर्विशेषोऽवश्यमभ्युपगन्तव्यः,

\newpage
\noindent
अन्यथा तदविशेषे द्वित्वमेवं त्रित्वमेवोत्पद्यत इति नियमो न स्यात्~। न स्योच्च रूपादिषु चतुर्विंशतिरेव गुणा इति, कर्मसु पञ्चैवेत्यादिनियम इति चेत्, सत्यम्; क्वचित् तात्त्विकीं क्वचिद् भाक्तीं सङ्ख्यामेवोपादायेति वक्ष्यामः~। सन्तु तर्हि द्वित्वादयः सामान्यविशेषाः घटत्वादिवत्, कृतं गुणान्तरकल्पनया ? गुणपक्षेऽपि सामान्यद्वारैव प्रत्ययानुगमः समर्थनीयः तद्वरं लाघवात् तदेवास्तु~। न; अविरोधात् परापरभावाभावाच्च~। अन्यूनानतिरिक्तव्यक्तिवृत्तेश्च सामान्यद्वयस्थानभ्युपगमात्, भेदे प्रमाणाभावात् विरुद्धधर्माध्यासो हि तदिति~। प्रत्ययवैलक्षण्यं प्रमाणमिति चेत्, नः धर्मविरोधमन्तरेण तस्यैवानुपपत्तेः~। अन्यथा बुद्धिव्यपदेशपौनरुक्तयोच्छेदप्रसङ्गादिति~।

अथवा एकादीत्यनेन गणितमुपलक्ष्यते~। गुणत्वे सतीति चाधिकारोल्लभ्यते~। तेन गणितव्यवहारहेतुर्गुणः सङ्ख्येति लक्षणमुक्तं भवति~। लक्षणार्थश्च पूर्ववदिति~। अन्यथैकव्यवहारहेतुरिति लक्षणे द्वित्वाद्यव्याप्तिः, द्वित्वव्यवहारे हेतुरित्युक्ते च त्रित्वाद्यव्याप्तिः~। एकादिसमस्तव्यवहारहेतुरिति विवक्षितेऽसिद्धिः स्यादिति~।

\hangindent=2cm {\knu (१३०) सा पुनरेकद्रव्या चानेकद्रव्या च~। तत्रैकद्रव्यायाः सलिलादिपरमाणुरूपादीनामिव नित्यानित्यत्वनिष्पत्तयः~। अनेकद्रव्या तु द्वित्वादिपरार्धान्ता~। तस्याः खलु एकत्वेभ्योऽनेकविषयबुद्धिसहितेभ्यो निष्पत्तिः~। अपेक्षाबुद्धिविनाशाद्विनाश इति~॥}

एवं प्रमाणतर्कलक्षणैर्व्यवस्थापितायाः सङ्ख्याया गुणान्तरेभ्यो विशेषं निरूपयति \textendash\ {\knu सा पुनरिति~।} रूपादयो हि गुणा एकद्रव्या एव~। संयोगादयस्त्वनेकद्रव्या एव~। सङ्ख्या पुनरेकद्रव्या चानेकद्रव्या च~। एकं द्रव्यमाश्रयतया सम्बन्धि यस्याः सा एकद्रव्या~। अनेकं भिन्नं द्रव्यमाश्रयतया सम्बन्धि यस्याः साऽनेकद्रव्या~।

तत्रैकद्रव्याया नित्यत्वाद्यतिदेशेन दर्शयति \textendash\ तत्रेति~। तत्र तयोर्मध्ये~। सलिलमादिर्यस्य स सलिलादिः, स चासौ परमाणुश्चेति सलिलादिपरमाणुः; तस्य रूपं तदेवादिर्येषां द्व्यणुकादिरूपाणां तानि~। तथा तेषामिव नित्यत्वमनित्यत्वं निष्पत्तिश्चैकद्रव्यायाः~। एतदुक्तं भवति, यथा सलिलपरमाणौ रूपं नित्यं तथैकत्वमपि~। यथा द्व्यणुकादौ रूपं कारणगुणप्रक्रमेणोत्पद्यते \renewcommand{\thefootnote}{1}\footnote{I तथा तस्यैकत्व \textendash\ क~।}तथैकत्वमपि~। यथा द्व्यणुकादावाश्रयस्य विनाशात् तद्रूपं विनश्यति तथैकत्वमपीति~।

\newpage
अनेकद्रव्यामाह \textendash\ {\knu अनेकेति~।} द्वित्वमादिर्यस्याः सा द्वित्वादिका~। परार्द्धमन्तो यस्याः सा पराद्धान्ता~। पराद्धात् परतः सङ्ख्याव्यवहारो नास्तीत्यागमदर्शनादवगन्तव्यम्~।

\begin{sloppypar}
तस्याः सङ्क्षेपेणोत्पत्तिविनाशप्रकारमाह \textendash\ तस्या खल्विति~। 'खलु'शब्दोऽवधारणे, निष्पत्तिरेव न त्वभित्यक्तिमात्रम्~। एकत्वे चैकत्वानि चैकत्वानि, तेभ्यः~। अनेनासमवायिकारणमुक्तम्~। अनेकद्रव्येत्यनेन प्रागेव समवायिकारणस्य दर्शितत्वात्~। न एकमनेकम्, तदेव विषयो यस्याः बुद्धेः साऽनेकविषया बुद्धिः~। तत्सहितेभ्य इति निमित्तकारणमुक्तम्~। कथमेतत् ? अथैकत्ववद् द्वित्वादिकमपि कारणगुणेभ्य एव जातमपेक्षाबुद्ध्या \renewcommand{\thefootnote}{1}\footnote{I 'एकपृथक्त्वात्' इति मु. कि. क पुस्तकयोर्नास्ति~।}एकपृथक्त्ववद् व्यज्यत इति किं न स्यात् ? न स्यात्, द्वयणुकादौ हि कारणगुणप्रक्रमेण द्वित्वादिकमुत्पद्यमानमेकत्वमिवैकद्रव्यं प्रसज्येत~। त्रित्वादिकं तु नोत्पद्येत कारणभावात्~। द्रव्यान्तरापेक्षया परार्धपर्यन्तं कारणस्यैकत्वादीनि सन्तीति कार्ये परार्द्धपर्यन्तमुत्पद्यमानमविरुद्धमिति चेत्; न; द्रव्यान्तराणामकारणत्वात्~। तथा च कारणगुणपूर्वकत्वानुपपत्तिः~। कारणाकारणसङ्ख्याभ्यः कार्याकार्यसङ्ख्याः स्युः संयोगवदिति चेत्, न, तद्वदेव तर्हि तासां न कारणमाश्रयः स्यात्~। तथा च कारणपरमाणुभ्यां द्व्यणुकस्य त्रित्वादिञ्यवहारो नोपपद्येत~। किञ्च तैरेवैकत्वैरेकत्वं तैरेव द्वित्वादिकमिति महद्वैषम्यम्, कारणाविशेषे कार्यवैचित्र्यानुपपत्तेः~। कार्थमपेक्ष्यैकत्वं स्वाश्रयमपेक्ष्य च द्वित्वादिकमिति चेत्, न; कार्यानित्यमेकमनेकव्यक्तिवृत्ति सामान्यमिति सामान्यलक्षणापत्तौ गुणत्वव्याघातात्~। नानित्यं कार्यत्वे सति सततोत्पत्तिप्रसङ्गोऽपेक्षणीयान्तराभावादविनाशप्रसङ्गश्च~। \renewcommand{\thefootnote}{2}\footnote{आश्रयनाशविरोधि \textendash\ पा. ८. पु~।}कारणविनाशविरोधि गुणोत्पादयोरसम्भवात् द्वित्वद्विपृथक्त्वादेश्च समानदेशत्वे संमानेन्द्रियग्राह्यत्वे च सति व्यञ्जकनियमानुपपत्तिरिति~। तस्मादपेक्षाबुद्धिरेव द्वित्वादेरुप्पदिका नियामिका च, तद्विनाशश्च तद्विनाशक इति सङ्क्षेपः~॥
\end{sloppypar}

\begin{sloppypar}
\hangindent=2cm {\knu (१३१) कथम् ? यदाबोद्धुश्चक्षुषा समानासमानजातीययोर्द्रव्ययोः सन्निकर्षे सति तत्संयुक्तसमवेतसमवेतैकत्व\renewcommand{\thefootnote}{3}\footnote{सामान्ये \textendash\ जे~।} सामान्यज्ञानोत्पत्तावे\renewcommand{\thefootnote}{4}\footnote{'एकत्कसामान्यज्ञानोत्पत्तौ' इत्यधिकं 'दे' पुस्तके~।}कत्वसामान्यतत्सम्बन्ध\renewcommand{\thefootnote}{5}\footnote{तत्र ज्ञानेभ्यः \textendash\ कं. कि. व्यो (४५७)~।}ज्ञानेभ्य ए\renewcommand{\thefootnote}{6}\footnote{एकत्वगुणयोः \textendash\ कि~।}कगुणयोरनेकविषयिण्येकाबुद्धिरुत्पद्यते तदा तामपेक्ष्य एकत्वाभ्यां स्वाश्रययोर्द्वित्वमारभ्यते~। ततः पुनस्तस्मिन् द्वित्वज्ञानमुत्पद्यते~। तस्माद्द्वित्वसामान्य\textendash}
\end{sloppypar}

\newpage
\hspace{1.3cm} \hangindent=2cm {\knu ज्ञाना\renewcommand{\thefootnote}{1}\footnote{I दपेक्षाबुद्धिविनाशाद् द्वित्वगुणस्य विनश्यत्ता, द्वित्वगुणबुद्धितः समन्यबुद्धेरपि विनश्यत्ता, द्वित्वगुणज्ञानत्सम्बन्धेभ्यो द्वे द्रव्ये इति द्रव्यबुद्धेरुत्पद्यमानतेत्येकः कालः \textendash\ दे~।}दपेक्षाबुद्धेर्विनश्यत्ता, द्वित्वसामान्यत\renewcommand{\thefootnote}{2}\footnote{तत्सम्बन्धतज्ज्ञानेभ्यो \textendash\ मु. भाः, तज्ञाजनतंत्सम्बन्धेभ्यो \textendash\ व्यो (४५७)~।}त्सम्बन्धज्ञानेभ्यो द्वित्वगुणबुद्धेरुत्पद्यमानता, इत्येकः कालः~॥}

\renewcommand{\thefootnote}{3}\footnote{तदेवम् \textendash\ कि.~।}तदेतत् द्वित्वबुद्धिमात्रे न्यायप्राप्तां शिष्यजिज्ञासामनूद्य विशदयति \textendash\ {\knu कथमिति~।} यदा यस्मिन् \renewcommand{\thefootnote}{4}\footnote{क्षणे \textendash\ पा. ८. पु.~।}काले बोद्धुश्चेतनस्य बुद्धिरुत्पद्यत इत्यग्रे भविष्यति~। चक्षुषेत्युपलक्षणं त्वचा वा लिङ्गेन वेत्यपि द्रष्टव्यम्~। {\knu समानासमानजातीययो}रित्यनियमं दर्शयति~। {\knu सन्निकर्षेसतीति} \textendash\ सम्बन्धे सतीत्यर्थः~। तत्संयुक्तमिन्द्रियसंयुक्तं यद्द्रव्यं तत्समवेतौ यावेकत्वगुणौः तयोः समवेतं यदेकत्वसामान्यं तत्र ज्ञानोत्पत्तौ सत्यामेकत्वसामान्यात् तत्सम्बन्धाच्च समवायलक्षणात्, तज्ज्ञानाच्च पूर्वोत्पन्नादेकत्वगुणयोर्विषयभूतयोः~। अत एवानेकविषयिण्येका बुद्धिरुत्पद्यते उत्पन्ना भवति~। यदोत्पन्ना भवति तदा तामपेक्ष्यैकत्वाभ्यां स्वाश्रययोर्द्वित्वमारभ्यते, द्वित्वारम्भणाय व्याप्रियते, द्वित्वारम्भानुगुणापेक्षाबुद्धिः सहकारिणी लभ्यते इति~। यदा तदेत्येककालतानिर्देशोपपत्तिः~। अपेक्षाबुद्ध्युत्पादऽद्वित्वोत्पद्य मानतयोरेककालत्वादिति~।

अत्र च द्रव्यज्ञानात्पूर्वं गुणज्ञानम्, गुणज्ञानाच्च पूर्वं सामान्यज्ञानमिति विशेषणत्वादवसेयम्; व्यवच्छेदकस्य तथाभावात् स्वज्ञानेन च व्यवच्छेदकत्वात्~। अनुनसंहितविशेषणस्य विशिष्टप्रत्ययादुपपत्तेः~। तथा च कार्यमेव कारणे प्रमाणमिति द्वित्वस्यापेक्षाबुद्धिसत्त्वासत्त्वाधीनं सत्त्वासत्त्वमिति द्वे द्रव्ये इति द्रव्यज्ञानस्य सङ्कटप्रविष्टतया तदन्तर्भाव्य विनाशं व्युत्पादयति~। {\knu ततः पुनरिति} \textendash\ तत उत्पन्ने द्वित्वे सति तस्मिन्नेव वर्त्तमानं यत् द्वित्वसामान्यं तत्र ज्ञानमुत्पद्यते~। पुनरिति पूर्वोक्तसम्बन्धानुकर्षणार्थम्~। तस्मादित्यानन्तर्यार्थम्~। द्वित्वसामान्यज्ञानमेवापेक्षाबुद्धेर्विनश्यत्ता विनाशकारणसान्निध्यम्~। यद्यपि सामान्यज्ञानमेवापेक्षाबुद्धेर्विनाशकारणं तदुत्पत्तिरेव च सान्निध्यं तथापि तस्य हेतुत्वमाविष्कर्तुं पञ्चमीनिर्देशः~। द्वित्वसामान्यतज्ज्ञानात् तत्सम्बन्धाच्च द्वित्वगुणबुद्धेरूत्पद्यमानता \textendash\ उत्पत्तिकारणसान्निध्यम्, पूर्ववत् पञ्चम्युपपत्तिः, यदेवापेक्षाबुद्धेर्विनाशकं तदेव गुणबुद्धेरुत्पादकमिति विनश्यत्तोत्पद्यमानतयोरेकः काल इत्युपपद्यते~॥

\hangindent=2cm {\knu (१३२) तत इदानीमपेक्षाबुद्धिविनाशाद्द्वित्वगुणस्य विनश्यत्ता, द्वित्वगुणबुद्धितः सा\renewcommand{\thefootnote}{5}\footnote{सामान्यबुद्धेर्विनिश्यत्ता \textendash\ कि~। }मान्यबुद्धेरपि विनश्यत्ता, द्वित्व \textendash\ }

\newpage
\indent
\hspace{1.3cm} \hangindent=2cm {\knu गुण\renewcommand{\thefootnote}{1}\footnote{I तज्ज्ञानतत्सम्बन्धेभ्यो \textendash\ व्यो (४५७); तत्सम्बन्धतज्ज्ञनेभ्यो \textendash\ कं; कि~।}ज्ञानतत्सम्बन्धेभ्यो द्वे द्रव्ये इति द्रव्यबुद्धेरुत्पद्यमानत्येतकः कालः~॥}

तत इदानीं तदनन्तरं तस्मिन् कालेऽपेक्षाबुद्धेर्विनाशाद् द्वित्वगुणस्य विनश्यत्ता, द्वित्वगुणबुद्धितः सामान्यबुद्धेरपि विनश्यन्ता, अपेक्षाबुद्धिविनाशगुणबुद्धयोर्द्वित्वगुणद्वित्वगुणद्वित्वसामान्यबुद्धिविनाशकतया तयोर्यौगपद्ये तयोर्विनश्यत्तायौगपद्यमित्यर्थः~। द्वित्वगुणतज्ज्ञानतत्सम्बन्धेभ्यो 'द्वे द्रव्ये' इति द्रव्यज्ञानस्योत्पद्यमानता~। गुणज्ञानमेव हि चरमं द्रव्यज्ञानस्योत्पादकं विशेषणतत्सम्बन्धयोः पूर्वसिद्धत्वात्, अपेक्षणीयान्तराभावाच्च तच्च तदैवेति भवत्येकः कालः~॥

\hangindent=2cm {\knu (१३३) त\renewcommand{\thefootnote}{2}\footnote{भाष्यमिदं जे दे पुस्तकयोर्नास्ति~।}दनन्तरं 'द्वे द्रव्ये' इति द्रव्यज्ञानस्योत्पादः, द्वित्वस्य विनाशाः, द्वित्वगुणबुद्धेर्विनश्यत्ता, द्रव्यज्ञानात् संस्कारस्योत्पद्यमानतेत्येकः कालः~॥}

\begin{sloppypar}
तदनन्तरं द्वे द्रव्ये इति ज्ञानस्योत्पादः; पूर्वनिर्दिष्टेभ्यो द्वित्वगुणतज्ज्ञानतत्सम्बन्धेभ्यो द्वित्वस्थ विनाशः; पूर्वनिर्दिष्टादपेक्षाबुद्धेर्विनाशात् $\rightarrow$ \renewcommand{\thefootnote}{3}\footnote{$\rightarrow$ $\leftarrow$ एतच्चिह्नान्तर्गतः पाठः जे पुस्तके नास्ति~।}पूर्वनिर्दिष्टद्वित्वबुद्धितः सामान्यबुद्धिविनाशः, $\leftarrow$ द्रव्यबुद्धयुत्पत्तिरेव संस्कारस्योत्पादिका गुणबुद्धेर्विनाशिकेति तयोरूत्पद्यमानताविनश्यत्तेत्यैकः कालः~॥
\end{sloppypar}

\begin{sloppypar}
\hangindent=2cm {\knu (१३४) त\renewcommand{\thefootnote}{4}\footnote{तदनन्तरं द्वित्वगुणबुद्धिविनाशो \textendash\ कि~।}दनन्तरं द्रव्यज्ञानाद् द्वित्वगुणबुद्धेर्विनाशः द्रव्यबुद्धेरपि \renewcommand{\thefootnote}{5}\footnote{संस्कारात् \textendash\ कं~।}संस्कारादिति~॥}
\end{sloppypar}

तदनन्तरं द्रव्यज्ञानाद्गुणबुद्धेर्विनाशः संस्कारस्योत्पाद इत्यपि द्रष्टव्यम्~। क्षणान्तर संस्कारबुद्धेरपि विनाश इति विशेषः~। 

\hangindent=2cm {\knu (१३५) \renewcommand{\thefootnote}{6}\footnote{अनेन \textendash\ पा. ६. पु~।}एतेन त्रित्वाद्युत्पत्तिरपि व्याख्याता~। एकत्वेभ्योऽनेकविषयबुद्धिसहितेभ्यो नि\renewcommand{\thefootnote}{7}\footnote{द्वित्वादिनिष्पत्तिः \textendash\ कि~।}ष्पत्तिरपेक्षाबुद्धिवि\renewcommand{\thefootnote}{8}\footnote{विनाशात् \textendash\ कि; जे~।}नाशाच्च विनाश इति~॥}

{\knu एतेन त्रित्वाद्युत्पत्तिर्व्याख्याता} \textendash\ एतेन प्रकारेण \textendash\ उत्पत्तिरित्युपलक्षणम्, विनाशोऽपि व्याख्यातः~। तमेव प्रकारमाह \textendash\ {\knu एकत्वेभ्य} इति~। तदयं प्रक्रियासङ्क्षेपः;

\end{document}