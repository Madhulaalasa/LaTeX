\documentclass[11pt, openany]{book}
\usepackage[text={4.65in,7.45in}, centering, includefoot]{geometry}
\usepackage[table, x11names]{xcolor}
\usepackage{fontspec,realscripts}
\usepackage{polyglossia}
\setdefaultlanguage{sanskrit}
\setotherlanguage{english}
\setmainfont[Scale=0.65]{Shobhika}
\newfontfamily\s [Script=Devanagari, Scale=1]{Shobhika}
\newfontfamily\regular{Linux Libertine O}
\newfontfamily\en [Language=English, Script=Latin]{Linux Libertine O}
\newfontfamily\mbh [Script=Devanagari, Scale=0.65, Color=purple]{Shobhika-Bold}
\newfontfamily\qt [Script=Devanagari, Scale=0.65, Color=violet]{Shobhika-Regular}
\newcommand{\devanagarinumeral}[1]{
\devanagaridigits{\number \csname c@#1\endcsname}} % for devanagari page numbers
\XeTeXgenerateactualtext=1 % for searchable pdf
\usepackage{enumerate}
\pagestyle{plain}
\usepackage{fancyhdr}
\pagestyle{fancy}
\renewcommand{\headrulewidth}{0.5pt}
\usepackage{afterpage}
\usepackage{multirow}
\usepackage{multicol}
\setlength{\columnseprule}{0.5pt}
\usepackage{wrapfig}
\usepackage{vwcol}
\usepackage{microtype}
\usepackage{amsmath,amsthm, amsfonts,amssymb}
\usepackage{mathtools}% <-- new package for rcases
\usepackage{graphicx}
\usepackage{longtable}
\usepackage{setspace}
\usepackage{footnote}
\usepackage{perpage}
\MakePerPage{footnote}
\usepackage{xspace}
\usepackage{array}
\usepackage{emptypage}
\usepackage{hyperref}% Package for hyperlinks
\hypersetup{colorlinks, 
citecolor=black, 
filecolor=black, 
linkcolor=blue, 
urlcolor=black}

\renewcommand{\baselinestretch}{0.2}

\begin{document}
\thispagestyle{empty}

\noindent 
\hspace{4cm} महाभाष्यप्रदीपोद्दयोतव्याख्या छाया~। \hspace{3cm} १५\\

\noindent
\rule{1\linewidth}{0.5pt}\\

\begin{multicols}{2}
(उद्व्योतः ) भाष्योक्तहेतुमुपपादयति \textendash\ भिन्नेन्द्रियेति~। सा \textendash\ मान्यव्यतिरिक्तत्वे सति श्रोत्रेन्द्रियग्राह्यत्वं हि शब्दलक्षणमिति भावः~॥ यदि च द्रव्येति~। यदि द्रव्यं शब्दशब्दाभिधेयं स्यादित्यर्थः~। तथाहि सति द्रव्यादीनां तत्रान्तरे निरूपितत्वात्तस्य१ वैयर्थ्यापत्तिरित्यपि बोध्यम्~। तत्र२ द्रव्यस्य क्रियाद्याश्रयत्वात्पूर्वमुक्तिः~। क्रियागुणौ च जात्याश्रयाविति तयोस्ततः पूर्वमुक्तिः~। वैशेषिकसिद्धगुणानामपि सङ्ख्यासंयोगविभागादीनां क्रियात्वं ध्वनयितुं गुणात्प्राक् क्रियोक्तिरिति ध्येयम्~। यत्तर्हि शुक्ल इत्यादिगुणस्य शब्दत्वशङ्कापरे वाक्ये {\qt इति}शब्देन प्रकारार्थेन सुरभ्यादिभेदभिन्नगन्धादिलिङ्गसंख्यादिरूपगुणसङ्ग्रह इति न न्यूनता~॥ नजु शब्दस्यापि गुणत्वात् {\qt गुणो नाम} सः इत्युत्तरमसङ्गतम्~। तस्य द्रव्यत्वे {\qt द्रव्यं नाम सत्} इत्यु \textendash\ त्तरासंगतिरिति चेत्, न~। {\qt उक्तद्रव्याश्रितो गुणो नाम सः} इत्यर्थात्~॥

\noindent
\rule{1\linewidth}{0.5pt}\\

ननु शब्दशब्दस्य कर्मघञन्तत्वेनार्थपरत्वमेवास्तु, अत आह [ १४ पृ. प्र.२ यप० ] तत्र \textendash\ यदि चेति~॥\\

[उ० १मप०] तदभिप्रेत्याह \textendash\ भाष्योक्तेति~॥\\

[उ० १मप० ] हेत्वसिद्धि परिहरन्हेतौ हेतुमाह \textendash\ [ १ मप० ] सामान्येति~॥ अनेन शब्दनिष्ठसामान्यैऽतिप्रसङ्गो वारितः~॥ विशेष्यदलेन द्रव्यादिनिरासः~॥ हि \textendash\ यतः~॥ शब्देति~। शब्दसा \textendash\ मान्येत्यर्थः~॥ एतेन \textendash\ अर्थबोधायानुच्चार्यमाणत्वान्न तत्त्वम् \textendash\ इति रत्नोक्तमपास्तम्~।  अर्थबोधायानुच्चार्यमाणेsनुकरणे चाव्याप्तेः~॥\\

[उ० ] अत एव यथाश्रुतमनुपपन्नं व्याचष्टे \textendash\ [ ३ यप० ] यदि द्रव्यमिति~। स्यादित्यनेन {\qt अभविष्यत} इत्यत्र कालाविवक्षा सूचिता~। एवं च {\qt अवक्ष्यत्} इत्यस्यापि {\qt वदेत्} इत्यर्थो बोध्यः~॥ हेतुहेतुमद्भावरूपलिङ्गनिमित्तत्वात् {\qt भूते च} इति लृङ् इति केचित्~॥ इत्यर्थ इति~। इति मूलभूतार्थ इत्यर्थः~। ननु विनिगमनाविरहात्तथोक्तिः~। न च स्फुटप्रतीतिभवनं तथा, कर्मघञ६न्तानुशासनसत्त्वेन तस्यावश्यकत्वेऽपि तत्वात्~। न च लोकप्रसिद्धिस्तथा, लोकप्रसिद्धितः शास्त्रप्रसिद्धेः प्राबल्येन तस्या अविशिष्टत्वात, अत आह \textendash\ तथाहीति~॥ अन्यार्थत्वस्यावश्यकत्वेन वैपरीत्यं नेति सूचनायाह \textendash\ [४ र्थप० ] दीनामिति~॥ यद्भिन्नेष्वित्यत्र वक्तव्यमतमाह \textendash\ तत्रेति~॥ तत्र भासमानानां मध्ये इत्यर्थः~॥ नन्वेवमपि शुक्लादिगुणस्य चिरतरसमयस्थायितया सर्वसंमतप्रत्यक्ष \textendash\ तया च पूर्वमुक्तिर्युक्ता न क्रियायाः, तद्विलक्षणत्वादत आह \textendash\ [ ६ष्ठ \textendash\ प० ] वैशैषिकेति~। तन्मतेत्यर्थः~॥ प्रसङ्गादत्रैवाह \textendash\ [ ८ मप० ] यत्तर्हीति~॥ [९ मप० ] प्रकारेति~।  सादृश्येत्यर्थः~॥

\noindent
\rule{1\linewidth}{0.5pt}\\

१ तस्य \textendash\ द्रव्यानुशासनस्य~।\\ 

२ भाष्ये द्रव्यक्रियागुणजातीनां निरसनीयानां मध्ये द्रव्यस्य प्रागुक्तौ बीजमाह \textendash\ तत्रेति~।\\

३ {\qt निमिषितमिति} इति मुद्रिते इतिशब्दसहितः पाठो दृश्यते; स च टीकाकारैरितिशब्दस्याव्याख्यानादनुपयोगाल्लिखितेष्वनुप \textendash\ लम्भाच्चात्र न गृहीतः~।

\columnbreak

\begin{center}
(प्रश्नभाष्यम्)
\end{center}

{\qt यत्तर्हि तदिङ्गितं चेष्टितं निमिषितम्३, स शब्दः ?}\\

( प्रदीपः ) अनेनैव न्यायेन गुणक्रियासामान्यानां निराकृ \textendash\ तेऽपि शब्दत्वे प्रपञ्चार्थं तत् चोद्यपूर्वकं निराकरोति \textendash\ यत्तर्हीति~।  गोशब्दार्थे चैषां संभवात् शब्दत्वमाशङ्क्यते~। परिहारस्तु पूर्ववत्~॥ तत्रेङ्गितम्$=$अभिप्रायस्यसूचकः शरीरव्यापारः~। चेष्टितं$=$काय \textendash\ परिस्पन्दः~। निमिषितम्$=$अक्षिव्यापारः~॥\\

( उद्द्योतः ) श्रोत्राग्राह्यत्वादेव क्रियादीनां शब्दत्वे निरस्ते किमुत्तरग्रन्थेने४त्यत आह \textendash\ अनेनैवेति~॥ गोशब्दार्थे चैषां संभ \textendash\ वादिति~।  अभेदसंभवात् , तेन {\qt तदभिन्नाभिन्नस्य तदभिन्नत्वं} इति न्यायेन शब्दत्वाशङ्का तेषामिति भावः~॥ यद्वा गोशब्दार्थे गुण समूहितया एषां संभवादित्यर्थः~॥ शरीरव्यापारः$=$रोमाञ्चादिः~॥\\

\noindent
\rule{1\linewidth}{0.5pt}\\

\noindent
अन्यथा तदानर्थक्यं स्पष्टमेवेति भावः~॥ {\qt यत्तत्} इति नपुंसकनिर्देशेन चेत्यपि बोध्यम्~॥ आदिना स्पर्शादिपरिग्रहः~॥ अग्र आदिना शक्तिरूपकारकपरिग्रहः~। तत एवात्रैवाह \textendash\ [ १० मप० ] नन्विति~॥ [ १२ शप० ] इति तदर्थादिति~। इत्युत्तरार्थादित्यर्थः~॥ {\qt इत्यर्थात्} इति पाठेsप्येवम्~॥ अनेन {\qt द्रव्यं नाम} तत् इत्यस्य {\qt वक्ष्यमाणगुणाश्रयो द्रव्यं नाम तत्?} इत्यर्थः सूचितः~। लिङ्गसंख्याद्यन्वयि द्रव्यमित्यादि तु स्वशास्त्रव्यवहारोपयोगीति बोध्यम्~॥

[ २ यप्र० १ मप० ] तत्रान्तरस्थक्रमेणाह \textendash\ कैयटे \textendash\ गुण \textendash\ क्रियेति~॥ निराकृतेऽपि \textendash\ तत्प्रायेऽपि~॥ प्रपञ्चार्थम् \textendash\ प्रसिद्ध्यन्त \textendash\ रावगमाय~॥ अत एव सर्वत्र नामपदोक्तिः~॥ तत् \textendash\ तेषां शब्दत्वम्~॥ यत्तर्हीतीति~। यदि द्रव्यं न शब्दः, तहींङ्गितादेस्तदन्य \textendash\  त्वात्तत्तमस्तीत्यर्थः~। तर्हि यत्तत् \textendash\ इङ्गितादि स किं शब्द इति योजना बोध्या~॥ एवमग्रेऽपि सर्वत्र~॥\\

[ २ यप्र० ] कैयटे \textendash\ [ ४ धैप० ] तत्रेङ्गीति~। त्रयाणां मध्य इत्यर्थः~॥\\

[ २ यउ० १ मप० ] निरस्ते \textendash\ निरस्तप्राये~॥\\

[द० भागे उ० ३ यप० ] कैयटेन शङ्काबीजे व्याख्यायामुक्तमेव. प्रदर्श्यत इति ध्वनयन्नाह \textendash\ अभेदेति~॥ ५नन्वेवमपि शब्देनाह यद्वेति~॥ एषां \textendash\ क्रियादीनाम्~॥ संभवादिति~। तथा च गोशब्दे तत्समूह्यभेदोऽपि सिद्ध एव तदन्यावयविनोऽत्रानुपलम्भादिति बोध्यम्~॥\\

[उ० ६ ष्ठप० ] कायपरिस्पन्दस्य वक्ष्यमाणत्वादाह \textendash\ रोमाञ्चादिरिति~॥ अनेन येनेङ्गयते भावः प्राणिनंं तत \textendash\ इङ्गितम् इति व्युत्पत्तिः सूचिताः~॥

\noindent
\rule{1\linewidth}{0.5pt}\\

४ उत्तरग्रन्थेन तदिङ्गितमित्यारभ्य आकृतिर्नाम सः इत्यन्तग्रन्थेन~।\\

५ द्रव्ये शब्दत्वं नेत्यर्थः~। (र. ना. )\\

६ कर्मघञन्तस्य शब्दशब्दस्यानुशासनसत्त्वेनेत्यर्थः~। (र. ना.)\\

७ तदभिन्नाभिन्नस्येति न्यायेन कथञ्चिद्गुणादीनां गोशब्दवाच्यत्व \textendash\  सिद्धावपि साक्षाद्गोशब्देन ते नोच्यन्तेऽत आह \textendash\ नन्वेवमपीति~॥
\end{multicols}

\fancyhead[RE]{[ १ अ. १ पा. १ पस्पशाह्रिकै}
\fancyhead[LO]{शास्त्रप्रयोजनाधिकरणम् ]}
\fancyhead[LE,RO]{\thepage}
\fancyhead[CE]{उद्द्योतपरिवृतप्रदीपप्रकाशितमहाभाष्यम्~।}
\fancyhead[CO]{महाभाष्यप्रदीपोद्दयोतव्याख्या छाया~।}
\cfoot{}
\newpage
%%%%%%%%%%%%%%%%%%%%%%%%%%%%%%%%%%%%%%%%%%%%%%%%%%%%%
\renewcommand{\thepage}{\devanagarinumeral{page}}
\setcounter{page}{16}

१६ उद्द्योतपरिवृतप्रदीपप्रकाशितमहाभाष्ये \textendash\ [ १ अ. १ पा. १ पस्पशाह्निके

\begin{multicols}{2}
\begin{center}
(उत्तरभाष्यम्)
\end{center}

{\qt नेत्याह, क्रिया नाम सा~॥}

\begin{center}
(प्रश्नभाष्यम्)
\end{center}

{\qt यत्तर्हि तत् शुक्लो नीलः कपिलः कपोत इति, स शब्दः ?}\\

( प्रदीपः ) शुक्लो नील इति~। द्रव्यस्य प्रागुपन्यासाद्गुण \textendash\ मात्राभिधायिनोsत्र शुक्लादयो द्रष्टव्याः~॥ 

\begin{center}
( उत्तरभाष्यम् )
\end{center}

{\qt नेत्याह, गुणो नाम सः~॥}

\begin{center}
( प्रश्नभाष्यम् )
\end{center}

{\qt यत्तर्हि तत् भिन्नेष्वभिन्नं छिन्नेष्वच्छिन्नं सामान्य \textendash\ भूतं, स शब्दः ? }\\

( प्रदीपः ) भिन्नेष्वभिन्नमिति~। अनेन सामान्यस्य {\qt एकत्वं} कथ्यते~। {\qt छिन्नेष्वच्छिन्नं} इत्यनेन तु {\qt नित्यत्वम्}~॥ सामान्यभूतमिति~।  सत्ताख्यं महासामान्यं गोत्वादेः सामान्य \textendash\ विशेषस्योपमानं निर्दिष्टम्~। सामान्यमिव \textendash\ सामान्यभूतम्~। भूतशब्द उपमार्थे, यथा \textendash\ पितृभूत इति~॥

\noindent
\rule{1\linewidth}{0.5pt}\\

[ १ मभा० १ मप० ] भाष्ये \textendash\ क्रियेति~। प्राधान्येन कृदन्त \textendash\ धातुवाच्यत्वादिति भावः~॥\\

[ २ यभा० १ मप० ] कपोतं$=$चित्रः~। {\qt इति} शब्दस्य प्रयोजनं तूक्तम्~॥\\

[ ३यभा० १ मप०] गुणो नामेति~। द्रव्यकर्मेतरसामान्यवत्त्वाच्छब्दस्य गुणत्वेsपि रूपादीनां गुणत्वयोगेनैव शब्दसाधर्म्यम्, न तु शब्दस्वयोगेनेति न दोषः~॥\\

[ २ यप्र० २ यप० कैयटे \textendash\ तुरुक्तवैलक्षण्ये~॥\\

[ द० भागे उ०१ मप० ] न्यूनतां परिहरति \textendash\ अनेकेति~। सथा च {\qt नित्यमेकमनेकानुगतं सामान्यम्} इति तल्लक्षणं सूचितम्~॥ तत्राद्यकृत्वमाह \textendash\ [ २ यप ० ] नित्येति~॥ एकमिति स्वरूपकथनपरं न विशेषनिवृत्त्यर्थम्~। तेषां प्रतित्वं भिन्नत्वात्~। तद् एकत्वेनेति~॥ प्रमाणमिति~। सामान्यसद्भावे इति शेषः~॥\\

[ उ० ४ र्थप० ] एवंहीत्यस्य नेदं युक्तमित्यत्रान्वयः~॥ सामान्यविशेषयोरुपमानोपमेयभावाङ्गीकारे हीति तदर्थः~॥ सर्वेति~। सामान्यविशेषरूपेत्यर्थः~॥ [ ६ ष्ठप० ] अध्याहारे इति~॥ गोत्वा \textendash\ दिरूपसामान्यविशेषवाचिशब्दाध्याहार इत्यर्थः~॥ नन्वेवं झटिति श्रुत्यनुपपत्तिरत आह \textendash\ किंत्विति~। अर्थान्तरपरतया तस्योप \textendash\ पत्तिरिति भावः~॥ ममु खरूपवाचित्वे न मानमत आह \textendash\ \\

\noindent
\rule{1\linewidth}{0.5pt}\\

१ {\qt व्यक्तयस्तु पदा \textendash\ } इति मुद्रितपाठः~।\\

२ अनेन छायाग्रन्थेन वृद्धिसूत्रस्थप्रदीपे {\qt वृत्तिविषये च} इत्यत्र झश्चकारः स चान्ते योज्य इति बोध्यते~। एवं हि वृद्धिसंज्ञासूत्रे कैयटः \textendash\ {\qt प्रमाणभूत इति} \textendash\ प्रामाप्यं प्राप्त इत्यर्थः~। भू प्राप्ता \textendash\ 

\columnbreak

(उद्योतः) एकत्वमिति~। अनेकसमवेतत्वस्याप्युलक्षणम्~। नित्यत्वेन संयोगनिरासः~। एकत्वेन प्रत्यभिज्ञप्रत्यक्षरूपं प्रमाणं दर्शितम्~।  भूतशब्दप्रयोगालुपपत्तिं परिहरति \textendash\ सत्ताख्यमिति~। एवं हि {\qt वृक्षवदाम्रः} इत्याद्यापत्तैः, सामान्यश्रुतेः सर्वसामान्यविषयत्वेन प्रवृत्तायाः संकोचे कारणाभावाच्च, उपमाकथनस्य प्रकृतेऽनुपयोगाच्च, अध्याहारे गौरवाच्च \textendash\ नेदं युक्तम्~। किं तु स्वरूपवाची सः~। वृद्धिसंज्ञासूत्रस्थभाष्यप्रयोगस्य {\qt प्रमाणभूतः} इत्यस्य स्वयं करिष्यमाणव्याख्यानरीत्याsस्यापि व्याख्यानसंभवात्~॥ {\qt पितृ \textendash\ भूतः} इत्यत्रापि {\qt अन्यत्रान्यशब्दप्रयोगः सादृश्यपरः} इति सादृश्यप्रतीतिः, न त्वस्य सादृश्यवाचकत्वे मानमस्तीति दिक्~॥

\begin{center}
(उत्तरभाष्यम्)
\end{center}

{\qt नेत्याह, आकृतिर्नीम सा~॥}\\

(उद्द्योतः) भाष्ये \textendash\ आकृतिः \textendash\ जातिः संस्थानं न, आक्रियते \textendash\ व्यवच्छिद्यते स्वाश्रयोsनयेति व्युत्पत्तेरिति भावः~॥ शङ्कापरभाष्ये {\qt सामान्यभूतम्} इत्यस्य तद्व्यञ्जकं चेत्यप्यर्थो बोध्यः~॥ जातिमात्रपरत्वे {\qt जात्याकृतिव्यक्त्यः पदार्थः} इति गौतमसूत्रेण तस्यापि पदार्थत्वबोधनात् प्रत्यक्षादौ तद्भानाच्च तस्य शब्दत्वाशङ्कातत्समाधानाभावेन न्यूनता स्यात्~॥ {\qt सिद्धे शब्दार्थ} \textendash\ इति वार्तिकव्याख्याने

\noindent
\rule{1\linewidth}{0.5pt}\\

\noindent
[ ७ भप० ] वृद्धीति~। २अत एव चोक्तिः, स चान्ते योज्यः~॥ सामान्यशब्दस्तद्वृत्तिधर्मपरः~। {\qt भूतं} इति भूप्राप्तावित्यस्य आधृ \textendash\  षाद्वा इति णिजभावे रूपम्~। प्राप्तमिति तदर्थ इति भावः~॥ ननु पितृभूत इत्यस्यानुरोधेन तथैव कुतो नात आह \textendash\ [ ८ सप० ] पितृभूत इति~॥ इति \textendash\ इति न्यायेन~॥ [ १ ०भ० ] दिगिति~। तदर्थस्तु तत्रापि स्वरूपपरत्वेनैव निर्वाहः~। किं च {\qt प्रमाणभूतः} इत्यादाविव निर्वाहस्तत्र~।  किं च साभान्यं च तद्भूतं चेति कर्मधारयः~। भूतशब्दः परमार्थवाची सामान्यस्यापोहरूपतां प्रतिक्षेप्तुमुपात्तः, तस्याघटादिव्यावृत्त्यात्मनो घटोऽयं घटोऽयमित्याद्यनु \textendash\ गताकारप्रत्ययविधिमुखप्रवृत्तशब्दविषयत्वायोगेनाप्रामाणिकत्वादिति बोध्यमिति~॥\\

यत्तु \textendash\ आकृतिर्जातिरत्र वक्ष्यमाणव्युत्पत्तेः~। अवयवसंयोगरूपाकृतेर्गुणत्वात् {\qt गुणो नाम} इत्यनेन गतार्धत्वादितिकृष्णः तदसदिति ध्वनयन् तथैव वाच्य आकृति पदस्वारस्यादित्याह \textendash\ [ २ य \textendash\ उ० १ मप० ] संस्थानं चेति~। रूपक्रियादिविशिष्टमवयवसंस्थानं चेत्यर्थः~।  तादृशतदाकार इति यावत्~। अत एव गां लिखेत्यादौ लेखनकर्मत्वोपपत्तिः~। {\qt पिष्टकमय्यो गावः} इस्यत्राकृतेः पदार्थत्वमावश्यकम्~॥ एवं व्याख्याने बीजान्तरमाह \textendash\ [ ४ र्थप० ] \textendash\ जातीति~॥ यस्तु पेति~। अत्र {\qt तुशब्दो विशेषणार्थः~॥} किं

\noindent
\rule{1\linewidth}{0.5pt}\\

\noindent
वित्यस्याधृषाद्वेति णिजभावे रूपम्~। वृत्तिविषये च प्रमाणशब्दः {\qt प्रामाण्ये वर्तते} इति, एतदनुरोधेन {\qt सामान्यभूतं} इति भाष्यस्य व्याख्यानमाह \textendash\ सामान्येत्यादिना~। एवब्च सामान्यभूतमित्यस्य {\qt सामान्यत्वं प्राप्तं} इत्यर्थः~।
\end{multicols}

\newpage
% शास्त्रप्रयोजनाधिकरणम् ] महाभाष्यप्रदीपोद्द्योतव्याख्या छाया~। १७

\begin{multicols}{2}
\noindent
भाष्ये तस्या अपि१पदार्थत्वं नित्यत्वञ्च कैयटो वक्ष्यति~।  {\qt तदभिन्नाभिन्नस्य तदभिन्नत्वम्} इति न्यायस्तु यत्र २स्वरभिन्नद्वारकः स्वस्व रूपानुगमस्तद्विषयः~। यथा \textendash\ परमकारणकार्योपादानके परमकारणाभेद~।  अत एव ३सुखदुःखयोर्नैक्यम्~। नापि कुण्डलकटकयीस्तत्, कुण्डलारम्भवस्वर्णाव्यवानुगमस्य कटकेऽभावात्~। एवं गुणसमूहस्या \textendash\  तद्भूतावयवभेदस्य वाच्यत्वेऽपि समूहिनां तत्तद्रूपेणावाच्यत्वान्न तेपां शब्दत्वाशङ्केति भावः~॥

\begin{center}
( प्रश्नोपसंहारभाष्यम् )
\end{center}

{\qt ४कस्तर्हि शब्दः ?}\\

( प्रदीपः ) द्रव्यादिषु निरस्तेषु पृच्छति \textendash\ कस्तर्हीति~॥ \\

(उद्द्योतः ) पृच्छतीति~। प्रश्नविषयविशेषान्तराभावात्सामा \textendash\ न्यरूपेण पृच्छतीत्यर्थः~॥

\noindent
\rule{1\linewidth}{0.5pt}\\

\noindent
विशिष्यते ? अप्रधानोपसर्जनभावस्यानियमेन६ पदार्थत्वम्~। यदा भेदविवक्षा७ विशेषगतिश्च तदा व्यक्तिः पदार्थोऽङ्गम् जात्याकृती~। यथा गौस्तिष्ठति गौर्गच्छतीति~। ८तयोः स्थाना \textendash\ दीनामसंभवात् सा पदार्थः~। यदा पुनरभेदो विवक्षितः सा \textendash\ मान्यगतिश्च तदा जातिः पदार्थः, इतरे तिशेषणे~। यथा {\qt गौर्न पदा स्प्रष्टव्यः}? इति तद्वार्तिक(तद्भाष्य)कारः~॥ तस्यापि \textendash\  संस्थानस्यापि~॥ नन्वेवमपि तस्य पदार्थत्वं न भाष्यादिसंमतमिति न साऽत आह \textendash\ सिद्धे इति~॥ पदार्थत्वमिति~। उक्तमिति शेषः~॥ नन्वेवमपि व्यक्तिवत्तस्यानित्यत्वेन {\qt भिन्नेषे \textendash\ } इति लक्ष \textendash\ णानाक्रान्तत्वाच्छङ्काया एवाभावेन कथमुत्तरे तथाऽत आह \textendash\ नित्येति~। तत्रैवेति भावः~॥ ननु तदभिन्नेति न्यायेन शङ्कापक्षे एवं निराकरणे न्यायस्य निर्विषयतापत्तिरत आह \textendash\ तदभिन्नेति~॥ ९पक्षेण शङ्कायामेवं निराकरणाशयमाह \textendash\ एवमिति~॥ तद्वदित्यर्थः~॥ समूहिनाम् \textendash\ अवयवानाम्~॥ ननु यः शब्दते स शब्दो नेत्याह अर्थरूपं तत् इत्येकेन वाक्येनेष्टसिद्धेरनेकवाक्यवैयर्थ्यमिति चेत्~। न, मन्दबुद्धीनां स्पष्टप्रतिपत्त्यर्थं तथोक्तेः~॥\\

[ १ म उ० १ मप० ] पूर्वप्रश्नतोऽत्र विशेषमाह \textendash\ प्रश्नेति~॥

\noindent
\rule{1\linewidth}{0.5pt}\\

१ पदार्थत्वं कैयटो नित्यत्वञ्चः इति अ. पाठः~।\\

२ स्वाभिन्नद्वारक इति~। गुणेन गुणिनोऽभेदः, तेन च शब्दस्यामेद इति शब्दस्थ गुणेनाभेदः स्वीक्रियते {\qt तदभिन्न \textendash\ } इति न्यायेन~। परन्तु यत्र स्वाभिन्नेन स्वस्वरूपानुगमो भवति तत्रैवायं न्यायः प्रवर्तते, अत्र तु गुणाभिन्नेन क्रियाsभिन्नेन वाऽर्थेन गुणस्य क्रियाया वा स्वरूपाननुगमान्नायं न्यायः प्रवर्तत इति, एवं निराकरणेपि न न्यायस्य निर्विपयतापत्तिरिति भावः~।\\

३ वस्तुतस्तु सुखदुःखयोः कटककुण्डलयोरपि कारणात्मनाऽभेद एव~। कार्यरूपेण तु भेदः~। तदाहुः \textendash\ {\qt कार्यरूपेण नानात्वमभेदः कारणात्मना~। हेमात्मना यथाभेदः कुण्डलाद्यात्मना भिदा~॥} इति~। तदभिन्नाभिन्नस्येति न्यायोदाहरणमप्येतदेवेति बोध्यम्~। (र.ना.) अविचारितरमणीयमेतत्~।\\

४ कस्तर्हीति~। यदि द्रव्यगुणादि न शब्दस्तर्हि गौरित्यत्र कः शब्दः पृच्छयत इत्यर्थः~। अत्रापि शब्दशब्दस्यार्थपरत्वेन शब्दपरत्वा \textendash\ भावेनानुपयोगात् इतिःशब्दो न प्रयुक्तः~। अर्थपरस्य च तस्य\\

३ प्र०पा० 

\columnbreak

\begin{center}
( समाधानभाष्यम् )
\end{center}

येनोनच्चारितेन सास्नालाङ्गूलककुदखुरविषाणिनां संप्रत्ययो भवति, स शब्दः~॥\\

(प्रदीपः ) उत्तरमाह \textendash\ येनोच्चारितेनेति~। वैयाकरणा वर्णव्यतिरिक्तस्य पदस्य वाक्यस्य वा वाचकत्वमिच्छन्त्ति~। वर्णानां प्रत्यैकं वाचकत्वे द्वितीयादिवर्णोच्चारणानर्थक्यप्रसङ्गात्~। आनर्थ \textendash\ क्ये तु प्रत्येकमुत्पत्तिपक्षे यौगपद्येनोत्पत्त्यभावात्, अभिव्यक्तिपक्षे तु क्रमेणैवाभिव्यक्त्या समुदायाभावात्~। एकस्मृत्युपारूढानां वाचकत्वे {\qt सरः रसः} इत्यादावर्थप्रतिपत्त्यविशेषप्रसङ्गात्तद्व्यतिरिक्तः स्फोटो नादाभिव्यङ्ग्यो वाचको विस्तरेण वाक्यपदीये व्यवस्थापितः~। उश्चारितेन प्रकाशितेनेल्यर्थः~॥\\

( उद्द्योतः ) भाष्ये \textendash\ उच्चारितेनेति~।  शारीरमारुताभिहतकण्ठादिस्थानैरवयवद्वाराऽभिव्यक्तेन येनेत्यर्थः~।

\noindent
\rule{1\linewidth}{0.5pt}\\

\noindent
नच कर्मव्युत्पत्त्या शब्दत्वनिरासेस्पि भावव्युत्पत्तिरूपप्रकारान्तरं संभवतीति वाच्यम्, तत्र क्रियारूपमेवोत्तरमित्याशयात्~। अन्यस्य प्रकारस्य निर्वक्तुमशक्यत्वात्~। तत्र भासमानशब्दस्य तत्त्वस्र्येष्टत्वाच्चेति भावः~॥\\

[२ यभाष्ये १ मप० ] उच्चारितेनेतिविशेषणेन गृहीतसंकेतरय स्वरूपसतो बोधकत्वं निरस्तम्~॥ प्रकाशितस्यैव तत्वात्~॥\\

[ वामे २ यप्र० १ मप० ] कैयटे \textendash\ उत्तरमिति~। सिद्धान्ती \textendash\ त्यादिः~।  शब्दपदं करणघञन्तमिति भावः~॥\\

[ २ यप्र० २ यप० ] पदस्यापि कल्पितत्वादाह \textendash\ कैयटे \textendash\ वाक्यस्य चेति~॥\\

[२ यप्र० ५ मप० ] कैयटे \textendash\ तुरप्यर्थे~॥\\

[ २ यप्र० ६ ष्ठप० ] कैयटे \textendash\ आदिना \textendash\ {\qt जरा, राज?} नदी, दीन? इत्यादिपरिग्रहः~। तत्र \textendash\ प्रसङ्गादिति~। वर्णानामुभयत्रा \textendash\ विशेषादिति भावः~॥\\

[ २ यउ० २ यप० ] \textendash\ स्थानैः~। {\qt साक्षादवयवद्वारायाऽभि \textendash\ } इति पाठः~॥ तत्राखण्डस्थले साक्षात्~। सखण्डस्थलेअवयवद्वारेति बोध्यम्~॥

\noindent
\rule{1\linewidth}{0.5pt}\\

\noindent
पृच्छधात्वर्थफलेन सहाभेदेनान्वयः\\

५ येनोच्चारितेनेति~। येनेति करणे तृतीया, येन प्रकाशितेनस्फोटेन सास्नादिमतां ज्ञानं भवति सः \textendash\ स्फोटः शब्द इत्यर्थः~। नतु स्फोटस्यास्मिन्भाष्येsनुल्लेखादेवमर्थस्वीकारे किं बीजमिति चेत्, उत्तरभाष्यमेवात्र मानम्~। तेन हि ध्वनेः शब्दत्वमुच्यते~।  ई्दृशार्थाभावेऽस्यापि स एवार्थः स्यादित्येतदस्य भाष्यस्य स्फोटपरत्वव्याख्याने बीजम्~। अन्यथाऽनन्वयापत्तिरिति~।\\

६ अनियमेन विशेष्यविशेमणभावानापन्नस्य पदार्थत्वं विशेष्यते इति योजना~। ( र. ना. )\\

७ जातिव्यक्त्याकृतीनां परस्परमिति शेषः~। विशेषगतिरित्यत्रापि व्यक्तीत्यादिः~। एवमग्रेऽपि {\qt पुनरभेदः} इत्यत्र {\qt सामान्यगति}रित्यत्रापि चेमावेव शेषौ क्षेयौ~। ( र. ना. )\\

८ तयोः \textendash\ जात्याकृत्योस्तिष्ठत्यर्थासम्भवाद्व्यक्तिः पदार्थ इ्त्यर्थः~।\\

५ तदभिन्नेतीत्वादिः~। ( र. ना. )
\end{multicols}

\newpage
% १८ उद्द्योतपरिवृतप्रदीपप्रकाशितमहाभाष्ये \textendash\ [ १ अ. १ पा. १ पस्पशाह्निके

\begin{multicols}{2}
१अत्र विषाणान्तैरवयवैर्गुणादयोऽप्युपलक्ष्यन्ते~॥ संप्रत्ययः \textendash\ ज्ञानम्~॥\\

ननु प्रत्यायकशब्दस्य वर्णसमूहरूपतया {\qt येन}इत्येकवचनमयुक्तम्~। न च {\qt वनम्} इतिवत् समूहाभिप्रायं तत्, समूहस्य स्थिरस्य निरूपयितुमशक्यत्वादत आह \textendash\ वैयाकरणा इति~॥ वर्णव्यतिरिक्तपदानभ्युपगमे एकैकस्य वर्णस्य वाचकत्वं, समुदायस्य वा, नाद्य इत्याह \textendash\ वर्णानां प्रत्येकमिति~॥ द्वितीयमनूद्य दूषयति \textendash\ आनर्थक्ये तु प्रत्येकमिति~। प्रत्येकमानर्थक्ये तु समुदायस्य वाचकत्वमुपेयम् , तत्तु न युक्तम् \textendash\ इति शेषः~। यतस्तत्र नयद्वयम् \textendash\ उत्पद्यमानसमुदा \textendash\ यस्य \textendash\ अभिव्यज्यमानसमुदायस्य वा वाचकत्वमिति~।  तत्राद्यं दूषयति \textendash\ उत्पत्तिपक्षे इति~। {\qt उत्पत्त्यभावात्} इत्यस्य {\qt समुदायाभावात्} इत्यनेनान्वयः~॥ द्वितीयं दूषयति \textendash\ अभिव्यक्तीति~। उत्पन्नाभिव्यक्तेर्ध्वस्तत्वादित्यर्थः~॥ ननु पक्षद्वये५पि पूर्वपूर्ववर्णानुभवजसंस्कारसहकृ तान्त्यवर्णानुभवहेतुकैकस्मृत्युपारूढानां वाचकत्वमस्त्वत आह \textendash\ एकस्मृतीति~॥ ननु तस्य नित्यत्वे सर्वदा र्थप्रतीतिप्रसङ्गोऽत आह \textendash\ नादेति~। नादः \textendash\ वर्णः~॥ ननु येन क्रमेणानुभूता वर्णस्तेनैव क्रमे \textendash\ 

\noindent
\rule{1\linewidth}{0.5pt}\\

[ २ यउ० ३ यप० ] न्यूनतां परिहरति \textendash\ अत्रेति~। अनेन प्रागुक्तरीतिः सूचिता~। वहुवचनं तु व्यक्त्यभिप्रायमिति भावः~। यत्तु \textendash\ संप्रत्ययः स्मरणम् {\qt एकसंबन्धि \textendash\ } न्यायेन \textendash\ इति रत्नकृत्~। तन्न, पदार्थोपस्थितेः स्मृतित्वस्य मञ्जूषायां निराकृतत्वात्~। अत आह \textendash\ ज्ञानमिति~॥\\

[ २ यउ० ५ भप० ] समूहाभीति~। अनुद्भूतावयवभेदक \textendash\ समूहेत्यर्थः~॥ समूहेति~। {\qt तद्वदत्र} इत्यादिः~॥ स्थिरस्य \textendash\ चिर कालस्थायिनः~॥\\

[ २ यज० ] पदाधिक्यपरिहारायाह \textendash\ [ ८ मप० ] द्वितीय \textendash\ मनूद्येति~। आद्यदोषमित्यर्थः~। तत्राद्यमाह \textendash\ प्रत्येकमानर्थक्ये त्विति~। वर्णानां प्रत्येकं वाचकत्वे द्वितीयादिवर्णानामानर्थक्ये त्वित्यर्थः~॥ शेषपूरणमाह \textendash\ [ ९ मप० ] समुदेति~। तत्तु \textendash\ तदपि~॥ [१०मप०] तञ्र \textendash\ समुदायस्य वाचकत्वे~॥ नयेति~। मतेत्यर्थः~॥ उत्पद्येति~। षष्ठीसमासः~॥ एवमग्रेऽपि~॥\\

[२ य उ०] क्रमेणैवेत्यत्र हेतुमाह \textendash\ [१३ शप०] उत्पन्नेति~॥ [१५ शप०] मस्त्वेति~। एवं च पक्षद्वयमप्यदुष्टमिति भावः~॥\\

[ २ य उ० ] ध्वनिवारणायाह \textendash\ [ १७ शप० ] वर्ण इति~॥ वर्णसत्त्वपक्षे साक्षाद् ध्वनिव्यङ्ग्यत्वस्य २तत्रासंभवादिति भावः~॥ ननु येन क्रमेणानुभवास्तेनैव क्रमेण स्मरणानीति दोषनिवारणमत्र दुर्वचम्, स्मृतेरत्रैकत्वादत आह \textendash\ ननु येनेति~। नन्विति स्फोटा \textendash\ भावशङ्कायाम्~॥ [१९ शप०] पूर्वोक्तेति~। अर्थप्रतीत्यविशेषप्रसङ्गरू \textendash\ पादित्यर्थः~।  तदुक्तम् \textendash\

\begin{quote}
{\qt ३यस्यानवयवः स्फोटो व्यज्यते वर्णबुद्धिभिः~।\\
सोऽपि पर्यनुयोगेन न केनापि विमुच्यते~॥~इति~॥}
\end{quote}

[२१ शप ०] प्रत्यय इति~। न चायं वर्णगोचरः, तेषामनेक \textendash\ 

\noindent
\rule{1\linewidth}{0.5pt}\\

१ अत्र विषाणेति \textendash\ भाष्ये {\qt ककुदखुरविषाणिना} इत्येतत् शुक्लादिगुणानां क्रियादीनाञ्चोपलक्षकमिति भावः~॥\\

२ स्फोटे \textendash\ इत्यर्थः~। (रः ना. ) ३ मते इति शेषः~। पर्यनुयोगेनेत्यस्य

\columnbreak

\noindent
णैकस्मृत्युपारूढानां वाचकत्वेन नार्थप्रतीत्यविशेषप्रसङ्गः, अन्यथा त्वयाऽपि वर्णानां स्फोटव्यज्जकत्वाभ्युपगमात् कथं पूर्वोक्तदोषान्निस्तार \textendash\ स्तवापीत्यत आह \textendash\ विस्तरेणेति~। इदं \textendash\ {\qt एकं पदम्} {\qt एके वाक्यम्} इति प्रत्ययः स्फोटसत्त्वे तदैक्ये च प्रमाणम्~। अनुभवक्रमेणैव स्मरणमित्यर्थं दृढप्रमाणाभावाच्च~। क्रमेणानुभूतानां व्युत्क्रमेणापि स्मरणदर्शनात्~। मम तु यथा पटे नानारञ्जकद्रव्याहितनानावर्णोपरागः क्रमेण, तथा एकस्मिन्नेव तस्मिन्नुच्चारणक्रमेण क्रमवानेव तत्तद्वर्णस्वरूपानुरागः, स च स्थिरः, तस्य च मनसा ग्रहणमिति न दोषः~॥ तत्र {\qt येन} इत्येकवचनेन {\qt विषाणिनाम्} इति बहुवचनेन चानेकोच्चारणविषयगौरित्यस्यैकत्वं सूचितम्~। एवं च तेनैव दृष्टान्तेन घटपटादिबोधकसाधारण्येन वाचकस्यैकत्वं सूचितम्~। भेदप्रत्ययस्त्वौपाधिक इत्याद्यस्मत्कृ \textendash\ तमञ्जूषायां शक्तिवादे द्रष्टव्यम्~। नन्वेवं तस्य ताल्वोष्ठपुटव्यापारविषयत्वाभावात् {\qt उच्चारितेन} इत्ययुक्तमत आह \textendash\ प्रकाशितेनेति~। अभि \textendash\ व्यञ्जकैरिति शेषः~॥

\noindent
\rule{1\linewidth}{0.5pt}\\

\noindent
मिति प्रतीत्या नानात्ववत्त्वात्~। न चायं आन्तः, बाधाभावादित्याद्यन्यत्र बोध्यम्~॥ युक्तयन्तरेण तं संसाध्य कैयटोक्तदोषमपि प्रतिपादयन् तां ४खण्डयितुं हेतुमाह \textendash\ अनुभवेति~। तदुक्तम् \textendash\ 

\begin{quote}
{\qt नानुभूतिकमाद्वर्णाः क्रमिकाः स्मृतिगोचराः~।\\
५गोचरोsनुभवो न स्यात्६क्रमस्ते कुतः पुनः~॥~इति~॥}
\end{quote}

अत्र हेतौ हेतुमाह \textendash\ [२२ शप०] क्रमेणेति~। प्रपूर्वदिनानुभूतं विहाय पूर्वदिनानुभूतस्मृतिस्थल इति भावः~॥ नन्वेवं वर्णानां स्फोट \textendash\ व्यञ्जकत्ववादिनस्तव कथं निर्वाहोऽत आह \textendash\ [२३ शप०] मम त्विति~॥ तस्मिन् \textendash\ स्फोटे~। उच्चारणक्रमेणेति~। न तु भवदुक्तरीत्येति भावः~॥ ननु तस्यापि क्षणिकत्वे प्रागुक्तदोष एवात आह \textendash\ [२५ शप०] स चेति~। स्फोटश्चेत्यर्थः~॥ पारमार्थिकत्वादाह \textendash\ स्थिर इति~॥ तस्य श्रोत्राग्राह्यत्वादाह \textendash\ तस्य चेति~। इदमन्यत्र स्पष्टम्~॥ न चैवं प्रतिज्ञाविरोधः, लिप्यादिवत्तदुपायत्वात्~। व्यञ्जकानुशासनद्वारा व्यङ्गयैवानुशासनाच्च~॥ ननु {\qt येन} इत्येकवचनस्य स्फोटाभिप्रायकत्वेऽपि प्रश्नवाक्यवदेकवचनमेव विषाणिनामित्यत्र युक्तमत आह \textendash\ [२५ शप० ] अत्रेति~। भाष्य इत्यर्थः~॥ तत्र इति पाठेऽप्येवम्~॥ नेकोच्चारणेति~। व्यक्तिभेदात्कर्तृभेदाच्चेत्यर्थः~॥ अधिकमाह \textendash\ एवं चेति~। तस्य तत्त्वसूचने चेत्यर्थः~॥\\

[ १८ ५० उ० २८ प० ] वाचकेति~। स्फोटस्येत्यर्थः~॥ सूचितमिति~। सुष्ठूचितमित्यर्थः~। समुचितमिति पाठत्तु सुगम एव~। कथं तहिं घटपटादिशब्दानां भेदप्रतीतिरत आह \textendash\ भेदेति~। उपाधिर्वर्णरूपः~॥\\

[ उ० २९ प० ] एव \textendash\ तदतिरिक्तत्वे~॥ तेषां बहुत्वात्तैरेव व्यञ्जितादर्थप्रतीतेरिष्टत्वाच्चाह \textendash\ [ ३० शप० ] अभीति~॥ एकेनैवाभिव्यकत्येष्टसिद्धेरन्यानर्थक्यम्, निरवयवस्य सर्वात्मनाsभिव्यक्त्तत्वैऽप्येकेनान्येषां स्फुटतराभिव्यक्त्यर्थत्वात्तरतमेन्द्रियसंबन्धेन स्फुट \textendash\ 

\noindent
\rule{1\linewidth}{0.5pt}\\

\noindent
प्रश्नेत्यर्थः~। (र०ना०)४युक्तिमित्यर्थः(र०ना०)५अनु \textendash\ भवान्तरेत्यादिः~। ( र. ना. ) ६ अनुभवक्रमेत्यर्थः~। ( र. ना. )
\end{multicols}

\newpage
% शास्त्रप्रयोजनाधिकरणम् ] महाभाष्यप्रदीपोद्द्योतव्याख्या छाया~। १९ 

\begin{multicols}{2}

\end{multicols}

\begin{center}
(समाधानान्तरभाष्यम्)
\end{center}

 अथवा \textendash\ प्रतीतपदार्थको लोके ध्वनिः शब्द इत्युच्यते~। तद्यथा \textendash\ शब्दं कुरु,
मा शब्दं कार्षीः, शब्दकार्ययं माणवकः, इति ध्वनिं कुर्वन्नेवमुच्यते~। 
तस्मात् ध्वनिः शब्दः~॥ 

 (प्रदीपः ) अथवेति~। अन्यत्र ध्वनिस्फोटयोर्भेदस्य व्यव \textendash\ 
स्थापितत्वादिहाभेदेन व्यवहारेsपि न दोषः, द्रव्यादयो न शब्दवाच्या इत्यत्र
तात्पर्यात्~॥ ध्वनिं कुर्वन्निति~। १विधिप्रतिषेध \textendash\ 
योरप्रवृत्तविषयत्वात्कथमस्य त्रिभिः संबन्धः ? उच्यतेशब्दं कुर्विन्न
{\qt शब्दं कुरु} इत्युच्यते विरामाशङ्कायां तन्निवृत्तये~। 
तथाऽनभिमतशब्दश्रवणोद्वेजितेनोच्यते \textendash\ मा शब्दं कार्षीरिति~॥ 



स्फुटतरक्रमेणान्त्यसंबन्धे स्फुटतमा व्यक्तिर्नं प्राक् संबन्धे यथा तथा
गकारादिध्वनिभिस्तरतमादिभावेन स्फोटभिव्यक्तिस्वीकार इति
तदनन्तरमेवार्थधीर्न ततः प्राक्~। एवमर्थाभिव्यक्तिस्तु न, स्फुटत्वस्य
संनिकर्षभेदेन प्रत्यक्षज्ञाननियतत्वात्स्फोटज्ञानस्य च प्रत्यक्षत्वात्~। 
अर्थो यद्यभिहितो नाभिव्यक्तः, संदिग्धश्चन्नाभिहित इत्याद्यन्यत्र
द्रष्टव्यम्~॥ 

 [द० १ मभा०] ननूक्तलक्षणलक्षितस्य लौकिकबहिर्भूतत्वात्कथमनुशासनम् ,
तदनुशासने च {\qt लौकिकानां} इत्याययुपक्रमविरोधापत्तिः~। यद्यपीदं
कौस्तुभादुक्तरीत्या सुसमाधानम् , तथापि {\qt लौकिकानां}
इत्यादिभेदेनोक्तेर्बहुवचनान्तेनोक्तेस्तथोदाहृतत्वाच्च
तथाऽनभिमतत्वस्य व्यवहारदशायां भगवदिष्टत्वेन
तदस्वारस्यादुपक्रमविरोधश्च~। किं च यस्मादर्थाप्रतीतिः कारणवशात्तस्य
शब्दत्वानापत्तिः~। किं च वर्णेभ्योऽर्थाप्रतीतिरेवेति वर्णानां
शब्दत्वव्यवहारः सार्वजनीनो विरुध्येत~। किंच
प्रागुक्तशब्दसामान्यलक्षणानाक्रान्तत्वं तस्य, मनसा तस्य ग्रहणात् \textendash\ अत
आह \textendash\ [१ मप० ] भाष्ये \textendash\ अथवेति~॥ प्रतीतः पदार्थ इति कर्मधारय एव~। 
स्वार्थे कन्~। तस्य च तद्वोधके लक्षणा३~। ध्वनय एव वर्णाः~। 
अन्यत्रोक्तरीत्या अनुभवेन च वर्णस्थले ध्वनीना. मावश्यकत्वेन तैरेव
स्फोटत्वेनाभिमताभिव्यक्तिसंभवे तयोरन्तरा वर्णाङ्गीकारे गौरवात्~। 
दूरगतकोलाहलश्रवणे घत्वादीनां श्रूयमाणध्वनिगतानामेव
दूरत्वदोषेणानभिव्यक्तिसंभवाच्च~॥ एव चैकैकवर्णोऽपि
सहकारिवर्णान्तरसंबन्धादनेकपदरूप इति सोऽपि तच्छक्तिरिति
गकारौकारविसर्जनीया गोर्वाचका इति लोकस्याविचारितरमणीये तत्समूहे
वाचकत्वग्रहः~। अखण्डोऽपि स्फोटो गकाराद्यवस्थाभ्यो
भिन्नोऽभिन्नश्च 



 १विधिप्रतिषेधयोः \textendash\ {\qt शब्दं कुरु} \textendash\ मा शब्दं कार्षीः?. इत्यनयोः
अप्रवृत्तव्रिषयत्वात् \textendash\ भूतभविष्यत्कालिकविषयत्वात् ध्वनिं
कुर्वन्नित्यनेन कथं सम्बन्ध इत्यर्थः~। तृतीयेन सम्बन्धस्य
सिद्धत्वाद्वाभ्यां कथमित्येवाशङ्क्यते~। अत एव द्वयोरुत्तरितम्~। 

२ कानि पुनरिति~। अत्र शब्दानुशासनपदस्य शब्दज्ञानफलक \textendash\ 
व्याकरणाध्ययनेऽर्थे वृत्तिरित्युद्द्योतः~। तेन
शब्दज्ञानफलकव्याकरणाध्ययनस्य कानि प्रयोजनानि \textendash\ इति भाष्यार्थः~। पुनः शब्दः
शब्दानुशासनपदस्य विशिष्टार्थे वृत्ति बोधयति~। अथवा अथशब्दातुशासनम्
इत्यनेन व्याकरणस्य शब्दानुशासनरूपं प्रयोजनमुक्तम्, इदानीं
तच्छब्दानुशासनस्य प्रयोजनं पृछ्यते~। अत एव {\qt प्रयोजनप्रयोजनानि तु
रक्षोहादीनि} इत्युक्तं कैयटेन~॥ 





 (उद्द्योतः ) भाष्ये \textendash\ अथवा प्रतीतपदार्थक इति~। लोके व्यवहर्तृषु
पदार्थबोधकत्वेन प्रसिद्धः श्रोत्रेन्द्रियग्राह्यत्वाद्वर्णरूपध्व \textendash\ निसमूह
एव शब्द इत्यर्थः~। तस्यार्थबोधकताऽप्यविचारितरमणीयस्यैव लोके सिद्धा,
तादृशस्यैव शास्त्रेणान्वाख्यानमिति तात्पर्यम्~। तदाह \textendash\ अन्यत्रेति~। 
संग्रहादौ, तपरसूत्रे भाष्ये चेत्यर्धः~। ध्वनिपदेनात्र वैखरी,
स्फोटपदेनाभिव्यक्तकत्वादिको मध्यमावस्थ आन्तरः शब्द उच्यते~॥ कथमस्येति~। 
कुर्वन् इत्यस्य वर्तमानार्थत्वादित्यर्थः~॥ 

 ( शास्त्रप्रयोजनाधिकरणम्) 

 ( प्रयोजनजिज्ञासाभाष्यम् ) 

 २कानि पुनः शब्दानुशासनस्य प्रयोजनानि ? 



चरमावस्थया विसर्जनीयादिरूपया पदादिरूपेण व्यज्यत इति वर्णेषु
तत्त्वव्यवहारोपपत्तिर्लोकस्येति बोध्यम्~॥ 

 [१मभा० २ यप०] भाष्ये {\qt इत्युच्यते} इत्यनेन सिद्धान्तत्य
नभिमतत्वमस्य सूचितम्~॥ 

 [१ मभा०] प्रतीतत्वमेव विशदयति \textendash\ भाष्ये [२ यप०] तद्यथेति~॥
ध्वनिं कुर्वन्नित्यैवमुच्यत इति संबन्धः~॥ यद्वाध्वनिं कुर्वन्नेवमुच्यते
~। एवंपदार्थ एव \textendash\ इतीत्यन्तः~॥ 

 [१ मभा० ४ रथप०] तथा चार्यं हानोपादानोपेक्षात्मा त्रिविधो व्यवहार
इत्यन्वयव्यतिरेकाभ्यां तस्य श्रावणत्ववाचकत्वयोरवसायः~। तद्ध्वनयन्
तदनुसारेणोपसंहरति \textendash\ भाष्ये \textendash\ तस्मादिति~। लोकव्यवहारादित्यर्थः~॥
ध्वतिः \textendash\ उक्तरूपः~॥ 

 [१ मप्र० ४ र्थप०] रप्रवृत्तेति \textendash\ कैयटे~। प्रवृत्तस्य प्रेरणा \textendash\ 
निषेधयोरसंभवादिति भावः~॥ एकेन संबन्धसंभवेऽपि समुदायेन
तदसंभवादाह \textendash\ त्रिभिरिति~॥ [६ ष्ठप०] तत्र तथाऽनभीति~। एवमुदासीनेन
वस्तुस्थितिः कथ्यते \textendash\ शब्दकारीति \textendash\ इति बोध्यम्~॥ 

 [१ म उ० १ म०प०] तदभिप्रेत्याह \textendash\ लोक इति~॥ तदर्थ \textendash\ माह \textendash\ [ २ यप० ]
व्यवेति~। अस्य बोधे प्रतीतौ वाऽन्वय इत्याह़ \textendash\ पदार्थेति~। अत एव
वक्ष्यति \textendash\ तस्येति~॥ प्रतीतपदार्थमाह \textendash\ प्रेति~॥ वर्णरूपत्वं ध्वनौ
तत्समूहे च बोध्यम्~॥ ननूक्तरीत्या तस्य नार्धबोधकत्वमत आह \textendash\ [ ३ यप० ]
तस्येति~॥ नन्वेवं कथं तस्य तत्त्वोक्तिरत आह \textendash\ तादृशेति~॥
अविचारितरमणीयस्यैवेत्यर्थः~॥ परं त्विदं तद्वद्वृथैवेति बोध्यम्~॥
एतदर्थमेवेयमुक्तिरिति भावः~॥ [ ५ मप०]आदिना हर्यादिपरिग्रहः~॥ 

 [१ म उ० ५मप०] नात्र \textendash\ कैयटे~॥ इदं चाग्रे सर्व स्फुटी \textendash\ भविष्यति~॥




३ इदं चिन्त्यम्~। प्रतीतः पदार्थो यस्मादिति बहुब्रीहिणा लक्षणां
विनैवाभीष्टार्थलाभात्~। (र. ना. )~। अत्र छायायां {\qt कर्मधारय एव}
इत्युक्तावपि {\qt इदं चिन्त्यं} इत्युक्त्वा बहुव्रीहिप्रदर्शनं साहसमात्रमेव
~। तद्वोधके लक्षणां वदतश्छायाकर्तुर्बहुब्रीहिरञ्चात इति वा
साहसिकस्यास्याभिप्रायः~। वस्तुतस्तु \textendash\ {\qt प्रतीतपदार्थकः} इति भाष्ये
प्रतीतशब्द प्रसिद्धर्थक इत्युद्दधोतदर्शनादवगम्यते, न शातार्थकः~। येन
{\qt यस्मातः इति बहुव्रीहिलब्धस्य} पञ्चम्यर्थस्य जन्यत्वस्य ज्ञानांशेsन्वयः
समाशङ्कयेत~। एवञ्च प्रसिद्धपदार्थबोधक इत्यर्थो लक्षणां विना दुर्लभ एव~। 
तत्र बदुब्रीह्याश्रयणे {\qt प्रसिद्धपदार्थहेतुः} इत्येवार्थः स्याद्~। स च
शब्दस्य पदार्थाहेतुत्वान्नैव सम्भवति \textendash\ इति~। 

२० उद्द्योतपरिवृतप्रदीपप्रकाशितमहाभाष्यम्~। 

 [१ अ. १ पा.१ पस्पशाह्निके



 ( प्रदीपः ) कानि पुनरिति~। किं संध्योपासनादिवद्व्या \textendash\ करणाध्ययनं
नित्यं कर्म \textendash\ अथ काम्यमिति प्रश्नः~॥ 

 (उद्योतः ) भाष्ये \textendash\ कानि पुनः शब्दानुशासनस्येति~। 
शब्दज्ञानफलकव्यारणाध्ययनस्येत्यर्थः~। तदाह \textendash\ किं सन्ध्येति~। 
शब्दज्ञानद्वारा \textendash\ अध्ययनस्य तदज्ञानप्राप्तप्रत्यवायपरिहारः फलम् , उत
तदतिरिक्तमपि किंचिदिति प्रश्नाशय इत्यर्थः~॥ प्रयोजनशब्दः
करणव्युत्पत्त्या १प्रयोजकपरोऽपि~। तेन व्याकरणाध्ययनस्य प्रयोजकं
किंचिदस्ति, न वा \textendash\ इत्यपि प्रश्नाशयो बोध्यः~। अत एवोत्तरे आगमपदो



 [ २ य उद्दयो० ] ननु शब्दानुशासनपदेनात्र व्याकरणं तद्वाच्यं
दुर्ग्रहम् तस्यानधीतस्य रक्षादिप्रयोजनासंभवात्~। तस्याक्रि \textendash\ यात्मकत्वेन
नित्यत्वकाम्यत्वासंभवेन कैयटासांगत्याच्च~। अत एव आदि \textendash\ असंगत्यापत्तेश्च~। 
अतो लाक्षणिकमर्थमाह \textendash\ [ १ मप० ] शब्देति~॥ [२ यप० ] तदाहेति
~। लाक्षणिकमर्थं हृदि कृत्वाऽऽहेत्यर्थः~॥ शब्दज्ञानफलकत्वादेव नेदं
साक्षात्प्रयोजनप्रश्नपरमित्याह \textendash\ शब्देति~॥ [ ३यप० ] \textendash\ ध्ययनेति~। 
व्याकरणेत्यादिः~। परिहारःस एव~॥ अपिना तस्य समुच्चयः~॥ आद्य
नित्यत्वम्~। द्वितीये काम्यत्वम्~॥ [ ४ र्थप० ] किंचिदिति~। अनेन
{\qt भाष्यस्थपुनःशब्दो विशेषे} इति सूचितम्~॥ प्रयोजनप्रश्नार्थस्तुं न, फलस्
प्रागुक्तत्वात्~। तद्वितीये तदवगतेश्च~॥ उत्तरानुरोधेनाह \textendash\ [ ४ र्थप० ]
प्रयोजेति~। {\qt बाहुलकात्कर्तृल्युडन्तः} प्रयोजनशब्दः प्रयोजकपरः इति
कृष्णकौस्तुभाद्युक्त्यसांगत्यध्वसायाह \textendash\ करणेति~। प्रयुज्यते
प्रवर्त्यतेऽनेनेतीत्यर्थः~॥ 

 [उ० ५ मप० ] अपिना फलपरसमुच्चयः~। तत्राद्यः
पुंलिङ्गोऽन्त्यो नपुंसकलिङ्ग इति बोध्यम्~। अत एव वक्ष्यति \textendash\ [ ७ मप०
] नपुंसकेति~॥ तेन \textendash\ आद्यपरत्वेन~॥ 

 [ ६ ष्ठप० ] किंचिदिति~। प्रत्यवायपरिहारातिरिक्तमपि
किंन्चित्मवर्तकमस्ति उत तद्रूपमेवास्तीत्यर्थः~॥ नन्वेनं प्रश्नोत्तरयोः
क्ं वचनभेदोऽत आह \textendash\ [ ७ मप० ] तत्रैकेति~। एकशेषे इत्यर्थः~॥
अत्रेदं बोध्यम् \textendash\ पुनःशब्दः पश्चादर्थकोऽपि सुयोजः, तादृशैतदध्ययनस्य
पुनः पश्चात् शब्दज्ञानानन्तरम् अन्यानि \textendash\ वा प्रयोजनानि 



१ \textendash\ {\qt रक्षोहलाधवासंदेहाख्यानि} चत्वारि फलानि, आगमस्तु प्रवर्तकः , इत्येवं
शब्दकौस्तुभेऽपि स्पष्टीकृतम्~॥ ( दाधि. ) 

२ एकशेष इति~। प्रयोजकपरस्य पुंल्लिगस्य प्रयोजनशब्दस्य
नपुंसकप्रयोजनशब्दस्य च~। 

३ प्रयोजनमिति~। ननु {\qt नीलो घटाः} इति अयोगवारणायासति विशेषानुशासने
विशेष्यविशेषणवाचकपदयोः समानवचनकत्वनियमाश्रयणात् {\qt  प्रयोजनम्} इति
वक्तुमुचितम्~। न च वेदाः प्रमाणमितिवदत्र प्रयोजनत्वेऽन्वयबोधः,
रक्षादिप्रत्येकपदार्थे प्रयोजनत्वपर्याप्तेरिति चेत् , न; {\qt प्रयोजनानि}
इत्यत्र पुन्नपुंसकप्रयोजकप्रयोजनार्थबोधकानां
प्रयो \textendash\ जनशब्दानामेकशेषमतिदेशप्राप्तैकवद्भावं च बोधयितुं तथा प्रयोगः~। 
एकवद्भावातिदेशे वैक्वचनमप्युपपद्यते,बहुत्वरूपोsर्थश्च न निषिध्यते~। 
अतिदिश्यमानधर्मविरुद्धस्याश्रयकार्याभाव एवातिदेशस्वभाव इति तात्पर्यम्~। 





पादानम्~। {\qt नपुंसकमनपुंसकेन} \textendash\ इस्येकैशेपः२~। तत्रैकवद्भावस्य
वैकल्पिकत्वात्प्रश्ने बहुवचनम् , उत्तरे एकवचनं बोध्यम्~॥ 

 ( प्रयोजनभाष्यम् ) 

 रक्षोहागमलध्वसंदेहाः प्रयोजनम्३~॥ 

 (रक्षापदार्थनिरूपणभाष्यम् ) 

 रक्षार्थं वेदानामध्येयं व्याकरणम्~। लोपागमवर्णविकारज्ञो हि
सम्यग्वेदान्परिपालयिष्यतीति~॥ 

 ( प्रदीपः ) पारम्पर्येण पुरुषार्थसाधनतामस्याह \textendash\ रक्षेति~। लोके
लोपाद्यदृष्टं वेदे दृष्ट्वा भ्राम्येदवैयाकरणः, वैयाकरणस्तु 



किं दृष्टानि किं वाऽदृष्टानि पुरुषार्थसाधनभूतानीति~। 
पुरुषार्थसाधनभूतानीति तु तत्पक्षेऽप्यावश्यकमितीति~॥ यद्यपि
प्रतिज्ञाऽवाक्येनैवानुवन्धचतुष्टयं सूचितम् , तथापि साधुशब्दज्ञानस्य
सुखदुःखाभावन्यतरारूपत्वेन साक्षादपुरुषार्थत्वमिति तत्साधनत्वमेवास्य
दृष्टद्वारकं वाच्यम्~। तत्रादृष्टं द्वारं प्रत्यवायपरिहाररूपं
स्वयमुपेक्ष्य दृष्टद्वारा तद्वोधकतानयोः६ स्फुटैव~॥ 

 [ २ यभा० १ मप० ] ननु साधुरक्षा नाध्ययनस्य प्रयो \textendash\ जनमत
आह \textendash\ भाष्ये \textendash\ रक्षार्थमिति~। अर्थशब्दः प्रयोजनवाची~॥ स्वरूपसताsनेन
वेदरक्षा दुष्कराऽत आह \textendash\ अध्येत व्येति~॥ अनेनैतच्छास्त्रस्येयं संज्ञेति
दर्शितम्~॥ एवमग्रेऽपि~॥ अत एव प्रश्ने {\qt शब्दानुशासनस्य} इति तथा
व्याख्यातम्~। अतथाव्याख्यातृणां त्वेतदर्थविरोधोsपि स्पष्ट एवेति बोध्यम्
~॥ ननु पाठमात्रेणाधीतेनाप्यनेन कथं वेदरक्षाऽत आह \textendash\ लोपेति ~॥प्रकृति \textendash\ 
प्रत्ययविभागादेरप्युपलक्षणमिदम्~॥ हि \textendash\ हेतौ~॥ तथा च न केवलं तथाध्ययनं
किंतु वाक्यार्थावगतिपर्यन्तमपीत्यनुपदमेव स्फुटीभविष्यति~॥ 

 [ १ मप्रदी० ] कैयटे \textendash\ मस्येति~। एतदध्ययनस्येत्यर्थः~॥ इद \textendash\ 
मुपलक्षणम् \textendash\ अस्यानुष्ठापकं मानमागमरूपमस्तीत्वाहेत्यपि कोध्न~। 
यद्वा \textendash\ अस्याध्ययनस्य
पारम्पर्येणाविशिष्टाध्ययनविधानेनाङ्गाध्ययनविधेर्लाभात्तद्वारा
धर्मादिसाधनतामित्यर्थः कैयटस्य तत्पक्षे वोध्यः~॥ 

 [प्र० २ यप० ] कैयटे \textendash\ वेदे इति~॥ त्मना देवेभिः इत्या \textendash\ दावित्यर्थः
~॥ 



४ तद्द्वैते इति पाठो भाति~। फलसंदेहे इति तदर्धः~। फल \textendash\ सुसङ्गतस्यापि
छायाग्रन्थस्य वैपरीत्याशङ्कनमनुचितमेव~। तथाहि \textendash\ पुनःशब्दः
प्रयोजनप्रश्नार्थो न, फलस्य \textendash\ शब्दानुशासनरूपस्य प्रागुक्तत्वात्~। 
तद्वितीये \textendash\ ( प्रयोजनप्रयोजनरूपे ) विशेषे तदतिरिक्तमपि किञ्चिदित्युक्ते
पुनःशब्दस्यावगतेश्चेति तदर्धः~। विपरीतपाठानुसारेण पुनः शब्दस्य
फलप्रश्नपरत्वे स्वीकृते {\qt प्रथोजनप्रश्नार्थस्तु} न इति तत्पूर्वग्रन्थस्य
का गतिरिति त एव प्रष्टव्याः~। 

५ अथ शब्दानुशासनमित्येतद्रूपेणेत्यर्थः~। (र. ना. ) 

६ अथ शब्दानुशासनमिति रक्षोहेत्यादिप्रतिज्ञावाक्ययोरित्यर्थः~। (र. नाः )
७ वेदेन सहेति शेषः~। (र. ना. ) 

शास्त्रप्रयोजनाधिकरणम् ] महाभाष्यप्रदीपोद्द्योतव्याख्या छाया~। 

 २१ 

 



न भ्रमति, वेदार्थं चाव्यवस्यति~। तत्र लोपागमयोरुदाहरणंदेवा
अदुह्रेति~। दुहेर्लङो झस्यादादेशे कृते लोपस्त {\qt आत्मनेपदेषु} इति तलोपः,
{\qt बहुलं छन्दसि} इति रुटि सति रूपमेतत्~। वर्णविकारो यथा \textendash\ उद्ग्राभं च
निग्राभं~। चेति~। {\qt हृग्रहोर्भश्छन्दसि हस्येति वक्तव्यम्} इति
भकारः~। {\qt उदि ग्रहः} इत्यत्र उद्ग्राभनिग्राभौ च छन्दसि
स्रुगुद्यमननिपातनयोः इति वचनादुन्निपूर्वाद्ग्रहेर्घञ्~॥ 

 (उद्द्योतः) पुरुषार्थः$=$धर्मो मोक्षश्च~॥ ननु शिष्याचार्यसंबन्ध एव
महान् वेदरक्षाहेतुः, किं व्याकरणेनेत्यत आह \textendash\ लोके इति~॥ भ्राम्येदिति~। 
भ्रमेण पाठान्तरं कल्पयेदित्यर्थः~। न चाचार्यवचनात् 



 [प्रदीपे सम्यगित्यादेरर्थमाह कैयटे \textendash\ [२ यप० ] \textendash\ वैयेति~। न
भ्राम्येति~। सिद्धं पाठं त्यक्त्वा पाठान्तरे न कल्पयेत्~। 
तत्तल्लक्षणदर्शित्वादनायासेन यथावस्थितपाठस्थापनं कर्तुं प्रभवेदिति यावत्
~॥ एवं च भवति वेदरक्षा फलमप्रधानम्, प्रधानं तु धर्मादिरूपपुरुषार्थएव~॥


 [प्रः ३ यप० ] कैयटे \textendash\ तत्र \textendash\ लोपादीनां मध्ये~॥ यद्यपि {\qt त्मना
व्योमन्} इति लोपमात्रोदाहरणम्, ब्राह्मणासः इत्यागममात्रस्य तथापि
लाघवादाह \textendash\ लोपागमेति~॥ 

 [ प्रदीपे ] यद्यपि वर्णविकारमात्रस्य जभार गृभ्णामि इत्यु \textendash\ दाहरणम्
, तथापि विशेषं प्रतिपादयितुं छान्दसप्रत्ययसंवलितस्य उदाहरणमाह \textendash\ कैयटे [
६ ष्ठप० ] उद्ग्राभं चेति~॥ 

 [ प्रदी० ९ मप० ] निपातनयोरिति वार्तिके पाठः~॥ एवं च निग्राभमिति
छन्दस्येव, उद्ग्राभमित्यत्रार्थाशा एव निपातनभ्, अन्यांशेऽनुवादः~॥
यद्यपि संनिवेशविशेषयुक्ताः पात्रविशेषाः स्रुचः , तथापि जुहूपभृतोरेव
ग्रहणम्~। {\qt  उद्ग्राभं चेति जुहूमुद्यच्छति निग्राभंचेति उपभृतं नियच्छति}
इति वचनात्~॥ 

 [ उद्दयो० ] शुद्धकाम्यत्वनिरासाय कामो य पुरुपार्थं इत्याह \textendash\ ~। [१
मप० ] धर्म इति~। विनिगमनाविरहाद्धर्मस्य तत्साधनत्वात \textendash\ त्त्वेऽपि
स्वतोऽतत्त्वाद्वक्ष्यमाणरीत्या चाह \textendash\ मोक्षश्चेति~॥ एतेन
{\qt अर्थकामसाधकत्वमपि संभवतीति तत्त्यागोsनुचितः} इत्यपास्तम्~। तयोः
स्वतोऽतत्त्वात, परमपुरुषार्थसाधकत्वसंभवे तत्कथनानौचित्या \textendash\ च्च~। 
१विशिष्टाध्ययनविधेश्च तथैव सत्त्वाच्च~॥ 

 [ उ० २ यप० ] किं व्येति~॥ तदध्ययनेन किएमित्यर्थः~॥ लोक इतीति
~। आत्मनेत्यादावित्यर्थः~। 



१ अङ्गविशिष्ट्वेदकर्मकाध्ययनविधेरित्यर्थः~। (र.ना.) २ पाठभ्रंशे \textendash\ 
त्यर्थो भाति (र.ना.) छायायां {\qt भ्रंशः इत्यस्यापभ्रंश} इत्यर्थं उच्यते, स
एवोचितः~। व्याकरणाभावे आचार्याणामपि तदाचार्यपूर्वकमेवज्ञानमिति तेषां
क्वचिदपभ्रंशस्यापि सम्भवेन तेष्वविश्वास इत्यर्थः स्यादिति
पाठ \textendash\ अंशेत्यादिकल्पनमनुचितम्~। ३ अत्राद्यशब्दः प्रभृत्यर्थकः~। १र.ना.)
वस्तुतस्तु ननु व्याकरणाभावेsपि वेदे पदक्रमजटादीनां लक्षणभिन्नानां
ख्यापकानां सत्वेन न भ्रंशसंभव इति चेत्, नः पदादिविषयेऽपि भ्रमशङ्कायां
व्याकरणमेव निर्णायकमिति पदादयस्तल्लक्ष्या एव~। तथा





भ्रमनिवृत्तिः, भ्रंशशङ्कया तेष्वविश्वासात्~॥ वेदार्थावगमद्वाराsपि
व्याकरणं पुरुषार्थसाधनमित्याह \textendash\ वेदार्थं चेति~। एवञ्च वेदार्थ \textendash\ 
ज्ञानवूर्वकं शुद्धतत्तन्मन्त्रैस्तद्विहितकर्मानुष्ठानेन स्वर्गसुखम् ,
उपनिष दर्थज्ञानेन वक्ष्यमाणरीत्या वा मोक्षश्च पुरुषार्थो
व्याकरणाध्ययनस्य फलमिति भावः~॥ विकार इति~॥ लोपस्तु न विकार इति
पृथग् \textendash\ गणितः~। भावरूप एव चादेशोsत्र विकारः~॥ हृग्रहोर्मं इति~। इदं च
गृभ्णामि इत्याद्यधर्मावश्यकम्~। एतेन {\qt उद्ग्राभनिग्राभौ चेत्यनेतैव
भस्यापि सिद्धत्वादेतदुपन्यासश्चिन्त्यः} इति परास्तम्~। अन्यार्धनावश्यकेन
तेनैव भस्य सिद्धौ तदंशे निपातनाश्रयणवैयर्थ्यात्~॥
निपूर्वाद्ग्रहेर्घञ्विधायकाभावादाह \textendash\ उद्ग्रभनिग्राभौ चेति~॥ 



 [ उद्दयो० ] ननु तस्य भ्रमे इष्टापत्तिरत आह \textendash\ [ उ० ३ यप० ]
भ्रमेणेति~। एवं च तस्य न व्याकरणसाम्यमिति भावः~॥ [४ र्थप० ]
भ्रंशेति~। २अपभ्रंशेत्यर्थः~॥ न च पदालक्षणं ख्यापकम्, तस्य
व्याकरणलक्ष्यत्वात्त्वलक्ष्याव्यत्यये निर्विवादत्यात~। न च
प्रातिशाख्यादितस्तन्निवृत्तिः४, तेषामनन्तासु शाखासु
क्चित्कचिदुत्साददर्शनात्~। न च तद्विरोधादस्यात्यन्तबाधः,
शाखान्तरपरतयाऽत्यन्तबाधकत्वस्याकल्पनात्~॥ एतेन \textendash\ 

 वेदरक्षाऽपि नैतस्मादृतेऽध्येतृपरम्पराम्~। 

 संप्रदायोऽनुवृत्तश्चेद्वेदस्तेनैव रक्ष्यते~॥ 

 प्रातिशाख्यविरोधे च बाध्यते व्याक्रिया वरम् 

 इति भद्टोक्तमपास्तम्~॥ न च मीमांसयाऽर्थावगमः,
जर्भरीत्यादावत्यन्ताप्रसिद्धे तस्या अव्यापारादिति दिक्~॥ 

 [ उद्दयो० ] तदेतद्धुनयन्नाह \textendash\ [ उ० ४ र्थप० ] वेदार्थेति \textendash\ ~। 
वेदार्थावगमपूर्वेकवेदरक्षणद्वारेत्यर्थः~॥ अपिरुक्तरीत्या
शुद्धरक्षाद्वारसमुन्नायकः~॥~। [ ५ मप०] व्याकेति~। तदध्य \textendash\ यनमित्यर्थः
~॥ {\qt पारम्पर्येण} इति कैयटतात्पर्यमाह \textendash\ एवं चेति~। अस्य वेदार्थनिश्चयकत्वे
वेत्यर्यः~॥ [ ६ ष्ठप० ] शुद्धेति~। अनेन वेदार्थावगमस्य५
रक्षाऽप्यवान्तरं प्रयोजनं सूचितम्~॥ अत एवाध्ययनमस्य न पाठमात्रं
किंतु वाक्यार्थावगतिपर्यन्तम्~। अन्यथा तदज्ञानाद्वेदरक्षाऽपि न स्यात्~। 
सति तु तस्मिन्नज्ञानार्था \textendash\ न्तरज्ञानसंशयानां निवृत्त्या सः~। अत एव
{\qt ज्ञेयश्च} इति तत्रोक्तम्~॥ उपनिषदर्थशानस्यान्यथाऽपि संभवात् {\qt यतोवाच}
इति श्रुत्या ब्रह्मणस्तथा शब्दाबोध्यत्वस्योक्तत्वाच्च \textendash\ [ ७ मप० ]
वक्ष्येति~॥ गौणप्रयोजनकथनावसरे इति भावः~॥ 



च पदादिपु व्याकरणेन अंशाभावः तैश्च वेदे भ्रंशाभाव इति अवैया \textendash\ करणः
पदादिषु भ्रान्तो वेदे भ्राम्येदिति स्वलक्ष्यस्य \textendash\ व्याकरणलक्ष्यस्य पदादेः
अव्यत्यये \textendash\ भ्रंशाभावे व्याकरणप्रयोजनरय निर्विवादत्वात् \textendash\ इति हि तदाशयः~। 
अत्र {\qt आद्यः} इति पदच्छेदः प्रकृतार्थासाधकः~। ४असाश्चुत्वभ्रमनिवृत्तिः~। 
( र. ना. ) भ्रंशशङ्काया निवृत्तिरित्युचि \textendash\ तम्~। ५ इदमवान्तरपदान्वयि~। 
(र. ना. \ अत व्याकरणाध्ययनस्य वेदरक्षणं मुख्यं प्रयोजनम्, तथा
वेदार्थज्ञानरक्षणमपि तस्यावान्तर \textendash\ प्रयोजनमिति छायातात्पर्यम्~। 

२२ उद्द्योतपरिवृतप्रदीपप्रकाशितमहाभाष्यम्~। [ १ अ. १ पा. १
पस्पशाह्निके 



 ( ऊहपदार्थनिरूपणभाष्यम् ) 

 १ऊहः खल्वपि~। न सर्वैर्लिङ्गैर्न च सर्वाभिर्वि \textendash\ भक्तिभिर्वेदे मन्त्रा
निगदिताः~। ते चावश्यं यज्ञगतेन२ यथाऽर्थ विपरिणमयितव्याः~। तान्नावैयाकरणः
शक्नोति यथाऽर्थं विपरिणमयितुम्~। तस्मादध्येयं व्याकरणम्~॥ 

 ( प्रदीपः ) ऊहः खल्वपीति~। इह यस्मिन्यागे इतिकर्त \textendash\ व्यतोपदिष्टा
यागान्तरेणोपजीयव्ते सा प्रकृतिः~। येन चोपजीव्यते सा विकृतिः~। 
{\qt प्रकृतिवद्विकृतिः कर्तव्या} इति मीमांसकैर्व्यवस्थापिते न्याये
प्रकृतिप्रत्ययादीनामूहं वैयाकरणः सम्यग्विजानाति~। 
तत्राग्नेर्मन्त्रोऽस्ति \textendash\ {\qt अग्नये त्वा जुष्टं निर्वपामि} इति इति, तत्र
{\qt सौर्यं चरुं निर्वपेद् ब्रह्मवर्चसकामः} ३इति 



 [उद्द्योते] ननु लोपस्यापि वर्णविकारत्वात्पृथगुक्तिरयुक्ता अत
आह \textendash\ [ उ० ८ मप० ] लोपस्त्विति~। अत्र तत्त्वेनाभिमत इत्यर्थः~॥
तदेवाह \textendash\ [ ९ मप० ] भावेति~। चो हेतौ~॥ ए्वं चान्यत्र तथा
सत्त्वेऽपि न क्षतिरिति बोध्यम्~॥ 

 [ उ० १० मप० ] आदिना {\qt जभार} इत्यस्य परिग्रहः~॥ [ ११ शप० ]
अपिना घञ्समुच्चयः~॥ [ १ २ शप० ] नैव भस्य सिद्धाविति पाठः~॥
वैयर्ध्यादिति~। अन्यथा {\qt सख्यशिश्वी} इत्यनेन ङीष इवेकारलोपस्यापि
निपातनादेव सिद्धौ {\qt यस्येति च} इत्यत्रस्थस्य {\qt इकारस्येति परे लोपः}
इत्यंशस्य {\qt अतिसखेरागच्छति अतिसखेः स्वम् इत्युदाहरणम्} इति
भाष्यस्यासंगतिः स्यादिति {\qt भावः}~॥ तत्र {\qt उदि ग्रहः}
इत्यनेनोत्पूर्वाद् ग्रहेर्लोकेऽपि घञः सिद्धत्वादाह \textendash\ निपूर्वादिति~॥
एवं च निग्राभमित्यस्यासिद्धिरिति भावः~॥ 

 [भाष्ये] लिङ्गैः \textendash\ सर्वलिङ्गबोधकपदैः~॥
विभक्तिभिः \textendash\ सर्वविभक्यन्तपदैः~। {\qt युक्ताः} इति शेषः~। नञो निगदिता
इत्यत्रान्वयः~॥
ब्रीह्यादिद्रव्यसंबन्धिनोऽवघातादेर्नीवारादिसंबन्धलक्षणसंस्कारोहे यद्यपि
व्याकरणाध्ययनस्य नोपयोगः, तथापि मन्त्रोहे उपयोग इति सूचयन्नाह \textendash\ मन्त्रा
इति~॥ प्राकृतकरणस्वरूपप्रयुक्तत्वे मन्त्राणां विकृतौ
तत्स्वरूपाभावात्तदप्राप्त्या नोहः फलम्, किंतु अपूर्वप्रयुक्तत्वे
मन्त्राणामिति सिद्धान्ताद्विकृतावपि तस्य सत्त्वेन मन्त्रप्राप्त्या तदूहः
फलमिति सूचनायाह \textendash\ ते चावश्यमिति~। यथाकथमिति पाठे
प्रकृतिमुख्यार्थामुख्यार्थादिभेदेनोहः~। न तु यथाकथंचिदित्यर्थः~॥
विपरिणेति~॥ तत्तलिङ्गादिबोधनक्षमपदयुक्ततया पाठ्या इत्यर्थः~॥ तेषां
तथा पठनमेवोहपदार्थः~॥ ननूह्यमानपदानां वेदादेवोद्धृत्य प्रयोगो भविष्यति
किं व्याकरणेनात आह \textendash\ तान्ना \textendash\ 



१ ऊह इति~। अत्र पूर्ववत्प्रयोजनमित्यस्यानुषङ्गः~। २ {\qt तेन पुरुषेण यथायथं
वि \textendash\ } इति मुद्रितपाठः~। ३ {\qt इति सौर्ये चरौ} इति मुद्रितपाठः~। ४ {\qt भाष्ये}
ऊहः खल्वपीत्यस्य अयोजनमिति शेषः~। खल्वपीति निपातसमुदायो निश्चयार्थः~। 
इशति मुद्रितपाठः~। 

५. द्विदेवत्येति~। आग्नेयसौर्ययागयोः प्रकृतिविकृतिभावावगमश्च
एकदेवत्यत्व \textendash\ ओषधद्रव्यत्वसाम्येन च बोध्यः~। दर्शपूर्णमासयोः प्रया \textendash\ 





सौर्यचरौ मन्त्र ऊह्यते \textendash\ सूर्याय त्वा जुष्टं निर्वपामि \textendash\ इति विस्तरेण
भर्तृहरिणा प्रदर्शित ऊहः~॥

 (उद्द्योतः ) भाष्ये \textendash\ {\qt खल्वपि} इति निपातसमुदायो निश्चयार्थः~। {\qt ऊहः
खल्वपि} इत्यस्य प्रयोजनमिति शेषः~। ननु यत्र प्रकरणे ये मन्त्राः
पठितास्तत्र तेषां तथैव प्रयोग इष्ट इति नोहेन प्रयोजनमित्याशङ्क्य
प्रकृतावूहाभावेऽपि विकृतावूह इति दर्शयितुं प्रकृतिशब्दार्थमाह \textendash\ इहेति~। 
यस्मिन् \textendash\ आग्नेयादौ~। यागान्तरेण \textendash\ सौर्यादिना~। प्रकृतिविकृत्यवगमश्च
अद्विदेवत्यत्वैकदेवत्यत्वादिसाम्येन बोध्यः~॥ अन्यत्र पठितमन्त्राणां
कथमन्यत्र गमनमत आह \textendash\ प्रकृतिवदिति~। यथा \textendash\ सोपकारा प्रकृतिरनुष्ठीयते, तथा
विकृतिरपीति तदर्थः~। तदनेनोपकारातिदेशे तत्पृष्ठभावेन पदार्था६
अप्यतिदिश्यन्ते, इतिकर्तव्यतायां भावनायाः साकाङ्क्षत्वात्~। एतन्मूलकमेव
{\qt प्रकृतिवत्} 



वैयेति~। यथा प्रत्यक्षश्रुतिमूलमपि स्मृतिप्रणयनं धर्मावगमे प्रमाणम्,
एवं व्याकरणमप्यूहे इति भावः~॥ उपसंहरति \textendash\ तस्मादिति~॥ यस्मात्तेन७
तद्दुष्करं तस्मात्तदर्थमेतदध्येयमित्यर्थः~॥ 

 [ प्रदीपे] भाष्ये लिङ्गविभक्तिग्रहणं
प्रकृत्यादेरुपलक्षणमित्यभिप्रेत्याह \textendash\ कैयटे [४र्तप०] प्रकृतीति
~॥ आदिना लिङ्गपरिग्रहः~॥ सम्यक् \textendash\ यथोचितम्~॥ 

 [प्र ५ मप०] कैयटे \textendash\ तन्त्रविकृतौ~॥ तत्र \textendash\ [६ ष्ठप० ] इति सौर्ये
इति~। इति विहितसूर्यदेवताकचरुद्रव्यकयाग इत्यर्थः~॥ 

 [ उ० १ मप० ] क्रमप्राप्तमूहं कथयतीत्याह \textendash\ उद्दयोते \textendash\ भाष्ये ऊहः
खल्वपीत्यस्येति~॥ ननु रक्षार्थमित्याद्यसंबद्धपदव्यवायेन
विच्छिन्नत्वादनुषङ्गो न~। तदुक्तम् \textendash\ {\qt अनन्तरेण संबन्धः स्यात्
परस्याप्यनन्तरे} इति, अत आह \textendash\ शेष इति~॥ यद्यपि खलुशब्द एव निश्चयार्थः,
अपिरन्यसमुच्चायक इति सुवचम्, तथापि परिगणनादेव समुच्चयः सिद्ध
इत्याशयेनाह \textendash\ खल्वपीति निपातेति~॥ [३ यप० ] ये मन्त्रा इति~। 
यादृशा इति यावत्~॥ [ उ० ४ र्थप० ] प्रकृतीति~। इदं च लक्षणं
संभवाभिप्रायं बोध्यम्~। वस्तुतस्तु यत्राङ्गजातं पूर्णमुपदिष्टं सा
प्रकृतिः, यत्र नोपदिष्टं सा विकृतिरिति सार्वत्रिकं लक्षणं बोध्यम्~॥
नन्वेवमपि कथं प्रकृतिविकृतिभावावगमोsत आह \textendash\ [ ६ ष्प० ] प्रकृतीति~। 
तद्भावावगमश्चेत्यर्थः~॥ आदिनो ओषधिद्रव्यकत्वादिपरिग्रहः~॥
प्रकृतितुल्या या विकृतिः सा कर्तव्येत्यर्थे \textendash\ विकृतौ प्रकृतितुल्यत्वस्य
वाधितत्वात्तत्कथनानुपयोगादतिदेशवैयर्थ्यापाताच्च, तद्वदियं कार्ये \textendash\ 
त्यर्थे \textendash\ तत्तत्यत्वस्यान्यथा दुर्वचत्वान्न्यायार्थमाह \textendash\ [ ८ मप० ]
यथेति~॥ प्रकृतावङ्गानि येनोपकारेण तामुपकुर्वन्ति यथा, विकृतावपि तेन
तेनोपकारेणैव तानि तामुपकुर्वन्तीत्यर्थः~॥ वैदिकस्य 



जादीनाञ्च द्विदेवत्यत्व \textendash\ पशुद्गव्यत्वसाम्येन च बोध्यः~। ६ ननु {\qt प्रकृति \textendash\ 
वद्विकृतिः \textendash\ } इत्यतिदेशेनोपकारातिदेशेऽपि पदार्थानामनतिदेशान्न विकृतौ
पदार्थान्वयः स्यादित्याशङ्कायामाह \textendash\ पदार्था अपीति~। उपकारान्वयमात्रेण
विकृतेः कथंभावाकांक्षाया अनिवर्तनात्ततपृष्ठभावेन तेऽप्यतिदिश्यन्त
इत्यर्थः~। ७ अवैयाकरणेनेत्यर्थः~। 

शास्त्रप्रयोजनाधिकरणम् ] महाभाष्यप्रदीपोद्द्योतव्याख्या छाया~। 



इति \textendash\ इति मीमांसकसरणिः~॥ पदार्थान्तर्गताश्च मन्त्राः~। तत्र
अग्निसंबन्धिनिर्वापप्रकाशमन्त्रस्थाग्निपदस्य
सूर्यसंबन्धिनिर्वापप्रकाशनासमर्थ्वात्तदपहाय तत्स्थाने
{\qt सूर्याय} इत्यूह्यमित्याह \textendash\ तत्राग्ने \textendash\ रिति~॥ मन्त्र ऊह्यत इति~। 
यद्यप्यूहे न मन्त्रत्वम्, तथाप्येकदेशस्योहेऽपि अनेकपदसमुदाये
मन्त्रत्वप्रत्यभिज्ञानात्तद्धटिते समुदाये मन्त्रत्वव्यवहारः, कर्मणः
साङ्गत्वं चेति बोध्यम्~। सोऽयं प्रकृत्यूहः~। अन्ये \textendash\ ऽप्यूह्याः~। ऊहज्ञस्य
हि आर्त्विज्यलाभेन द्रव्यप्राप्तिद्वारा ऐहिकसुख \textendash\ सिद्धिः फलमिति बोध्यम्~। 
भाष्ये \textendash\ लिङ्गपदं च प्रकृत्यादेरुपलक्षणम्~॥ यथायथमिति१~। 
अर्थप्रकाशनसामर्थ्यानतिलङ्घनेनेत्यर्थः~॥ 



विधेरुपकारापेक्षायां करणस्य वैदिकत्वात्तथाविधानामेवाङ्गानामपे \textendash\ 
क्षितत्वात्तेषामेव प्रकृतौ
क्लृप्तसामर्थ्यत्वात्तान्येवौपदेशिकसंबन्धाभावे \textendash\ प्यतिदेशेन प्राप्यन्ते~। 
तदाह \textendash\ [ ९ मप० ] तदनेनोपेति~॥ अत एवोपकारस्यासंभवे
सामान्येनातिदेशेऽपि केषाञ्चिन्निवृति साकल्येनैकदेशेन वा~॥ [ ९
मप० ] इति केति~॥ निरितिकर्तव्यताकस्य कर्मणोsभावात् सर्वत्राख्याते
भावनाया विधेयत्वात्तस्याश्चाशत्रययोगित्वात् {\qt किंकेन} \textendash\ कथम् इति
साध्यसाधनेतिकर्तव्यताकाङ्क्षानियतत्वादितिकर्तव्यतां विना
साध्यसाधनयोरन्वयायोगातदावश्यकत्वमिति भावः~॥ नन्वस्तूक्तरीत्या
पदार्थातिदेशः, मन्त्रातिदेशस्तु दुर्घटोऽत आह \textendash\ [११ शप०]
पदार्थान्तर्गतेति~॥ 

 [ उद्दयोते ] एकदेशस्यानुपवारान्नानुवृत्तिरित्यत्र कैयटेन सम्यक्
सत्वोपपादनाय दत्तमुदाहरणमुपपादयंस्तमवतारयति \textendash\ [ ११ शप० ] तत्रेति~॥
प्रकृतावित्यर्थः~॥ तेन मन्त्रेण प्रकृतिसमवेताग्नि \textendash\ 
संबन्धिनिर्वापप्रकाशनरूपः प्रकृत्युपकारः कृत इति विकृतेरपि
तत्समवेतदेवतासंबन्धिनिर्वापप्रकाशनरूपः स कार्यः~। स चाग्नि \textendash\ शब्दयुतेन
तेन कर्तुमशक्यः~। तदाह \textendash\ [१२ शप० ] मन्त्रस्थाग्नीति~॥ नन्वेवमपि
संप्रदानत्वस्याबाधादेकारानिवृत्तौ सूर्ये इति स्यादिति चेत, न,
अकारान्तेतरप्रकृतिसंनिधानेनैवैकारेण संप्रदानत्वस्य
वाच्यत्वेनाकारान्तप्रकृतिसंनिधौ यशव्देन वाच्यत्वेनाग्निशब्दनिवृत्तौ
तस्यापि निवृत्तेः~। तथा च तद्घटितदृश्यमानं रूपमेव समुदितं समुदितस्थाने
ऊह्यते~। तदाह \textendash\ [१३ शप०] तदपेति~॥ {\qt अग्रये} इति पदमपहारयेत्यर्थः
~॥ तत्राग्नेरितीति~। प्रकृतावग्नि \textendash\ संबन्धिनिर्वापप्रकाशकमन्त्र
इत्यर्थः~॥ 

 [उ० १४ शप] न मन्त्रत्वमिति~। {\qt अनाम्नाता अमन्त्राः}
इत्याद्यापस्तम्बाद्युक्तेरिति भावः~॥ एकदेशविकृतन्यायेन लौकिकेन
मल्लग्रामादिवद् भूयसा व्यपदेश इति न्यायेन च क्रमेणाह \textendash\ [१५ शप० ]
अनेकपदेति~। अवशिष्टेत्यादिः~॥ तद्घटिते ऊहितपदघटिते~॥ [ १६
शप० ] प्रकृत्यूह इति~। एकाररूपप्रत्ययोहस्तु नान्तरीयकः,
तदर्थस्यावाधात्~॥ एतेन {\qt सोsयमुभयोहः} इति कृष्णोक्तमपास्तम्~॥
प्रकृतिलिङ्गोहस्तु \textendash\ देवीरापः शुद्धाः 



१ {\qt यथार्थमिति} थ \textendash\ पाठः~। 

२ आगम इति~। भत्रापि पूर्ववत् {\qt खल्वपि} इति निपानसमुदायो 

निश्चयार्थ~। आगमः खल्वपि प्रयोजनम् \textendash\ इति तदर्थः~। अत्र प्रयोजनशब्दः
प्रयोजकपरः~। प्रयोजकत्वमुपपादयति \textendash\ ब्राह्मणेनेत्यादिना~। 





 (आगमपदार्थनिरूपणभाष्यम् ) 

 २आगमः खल्वपि~। {\qt ब्राह्मणेन निष्कारणो धर्मःषडङ्गो वेदोsध्येयो ज्ञेयः} इति~॥ प्रधानं च षट् \textendash\ खङ्गेषु व्याकरणम्~। ३प्रधाने कृतो यत्नः
फलवान्भवति~॥ 

 ( प्रदीपः ) आगम इति~। आगमः४ प्रयोजनः प्रवर्तको नित्यकर्मतां
व्याकरणाध्ययनस्य दर्शयति~। {\qt ५प्रयोजन}शब्देन च फलं \textendash\ प्रयोजकश्चोच्यते~॥
निष्कारण इति~। दृष्टं कारणमन \textendash\ 



स्थं इत्यप्सु विनियुक्तमन्त्रस्याज्ये {\qt देवाज्य शुद्धमसि} इति~॥
वचनोहोऽप्ययम् \textendash\ आप इत्येवंपरसंख्यारोपकृतवहुवचनसत्वात्~॥
वचनमात्रोहस्तु \textendash\ {\qt मा भैर्मा संविक्था} इति
पुरोडाशेऽवदान \textendash\ मन्त्रस्य धानास्ववदाने {\qt मा भेष्ट मा संविजिध्वम्}
इत्यादि इति मिश्रादयः~॥ याज्ञिकास्तु {\qt संविग्ध्वम्} इत्यूहमाहुः~॥ तत्र
वेदभाष्ये हे पुरोडाश सा भैः \textendash\ भयं मा कार्षीः, मा च संविक्थाः \textendash\ कम्पिष्ठा
इत्यर्थ इति ओविजीत्यस्य रूपम् ,
इडभावश्छान्दसत्वादागमशाstraनित्यत्वाद्वा~। विजिरस्तु न, तस्य
कम्पनार्थत्वाय धातूनामनेकार्धत्वाश्रयणापत्तिरित्याशयेनोक्तमिति
तद्विरोधापत्त्या याज्ञिकोक्तं न युक्तमिति तदेव युक्तमिति हरदत्ताशयः~॥
याज्ञिकाशयस्तु \textendash\ तदपेक्षया छान्दसत्वादिकल्पना गरीयसीति प्रकृतावनिटूकतया
विजिर एवायं प्रयोगो न तु तस्येति~॥ वस्तुतस्तु ओविजी इत्यस्य रूपे मा भैः
इत्यस्य वैयर्थ्यम् , तस्यैव तंत्रादिना विशेष \textendash\ तात्पर्यग्राहकाभावेन न
त्वदिष्टार्थद्वयसंभवात्~। तस्माद्विजिर एव रूपम्~। पृथग्भावं मा मन्यस्व
ति त्वर्थ इति बोध्यम्~॥ तदेतद \textendash\ भिप्रेत्याह \textendash\ [१६ शप०] अन्येsप्यूह्या
इति~॥ संभवादाह \textendash\ ऊहज्ञेति~॥ हिश्चार्थे सिद्धिपदोत्तरं योज्यः~॥ तेन
प्रागुक्तरीत्या स्वर्गसुखादिसमुच्चयः~। अनेनैवंरीत्या
पूर्वत्राप्यैहिकसुखसिद्धेः फलत्वं सूचितमिति बोध्यम्~॥
तावताप्युक्तरीत्या तत्र साक्षात्त्वस्य बाधि \textendash\ तत्वात्कैयटोक्तं
पारम्पर्यमविरुद्धम्~॥ न्यूनतां परिहरति \textendash\ भाष्ये लिङ्गेति~॥ चेन
विभक्तिसमुच्चयः~॥ 

 [भाष्ये ] क्रमप्रप्तमागममाह \textendash\ आगमः खल्वपीति~। शेष \textendash\ पूरणादि
प्राग्वत्~॥ तमेवागमं दर्श्यति \textendash\ ब्राह्मणेनेति~॥ [भाष्ये ]
अवश्याध्ययनायैवात्याङ्गतोsस्य वैलक्षण्यमाह \textendash\ भाष्ये \textendash\ प्रधानं चेति~। 

 [भाष्ये ] एतदेवाह \textendash\ भाष्ये \textendash\ प्रधाने चेति~॥ क्वचिन्निश्चपाठः~॥ अत
एवोद्योते तथा पाठो धॄतः~॥ 

 [ प्रदीपे ] यत्तु यद्यप्यागम्यते \textendash\ प्रमीयते
हिताहितत्प्राप्तिपरिहारावनेनेत्यागमः \textendash\ अनादिरुपदेशः
प्रत्यक्षानुमानिकश्रुतिरूपः, तथाप्यत्रागमशब्देन तत्फलं लक्ष्यते~। तच्च
नियोज्यस्य हिताहितप्राप्तिपरिह्वा



३ {\qt प्रधाने च} इति मुद्रितपाठः~। 

४ {\qt प्रयोजकः} इति अ. पाठः~। 

५ प्रयोजनशब्देन \textendash\ रक्षोहागमलघ्वित्यत्रयप्रयोजनशब्देन~। 

२४ उद्द्योतपरिवृतप्रदीपप्रकाशितमहाभाष्ये~। [१ अ. १ पा. १ पस्पशाह्निके




पेक्ष्येत्यर्थः~॥ प्रधानं चेति~। पदपदार्थावगमस्य व्याकरण \textendash\ 
निमित्तत्वात्तन्मूलत्वाद्धाक्यवाक्यार्थावसायस्येति भावः~॥ 

 ( उद्द्योतः ) ननु वक्ष्यमाणागमस्य न व्याकरणफलत्वमत आह \textendash\ प्रवर्तक इति
~॥ नित्यकर्मतामिति~। एतेन नित्यकाम्यता व्याकरणाध्ययनस्येत्युक्तम्~। 
नित्यत्वं १चाध्येयज्ञेयपदार्थविशेषणीभूता \textendash\ 



ररूपम् , उपदेशस्य नियोज्यप्रयोजनैकसाध्यत्वात् \textendash\ इति केनचिदुक्तम्~। तन्न
आगमशब्दस्य लाक्षणिकत्वे निर्मूलमुख्यार्थत्यागापत्तेः~। 
उत्तरग्रन्थासंबद्धत्वापत्तिः~। ४अस्य प्रत्यक्षवेदेष्वनुपलम्भेन
प्रत्यक्षश्रु \textendash\ त्यरूपत्वात्~। अनुमापकाभावादनुमितत्वस्याप्यभावाच्च~। न च
व्याकरणाध्ययनेनैवास्यानुमानम् , अस्योन्याश्रयापत्तेः, एतदध्ययनेन
तदनुमानम्~। तदनुमितेन चैतदध्ययनमितीति दिक्~। तदेतद् ध्वनयन्नाह
कैयटे \textendash\ आगमः प्रयोजन इति~॥ 

 [ प्रदीपे ] ननु रक्षादिरूपफलसत्त्वेन कथं तदभावोऽत आह \textendash\ कैयटे
[ ३ यप० ] मनपेक्ष्येत्यर्थः इति~॥ तथा च न तदभावः प्रतिपाद्यते,
किंतु प्रवृत्तौ तदनपेक्षा, नित्यत्वादेव प्रवृत्तिसिद्धेः~। अत एव
द्वैरूप्यं प्रागुक्तम्~॥ 

 [ प्रदी० ५ मप० ] निमित्तत्वात् \textendash\ निमित्तकत्वात्~॥ तन्मू \textendash\ 
लत्वात् \textendash\ पदपदार्थधीनिमित्तकत्वात्~॥ वाक्यार्थेति~। वेदार्थावग \textendash\ 
मस्येत्यर्थः~॥ 

 [ उ० १ मप ] व्याकरणफलत्वमिति \textendash\ षष्ठीतत्पुरुषः~। 
कैयटस्थप्रयोजनशब्दोsपि फलपर इति भावः~॥ ननु नित्यकर्मतामि \textendash\ त्यस्य
नित्यतामित्यर्थ कर्मपदानर्थक्यम् , किंच रक्षादिप्रयोजनसत्त्वे कथं
निष्प्रयोजनत्वेन नित्यत्वम्, नित्यत्वे वा कथं रक्षादिप्रयो \textendash\ जनोक्तिः~। न
चैकस्य द्वैरूप्यम्, विरोधादत आह \textendash\ [ २ यप० ] एतेनेति
कर्मपदयुक्तकथनेनेत्यर्थः~। फलाभिसन्धिना यत् क्रियते तत् \textendash\ कर्मेति
व्युत्पत्त्या च कर्मपदं काम्यपरम्~॥ {\qt एकस्य तूभयत्वे संयोगपृथक्त्वम्}
इति (पू. मी. ४~। ३~। ४ ) न्यायेन प्रमा \textendash\ णद्वयसंबन्धादेकस्यापि द्वैरूप्यम्
इत्यकेन {\qt खादिरो यूपो भवति खादिरं वीर्यकामस्य} इति
विहितखादिरोदाहरणकेनात्र द्वैरूप्ये न विरोधः~॥ अत एव
फलचतुष्ट्यबहिर्भावेतैवादावन्ते वाऽऽगमे 



१ अध्येयज्ञेयेति~। ब्राह्मणेन निष्कारणेनेत्यागमगतेत्यादिः~। 

२ काम्यमेव नेति सूचयति इति {\qt कश्चित}शब्दरहितो ज. पुस्तकपाठः~। 

३ {\qt शब्द इति चेन्नातः प्रभवात्प्रत्यक्षानुमानाभ्याम्} इति ब्रह्मसूत्र
( १~। ३~। २८ ) भाष्ये \textendash\ 

 अनादिनिधना नित्या वागुत्सृष्टा स्वयंभुवा~। 

आदौ वेदमयी दिव्या यतः सर्वाः प्रवृत्तयः~॥ 

नाम रूपं च भूतानां कर्मणां च प्रवर्तनम्~। 

वेदशब्देभ्य एवादौ निर्ममे स महेश्वरः~॥ 

सर्वेषां तु स नामानि कर्माणि च पृथक् पृथक्~। 

वेदशब्देभ्य एवादौ पृथक्संस्थाश्च निर्ममे~॥ 

इत्यादिस्मृतीतां श्रुतिमूलिकानां तथा \textendash\ 

{\qt समाननामरूपत्वाच्चावृत्तावप्यविरोधो} दर्शनात्स्मृतेश्च





ध्ययनज्ञानयोर्बोध्यम्~। अत्र ब्राह्मणेनेत्युक्तेरन्यस्यैवमध्ययनं
काम्य.२मेवेति सूचयतीति कश्चित्~॥ निष्कारण इति~। कारणशब्द फलपरः~॥ ननु
नित्यत्वेऽपि प्रत्यवायपरिहाररूपं फलमस्त्येवेत्यत आह \textendash\ दृष्टमिति~। 
अध्ययनादिनिष्ठं कारणानपेक्षत्वं विषये आरोप्य {\qt निष्कारणो वेदः इति प्रयोगः
~।} आगमपदेन श्रुतिः~। श्रुता३



वाच्ये मध्ये तदुक्त्या फलसंदंशपातेन नित्यस्यापि फलमस्तीति सूचितमिति
भावः~॥ ननु निष्कारणशब्देन नित्यत्वं साङ्गवेदस्य प्रतिपाद्यम्~। 
तदयुक्तम्, आनुपूर्व्या अनित्यत्वात्, तस्य काम्यत्वा \textendash\ संभवाच्च~॥
अङ्गानां तत्त्वस्य दूरापास्तत्वाच्च, तावतापि प्रकृते \textendash\ ष्वसिद्धेश्च~। अत
आह \textendash\ [ उ० ३ यप० ] नित्यत्वं चेति~॥ नयोरिति~। 
धर्मत्वेनाभिमतयोरिति भावः~॥ [४ र्थप० ] अत्रेति~। आगम इत्यर्थः
~॥ रन्यस्यैवमध्ययनमिति~॥ क्षत्रियस्य वैश्यस्य च साङ्गं वेदाध्ययनं
तथा ज्ञानं चेत्यर्थः~॥ [ ५ मप० ] कश्चिदिति~। रत्नकृदित्यर्थः~॥
अत्रारुचिबीजं तयोर्नित्याध्ययनविधायकस्मृत्य \textendash\ न्तरादिविरोधापत्तिरिति~॥
तस्माद्ब्राह्मणपदं त्रैवर्णिकोपलक्षणमिति बोध्यम्~॥ वस्तुतस्तु कलौ
क्षत्रियवैश्ययोरभावं सूचयितुं तथोक्तमिति यथाश्रुतमेव तत् साधु~। तथा च \textendash\ 


 कलौ न क्षत्रियाः सन्ति कलौ नो वैश्यजातयः~। 

 ब्राह्मणाश्रैव शूद्राश्च कलौ वर्णद्वयं स्मृतम्~॥ 

 इति स्मृतिरिति तत्त्वम्~॥ निष्कारण इसि \textendash\ इति प्रतीकान्तर्गतप्रतीकम्~॥
यद्वा \textendash\ {\qt निष्कारण इति} \textendash\ इत्यत्र भाष्ये~॥ ननु निष्कारणशब्दस्य
निर्हेतुपरतया लब्धस्यानादिस्वरूपनित्यत्वस्य वेदे कथंचित्संभवेऽप्यङ्गानां
पौरुषेयत्वेन तत्त्वाभावः~॥ किंचाध्ययनज्ञानयोस्तत्त्वाभावापत्तिरत
आह \textendash\ [ ५ मप ] कारणेति~। कार्यतेस्नुशष्ठायतेऽनेनेति व्युत्पत्तेः
~। तथा च ततो निर्गतो निष्कारणो नित्य इति यावदिति तदर्थो बोध्यः~॥ 

 [उद्दयोते ] तद्ध्वनयन्नुक्तरीत्या यथाश्रुते तस्य
तदनिष्टत्वादसंगतिपरिहारायाह \textendash\ [ उ० ७ मप० ] अध्ययनेति~। आदिना
ज्ञानपरिग्रहः~। ब्राह्मणेनेत्यस्य तादृशो वेदो५ध्येयो
क्ज्ञेयश्चेत्यनेनान्वयः~॥ तथा च
संध्योषासनादिवत्साङ्गवेदाध्ययनतदर्थज्ञानरू \textendash\ 



इति ब्रह्मसूत्र ( १~। ३~। ३० ) भाष्ये \textendash\ 

धाता {\qt यथापूर्वमकल्पयत्} इति श्रुतेस्तदर्थभूतानां 

तेषां ये यानि कर्माणि प्राक् सृष्ट्यां प्रतिपेदिरे~। 

तान्येव ते प्रपद्यन्ते सृज्यमानाः पुनः पुनः~॥ 

ऋषीणां नामधेयानि याश्च वेदेषु दृष्टयः~। 

शर्वर्यन्ते प्रसूतानां तान्येवैभ्यो ददात्यजः~॥ 

यथर्तुष्वृतुलिङ्गानि नानारूपाणि पर्यये~। 

दृश्यन्ते तानि तान्येव तथा भावा युगादिषु~॥ 

यथाभिमानिनोऽतीतास्तुल्यास्ते सांप्रतैरिह~। 

देवा देवैरतीतैर्हि रूपैर्नामभिरेव च~॥ 

इत्यादिस्मृतीनां च शंकरभगवत्पादैरुपन्यस्तत्वाद्वेदे कुशिकजमद \textendash\ 
ग्न्यादिनामसत्त्वेऽपि नार्वाचीनत्वशङ्केति दिक्~॥ इति दाधिमथाः
~॥ब्राह्मणेन निष्कारणो धर्मं इत्यागमस्येत्यर्थः~। (र. ना. ) 

शास्त्रप्रयोजनाधिकरणस् ] महाभाष्यप्रदीपोद्द्योतव्याख्या छाया २५



वपि १जनमेजयाद्युपाख्यातवत्पौरुषेयव्याकरणस्यापि नित्यताबोधनमुपपद्यते~। 
यद्वा {\qt तेन प्रोक्तम्} इति सूत्रोक्तभाष्यरीत्या वेदार्थ
वद्व्याकरणार्थोऽपि {\qt अनादित्वरूपनित्यतावान्} इति श्रुत्या तस्य
नित्यत्वबोधनमिति बोध्यम्~। धर्मत्वं च वेदस्य पुरुषयत्नसाध्यतया
धर्मत्वेनाभिमताध्ययनज्ञानकर्मत्वेनौपचारिकमित्याहुः~॥ षडङ्गः \textendash\ 
शिक्षा \textendash\ कल्प \textendash\ व्याकरण \textendash\ निरुक्तछन्दो \textendash\ ज्योतिषाङ्गसहितः, वेदः
स्वशाखारूपः,अध्येयः \textendash\ प्रेरणाविषयाध्ययनकर्म वेद इति बोधः~॥ यद्यपि
{\qt तेन प्रोक्तम्} इति सूत्रे भाष्ये वक्ष्यमाणया रीत्या वेदत्वं
४शब्दतदर्थोभयवृत्ति, तथापि {\qt ज्ञेयः इत्यस्य} सम्यगर्थबोधपर्य वसाय्य



पोऽयं धर्मो नित्य इति फलितम्~॥ यथा नित्यस्याकरणे प्रत्यवाय \textendash\ 
स्तत्करणे च तदभावः, तथा न्याकरणाध्ययनस्याननुष्ठानेऽनुष्ठाने च बोध्यम्~॥
यत्तु \textendash\ {\qt ब्राह्मणेन \textendash\ } इत्यादिः श्रुतिर्न, पठ्यमानवेदेषु अनुपलम्भात्
~। तस्मात् स्मृतिरेवेयम् \textendash\ इति भट्टैरुक्तम्~। तन्न,
भाष्योक्तागमपदास्वारस्यापत्तेः, तस्य वेदपरत्वेनोपपत्तौ उप \textendash\ 
चरितस्मृतिपरत्वानौचित्याच्च, पठ्यमानस्मृतिष्वनुपलम्भस्य तुल्य \textendash\ त्वाच्च~। 
भगवतः सिद्धानुवादकत्वेन भाष्यीयत्वाद्यभावाच्च~। तदेतद् ध्वनयन्नाह \textendash\ [ ८
मप० ] आगमेति~॥ नन्वेवं पौरुषेयव्याकरणस्य कथमागमबोध्यत्वम्~। न
चैवं स्मृतिरेवेयमास्ताम्, पौरुषेयतयाsस्या आगमत्वोक्त्यसांगत्यापत्तेः~। न
च वेदमूलकत्वे तस्याः स्मृतेरप्यागमत्वम् , निल्यानित्यसंयोगविरोधात्~। 
स्मृतेर्वेद \textendash\ मूलकतया स्मृतिविधेयस्य वेदविधेयत्वं स्वीकार्यम्, न चैतत्
संभवतिव्याकरणस्यानित्यस्याध्ययनं नित्येन वेदेन विधीयते इति, अत आह \textendash\ 
श्रुतावपीति~। तथा च तद्वत्प्रवाहनित्यतया नित्यविधिविषयत्वोपपत्तिः~। 
न च प्रत्यक्षश्रुत्यमूलकत्वान्न ते प्रबाहनित्यत्वकल्पनं युक्तम्,
{\qt तस्मादियं व्याकृता वागुच्यते} इति प्रत्यक्षश्रुतिमूलकत्वसत्त्वात्~। 
पूर्वकाले {\qt अग्निमीळे} इत्यादिका वागेकात्मिका विभागशून्याऽऽतीत्~। 
ततो विभागार्थ देवैः प्रार्थित इन्द्रो देवान्प्रति इन्द्रेण याचित
एकपात्रे वायोः स्वस्य च सोमरसग्रहणरूपे वरे देवैस्तथा करणेन संतुष्टो
विभागं कुर्यादित्यर्थकमिदं श्रुतावुक्तम्~॥ न च
संप्रदायव्याकृतनित्यवेदाभिप्रायेयम्, वागुच्यत इत्येव सिद्धे व्याकृता
इत्यस्यानर्थक्यापत्तेः~। तस्मादियं वागिदानीमपि
पाणि \textendash\ न्यादिभिर्व्याकृतोच्यत इत्यर्थकमिदं श्रुतावुक्तम्~॥ 

 [उ० १० मप०] सिद्धान्तमतेनाह \textendash\ यद्वेति~॥ ननु {\qt चोदना} \textendash\ लक्षणोऽर्थो
धर्मः इति धर्मलक्षणानाक्रान्तत्वात् कथं वेदस्य धर्म \textendash\ त्वमत आह \textendash\ [१ २
शप०] धर्मत्वं चेति~॥ यत्तु \textendash\ वेदोपकारि \textendash\ 
त्वाच्छ्रुतिलिङ्गादिषट्कमेवाङ्गम्, शब्दात्मकत्वेन तेषां
श्रुत्यवयवत्व \textendash\ संभवात् \textendash\ इति, तन्नः तथा सत्यध्येयो ज्ञेयश्चेत्येव सिद्धे
षडङ्ग इत्यस्य वैयर्थ्यापत्तेः~। तद् ध्वनयन्नाह \textendash\ [ १४ शप० ] शिक्षेति
~॥ [१४ शप०] सहित इति~। अनेन बहुब्रीहिः सूचितः~॥ यतु \textendash\ न
तेषामङ्गत्वम्, तद्धि तादर्थ्यात्तदवयवत्वाद्वा वाच्यम्~। तत्र 



१ ननु पाणिनीयव्याकरणस्यागमबोधितत्वं कथमुपपद्येत ? यत आगमो नित्यः ,
एतच्च व्याकरणं मुनित्रयप्रणितत्वादनित्यमिति नित्यानित्यसंयोगविरोधात्
श्रुतिबोधितत्वमस्य न स्यादत आह \textendash\ जनमे \textendash\ जयादीति~। 
स्वमतेनाप्युपपादयति \textendash\ यद्वां तेनेति~। तत्सूत्रे भाष्ये \textendash\ यद्यप्यर्थो
नित्यः, या त्वसौ {\qt वर्णानुपूर्वी सानित्या} इत्युक्तम्~। 

४ प्र ०पा० 



{\qt ध्ययनं कार्यम्} इत्यर्थं तात्पर्यम्~। अत एव समानसमीहमानानाम \textendash\ 
धीयानानां च केचिदर्ैर्युज्यन्ते {\qt केचिन्न} इति वक्ष्यति भाष्यकृत्~। 
अध्ययनतथाज्ञानयोश्च समुच्चयः~। अनेनातादृशस्याधर्मत्वं सूचयति~। श्रुत्वा
मिलितस्यैव धर्मत्वबोधनात्~। तथा च व्यासः \textendash\ 

वेदार्थज्ञो जपं जप्त्वा तथैवाध्ययनं द्विजः~। 

कुर्वन् स्वर्गमवाप्नोति नरकं तु विपर्यये इति~॥ 

जपपदं मन्त्रसाध्यकर्ममात्रोपलक्षणम्~॥ ज्ञाधातोश्च ज्ञानानुकूलो
मनःप्रणिधानऽविषयेन्द्रियसंयोगसंपादनादिरूपो व्यापारोऽर्थः, तस्यैवात्र
विधेयत्वम्~। एतेन \textendash\ ज्ञानस्याविधेयत्वादिदं चिन्त्यमित्यपास्तम्~॥ 



नाद्यः, तादर्थ्यवोधकश्रुत्याद्यभावात्~। नान्त्यः, अनित्यस्य
नित्यावयवत्वायोगात् \textendash\ इति~। तन्न, {\qt मुखं व्याकरणं प्रोक्तम्} इ्त्यने \textendash\ 
नाङ्गत्वेन संस्तवादुपकारकत्वाद्वाsज्ङ्गत्वसंभवात्~॥ यत्तु \textendash\ वेदत्वाव \textendash\ 
च्छिन्नस्यसकलशाखस्येदानीमध्ययनाद्यसंभत्वात्कलिकालप्रधानपाणिनिव्याकरणेन
सह तस्य तत्प्रतिपादनमयुक्तम् \textendash\ इति, तन्निरासा \textendash\ याह \textendash\ स्वेति~॥ अयं {\qt 
चार्थो बोध्यः~।} अध्येये यत् कर्मणि जायमानः {\qt प्रैषा \textendash\ } इति सूत्रेण
विधिरूपत्वे~। तदाह \textendash\ [१५ शपं०] अध्येय इत्यत इति~। 
व्युत्पत्तिवैचित्र्यात्तथाsन्वयं इति भावः~॥ [१७ शप०] निष्ठमिति~। 
व्यासज्यवृत्तीत्यर्थः~। एवं च शब्दस्य साक्षादध्ययनवदर्थस्य
ज्ञानद्वाराऽध्ययनसंभवादर्थबोधपर्यवसाय्य \textendash\ ध्ययनं
कार्यमित्यर्थाच्चाध्येय इत्येव सिद्धे ज्ञेय इति व्यर्थमिति भावः~। अत एव
सामर्थ्यादाह \textendash\ [ १७ शप० ] सम्यगिति~॥ नन्वध्ययनमात्रेण
सम्यगर्थबोधस्यैव जायमानत्वेनेदमयुक्तम्, अत आह \textendash\ [१८ शप०] अत एवेति~। 
तस्योक्तार्थतात्पर्यकत्वादेवेत्यर्थः~॥ [२० शप०] तथेति~। 
सम्यगर्थविषयकज्ञानेत्यर्थः~॥ समुच्चय इति~। {\qt अहरहर्नयमानो गामश्वं
पुरुषं पशुम्} इतिवत् विना च समुच्चय इत्यर्थः~॥ क्वचित् {\qt ज्ञेयश्च}
इत्येव पाठः~॥ [२०शप०] नातादृशस्येति~। प्रत्येकस्येत्यर्थाः~। 
अनेनेत्यस्यार्थमाह \textendash\ श्रुत्येति~॥ अत्र संमतिमाह \textendash\ [२१ शप०] तथा चेति~॥
जप्त्वाजपं कृत्वा~। तथाशब्दः७ समुच्चये~॥ एवः क्रियावधारणे~॥ विपर्यये
प्रत्येककरणे~॥ न्यूनतां परिहरति \textendash\ [ २४ शप० ] जपेति~॥ मन्त्रस्यैव
जपसत्त्वादाह \textendash\ मन्त्रेति~॥ विनिगमनाविरहादाह \textendash\ मात्रोपेति~॥ कस्यचिदुक्तिं
खण्डयितुं सर्वात्मनाऽध्येय इत्येतत्तुल्यतां ज्ञेय इत्यत्र कथयति \textendash\ [
२४ शप० ] ज्ञाधातोरिति~॥ आदिना स्थित्यादिपरिग्रहः~। तस्यैवेति~। तावतैव
ज्ञानविधित्वसिद्धेः~। फूत्कारादिविधिना पाकविधिवदिति भावः~॥ [२४ शप०
] ज्ञानस्येति~। तस्य कर्तुमर्क्तुमन्यथाकर्तुं
चाशक्यत्वाद्वाक्यादिरूपप्रमाणपरतन्त्रस्य तस्य विधेयताया असंभवाच्चेति
भावः~॥ [२६ शप० ] पास्तमिति~। उक्तयुक्तेः~। 
प्रतिमादावीश्वरबुद्धिवत्तस्यापि तथा कर्तुं शकयत्वाच्चेति भावः~॥ 



२ वेदस धर्मत्वमौपचारिकमिति संबन्धः~। (र. नाः ) ३ अध्येय इत्यतः प्रे \textendash\ 
इति घ. पाठः~। ४ {\qt भयनिष्ठं} इति घ. पाठः~। ५ प्रणि \textendash\ धानस्य संयोगस्य च
संपादनेऽन्वयः~। ( र. ना. ) ६ अध्येयशब्दे प्रैषेति सूत्रेण कर्मणि जायमानो
यत् विधिरूपत्वे पर्यवस्यतीति संबन्धः~। ( र. ना. ) ७ कुर्वन्नित्यनेन
संबद्धयते इति शेषः~। (र. ना. ) 

२६ उद्द्योतपरिवृतप्रदीपप्रकाशितमहाभाष्ये \textendash\ [१अ.१पा.१पस्पशाह्निके 



ननु गुणानां परस्परसंबन्धाभावात् {\qt षङङ्मेषु प्रधानम्} इत्यनुपपन्नमत
आह \textendash\ पदेति~। एवं वेदार्थज्ञाने व्याकरणं प्रधानं कारणमिति भावः
~॥भाष्ये \textendash\ प्रधाने कृतो यत्न इति~। १प्रधानविषये निर्वृत्त इत्यर्थः~। 
अन्यथा २यत्नविषयकृतेरसंभवादसंगतिः स्पष्टैव~॥ भाष्ये \textendash\ {\qt  फलवान्}
इत्यत्र फलपदेन वाक्यार्थावगमो विवक्षितः~। एतच्छास्त्रं च पद \textendash\ 
पदार्थावगमद्वारा वाक्यार्थावगमोपयोगीति बोध्यम्~॥ एवञ्च व्याक \textendash\ 
रणानध्ययनप्रयुक्तस्य वेदार्थाज्ञानप्रयुक्तस्य च प्रत्यवायस्य
परिहारकतयाऽस्य नित्यत्वमिति ३बोध्यम्~॥ 

 (लघुपदार्थनिरूपणभाष्यम् ) 

लघ्वर्थं चाध्येयं व्याकरणम्~। {\qt ब्राह्मणेनावश्यं}



 [उद्दयोते] तत्र वर्तमानं स्वाङ्गापेक्षया प्रधानमिति न भाष्यार्थः~। 
तदपेक्षया तत्त्वस्य शिक्षादावप्यविशिष्टत्वात् अतः तेषां मध्ये यत्
प्रधानम् अर्थात् तन्निरूपितप्राधान्यवद् इति तदर्थो वाच्यः सोऽपि न,
अङ्गानां मिथोऽसंबन्धेन गुणप्रधानभावाभावात्~। तदाह \textendash\ [ उ० २६ शप० ]
ननु गुणेति~॥ 

 [उद्द्योते] फलितमाह \textendash\ [उ० २८ शप०] एवं चेति~। अस्य
तन्मूलतन्निमित्तत्वे चेत्यर्थः~॥ ज्ञाने \textendash\ ज्ञानं प्रति~। एवं च
षट्स्वङ्गेषु वर्तमानं व्याकरणं वेदार्थज्ञानं प्रति प्रधानं \textendash\ मुख्यं
कारणमिति तदर्थो बोध्यः~॥ स्वरूपार्थप्रतिपत्तिभ्यामस्य तदुपकारकत्वमिति
सर्वोपजीव्यत्वादस्य प्राधान्यम् ,५अन्येषां त्वेकदेशे व्यापारः~। यथा
चैतत्तथाऽन्यत्र स्पष्टम्~॥ अत एवाङ्गत्वेन संस्तवेऽप्यस्य मुखत्वे \textendash\ 
नोक्तिः~। यथा सर्वशरीरावयवानां मध्येऽन्नपानादिप्रवेशनद्वारा
शरीरनिर्वाहकं मुखम् , एवमिदमपि वेदस्वरूपनिर्वाहकमिति भावः~॥ 

 [उ० २९ शप० ] प्रधानेति~। व्याकरणाध्ययनविषय इत्यर्थः~॥ कृतः
इत्यस्यार्थमाह \textendash\ निर्वृत्त इति~॥ अन्यथा \textendash\ उक्ता \textendash\ र्थाभावे~॥ ननु
फलवानित्यत्र फलशब्दस्य व्याकरणज्ञानार्थकत्वे
उक्तार्थाप्रतिपादकत्वापत्तिरत आह \textendash\ [ ३१ शप० ] फलेति~॥ वाक्यार्थेति
~। वेदार्थत्यर्थः~॥ एवमग्रेsपि~॥ फलितमाह \textendash\ एतदिति~॥ [ ३२ शप० ]
वाक्यार्थेति~। वाक्येत्यादिः~॥ परमपर्य \textendash\ वसितमुपसंहरति \textendash\ एवं चेति~॥
अस्य तथा तदुपयोगित्वे चेत्यर्थः~॥ 



१ उपासनीयं यत्नेन शास्त्रं व्याकरणं महत्~। 

प्रदीपभूतं सर्वासां विद्यानां यदवस्थितम्~॥ १~॥ 

इदमाद्यं पदस्थानं मुक्तिसोपानपर्वणाम्~। 

इयं सा मोक्षमाणानामजिह्मा राजपद्धतिः~॥ २~॥ 

रूपान्तरेण ते देवा विचरन्ति महीतले~। 

ये व्याकरणसंस्कारपवित्रितमुखा नराः~॥१~॥ 

एवं पदमञ्जर्यां प्राधान्यमस्य स्पष्टीकृतम्~। ( दा. म. ) 

२ कृधातोर्यत्नार्थकत्वाद्यत्नविषयो यत्नो न संभवतीत्यसङ्गतिरित्यर्थः~। 

३ फलचतुष्टयस्य प्राक्पश्चाद्वा प्रवर्तके वक्तव्ये मध्ये तु तदुक्त्या
फलसंदंशेन नित्यानामपि फलपर्यवसानमस्तीत्यादि शब्दकौस्तुभे स्पष्टम् (
दाधि. )~। १ 

४ {\qt ज्ञातुम्} इति प. पुस्तकपाठः~। 

५ अन्येषामिति~। व्याकरणातिरिक्तानामित्यर्थः तथा चोक्तं शिक्षायाम् \textendash\ 





शब्दा ज्ञेयाः? इति, न चान्तरेण व्याकरणं लघुनोपायेन शब्दाः शक्या
४विज्ञातुम्~॥ 

 ( प्रदीपः ) लघ्वर्थमित्ति~। लाघवेन शब्दज्ञानमस्य प्रयो \textendash\ जनम्~॥
ब्राह्मणेनेति~। अध्यापनं ब्राह्मणस्य वृत्तिः~। न चाशब्दज्ञं
तमुपश्लिष्यन्ति शिष्या इति~॥ 

 (उद्द्योतः) {\qt अवश्यम्}इत्यनेन विवक्षितं दर्शयति \textendash\ अध्या \textendash\ पनमिति~। लाघवं
च प्रतिपदपाठापेक्षया प्रकृत्यादिविभागेनान्वाख्यानस्य विवक्षितम्~॥ 



 [१ मभाष्ये] क्रमप्राप्तं लघुरूपप्रयोजनमाह \textendash\ भाष्ये \textendash\ लघ्वर्थं चेति
~॥ सकलसाधुशब्दज्ञानहेतुभूतप्रक्रियारूपलघुभूतोपायज्ञानार्थ \textendash\ मित्यर्थः~॥


 [भाष्ये] नतु किमर्थं साधुशब्दज्ञानं यदर्थं तादृशोपायज्ञानापेक्षा,
अतस्तत्र हेतुमाह \textendash\ भाष्ये \textendash\ ब्राह्मणेनेति~। अत्रापि
क्षत्रियविशोरध्यापनाभावादुक्तहेतोश्च ६{\qt ब्राह्मणेन} इत्युक्तम्~॥
इति \textendash\ हेतौ~॥ 

 [भाष्ये] एतदेवाह \textendash\ भाष्ये \textendash\ न चान्तेति~॥ व्याकरणाध्ययनं विना न
हीत्यर्थः~॥ प्रतिपदपाठरूपगुरूपायस्य सत्त्वादाह \textendash\ लघु \textendash\ नोपायेनेति~। 
लघुभूतेनोक्तान्योपायेनेत्यर्थः~॥ क्वचित् {\qt उपाया} \textendash\ न्तरेणः इत्येव पाठः~॥


 [ प्रदीपे ] तत्फलितमाह कैयटे \textendash\ लाघवेनेति~॥ 

 [प्रदीपे] ननु तस्य वृत्तित्वेऽपि प्रकृते किम् , अतो व्यतिरेकमुखेन
तद् द्रढ्यन्नाह \textendash\ कैयटे \textendash\ न चाशब्दज्ञमिति~। {\qt अपशब्दज्ञं इत्यपपाठः~।} एवं च
वृत्त्यनिर्वाह इति तदावश्यकत्वम्~। {\qt न च \textendash\ } इत्यस्य न ह्रीत्यर्थः~॥
इति \textendash\ समाप्तौ~॥ 

 [ उद्द्योते ] वृत्तिविधिं विनेव सिद्धस्यार्थस्यानुवादेन विधिकल्पने
मानाभावं ध्वनयन्नाह \textendash\ उद्योते \textendash\ अवश्यमिति~। विधिकल्पकत्वे तु तदानर्थक्यं
स्पष्टमेव~॥ 

 [उदद्योते कैयटोक्तलाघवं न शब्दज्ञाननिष्ठमित्याह \textendash\ उद्द्योतेलाघवं चेति
~॥ प्रतिपदेति~। आनन्त्येन तस्य कर्तुमशक्यत्वात्~॥ प्रकृत्येति~। 
सामान्यविशेषलक्षणावच्छास्त्रबोध्येनेति भावः~॥ 



मुखं व्याकरणं तस्य ज्यौतिषं नेत्रमुच्यते~। 

निरुक्तं श्रोत्रमुद्दिष्टं छन्दसां विचितिः पदे~। 

शिक्षा घ्राणं तु वेदस्य हस्तौ कल्पान् प्रचक्षते~॥ 

 ज्यौतिषं \textendash\ स्वाध्यायोपयोगिनमनुष्ठानोपयोगिनं च कालविशेषं प्रतिपादयति~। 


 निरुक्तं तु \textendash\ व्याकरणस्यैव परिशिष्टप्रायम्~। बाहुलकादिसाध्यानां
लोपागमविकारादीनां प्रायशस्तत्र संग्रहात्~॥ 

 छन्दोविचितिरपि \textendash\ {\qt गायञ्या यजति त्रिष्टुभा शंसति~।} इत्यादौ श्रुतानां
गायव्यादिशब्दानां लक्षणतोऽर्थमाचष्टे~॥ 

 शिक्षाऽपि \textendash\ अध्ययनकाले कर्मणि च मन्त्राणामुञ्चारणप्रकारं
प्रतिपादयति~॥

 कल्पसूत्राण्यपि \textendash\ प्रतिशाखं शाखान्तराधीतेन न्यायप्राप्तेन
चाङ्गेनोपेतस्य कर्मणः प्रयोगं कल्पयन्ति~॥ (दाधि. ) 

६ ब्राह्मणपदं त्रैवर्णिकोपलक्षणं \textendash\ कलौ तयोरभाव इति वा \textendash\ उक्तहेतोः~। 

शास्त्रप्रयोजनाधिकरणम् ] महाभाष्यप्रदीपोद्द्योतव्याख्या छाया~। २७ 



 ( असंदेहपदार्थनिरूपणभाष्यम् ) 

 असंदेहार्थं चाध्येयं व्याकरणम्~। याज्ञिकाः पठन्ति \textendash\ 
स्थूलपृषतीमाग्निवारिणीमनड्वाहीमालभेत इति~। तस्यां संदेहः \textendash\ स्थूला चासौ
पृषती च, स्थूलानि वा पृषन्ति यस्याः स स्थूलपृषतीति तां नावैयाकरणः
स्वरतोऽध्यवस्यति~। यदि पूर्वपद् प्रकृतिस्वरत्वं ततो बहुव्रीहिः,
अथ \textendash\ अन्तोदात्तत्वं तत्पुरुष इति~॥ 

 ( प्रदीपः ) असंदेहार्थमिति~। संदेहस्य प्रागभावोऽत्र द्रष्टव्यः, न
तु प्रध्वंसाभावः~। न दि वैयाकरणस्य संशय उत्पद्य विनश्यति, इतरस्यैव
तदुत्पादात्~। स्वरत इति~। पूर्वपद \textendash\ प्रकृतिस्वराद्बहुव्रीह्यर्थवसाय
इत्यर्थः~। 



 [भाष्ये ] असंदेहस्य फलत्वं कमप्राप्तमाह \textendash\ असंदेहेति~॥
संदेहप्रागभावपरिपालनार्थं चेत्यर्थः~॥ 

 [ भाष्ये ] ननु कुत्रावैयाकरणस्य संदेहोऽत आह \textendash\ भाष्ये \textendash\ 
याज्ञिका इति~॥ 

 [ भा० ३ यप० ] तस्यामिति \textendash\ भाष्ये~। श्रुतावित्यर्थः~। 
स्थूलपृषत्यामित्यर्थोऽनड्वाह्यामित्यर्थो वा~॥ यत्तु \textendash\ पृषतो
मृगविशेषस्तज्जातीया, पृषतजातित्वात् ङीप्~। सैवानोवहनयोग्या
चानड्वाहो \textendash\ इति कृष्णः~। तन्न~। तस्या गवि रूढत्वात्~। 
तस्मात्मत्त्वर्थलक्षणया पृषतीशब्देन पृषद्वती$=$बिन्दुमती गौरुच्यते ,
स्थूलाsपि सैव, बिन्दुषु न~। तत्र कर्मधारये गौरेवोच्यते, स्थूलाऽपि~। 
बिन्दुषु न स्थौल्यान्वयः~। बहुव्रीहौ तु स्थौल्यान्वयः पृषत्सु, न गवि~॥
न च कर्मधारयस्य स्वपदार्थप्राधान्येन बहुब्रीहेः प्राबल्यादसंदेह एवेति
वाच्यम्~। अन्यपदार्थप्रधानाग्निवारुणीशब्दसमभिव्याहारेण तस्यापि संभवात्
~॥ न च {\qt व्याख्यानतः} इति परिभाषयाः निर्णयः ; पौरुषेयेषु
लक्ष्यानुरोधेन तया निर्णयेऽप्यपौरुषेयेऽदृष्टार्थविषये वेदे तया तदसंभवात्
~। तस्मात् स्वरतो निर्णयो वैयाकरणैकसाध्यः~। तदेव
व्यतिरेकमु्खेनाह \textendash\ भाष्ये \textendash\ तामिति~॥ 

 [भाव्ये] स्वरतोऽध्यवसायप्रकारमाह \textendash\ [भा०५मप० ]यदीति~॥ 

 [ प्रदीपे ] कैयटे \textendash\ अत्र द्रष्टव्य इति~। न तु सर्वत्रेति भावः~॥
एतेन \textendash\ {\qt संदेहाभाव एवार्थः~।} कैयटोक्तं तु चिन्त्यम्, ईदृशाग्रहे
मानफलयोरभावात्~। अनुभवविरोधात्~। शङ्कासमाधिपरसमस्त \textendash\ भाष्यासंगतेश्चःइति
रत्नोक्तम् ) \textendash\ {\qt प्रध्वंस एव सः}, उत्पन्नस्य तस्य हि वैयाकरणेन
शास्त्रहितसंस्कारवता निरासः सुकरः \textendash\ इति कृष्णोक्तं चापास्तम्~। उक्तहेतोः
~। अवैयाकरणवैयाकरणभेदेन भगवता शङ्कासमाध्योरुक्तेश्चेति दिक्~। 
तदाह \textendash\ कैयटे \textendash\ न हीति~॥ न च {\qt व्याख्यातृपरम्परया तदभावात् किमनेन \textendash\ ?}
प्रतिपदं प्रतिपुरुषं प्रतिक्षणं जायमानसंशयानां तैर्निरसितुमशक्यत्वात्
~। तेषां सर्वदा सांनिध्याभावाच्च~॥ नचास्यार्थनिश्चायकत्वे गवादीनां
पुरुषाद्यर्थकताऽपि स्यात् , अर्थमनूद्य शब्दानामिदमनुशासनं
नार्थानामिति व्युत्पत्त्यर्थ तदाश्रयणेऽपि यत्र प्रमाणान्तराविरोधः, तत्र
तत्सत्त्वमन्यत्र नेत्येवं देवताधिकरणन्यायेनाङ्गीकारात्~। स च पदार्थविषयो
न वाक्यार्थविषयः~। पदानामेवानुशासनादिति वोध्यम्~। तदेतद् ध्वनयन्
वक्ष्यते \textendash\ तां नेति ( भाष्ये )~॥



तत्परिपालनं \textendash\ असंदेहपरिपालनं तदर्थमध्येयं व्याकरणम्~। 





 ( उद्द्योतः)नन्वसंदेहः$=$संदेहाभावः~। स च नात्यन्ताभावः~। तस्य
नित्यत्वात्~। न च ध्वंसः , तत्तद्विषयकव्याकरणज्ञानवतः संदेह \textendash\ स्यैवाभावेन
तदसंभवात्~। यस्य त्वधीतव्याकरणस्यापि क्वचित्संदेहः स
तद्विषयकवैयाकरणत्वाभाववानेव, वेदनार्थे हि तत्र प्रत्ययोऽत
आह \textendash\ प्रागभाव इति~। तत्१परिपालनं च शास्त्राध्ययनफलम्~। न तु स्वरूपं,
तस्याजन्यत्वात्~। तत्पालनं च प्रतियोगिजनककारणविध्वंसनं, प्रागभावस्य
प्रतियोगिजन्मनाश्यत्वात् \textendash\ इति बोध्यम्~॥ भाष्ये \textendash\ याज्ञिकाः पठन्तीति~। 
यज्ञकाण्डभवा वैदिकाः शब्दाः \textendash\ याशिकाः~। ते पठन्तीत्यर्थः~। ऋषिः पठति
{\qt शृणोत ग्रावाणा} इतिवत्प्रयोगः~। तत्र ऋषिः$=$वेदः~। सन्सूत्रे भाष्ये एष
प्रयोगः~। ताम् \textendash\ अनड्वाहीम्~। कस्तर्हि निश्चयस्तत्राह \textendash\ पूर्वपदेति~॥ 



 [ प्र० ४ र्थप० ] कैयटे \textendash\ स्वरात् \textendash\ पठ्यमानात्~॥ स्थूलशब्दः
फिट्स्वरेणान्तोदात्तः~। अर्थविशेषाश्रयबहुव्रीहिनिबन्धनपूर्वपदप्रकृति \textendash\ 
स्वरस्य तेन विनास्संभवात्तस्य निर्णायकत्वं युक्तमिति भावः~॥ अवैयाकरणस्य
त्वनिर्णिय एव, वृत्तिपदार्थज्ञानादेरप्यभावेन विपरीत \textendash\ ~। 
निर्णयस्याप्यभावात्~। अन्यस्य कस्यचिल्लौकिकस्य निर्णायकस्यासंभवाच्च~॥
एतेन \textendash\ तस्य विपरीतनिर्णयः स्याद् \textendash\ इति कृष्णोक्तमपास्तम्~॥ 

 [उद्द्योते ] एतदनभिज्ञः शङ्कते \textendash\ [उ० १म०प० ] नन्बसंदेहेति~॥
[२ यप०] नित्यत्वादिति~॥ एवं च फलत्वासंभवस्तस्य~॥ [ ३ यप० ]
तदेति~। ध्वंसेत्यर्थः~॥ ननु तदध्ययनसत्त्वेन कथं तत्त्वम्, तत्रापि
प्रत्ययसत्त्वादत आह \textendash\ [ ४ र्धप० ] वेदनेति~॥ यतो वैयाकरणशब्दे इत्यर्थः
~। उक्तहेतोरिति यावत्~॥ तथा च वैयाकरणत्वमव्याप्यवृत्तीति भावः~॥
तस्याजन्यत्वेनाफलत्वा \textendash\ दाह \textendash\ [५ मप०] तत्परीत्ति~॥ शास्त्रेति
~। व्याकरणेत्यर्थः~॥ ननु तस्य निरवयवत्वेन पोषणाद्यसंभवात्कथं परिपालनमत
आह \textendash\ [६ ष्ठप०] तत्पालनमिति~॥ कारणेति~। संदेह२जनकप्रतिबन्धका \textendash\ 
भावरूपेत्यर्थः~॥ ननु तस्य कथं तत्त्वमत आह \textendash\ [७ मप०] प्रागेति~। तथाच
कोटिद्वयोपस्थितिः सुतरां नेति भावः~॥ 

 [उद्दयोते] यत्तु \textendash\ {\qt यज्ञं विदन्ति ते याज्ञिकाः}, उक्थादित्वाठ्ठक्~। 
पाठसमयेऽध्यापकादीनां निर्णिनीषाया अनावश्यकत्वाद् याज्ञिका इत्युक्तम् , न
{\qt त्वध्येत्रादयः} इति कृष्णः~। तन्न, अस्याः श्रुतित्वेनं {\qt इति श्रूयते
तत्र संदेहं कुर्वन्ति} इति वाच्ये तथोन्यसांगत्यापत्तेः~। तदेतद्
ध्वनयन्नाह \textendash\ [ उ० ८ मप० ] यज्ञेति~॥ लक्षणयाऽर्थमrह \textendash\ काण्डेति
~॥ ननु न तेषां पठनसंभवोऽत आह \textendash\ [ ९ मप० ]ऋषिरिति~॥ एवं च बोधने
लाक्षणिकः प्रयोग इति भावः~॥ तत्र पूर्ववाक्ये तत्प्रयोगस्य निर्मूलत्वं
निराचष्टे \textendash\ [ १० मप० ] सन्निति~॥ 

 [उद्दयोते] पूर्वव्याख्यानासांगत्यं ध्वनयन् स्थूल्पृषतीमिति विहाय
व्याचष्टे \textendash\ [उ० १० मप०] अनड्वाहीमिति~। यद्यपि
स्थूलपृषतीशब्देऽर्थसंदेहः, तथापि पर्यन्ततो गव्येव संदेह इति भावः~॥ 

 [उद्दयोते] नन्वेवं भगवता निश्चयो नोक्त इति संशय एवेत्याशये
नाह \textendash\ [उ० ११ शप० ] कस्तर्हीति~॥ 



२सन्देहजनकं यत्प्रतिबन्धकाभावरूपं कारणं तद्विध्वंसनमित्यर्थः(र.ना.) 

२८ उद्द्योतपरिवृतप्रदीपप्रकाशितमहाभाष्यम्~। 

 [९ अ. १ पा. १ पस्पशाह्निके



 (व्याकरणाध्ययनसाधकागमप्रतीकभाष्यम् ) 

 इमानि च भूयः शब्दानुशासनप्रयोजनानि \textendash\ ते \textendash\ ऽसुराः~। दुष्टः शब्दः~। 
यदधीतम्~। यस्तु प्रयुङ्क्ते~। अविद्धांसः~। विभक्तिं कुर्वन्ति~। यो वा
इमाम्~। चत्वारि~। उत त्वः~। सक्तुमिव~। सारस्वतीम्~। दशम्यां पुत्रस्य~। 
सुदेवो असि वरूण \textendash\ इति~॥ 

 ( प्रदीपः ) मुख्यानि प्रयोजनानि प्रदर्श्यानुषङ्गिकाणि प्रद \textendash\ 
र्शयति \textendash\ इमानि चेति~॥ भूय इति~। पुनरित्यर्थः~। 
आनुषङ्गिकत्वाञ्चैषां वर्गद्वयोपादानम्~॥ 

 ( उद्व्योतः ) ननु अम्लेच्छत्वादीनां बहुत्वात् {\qt बहूनामनुग्रहो
न्याय्यः} इति प्रथमतोsभिधानं युक्तमत आह \textendash\ मुख्यानीति~। प्रधानत्वादेव
तेषां प्रथमतोsभिधानमिति भावः~॥ तेषां प्रधानत्वं च
{\qt पदपदार्थज्ञानाधीनत्वेनान्तरङ्गत्वात्}~। वक्ष्यमाणानां च वहिरङ्ग \textendash\ 
शब्दापशब्दप्रयोगविधिनिषेधविषयत्वादानुषङ्गिकत्वं बोध्यम्~॥ ननु
{\qt भूयः} इत्येकवचनमनुपपन्नं {\qt प्रयोजनानि} इत्यनेन सामानाधि \textendash\ 
करण्यादत आह \textendash\ पुनरिति~॥ इमानि चेति~। च \textendash\ समुच्चये~॥
भाष्ये \textendash\ प्रयोजनानीति~। प्रवर्तकानि फलानि चेत्यर्थः~। तत्र वक्ष्यमाण \textendash\ 
वाक्यानि प्रवर्तकानि, तद्बोध्यानि फलानीति विवेकः~॥ ननु रक्षादिभिः 



 [भाष्ये] इमानीति \textendash\ भाष्ये वक्ष्यमाणानीत्यर्थः~॥ 

 [भाष्ये] प्रयोजनेयत्तायाः स्पष्टत्वाय तान्यादौ प्रतीकं धृत्वा
गणयति \textendash\ भाष्ये \textendash\ तेऽसुरा इत्यादि~॥ {\qt विभक्तिं कुर्वन्ति} इति
वाक्यार्थकथनम्~। {\qt सारस्वतीं} इति तद्वाक्यस्थपदोक्तिः~॥ 

 [ उ० २यप० ] उद्द्योते \textendash\ इतीति~। इति न्यायेनेत्यर्थः~। 

 [४र्धप०] पदार्थज्ञानेति~। तन्मात्रेत्यर्थः~॥ 

 [उ० ७मप०] पुनरितीति~। पश्चादिति तदर्थः~॥ अव्ययमिति भावः~॥ यद्यपि
तस्याधिक्यमप्यर्थः , तथापि योग्यतया तस्यैव ग्रहणम्~॥ 

 [उ० ८मप०] चेति~। पश्चादिति तस्यार्थः~। इति \textendash\ बोध्यानीति शेषः
~॥ प्रागुक्तरीत्याऽर्थद्वयं बोध्यम्~॥ तत्र \textendash\ तयोर्मध्ये~॥ 

 [उ० ९मप०] ननु रक्षादिभिरिति~। यद्यप्युक्तरीत्याऽन्तरङ्ग
बहिरङ्गभावः, तथापि प्रधानत्वाप्रधानत्वयोर्मानाभावः~। किं चास्तु तयोरपि
सत्त्वम् , तथापि क्रमेण तैः सह पाठ एवैषामुचित इति भावः~॥ [ ११ शप० ]
बोधयितुमिति~। तथा च तथा पाठे स्वस्य ज्ञानसत्त्वेऽपि परस्य ज्ञानं न
स्यादिति परबोधायैव परमुक्तिः~॥ एवं च कैयटे चः \textendash\ हेतौ~। यतो
हेतोरानुषङ्गिकत्वमेषाम्, अत इत्याद्यर्थो बोध्य इति भावः~॥ 

 [भा्ये] तत्रादावुद्देशक्रमेणाभ्लेच्छत्वं फलं वदन् तद्वाक्यं
ब्राह्म \textendash\ णस्थमुक्तप्रतीकसंघटितमाह \textendash\ भाष्ये \textendash\ तेsसुरास्तेऽसुरा इति~॥
इयं रीतिरग्रे सर्वत्र बोध्या~॥ 

 [भाष्ये] तत्र निषेधस्य सविषयनिन्दापेक्षायां विषयस्य निन्दितत्वं
फलमाह \textendash\ भाष्ये \textendash\ म्लेच्छो ह वा इति~॥ यत्$=$यस्मात् एषोऽप \textendash\ शब्दो
म्लेच्छः \textendash\ म्लेच्छतयाऽतिप्रसिद्ध इत्यर्थः~॥



१ ते सुराः इति प्रथमप्रतीकस्य अ \textendash\ पुस्तके न पाठः~। 



सहैवैतानि कुतो न पठितानि ? इत्यत आह \textendash\ आनुषङ्गिकत्वा दिति~। 
एषामानुषन्गिकत्वं तेषां प्रधानत्वं बोधयितुं तथोक्तिरिति भावः~॥ 

 (भाष्यम् ) 

तेऽसुराः१ \textendash\ 

{\qt तेऽसुरा हेऽलयो हेऽलय इति कुर्वन्तः परा \textendash\ वभूवुः~। 
तस्माद्ब्राह्मणेन न म्लेच्छितवै नापभाषितवै, म्लेच्छो ह वा एष यदपशब्दः~॥} म्लेच्छा मा भूमेत्यध्येयं व्याकरणम्~। तेऽसुराः~॥ 

 ( प्रदीपः ) तेऽसुरा हेऽलय इति~। निन्दाऽर्थवादेन {\qt न
म्लेच्छितवै} इति म्लेच्छनं निषिध्यते~। तत्र केचिदाहुः \textendash\ 
{\qt हैहेप्रयोगे हैहयोः} इति प्लुते प्रकृतिभावे च कर्तव्ये तदकरणं
म्लेच्छनमिति~। पदद्विर्वचने कार्ये वाक्यद्विर्वचनं लत्वं च म्लेच्छन \textendash\ 
मित्यपरे~। न म्लेच्छितवा इत्यस्य पर्यायः \textendash\ नापभाषितवा इति~। 
{\qt कृत्यार्थे \textendash\ } इति तवैप्रत्ययः~॥म्लेच्छ इति~। कर्मणि घञ्~॥ 

 (उद्द्योतः ) ननु {\qt पराबभूवुस्तस्माद्ब्राह्मणेन \textendash\ } इति श्रुतौ
पराभवस्यापशब्दभाषणफलत्वावगतेः {\qt पराभूता मा भूमेत्यध्येयम्} इति
वक्तुमुचितमत आह \textendash\ निन्दाऽर्थवादेनेति~। {\qt द्रव्यसंस्कारकर्मसु
परार्थत्वात्फलश्रुतिरर्थवादः स्यात्} (पू० मी० ४~। ३~। १) इति न्यायेन
{\qt यस्य पर्णमयी जुहूर्भवति न स पापं श्लोकं शृणोति} 



 [ भा० ५ मप० ] अनपभायणस्याम्लेच्छत्वफलकत्वाद्व्याकरणा \textendash\ 
ध्ययनस्याम्लेच्छ्त्वं फलम् \textendash\ इत्युपसंहृतदार्ढ्याय पुनः {\qt तेऽसुराः
इत्युक्तम्~॥} एवमग्रे सर्वत्र ज्ञेयम्~॥ 

 [प्रदीपे] ननु किमनृतं यत्कृतस्तेषांपराभवः, अत आह \textendash\ कैयटे [२ यप]
तत्रेति~। श्रुतावित्यर्थः~॥ [३ यप०] प्रकृतीति~। 
तत्प्रयुक्तेत्वादिः~॥ {\qt प्लुतप्रगृह्याः} इत्यनेनेति भावः~॥ 

 [प्र० ६ष्ठप०] अत एवाह \textendash\ कैयटे \textendash\ कृत्यार्थ इति~॥ उभयत्रे \textendash\ त्यादिः~॥


 [प्रदीपे] यत्तु \textendash\ कर्मणि करणे वा घञ् \textendash\ इति कृष्णः~। तन्न, करणे तस्य
दौर्लभ्यात्~। तद्भुनयन्नाह कैयटे [६ ष्ठप० ] कर्मणीति~॥ 

 [उद्दयोते] {\qt तेऽसुराः} इत्यादेः श्रुतित्वेऽप्यनर्थयुक्तत्वादाह \textendash\ 
उद्दयोते \textendash\ परेति~॥ [३ यप०] निन्दाऽर्थवादेनेतीति~। तृतीयया पूर्वत्वं
सूच्यते~। तथाच निन्दाऽर्थवादपूर्वकं {\qt न म्लेच्छितवै} इत्यनेन म्लेच्छनं
निषिध्यते \textendash\ न विधीयत इति कैयटार्थो बोध्यः~॥ न च {\qt तेऽसुराः} इति
आर्थवादिकफलनिर्देश एव, {\qt म्लेच्छो हः इत्यस्य वाक्यस्यानर्थक्यापत्तैः~। न
च म्लेच्छस्वरूपप्रदर्शनाय तत्, नापभाषि \textendash\ } इत्यनेनैव गतार्थत्वात्~॥
तस्मात् कर्माङ्गेऽपशब्द \textendash\ निषेधे फलश्रुतेरर्थवादत्वमेव न्याय्यम्~। 
तदाह \textendash\ द्रव्येति~। जैमिनीयमिदम्~। अस्यार्थःकौस्तुभव्याख्यातोsवगन्तव्यः~॥
तत्र द्रव्यस्योदाहरणमाह \textendash\ [५ मप०] यस्येति~॥ [५ मप०] स
पा \textendash\ 

शास्त्रप्रयोजनाधिकरणम् ] महाभाष्यप्रदीपोद्द्योतव्याख्या छाया~। २९



इत्यादावपापश्लोकश्रवणस्य प्राशस्त्यमात्रपरत्ववदस्य
निन्दामात्रपरत्वादिति भावः~। प्रकरणाच्च क्रत्वङ्गोऽयं१निषेधः~। 
केचिदाहुरिति~। अत्रारुचिबीजं तु {\qt सर्वः प्लुतः साहसमनिच्छता विभाषा
कर्तव्यः} इति~॥ पदेति~। {\qt सर्वस्य द्वे} इत्यत्र पदग्रहणस्य
वार्तिककृतोक्तत्वा \textendash\ 



षेति~। पापश्लोकश्रवणाभावस्येत्यर्थः~॥ प्राशस्त्येति~। {\qt पर्णमयी
जुहूः} इत्यादिः~॥ मात्रपदेन स्वार्थपरस्वनिरासः~॥ एवमग्रेऽपि~॥
अस्य \textendash\ तेऽसुरा इत्यर्थवादस्य~॥ ननु निषेधे द्रव्यत्वाभावेनानत्ङ्गत्वेन
तदन्यान्यतरत्वस्याप्यभावेन दृष्टान्तवैलक्षण्यात्कथं {\qt द्रव्य} \textendash\ इति
न्यायसंचारोऽत आह \textendash\ [ ७ मप० ] प्रकेति~॥ चरत्वर्थः~। {\qt क्रत्वङ्गमयम्}
इति पाठः~। अङ्गशब्दस्य नित्यक्लीबत्वात्~॥ यत्तु {\qt यर्वाणो नाम}
इत्यादुत्तरत्र भगवदुक्तेरयं निषेधः क्रत्वङ्गम् \textendash\ इति कृष्णः~। तन्न,
प्रकरणलब्धस्यैवार्थस्याग्रे भगवता कथनात्~॥ 

 [ उद्दयोते ] यत्तु रत्नकृत् \textendash\ इदमयुक्तम् {\qt ई चाक्रवर्मणस्य}
इत्यसुतवद्भावस्य {\qt ई}ग्रहणाभावेनान्यत्रापीष्टत्वेनात्र प्लुते सत्यपि
पूर्वरूपसंभवात्, पूर्वरूपं जायमानं प्लुतात्मकमेव भवतीति तदनुच्चा \textendash\ रणं
दोषः \textendash\ इति तु {\qt युक्तम्} \textendash\ इति~। तन्न, हैहे.३ \ldots \textendash\ तदेतद्
ध्यनयन्नाह \textendash\ [ ७ मप० ] अत्रेति~॥ अत एवाह \textendash\ [ ९ मप० ] पदेतीति~॥
यत्तु \textendash\ {\qt पदस्य} इत्यधिकारात् \textendash\ इति कृष्णः~। तन्न, तस्याग्रे सत्तेsपि
पूर्वमभावात् ; अत आह \textendash\ [ ५ मप० ] तोक्तत्वादिति~। यदि तु {\qt सर्वस्य
इतिवत्तत्रत्यपदस्य} इत्यस्याप्यधिकारः \textendash\ इत्युच्यते, तर्ह्यस्तु सोऽपि
साधुः~। यत्तु रत्नकृत् \textendash\ असुरोक्त्यनुकरणतया पदस्थानापन्न \textendash\ स्यैव वेदे
द्विरुक्तत्वेन वाक्यद्वित्वस्य तैरकृतत्वादिदमयुक्तम्~। बहु \textendash\ कृत्वो
वाक्यप्रयोगस्तु वीप्साद्योतनार्थम् \textendash\ इति, तन्न, मिथोविरुद्धवात्~। 



१ क्रत्वङ्ग इति~। सप्तम्यन्तम्~। तथाच क्रत्वङ्गे क्रतूपकारके ऋतुशेषे
भाषणेsयं निषेध इत्यर्थः~॥ {\qt ऋत्वङ्गेऽयं इत्येवं संहितालेखे
पूर्वरूपसूचकेsर्धाकारचिह्ने ईषद्वक्ररेखात्मके भ्रान्त्या सरललेखाज्ञानेन}
क्रत्वङ्गोयं इति पाठो लेखकेन प्रमादाल्लिखितो भवेत्~॥ यदि च
छायाकर्तुर्नैव भ्रमसंभवः, तस्य
नागेशान्तेवासित्वान्नागेशहस्तलिखितपुस्तकस्यापि सुलभत्वम् \textendash\ इति विभाव्यते
तदा \textendash\ अङ्गं अस्य अस्ति इति अङ्गः {\qt अर्शआदिभ्योऽच्?} \textendash\ अङ्गी,
{\qt येनाङ्गविकारः} इति सूत्र तथाऽङ्गीकारात्~॥ एवं च {\qt क्रतुः अङ्गः \textendash\ अङ्गी
यस्य} इति बहुव्रीहौ {\qt क्रत्वङ्गः क्रतूपकारकः क्रतुशेषः \textendash\ अयं निषेधः
इत्यर्थेन पुंलिङ्गनिर्देशो नैव विरुद्धः \textendash\ इत्येवं सर्वं समञ्जसम्~॥
एतज्ज्ञात्वैव छाया कृताक्रत्वङ्गमयम्} इति पाठः {\qt अङ्गशब्दस्य}
नित्यक्लीबत्वात् इत्युक्तम्~॥ एवंच मूलच्छायाविरोधात् {\qt क्रत्वङ्गः}
इत्यस्य स्थाने क्रत्वर्थः इति पाठकल्पनं Bengal Asiatic Society पुस्तके
चिन्त्यमेव \textendash\ इति दाधिमधाः~। {\qt अङ्गत्वपरादङ्गशब्दादर्शभाद्यचि} साधुरयमिति
वयम्~। (रघुनाथाः)~। एतत्सर्वं पण्डितद्वयोक्तमपिप्रामादिकमेव~। म. म.
शिवदतैः क्रत्वङ्गेऽयं? इति पाठं प्रकल्प्य प्रदर्शितं पाण्डित्यं
मीमांसाsनभिज्ञत्वन्तेषां प्रदर्शयति~। तथाहि \textendash\ विधिवाक्यानां
कथैभावाकाङ्क्षासत्वेन \textendash\ अस्य तेऽसुरा इत्वादिवाक्यस्य च
भाव्याकाङ्क्षासत्वेनोभयाकाङ्क्षारूपप्रकरणप्रमाणबलादयं निषेधः
क्रत्वङ्गमित्येवोद्दयो \textendash\ 





दिति भावः ~॥ ननु नेदं द्वित्वं, किं त्वैच्छिकः पुनः प्रयोगोsत आह \textendash\ लत्वं
चेति~। {\qt न म्लेच्छितवै} इत्येतत् म्लेच्छमाषाविषयमिति
भ्रम \textendash\ निवृत्त्यर्थं तद्विवरणं \textendash\ नापभाषितवै? इति~। {\qt ह वै}
इति \textendash\ प्रसिद्धौ~। अपशब्दत्वं व्याकरणानुगतशब्दस्येषद्भ्रंशन एव प्रसिद्धमिति
भावः~॥ 



अन्यथानुकरणत्वासंभवाच्च~। तदेतद् ध्वनयन्नाह \textendash\ [ १० मप० ] ननु
नेदमिति~॥ पुनरिति~। तात्पर्यद्योतनार्थमिति शेषः ~॥ {\qt अनावृत्तिः
शब्दात् अनावृत्तिः शब्दात्} इति व्याससूत्रवदिति भावः~॥
आम्रेडितस्वररहितपाठस्यैव सत्त्वेन तदनुपत्तिरूपदोषाभावादाह \textendash\ लत्वं चेतीति
~॥ शाखान्तरे {\qt हेलवो हेलवः} इति पाठे वत्वमपि तदिति ज्ञेयम्~॥
भाष्यकैयटसाधारण्येनाह \textendash\ [ ११ शप० ] न म्लेच्छीति~। 
इत्येतत् \textendash\ विधिवाक्यम्~। म्लेच्छभाषेति~। यवनभाषानिषेधेत्यर्थः~॥ सा
निषेध्यत्वेन विषयो यस्येत्यर्थ तु यथाश्रुतमेव साधु~॥ {\qt न म्लेच्छभाषां
शिक्षेत्} इति तद्विषयनिेषेधस्यातिरिक्तस्य सत्त्वादाह \textendash\ [ १ १ शप० ]
भ्रमेति~॥ न चैवं तस्याः स्मृतेर्मूलान्तरं कल्प्यं स्यात् प्रमाणतया
समागतस्यार्थस्य तत्कल्पनायां बाधकाभावादिति भावः~॥ 

 [ उ० १२ शप० ] उद्योते \textendash\ ह वै इतीति~। निपातसमुदायो द्वौ वेत्यर्थः
~॥ न चानेन वैदिकशब्दमात्रापभाषणं निषिध्यते, अर्थवादे
सामान्यापभाषणानुवादेन पराभवस्योक्तत्वात्~। न म्लेच्छितवा इत्यस्य
म्लेच्छभाषाविषयत्वाभावे युक्त्यन्तरं ध्वनयन्नत्र हेत्वन्तर \textendash\ माह \textendash\ [ १३
शप० ] अपेति~॥ 

 [उ० १३ शप० ] एवो व्युत्क्रमे, ईषद्भ्रंशतो व्याकरणानुगतश \textendash\ 
ब्दस्यैतेत्यर्थः~। तथाच तेन तन्नग्राह्यम्, तथा न वाच्यं वेति भावः~॥ 



ताभिप्रायः~। भाषणस्य वाक्येनाविधानात्तस्य भाव्याकाङ्क्षा
नसम्भवतीत्युभयाकाक्षारूपप्रकरणस्यासम्भव एवेति{\qt क्रतूपकारकेक्रतुशेषे
भाषणेऽयं निषेधः} इत्यर्थं प्रदर्शयन्तः {\qt प्रकरणाच्च क्रत्वङ्गोsयं निषेधः}
इत्युद्वयोतं कथं योजयन्तीति त एवात्र प्रष्टव्याः~। यदि च
{\qt क्रत्वर्थोऽय}ं इति पाठं क्वचिदुपलभ्य वङ्गदेशमुद्रितपुस्तके
निर्दिंष्टस्ततः स साधरेवेति नाविदितं मीमांसकसरणिज्ञानवताम्~। 
{\qt अङ्गत्वपरादङ्गशब्दादर्शआद्यचि \textendash\ ?} इत्याद्यक्तिस्तु
रघुनाथानामुपहसनीयैवेत्यलं परदोषाविष्करणेन~। इदानीमुपलब्धपुस्तकेषु
{\qt क्रत्वङ्गोऽवं} इति पाठदर्शनात्स एवात्र स्थापितः~। 

२ इदं च भाष्यादिषु प्रसिद्धं श्रुतिपाठमनुसृत्य व्याख्यातम्~। अयं च
पाठः क्वचिच्छाखायामन्वेषणीयः~। माध्यंदिनानां शतपथ \textendash\ ब्राह्मणे तु
{\qt हेलवोऽहेलव इति वदन्तः} इति पठित्वा {\qt तस्माद्ब्राह्मणो न म्लेच्छेत्} इति
पठ्यते~। तत्र यकारस्थाने वकारोsपशब्द इति स्पष्टम् इति श० कौ०~। 

३ इत्यनेन तस्य बाधादित्यर्थकः पाठो भाति~। ( र. ना. ) 

४ पश्चात्पठितस्यापि पदस्मेत्यस्य पूर्वत्रानुकर्षणमित्यर्थः~। (र.ना.)
नायंछायाऽभिप्रायः, किन्तु वार्तिककृताऽत्र सूत्रे उक्तस्य पदग्रहृण \textendash\ 
स्यापि सर्वस्येतिवदधिकार इत्युच्यते \textendash\ इत्येवाभिप्रायः~। 

३० उद्द्योतपरिवृतप्रदीपप्रकाशितमहाभाष्यम्~। 

 [ १ अ. १ पा. १ पस्पशाह्निके



ननु म्लेच्छो नाम पुरुषविशेषो देशविशेषो वा, स कथमपशब्दोऽत आह \textendash\ घञिति
~। निन्दावचनान्म्लेच्छतेरिति भावः~। निन्दा च शास्त्रबोधितविपरीतोच्चारणेन
पापसाधनत्वात्~। एवं च म्लेच्छा इत्यस्य निन्द्या इत्यर्थ इति दिक्~॥ 

 (भाष्यम्) 

दुष्टः शब्दः \textendash\ 

 दुष्टः शब्दः स्वरतो वर्णतो वा 

 मिथ्याप्रयुक्तो न तमर्थमाह~। 



 [ उद्दयोते ] पुरुषे देशस्थत्वादेव वृत्तेराह \textendash\ [ १४ शप० ] यत्तु
म्लेच्छधातोरव्यक्तवाचित्वाद् अपशब्दस्य च
मन्दोच्चारणबोधकत्वादव्यक्तत्वसंभवात्सामानाधिकरण्यम्~। न चैवं कथं
निन्द्यत्वम् , व्यक्तापेक्षयाsव्यक्तस्य लोके निन्दितत्वात् {\qt अयं न}
स्पष्टं वदति किंत्वव्यक्तम् इति कृष्णः~। तन्न, एवं
सत्यपभाषणस्योक्तरीत्याsव्यक्तत्वेऽपि दृष्टादृष्टक्रतुवैगुण्यानापादकत्वेन
निषेधस्य क्रत्वङ्गत्वासंभवापत्तेः~। तदेतद् ध्वनयन्नाह \textendash\ [ १५ \textendash\ शप० ]
निन्देति~। धातूनामनेकार्थत्वादिति भावः~॥ तोच्चारणेन \textendash\ तद्विषयत्वेन~॥
शास्त्र३बोधितत्वं तद्विपरीतत्वं चोच्चारणक्रियाया यथाकथंचिद्विशेषणं
बोध्यम्~॥ एवं च \textendash\ तस्य तथा पापसाधनत्वे च~॥ तथा तस्य तद्विषयत्वेन
तत्त्वात्तत्त्वम्, तथा तत्प्रयोक्तुः पापस्य जायमानत्वेन तथोच्चारणेन
तत्साधनत्वात्तेषामपि निन्दाविषयत्वम्~। तदाह \textendash\ म्लेच्छा इति~॥
तत्फलप्रदर्शनमुखेनैव त्तेऽसुरा इत्यर्थवादः~॥ एतेन \textendash\ अपशब्दस्य म्लेच्छत्वं
जातिम्ले \textendash\ च्छकृतमन्दोच्चारणविषयतया गौणम्~। तेन म्लेच्छ इति गौण्युक्तिः~। 
तथा च यस्मादेषोऽपशब्दोऽतिप्रसिद्धो म्लेच्छस्तस्मात् तेऽसुराः देवान्प्रति
तथा वक्तव्ये तथाऽपशब्दं कुर्वन्तः पराभवमापुः~। तस्माद्
ब्राह्मणेनापशब्दभाषणं सर्वथा वर्ज्यमित्यर्थो वाच्यः \textendash\ इति
कृष्णोक्तमपास्तम्~। उक्तरीत्या मुख्यतयैव निर्वाहे गौणपरतया
योजनऽनौचित्यात्~। तेषां तथोच्चारणनियमाभावाच्चेति दिक्~॥
ब्राह्मणेनेत्युक्तिस्तूक्ताभिप्रायैवेति बोध्यम्~। तदाह \textendash\ दिगिति~॥ 

 [भाष्ये ] \textendash\ एवं वर्णापराधेऽनर्थमुक्त्वा वाक्यापराधेsप्याह \textendash\ 
भाष्ये \textendash\ दुष्ट इति~॥ अपशब्दप्रयोगे प्रत्यवायप्रतिपादकमिदं.
शिक्षावाक्यम्~॥ यद्यपि तत्र {\qt मन्त्रोहीनः} इति पठितम्, तथाप्य \textendash\ 
त्रैतत्पाठस्य फलमग्रे स्फुटीभविष्यति~॥ 



१ {\qt दुष्टः शब्दः} इति अ. पुस्तके न~। 

२ {\qt नन्विन्द्रशत्रुशब्दस्य} लौकिकत्वे कथं स्वरप्रयुक्तयोर्गुणदोषयोः
प्रसक्तिः, स्वरस्य वेदमात्रविषयकत्वादिति चेन्न~। स्वरविधौ
छन्दोऽधिकाराभावात्~। एतावानेव हि भेदः \textendash\ यच्छन्दसि
त्रैस्वर्यमेकश्रुतिश्च व्यवस्थयाऽऽश्रीयते~। लोके {\qt त्वैच्छिको विकल्पः}
इति श. कौ.~॥ 

३ शास्त्रबोधिताद्विपरीतस्योच्चारणमिति समासः~। इयं छाया तु हेया~। (र.
ना. ) वस्तुतः {\qt छाया हेया} इति कथनमुन्मत्तप्रलपितमेव~। तथा हि \textendash\ {\qt निन्दा च
शास्त्रबोधितविपरीतोच्चारणेन पापसाधनत्वात्} इत्युद्द्योतेनापशब्दस्य
निन्द्यत्वं प्रसाध्यते~। तत्र शास्त्रवोधितं यदुच्चारणं तद्विपरीतं
यदुच्चारणं तद्विषयत्वादपशब्दः पापसाधकः, पापसाधनत्वाच्च निन्दा भवतीति
तदर्थः~। शास्त्रेण शब्दो बोध्यते न 





 स वाग्वज्रो यजमानं हिनस्ति 

 यथेन्द्रशत्रुः खरतोऽपराधात्~॥ इति~॥ 

 दुष्टाञ्छब्दान्मा प्रयुक्ष्महीत्यध्येयं व्याकरणम्~॥ दुष्टः शब्दः~॥ 

 ( प्रदीपः ) दुष्टः शब्द इति~। स्वरेण \textendash\ स्वरतः, आद्या \textendash\ 
दित्त्वात्तसिः~॥ मिथ्याप्रयुक्त इति~। यदर्थप्रतिपादनाय प्रयु \textendash\ 
क्तस्ततोऽर्थान्तरं खरवर्णदोषात्प्रतिपादयन्नाभिमतमर्थमाहेत्यर्थः~। वागेव
वज्रो हिंसकत्वात्~। यथेन्द्रशत्रु२शब्दः खरदोषाद्यजमानं 



 [भाष्ये ] स इति~। स्वरादिदुष्ट इत्यर्थः~॥ 

 [भाष्ये ] तथाचैतादृशऋत्विकृतापराधाद्यजमाननाशज्ञानेन को
नामावैयाकरणमृत्विजं वृणुयादिति ऋत्विक्त्वेन द्रव्यार्जनादिकं कर्तुं
व्याकरणमध्येयमिति फलितम्~॥ तदाह \textendash\ दुष्टानिति भाष्ये~॥ इदं
भाष्यं \textendash\ शिक्षाश्रुत्यादिकं च डित्थादीनां व्युत्पन्नत्वमिति
स्वीकर्तृशाकटायनादिमत इति भावः~॥ 

 [प्रदीपे ] {\qt खरेण वर्णेन} वाऽन्यथाभूतो दुष्टः शब्दः प्रयुक्तो मिथ्या
मृषा~। कुत इत्यत आह \textendash\ न तमर्थमाहेति~। यदर्थ \textendash\ इत्यादिकैयटोक्त एवार्थःइति
कृष्णोक्तव्याख्याने प्रयुक्त इत्यस्य प्रतीयमानविधेयत्वभङ्गापत्तिश्च~। अत
आह \textendash\ [ कैयटे २ यप० ] मिथ्याप्रयुक्त इतीति ~॥ 

 [प्र० ३ यप० ] भाष्यस्थे {\qt न तं इत्यादेरथमाह} \textendash\ कैयटे \textendash\ नाभिमतमिति~॥
एवं च मिथ्याप्रयुक्तत्वमत्र हेतुः~। तत्तत् \textendash\ स्वरवर्णानुपूर्वीकमेव हि
पदादि तत्तदमिति तत्र तत्र प्रयुक्तं सत् तं तमर्थं बोधयति नान्यथेति भावः
~॥ 

 [ प्रदीपे ] वाग्वज्रः इत्यत्र द्वन्द्वे निर्देशानुपपत्तिः,
बहुव्रीहौप्रकृतपरामर्शिसः इत्यादिना समानाधिकरण्यमनुपपन्नमत आह \textendash\ [
कैयटे ४ थप० ] वागेवेति~॥ स वागेव सन् वज्रो भूत्वा
यजमानंमारयेच्चेत्यर्थः~॥ रूपके साधारणधर्म उपस्थितत्वात् हिंसैवेति
सूचयन्नाह \textendash\ हिंसेति~॥ एतेन हिनस्तीति लेट्, न तु लट् ,
वर्तमानत्वानुपयोगादिति सूचितम्~॥ 

 [ प्रदीपे ] तत्रेन्द्रशत्रुशब्दस्यार्थपरत्वे
दृष्टान्तत्वानुपपत्तिरर्थासंगतिश्चात आह \textendash\ [ कैयटे ४ थप० ] शब्द इति~॥
पुरा किल विश्वरूपाख्ये त्वः पुत्रे इन्द्रेण हते कुपितस्त्वष्टा इन्द्रस्य
हन्तारं 



तूच्चारणक्रियेति तस्याः शास्त्रबोधितत्वं तद्विपरीतत्वञ्च यथा
कथञ्चिद्बोध्यमिति छायाभिप्रायः~। यदि च शास्त्रबोधिताद्विपरीतस्य
शब्दस्योच्चारणमित्यर्थोऽवकल्पेत तदा पापसाधनत्वे हेतुत्वं शब्दोच्चारणस्य
स्यात् , तच्च नेष्टम् किन्तु विपरीतोच्चारणस्यैव हेतुत्वमिष्टमिति
छायदर्शितदिशैवावसेयमिति सुधियो विदाङ्कुर्वन्तु~। 

४ इदं चिन्त्यम् \textendash\ मिथ्याप्रयुक्तत्वमत्र हेतुरिति वक्ष्यमाणस्ववचचन \textendash\ 
विरोधात्~। भवदुक्तव्याख्याने वाक्यभेदाच्च~। ( र.ना. ) अत्र र. ना.
पण्डितानां {\qt इदं चिन्त्यम्} इत्युक्तिग्रन्थानवबोधमूलिकैव~। नेयं
छायोक्तिः, किन्तु स्वरेण वर्णेनेत्यारभ्य \textendash\ कैयटोक्त एवार्थ
इत्येतत्पर्यन्तं कृष्णग्रन्थस्यैवायमनुवादः~। अतो
{\qt वक्ष्यमाणस्ववचनविरोधात्} इत्युक्तिः साहसमात्रमेवेति~। 

शास्त्रप्रयोजनाधिकरणम् ] महाभाष्यप्रदीपोद्द्योतव्याख्या छाया~। ३१ 



{\qt हिंसितवानित्यर्थः~। 
इन्द्रस्याभिचारोवृत्रेणारब्धस्तत्र}इन्द्रशत्रुर्व \textendash\ र्धस्व इति
मन्त्र ऊहितः~। तत्रेन्द्रस्य शमयिता शातयिता वा भव \textendash\ इति क्रियाशब्दो५त्र
शत्रुशब्द आश्रितो न तु रूढिशब्दः, तदाश्रयणे हि
बहुव्रीहितत्पुरुषयोरर्थभेदः~। तत्रेन्द्रामित्रत्वे सिद्धे सति {\qt इन्द्रस्य
शत्रुर्भव} इत्यत्रार्थे प्रतिपाद्येsन्तोदात्ते प्रयोक्तव्य आद्युदात्त
ऋत्विजा प्रयुक्तइति \textendash\ अर्थान्तराभिधानादिन्द्र एव वृत्रस्य शातयिता संपन्नः
~। इन्द्रशत्रुत्वस्य च विधेयत्वात् संबोधनविभक्तेरनुवाद्यविषयत्वादिहाभावः
~। यथा \textendash\ राजा भव युध्यस्वेति~॥ ऊह्यमानस्य चामन्त्रत्वात् {\qt यज्ञकर्मणि \textendash\ } इति
जपादिपर्युदासेन मन्त्राणामेकश्रुतिर्विधीयमाना नेह भवति~॥ 



पुत्रान्तरमुत्पिपादयिषुराभिचारिकं यागं कृतवानित्यादि श्रुत्यादौ
प्रसिद्धम्~। तदाह \textendash\ इन्द्रस्येति~॥ 

 [ प्र० कैयटे ५ मप० ] तत्रेन्द्रेति~। याग इत्यर्थः~॥ 

 [प्रदीपे ] विनिगमताविरहादाह \textendash\ [ कै० ६ ष्ठप० ] शातयिता भवेतीति
~। तस्यासिद्धत्वात्~॥ {\qt वर्धस्व} च इत्यपि बोध्यम्~॥
शब्दोऽत्र \textendash\ ऊहिते वाक्ये~॥ हि \textendash\ यतः~॥ रर्थाभेद इति~। अर्थभेदो न
स्यादित्यर्थः~॥ 

 [ प्रदीपे ] क्रियाशब्दाश्रयणे त्वर्धभेद इत्याह \textendash\ तत्रेन्द्रेति
[कै० ८ मप० ] तत्र \textendash\ रहितवाक्ये~। प्रयोक्तव्यादावस्यान्वयः~॥
नन्विन्द्रसपत्नत्वविशिष्ट एव कुतो न विधेयोऽत आह \textendash\ सिद्धे इति~॥
सतिसप्तम्या हेतुत्वं \textendash\ २अस्यातदर्थप्रतिपादनसूचने~॥ शत्रुः \textendash\ शातयिता~॥
भाद्युदात्त इति~। व्युत्पत्तिपक्षे {\qt ॠज्रेन्द्राग्र \textendash\ } इत्यादिना
तस्य रनन्तत्वस्योक्तत्वान्नित्वादाधुदात्तत्वस्~। सिद्धान्तमते तु
{\qt ग्रामादीनां च} इति~। तदेव च बहुव्रीहौ पूर्वपदप्रकृतिस्वरेणावतिष्ठत
इति भावः~॥ [१० मप०] अर्थान्तरेति~। बहुब्रीहिगम्येत्यादिः~॥
[१२ शप० ] राजेति~। तस्य सिद्धत्वे तु राजन् युध्यस्वेति
भवत्येवेति भावः~॥ 

 [उद्द्योते ] तस इति~। तसिल इत्यथः~॥ स्वरेणेतीति~। 
{\qt इतराभ्योऽपि दृश्यन्ते} इत्युत्तेस्तृतीयार्थे इति भावः~। एवं
वर्णतः इत्यत्रापि बोध्यम्~॥ हीनशब्दस्यापाठाद् {\qt हीयमान \textendash\ 
इत्यस्याप्राप्तेराह} \textendash\ आाद्यादीति~॥ 

[ उ० २ यप० ] तमित्यस्येति~। यत्तदोनित्यसंबन्धादिति भावः~॥
अध्याहाराभावजलाघवमाह \textendash\ मिथ्येति~। {\qt प्रतिपादयन्} इत्यन्तमिति भावः~॥ 





स्पष्टम्~। २ अस्य इन्द्रामित्रत्वस्यसिद्धस्य तदर्थप्रतिपादनसूचने
हेतुत्वंसतिसप्तम्या बोध्यते इति संबन्धः~। (र. ना. ) वस्तुतोऽत्र
\textendash\ अस्या \textendash\ तदर्थप्रतिपादनसूचने? इत्येव पाठः समुपलभ्यते~। प्रथममुद्रणावसरे
म.म.शिवदत्तेनापि अस्यातदर्थप्रतिपादनसूचने (?)इत्येव पाठः प्रदर्शितः,तस्य
दुरूहत्वाच्च शङ्काचिह्नमपि तत्र मुद्रितम्~। पुनर्मुद्रणावसरे र.
ना.पण्डितेन शङ्काचिह्नदर्शनभ्रान्तेन {\qt अस्य तदर्थ \textendash\ } इति नञ्
रहितपाठमुल्लिख्य ग्रन्थसम्बन्धप्रदर्शनाय प्रयतितम्~। तदेत \textendash\ 



 उद्द्योतः) असर्वनामत्वात्तसोऽप्राप्तेराह \textendash\ स्वरेणेति~। आद्या \textendash\ 
दित्यादिति च~। {\qt तं} इत्यस्य यच्छब्दापेक्षित्वादाह \textendash\ यदर्थेति ~॥
{\qt मिथ्याप्रयुक्तः?} इत्यस्य विवरणं \textendash\ ततोऽर्थान्तरमिति~॥न केवलंतदर्थं
नाह \textendash\ इत्येतावत्, किंतु प्रत्यवायोत्पादनद्वारा हिंसकोऽपीत्याह \textendash\ 
वागेवेति~। यथा \textendash\ अध्वर्युकृताद्धोमाद्यजमाने धर्मोत्पत्तिः,
एवंतत्कृतापशब्दप्रयोगात्तस्मिन् प्रत्यवाय इत्याशयः~॥ तत्र
दृष्टान्तमाह \textendash\ यथेति~॥ १वृत्रेणेति \textendash\ द्देतौ तृतीया, फलमपीह हेतुः~॥
मन्त्र इति~। न चोहितस्य कथं मन्त्रत्वमिति वाच्यम्~। मन्त्रशब्दस्य
बोधकपरत्वात्~। अत एव {\qt मन्त्रो हीनः स्वरतः} इति प्रसिद्धमपि
शिक्षापाठं विहाय भाष्ये \textendash\ {\qt दुष्टः शब्दः} इति पठितं तत्रत्यमन्त्र



 [उ० ४ र्थप० ] उद्द्योते \textendash\ तदर्थमिति \textendash\ कर्मधारयः~॥ प्रत्येति~। 
यजमानस्येत्यादिः~॥ 

 [ उद्दयोते ] नन्वन्यकृतापराधेन कथमन्यनाशः~। किं चात्र नाशो न
यजमानस्य किंतु तत्पुत्रस्येति वैषम्यम् ; अत आह \textendash\ [ उ० ५ मप० ] यथेति~। 
अत्र चेदमेवं वचनं मानम्~। तथा च न तत्प्रतिपादने तात्पर्यम् , किंतु
प्रत्यवायोत्पत्तौ~। सा च स्वस्य नाशिका स्वीयस्य वा~। सोऽपि स्वस्यैव
नाश इति भावः~॥[६ष्ठप०] तत्र \textendash\ हिंसायाम्~॥ 

 [ उद्द्योते ] तत्र त्वष्टुः कर्तृत्वेन वृत्रेणेत्यनुपपत्तेराह \textendash\ [
उं० ७ मप० ] वृत्रेणेतीति~॥ द्देतुत्वस्यापि वस्तुतो
बाधादाह \textendash\ फलमिति~॥ तदप्युक्तप्रयोगहेतुत्वेन विवक्षितमित्यर्थः~॥ तथा च
वृत्रोत्पत्त्यर्थं त्वष्ट्रा स आरब्ध इत्यर्थो बोध्यः~॥ 

 [ उद्द्योते ] {\qt मन्त्रे स ऊहितः} इति दण्ड्युक्तव्याख्याऽयुक्ता,
तत्र तस्य तत्त्वाभावात् \textendash\ यागे ऊहितत्वात्~। तदेतद् ध्वनयन्नाह \textendash\ [ उ० ८
मप० ] न चोहीति~। सर्वस्येत्यर्थः~। {\qt अनाम्नाता अमन्त्राः}
इत्याद्यापस्तम्बाद्युक्तेरिति भावः~॥ यथपि लोके एकावयवा \textendash\ 
धिक्यविनाशयोरवयविप्रत्यभिज्ञानवदत्राप्येकावयवाभावे पदान्तरप्रक्षेपे
चाध्ययनकालावगतमन्त्ररूपत्वप्रत्यभिज्ञानात्प्रक्षिप्तपदमात्रस्यैवामन्त्रत्वं
न तु सर्वस्येति प्रागुक्तम्, तथाप्येकदेशोहस्थले तथा संभवेऽपि सर्वोहस्थले
न संभवतीत्याशयेनाह \textendash\ [ ८ मप० ] मन्त्रेति~॥ 



त्प्रामादिकम्~। तादृशपाठस्यासत्वेन कुड्याभावे चित्राभावात्~। ग्रन्थस्य
चायमभिप्रायःअस्य \textendash\ इन्द्रशत्रुर्वर्धस्वेत्यस्य इन्द्रस्य
शत्रुर्भवेत्यतो भिन्नार्थप्रतिपादनसूचने सतिसप्तम्या हेतुत्वमिति~। 

 ३ एवकारोऽप्यर्थः~। स्पष्टं चेदं पूर्वोत्तरमीमांसयोः~। ( र. ना.)
वस्तुतःअत्र च \textendash\ {\qt तत्कृतापशब्दप्रयोगात्तस्मिन् प्रत्यवायः} इत्यत्र \textendash\ च
इदमेव \textendash\ {\qt स वाग्वज्रो यजमानं हिनस्ति} इतिवचनमेव प्रमाणमिति \textendash\ 
छायाऽभिप्रायः~। अत्र पूर्वोत्तरमीमांसाविदामभिप्राय सुधियो
विभावयन्तु~। 

३२ उद्द्योतपरिवृतप्रदीपप्रकाशितमहाभाष्ये~। [१ अ. १ पा. १ पस्पशाह्निके




शब्दस्य शब्दमात्रपरता \textendash\ इति सूचयितुम्~॥ शत्रुशब्दस्यामित्रपर्यायत्वे
बहुब्रीहितत्पुरुषयोरविशेषोऽत आह \textendash\ तत्रेन्द्रस्येति~। शदेर्ण्यन्ता \textendash\ 
दौणादिकः क्रुन्~। प्रशादिगणे निपातनाद्ह्रस्वत्वम्~। शमेस्तु तत्वम्~॥
संबोधनविभक्त्यभावायाह \textendash\ इन्द्रशत्रुत्वस्येति~॥ नन्वत्र {\qt यज्ञ \textendash\ 
कर्मणि \textendash\ } इत्येकश्रुतौ प्राप्तायां पूर्वपदप्रकृतिस्वरकरणं स्वरापराध इति
वक्तुं युक्तम्~। न च तथैवास्तु, शतपथ१ब्राह्मणतद्भाष्यविरोधात्~। तत्र हि
आद्युदात्तत्वकरणमेव स्वरापराध उक्त इत्यादि
स्पष्टमस्मत्कृतैकश्रुत्यवादेऽत आह \textendash\ ऊह्यमानस्य चेति~। 
तदुक्तम् \textendash\ {\qt अनाम्नातेष्वमन्त्रत्वम्} ( पू० मी० २~। १~। ३४ ) इति~। न
चास्यामन्त्रत्वेलौकिकशब्देन क्रियमाणस्य कर्मणः कथं फलजनकत्वमिति
वाच्यम्~। {\qt सूर्याय त्वा} इत्यादेर्लौकिकघटितस्य ऊहविधायकशास्त्रबलेन
फलजनकत्ववदुपपत्तेरित्याहुः~॥ 

 (भाष्यम्) 

२यदधीतम् \textendash\ 

यदधीतमविज्ञातं निगदेनैव शब्द्यते~। 

अनग्नाविव शुष्कैधो न तज्ज्वलति कर्हिचित्~॥



 [उ० ९मप०] अत एव \textendash\ अमन्त्रत्वादेव~॥ [१ ०मप० ] पठित \textendash\ मिति~। 
व्यक्तमपीति शेषः~। अन्यथा दृष्टान्तासंगतिरेव स्यादिति भावः~॥ [१० मप०
] तत्रत्येति~। शिक्षापाठ्येत्यर्थः~॥ शब्देति~। बोधकेत्यर्थः~॥
मात्रशब्दः कार्त्स्न्ये~। यद्वाऽवधारणे~। [११ शप० ] स्यामित्रेति
~। सपत्नेत्यर्थः~॥ [१२ शप० ] रविशेष इति~। तयोरर्थभेदो न
स्यादित्यर्थः~॥ तत्रेन्द्रेति~। आभिचारिकयाग इत्यर्थः~॥ 

 [उद्दयोते ] ऐच्छिकत्वादाह \textendash\ [ उ० १३ शप० ] शमेरिति~॥ तत्वमिति~। 
प्रज्ञादिगणे निपातनादित्यस्यानुषङ्गः~। प्रत्ययविधानाय प्रकृतीनां
तत्रानुवादादिति भावः~॥ 

 [उ० १४ शप० ] नन्वत्रेति~। ऊहिते प्रागुक्तवाक्ये \textendash\ इत्यर्थः~॥
[१८ शप० ] तदुक्तमिति~। जैमिनिना भेदलक्षणे इति शेषः~॥
{\qt ऊह्यमानस्य मन्त्रत्वम् \textendash\ उत नः} इति संशय्य मन्त्रैकवाक्यत्वेन
मध्यनिवेशान्मन्त्रत्वमिति पूर्वपक्षोक्तौ \textendash\ उक्तम् \textendash\ {\qt अना \textendash\ } इति~। उक्तहेतोर्न
तत्त्वकारणत्वं किंत्वभियुक्तप्रसिद्धेः~॥ 

 [उ० २१ शप० ] कत्ववदिति~। शिक्षावचनं श्रुत्यादिकं~। चात्रापि
प्रमाणमिति भावः~॥ 

 [भाष्ये ] शब्द्यते \textendash\ पुनःपुनरुच्चार्यते~। भेर्येव तेन पुरुषेण ध्वनि \textendash\ 
मात्रमेवाभिव्यज्यत इति यावत्~॥ अनग्नाविवेति~। बहुब्रीहिपक्षे भस्मप्रदेश
इवेत्यर्थः~। नञा विरोधार्थकेन तत्पुरुषे जलप्रदेश इवेत्यर्थः~। 
प्रक्षिप्तमिति शेषः~॥



१ {\qt अथ यदब्रवीत् \textendash\ इन्द्रशत्रुर्वर्धस्वेति तस्मादु हैनमिन्द्र एव जघान~। 
अथ यद्ध शश्वदवक्ष्यत \textendash\ इन्द्रस्य शत्रुर्वर्धस्वेति शश्वदुह स
इन्द्रमेवाहनिष्यत्~॥} इति शतपथै 

१ का० ५ प्र० २ ब्राह्मणम्~॥ 

स त्वष्टा ग्रहप्रवारसमये इन्द्रशत्रुरितिपूर्वपदाद्युदात्ततया इन्द्रः
शत्रुर्यस्येति बहुव्रीहिसमासं कृत्वा मन्त्रं प्रयुक्तवानतस्तमिन्द्रो
जघान~। 





तस्मादनर्थकं माधिगीष्महीत्यध्येयं व्याकरणम्~॥

 यदधीतम्~॥ 

 (प्रदीपः) अविज्ञातमिति~। अविदितसुबादिसंस्कारत्वात्, अर्थापरिज्ञानाद्वा
~। निगदेनेति~। पाठमात्रेण~। न तज्जवलतीति~। निष्फलं भवति~॥
अनर्थकमिति~। निष्प्रयोजनम्~॥ 

 ( उद्ष्योतः ) नतु \textendash\ अधीतमविज्ञातं \textendash\ इति विरुद्धमत आह \textendash\ अविदितेति~॥ शुष्कैध
इति~। सान्त्क्लीबैधःशब्देनादन्तपुंलिङ्गेन वा निर्वाह्यम्~॥ निरुक्ते तु
यद्गृहीतमविज्ञातं इति पठ्यते~। तत्र गृहीतं शब्दतः, आअविज्ञातमर्थत इति
बोध्यम्~॥ भाष्येतस्मादनर्थकमिति~। अनेन निन्दार्थवादेन
निष्प्रयोजनाध्ययनानौचित्येन निष्प्रयोजनं नाध्येयमिति निषेधकल्पनादिति
भावः~। अतो निषिद्धमध्ययनं मा भूदिति व्याकरणमध्येयमिति तात्पर्यम्~॥ 

 ( भाष्यम्) 

२यस्तु प्रयुङ्क्ते \textendash\ 

यस्तु प्रयुङ्क्तं कुशलो विशेषे 

शब्दान्यथावद्व्यवहारकाले~। 



 [भाष्ये ] न तज्ज्वलतीति~। न दीप्यते न प्रकाशते \textendash\ इत्यर्थः~॥ 

 [ भाष्ये ] अधीतस्य स्वरूपेणाविज्ञातत्वं न संभवतीति तदनर्थकपरमिति
बोधयन्नुपसंहरति \textendash\ भाष्ये \textendash\ तस्मादिति~। 

 [ प्रदीपे ] कैयटे \textendash\ अविदितेति वबहुब्रीहिः~॥ 

{\qt स्थाणुरयं भारहारः किलाभूदधीत्य वेदं न विजानातिं योऽर्थम् ! योऽर्थज्ञ ४इत्सकलं भद्रमश्नुते नाकमेति ज्ञानविधूत \textendash\ पाप्मा~॥}

 इति श्रुतिमन्त्राद्यनुरोधेनाह \textendash\ अर्थापरीति~॥ 

 [ प्रदीपे ] तत्फलितमाह \textendash\ कैयटे \textendash\ निष्फलमिति~॥ 

 [ प्रदीपे ] अनर्थकमित्यस्याभिधेयशून्यमित्त्यर्थनिरासायाह \textendash\ कै \textendash\ 
यटे \textendash\ निष्प्रयोजेति~। दृष्टान्तबलेनैवमेव लाभादिति भावः~॥ 

 [ उद्दयोते ] विनिगमनाविरहादाह \textendash\ [ उ० २ यप० ] अदन्तेति~॥ इदं
श्रुतिस्थमिवेति ध्वनयन्नाह \textendash\ [ ३ यप० ] {\qt निरुक्ते त्विति~॥}
पाठभेदेऽप्यर्थस्य साम्यमित्याह \textendash\ तत्रेति~। तत्रापीत्यर्थः~॥ 

 [ उद्दयोते ] {\qt तस्मात्} इत्यस्याधिकस्य
पूर्वत्रानुक्तस्यार्थमाह \textendash\ [उ० ५ मप० ] अनेनेति~। {\qt यदधीतं}
इत्यनेनेत्यर्थः~। अनेन {\qt दुष्टः शब्दः} इत्यत्राप्येवं रीतिः सूचिता~॥ 

 [ उद्वयोते ] अत एव यथाश्रुतानुपपत्तेराह \textendash\ [ ७ मप० ] अत इति~। 
प्रकृत्यादिज्ञानेन पदादिभागार्थज्ञानेनानुष्ठाने तथा प्रागुक्तं फलमिति
परमतात्पर्यम्~॥



यद्यसौ अन्तोदात्ततया तत्पुरुषसमासेन वा इन्द्रस्य शत्रुरिति
व्यस्तनिर्देशेन वा ब्रूयात्तदा निश्चितमेव स इन्द्रं हन्यात् इति
सायणभाष्यम्~॥ 

२ अ. पुस्तके प्रतीकपाठो न दृश्यते~। 

३ मन्त्रैकवाक्यत्वहेतोर्न मन्त्रत्वज्ञापकत्वं किन्तु अभियुक्तप्रसिद्धे \textendash\ 
रित्यर्थः~। ( र. ना. ) 

४ इदव्ययमेवकारार्थम्~। स इवेत्यर्थः~। (र. ना. ) 

शास्त्रप्रयोजनाधिकरणम् ] महाभाष्यप्रदीपोद्दयोतव्याख्या छाया~। ३३ 



सोऽनन्तमाप्नोति जयं परत्र 

 वाग्योगवित् दुष्यति चापशब्दैः~॥ 

 कः ? 

 वाग्योगविदेव~॥ 

 कुत एतत्? 

 यो हि शब्दान् जानाति. अपशब्दानप्यसौ जा नाति~। यथैव हि शब्दज्ञाने
धर्मः, एवमपशध्दज्ञानेऽप्यधर्मः~। 

 अथवा भूयानधर्मः प्राप्नोति~। भूयांसोऽपशब्दाः,अल्पीयांसः शब्दा
इति१~। एकैकस्य शब्दस्य बहवोऽपभ्रंशाः~। तद्यथा \textendash\ गौरित्यस्य२ गावी
गोणी गोता ३गोपोतलिकेत्येवपादयोऽपभ्रंशाः~। 

 अथ योऽवाग्योगवित् अज्ञानं तस्य शरणम्~॥ 

 ( प्रदीपः ) यस्तु प्रयुङ्क्ते इति~। अनेनाभ्युदयहेतुत्वं
व्याकरणाध्ययनस्य दर्शयति ~॥विशेष इति~। स एव शब्दः क्वचिदर्थे
केनचिन्निमित्तेन प्रयुक्तः साधुः, अन्यथा त्वसाधुः~। यथा \textendash\ ४अश्वे \textendash\ अस्वशब्दो
धनाभावनिमित्तकः साधुः, जातिनिमित्तको \textendash\ 



 [भाष्ये ] सोऽनन्तमिति~॥ऐहिकापेक्षयाऽतिशयितत्वमनेनोच्यते
~। अन्तवत्त्वं त्वस्त्येव, कर्मजन्यस्य सर्वस्यानित्यत्वात्~॥
जयमित्ति~। उत्कर्षमित्यर्थः~। लक्षणया सुविशेषमिति यावत्~॥
परत्रपरलोके~॥ 

 [ भाष्ये \textendash\ [ ९ मप० ] ज्ञानेsप्येति~। ज्ञानेऽधर्मोsपीत्यर्थः~॥ 

 [भाष्ये ] तादृशाधर्मप्राप्तौ हेतुमाह \textendash\ भाष्ये \textendash\ [ ११ शप० ] भूयांस
इति~। इतिर्हेतौ~। भूयांसो ह्यपेति पाठे हि \textendash\ हेतौ~॥ 

 [ भाष्ये ] नन्विदानींतनव्यवहाराय संकेतितापशब्दानां कथं
नित्यसुशब्देभ्यौ भूयस्त्वम् , अतस्तदुपपादयति \textendash\ [भा० १२ शप०]
एकैकस्येति~। देशकालपुरुषतदीयशक्तिवैकल्यानां भूयस्त्वादिति भावः~॥ 

 [भा० १५ शप० ] अथशब्द आनन्तर्ये~। एवमग्रेऽपि~॥ विदुष एव
सर्वत्राधिकारदर्शनादाह \textendash\ अज्ञानमिति~। निषेधशास्त्राज्ञानमित्यर्थः~॥ 

 [प्रदीपे] तदेवाह \textendash\ कैयटे \textendash\ अनेनेति~। पद्येनेत्यर्धः~॥ 

 [प्रदीषे] अत्र रत्नकृत् \textendash\ अत्र स्वार्थे ईयसुन् {\qt अल्पाचतरम्} इत्यत्र
तरबिव~॥ एतेन कैयटोक्तं चिन्त्यम् \textendash\ इति~। यद्यपि आति \textendash\ शायनिकप्रकरणे एव
{\qt तादी घः} इत्यादौ वाच्ये गुरुसूत्रकरणस्य {\qt गरीयांसं यत्नम्}
इत्यादिभाष्यानुरोधाद्योगापेक्षज्ञापकत्वाङ्गी \textendash\ कारेणेदं समूलम् , तथापि
तस्यागतिकगतित्वेन प्रकृते पूर्वपक्षिणा साधुशब्दानां
बहुत्वस्यासाधुशब्दानामल्पत्वस्य प्रक्रान्तत्वेन तत्ख \textendash\ ण्डकेऽत्र
परमतापेक्षया प्रकर्षस्य तात्पर्यविषयतया तथैवोपपत्तौ तदङ्गीकारोsत्र
युक्तः~। अत एवानिष्टप्रयोगो नापाद्यः, इति तदुक्त \textendash\ 



१ प. पुस्तके इतिशब्दो न~। एकैकस्य हि शब्दस्य इति च. छ. क. पाठः~। २
{\qt त्यस्य शब्दस्य गावी} इति क. च. छ. पाठः~। 

३ {\qt पोतलिकेत्यादयो} बहवोपभ्रंशाः इति च. छ.~। ४ अश्वे इति~। अश्वशब्दे
प्रयोक्तव्येऽस्वशब्दः प्रयुज्यमानो यदि जातिनिमित्तः प्रयुज्येत 

५ प्र०पा० 





ऽसाधुः~। गति च गोणीशब्दः साधर्म्यात्प्रयुक्तः साधुः, जाति \textendash\ 
प्रयुक्तस्त्वसाधु:~॥ क इति~। वाग्योगविदः श्रुतत्वाद्दोषदर्शनाच्च
प्रश्नः~॥ प्रष्टैव परमतमाशङ्काह \textendash\ वाग्योगविदेवेति~॥
एवमपशब्दज्ञानेऽपीति~। यथा श्लैष्मिकद्रव्यसेवया श्लैष्मिक \textendash\ 
व्याधिसम्भवः, तद्विपरीतसेवया त्वारोग्यं तथाsत्रापि यथोक्तं न्याय्यमिति
भावः~॥ भूयांसोsल्पीयांस इति~। परमतापेक्षया प्रकर्षप्रत्ययः~। यदि
मन्यसे \textendash\ बहव: शब्दाः अल्पेऽपशब्दाः, अङ्गभूयस्त्वाच्च
फलभूयस्त्वमिति~। तन्न~। यन्माद्भूयांसोsपशब्दाः, अल्पीयांसः शब्दाः~॥
अज्ञानमिति~। तथा न् तिरश्चां ब्रह्म \textendash\ हत्यादिफलाभावः~॥ 

 (उ०){\qt एकः पूर्वपरयोः} इत्यत्र भाष्यपठित एकः शब्दः \textendash\ 
इत्यादिश्रुतिमूलकमाह \textendash\ (भाष्ये) \textendash\ यस्त्विति~। अभ्युदयहेतुत्वम् \textendash\ 
अदृष्टद्वारा पुरुषार्थसाधनत्वम्~। कुशलः \textendash\ अधीतव्याकरणः~। लक्षण \textendash\ 
स्मरणपूर्वकं यः प्रयुङ्क्ते \textendash\ इति फलति~॥ विशेष इति~। अर्थविशेष इत्यर्थः~। 
तदाह \textendash\ स एवेति~॥ साधर्म्यात् \textendash\ साधर्म्यमूलकादभेदा \textendash\ रोपात्~॥
भाष्ये \textendash\ यथावत् \textendash\ अम्बकृतादिदोषरहितम्~। वाचो योगः \textendash\ 
प्रकृतिप्रत्ययविभागेनार्थविशेषपरत्वं, तद्वेत्तीति \textendash\ वाग्योगवित्~॥
श्रुतत्वादिति~। एकत्व जयाप्तिदोषयोश्चकारेण समुच्चयप्रतिपादनाच्चेत्यपि



मयुक्तमेवेति दिक् तदभिप्रेत्याह; [कैयटे१०मप०] परमतेति[ प्र० १२
शप० कैयटे ( भाष्यस्थ \textendash\ ) {\qt इत्येवमादि}पदार्धमाह \textendash\ यस्मादिति~॥ तथा च
बह्वनर्थप्रतिविधानेन तदङ्गीकारेण वाऽल्पफलसाधनं प्रेक्षावतां न
युक्तमन्यत्रादृष्टत्वादिति तात्पर्यम्~॥ 

 [ प्रदीपे ] तस्य तदभावे दृष्टान्तमाह \textendash\ [ कै० १३ शप० ] यथा चेति
~॥ तस्मान्नित्याध्ययनमेवास्य युक्तम्, न तु शब्दव्यु \textendash\ त्पादनार्थत्वं,
तस्य प्रत्युतानर्थहेतुत्वादिति भावः~॥ 

 [उद्द्योते] एक इति~। {\qt एवं निषेधविधिविषयफलान्युक्त्वा
काम्यविधिविषयं फल्ं कथयन्} इत्यादिः~। मूलकमिति~। 
कात्यायनोक्तभ्राजाख्यश्लोकान्तर्गतं पद्यमिति शेषः~॥ 

 [उद्द्योते] प्रागुक्तैकवाक्यत्वायाह \textendash\ [उ० ३ यप० ] अदृष्टेति~॥
[४ र्थप०] अर्थविशेष इति~। शक्यलक्ष्यद्योत्यान्यतमे इत्यर्थः~॥
एवंचैवात्रार्थे निमित्तविशेषैस्तत्साध्वसाधुत्वज्ञानं व्याकरणैकशरणम्~॥
अत एवाह हरिः \textendash\ 

 अर्थप्रवृत्तितत्त्वानां ७शब्दा एव निबन्धनम्~। 

 तत्त्वावबोधः शब्दानां नास्ति व्याकरणादृते~॥ इति~॥
तावताऽप्यनिर्वाहादाह \textendash\ [ ५ मप० ] साधर्म्यमूलकेति~॥ 

 [उद्द्योते] वाग्योगविदित्यस्यार्थमाह \textendash\ [उ० ६ ष्टप० ] वाचो योग
इति~॥ एतेन \textendash\ वाचः शब्दस्य योगं घटनां वेत्तीति तथा \textendash\ इति कृष्णोक्तमपास्तम्
~। अर्थविशेषपुरस्कारेणैव शब्द \textendash\ साधुत्वस्यानेन प्रतिपाद्यत्वेन तज्ज्ञातुरेव
तत्त्वात्~। हेत्वन्तरमाह \textendash\ [ ८ मप० ] एकत्रेति~। अत एवान्ययोगव्यवच्छेदक
एवशब्दः प्रयुक्तो भाष्ये~। तथा च तस्यैव कर्तृत्वमिति भावः~। 



ततोऽसाधुः, धनाभावनिमित्तकः साधुरित्यर्थः~। ५ प्रकर्षे प्रत्ययः इति च.
क. पाठः~। ६ {\qt यथा च} इति मुद्रितपाठः~। ७ अर्थे शब्दस्य या
प्रवृत्तिस्तद्धेतुभूतजात्यादितत्त्वानां शब्दा एव वाचकत्वेनालम्बना \textendash\ 
नीत्यर्थः~। (र.ना.) 

३४ उद्द्योतपरिवृतप्रदीपप्रकाशितमहाभाष्ये \textendash\ 

 [ १ अ. १ पा. १ पस्पशाह्निके



बोध्यम्~॥ दोषदर्शनादिति~। शास्त्रस्याधर्महेतुत्वं दोष इत्यर्थः~॥
वादिप्रतिवादिपूर्वपक्षोत्तरपरतया व्याख्यानेsसामञ्जस्यं मत्वाऽऽह \textendash\ 
प्रष्टैवेति~॥ भाष्ये \textendash\ कुत एतदिति~। वाग्योगविदो दोषः कुत इत्यर्थः~॥ ननु
साधुज्ञानस्य यथा धर्मजनकता शास्त्रोक्ता न तथा \textendash\ ऽसाधुज्ञानस्याधर्मजनकता
शास्त्रोक्तेत्यत आह \textendash\ यथेति~। एवञ्च अपशब्दज्ञानं \textendash\ अधर्मसाधनं,
धर्महेतुशब्दज्ञानविरोधित्वात् ; यद्यद्विरोधि तत्तद्विरोधिफलकं,
यथा \textendash\ श्षैष्मिकद्रव्यसेवा \textendash\ इत्यनुमानं प्रमाणमिति भावः~। एवं च
शास्त्रस्याधर्महेतुतया तत्र प्रवृत्तिर्न स्यादिति नेदं वचो
व्याकरणाध्ययनानुष्ठापकं स्यादिति तात्पर्यम्~॥ नन्वत्वधर्मोऽपि तथापि
धर्माधिक्यायाध्येयं व्याकरणमत आह \textendash\ भाष्ये \textendash\ अथवेति~॥ अङ्गभूयस्त्वादिति~। 
अङ्गानां धर्मजनकानां भूय \textendash\ स्त्वादित्यर्थः~। एवं च यज्ञादेः
पापजनकत्वैऽप्यधिकधर्मजनकत्वाद्यथा 



{\qt दोषदर्शनात्} इति कैयटस्य वाग्योगविदोsपशब्दज्ञानप्रयुक्तदोष \textendash\ 
दर्शनादित्यर्थो न, {\qt प्रयोगकृतो दोषो न ज्ञानकृतः} इत्यस्य वक्ष्य \textendash\ 
माणत्वात् १तथा सति कोट्यन्तरानुत्थापकतया संशयस्याभावेन प्रश्नाभावापत्तेः
, २त्रिभिर्मिलित्वा वाग्योगविद एव कर्तृत्वापत्तेश्च~। अत एव मध्ये
३शेषपूरणम्~। तेनोक्तहेर्तुद्वय४नैककोट्युपस्थितिरन्येनान्य५कोट्युपस्थितिः
~। तदाह \textendash\ [९ मप०] शास्त्रस्येति~॥ हेतुत्वं दोष इति~। 
वाग्योगविद इति बोध्यम्~॥ [१० मप०] सामञ्जस्यमितिउत्तरस्य
दुष्टत्वेनासिद्धान्तत्वादिति भावः~॥ [११ शप० ] दोष इति~। 
कर्तृत्वाङ्गीकारे तस्याधर्महेतुत्वापत्तिरूपदोष इत्यर्थः~॥ तथा च
तददर्शनान्न तस्य कर्तृत्वम्, किं त्ववाग्योगविद इति भावः~॥ एवज्च
कैयटस्थत्वेन प्रश्न सर्वेषां समुच्चयः~॥ 

 [उ० १२ शप०] ननु साध्विति~। एवं च् {\qt यो हि? इत्यादि} \textendash\ 
ग्रन्थासंगतिरिति भावः~॥ दृष्टान्तसूचितमर्थमाह \textendash\ [ १३ शप० ] एवं चेति
~। दृष्टान्तसत्त्वे चेत्यर्थ:~॥ विशेषव्याप्त्यसंभवादाह \textendash\ [ १४ शप० ]
यद्यदिति~॥ यद्यपि तद्वलान्नापशब्दोद्देशेन शास्त्रप्रवृत्तिस्तथापि
साधुज्ञानव्यापारेणासाधुज्ञानस्यापि नान्तरीयकतया जननादामिक्षायां
वाजिनस्येवैकत्रोभयोरविरोधादिति भावः~॥ तत्फलितमाह \textendash\ [ १६ शप० ] एवं
चेति~॥ तस्यानुमानप्रमाणकत्वे चेत्यर्थः~॥ [ १५७ शप० ]
नन्वस्त्वधर्मोऽपीति~। तस्यैवेति भावः~॥ धर्मेति~। धर्माशेत्यर्थः~॥ तथाच
नान्तरीयाल्पदोषस्तेन सुनिरस इति भावः~॥ 

 [ उद्दयोते ] अङ्गत्वमुपकारकत्वमित्याशयेनाह \textendash\ [ १९ शप० ] 



१ दोषस्यापशब्दज्ञानप्रयुक्तत्वस्वीकारे सतीत्यर्थः~। ( र. ना. ) 

२ प्रयुङ्के आप्नोति दुष्यति इत्येतैरिति भावः~। ( र. ना. ) 

३ प्रयुङ्के दुष्यतीत्यनयोभिन्नकर्तुकत्वादेव मध्ये वाग्योगविदिति
शेषपूरणमित्यर्थः~। (र. ना. ) श्रुतत्व \textendash\ दोषदर्शनरूपहेतुद्वयेन
जयाप्तिर्वाग्योगविदो दोषश्चावाग्योगविदः , एकत्रजयाप्तिदोषयोश्चकारेण
समुच्चयप्रतिपादन \textendash\ हेतुना दोषोऽपिवाग्योविद एवेति कोटिद्वयम्~। ५
अन्येन \textendash\ उद्द्योतोक्तसमुच्चयहेतुना~। ६ स्त्रीणामपि पुरुषाणामिवाधिकारेsपि
पतिवत्नीनां पतिसंनिधाने गुरुसंनिधौ शिष्याणामिव राजसंनिधौ सचिवानामिव च
नाधिकार इत्यपि तत्र निर्णीतमिति दाधिमथाः, पतिवत्यादीनां पत्यादिसंनिधाने
कर्मण्यधिकाराभावेऽपि अपशब्दपरिवर्जनादावधिकारोऽस्त्येव~। ततश्च
पुरुषपदमुपलक्षणमित्येव वरमिति(र.ना.)





तत्र प्रवृत्तिस्तथा प्रकृतेऽपीति भावः~॥ अल्पीयांसः शब्दा इति~। एवं च
श्वदृतिप्रक्षिप्तक्षीरवदनुपादेयं तद्विज्ञानमिति भावः~॥ नन्वेवं
सामर्थ्यादवाग्योगविदेव {\qt दुष्यति} इत्यत्र कर्ताऽस्त्वत आह \textendash\ भाष्ये \textendash\ अथ
योऽवाग्योगविदिति~॥ 

 (बाधकभाष्यम् ) 

 वि॒षम उपन्यासः~। नात्यन्तायाज्ञानं शरणं भवितु \textendash\ मर्हति~। यो ह्यजानन् वै
ब्राह्मणं हन्यात् सुरां वा पिबेत् सोऽपि मन्ये पतितः स्यात्~॥ 

 ( प्रदीपः ) नात्यन्तायेति~। पुरुषाणां विधिनिषेधयोर \textendash\ 
धिकारात्तत्परिज्ञाने प्रयत्नस्य न्याय्यत्वात्~॥ 

 (उ० ) अत्यन्तमित्यर्थे {\qt अत्यन्ताय} इति अव्ययम्~। पुरुषा \textendash\ णामिति~। 
मनुष्याणामित्यर्थः~। तिरश्चामत्यन्तायोग्यत्वादस्त्वज्ञानं 



धर्मेति~॥ साधुशब्दानामित्यर्थ:~॥ अयं न्यायः कुत्र दृष्टोऽत आह \textendash\ एवं
चेति~। उक्तन्यायाङ्गीकारे चेत्यर्थः~॥ [२० शप०] पापेति~। 
हिंसाकृतेत्यादिः~॥ तेनैव तत्क्षालनं फलबाह्ुल्यमेव वेति भावः~॥ 

 [ उद्दयोते ] तत्फलितमाह \textendash\ [ उ० २२ शप० ] एवंचेति~। तथा
व्यत्यये चेत्यर्थः~॥
अत्रोक्तदृष्टान्तविरुद्धदृष्टान्तमाह \textendash\ श्वदृतीति~॥
श्वसंबन्धिचर्मपात्रेत्यथः~॥ यथा तदन्तर्गतत्वं क्षीरस्य तथा
बह्वपशब्दव्याप्तत्वमेकस्य सुशब्दस्येति भावः~॥ [ २२ शप० ] तदिति
~। साधुशब्देत्यर्थः~॥ 

 [उ० २२ शप० ] नन्वेवम् \textendash\ वाग्योगविदोऽधर्मबाहुल्येन
तद्विज्ञानस्यानुपादेयत्वेनास्य वचनस्यैतदननुष्ठापकत्वापत्तौ~॥ [२३
शप०] कर्तेति~। तस्य तत्सत्त्वेऽस्य सुतरां तत्सत्त्वादिति भावः~॥
[भाष्ये ] तस्य नाज्ञानमात्रेण निस्तारः, किं निषेधाधि \textendash\ 
कारिताऽप्यस्तीत्याह \textendash\ [ भा० १ मप०~। ] नात्यन्तायेति~॥ 

 [भाष्ये] अत्र हेतुमाह \textendash\ [भा० २ यप०] यो ह्यजानन्निति~॥ 

 [ प्रदीपे ] नन्वजानतां प्रतिषिद्धानुष्ठानतश्चेद्दोषस्तर्हि
तिरश्चामपि स्यादत आह [ कै० १ मप० ] पुरुषाणामिति~॥ 

 [उ० १ मप०] अव्ययमिति~। विभक्तिप्रतिरूपकमिति भावः~॥ 

 [ उद्द्योते ] अधिकारिलक्षणे ( पू० मी० ५~। १~। २ अ.)
मनुष्याधिकारित्वस्य शास्त्रे व्यवस्थापितत्वात् {\qt पुरुषाणाम्} इत्ययुक्तमत
आह [ उ० २ यप० ]६मनुष्येति~। तथा च तदेतदुपलक्षणमिति भावः~॥ मनुष्यस्य
तिर्यक्साम्याभावमाह \textendash\ तिरश्रामिति~॥ 



वस्तुतो मीमांसकंमन्ययोरुभयोरप्ययुक्तमेवैतत्~। यतः {\qt स्ववतोस्तु}
वचनादैककर्म्यं(पू.६~। १~। १७)इत्यधिकरणे दम्पत्योः सहाधिकारनिर्णयेन
पतिपत्नीनां पतिसन्निधानेऽपि अधिकारोऽस्त्येव~। अन्यथा
पत्न्यवेक्षणादिकर्मसु तासामधिकारः कथं सिद्ध्येत् ? स्त्रिया
अनधिकारनिराकरणे \textendash\ {\qt लिङ्गविशेषनिर्देशात्पुंयुक्तमै \textendash\ तिशायनः} (६~। १~। ६) इत्यनेन
स्त्रिया अधिकारमाक्षिप्य {\qt जातिं तु बादरायणोऽविशेषात् तस्मात्स्त्रयपि
प्रतीयेत जात्यर्थस्याविशिष्टत्वात्} ( ६~। १~। ७ ) इति सूत्रेण तस्या अधिकारः
प्रतिपाद्यते~। अतः {\qt नायशियां वाचं वदेत्} इति निषेधः स्त्रीणामपीति
पतिसन्निधाने पत्नीनां कर्मण्यधिकाराभावेऽपि
इत्याद्युक्तिर्मीमांसाऽनवबोधमूलैवेति~। 

शास्त्रप्रयोजनाधिकरणम् ] महाभाष्यप्रदीपोद्द्योतव्याख्या छाया~। ३५



शरणं, मनुष्याणां तु योग्यत्वात्तैर्निषिद्धानुष्ठानवर्जनार्थमवश्यं
शास्त्रं विज्ञेयम्~। अतोऽवाग्योगविदो महाननर्थ इति भावः~॥ तत्र
{\qt नात्यन्ताय} इत्यनेन शास्त्राज्ञानं नैव शरणं, दोषद्वयप्रसङ्गात्~। किं तु
दध्यादेर्ब्राह्मणत्वादज्ञानमेव पापाल्पत्वे प्रयोजकमिति ध्वनितम्~॥
भाष्ये \textendash\ यो ह्यजानन्निति~। निषेधशास्त्रमजानन् ब्राह्मणत्वादिज्ञान \textendash\ पूर्वकं
तद्वधादि कुर्यात्सोऽपि पतित एवेत्यर्थः~॥ 

 (सिद्धान्तव्याख्याभाष्यम् ) 

एवं तर्हि \textendash\ 

सोऽनन्तमाप्नोति जयं परत्र 

वाग्योगवित् दुष्यति चापशब्दैः~॥ 

कः ? 

अवाग्योगविदेव~। 

अथ यो वाग्योगवित्, विज्ञानं तस्य शरणम्~॥ 



 [उ०४ र्थप० ] महाननर्थ इति~। स एवेत्यर्थ:~। एवेन
स्वल्पस्याप्यर्थस्य निरासः~॥ ननु अत्यन्तायेत्यधिकम्, तदभावेना \textendash\ 
पीष्टसिद्धेरत आह \textendash\ तत्रेति~। भाष्ये इत्यर्थः~॥ [ उ० ५ मप ] दोषद्वयेति
~। {\qt यो ह्यजानन्} इति वक्ष्यमाणेत्यर्थः~॥ किंचिदज्ञानं शरणमेवेति ततो
लब्धस्य फलितमाह \textendash\ किंत्विति~। वध्यादेरित्यस्य ब्राह्मणत्वादावन्वयः~॥ 

 [ उद्दयोते ] अत्र ब्राह्मणादेर्न कर्मत्वम्~। तथा सति
ब्रह्महत्यादिपातित्याभावेन तदापादनासंगतेः, शास्त्राज्ञानस्यैव
प्रक्रान्तत्वाच्च, अत आह \textendash\ [ ७ मप० ] निषेधेति~। {\qt मन्ये} इति
निश्चयार्थकं, पूर्वत्र {\qt वै} इतिवत्~। तद्वललभ्यमर्थमाह \textendash\ [ ८ मप० ]
पतित एवेत्यर्थ इति~॥ 

 [भाष्ये ] अथ सिद्धान्ती यथार्थोत्तरं दातुं पुनरनुवदति \textendash\ [ भा० १
मप० ] एवं तर्हीति~। तस्यान्यथा निर्वाहाभावेन दोषवत्त्वस्य सुवचत्व
इत्यर्थः~॥ 

 [ भाष्ये ] तदेतदाह \textendash\ [ भा० ६ ष्ठप० ] विज्ञानमिति~। 
शास्त्रादिज्ञान पूर्वकप्रयोगरूपमित्यर्थः~॥ अत एव {\qt विज्ञानम्}
इत्युक्तम्~। तत्र च बहुव्रीहिः~। प्रयोगरूपोऽन्यपदार्थः~॥ जनकत्वं
षष्ठ्यर्थः~॥ तदेतदुक्तं [ कै० ३ यप० ] ज्ञानपूर्वकेति~। भाष्ये
ज्ञानमिति पाठे तु तस्य लक्षणा बोध्या~॥ 



१ वाग्योगविदो ज्ञानरूपात्प्रकरणादित्यर्थः~। ज्ञानस्यापि प्रकरण \textendash\ 
शब्दवाच्यत्वं संख्यासंज्ञासूत्रे भाष्ये~। संनिधेरित्युद्दयोतस्य बुद्धि \textendash\ 
संनिधेरित्यर्थः~। अत्रत्या छाया चिन्त्या~। ( र. ना. ) वस्तुतस्तु
स्वरूपदर्शनेनापि पराहतमेतत्~। यतः {\qt प्रकरणात्सामर्थ्यं बलीयः} इति प्रदीपे
प्रकरणशब्दस्य भवदभिमतेऽर्थ स्वीकृतेऽपि तदपेक्षया सामर्थ्यस्य बलीयस्त्वं
केनोपदिष्टमिति कथं प्रदीपस्योपपत्तिरिति त एव प्रष्टव्याः~। 
अत्रान्यतराकाङ्क्षासत्वेन \textendash\ दुष्यतीत्यत्र
कर्तुरूपार्थस्याकाङ्क्षितत्वेनोभयाकाङ्क्षाया असंभवेन {\qt प्रकरणात्} इति
प्रदीपोक्तिर्न सङ्गच्छेत, अतः प्रदीपाशयमुपवर्णयितुं {\qt सन्निधेरित्यर्थः}
इत्युद्द्योते उच्यते~। सन्निधिश्च स्थानरूपप्रमाणविशेषः~। स्थानञ्च
देशसामान्यम्,तद्विविधम् \textendash\ पाठसादेश्यमनुष्ठानसादेश्यच्च~। पाठसादेश्यमपि
द्विविधम्यथासङ्ख्य \textendash\ पाठः सन्निधिपाठश्च, अयमेवोद्द्योते प्रदर्शितः~। 
यथासङ्ख्यपाठेन \textendash\ 





 ( प्रदीपः ) १प्रकरणात्सामर्थ्यं बलीय इत्याह \textendash\ अवाग्यो \textendash\ गविदिति~। 
वाग्योगवित्तूभययज्ञोऽपि शब्दान्प्रयुङ्कते, नापशब्दान् \textendash\ इति
ज्ञानपूर्वकप्रयोगादभ्युदयभाग्भवति~॥ 

 (उद्द्योतः ) प्रकरणादिति~। संनिधेरित्यर्थः~। नन्वेवं
वाग्योगविदोऽपशब्दज्ञानाद्धर्मः स्यादत आह \textendash\ वाग्योगेति~। 
एवञ्चापशब्दप्रयोग एवाधर्महेतुः, न ज्ञानमात्रमिति न दोष इति भावः~॥ 

 ( स्थानजिज्ञासाभाष्यम् ) 

क्व पुनरिदं पठितम् ? 

 ( प्रदीपः ) श्लोकस्यापरिज्ञानात्पृच्छति \textendash\ क्व पुनरिति~। 
प्रातिपदिकार्थप्रश्ने चात्र तात्पर्यम्, किं तदस्ति यत्रेदं पठितमित्यर्थः
~। अत एव {\qt श्लोकाः} इति प्रथमान्तेनोत्तरम्~। अन्यथा {\qt श्लोकेषु} इति वाच्यं
स्यात्~॥



 [ प्र० कै० १ मप० ] इत्याहेति~। सिद्धान्तीति शेषः~। इत्याशयेन स
आहेत्यर्थः~॥ 

 [उद्दयोते] उभयाकाङ्क्षारूपप्रकरणस्याभावादाह \textendash\ [उ० १ मप०]
संनिधेरिति~। स्थानाख्यप्रमाणादन्यतराकाङ्क्षारूपादित्यर्थः~॥ दुष्यति
च \textendash\ इति चस्त्वर्थे इति भावः~॥ 

 [उद्दयोते] अथेत्यादि भाष्यं कैयटेन छायया व्याख्यातम्~। 
तदाशयमाह \textendash\ [ उ० १ मप० ] नन्वेवमिति~। २तस्य दोषकर्त \textendash\ त्वाङ्गीकार
इत्यर्थः~॥ वाग्योगेति~। तस्याप्यपशब्दज्ञानादधर्मोsपि स्यादित्यर्थः~॥
एवं च प्रागुक्तदोष एवेति भावः~॥ [ २ यप० ] एवं चेति १ तस्य
तदप्रयोगादौ चेत्यर्थः~॥ {\qt यस्तुप्रयुङ्क्ते} इत्यनेन एकः शब्द इति
श्रुत्या च तत्प्रयोगेण धर्म उक्तो न ज्ञानमात्रेण~। तथा
चापशब्दज्ञानमात्रेण नाधर्मः , अपशब्दप्रयोगस्तु न वैयाकरणेन क्रियत एवेति
तस्य सर्वदा धर्म एव, साधुप्रयोगादभ्युदयोऽपि~। अन्यस्य तु
तत्प्रयोगादधर्मवृद्धिरेव~। तदाह \textendash\ अपशब्देति~॥ 

 [२ यप्रदीपे १ मप०] श्लोकस्यापरीति~। तत्स्वरूपस्यापरीत्यर्थः~। 
प्रागिति भावः~। यद्वा तदाधारस्येत्यर्थाः~॥ अत एवाह \textendash\ [२ यप०]
प्रातीति~। विभक्त्यर्थप्रकारकेत्यादिः~॥
अस्याशाब्दत्वदाह \textendash\ तात्पर्यमिति~। तदाह \textendash\ किं तदिति~॥ इत्यर्थ इति~। 
इति फलितार्थ इत्यर्थः~। अत एव \textendash\ तत्र तात्पर्यादेव~॥ 



ऐन्द्राग्नमेकादशकपालं निर्वपेत् वैश्वानरं द्वादशकपालं निर्वपेत् \textendash\ इत्येवं
क्रमविहितास्विष्टिषु {\qt इन्द्राग्नीरोचना \textendash\ } इत्यादिमन्त्राणां यथासङ्ख्यं
विनियोगः~। सन्निधिपाठेन च \textendash\ {\qt शुन्धध्वं दैव्याय कर्मणे} इति मन्त्रः
इध्माबर्हिर्निर्वापविषययोर्मन्त्रानुवाकयोर्मध्यमेऽनुवाके
पठितः \textendash\ इध्मावर्हिः संपादनस्य मुष्टिनिर्वापस्य चान्तरालं
सान्नाय्यपात्राणां देश इति प्रत्यक्षसन्निधिना
सान्नाव्यपात्रशोधनशेषोऽयमिति निर्णीयते~। तथा दुष्यति चापशब्दैरित्यत्र
सन्निधिपाठात् वाग्योगविद एवान्वयः कर्तृत्वेन प्राप्तोऽपि
सामर्थ्यस्य \textendash\ शब्दसामर्थ्यरूपलिङ्गस्य बलीयस्त्वात्तेन अवाग्योगिद एव
कर्तृत्वेनान्वय इति सिद्धान्तः~। अत एव {\qt अज्ञानं तस्य शरणं}
इतिभाष्यावतरणे \textendash\ नन्वेवंसामर्थ्यादवाग्योगविदेव दुष्यतीत्यत्र
कर्तास्तु \textendash\ इत्युक्तमुद्द्योते~। अमुमेवार्थं सुस्पष्टं प्रतिपादयन्तीं
छायां चिन्त्यत्वेनाभिव्याहरन् ननु देवानांप्रिय इति निष्कर्षः~॥
२अवाग्योगविद इत्यर्थः~। (र.ना.) 

३६ उद्द्योतपरिवृतप्रदीपप्रकाशितमहाभाष्यम्~। 

 [१ अ. १ पा. १ पस्पशाह्निके



 (समाधानभाष्यम् ) 

भ्राजा नाम श्लोकाः~॥ 

 ( उद्द्योत्तः ) {\qt भ्राजाः नाम कात्यायनप्रणीताः श्लोका इत्याहुः~॥}

 ( प्रामाण्याक्षेपभाष्यम् ) 

किं च भो: श्लोका अपि प्रमाणम् ? 

किं चातः ? 

यदि प्रमाणम्, अयमपि प्रमाणं भवितुमर्हति \textendash\ 

यदुदुम्बरवर्णानां घटीनां मण्डलं महत्~॥ 

पीतं न गमयेत्सर्गे किं तत्ॠतुगतं नयेत्~।  इति~॥ 

 ( प्रदीपः ) आप्तोक्तत्वापरिज्ञानादाह \textendash\ किं च भो इति~॥
यदुदुम्बरेति~। अयं श्लोकः सौत्रामणियागे सुरापाणस्य दुष्त्वमुद्भावयति
~॥ 

 ( उद्व्योतः ) भाष्ये \textendash\ किं चात इति~। अतः प्रश्नात्किं विव \textendash\ 



 [उद्द्योषे ] ननूत्तरे उक्तो वस्तुत आधारपदार्थ एव कोऽत
आह \textendash\ [द० उ० १ मप०] भ्राजेति~॥ 

 [वामे २यभा ० १मप ० ] चः \textendash\ प्रश्नसमुच्चये ~॥ वेदस्मृतिशास्त्रविषयाः
श्लोका अपि किं धर्माधर्मयोः प्रमाणमिति भाष्यार्थः~॥ 

 [भाष्ये] परस्तद्वाचा {\qt यदि प्रमाणं तर्हि अयमपीति} तदीयम \textendash\ 
भिप्रायमभिधापयितुं पृच्छति \textendash\ [ २ यप० ] किं चात इति~॥ 

 [भाष्ये ] स स्वाशयमाह \textendash\ [ ३ यप० ] यदीति \textendash\ भाष्ये~॥ अयमपीति~। 
धर्ममनधिकृत्य येन केनापि पठितस्यापि प्रामाण्ये
वक्ष्यमाणश्लोकस्यापीश्चरप्रणीतस्य प्रामाण्यापत्तिरिति भावः~॥ 

 [वामे प्रदी० १ मप० ] आाप्तोक्तत्वेति~। भ्राजाख्यानामित्यादिः~॥ 

 [ उद्दयोते ] यत्तु \textendash\ निपातसमुदायोsयं किमित्यर्थे, श्लोकप्रामाण्ये
किमनिष्टमित्यर्थः \textendash\ इति कृष्णः ? तन्न, तेषां प्रयोगासांगत्यापत्तेः,
तत्प्रामाण्येऽनिष्टाभावाच्च~। तदेतद् ध्वनयन्नाह \textendash\ [उ० १ मप०] अत
इति~॥ 

 [द्वितीये उ० २यप ०] ताम्रमिति~। तथा च लोहिनीनां कलशीनां विशालं
मण्डलं लक्षणया ततत्स्थ सुराद्रव्यं, स्वरूपतस्तस्य पानकर्मत्वायोग्यत्वात्
~। तत् पीतं सद् यदि पातारं स्वर्गं न गमयेत्त् तत् तहिं
सौत्रामणीग्रहपात्रस्थं खल्पं तत् \textendash\ सुराद्रव्यं पीतं किं नयेत्~। नेष्यत्येव
~। तथा च तैत्तिरीयब्राह्मणम् ब्राह्मणं परिक्रीणीयादुच्छेपणस्य पातारं
ब्राह्मणो ह्याहुत्या उच्छेषणस्य पाता यदि ब्राह्मणं न विन्देत्
वल्मीकवपायामवनयेत् सैव ततः प्रायश्चित्तिः? इति~॥ एवञ्च स्वर्ग \textendash\ 
साधनत्वेन श्रुता सुराऽल्पत्वात् सम्यक स्वर्गं नेतुमशक्ता~। तस्मा \textendash\ 



{\qt अहम् शुभम् ते मेअस्ति अस्मि वरम्} इत्यादिवत् {\qt प्रमाणम्?}
इत्येतदपि विभक्तिग्प्रतिरूपकमव्ययमपि इत्येवमेव मेदिनी \textendash\ काराद्याशयः~। अत
एव श्लोकाः प्रमाणम् {\qt वेदाः प्रमाणम्} इत्यादिषु न
सामानाधिकरण्यानुपपत्तिः~। एतदजानानामिव विद्वद्वराणामत्र
सामानाधिकरण्योपपत्तये कोलाहलः स्वकीयवावदूक्ताख्यापना \textendash\ 





क्षितमित्यर्थः~॥ उदम्बरं$=$त्ताम्रम्~॥ दुष्टत्वमिति~। इदमुपलक्षणं \textendash\ 
अन्यत्रप्यदुष्टत्वस्य~॥

 (समाधानभाष्यम्)

 प्रमत्तगीत एष तत्रभवतः~। २यस्त्वप्रमत्तगीत \textendash\ स्तत्प्रमाणम्~॥ यस्तु
प्रयुङ्क्ते~॥ 

 ( प्रदीपः ) प्रमत्तगीत इति~। प्रमादेन \textendash\ विप्रतिपन्नत्वेन गीत इत्यर्थः~। 
कात्यायनोपनिबद्धभ्राजाख्यश्लोकमध्यपठितस्य त्वस्य३ श्लोकस्य
श्रुतिरनुग्राहिकाsस्ति \textendash\ {\qt एकः शब्दः सुज्ञातः सुप्रयुक्तः स्वर्गे लोके
कामधुग्भवति} इति~॥

 (उद्द्योतः ) ननु कर्तरिक्तान्तप्रमत्तशब्दस्य गीतशब्देन समा \textendash\ 
सोऽनुपपन्नः, {\qt तत्रभवतः} इत्यनेन साकाङ्क्षत्वादत आह \textendash\ प्रमादेनेति~। भावे
क्तान्त इति भावः~। विप्रतिपन्नत्वेन वेदविषय \textendash\ विप्रतिपत्त्याश्रयत्वेन~। 
४तत्वं स्वस्मिन्नारोप्य दैत्यनाशाय पूज्यस्यापि भगवत ईश्वरस्य तथोक्तिरिति
स न प्रमाणमिति भावः~॥ [५भाष्ये \textendash\ 



त्सौत्रामणीं हित्वा पानागारे एव मद्यं पातव्यमित्यर्थः~॥ तदेतदभि \textendash\ 
प्रेत्याह \textendash\ [उ० २यप० ] इदमुपेति~। [उ० ३यप०] अन्यत्रेति च~॥
अन्यत्र \textendash\ यागातिरिक्ते पानागारे तत्पानादुष्टत्वस्यापीत्यर्थः~॥ अपिना
तस्यापि समुच्चयः~॥ 

 [भाष्ये] एतन्निराचष्टे \textendash\ [द० भा० १ मप०] प्रमत्तेति~॥ 

 [द० भा० १मप०] {\qt तत्रभवतः} इति पूर्वान्वयी, नोत्तरान्वयी {\qt यस्तु}
इत्यनेन विच्छेदात्~। तथा च तत्रभवत एष श्लोकः प्रमादेन पठित इति तदर्थो
बोध्यः~॥ 

 [भाष्ये] नन्वेवं कथमस्यापि न प्रामाण्यम्, अत आह \textendash\ [ वामे भा० १
मप०] यस्त्वप्रेति~॥ 

 [द० प्र० १ मप० ]यद्यपि प्रमादश्चैतसोऽसमाधानम्,
{\qt प्रमादोऽनवधानता} इत्युक्तेः, तथापि अकृते विशेषमाह \textendash\ कैयटे विप्रतीति
~॥ 

 [ द० उ० १ मप० ] कर्तरीति~। प्रमाद्यतीति प्रमत्त इति भावः~॥
[उ० २यप०] साकाङ्क्षेति~। तथा च {\qt सापेक्षम् \textendash\ } इति
न्यायादसामर्थ्यमिति भावः~॥ भावे क्तान्त इति~। भावे यः क्तस्त \textendash\ दन्त
इत्यर्थः~। प्रमत्तशब्द इति शेषः~॥ 

 [उद्दयोते] अत एव तदर्थमाह \textendash\ [उ ३ यप०] वेदेति~॥ 

 [उद्दयोते] ननु कोऽसौ तत्रभवान् ? यदीश्वरः, कथं तत्र
विप्रतिपत्तिः ? अथान्यः, कथं तत्र पूज्यत्वं, भ्रमादिसंभवात्~। अत आह \textendash\ [
उ० ४ र्थप० ] तत्त्वमिति~। विप्रतिपत्त्याश्रयत्वमित्यर्थः~॥ आरोपे
बीजमाह \textendash\ दैत्येति~॥ तत्रभवत इत्यस्यार्थमाह \textendash\ पूज्येत्यादि~॥
भगवतः \textendash\ षड्विधैश्वर्यपूर्णस्य~॥ ईश्वरस्य \textendash\ नियन्तुः~॥ 



येवेति दाधिमथाः~॥ प्रमाणत्वे कत्वान्वयेन उपपत्तौ अक्लृप्तागतिक \textendash\ 
गत्यव्ययत्वकल्पनमयुक्तमिति वयम्~। ( र. ना. ) २ यस्य कस्यापि यः
श्लोकोऽप्रमादेन पठितस्तत्प्रमाणमित्यर्थः~। ३ अस्य \textendash\ यस्तु प्रयुङ्क्ते कुशल
इत्यस्य~। ४ तत्त्वं \textendash\ वेदविषयविप्रतिपत्त्याश्रयत्वम्~। ५ अ. पुस्तके [
] चिह्नगतो न पाठः~। 

शास्त्रप्रयोजनाधिकरणम् ] महाभाष्यप्रदीपोद्द्योतव्याख्या छाया~। ३७



यस्त्वप्रमादेन गीतो यस्य१ कस्यापि तत्प्रमाणमिति २प्रतिनिर्देशलिङ्गम्~। 
] ननु कथं ज्ञातं अयमप्रमत्तगीतः इति \textendash\ अत आह \textendash\ श्रुतिरनुग्राहिकेति३~॥ 

 (प्रयोजनभाष्यम् ) 

अविद्वांसः \textendash\ 

अविद्वांसः प्रत्यभिवादे नाम्नो ये न प्लुतिं विदुः~। 

कामं तेषु तु विप्रोष्य स्त्रीष्विवायमहं वदेत्४~। 

स्त्रीवन्मा भूमेत्यध्येयं व्याकरणम्~॥ 

अविद्धांसः~॥ 



 [ उद्द्योते ] अस्यापि पूर्ववदेवार्थ इत्याह \textendash\ [ उ० ६ ष्ठप० ]
भाष्येयस्त्वेति~॥ ६तस्यात्रासंबन्धं सूचयन्नाह \textendash\ यस्येति~॥ प्रतीति~। 
तदित्यत्रेति भावः~॥ एतेन \textendash\ अत्रापि यस्त्वप्रमादेन पठितः स तत्रभवतः
प्रमाणमेव \textendash\ इति कृष्णोक्तव्याख्यानम् \textendash\ {\qt मद्यपानादिनोन्मत्तस्य}
वेदविरुद्धमार्गानुयायिना केनचिह्वाह्येन पठित इति न तस्य प्रामाण्यम्,
तत्रभवतः \textendash\ पूज्यस्य महर्षः, कर्तुः शेषत्वविवक्षायां षष्ठी, न तु
कृद्योगलक्षणा, {\qt न लोक \textendash\ } इति निषेधात् अप्रमत्तगीतः \textendash\ अप्रमादेन
{\qt सावधानतया साधनेन गीतस्तत्प्रमाणं वेदवत्?} \textendash\ इति रत्नोक्तं चापास्तम्~। 
दोषगणसत्त्वस्योक्तप्रायत्वात्~। तुशब्दासांगत्यापत्तेश्च~। 
वस्तुस्थितेस्तथाऽसत्त्वाच्चेति दिक्~॥ 

 [भाष्ये ] उ्त्वैलक्षण्यसूचनाय भाष्ये \textendash\ श्रुतौ तुशब्दः~॥ नन्वेतावता
किमत आह \textendash\ स्त्रीवदिति ~॥ एतदध्ययनेन प्लुतज्ञाने तत्र तत्करणात्स्त्रीसाम्यं
नेति तत्फलकमिदमध्येयम्~। अन्यथा तत्साम्यं स्यादिति शाब्दोऽर्थः~॥ 



१ यस्य कस्यापीत्यनेन \textendash\ वेदविरुद्धवाद ईश्वरस्यापि अप्रमाणम्~। 
अप्रमत्तगीतश्च तदितरस्यापि प्रमाणमिति~॥ 

२ ननु {\qt तत्प्रमाण इति भाष्ये तच्छब्दस्य नपुंसकलिङ्गव्यवहारो} \textendash\ ऽनुचितः ,
उपक्रमे {\qt यस्तु} इति पुंल्लिङ्गव्यवहारादत आह \textendash\ प्रति \textendash\ निर्देशेति~। 
विधेयलिङ्गतया तत्त्वमित्यर्थः~। ३ तथा साधुष्वपि
सामान्यलक्षणाद्विशेषलक्षणान्वितस्य प्रयोगविशेषे फलविशेषः स्मर्यते~। तथा
च \textendash\ 

नाकमिष्टसुखं यान्ति सुयुक्तैर्वडवारथैः~। 

अथ पत्काषिणो यान्ति येऽचीकमतभाषिणः~॥

इति पदमञ्जर्यां निर्णीतम्~। ( दा. म. ) 

४ अभिवादात्परं विप्रो ज्यायांसमभिवादयन्~। 

असौनामाहमस्मीति स्वं नाम परिकीर्तयेत्~॥ 

इति मनुनापि नाभोच्चारणस्य विहितत्वेन, 

भात्मनाम गुरोर्नाम नामातिकृपणस्य च~। 

श्रेयस्कामो न गृह्णीयाज्ज्येष्ठापत्यकलत्रयोः~॥ 

इति स्मृतौ नामोच्चारणस्य निषिद्धवेन षोडशिग्रहणाग्रहणवद्वि \textendash\ 
कल्पापत्तिरिति तु न सम्यक्, यतो वेदाङ्गज्यौतिषे \textendash\ 

नक्षत्रदेवता एता एताभिर्यज्ञकर्मणि~। 

यजमानस्य शास्त्रज्ञैर्नाम नक्षत्रजं स्मृतम्~॥

इत्युक्तत्वेन, बौधायनेनापि {\qt पुत्रस्य नाम गृह्णाति रौहिणाय तिष्यायेति}
इत्युक्तत्वेनापस्तम्बेनापि {\qt नाक्षत्रं नाम च निर्दिशति~। तद्रहस्यं
भवति} इत्युक्तत्वेनाश्चलायनश्रौतकारिकायां \textendash\ {\qt निर्दिशेद्यजमानः} स्वं नाम
सांव्यवहारिकम्~। 





 ( प्रदीपः ) स्त्रीष्विवेति~। प्रत्यभिवादे हि गुरुणा प्लुतः कार्यः~। 
यस्तु प्लुतं कर्तुं न जानाति स स्त्रीवद्वक्तव्यः \textendash\ अयं \textendash\ अहं \textendash\ इति, न तु
{\qt अभिवादये दैवदत्तोsहम्} इत्यादिना संस्कृतेन वाक्येनेत्यर्थः~॥ 

 ( उद्व्योतः ) भाष्ये \textendash\ अविद्वांस इति~। अत्र योग्यत्वाद्व्याकरणं कर्म
बोध्यम्~॥ प्रत्यभिवादे$=$अभिवादानन्तरप्रयोज्ये
वचनेनाम्नः \textendash\ $=$अभिवादकनाम्नष्टेःप्लुतिं कर्तुं ये न जानन्ति तेषु कामं
निःशङ्कं विप्रोष्यन्प्रवासादेत्य~। यद्वा \textendash\ कामं$=$विध्यनुसारमन्तरै \textendash\ 



 [ प्रदीपे ] अयमित्यादि~। {\qt अयमहं नमामि} इति वदेत्~। एतदभिनयेन
कृताञ्जलिपुटस्य हृदयदेशप्रापणादभिवादतभ्रान्तिज \textendash\ ननेन तान्प्रीणयेत्, न तु
{\qt अभिवादये भारद्वाजगोत्रकृप्णशर्माहं भोः इत्यादिना
नामगोत्रोच्चारणपूर्वकपादग्रहणं कुर्यादित्यर्थः~॥ यद्वा \textendash\ उक्ताशयेन
अयमहमस्मि} इत्येव वदेत् , न तु {\qt नमामि} इत्याद्यपि \textendash\ इत्यर्थः~। तदेतद्
ध्वनयन्नुभयसाधारण्ये \textendash\ नाह \textendash\ कैयटे \textendash\ प्रत्येत्यादिना~॥ प्रत्यभिवादेन ह्यसौ
संस्कृतोऽभिवाद्यो धर्मार्हो भवति स्वयं चाभ्युदयेन युज्यते~। तेन
तदनभिज्ञा नाभिवाद्या इति भावः~॥ 

 [उद्द्योते २ यप० ] वचने \textendash\ आशीर्वचनवाक्ये~। उपात्तस्येति शेषः~॥
यथाश्रुतानुपपत्तिः शेषपूरणेन व्याचष्टे \textendash\ टेरित्यादि~॥ कामं यथेष्टम्
~। तत्फलितमाह \textendash\ निरिति~॥ विप्रोष्येत्यस्य
गत्वेतिरत्नोक्तव्याख्यानमयुक्तमुदक्षरत्वाद्, अत आह आक्न \textendash\ [ ४ र्थप० ]
प्रवासादिति~॥



 नाक्षत्रं च यथा कृष्णशर्मा रौहिण इत्यपि~॥ 

 इत्युक्तत्वेन, गोभिलगृह्यसूत्रे \textendash\ {\qt अभिवादनीयं नामधेयं कल्पयित्वा~। 
देवताश्रयं वा नक्षत्राश्रयं वा~। गोत्राश्रयमप्येके}
इत्युक्तत्वेनाश्वलायनगृह्मसूत्रेऽपि \textendash\ {\qt अभिवादनीयं च समीक्षेत
तन्मातापितरावेव विद्यातामोपनयनात्} इत्युक्तत्वेन,
गोभिलगृह्णे \textendash\ सोष्यन्तीहोमप्रकरणे {\qt पुमानयं जनिष्यते \textendash\ असौनामेति नामधेयं
गृह्णाति~। यत्तद्गुह्यमेव भवति} इत्युक्तत्वेन {\qt अमुष्यासाविति पतिनाम
गृह्णीयादात्मनश्च इत्यादिगृह्यवा्क्य \textendash\ पर्यालोचनेन} जबाला तु नामाहमस्मि
सत्यकामो नाम त्वमसि~। स सत्यकाम एव जाबालो ब्रुवीथाः? इत श्रुतावपि
नामोश्चारणस्य विहितत्वेन दशरथतीतादिकृतरामनामोच्चारणस्य श्री
मर्यादापुरुषोत्तमावतारवाञ्छामानुषविग्रहश्रीरामकृतदशरथसीतादिनामोच्चारणस्योपलम्भेन
सांव्यवहारिकाभिवादनीयेतिसमाख्यया नामोच्चा रणस्यावश्यकर्तव्यतया
तत्तत्कर्मंनियततत्तन्नामनिर्देशस्य कर्मान्तरे ग्रहणनिषेधपरा
निषेधस्मृतिरिति सामान्यविशेषरीत्या व्यवस्था कल्पनीयेति दिक् इति दाधिभथाः
~॥ 

५ स्त्रीवदिति~। यथा स्त्रीषु नामगोत्रोच्चारणपूर्वकमभिवादनं न क्रियते
{\qt किन्तु अयमहं नमामि} इत्युक्त्वाsभिनयमात्रं क्रियते, तथाऽविद्वत्सु
व्यवहर्तव्यमित्यः~। यत्तु {\qt भाषाशब्दैरेव} तेषां नमस्कारं कुर्यात् इति,
तदप्रामाणिकं~। संस्कृतशब्दैर्व्यवहारे न किञ्चिद्बाधकम्~॥ 

६ तस्य \textendash\ तत्रभवत इत्यस्य, अत्र \textendash\ यस्त्वत्रेत्यादौ~॥ 

३८ उद्द्योतपरिवृतप्रदीपप्रकाशितमहाभाष्यम्~। 

 [ १ अ. १ पा. १ पस्पशाह्निके



वेत्यर्थः~। {\qt अयमहम्} इत्यनुकरणस्यानितिपरत्त्वेऽपि कृञ्योगाभावा \textendash\ 

द्गतित्वाभावेन {\qt वदेत्} इत्यस्य तिङन्तत्वेन च समासाभाव इत्याहुः~॥~। अत्र
क्वचित् अयमहमिति वदेत्~। अनुकरणं चानितिपरमिति

गतित्वात्समासः इति पाठो दृश्यते सोsपपाठः~॥ भाष्ये \textendash\ स्त्रीव \textendash\ 

दिति~। तद्वत्करणेन सू्चितशास्त्रज्ञानित्वकृतापमानप्रयुक्तदुःखभाजो

मा भूमेति भावः~॥

 ( प्रयोजनभाष्यम् )

 विभक्तिं कुर्वन्ति \textendash\ 

 याज्ञिकाः पठन्ति \textendash\ {\qt प्रयाजाः सविभक्तिकाःकार्याः} इति~। न चान्तरेण
व्याकरणं प्रयाजाः



 [ उ० ७ मप० ] यत्तु अनुकरणं चेति गतिसंज्ञायां गति \textendash\ 

समासेपि अनुकरणात्वादेवास्यवामीयमित्यादिवद् विभ \textendash\ क्तेर्न लुक् \textendash\ इति
कृष्णः~। तदसद् इति ध्वनयन्नाह[७मप०]अयमिति~॥ आहुः \textendash\ सिद्धान्तविदः
~॥ कैयटीयपाठं दूषयति[ ७ मप० ] अत्रेति~। ३ {\qt वाक्येनेत्यर्थः}
इत्यग्र इत्यर्थः~। उक्त \textendash\ 

हेतोः , मिथोविरुद्धत्वाच्चेति भावः~॥

 [ उ० ९ मप० ] तत्फलितमाह \textendash\ [ ९ मप० ] तद्वत्करणे \textendash\ 

नेति~। स्त्रीवदित्युक्त्येत्यर्थः~॥ शास्त्राज्ञानकृताप इति पाठः
~॥

 भूमेतीति भावः इति पाठः~॥

 न च

 आयुष्भान् भव सौम्येति वाच्यो विप्रोsभिवादने~। 

 अकारश्चास्य नाम्नोsन्ते वाच्यः पूर्वाक्षरः प्लुतः~॥

 इति मनुस्मृतेरेव ज्ञाने नेदं फलमिति वाच्यम्~। {\qt ऊकालोच् \textendash\ }

इति लक्षितस्य तस्यैतदधीनज्ञानत्वात्~॥ न च लोकप्रसिद्धयैव
त्रिमात्रःप्लुत इति सुज्ञानम् , लोकप्रसिद्धेरपि शास्त्रमूलकत्वात्,
{\qt प्लुङ् गतौ}इति धातुसिद्धस्य तस्य त्रिमात्रेऽप्रसिद्धेरिति बोध्यम्~॥

 [भाष्ये] \textendash\ विभक्त्तिं कुर्वन्तीति \textendash\ लेट्~। विभक्त्यन्तं
कुर्युरित्यर्थः~। 

अत एव वचनैकवाक्यता~॥ कुत्रेत्याकाक्षायामाह \textendash\ याज्ञिका

इत्यादि~। पूर्ववत्~॥

[भाष्ये]ननु किमेतावताऽत आह \textendash\ न चान्तेति \textendash\ भाष्ये~॥

च् \textendash\ यतः~॥ अतोsध्येयं व्याकरणमिति शेषः~॥ विभक्तयादिज्ञानं

चैतदधीनम्~॥ याज्ञिकवृद्धव्यवहारात्तज्ज्ञानम \textendash\ इति तु न, तस्या \textendash\ 

प्येतन्मूलकत्वात्~। तस्यैव प्रमाणत्वे मीमांसोच्छेदापत्तेश्चेति दिक्~॥

[ग्रदीपे) \textendash\ आधानोत्तरमाधातुर्महासंकटादौ विहितपुनराधेयेष्टा \textendash\ 

विदं \textendash\ प्रयाजा इति समाम्नातम्~। प्रयाजेषु यागेषु श्रूयमाणं



१ {\qt ऊह्यमाना अभि शब्द} इति अ. पाठः~॥

२ उद्दोते शङ्कात्रयमत्रोपस्थापितम् , तच्चेत्थम् \textendash\ {\qt तत्किमनेनं वच \textendash\ 

नेन} इत्यनेन {\qt प्रयाजाः सविभक्तिकाः} कार्याःइत्यस्यानर्थक्यशङ्का,

{\qt यत्किङ्चित्प्रातिपदिक \textendash\ } इत्यनेन प्रयाजमत्रगताशेषपदेभ्योsपि
विभक्तिशङ्का, {\qt सर्वा विभक्तयः} इत्यनेन प्रथमादिसप्तम्यन्तानामाशङ्केति

शङ्कात्रयस्यापि छायायां {\qt अपूर्वेति~। यत इत्यादिः~। अनेनार्थक्य \textendash\ 

शङ्कापरिहृता} इत्यादिग्रन्थेन क्रमेणोत्तराणि विहितानीति तत एव

ज्ञेयानि~। ३ कैयटीयापपाठस्थलं दर्शयति \textendash\ वाक्येनेत्यर्थ इत्यग्रे 





 सविभक्तिकाः शक्याः कर्तुम्~॥ विभक्तिं कुर्वन्ति~। 

 ( प्रदीपः ) प्रयाजा इति~। प्रयाजमन्त्रा १ऊह्यमानाऽग्नि \textendash\ 

शब्दप्रकृतिकविभक्तियुक्ता इत्यर्थः~। यथा {\qt समिधः समिधोऽग्ने अग्न
आज्यस्य व्यन्तु} अग्निरग्निरिति~। 

 (उद्व्योतः) ननु २प्रकृतौ प्रयाजमन्त्राः सविभक्तिका एव पठ्यन्ते,

तत्किमनेन वचचनेन ? इत्यत आह \textendash\ प्रयाजेति~। ऊह्यमाना याsग्नि \textendash\ 

शब्दप्रकृतिकविभक्तिस्तद्युक्ता इत्यथः~॥ ननु {\qt अग्निशब्दप्रकृतिकः}
इत्यसङ्गतम्~। अपूर्वविभक्तिमात्रविधाने {\qt न केवलः} प्रत्ययः इति निषेधात्
यत्किच्चित्प्रातिपदिकप्रकृतिकाः सर्वा विभक्तयः प्रयोक्तव्या इति
प्राप्नोति \textendash\ इति चेत्, न~। {\qt त्वमग्ने} प्रयाजानुयाजानां पुरस्तात्



सविभक्तिकत्वम् {\qt आनर्थक्यात्तदङ्गेषु?} ( पू. मी. ३~। १~। १७)

इति न्यायेन तदुपकारकेपु मन्त्रेष्ववतरतीति प्रयाजपदं मन्त्रपर \textendash\ 

मित्याह \textendash\ कैयटे \textendash\ प्रयाजमन्त्रा इति~। {\qt समिधोऽग्न} आज्यस्यव्यन्तु इत्यादय
इत्यर्थः~। 

 [ प्रदीपे \textendash\ {\qt न केवला \textendash\ } इति निषेधात्केवलविभक्तिविधानस्य

दुर्वचत्वादाह \textendash\ कैयटे \textendash\ उह्येति~॥

 [प्रदीपे] \textendash\ तदभिप्रेत्यैव प्रयोगस्वरूपमाश्वलायनाद्युक्तं
दिक्प्रदर्श.

नाय कैयटे आह \textendash\ यथा समिधः समिधोऽग्रेऽग्न इति~॥

 [ उद्द्योते ]तदभिप्रेत्यैवाह \textendash\ उद्द्योते \textendash\ [ १ मप० ] ननु

प्रकृताविति~। एवे चातिदेशेन तथैव तत्र तेषां प्राप्तिरिति विधान \textendash\ 

मनर्थकमिति भावः~॥ तदाह \textendash\ तत्किमिति~॥

 इदं च नाग्निशब्दे विशेषणम् , किंतु विशिष्टविशेषणमित्याह \textendash\ उद्द्योते [
२ यप० ] ऊह्यमानेति~॥ तत्र शङ्कते \textendash\ [ ३ यप० ]नन्वग्नीति~॥ [४
र्थप०] अपूर्वेति~। यत इत्यादिः~॥ अनेनानर्थक्यशङ्का परिहृता~॥
दोषान्तरं सूच्यन्नाह \textendash\ सर्वा इति~। 

४तत्राद्यां समाधत्ते \textendash\ त्वमञ्न इति~। यथासंख्यमन्वयः~॥ एतेन

श्रुतानामेव {\qt समिधः} इत्यादिघटकविभक्तीनां स्थाने सर्वासां पर्या \textendash\ 

येण प्रयोगः प्राप्नोतीत्याशङ्काऽपि समाहिता~॥ अत्रेदं बोध्यम् \textendash\ 

यद्यपि {\qt विभक्तयो भवन्ति वाचो विधृत्यै} इति तैत्तिरीयसंहिता \textendash\ 

विधिः~। तत्रेष्टौ प्रयाजानामाग्नेयत्वसंपादनाय सर्वविभक्तयस्ततः

प्राप्ताः~। तत्राग्नेर्देवताया एकत्वात् द्विवचनवहुवचनयोरसंभव एव,

तथा सति सर्वैकवचनानि प्राप्तानि~। ६तथोत्तमेऽपि७ आदौ मध्ये चान्तेच, अस्य
विधेरपि विशेषात्~। तावता ८साहित्यस्यापि संभवात् प्रयाजानुयाजेष्वेव
विभक्तीः कुर्यादिति तत्प्रकरणस्थब्राह्मणवाक्यमपि

साधारणम्~। तच्च सर्वं पुनराधेयमाग्नेयं न कार्यम् , किंतु प्रया \textendash\ 



तदुत्तरमिति भावः~॥

४ तत्राद्यगमिति~। आनर्थक्येतरशङ्काद्वयस्याद्यां~। अशेषपदेभ्योऽपि
विभक्तिशङ्कामित्यर्थः~॥

५ अग्निशब्दप्रकृतिका एव ता इति व्यवस्थापनेनेत्यर्थः~॥

६ पञ्जमे प्रयाजे इत्यर्थः~। (र. ना. )

७ आदिमध्यावसानेषु प्रयाजेषु सविभक्तिकत्वविधेस्तुल्यत्वादि \textendash\ 

त्यर्थः~। ( र. ना. )

८ विभक्तीत्यादिः~। ( र. ना.~। 

शास्त्रप्रयोजनाधिकरणम् ] महाभाष्यप्रदीपोद्द्योतव्याख्या छाया~। ३९



पश्चात् इति मन्त्रवर्णेनाग्निशब्दप्रकृतिका एव ता इत्याशयात्~। 
तत्राग्नेदेवताया एकत्वादेकवचनमेव~। विभक्तयश्च प्रथमाद्वितीयातृतीया \textendash\ 



जानुयाजेष्वेवाग्निविषया विभक्तिः कार्या इत्यर्थकम्~। तत्रानुयाज \textendash\ 
विभक्तिरूपमाश्वलायनेनोक्तम् \textendash\ देवबहिरग्ने वसुवने वसुभ्धेयस्य वेतु देवो
नराशंसाग्नौ वसुवने वसुधेयस्य वेत्वे(त्विति)~। तथा चाज्यभागौ
पत्नीसंयाज्याभ्यः प्रकृतिवदेव~॥ आपस्तम्बेन त्वन्यथोक्तम् \textendash\ देवेऽग्नौ
देवोऽग्निरिति द्वयोरनुयाजयोर्विभक्तिमुक्त्वा प्रयाजेन वरदकरोतीति
संहितास्थवाक्येन विभक्त्यन्तं तदुच्चार्यम्~। प्रयाजेन यांज्यया
वषद्कारपूर्वं यजेदित्यर्थकेन तृतीयदोषनिरासः~। तथा च दोषद्वयमस्त्येव,
तथापि अग्नेऽग्नेऽग्नावग्नेऽग्निनाग्नेऽग्निमग्न इति चतुर्षु
प्रयाजेषु चतस्रो विभक्तीर्ददाति नोत्तमे इति
पक्षान्तरप्रतिपादकापस्तम्बेन सर्वनिरासः। तथा
च्वोत्तमवर्जमग्निप्रकृतिकसंबुद्धिसप्तम्येकवचनतृतीयैक \textendash\ वचनद्वितीयैकवचनानि
याज्यातः पूर्वमग्निशब्दात्पूर्व वा प्रयोज्या \textendash\ नीति सिद्धम्~॥२
द्वितीयशङ्काऽप्यनेन समाहिता~। तदेतदभिप्रे \textendash\ त्याह \textendash\ [ ७ मप० ]
तत्रेत्यादि~। तत्र \textendash\ तत्रापीत्यर्थः~॥ सकल \textendash\ विभक्तीनामपि मध्ये इति
यावत्~॥ तत्रापि संबुद्धिटाङे५न्तानां न प्रयोगः~। {\qt व्द्यक्षराः सन्तो
द्वित्वेनावृत्त्या चतुरक्षरा भवन्ति} इति वचनात्~। संबुद्ध्यस्तस्य
व्द्यक्षरत्वेऽपि द्वित्वोत्तरं पूर्वरूपे अग्नेऽग्ने इति
त्र्यक्षरत्वात्~। टाङेन्तयोरादित एव तत्त्वाभावाच्च~॥ तथा ङस्यन्तमपि न
प्रयोज्यम् ,{\qt  न शब्दजामि पठेत्~। शब्दजामि हि तद् भवति
यत्पञ्चम्यन्तम्} इति वचनात्~॥ तथा {\qt मात उत्तमे} इति वचनेन
निषेधादुत्तमप्रयाजमन्त्रे न तत्प्रयोग इत्याहुः~॥ तथा चादितश्चतुर्षु
प्रयाजमन्त्रेषु अवशिष्टेषु यथाक्रममाद्ये पक्षे चतस्रो विभक्तयः
संबुद्धिसप्तम्येकवचनतृतीयैकवचनद्वितीयैकवचनरूपा
अग्निशब्दप्रकृतिविशिष्टास्ततः पूर्वं प्रयोज्याः इति सिद्धम्~। स च
प्रयोगो ये यजामहे भूर्भुवः सुवः समिधोऽग्नेऽग्न आज्यस्य व्येतु वौषट् इति
~॥ द्वितीयपक्षे तु \textendash\ ये यजामहे भूर्भुवः सुवः 



१ \textendash\ यद्यपि प्रकृतौ प्रयाजमन्त्राः सविभक्तिका एव पठ्यन्ते~। तथापि
यद्याधानादनन्तरं यजमान उदरव्यथावान् स्यात्, यदि वा संवत्सरमध्ये तस्य
महती विपत् स्यात् , तदा नैमित्तिकी पुनराधेयेष्टिर्विधीयते~। 
तत्रेदमाम्नातम् \textendash\ {\qt प्रयाजाः सविभक्तिकाः \textendash\ } इति तत्र केवलविभक्तेः
प्रयोगानर्हत्वात् प्रकृतिराक्षिप्यते~। सा चग्निशब्दरूपा, न तु या काचित्,
निरुक्ते देवताकाण्डे{\qt अद्य किंदेवताः प्रयाजानुयाजाः} इति प्रश्नमवतार्य
मन्त्रवर्णादीनुदाहृत्य {\qt आग्नेया इति तु स्थितिः} (८~। २२~। ८)
इत्युपसंहारात्~। तस्मादादितश्चतुर्षु प्रयाजेषु चतस्रो
विभक्तयोsग्निशब्दप्रकृतिकाः पठ्यन्ते \textendash\ इत्यादि श्रौतग्रन्थेषु
द्रष्टव्यम् इति शब्दस्तुकौभस्य~। 

 {\qt ताश्च विभक्तयः} सूत्रे स्पष्टमुदाहृताः \textendash\ {\qt अग्नेऽग्नेऽग्नावग्ने
ऽश्निनाग्नेऽग्निमग्ने} इति चतुर्षु प्रयाजेषु चतस्रो विभक्तीर्ददाति
इति, समिधो अग्ने आज्यस्य वियन्तु, तनूनपादग्न आज्यस्य
व्येत्वित्याद्यासु प्रयाजयाज्यासु संबुद्ध्यन्ता अग्निशब्दा आम्नाताः~। 
तेभ्यःपुरस्तात्क्रमेण संबुद्धिसप्तमीतृतीयाद्वितीयाविभक्त्यन्ता
अग्निशब्दाः



१षष्ठीसप्तम्य एवेति श्रौतसंप्रदायः~॥ 



समिधोऽग्नेऽग्ने आज्यस्य व्येतु वौषट्, ये यजामहे भूर्भुवः सुवः
तनूनपादग्नावग्न आज्यस्य व्येतु वौषट् ये यजामहे भूर्भुवः
सुवरिडोsग्निनाsग्ने आज्यस्य व्येतु ( वौषट् ), ये यजामहे भूर्भुवः
सुवर्बर्हिरग्निमग्ने आज्यस्य व्येतु वौषट् इति~। प्रकृतौ तु \textendash\ {\qt समिधोऽग्ने
आज्यस्य व्येतु तनूनपादग्न आज्यस्यव्येतु इडोऽग्न आज्यस्य व्येतु
बर्हिरग्न आज्यस्य व्येतु} इति श्रुतौ सं
बुद्ध्यन्ताग्निशब्दघटितमात्रपाठः~॥ छान्दोग्यब्राह्मणे तु अन्यथा विभक्तय
आम्नाताः~। ताश्च हरिटीकायां प्रदर्शिताः~। तथा हि \textendash\ ये यजामहे समिधः
समिधोऽsग्नेsग्न भाज्यस्य \textendash\ व्येतु वौषट्, ये यजाभहे नराशंसोऽग्निमग्ने
आज्यस्य व्येतु वौषट्, ये यजामहे इडोऽग्निनाऽग्ने आज्यस्य व्येतु
वौषट्, ये यजामहे बहिर्रग्नि \textendash\ रग्न आाज्यस्य व्येतु वौषट् इति~॥ तत्र
तनूनपात्स्थाने नराशंसः प्रवरभेदेन स्थितः~॥ तथा
चकातीयसूत्रम् \textendash\ वशिष्ठशुनकवध्यश्व \textendash\ 
संकृतिराजन्यवैश्यकण्यात्रिकश्यपप्रजापशुकामानां नराशंसो द्वितीयः,
{\qt तनूपादन्येषाम्} इति~॥ आश्वलयनोsपि पुनराधेप्रकरणे
इत्थम्प्रयाजानुयाजान्विभक्तिभिर्यजेत \textendash\ समिधः समिधोऽग्रेऽग्न आज्यस्य
व्येतु, तनूनपादग्निममन आज्यस्य व्येतु, इलोऽग्निनाऽग्न आज्यस्य
व्येतु, {\qt बर्हिरग्निरग्न आज्यस्य व्येतु} इति~॥ वातीयब्राह्मणे तु
पुनराधेयप्रकरणे \textendash\ ये यजामहेऽग्न आज्यस्य व्येतु वौषट् ये
यजामहेऽग्निनाज्यस्य व्येतु ये {\qt यजामहेऽग्निराज्यस्य व्येतु} इत्युक्तम्
~॥ तत्सूत्रे तत्प्रकरणेऽप्येवम्~॥ तत्सूत्रे प्रकृते तु समिधः समिधो
व्येतु तनूनपादग्न आज्यस्य व्येतु नराशंसोऽग्न आज्यस्य व्येतु इति~। 


 \ldots\ldots\ldots\ldots\ldots.~। 
भारद्धाजसूत्रेऽप्येवम् \textendash\ इति बोध्यम्~॥ इदं सर्वं यथासंप्रदायं
 \textendash\ ग्रन्थादुपगन्तव्यम्~॥ तदेतन्मनद्वयं हृदि निधायाह \textendash\ [८ मप०]
विभक्तयश्चेत्यादिना~॥ 



प्रयुक्ताः इति प्रथमकाण्डे तृतीयप्रपाठके प्रथमानुवाके तैत्तिरीय \textendash\ 
ब्राह्मणसायणभाष्यस्य चानुरोधादवसीयते \textendash\ {\qt नागेशविरचितभाष्य \textendash\ प्रदीपोद्दयोते
केनच्चिद्विद्वत्तरेण षष्ठीविभक्तेरनाम प्रक्षिप्तम्} इति; तत्तु न सम्यक्
इति दाधिमथाः~॥ 

 येषु प्रवरेषु यासु विभक्तिषु कृतद्विर्वचनानां चतुरक्षरत्वं भवति
तद्विभक्त्यन्तानामेव प्रयोगश्च विहितत्वेन संबुद्धितृतीयाचतुर्थीषु
द्विर्वचने चतुरक्षरत्वाभावात् पञ्चम्या विशेषतो निषेधतश्च
प्रथमाद्वितीयाषष्ठीसप्तम्येकवचनान्तानामेन् प्रयोगः करणीयः ,
तत्प्रवरविहितविभक्तिसंग्रहार्थं षष्ठीविभक्तिनाम्नोsपि
सर्वशाखासाधारणेsत्रावश्यकत्वम् \textendash\ इति केचित्~॥ इदमेवोचितम् , तदुदाहरणं तु
{\qt आदीक्षिष्टायं ब्राह्मणः \textendash\ } इत्यस्य मन्त्रस्य प्रयोगकाले शेषत्वेन
ब्राह्मणनामथेयविशेषं तदीयप्रवरं पठन्ति \textendash\ असौ
देवदत्तोऽमुष्यपुत्रोऽमुष्यपौत्रोsमुष्य नप्ताऽमुष्याः पुत्रोऽमुष्याः
पौत्रोऽमुष्या नप्ता इति~। एतेपु प्रवरनामधेयेषु षष्ठया अप्यूहेन तस्या
अप्यावश्यकत्वमिति~॥ 

२ द्वितीयशङ्का सर्वविभक्तिविषया~। (र. मा. ) 

४० उद्द्योतपरिवृतप्रदीपप्रकाशितमहाभाष्ये~। [१ अ. १ पा. १
पस्पशाह्निके



 ( प्रयोजनभाष्यम् ) 

१यो वा इमाम \textendash\ 

 {\qt यो वा इमां पदशः स्वरशोऽक्षरशो २वाचं विदधाति स आर्त्विजीनो
भवति~। आर्त्विजीनाःस्यामेत्यध्येयं व्याकरणम्~॥ यो वा इमाम्~॥}

 ( प्रदीपः ) पदश इति~। पदं पदमिति {\qt संख्यैकवच} \textendash\ नाच्च वीप्सायाम् इति शस्
~॥स्वरश इति~। स्वरः \textendash\ उदात्तादिः~॥ अक्षरश इति~। अक्षरं \textendash\ व्यज्नसहितोsच्
~॥ आ र्त्विजीन इति~। 
ऋत्विजमर्हतीति$=$आर्त्विजीनः$=$यजमानः~। ऋत्विक्कर्मार्हतीति \textendash\ याजकोऽप्यार्त्विजीनः
~। {\qt यज्ञर्त्विग्भ्यां घखञौ} इति सूत्रेण {\qt यज्ञर्त्विग्भ्यां
तत्कर्मार्हतीति चोपसंख्यानम्} इति वातिंकेन च खञ~। {\qt विद्वान्यजेत
विद्वान्या जयेत्} इति द्वयोरपि विदुषोरधिकारात्~॥ 

 (उद्व्योतः) व्यञ्जनसहित इति~। यथा {\qt ये यजाभहे \textendash\ इति पञ्चाक्षरम्} इति~। 
{\qt वर्णं याहुः} पूर्वसूत्रे इति भाष्यात् वर्ण



 [ भाष्ये \textendash\ भाष्ये \textendash\ यो वा इमामिति~। यः वै \textendash\ निश्चयेन
इमां \textendash\ संनिहितामपरोक्षां वाचं \textendash\ वेदरूपां वार्णी पदं \textendash\ पदं स्वरं \textendash\ स्वरम्
अक्षरं \textendash\ अक्षरं विदधाति \textendash\ संस्करोति स आर्त्त्विजीनो भवतीत्यर्थः~॥ 

 [ भाष्ये ] \textendash\ ननु किमेतावताऽत आह \textendash\ [ भाष्ये ]आर्त्विजीना इति~। 
यजमाना ऋत्विजश्चेत्यर्थः~। ४ आर्त्विजीयायाम्
विदुषो \textendash\ विहितत्वेनाविदुषोरविहितत्वेन तत्सिद्धेर्व्याकरणायत्तत्वेन
व्याकरणेन वैदुष्यसिद्धौ तत्वसिद्धिरिति भावः~॥ तथा
चैहिकामुष्मिकफललाभायाध्येयं व्याकरणमिति तत्त्वम्~॥ न
चाध्यापकप्रातिशाख्यादितस्तज्ज्ञानमिति नेदं फलमिति वाच्यम्~। 
दत्तोत्तरत्वात्~॥ 

 [ प्रदीपे \textendash\ कैयटे \textendash\ शसिति~। एवमुत्तरयोरपि ज्ञेयम्~॥ अक्षर \textendash\ शब्दस्य
पृथगुक्तेरचः \textendash\ स्वरान \textendash\ इत्याह \textendash\ उदात्तेति~॥ ब्रह्मादिव्यावृत्तये
आह \textendash\ व्यज्जनेति~॥ हलामप्राधान्याद्वैपरीत्यशङ्का तु न कार्या~॥ 

 [ प्रदीपे ] \textendash\ कैयटे [ ७ मप० ] वार्तिकेन \textendash\ तेनैव~॥ इतीति~। इति
श्रुतेरित्यर्थः~॥ 

 [उद्द्योते १ मप०] \textendash\ त एवात्र मानमाह \textendash\ उद्द्योते \textendash\ यथेति~॥
भाष्यात् \textendash\ उत्तराह्निकस्थात्~॥ मात्रशब्दः कात्स्न्यैऽवधारणे वा~॥ 



१ अस्य प्रतीकस्य अ. पुस्तके न पाठः~। २ क्षरचश्च इति च. छ. पाठः~। ३
{\qt चत्वारि पदजातानि} इति च. छ. पाठः~। ४ क्रियायामिति शेषः~। 
ऋत्विकुक्रियायां यजमानक्रियायां चेत्यर्थः~। (र. ना. ) ५ {\qt शृङ्गेति
शेर्डादेशः} इति छायाकारोक्तिस्तु {\qt शेश्छन्दसि बहुलम्} इति शिलोपे नलोपे च
रूपसिद्धौ चिन्त्यैवेति दाधिमथाः~। लोपद्वयकल्पनापेक्षया जसो डादेशकल्पनमेव
वरं प्रक्रियालाघवादिति वयम्~। (र. ना. ) वस्तुतः
पण्डितद्वयोक्तमप्येतत्प्रामादिकम्~। शिलोपवादितः \textendash\ जसः शिआदेशे
नुमागमात्पूर्वमपरतनिमित्त \textendash\ 





मात्रमित्यन्ये~॥ सकृत्प्रयुक्तात्विजीनशब्दस्योभयपरत्वे युक्तिमाह \textendash\ ~। 
विद्वानिति~। वेदार्थज्ञ इत्यर्थः~॥ 

 ( प्रयोजनभाष्यम् ) 

चत्वारि \textendash\ 

चत्वारि शृङ्गा त्रयो अस्य पादा 

द्वे शीर्षे सप्त हस्तासो अस्य~। 

त्रिधा वद्धो बृषभो रोरवीति 

महो देवो मर्त्या आविवेश~॥ इति~। 

 ३चत्वारि शृङ्गाणि \textendash\ पदजातानि, नामाख्यातोप \textendash\ सर्गनिपाताश्च~॥ त्रयो अस्य
पादाः \textendash\ त्रयः कालाः भूतभविष्यद्वर्तमानाः~॥ द्वे शीर्षे \textendash\ द्वौ
शब्दात्मानौनित्यः कार्यश्च~॥ सप्तहस्तासो अस्य \textendash\ सप्त विभक्तयः~। त्रिधा
वद्धः \textendash\ त्रिषु स्थानेषु वद्धः \textendash\ उरसि कण्ठे शिरसीति~। वृषभो वर्षणात्~॥
रोरवीति$=$शब्दं करोति~॥ 



 [ २यभा०] \textendash\ अथास्यापवर्गसाधनत्वमाह \textendash\ भाष्ये \textendash\ चत्वारीति~॥ 

 [ २ यभा० ]भाष्ये \textendash\ शृङ्गेति~। शेर्डादेशः५~॥~। {\qt प्रकृत्यात्रयो}
अस्येति प्रकृत्यान्तःपादं इति प्रकृतिभावः~॥ शीर्ष~। शीर्षशब्दो.
ऽकारान्तः क्लीबः~॥ हस्तास् इत्यत्र आज्जसेः इत्यसुक्~॥ मह \textendash\ 
शब्दोऽकारान्तो {\qt घञर्थे कविधानं} इति कर्मणि कान्तः, महानित्यर्थः
~॥ चतुः शृङ्गाद्यवच्छिन्नवृषभरूपता शब्दस्य प्रमीयमाणाऽपि
प्रमाणबाधितेति गौण्या वृत्त्याऽबाधितार्धविषयतया व्याचष्टे [ ६ ष्ठप०
] चत्वारि शृङ्गाणीत्यादिना~। अत एव शब्दस्य वृषभत्वेन निरूपणमिति
कैयटेनोक्तम्~॥ 

 [ २ यभा० ] \textendash\ तत्कलितमाद्यमर्थमाह \textendash\ भाष्ये \textendash\ [६ ष्ठप० ]
चत्वारिपदजातानीति~॥ 

 [२यभा०] \textendash\ ६शृगान्तराण्याह \textendash\ भाष्ये [६ष्ठप०] \textendash\ नामेति~॥ 

 [ २ यभा० ] \textendash\ त्रिभिः प्रकारैर्बद्ध इति वाच्ये
तात्पर्यवृत्त्याऽऽह भाष्ये [ १० मप० ] त्रिष्विति~॥ बद्ध हृति~। 
व्यञ्जकध्वन्य \textendash\ नुत्पत्तावनभिव्यक्त इत्यर्थः~॥ 

 [२ यभा० ] \textendash\ भाष्ये [ १ शप० ] वृषभः~। वृषेरौणादिकोऽभच् किच्च
~। तदाह \textendash\ वर्षणादिति~॥ इदं प्राग्वद् बोध्यम्~॥ 



कत्वेनान्तरङ्गत्वात् {\qt शेश्छन्दसि बहुलम्} इति लोपे प्रत्ययलक्षणेन नुमि
प्रत्यक्षलक्षणेनैव दीर्घे तेनेव च नलोप इति पुनः पुनः प्रत्ययलक्षणरूपं
गौरवमेवदुरूहम्~। न त्वस्मिन् पक्षे किञ्चिदपि लाघवमिति स पक्षः पराहतः~। 
जसो डादेशवादिनस्तु \textendash\ जसो डादेशे कर्तव्ये प्राप्तस्य शिआदेशस्य बाधः, {\qt डा
आदेशः}, टिलोपश्चेति कल्पनावयमिति तदपेक्षया शेर्डादेशकल्पनं न गरीयः~। अतः
शेर्लोपपक्षे पुनः पुनः प्रत्ययलक्षणरूपमेव गौरवं प्रदर्शनीयमिति छायोक्तिः
सम्यगेव~॥ ६ परा \textendash\ पद्यन्तीमध्यमावैखरीरूपशृङ्गापेक्षयाsन्यानीत्यर्थः~। 
(र. ना. ) 

शास्त्रप्रयोजनाधिकरणम् ] महाभाष्यप्रदीपोद्दयोतव्याख्या छाया~। ४१ 



कुत एतत् ? 

रौतिः शब्दकर्मा~॥ 

महोदेवो मर्त्या आविवेशेति~। महान्देवः$=$शब्दः, मर्त्याः \textendash\ मरणधर्माणो
मनुष्याः, तानाविवेश~। महता देवेन नः साम्यं यथा स्यादित्यध्येयं व्याकरणम्
~॥ 

 ( प्रदीपः ) चत्वारीति शब्दस्य वृषभत्वेन निरूपणम्~॥ त्रयः काला
इति~। लडादिविषयाः~॥ नित्यः कार्यश्चेति~। व्यङ्ग्यव्यज्जकभेदेन
~॥ सप्त विभक्तय इति~। सुप इत्यर्थः~॥ केचित्तु
तिङामपरिग्रहप्रसङ्गात्सह शेषेण सप्त कारकाणि विभक्ति \textendash\ शब्दाभिधेयानि \textendash\ इति
व्याचक्षते~॥ वर्षणादित्ति~। कामानां ज्ञानपूर्वकानुष्ठानफलत्वात्~॥
महतेति~। परेण ब्रह्मणेत्यर्थः~॥ 

 (उद्क्योतः) चत्वारीति \textendash\ ऋड्मन्त्रः~॥ शब्दस्येति~। 
शब्दशास्त्रप्रतिपाद्यस्येत्यर्थः~। वृषभाकारः शब्दशास्त्रप्रतिपाद्यः
शब्दो रोरवीति~। 



 [१ मभाष्ये] \textendash\ भाष्ये [१मप०] कुत एतदिति~। उक्तोऽर्थः~। 
तस्य \textendash\ कस्मादित्यर्थः~। रोरूयमाणमर्थमाह \textendash\ भाष्ये [ ३ यप० ] महानिति~॥ 

 [ प्रंदीपे ] \textendash\ ननु कालस्यामूर्तत्वेन कथमवयवत्वमत आह \textendash\ कैयटे \textendash\ लडादीति~॥
तथा च लडादीनां दशत्वेन त्रित्वासंभवा \textendash\ त्तथा दुर्वचत्त्वेऽपि
तद्द्योत्यकालस्य प्रसिद्धतरस्य त्रित्वात्तत्र विहितत्वेन लडादीन् त्रेधा
विभज्य तद्वद्योत्तकास्त्रेधा लडादयः पादा इत्यर्थः~॥ तेषामप्रयोगेऽपि
तदादेशानां तत्संज्ञकानां प्रयोगसमवायात्~॥ एतेन
{\qt कालस्यैकत्वादवान्तरभेदेनाधिक्याच्चेदं चिन्त्यम्} हत्यपास्तम्~। 
उक्तहेतोः, तावतैव प्रसिद्धतरेण सर्वेषां ग्रहणे सिद्धेऽवान्तरभेदग्रहणे
मानाभावाच्च~॥ यत्तु \textendash\ शब्दस्यात्मानौ {\qt ममायमात्मा} इत्यादाविवाभ्यहितत्वेन
प्राधान्यमात्मशब्देनोच्यते स्वरूपत्वेन च प्राधान्यम्~। तेन द्वौ शब्दौ
व्यज्जकौ स्फोटनादौ शीर्षे~। केचित्तु \textendash\ शब्दस्यात्मानौ वाच्यतया
प्रधानभूतावर्थौ जातिव्यक्ती \textendash\ इत्याहुः \textendash\ इति कृष्णः~॥ तन्न \textendash\ इति
ध्वनयन्नाह \textendash\ कैयटे \textendash\ [ ३ यप० ] व्यङ्ग्येति~॥ 

 [४ प० ]कैयटे केचित्विति~। अत्रारुचिबीजं तु त्रयः काला इत्यनेनैव
तेषां संग्रहः, अन्यथा~॥ 

 [ उद्थोते ] \textendash\ प्रसिद्धशब्दस्य
तादृशशृङ्गाद्यवयवकत्वासंभवादाह \textendash\ उद्द्योते \textendash\ शब्दशास्त्रेति~। शब्दब्रह्मण
इत्यर्थः~॥ {\qt स मर्त्यानाविवेशेत्याश्चर्यमत्र प्रतिपाद्यम्} इति
रत्नकृदुकत्यसांगत्यं ध्वनयन्नादौ मन्त्रतात्पर्यार्थमाह \textendash\ [ २ यप० ]
वृषेति~॥ शब्द इति~। शब्दब्रह्मरूप इत्यर्थः~॥ रोरवीति \textendash\ शब्दं करोति,
किमित्यत आह \textendash\ [ ३ यप० ] महानिति~॥
वक्ष्यमाणभाष्यानुरोधेनाह \textendash\ स्वाभेदमिति~। तेष्विति भावः~॥ नन्वेतावताऽतपे
प्रकृते किमत आह \textendash\ [ ४ र्धप० ] महत इति~॥ अभेदस्य प्रतिपाद्यत्वादाह
[ ५ मप० ] इवेति~॥ यद्वा यथाकथंचिद्व्याकरणज्ञस्य
तदाविष्टत्वभावादाह \textendash\ इवेति~॥ 



 १ तिङामिति~। चत्वारीत्यादिभाष्यव्याख्याने तिङामपरिग्रहान्न्यूनतावारणाय
{\qt सप्तविभक्तयः} इत्यनेन तिङामपि ग्रहणम्~। शेषः षष्ठ्यर्थं एकः ,
कर्त्रादिकारकाणि षट् \textendash\ इति सप्त भवन्ति~। कारकपदेन च तिडां
शेषातिरिक्तसुपाञ्च ग्रहणम्~। एवञ्च सप्तविभक्तय इत्यनेन सुपां तिङाञ्च
ग्रहणमिति भावः~॥ २ {\qt स्यर्मूर्धन्या} इति च. पाठः~। ३ अत्र मूल \textendash\ 

६ प्र० पा०





महान् देवः \textendash\ अन्तर्यामिरूपः शब्दो मर्त्यानाविवेश \textendash\ स्वाभेदमाविष्कृत \textendash\ वानिति
मन्त्रतात्पर्यम्~। महतो देवस्य \textendash\ शब्दब्रह्मणो व्याकरणज्ञाप्यतया
व्याकरणज्ञस्तदाविष्ट इव भवतीति यावत्~॥
भाष्ये \textendash\ पदजातानि$=$परापश्यन्तीमध्यमावैखरीरूपाणि~। अत एवाग्रे निपाताश्च इति
चकारः संगच्छते~॥ नामाख्यातेति~। नामशब्देन सुबन्तं, नमति$=$आख्यातार्थं
प्रति विशेषणीभवतीति व्युत्पत्तेः~। आख्यातं \textendash\ तिङन्तम्~। उपसर्गनिपातयोः
पृथुगुपादानं गोवलीवर्दन्यायेन~। च्यङ्ग्यः \textendash\ आन्तरः~। व्यञ्जकः \textendash\ वैखरीरूपः~॥
तिङ्विभक्तीनां पटत्वदाह \textendash\ सुप इति~॥ सह शेषेणेति~। शेषः \textendash\ षष्ठयर्थः,
तत्सहितकारकाभिधायकत्वेन सुप्तिङोरुभयोरपि संग्रह इति भावः~॥
भाष्ये \textendash\ उरसीत्यादि~। 

 {\qt हकारं पञ्चमैर्युक्तमन्तस्थाभिश्च संयुतम् 

 उरस्यं तं विजानीयात्~।}

 {\qt २मूर्धन्या ऋटुरषाः}



 [ उद्द्योते ]{\qt पदजातान्येव दर्शयति \textendash\ नामेत्यादि} इति
दण्डिकृष्णाद्युक्त्यसांगत्यायाह \textendash\ उद्दयोते [ ६ ष्ठप० ] परेति~॥
तद्वीजमाह \textendash\ अत एवेति~॥ 

 [ उद्दयोते ] \textendash\ विविक्ते तदर्थमाह \textendash\ उद्दयोते [ ७ मप० ] नामेति~॥
व्युत्पत्तेरिति~। ३सा च स्वार्थद्वारा सुबन्तस्यैवेति भावः~॥
आख्यातं \textendash\ तच्छब्दप्रतिपाद्यम्~॥ निपातेनोपसर्गग्रहणस्य सिद्धत्वादाह \textendash\ [ ९
मप० ] उपसेति~॥ गोबलीति~। इदं च कैयटादिरीत्या~। सिद्धान्ते तु
ब्राह्मणवसिष्ठेति बोध्यम्~॥ तथा चाद्येन धात्वर्थद्योतकसर्वा \textendash\ 
अन्त्येन वाचकं त इति भेदः षतेन \textendash\ {\qt आख्यातशब्देनोपसर्गग्रहणात् तेन
कर्मप्रवचनीयग्रहणम्} \textendash\ इति कृष्णोक्तमपास्तम्~। तेषामपि तद्विशेषद्योतकत्वेन
तेनैवान्यवद्ग्रहणसंभवात्~। पृथकप्रयोगाभावेऽप्याख्यातान्तर्भावासंभवाच्च
~॥ 

 [उद्द्योते]केचिन्मतरीत्या कैयटयोजनानिरासाय तद्या \textendash\ चष्टे \textendash\ 
उद्दयोते \textendash\ [ १० मप० ] आन्तर इति~। मध्यमावस्थ इत्यर्थः~॥ [११ शप०
]षट्त्वादिति~। त्रिकाणामित्यादिः~। {\qt त्रयः कालाः} इत्यनेन तेषां
संग्रहाच्चेत्यपि बोध्यम्~॥ सुप इतीति~। {\qt ख्रौजस् \textendash\ }
इत्यादिप्रथमादित्रिकरूपार इत्यर्थः~॥ 

 [ उद्द्योते ] एवं च न वैरूप्यं नापि काचनामुपपत्तिरिति
कैयटादयश्चिन्तया एवेति बोध्यम्~। तदेतद्ध्वनयन्नाह \textendash\ उद्वयोते [ १२ शप०
] इति भाव इति~। अत एव {\qt षट्त्वात्} इत्येव हेतुः पूर्वमुक्तः~॥ आदाने
यथा हस्तस्य व्यापारः, तथा प्रातिपदिकार्थाभिधाने विभाक्तिव्यापार इति
तासां हस्तत्वमिति भावः~॥ 

 [उद्द्योते] \textendash\ ननूरसो न स्यानत्वं वण्ठादिबहिर्भावादत आह \textendash\ उद्द्योते [
१४ शप० ] हकारमिति~॥ यत्तु रत्नकृत् \textendash\ अनुस्वारयमानां नासिकारूपं शिरः
स्थानम् \textendash\ इति~। तत्र, तयोर्भेदस्य स्पष्टत्वात्~। 
तद्ध्वनयन्नाह \textendash\ स्युर्मूर्धन्या इति~॥ अयं चाग्रे हेतुः~॥ 



पुस्तकपाठ सन्दिग्धः {\qt साञ्चस्यार्थद्वा} \textendash\ इत्युपलभ्यते~। ; {\qt गोशब्दः
स्त्रीगवीणामेव वाचकः प्रायेण} इत्येकशेषप्रकरणे वक्ष्यमाणत्वादत्र कथं
{\qt गोवलीवर्दन्यायेन} इत्युक्तमत आह \textendash\ कैयटादीति~॥ दाधि०~॥ 

५ अत्र पाठः खण्डितः~। {\qt द्योतक्तसर्वेषां ग्रहणम् , अन्त्येन वाचका \textendash\ 
नामिति भेदः} इत्याशयकःतर्कगोचरः~॥ 

४२ उद्ह्योतपरिवृतप्रदीपप्रकाशितमहाभाष्ये \textendash\ 

 [ १ अ. १ पा. १ पस्पशाह्निके



इत्युक्तेमूर्धैव शिरः~॥ {\qt कण्ठे} इत्यनेन मुखान्तर्गतकण्ठादिस्थान \textendash\ 
मुपलक्ष्यते~॥ ननु वर्षति$=$मूत्रेण भूमिं सिञ्चतीति \textendash\ वृषभ इत्युच्यते,
तत्कथं शब्द एवम् ? इत्यत आह \textendash\ कामानामिति~। मेघ उदक \textendash\ 
मिवोक्तचतुश्शृङ्गादिगुणवैशिष्ट्येन ज्ञानपूर्वकं प्रयोगेण कामानां
दानादित्यर्थः~। द्विशृङ्गद्विपाद्धिकरकण्ठमात्रवद्धलौकिकवृषभादस्य
व्यतिरेकः~॥ किं रोरवीति, तत्राह \textendash\ भाष्ये \textendash\ महान् देव इत्यादि~। 
महान् \textendash\ परब्रह्मस्वरूपो देवः$=$अन्तर्यामिरूपः शब्दो मर्त्येष्वाविष्ट
इत्यर्थः~। तदावेशफलमाह \textendash\ महता देवेनेति~। साम्यं$=$सायुज्यम्~। तदुक्तं
हरिणा \textendash\ 

अपि प्रयोक्तुरात्मानं शब्दमन्तरवस्थितम्~। 

प्राहुर्महान्तमृषभं येन सायुज्यमिष्यते~॥ 

आह च \textendash\ चत्वारि शृ्ङ्गा त्रयो अस्य पादा इत्यादि ऋग्मन्त्रः इह द्वौ
शब्दात्मानौ \textendash\ कार्यौ नित्यश्च~। तत्रान्त्यः सर्वेश्वरः सर्व \textendash\ शक्तिर्महान्
शब्दवृषभः,तस्मिन् खलु वाग्योगवित् शास्त्रजशब्दज्ञानपूर्वकं प्रयोगेण
क्षीणपापः पुरुषो विच्छिद्याहंकारग्रन्थीन् अत्यन्तंसंसृज्यते? इति तदर्थं
तद्व्ख्यातारः~॥



 [उद्द्योते]एवेनान्यनासिकादिव्यावृत्ति~॥न्यूनताम्ं परिहरति \textendash\ कण्ठ
इति~॥ उद्द्योते[१७शप०]इत्युच्यत इति~। लोके इति
शेषः~॥एवं \textendash\ वृषभः~। तत्वस्याभावात् ~। तथा च
ज्ञानपूर्वकप्रयोगेणफलस्य कामस्य सकलस्य वर्षणादिति भाष्यार्थो
बोध्यः~। तदाह \textendash\ उद्दयोते [ १८ शप० ] मेघ इति~॥ इवशब्दोत्तरम् अयं
शब्दरूपो वृषभः सर्वान् कामान् वर्षति इति शेषः~॥ अत्र हेतुं
प्रतिपादयन् कैयटार्थमाह \textendash\ वोक्तेति~॥ एवं च वृषभस्यातिशयः~॥
कामानां \textendash\ स्वर्गमोक्षादीनाम्~। एतावतैवास्य कामवर्षकत्वम्~॥
अतिशयवैशिष्ट्यादेव ततोऽस्य वैलक्षण्यमाह \textendash\ द्विशृङ्गेति~॥ अग्रिमयोः
करत्वेन ग्रहणात् {\qt द्विपात्} इत्युक्तम्~॥ 

 [ उद्दयोते ]तदर्धमाह उद्वयोते [ २१ शप० ] महानिति~। 
एतेनाविष्ठः सन् शब्दान् करोतीति मन्त्रार्थः~॥ न च शब्दस्य शब्दे कथं
कर्तृत्वम्~। न वाय्वादिशब्दाभिप्रायेण तत् , अनित्यतापत्तेरिति वाच्यम्~। ३
तेन क्रियाविशेषस्य लक्ष्यत्वेन शब्दो ध्वनिः स्वव्यञ्जकमपेक्षत
इत्यर्थसंभवादिति तथाङ्गीकारात्~॥ यद्वा \textendash\ परादिषु पूर्वस्याः परत्र
कर्तृत्वम्~॥ [२३ शप० ] फलमाहेति~। तत्फलं सूच्चयन् एतावता प्रकृते
किमायातम् \textendash\ तदाहेत्यर्थः~॥ महता देवेनेतीति~। शब्दब्रह्मणेत्यर्थः~॥
यत्तु \textendash\ साम्यम् \textendash\ अवैकल्यं पूर्णतेति यावत् \textendash\ इति रत्नकृत्~। तन्न \textendash\ इति
ध्वनयन्नाह \textendash\ सायुज्यमिति~। ऐक्यमित्यर्थः~॥ इदमेव ध्वनयन्नत्र
संमतिमाह \textendash\ तदुक्तमिति~। [ २६ शप० ] मिष्यते इति~॥ इति पाठः~॥
तद्व्ख्यातृग्रन्थमाह \textendash\ [ २७ शप० \textendash\ ] आह चेति~॥ उक्तमर्धमाह \textendash\ [ २८
शप० ] इहेति~। मन्त्रइत्यर्थः~। प्रतिपाद्यविति शेषः~। कार्यः \textendash\ व्यञ्जको
वैखरीरूपः~॥ नित्यः \textendash\ व्यङ्ग्य आन्तरो मध्यमावस्थः~॥ अयं
चाविद्यावृत्तत्वेन घटस्थितदीप इ्व नानाविषयान् भासयन्४



१ निपाताश्चैति~। चशब्दोऽयं परापश्यन्तीमध्यमावैखरीणामपि संग्रहं
बोधयति; अन्यथा तस्यानर्थक्यापत्तेः~। पदजातानीत्यस्य नमेत्याद्येव
व्याख्यानसत्वे चशब्दानुपयोगः~। नामाख्यातादीनामपि परेत्यादिरूपचतुष्टयं
वर्तते एव~। एवमेव पूर्वस्मिन्नपि भाष्यव्याख्याने बोध्यम्~॥ . २ {\qt तुरीयं
हृवा एत \textendash\ } इति पाठच्छ्याकारदृष्टः~॥ 





 (भाष्यम् ) 

अपर आह \textendash\ 

चत्वारि वाक्परिमिता पदानि 

तानि विदुर्ब्राह्मणा ये मनीषिणः~। 

गुहा त्रीणि निहिता नेङ्गयन्ति 

तुरीयं वाचो मनुष्या वदन्ति~॥ 

{\qt चत्वारि वाक्परिमिता पदानि} चत्वारि
पदजातानि \textendash\ नामास्यातोपसर्गनिपाताश्च१~॥ {\qt तानि विदुर्ब्राह्मणा ये मनीषिणः~।} 
मनस ईषिणः \textendash\ मनीषिणः~। {\qt गुहा त्रीणि निहिता नेङ्गयन्ति} गुहायां त्रीणि
निहितानि नेङ्गयन्ति $=$न चेष्टन्ते, न निमिषन्तीत्यर्थः~॥ {\qt तुरीयं वाचो}
मनुष्या वदन्ति~॥ तुरीयं२ वा एतद्धाचो~। यन्मनुष्येषु वर्तते~। 
चतुर्थमित्यर्थः~॥ चत्वारि~। 

 (प्रदीपः) {\qt चत्वारि} इत्यनेनैकदेशेन सदृशेन वाक्यान्तरमपि सूच्यत
इत्याह \textendash\ अपर आहेति~। परिमितानि \textendash\ इति प्राप्ते {\qt शेश्छन्दसि बहुलम्} इति
शेर्लोपे परिमिता \textendash\ इति भवति~। परे \textendash\ 



\ldots तदाह \textendash\ सर्वेश्वर इत्यादि~॥ [ २९ शप० ] तस्मिन्नित्यस्य
संसृज्यत इत्यत्रान्वयः वाग्योगविद् \textendash\ उक्तलक्षणो वैयाकरणः~॥ 

 [भाष्ये ] \textendash\ ईषन्ति ते \textendash\ ईषिणः, औणादिक इनिः कर्तरि~। 
शकन्ध्वादित्वात्पररूपम्~। मनस इतिं कर्मणि षष्ठी~॥ {\qt सुप्यजातौ} इत्यस्य
तु न प्राप्तिः, मनःशब्दस्य जातिवाचकत्वात्~। तदेतदभिप्रेत्याह \textendash\ भाष्ये [
८ मप० ] \textendash\ मनस ईषिण इति~॥ अन्यथा मन ईषन्ति ते \textendash\ मनीषिण इत्युक्तं स्यात्~॥


 [ भाष्ये ] \textendash\ भाष्ये [ ९ मप० ] \textendash\ निहितेति~। परिमितेतिवत्~। 
नेङ्गयन्तीत्यत्र हेतुरयम्~। यतो गुहायां त्रीणि परादीनि
त्रिभिरंशैरवच्छिन्नानि च नामादीनि एदानि निहितानि अतो नेङ्गयन्ति, अस्यैव
व्याख्यानं {\qt न चेष्टन्ते} इति, अस्य न प्रकाशन्त इत्यर्थः~॥ [ १०
मप० ] न निमिषन्तीत्यर्थ इति~। स्फुटप्रति. पत्त्यर्थमनेकपर्यायोपादानम्
~॥ 

 [ भाष्ये ] \textendash\ नन्वेवं कथं मनुष्याणामवैयाकरणानां शब्देन व्यवहारोऽत
आह \textendash\ भाष्यै [ १ १ शप० ] \textendash\ तुरीयमिति~। वाचः परादेर्नामादेश्च
प्रत्येकं तुरीयं चतुर्थ भागं वैयाकरणतोऽन्ये मनुष्या वदंन्तीत्यर्थः~॥ 

 [भाष्ये ]नन्वतादिलौकिकवैदिकव्यवहारे कथं भागत्रयो त्सादोऽत
आह \textendash\ भाष्ये [ ११ शप० ] तुरीयमिति~॥ तदर्थमाह \textendash\ [ १२ सप० ]
चतुर्थमिति~॥ ह वै इति प्रसिद्धौ~॥ वै इत्येव पाठ वैशब्दो निश्चये~॥


 [ प्रदीषे ] \textendash\ कैयटे [ १ मप० ] नैकदेशेनेति~। प्रतीक \textendash\ ग्रहणकाल
उक्तेनेत्यर्थः~॥ तत्सूचने बीजमाह \textendash\ सदृशेनेति~॥ वाक्यान्तरमिति~। एवं
च प्रतिपाद्यभेदाद्वक्तृभेदेन व्यपदेश इत्याशयेन {\qt अपर आह} इत्युक्तमिति
भावः~॥ वस्तुतस्तु प्रोक्तमत्रस्य यज्ञादावग्निविषयत्वेनान्यत्र
वर्णितत्वात् केनचिच्छिष्येणोक्ते 



३ तेन \textendash\ कर्तृत्वेन~। अत्रापि सन्दिग्ध एव पाठ उपलभ्यते \textendash\ {\qt तेन क्रियाविशेषस्य
लक्ष्यत्वेन शब्दो ध्वनिः स्वव्यञ्जकमपेक्षत इत्यसम्भवादिति
कृथाङ्गीकारात्} इति~॥ ४ अपि न गृह्यते व्यञ्जकं विनेत्यर्थकः पाठो भाति~। 
(र. ना. ) 

शास्त्रप्रयोजनाधिकरणम् ] महाभाष्यप्रदीपोद्द्योतव्याख्या छाया
 ~। 

 ४३



मितानि$=$परिच्छिन्नानि, एतावन्त्येवेत्यर्थः~॥ मनीषिशब्दः
पृषोदरादित्वात्साधुः~॥ कथं मनीषिण एव विदन्तीत्याह \textendash\ गुहेति~। अज्ञानमेव
गुहा, तस्यामित्यर्थः~। {\qt सुपां सुलुक्} इति सप्तम्या लुक्,
व्याकरणप्रदीपेन तु तानि प्रकाशन्ते~। तत्र चतुर्णां पदजातानामेकैकस्य
चतुर्थं भागं मनुष्याः$=$अवैयाकरणा वदन्ति~। नेङ्गयन्तीत्यस्यैः व्याख्यानं न
चेष्टन्ते, न निमिषन्तीति~॥ 

 (उद्व्योतः) एतावन्त्येवेति~। वाक्परिमित्तानीति षष्ठीतत्पु \textendash\ रुषः
~। पदजातानि \textendash\ परापश्यन्तीमध्यमावैखर्यः, नामादीनि च~। नामादिमध्ये च
एकैकं चतुष्पादम्~॥ भाष्ये \textendash\ मनस ईषिणः \textendash\ 



मंत्रे भगवता व्याख्याते छात्रान्तर्गतेनान्येनासंदिग्धवाग्विषयमन्त्रे
उक्ते तं व्याख्यातुमाह \textendash\ {\qt अपर आह} इतीति बोध्यम्~॥ भगवतै \textendash\ 
वास्यान्यमतत्वेनोक्तिरित्यपरे~॥ 

 [ प्रदीपे ] \textendash\ कैयटे [५ मप० ] इत्याहेति~। इत्यतस्तत्र 

हेतुमाहेत्यर्थः~॥ 

 [ प्रदीपे ] गुहापदार्थमाह कैयटे [६ ष्ठप० ] अज्ञानमेवेति~। 
शास्त्राद्यज्ञानमपीत्यर्थः~। एवोऽयर्थः~। अत एव्र {\qt गुहा अज्ञानं
हृदयादिरूपा च} इत्युद्वयोत उक्तम्~॥ गुहायामिति प्राप्तं आह \textendash\ सुपामिति~॥
तेषां वेदने हेतुमाह \textendash\ व्याकेति~॥ 

 [ प्रदीपे \textendash\ तदाह \textendash\ कैयटे [ ७ मप० ] तत्रेति~। तेषामित्यर्थः~॥ 

 [ उद्वयोते ]वागिति सुपां सुलुक \textendash\ ? इति षष्ठ्यन्तमित्या \textendash\ 
शयेनाह \textendash\ कैयटे \textendash\ परिमितेत्यादि~॥ कृष्णोsप्येवम्~॥ तदसत्, न्यायेनैव
निर्वाहे तत्कल्पनानौचित्यात् इति ध्वनयन्नाह उद्दयोते \textendash\ वाक्परीति~॥ तथाच
वाचः \textendash\ शब्दस्य परिमितानि \textendash\ परिच्छिन्नानि तावन्त्येव पदानि सन्ति तान्येव
विदन्तीत्यर्थो वोध्यः~॥ 

 [उद्ध्योते \textendash\ प्राग्वत् कृष्णाद्युक्तेरसांगत्याय चत्वारि वाक्प \textendash\ 
रिमिता पदानीत्यस्याद्यार्थस्य भाष्योक्तमर्थमाह उद्द्योते \textendash\ [ २ यप० ]
परेति~॥ चत्वारि वाक्परिमिता पदानीत्यस्य द्वितीयार्थ भाष्योक्तं
विशेषं वक्तुमनुवदति \textendash\ नामादीनि चेति~॥ चरमपादानुरोधेनाह \textendash\ नामादीनां मध्ये
चेति~॥ चतु्पादमिति~। परादिभिरेवेति भावः~। ये ब्राह्मणा मनीषिणश्च ते
तानि सर्वाणि पदजातानि विदन्ति प्रवचनावृत्त्या पारगमनस्य
वर्णान्तरेष्वभावात् ब्राह्मणा इत्युक्तम्~॥ यद्वा \textendash\ वेदाख्यं ब्रह्म
साङ्गोपाङ्गं विदन्तीति ब्राह्मणाः~। ब्राह्मोऽजातौ? इति
टिलोपाभावश्छान्दसः~। 

 [उद्द्योते ] \textendash\ अत एव तथैव व्याचष्टे उद्द्योते \textendash\ [ ४ र्थप० ]



१ नामाख्यातोपसर्गनिपातश्चेतिच शब्देन परापश्यन्तीत्यादी \textendash\ नामपि संग्रहात्
परादीनामित्युपादानम्~। तत्र त्रयोंशा मध्यमान्ता हृदयादिरूपायां गुहायां
वर्तन्ते इति चतुर्थं भागं वैखरीरूपं मनुष्या वदन्तीत्यन्वयः~। 
अवैयाकरणोऽज्ञानात् वैखरीरूपमेव जानाति व्यवहरति च~। वैयाकरणस्तु
परादिमध्यमान्ता जानाति, व्यवहरति च वैखरीरूपेणैवेति तात्पर्यम्~॥ २
अज्ञानमिति~। अवैयाकरणपक्षे गुहा \textendash\ अज्ञानम् ~।अवैयाकरणपक्षे
हृदयादिरूपेत्यर्थः~। वैयाकरणानामज्ञानासम्भवात्~॥ ३ ननु भाष्यकृतः स्वत
एव प्रमाणत्वेन तत्र हर्यादीनामुपन्यासो वृथेति शङ्कावारणायाह \textendash\ हरिरपीति~। 
अपिशब्देन हरिसंमतिप्रदर्शन 





चित्तशुद्धिक्रमेण वशीकर्तारः, विषयान्तरेभ्यो व्यावृत्त्या हिंसका वा~। 
ते च वैयाकरणाः~॥ अन्थेषां तदज्ञाने हेतुमाह \textendash\ गुहेति~। 
एकैकस्य \textendash\ नामादिरूपस्य चतुर्थां भागम्, एवैकस्य चतुरंशत्वात्~। १ परादीनां वा
त्रयोंsशा मध्यमान्ताः~। गुहा \textendash\ अज्ञानं, हृदयादिरूपा च~। वैयाकरणस्तु
शास्त्रबलेन \textendash\ तद्वललब्धयोगेन च गुहाऽन्धकारं विदार्य सर्वं जानातीति भावः~॥
न निमिषन्तीति \textendash\ ज्ञानसामान्य \textendash\ विषयत्वाभाव उच्यते~॥ भाष्ये \textendash\ यन्मनुष्येषु
वर्वते \textendash\ ज्ञानविषयतया तत् वाचश्चतुर्थमित्यन्वयः~। तस्मात्सकलपदज्ञानाय
मनीषित्वाय वा \textendash\ ऽध्येयं व्याकरणमिति शेषः~। ३हरिरप्याह \textendash\ 



चित्तेति~॥ ४ईष ऐश्वर्ये इत्यस्य रूपमित्याशयेनाह \textendash\ वशीति~॥ ईष
गतिहिंसादर्शनेष्वित्यस्य रूपमित्याशयेनाह \textendash\ विषेति~। मुख्यहिंसाया
असंभवादिति भावः~॥ ते चान्ये न \textendash\ इत्याह \textendash\ ते चेति~। वैयाकरणा एवेत्यर्थः~। 
वक्ष्यमाणहेतोरिति भावः~। अत एवाह \textendash\ अन्येषामिति~॥ अयं त्व
कैयटीयैवकारसूचितोऽर्थः~॥ मनीषिशब्दः पृषोदरादित्वात्साधुरिति
कैयटस्तु {\qt शकन्ध्वादित्वं नैव कल्प्यम्, कल्पनायां वाsविशेषः?} इत्याशयकः~॥


 [उद्द्योते] \textendash\ एवं द्वितीयव्याख्यानाशयेनोक्ते कैयटे न्यूनतां
परि \textendash\ हरन् पदानीत्यस्याद्यव्याख्यानाशयेनाह \textendash\ उद्दयोते [ ७ मप० ] परेति
~॥ एवं च तासां मध्ये चतुर्थभागं वैखर्याख्यं मनुष्या वदन्तीत्यर्थः~॥ [
७ मप० ] आदिना ५नाभ्याधारयोर्ग्रहणम्~॥ तेषां वेदने कैयटोक्तहेतुं
विशदयति \textendash\ वैयेति~॥ नामादिपक्षे आह \textendash\ शास्त्रेति~॥ परादिपक्षे आह \textendash\ तद्वलेति
~॥ गुहाsन्धेति~॥ अज्ञानान्धकारं हृदयाद्यन्धकारं चेत्यर्थः~॥ ननु {\qt न
निमिषन्ति} इत्यस्य कथं तद्व्याख्यात्वमर्थभेदादत आह \textendash\ [ ९ मप० ]
ज्ञानेति~। धातूनामनेकार्थत्वादिति भावः~॥ 

 [उद्द्योते] \textendash\ तस्याकाशवृत्तित्वेन मनुष्यावृत्तित्वादाह \textendash\ उद्द्योते 

 [ १० मप० ] ज्ञानेति~। प्रतिपुरुषं व्यवहारसाधनत्वेन त्वेषु
यद्वर्तते तत्तुरीयम्~। नानादेशेषु नानाशाखासु च कथं कथमपि
विप्रकीर्णाञ्छव्दानन्विष्यापि तुर्यभाग एव ज्ञायते नान्य इति
प्रतिपुरुषसमवेताज्ञानविषयत्वोक्तौ नानादिव्यवहारविरोधः~॥ भाष्ये न्यूनतां
परिहरति उद्दयोते [ ११ शप० ] तस्मादिति~॥ नामादि \textendash\ पक्षे आह \textendash\ सकलेति~॥
परादिपक्षे आह \textendash\ मनीति~॥ {\qt य च} इति पाठः~॥ {\qt वा} इति पाठे स समुच्चये~॥
अत्रैव संमतिमाह \textendash\ [ १२ शप० ] हरिरिति~॥ एतद् \textendash\ व्याकरणम्~। अन्यतो
विशेषा



एव तात्पर्यम्~। सा च भाष्यकृदुक्तग्रन्थस्य व्याख्यानं मदीयमुचित \textendash\ 
मित्यत्राथ~। वैखरीमध्यमापश्यन्तीरूपायास्त्रय्या वाचः शष्दाद्यनेक \textendash\ 
भेदभिन्नाया अद्भुतं \textendash\ अपूर्व सामान्यविशेषवत् प्रकृतिप्रत्ययादिविलक्षणं
व्याकरणमेव परं \textendash\ उत्तमं स्थानमित्यर्थः, योगिनामपि परायां भेदज्ञानं
प्रकृतिप्रत्ययादिरूपं न भवतीति त्रय्या वाचो भेद उक्तः~॥ ४ ईष उञ्छे
इत्यस्यैव वशीकरणस्य यथाकथंचिदादानरूपत्वेन संभवादेतद्रूपमिति बोध्यम्~। 
ऐश्वर्यार्थकस्य तालव्यान्तत्वात्~। इति दाधिमथाः~॥ ५ आधारो मूलाधारचक्रं
परास्थानं, नाभिः पश्यन्त्याः, हृदयं मध्यमायाः~॥ 

४४ उद्द्योतपरिवृत्तप्रदीपप्रकाशितमहाभाष्यम्~। 

 [१ अ. १ पा. १ पस्पशाह्निके 



वैखर्या मध्यमायाश्च पश्यन्त्याश्वैतदद्धुतम्~। 

अनेकतीर्थभेदायास्त्रय्या वाचः परं पदम्~॥ 

 तत्र श्रोत्रविषया वैखरी~। मध्यमा हृदयदेशस्था
पदादिप्रत्यक्षानुपपत्त्या व्यतहारकारणम्~। पश्यन्ती तु
लोकव्यवहारातीता~। योगिनां तु तत्रापि प्रकृतिप्रत्ययविभागावगतिरस्ति~। 
परायां तु नेति त्रय्या इत्युक्तम्, 

१स्वरूपज्योतिरेवान्तः परा वागनपायिनी~। 

तस्यां दृष्टस्वरूपायामधिकारो निवर्तते~॥ इत्युक्तेः~। 

 अनेकतीर्थभेदायाः \textendash\ नामाख्यातादि \textendash\ वीणावेणुमृदङ्गशब्दादिरूपानेकभेदायाः~। आह
खल्वपि \textendash\ चत्वारि वाक्परिमिता पदानि \textendash\ इति 



दाह \textendash\ अद्भुतमिति~। तस्या वाच एतद् \textendash\ व्याकरणमद्भुतं परम् \textendash\ उत्कृष्टं
पद \textendash\ स्थानमित्यन्वयः~। {\qt पदम्~॥ इति} इति पाठः~॥
तद्व्याख्यातृग्रन्थमाह \textendash\ [ १५ शप० ] तत्रेति~। तासां मध्य इत्यर्थः~॥
श्रोत्रेति~। स्व \textendash\ परान्यतदीयेत्यादिः~। मध्यमैव चास्यपर्यन्तं गच्छता तेन
वायुना कण्ठदेशं गत्वा मूर्धानमाहत्य परावृत्य तत्तत्स्थानेष्वभिव्यक्ता
परश्रोत्रेणापि ग्रहणयोग्या विराडधिदैवत्या वैखरी वागित्यर्थः~॥ प्रणव एव
व्यानोदानाभ्यां सह वैखरीरूपं प्रतिपद्यते इति यावत्~॥ इयं च
वाय्वादिपरिणामरूपा कण्ठताल्वाद्यभिघातजन्यैव~॥ अत एव \textendash\ 

वायोरणूनां ज्ञानस्य शब्दत्वापत्तिरिष्यत्ते~। 

कैश्चिद्दर्शनभेदोऽत्र प्रवादेष्वनवस्थितः~॥

 इति हरिः~॥ वायोः \textendash\ प्राणादिरूपस्य~। अणूनां \textendash\ शब्दतन्मात्र \textendash\ परमाणूनाम् , 

 {\qt ५} अभ्राणीव प्रचीयन्ते शब्दाख्याः परमाणवः~। इति तदुक्तेः~॥
ज्ञानस्य वक्तृज्ञानस्य, पराशक्तिसाहित्येन च तत्तत्परिणाम इति बोध्यम्~॥
कैश्चिदित्यनेन पराशक्तेरेव मध्यमावत् परिणामविशेषोऽन्येषां मते वैखरीति
सूचितम्~। तदाह \textendash\ प्रचादेषु \textendash\ शास्त्रेषु~। अत्र विषये
दर्शनभेदः \textendash\ सिद्धान्तभेदोsनवस्थित इत्यर्थः~॥ [ उ० १५ शप० ] मध्यमेति~। 
पश्यन्त्येव हृदयपर्यन्तमागच्छता तेन वायुना हृदयदेशेऽभिव्यक्ता
तत्तदर्धविशेषतत्तच्छब्दविशेषोल्लेखिन्या बुद्ध्या विषयीकृता
हिरण्यगर्भदैवत्या परश्रोत्रग्रहणायोग्यत्वेन सूक्ष्मा मध्यमा वागित्यर्थः
~। स्क्यं कर्णपिधाने सूक्ष्मतरवाच्वभिघातेनोपांशुशब्दप्रयोगे च
श्रूयमाणेयम्~॥ अस्या व्यवहारोपयोगित्वमाह \textendash\ पदेति~॥ [ उ० १६ शप० ]
पश्यन्तीति~। पराख्यमेव नाभिपर्यन्तमागच्छता तेन वायुनाऽभिव्यक्तं
मनोविषयः पश्यन्ती वागित्यर्थः~॥ इयं सूक्ष्मतरा ईश्वराधिदैवत्या योगिनां
समाधौ सविकल्पकज्ञानविषय इति भावः~॥ तत्र तदभावे हेतुमाह \textendash\ स्वरूपेति~॥ 

वैखरी शब्दनिष्पत्तिर्मध्यमा श्रुतिगोचरा~। 

आन्तरार्था च पश्यन्ती सूक्ष्मा वागनपायिनी~॥



१ परायां योगिनामपि प्रकृतिप्रत्ययविभागावगतिर्नास्तीत्यत्र मान \textendash\ 
माह \textendash\ खरूपज्योतिरित्यादि~। इयं परावागन्तः स्वरूपज्योतिरेव
अनपायिनी \textendash\ प्रकृतिप्रत्ययविभादिशून्याऽविचाला वर्तते~। स्वरूपमात्रपलब्धायां
तस्यां व्याकरणादीनामधिकारो निवर्तते इति तदर्धः~। {\qt स्वरूपज्योतिरेवान्तः}
सैषा वा \textendash\ ? इति कारिकापाठः~॥ ९ अ. पुस्तके न प्रतीकपाठः~। ३ {\qt पश्यन्नपि न
पश्यति वाचम्} इत्ति च, पाठः~॥ 





 ऋङ्मन्त्र इति तद्वयाख्यातारः~॥ 

 (भाष्यम् ) 

२उतत्वः \textendash\ 

उत त्वः पश्यन्न ददर्शं वाच \textendash\ 

मुत त्वः शृण्वन्न शृणोत्येनाम्~। 

उतो त्वस्मै तन्वं विसस्त्रे

जायेव पत्य उशती सुवासाः~॥

 अपि खल्वेकः३ पश्यन्न पश्यति, अपि खल्वेकः शृण्वन्न
श्रृणोत्येनामिति \textendash\ अविद्वांसमाहार्धम्~॥ 



 इत्प्युक्तम्~। आन्तरं स्वरूपज्योतिरेव~। सूक्ष्मा सूक्ष्मतरेति तदर्थः
~॥ [ उ० २० शप० ] मधीति~। लौकिक इति भावः~॥ अयं भावः \textendash\ विन्दोः
शब्दब्रह्मापरनामधेयं वर्णादिविशेषरहितं ज्ञानप्रधानं
सृष्ट्युपयोग्यवस्थाविशेषरूपं चेतनमिश्रं नादमात्रमुत्पद्यते~। 
एतज्जगदुपादानं रवादिशब्दव्यवहार्यम्~। एतत्सर्वगतमपि प्राणिनां मूलाधारे
संस्कृतपवनचलतेनाभिव्यज्यते~॥ ज्ञातमर्थं विवक्षोः पुंस इच्छयया जातेन
प्रयत्नेन योग एव मूलाधारस्थपवनस्य संस्कारः~। तदभिव्यक्तं शब्दब्रह्म
स्वप्रतिष्ठतया निस्पन्दं परा वाक्~। इयं सूक्ष्मतया योगिनां समाधौ
निर्विकल्पकज्ञानविषयः~॥ तदुक्तम् \textendash\ 

 अनादिनिधनं ब्रह्म शब्दतत्त्वं यदक्षरम्~। 

 विवर्तते ऽर्थभावेन प्रक्रिया जगतो यतः~॥

श्रुतिश्च \textendash\ {\qt वागेवार्थं पश्यन्ती} वागजीवीति वागर्थं निहितं संतनोति वाचैव
विश्वं बहुरूपं निबद्ध तदेकस्मादेकं प्रविभज्योपभुङ्क्ते इति~॥
एतदवस्थात्रयमपि सूक्ष्मतमसूक्ष्मतर \textendash\ सूक्ष्मप्रणवरूपम्~॥ [ २१ शप० ]
अनेकेति~। अनेके तीर्थतः शास्त्रतो भेदा यस्या इत्यर्थः~॥६
आदिभ्याम् \textendash\ उपसर्गनिपातापभ्रष्ट \textendash\ भेरीशब्दादिरूपसंग्रहः~॥
भागवतेsप्युक्तम् \textendash\ 

 स एव जीवो विवरप्रसूतिः प्राणेन घोषेण गुहां प्रविष्टः~। 

 मनोमयं सूक्ष्ममपेत्य रूपं मान्ना स्वरा वर्ण इति स्थविष्ठः~॥

इति~॥ विवरेषु \textendash\ आधारादिषु प्रसूतिरभिव्यक्तिर्यस्येत्यर्थः~॥ विन्दुतः
प्राक्प्रक्रिया तु मञ्जवादौ स्पष्टा~॥ विस्तरोsपि तत एव ज्ञेयः~॥ [
भाष्ये ] देवताप्रसादजनकं चैतदित्याह \textendash\ भाष्ये \textendash\ उत्तत्व इति~। 
अत्रोतशब्दोsप्यर्थे खल्वर्थे वा~। एवं च अन्यतरस्यार्थिको लाभः, अध्याहारो
वा~॥ त्वशब्दोsन्यार्थ एवत्वावच्छिन्नार्थो वा~॥ एकोऽवैयाकरणो निश्चयेन
एनां वाचं स्वरूपेण पश्यन्नपि स्वभ्यस्यन्नपि न पश्यति न जानाति,
अर्थापरिज्ञानात्~॥ एवमन्य एनां स्वरूपेण श्वृण्वन्नपि उक्तहेतोरेव न
शृणोति फलाभावात् इति भाष्यार्थः~॥ इतिरर्धार्थसमाप्तौ~॥ अत एवात्र
योग्यं कर्तारं सूचयनुपसंहरति \textendash\ अविद्वांसमिति~॥ 



४ वायोरिति~। कैश्चित् वायोर्वा अपूतां वा ज्ञानस्य वा
शब्दत्वापत्तिरिष्यते~। एवं दर्शनभेद एषु भिन्नेषु भिन्नेषु
प्रवादेष्वनवस्थित इतितदर्थः~॥

 ५ अणूनामित्यस्य शब्दतन्मात्रपरमाणूनामिति व्याख्याने प्रमाणमाह \textendash\ 
अभ्राणीवेति~। यथाऽभ्राणि वायुना एकत्रीभूय प्रवर्धन्ते तथा शब्दरूपाः
परमाणवः प्रवर्धन्त इत्यर्थः~॥ ६ आदिभ्यां \textendash\ 
नामाख्यातादि \textendash\ मृदङ्गशब्दादि \textendash\ इति आदिशब्दाभ्यां~। 

शास्त्रप्रयोजनाधिकरणम् ] महाभाष्यप्रदीपोद्द्योतव्याख्या छाया~। ४५ 



 उतो त्चरमै तन्वं विसस्त्रे$=$तनुं विवृणुते~। जायेव पत्य उशती सुवासाः~। 
तद्यथा \textendash\ १जीयेव पत्ये कामयमाना सुवासाः स्वमात्मानं विवृणुते, एवं वाक्
वाग्विदे स्वात्मानं विवृणुते~॥ वाङ्गो विवृणुयादात्मानमित्यध्येयं
व्याकरणम्~॥ उत त्वः~॥ 

 ( प्रदीपः ) उत त्व इति~। त्वशब्दोऽन्यवाची~। 
उतशब्दः \textendash\ अपिशब्दस्यार्थे~। स च भिन्नक्रमः, प्रत्यक्षेण शब्दस्वरूप
मुपलभमानोऽप्यर्थापरिज्ञानान्न पश्यतीत्यर्थः~॥ उतो इति~। {\qt उत \textendash\ उ} इति
निपातसमाहारः~॥ अविद्धांसमाहार्धमिति~। अविद्वल्लक्षणमर्थमर्द्धर्च
आहेत्यर्थः~॥ 

 (उद्द्योतः ) उत त्व इति~। {\qt उतशब्दोऽप्यर्थे~॥ बहूनामपि~।}
समानपृष्ठोदरपाणिपादानामध्ययनमधीयानानामेकः कश्चित्पश्यन्नपि
स्वभ्यस्ताध्ययनोsपि तीक्ष्णबुद्धिरपि सन्न पश्यति, अर्थानभिज्ञत्वात्~। 
अर्थपरिज्ञानफला हि वाक् \textendash\ इत्यभिप्रायः~। एवं \textendash\ एकः श्वृण्वन्नपि न
शृणोत्येनां वाचम्, य एव ह्यर्थ सम्यगवबुध्यते तेनैव सा सम्यक् श्रुता भवति
~॥ एवमधेनाविद्वांसं निन्दित्वाऽर्धान्तरेण विद्वासं स्तौति \textendash\ 



 [भाष्ये ] तदेवाह \textendash\ भाष्ये \textendash\ [ ८ मप० ] उतो त्वस्मै इति~। 
उतशब्दोऽप्यर्थे~। उशब्दो निश्चये~॥ एकस्मै स्वतनूमपि निश्चयेन
विसस्रे \textendash\ प्रकाशयति~। सृधातोः {\qt छन्दसि लुङ्लङ्लिटः इति वर्तमाने लिट्~।} 
धातूनामनेकार्थत्वात्प्रकाशने वृत्तिः~॥ एवं ददर्शेत्यत्रापि बोध्यम्~॥
अत एव भगवता तथैव व्याख्यातम्~॥ {\qt अमि पूर्वः} इत्यत्र {\qt वा छन्दसि}
इत्यनुवृत्त्या तन्वमित्यत्र यणादेशः~। तदेतद् ध्वनयन्नाह \textendash\ तनुं विवृणुत इति
~। अर्थरूपमित्यर्थः~॥ यतेन कृत्स्ना शब्दार्थरूपामिति कृष्णोक्तमपास्तम्,
निरुक्तादिविरोधात्~॥ उक्तं दृष्टान्तं विशदयति \textendash\ तद्यथेति~॥ पत्ये उशती
कामयमाना~। वशेः कान्त्यर्थात् शतरि संप्रसारणादौ रूपम्~॥
जाया \textendash\ गर्भग्रहणधारणादियोग्या यतः सुवासाः \textendash\ निर्णिक्तसूक्ष्मतमवस्त्रा तनूं
सर्वावयवां विवृणुते इत्यर्थः~॥ अत एव भाष्ये तनुमित्यस्य व्याख्यानं
स्वमात्मानमिति~। खीयं सकलस्वरूपमिति तदर्थः~। एवमग्रेऽपि वागेव
शास्त्राभ्यासादिप्रसन्ना दुर्ज्ञानतया पूर्वोक्तदार्ष्टान्तिके
कर्त्रीत्याह \textendash\ एवमिति~। वाग्विदे \textendash\ वैयाकरणाय~॥ स्वात्मानम् , अर्थ उक्तः~॥
नन्वेतावता किमत आह \textendash\ [ ११ शप० ] वागिति~॥ नः \textendash\ अस्मान् प्रति,
अस्मभ्यमिति वा~॥ 

 [ प्रदीपे ] तदर्थमाह कैयटे \textendash\ अविद्वदिति~॥ विशेष्यभूत \textendash\ 
वाक्यार्थाभयत्वेन स तमाहेत्यर्थः~॥ {\qt अर्धर्चः} इत्यत्र {\qt अर्धर्चाः पुंसि
च} इति पुंस्वम्~॥ अनेन \textendash\ {\qt उत्तरार्धं विद्वांसं वैयाकरणमाह}
इत्यर्थादुक्तप्रायम्~॥ 

 [ उद्द्योते ] तत्सर्व हृदि निधायाह \textendash\ उद्द्योते \textendash\ उतशब्द इति~। 
त्वशब्दस्य द्वितीयार्थाशयेनाह \textendash\ बहूनामपीति~॥



१ जाया पत्ये इति मुद्रित पाठः~॥ २ वाचां प्रसादेन तमेव स्तौति? इति घ.
पाठः~। अयमेव पाठो व्याख्यातश्छायया~॥ ३ प्रतीक \textendash\ स्यास्य अ. पुस्तके न
पाठः~॥ ४ सक्तुरिति १ षश्व धातोः कर्मणि 





उतो त्वस्मै~। अप्येकस्मै$=$कस्मैच्चिद्वैयाकरणाय तन्व$=$शरीरं विसस्त्रे$=$
विवृणोति \textendash\ प्रकाशयति~। सम्यक् जानं हि प्रकाशनमर्थस्य~। अर्थो हि वाचः
शरीरम्~॥ अथोपमायुक्तयोत्तमया २वाचाऽन्त्यपादेनार्थज्ञं प्रशंसति \textendash\ यथा जाया
विवृतसर्वाङ्गावयवा भूत्वोशती$=$कामयमाना भर्त्रे प्रेम्णा दर्शयत्यात्मानम्
~। कस्मिन् काले ? यदा सुवासाः$=$निर्णिक्तवासा नीरजस्का ऋतुकालेपु, तदा
ह्यतितमां स्त्री पुरुषं प्रार्थयते~। यथा स पुरुषस्तां यथावत्पश्यति,
शृणोति तद्वचनार्थः नान्यदा घनपटप्रावृतशरीराम्~। एवं \textendash\ स एवैनां वाचं पदशः
प्रकृतिप्रत्ययविभागेन विगृह्यार्थमस्याः पश्यति शृणोति चेति~। अयमर्थो
निरुक्ततद्भाष्ययोः (१ अ० १९ ख० ) स्पष्टः~। भाष्ये \textendash\ वाग्विदे$=$वैयाकरणाय~॥


 (भाष्यम्) 

३सक्तुमिव \textendash\ 

सक्तमिव तितउना पुनन्तो 

यत्र धीरा मनसा वाचमक्रत~। 

अत्रा सखायः सस्यानि जानते 

भद्रैषां लक्ष्मीर्निहिताsधिवाचि~॥ 

४सक्तुः \textendash\ सचतेर्दुर्धावो भवति, कसतेर्वा विपरी \textendash\ 



पश्यन्नपीत्यस्यार्थमाह़ \textendash\ स्वभ्येति~। अर्थानभिज्ञत्वे इष्टापत्तिं
निराचष्टे \textendash\ [ ४ र्थप० ] अर्थेति~॥ हि \textendash\ यतः~॥ एवं \textendash\ तेषां मध्य इति शेषः
~॥ एनामित्यस्य व्याख्या वाचमिति~॥ अत्र हेतुमुक्तरूपमेवाह \textendash\ [ ५ मप० ]
य एवेति~॥ हि \textendash\ यतः~॥ सम्यगिति~। अनेन न शृणोतीत्यस्य सम्यगश्रवणम् \textendash\ अर्थः
सूचितः~॥ एवमुक्तप्रकारेण उतोशब्दोsप्यर्थे इत्याशयेनाह \textendash\ [ ७ मप० ]
अप्येकस्मा इति~॥ सम्यग् ज्ञानमेव प्रकाशनमित्याह \textendash\ सम्यगिति~॥ हि यतः~॥
नन्वेवमपि कथं शरीरप्रकाशोऽत आह \textendash\ [ ८ मप० ] अर्थो हीति~। यत
इत्यर्थः~॥ [ ९ मप० ] पमेति~। जायासादृश्यरूपेत्यर्थः~॥ उत्तमयेति~। 
मध्यमादिरूपयेत्यर्थः~॥५ वाचमिति~। तत्कर्तृकप्रसादेनेत्यथैः~॥
तमेव \textendash\ वैयाकरणमेव~॥ [ १० मप० ] अङ्गं \textendash\ शरीरम्~॥ [ १३ शप० ] \textendash\ 
नार्थमिति तद्रूपमर्थमित्यर्थ:~॥ तस्यार्थं शब्दद्वारा वेत्यर्थः~॥ एवं \textendash\ 
अग्रेऽपि~॥ घनेति~। निबिडेत्य्ः~॥ [ १४ झप० ] स इति~। वैया \textendash\ करण
ए्वेत्यर्थः~॥ अत्र संमतिमाह \textendash\ [ १५ शप० ] अयमिति~॥ 

 [ भाष्ये ] क्रमप्राप्तं साधकमन्यदप्याह \textendash\ भाष्ये \textendash\ सक्तुमि \textendash\ वेति~। १ 

 [भाष्ये ] तत्र सक्तुशब्दार्थमाह \textendash\ भाष्ये प्रतीकं धृत्वा सक्तुः
सचतेरिति~॥ 

 [भाष्ये ] विनिगमनाविरहादाह भाष्ये \textendash\ कसतेरिति~॥ यद्वा यतः कस गतौ?
इत्यस्माद्विकासार्थकात् पृषोदरादित्वाद्विपरी \textendash\ तात्प्राग्वत् कर्मणि तुनि
रूपम्, अतो विकसितार्थप्रतिपादको 



तुन्प्रत्यये सक्तुशब्दो दुर्धावः \textendash\ दुश्शोध्य इत्यर्थको भवति~। कसतेर्वा
वर्णव्यत्य्येन विकसित इत्यर्थको भवतीत्यर्थ:~॥ ५ निरुक्ते तु {\qt वाचा} इति
तृतीयान्तमेवोपलभ्यते तेन {\qt वाचा}न्त्यपादेन इति पाठ एवोचितः~॥ 

४६ उद्द्योतपरिवृतप्रदीपप्रकाशितमहाभाष्यम्~। [ १ अ. १ पा. १
पस्पशाह्निके 



ताद्विकसितो भवति~। १तितउ$=$परिपवनं भवति, ततवद्वा, तुन्नवद्वा~। 
धीराः$=$ध्यानवन्तः, मनसा प्रज्ञानेन, वाचमकत$=$अकृषत~। 

 {\qt अत्रा सखायः सख्यानि जानते} अत्र सखायः सन्तः सख्यानि २जानते \textendash\ साजुज्यानि
जानते~। 

क्व ? 

य एष दुर्गो मार्गः, एकगम्यो वाग्विषयः~॥ 

के पुनस्ते ? 

वैयाकरणाः~॥ 

कुत एतत्? 

भद्रैषां लक्ष्मीर्निहिताsधिवाचि~। 

 एषां वाचि भद्रा लक्ष्मीर्निहिता भवति~। लक्ष्मी \textendash\ 
र्लक्षणाद्भासनात्परिवृढा भचति~॥ सक्तुमिव~॥ 

 ( प्रदीपः ) सचतेरिति~। {\qt षच \textendash\ सेचने} इत्यस्य~। दुर्धाव इत्ति~। 
दुःशोधः~। यथा \textendash\ तितउना सक्तोस्तुषाद्यपनीयते तथा



भवतीत्यर्थः~॥ तितेति~। {\qt तनोतेर्डउ: सन्वच्च} इति उउः~। 
विश्लेषोच्चारणसामर्थ्यात्संध्यभावः~। अव्युत्पन्नं वा~॥ 

 [भाष्ये ] परिपवनं \textendash\ परिशोधनम्~। करणे ल्युट्~॥ परि \textendash\ 
शोधनकरणत्वस्यातिप्रसक्तत्वात्तितउशब्दव्युत्पत्तिमाह \textendash\ ततेति ~॥ एतेन
तनोतेरेवेदं रूपमिति सूचितम्~॥

 [भाष्ये ] अस्याप्यतिप्रसक्तत्वादाह \textendash\ भाष्ये \textendash\ तुन्नेति~। तुन्नानि
छिद्गाणि यस्य सन्तीति भून्नि मतुप्~॥ 

 [भाष्ये ] [ ५ मप० ] \textendash\ वाचमिति~। शुद्धामिति शेषः~॥
अकृषत \textendash\ कृतवन्तः~॥ 

 [ भाष्ये ] अत्रासखाय इत्यत्र नञादि नेति ध्वनयन्नाह \textendash\ भाष्ये [ १०
मप० ] \textendash\ अत्र सखाय हति~॥ तदर्धमाह \textendash\ समानेति~। तस्यार्थः प्रागुक्तो बोध्यः
~॥ एवमग्रेऽपि~॥ कैयटेन तथावैयाकरणाद् भाष्ये {\qt समानख्यातयः सायुज्यानि
जानते} इति पाठो नास्तीति प्रतीयते~। सांप्रतं तु भाष्ये ४तथोपलभ्यते~॥


 [भाष्ये] पूर्वं विशिष्य कर्तुरनुल्लेखादाह \textendash\ [ भा० १४ शप० ] के
पुनरिति~। सायुज्यप्राप्तिकर्तारः पुनः के इत्यर्थः~॥ उत्तरमाह \textendash\ वैयेति~॥
कुत इति~। तेषामेव तत्प्राप्तिकर्तृत्वमिति कुत इत्यर्थः~॥
उत्तरमाह \textendash\ भद्रैषामिति~। यत इत्यादिः~॥ तदर्थमाह [१८ शप० ] \textendash\ एषामिति~। 
अधीत्यस्य निरर्थकत्वादनुपन्यासः~। शेषपूरणं वा~॥ 

 [ भाष्ये ] लक्ष्मीशब्दार्थमाह \textendash\ भाष्ये [ १८ शप० ] लक्ष्मी \textendash\ रिति
तथा च भासनार्थाद् {\qt लक्ष दर्शनाङ्कनयोः} इत्यस्मादौणादिको {\qt लक्षेर्मुट्
च} इतीकार इति भावः~॥ 

 [ प्रदीषे ] तदाह कैयटे [ २ यप० ] दुःशोध इति~॥ अत्रापि
प्राग्वत् खल्~॥ [ ३ यप० ] वाच इति~। पङ्चम्यन्तम्~। अपनीयन्त इति
शेषः~॥ 



१ तितउ इति~। तनोतेरेतद्रूपंपरिपवनार्थकं \textendash\ परिशोधनकरणार्थकं भवति, अथवा
ततवत् \textendash\ विस्तारयुक्तवस्त्वर्थ भवति, अथ वा तुन्नवत् \textendash\ 
बहुच्छिद्रवद्वस्त्वर्थकं भवतीति तात्पर्यम्~। २ जानते समानख्यात्यः 





व्याकरणेन वाचोsपशब्दा इत्यर्थः~॥ कसतेरिति~। पृषोदरादि \textendash\ 
त्वाद्वर्णव्यत्ययः~॥ ततवदिति~। विस्तारयुक्तमित्यर्थः~॥ तुन्न \textendash\ 
वदिति~। बहुच्छिद्रम्~। धीरा इति~। वैयाकरणाः~॥ वाचमक्रतेति~। 
अपशब्देभ्यो विविक्तां कृतवन्तः~। {\qt मन्त्रे धस} \textendash\ इति लेर्लुकि सति
{\qt अक्रत} इति रूपम्~॥अत्रा सखाय इति~। {\qt ऋचि तुनुघ \textendash\ } इति दीर्घ~। 
सखायः$=$समानख्यातयो भेदग्रहस्य निवृत्तत्वात्सर्वमेकमिति मन्यन्ते~॥
सख्यानीति~। सायुज्यानीत्यर्थः~॥ एकगम्य इति~। ज्ञानेनैव प्राप्यः
~॥ वाचीति~। वेदाख्ये ब्रह्मणि या लक्ष्मीर्वेदान्तेषु
परमार्थसंविल्लक्षणोक्ता सैषां निहितेत्यर्थः~॥ 

 ( उह्व्योतः ) भाष्ये \textendash\ सक्तुमिवेति~। २यत्र \textendash\ व्याकरणे धीराः ध्यानवन्तः
सन्तस्तितउना सक्तुमिव ध्यानयुक्तमनसा तत्करण \textendash\ ज्ञानेन
वाचमक्रत$=$अकृषत \textendash\ असाधुभ्यो विविक्तां कृतवन्तस्ते तत्क \textendash\ रणेन शुद्धचित्ता
अत्र$=$ब्रह्मप्रतिपादकशब्दे शब्दार्थयोरभेदबुद्धया
सखायः$=$समानख्यातयः \textendash\ समान्ज्ञानाः, तच्छब्दे ब्रह्मैकत्वज्ञान \textendash\ 



 [ प्रदीपे ] तदाह \textendash\ कैयटे [ ४ र्थप० ] विस्तारेति~॥ 

 [ प्रदीषे ] तदाह \textendash\ कैयटे [ ५ मप० ] बह्विति~। बहुव्रीहिः~। 

 [ प्रदीषे ] धाधातोरौणादिके कनि घुमास्था \textendash\  इतीत्वे धीरा इति रूपम्,
धातूनामनेकार्थत्वाद् ध्याने वृत्तिरिति कृष्णादयः~। ध्यैधातोरेव तथा
रूपमिति वयम्~॥ ध्यानं व्याकरणविषयकम्~। अत एव विशिष्टार्थमाह \textendash\ कैयटे [
५ मप० ] वैयेति~॥ 

 [प्रदीपे ] तवाह \textendash\ कैयटे [ ६ ष्टप० ] अपेति~॥ 

 [ प्रदीपे] अत एवाह \textendash\ कैयटे [ ८ मप० ] ॠचि त्विति~। भेदेति च~॥ 

 [ प्रदीपे ] एकगम्य इत्यत्यैकेनैव गम्य इत्यर्थ:~। तमाह \textendash\ वैयटे [
१० मप० ] ज्ञानेनैवेति~॥ 

 [ प्रदीपे ] कैयटे [ १ १ शप० ] वेदाख्ये इति~। 
वाचीत्यस्यावृत्त्योभयत्रान्वय इति भावः~॥ 

 [उद्द्योते ] सक्तुमिति स्पष्टार्थ, तदर्धमाह \textendash\ [ उ० १ मप० ] ये
यत्रेत्यादि~। विषयत्वं सप्तम्यर्थः~। [ २ यप० ] सक्तुमिवेति~। सक्तुं
पुनन्त इवेत्यर्थः~॥ शुद्धमनसोहेतुत्वादाह \textendash\ ध्यानेति~॥ तदपि
सव्यापारमित्याह \textendash\ तत्करणेति~॥ वाचमक्रतेति~। {\qt मन्त्रेघस \textendash\ } इति लेर्लुक्~। 
शुद्धामित्मर्थः~॥ तदाह[ ३ यप० ] अस्ाध्विति~॥ विविक्तामिति~। 
विवृत्तामिति क्वचित्पाठः~॥ अत्रासखाय इत्यत्र {\qt ऋचि तुमुघ \textendash\ } इति दीर्थ इति
ध्वनयन्नाह \textendash\ [ ४ रथप० ] ब्रह्मेति~॥ पतेन तत्र व्याकरणविषय इति
रलोक्तमपास्तम्~। तद्ध्वनयन्नाह \textendash\ ब्रह्मेति~। व्याकरणप्रतिपाद्ये इत्यर्थः
~॥ सखाय इत्यस्य व्याख्या समानेति~॥ तस्य व्याख्या समानेति~॥
तस्यार्थमाह \textendash\ [ ५ मप० ] तच्छब्दे इति~। ब्रह्मप्रतिपादकशब्दे इत्यर्थः
~॥ 



{\qt सायुज्यानि जानते} इति छायाकृद्दृष्टः पाठ:~॥ ३ {\qt ये यत्रः इति घ.पाठः~॥} ४
तथोपलभ्यते क्वचिदेव पुस्तके~। न पुनः सर्वैषु पुस्तकेषु~। अस्माभिरपि
द्वयोरेवोपलब्धस्तथा पाठः, त्रिषु नोपलब्धः~॥ (दाधि.) 

शास्त्रप्रयोजनाधिकरणम् ] महाभाष्यप्रदीपोद्द्योतव्याख्या छाया~। 

 ४७



 वन्तस्तेनैव दृष्टान्तेन सर्वपदार्थेषु ब्रह्मनिरूपिताभेदज्ञानवन्तः
सख्यानि$=$सायुज्यानि जानते, प्राप्नुवन्तीत्यर्थः~। यत एषां वाचिभद्रा
लक्ष्मीः स्वप्रकाशब्रह्मरूपा अधि$=$अधिकं निहिता$=$स्थिता भवतीति ऋगर्थ:~। 
सचतेर्दुधाव इति~। यतः सचतेः, अतो दुःशोधार्थप्रतिपादको
भवतीत्यर्थः~॥तितउशब्द इत एवभाष्या \textendash\ न्नपुंसकः~। {\qt चालनी तितउः पुमान्} इति
१कोशात्पुल्लिङ्गोऽपि~॥ विस्तारेति~। {\qt तनोतेर्डउ:} सन्वच्च इति
व्युत्पत्तेरिति भावः~॥ तुमेति~। तुदधातोः कर्मणि डउः सन्वच्चेति भावः
~॥ ध्यानं \textendash\ २ज्ञातार्थभानम्~॥ {\qt मनसा इति व्यापारपर,}
तदाह \textendash\ प्रज्ञानेनेति~॥ भाष्ये \textendash\ क्वेति~। {\qt र्कि तत् यत्र सायुज्यानि
प्राप्नुवन्ति} इति प्रश्नः~॥ उत्तरयति \textendash\ य एष इति~। 
दुर्गमार्गप्राप्यत्वात् दुर्गमार्गत्वोपचारः~॥ दुर्गत्वमेवाह \textendash\ एकेति~॥
ज्ञानेनैवेति~। निर्विकल्पकज्ञानेनैवेत्यर्थः~। अत एव दुर्गत्वं, {\qt नान्यः
पन्था:} इति श्रुतेरिति भाव:~॥ भाष्ये \textendash\ ३वाग्विषय इति~। अत एव
श्रवणोपपत्ति:~॥ वाक् वेदरूपाऽत्र~॥ वेदान्तेष्विति~। सत्यं
ज्ञानमित्यादिषु~। परमार्थसंविल्लक्षणा \textendash\ 
परमार्थब्रह्ममात्रविषयाsखण्डार्थरूपा सैषां वाचि निहितेत्यर्थः~। 
सर्वोऽपि वेदस्तेषां ब्रह्मपरः, वेदैश्च सर्वैरहमेव वेद्य
इत्युक्तेरिति भावः~॥ अयं भावः \textendash\ ये शास्त्रतः प्रकृतिप्रत्ययविभागेन
साधून् ज्ञात्वा 



तुल्ययुक्त्या आह \textendash\ त्तेनैवेति~॥ [ ७ मप० ] प्राप्नुवन्ति~॥ यतः इति
पाठः~॥ अत्र हेतुमाह \textendash\ यत इति~॥ एषां \textendash\ अत्र धीराणाम्~॥ [ ९ मप० ]
भवतीति~। अत इति शेषः~॥ 

 [ उद्द्योते ] असंबद्धत्वात्तदर्धमाह \textendash\ [ उ० ९ मप० ] यत इति~॥ षच
समयाये षच सेचने \textendash\ इत्यन्यतरस्माद् दुःशोधार्थादौणादिके कर्मणि तुनि कत्वे च
रूपमिति भावः~॥ अत एवाह \textendash\ अत इति~॥ भाष्ये दुर्धाव इत्यत्रेषद्दुःस्विति
दुःपूर्वाद्धावे कर्मणि खलित्याशयेनाह \textendash\ दुःशोध्यार्थेति~। 1 

 [ उद्द्योते १ \textendash\ [११ शप० ] नपुंसकः \textendash\ नपुंसकलिङ्गः~॥ कोशासंगतिं
परिहरति \textendash\ चालनीति~। अत एव {\qt स्याद्वास्तु हिङ्गु तितउ इति त्रिकाण्डशेष:
संगच्छते~॥}

 [ उद्द्योते ] तदाह \textendash\ [ उ० १२ शप० ] तनोतेरिति~। इदं सर्वं
शाकटायनादिरीत्या वोध्यम्~॥ 

 [उ्योते ] तदाशयमाह \textendash\ [ उ० १३ शप० ] तुदेति~। व्यथनार्थकादिति भावः
~॥ एवं च नातिप्रसक्तिरिति वोध्यम्~॥ 

 [ उद्दयोते ] तत्र ध्यानपदार्थमाह \textendash\ [उ० १४ शप०]
ज्ञातार्थभावनमिति~। सततपरिशीलनमित्यर्थः~। धारणेति यावत्~॥ 

 [ उद्द्योते ] प्रश्नोत्तरयोः सामानाधिकरण्याय प्रातिपदिकार्थप्रश्ने
तात्पर्यमित्याशयेनाह \textendash\ [ उ० १५ शप० ] किं तदिति~॥ [ १६



२ {\qt वस्तुतस्तूक्तभाष्यानुरोधादमरग्रन्थे} पुंस्त्वाऽयोगमात्रं व्यवच्छेद्यं
न त्वन्ययोगोऽपि~। तथा च पुंनपुसकवर्गे \textendash\ {\qt स्याद्वास्तु हिङ्गु तितउ} इति
त्रिकाण्डशेषः~। अत एव तितउमाचष्टे इति तितावयतीस्यभियुक्तग्रन्थां 





शास्त्रार्थध्यानवन्तो मानसज्ञानेन वाचमसाधुभ्यः पृथक् कृतवन्तस्ते
तज्ज्ञानपूर्वकैः साधुशब्दप्रयोगैर्लब्धान्तःकरणशुद्धयः, अत्र य एष दुर्गों
मार्गो ब्रह्मरूपस्तत्रात्मना सह समानख्यातयः \textendash\ त्यक्तमेदभावनाः
सख्यानि \textendash\ सायुज्यानि प्राप्नुवते~। यत एषां वाचिवेदाख्ये ब्रह्मणि
सर्वभासकब्रह्मरूपा सा अधि$=$अधिकं निहिता भवति~। 
एतच्छास्त्रसाध्यप्रयोगव्यङ्ग्यस्य ध्वनिरूपवैखरीरूपरूषितस्यैव
तैर्वाचकत्वस्वीकारेण तस्य चात्यन्तविचारे ब्रह्मातिरेके मानानुपलम्भेन
तत्तदुपाधिभिन्नचित्त एव बोधकतया तैर्ग्रहात्~। एवं सर्वबोधकेपु
ब्रह्मबुद्धौ जातायां तेनैव दृष्टान्तेन सर्वपदार्थेषु
ब्रह्मबुद्धिवैयाकरणानामिति~॥ यत्तु पदपदार्थवाक्यार्थमर्यादया
{\qt सत्यं \textendash\ } इत्यादितस्तेषां तद्वोधात्सा तेषां वाचि निहितेति, तन्न; {\qt यतो
वाचो निवर्तन्ते अप्राप्य मनसा सह} इति श्रुत्या ब्रह्मणस्तन्मर्यादया
शब्दावोध्यत्वोक्तेः~। तत्र हि अप्राप्यं इत्यस्य
संबन्धज्ञानेsविषयीभूय \textendash\ असंबंध्य चेत्यर्थः~। आद्यं शब्दे, अन्त्यं मनसि~॥
शुद्धचित्तस्य तूभयमपि बोधकमखण्डविषयस्य, दृश्यते त्वग्र्यया
बुद्ध्येति श्रुत्यन्तरात् ; न तु पदपदार्थमर्यादयेति बोध्यम्~। [
एवज्ज सर्वबोधकेषु ब्रह्मबुद्धौ जातायां तेतैव दृष्टान्तेन
सर्वपदार्थेष्वपि ब्रह्मबुद्धिवैयाकरणानामिति तात्पर्यम्~॥ ] 



 शप० ] दुर्गमार्गेति~। तस्य दुष्प्रापत्वादिति भावः~॥ [१७ शप०]
दुर्गत्वमिति~। अतिविषमत्वमित्यर्थः~। मार्गे इति भावः~॥ 

 [ उद्दयोते ] सविकल्पकनिवृत्तये आह \textendash\ [ उ० १७ शप० ] निर्विकेति~। 
अत एव \textendash\ तस्यान्यथा दुःसाध्यत्वादेव~॥ एवव्यव \textendash\ च्छेयमाह [ १८ शप० ]
नान्य इति~॥ अवागितिच्छेदनिरासा \textendash\ याह \textendash\ [ १९ शप० ] भाष्ये वाग्विषय
इति च्छेद इति~॥ अत एव \textendash\ विषयत्वादेव~॥ एवं च साधनसंपत्तिसत्ता सूचिता~॥
ननु कीदृशवाग्विषयत्वं तस्य, तत्राह \textendash\ [ १० शप० ] वाग्वेदरूपाऽत्रेति~॥


 [उद्द्योते २१ शप० ] परमार्थेत्यादि रूपान्तं समस्तमेकं पदम्~॥
नन्वेवं कथं सर्ववेदनिष्ठा सा लक्ष्मीरित्यत आह[२१ शप० ] सर्वोऽपीति
~॥ ब्रह्मपर इति~। तत्तात्पर्यक इत्यर्थः~॥ अन्येषामज्ञान \textendash\ 
निरासायाह \textendash\ [ २३ शप० ] अयं भाव इति~॥ शास्त्रतः \textendash\ व्याकरणतः~॥ 

 [ उद्ययोते ३१ शप० ] सैः \textendash\ वैयाकरणैः~। एवमग्रेऽपि [ ३१ शप० ]~॥
परमतात्पर्यमाह \textendash\ [ ३१ शप० ] एवमिति~॥ प्राचामुक्तिं खण्डयति [ ३३
शप० ] यत्विति~॥ अप्राप्येत्यस्योभयत्र संबन्धायाह \textendash\ [ ३६ शप० ]
तत्रेति~॥ [ ३७ शप० ] तूभयमिति~। घा \textendash\ नसरूपमित्यर्थः~॥
स्पष्टार्थमुपसंहरति [३८ शप० ] \textendash\ न त्विति~॥ 



{\qt अपि सङ्गच्छन्ते} शति श० कौ०~॥ २ {\qt ज्ञातार्थभावनं} इति मुद्रित \textendash\ पाठः~। ३
{\qt वाग्विषय इति च्छेदः} इति घ \textendash\ झ. \textendash\ पाठः~॥ ४ ब्रह्ममात्र. विषयज्ञानविषया
इति पाठः~॥ ५ मुद्रितपुस्तकेषु अयं पाठौ न दृश्यते~॥ 

४८ उद्द्योत्तपरिवृतप्रदीपप्रकाशितमहाभाष्ये~। [१ अ. १ पा. १
पस्पशाह्निके 



लक्षणादित्यस्यव्याख्या \textendash\ भासनादिति~॥ परिवृढा \textendash\ अज्ञाननिवर्तने प्रभ्वी~॥
एते च मन्त्रा: १सर्वानुक्रमेsन्यत्र विनियुक्ता अपि भाष्य \textendash\ 
प्रामाण्यादेतत्तात्पर्यका अपीति बोध्यम्~॥ {\qt  चत्वारि} \textendash\ इत्यादि ऋक् \textendash\ 
चतुष्टयं व्याकरणस्य मोक्षजनकत्वं प्रतिपादयति~॥ 

 ( भाष्यम्) 

 सारस्वतीम् \textendash\ 

 याज्ञिकाः पठन्ति \textendash\ {\qt आहिताग्निरपशब्दं प्रयुज्यं प्रायश्चित्तीयां
सारखतीमिष्टिं निर्वपेत्} इति~। प्रायश्चित्तीया मा भूमेत्यध्येयं
व्याकरणम्~॥ सारस्वतीम्~। ! 

 ( प्रदीपः ) प्रायश्चित्तीयामिति~। भवार्थ {\qt वृद्धाच्छः~॥}
प्रा्यश्चित्तीया इति~। प्रायश्चित्ताय$=$पापशोधनाय श्रुतिस्मृति \textendash\ विहिताय
कर्मणे हिताः$=$तन्निमित्तोत्पादका मा भूमेत्यर्थः~॥ 

 ( उद्द्योतः) भवार्थे इति~। प्रायश्चित्तसाधनत्वेन तद्भवत्वम्~॥ 

 ( भाष्यम्) 

 दशम्यां पुत्रस्य \textendash\ 

 याज्ञिकाः पठन्ति \textendash\ {\qt दशम्युत्तरकालं} पुत्रस्य जातस्य नाम विदध्यात्
घोषवदाद्यन्तरन्तस्थ \textendash\ 



 [ उद्द्योते ] परिगृहीतेति रत्लोक्तार्थनिरासायाह \textendash\ [ उ० ४१ शप०
]मज्ञानेति~॥~। विरोधं परिहरति \textendash\ एते चेति~॥ सर्वानुक्रमे
छन्दर्पिदेवतानामेव प्रतिज्ञापुरःसरमनुक्रान्तत्वादाह \textendash\ भाष्ये इति~॥ [
४२ शप० ] अन्यत्र \textendash\ अग्निध्यानादौ~॥ परमतात्पर्यार्थमाह \textendash\ [४३ शप ० ]
चत्वारीति~। चत्वारि शृङ्गा इत्यादीत्यर्थः~॥ 

 [ भाष्ये ] अग्निमादधानेन दानपरिग्रहादिवद्व्याकरणमवश्यम \textendash\ 
ध्येयमित्याह \textendash\ भाष्ये \textendash\ सारस्वत्तीमिति~॥ याज्ञिकाः पठन्तीति प्राग्वत्~॥
यद्यपि अनारभ्याधीतेन {\qt नानृतं वदेत् इति निषेधे} \textendash\ 
नार्थानृतवच्छब्दानृतस्यापि सत्त्येन सर्वत्रापशब्दप्रयोगे पुरुषदोषा बहवः,
तथापि दर्शपूर्णमासप्रकरणपठितेन तादृशेन वाक्येन क्रतोरपि
वैगुण्यमित्याशयेनाह \textendash\ आहिताग्निरिति~॥ ऋतुमध्य इति शेषः~॥ वस्तुतस्तु
अन्यत्र दोषाभावसूचकमेवात्राहिताग्निपदमित्यन्यत्र स्फुट्म्~॥ एवं च
नारायणोक्तं युक्तमेवेति बोध्यम्~। 

 [ भाष्ये ] भाष्ये सारस्वतीं सरस्वतीदेवताकाम्~॥ 

 [ प्रदीपे] कैयटे \textendash\ हिता इति~। {\qt तस्मै हितम्} इत्यर्थे वृद्धाच्छः
~॥ अत्र प्रायश्चित्तपदमुक्तकर्मपरमेवेति भावः~॥ प्राग्वत्
यशकर्मण्यपशब्दप्रयोक्तारो मा भूमेत्यध्येयं व्याकरणमिति फलितोऽर्थः~॥ 

 [उद्र्योते ] ननु श्रुतिस्मृतिविहितं पापशोधककर्मं प्रायश्चित्त \textendash\ 
मिति कथमिष्टेस्तद्भत्वमत आह \textendash\ उदद्योते \textendash\ प्रायश्चित्तेति~॥ तथा चात्र
प्रायश्चित्तपदमन्तःकरणशुद्धिपरमिति भावः~॥ 



१ {\qt सर्वानुक्रमभाष्यैऽन्यत्र} इति मुद्रितपाठः, छायादृष्टोsप्ययम्
~॥ 





मबृद्धं त्रिपुरुषानूकमनरिप्रतिष्ठितम्~। तद्धि प्रति \textendash\ ष्ठिततमं भवति~। 
{\qt द्व्यक्षरं चतुरक्षरं वा नाम कृतं कुर्यान्न तद्धितम्} इति~॥ 

 न चान्तरेण व्याकरणं कृतस्तद्धिता वा शक्या विज्ञातुम्~॥ दशम्यां
पुत्रस्य~॥ 

 (प्रदीपः) दशम्युत्तरकालमिति~। दशम्या उत्तर इति {\qt पञ्चुभी \textendash\ } इति
योगविभागात्समासः~। ततः कालशब्देन बहुव्रीहि~। कियाविशेषणं चैतत्~। दश
दिनान्यशौचं भवति \textendash\ इति दशम्युत्तरकालमित्युक्तम्~। येsपि गृह्यकाराः
पठन्ति \textendash\ {\qt दशम्यां पुत्रस्य} इति, तैर्दशम्यामिति सामीपिकमधिकरणं व्याख्येयम्
~॥ घोषवदादीति~। घोषवन्तो ये वर्णाः शिक्षायां प्रदर्शिताः, तदादि~॥
अन्तरन्तस्थमिति~। मध्ये यरलवा यस्य तदित्यर्थः~॥ त्रिपुरुषानूकमिति~। 
नामकरणे योऽधिकारी पिता तस्य ये त्रयः पुरुषास्ताननुकायति$=$अभिधत्त
इति त्रिपुरुषानूकम् , {\qt अन्येषामपि दृश्यते} इति दीर्घ:~॥ 

 (उद्द्योतः ) पञ्चमीति योगविभागादिति~। {\qt सुप्सुपा} इति तु युक्तम्~॥
ननु {\qt दशम्यां पुत्रस्य इति प्रयोजनस्वरूपगणने भाष्ये उक्तम्~।} गृह्येऽपि
तथेव पठ्यते, अतः {\qt दशम्युत्तरकालं इत्ययुक्तमत}



 [भाष्ये ] नामकरणे उपयोगमाह \textendash\ भाष्ये दशम्यामिति~॥ याज्ञिकाः पठन्तीति
पूर्ववत्~॥ {\qt वटे गवः सुशेरते मासिकं} इत्यादिवत् {\qt द्शम्याम्}
इत्यौपश्लेषिकेऽधिकरणे सप्तमीत्याशयेनाह \textendash\ दशम्युत्तरेति~॥ 

 [भाष्ये ] अबृद्धं \textendash\ बृद्धसंज्ञाक्षररहितम्~॥ 

 [ भाष्यै ] अत एव स्तौति \textendash\ तद्धीति~॥ कृतमिति~। तदन्त \textendash\ मेवेत्यर्थः~॥
एवमग्रेऽपि~॥ 

 नियमापूर्वफलिततदनिष्टाभावात् तद्धितान्तस्यापि प्राप्तिरिति
तन्निषेधति \textendash\ न तद्धितमिति~॥ तत्करणेन न केवलं नियमापूर्वहानिः, अपि तु
दुरितमपीत्यर्थः~॥ 

 नन्वेतावता किमत आह \textendash\ भाष्ये \textendash\ नचान्तेति~॥ नहीत्यर्थः~॥ विज्ञातुमिति~। 
{\qt अतोऽध्येयं व्याकरणम्} इति शेषः~। 

 [ प्रदीपे ] कैयटे [५ मप० ] सामीपिकमिति १~। उप समीपे श्लेषः
संबन्धस्तत्कृतमित्यर्थः~॥ [ ६ ष्ठप० ] घोषवन्त इति~। घोषशब्दो
धर्मपरः~। 

 [ प्रदीपे ] कैयटे [ ९ सप० ] त्रयः पुरुषा इति~। पितृपिता \textendash\ 
महप्रपितामहा इत्यर्थः~॥ तानिति~। तेषामन्थतममित्यर्थः~॥ इदं च~॥
फलितार्थकथनम् 

 [ उ्ष्योते ] योगविभागस्य भाष्यादृष्टत्वादाह \textendash\ उद्द्योते सुप्सुपेतीति
~॥ इदमेवं प्रतिपादयन्नाह \textendash\ [ २ यप० ]नन्विति~॥ 





२ {\qt तैरपिदश} \textendash\ इति पाठः,उद्द्योते तैरपीतिइति प्रतीकग्रहणात्~। 

शास्त्रप्रयोजनाधिकरणम् ] 

महाभाष्यप्रदीपोद्द्योतव्याख्या छाया~। 

४९ 

आह \textendash\ तैरपीति~॥ शिक्षायामिति~। ते च \textendash\ हृशः~। अन्तरन्तःस्था 

यस्येत्यर्थः~॥त्रिपुरुषानूकम्~। त्रिपुरुषशब्दो द्विगुः~। मूलविभुजादि \textendash\ 


त्वात्कः~॥ भाष्ये \textendash\ अनरीति~। अमनुष्ये \textendash\ अरिभिन्ने \textendash\ इति वाऽर्थः~॥ 

( भाष्यम्) 

सुदेवो असि वरुण \textendash\ 

सुददेवो असि वरुण यस्य ते स्प्त सिन्धवः~। 

अनु क्षरन्ति काकुदं सूर्म्य सुषिरामिव~॥

{\qt सुदेवो असि वरुण} \textendash\ सत्यदेवीऽसि, {\qt यस्य ते सप्त 

सिन्धवः~।} \textendash\ सप्त विभक्तयः, {\qt अनुक्षरन्ति काकुदम्}

काकुदं \textendash\ तालु~। कौकुः$=$जिह्वा, साऽसिमन्नुद्यत इति \textendash\ 

काकुदम्~। \textendash\ सूर्म्य सुषिरामिव~। तद्यथा \textendash\ शोभना \textendash\ 

मूर्मि सुषिरामग्निरन्तः प्रविश्य दहति, एवं ते 

सप्त सिन्धवः$=$सप्त विभक्तयस्ताल्ववुक्षरन्ति~। 

तेनासि संत्यदेवः~॥ सत्यदेवाः स्यामेत्यध्येयं 

[ उद्व्योते ] केयटेन वृत्यर्थस्योक्तत्वाद्विग्रहमाह \textendash\ [उ० ४ र्थप०
] 

अन्तरिजि~॥ अन्तःस्थाशब्द आकारान्तः~॥ 

[ उद्दयोते ] शब्दार्थमाह \textendash\ [ उ० ५ मप० ] त्रिपुरुषशब्दो 

द्विगुरिति~॥ समाहारद्विगुरित्यर्थः~॥ पात्रादित्वात् स्त्रीत्वाभाव 

इति भावः~॥ अमनुष्य इति~। देवतादौ पर्यवसितमित्यर्धः~॥ 

यद्गा \textendash\ अरौ प्रतिष्ठितं न भवतीत्य्थः~॥ तत्कलितमाह \textendash\ अरिभिन्ने 

इति~॥ त्भिन्ने मित्रादावित्यर्थः~॥ र 

[ भाष्ये ] तथाऽन्यमपि ऋ्मप्राप्तमृङ्मत्रमाह \textendash\ भाष्ये \textendash\ सुदेवो 

असीति~। {\qt प्रकृत्यान्तः \textendash\ इति प्रकृतिभावः~॥ 

[भाष्ये ] भाष्ये \textendash\ सुदेव इति~। हे वरुण त्वं सुदेवो असि~॥ 

अस्य व्याख्या \textendash\ सत्यदेवोऽसीति~॥ 

[भाष्ये ] तदेवाह \textendash\ भाष्ये \textendash\ यस्य ते सप्तेत्यादि~॥ 

[भाष्ये ] भाष्ये \textendash\ ताल्विति~। तस्य तत्र यौगिकत्वम्~॥ 

रूढिव्यावृत्तये आह \textendash\ काकुरिति~॥ साऽस्मिन्नुद्यत इति फलितार्थ \textendash\ }

कथनम्~॥ 

[भाष्ये ] अत्र दृष्टान्तमाह \textendash\ भाध्ये \textendash\ सूर्म्यमिति~॥ तदुप \textendash\ 

पादयति \textendash\ शोभनामूर्मिमिति~। अनेनोमिरयःप्रतिमा न तु समिः 

इति सूचितम्~॥ 

[भाष्ये ] उपसंहर्तु दार्ष्टान्तिके पुनर्योजयति भाष्ये \textendash\ एवं 

त इति~॥ शवमित्येव पाठे ते? इति शेषः~॥ 

[ भाष्ये ] निगमयति भाष्ये \textendash\ तेनासीति~॥ ननु किमेतावता \textendash\ 

ऽत आह \textendash\ सत्येति~। व्याकरणाध्ययनस्य सत्यदेवत्वं फलमित्यरथैः~॥ 

[ प्रदीषे ] कैयटे \textendash\ वरुणेति~। बार्हस्पते प्रगाथे इत्यादिः~॥ 

१ न्रिपुरुषेत्ति~। यद्यपि बहुषु प्रामाणिकपुस्तकेषु {\qt त्रिपुरुषशब्देन} 

द्विगुः इति पाठ उपलभ्यते तथापि समर्षकः पाठरश्छायाकारेण गृहीतः 

कवचिद्वृष्टश्य, स एवात्र संगृहीतः~। {\qt त्रिुरुपशब्दे द्विगुः} इति क्वचित्


मुद्रितपुस्तके पाठो दृश्यते स चप्रामाणिकः~॥ २ प्रतीकमेतत् अ,. 

फ. पुस्तकयोर्न पल्यते~॥ ३ {\qt काकुद्$=$जिह्वा} इति अ. पाठः~॥ 

२ काकुदमिति \textendash\ रैषिकोऽग् \textendash\ इति काकुद \textendash\ जिह्वा इति पाठे यो \textendash\ 

प्र.पा.

व्याकरणम्~॥ सुदेवो असि~॥ 

(प्रदींपः ) सुदेवो असीति~। वरुणस्येयं स्तुतिः~। यतो 

हेतोर्व्याकरणज्ञानाद्वरुर्णः सत्यदेवः, ततो हेतोरन्येऽपि सत्य \textendash\ 

देवा भवन्तीत्यर्थः~॥ सिन्धव इति~। नद् इव विभक्तय 

इत्यर्थः~॥ अनुक्षरन्तीति~। तात्यनुप्राप्य प्रकाशन्त इत्यर्थः~॥ 

साऽस्मिन्नुद्यत इति~। {\qt अनेकार्थत्वाद्धातृनां उत्क्षिप्यते} इत्यर्थः~। 

सूम्र्यमिति~। सूर्मि इति प्राप्ते {\qt अमि पूर्चः} इत्यत्र {\qt वा 

छन्दसि} इत्यनुवृत्त्या यणादेशः~॥ 

(उद्व्योतः ) सत्यदेवा इति~। वरुणे यथा \textendash\ सत्येन दीव्यदि 

द्योतत इति यौगिकोऽयं शब्दस्तथाऽन्येष्वपीत्यर्थः~॥ यस्येति~। पञ्चमी \textendash\ 

स्थाने षष्ठी~। यस्मात्ते काकुदं$=$तालु प्राप्य सिन्धवः सप्तसमुद्ररूपा: 

सप्तविभक्त्योऽनुक्षरस्ति, प्रकाशन्ते \textendash\ इत्यर्थः~॥ अनेकार्थत्वादिति~। 

वदेरधिकरणे {\qt घञर्थे \textendash\ } इति कः~। शवन्धुवत्पररू, पृषोदरादिल्वा \textendash\ 

दैकस्थोकारस्य लोपः~। नुदेवऽधिकरणे कः, पृषोदरादित्वान्नुलोपः~॥ 



[प्रदीपे ] सर्वमब्रतात्पर्या्माह \textendash\ कैयटे \textendash\ यत इति~। व्याक \textendash\ 

रण्षानाद्यतौ हेतोः साध्वसाधुशब्दद्यानपूर्वकं सुशब्दप्रयौगादित्यर्थः~॥


[ प्रदीप ] कैयटे \textendash\ नद्य इवेति~। समुद्रा इवेलर्थः~। लक्षणैवेति 

भावः~॥ विभक्तयः~। उक्तो५र्थः~। अनुक्षरन्तीत्यस्याथमाह \textendash\ ताल्वि \textendash\ 

ति~॥ अध्याहारेणाह \textendash\ अनुप्राप्येति~॥ क्वचित् {\qt प्राप्यःइत्येव पाठः~॥}

[ उद्द्योते ] तत्तात्पर्यमाह \textendash\ उदयोते \textendash\ वरुणे इति~॥ सत्ये \textendash\ 

नेति~। साधुशब्दप्रयोगाख्येनेत्यरथः~॥ {\qt देवितासि व्यवहरसि} इति 

कृष्णोक्तार्थनिरासायाह \textendash\ दीव्यति द्योतत इति~। अयं भावः \textendash\ 

द्रिविधमनृतम् \textendash\ अर्थानृतं शब्दानृतं च~। तत्र यथा भ्रान्तविप्रलम्भकै 

रजतत्वेनोपदिष्टस्य रङ्गस्य रजतस्थाने दानमनर्थहेतुः, तद्वत् साधु \textendash\ 

शब्दस्थानेऽसाधुशब्दप्रयोगः~। तदभावाद्वरुणः सत्यदेव इति यौगिकः~। 

न त्वश्वकर्णादिवद्रूढ इति भावः~॥ 

[उद्दयोते] तदर्थमाह उद्दयोते \textendash\ यस्येतीति~। व्यत्ययेनेति भावः~॥ 

प्राप्येत्यध्याद्वारलभ्यः, अन्यथाऽनुपपत्तेः~॥ सप्तवीति~। सप्तत्व. 

संख्यासामान्यात्प्रभमादिसप्तत्रिकाणीत्यर्थः~॥ तथा च्च श्रुतौ लुप्तोपमा


सू्चिता~॥ व्याख्यानेऽपि तथा~। अनेन क्रियाऽप्याक्षिप्ता भवतीति न 

न्यूनता~॥ प्रकाशन्त इति~। मूर्तिमत्य इत्यादिः, अत इत्यन्तश्च~॥ 

[ उद्दोते ] तद्ध्वनयन् \textendash\ {\qt अनेकार्थत्वात्} इते कैयटो \textendash\ 

क्तिवशादाह \textendash\ [ उ० ५ मप० ] वदेरिति~। उत्क्षेपार्थादिति 

भावः~॥ घजर्थे इतीति~। वार्तिकेनेति भावः~॥ शक्ध्वति~। 

संप्रसारणे पूर्वरूपे काकोरुदमिति पध्रीतत्पुरुषे इत्यादिः~॥ विनि \textendash\ 

गमनाविरहादाह \textendash\ पृषोदेति~॥ {\qt सास्मिन्नुद्यते} इकति भाष्यस्य 

संदिग्धत्त्रेनान्यथापि व्याख्यातुं शक्यत्वेन
संप्रमारणाद्यभावादेकार्थत्व \textendash\ 

कल्पनालाधवादाह \textendash\ नुदे्र्वेति~॥ 

ज्यम्~॥ ५ सूर्म्यमिति~। सुर्मिः $=$अयःप्रतिमेत्यपि दृश्यते ~॥ सूर्भि 

{\qt ज्वलन्तीमाश्लिष्य} इति रमृतेः~। केचित् \textendash\ विकृतविला मूषा \textendash\ इत्याहुः~॥ 

६ {\qt एवं तव सप्तः इति फ. पाठः~॥} ७ सत्यदेव इति~। सत्येन 

साधुशब्दस्वरूपेण प्रयोगाख्येन देविताऽसि, व्यवहरसीत्यर्थः~॥ 

८ {\qt वरुण सत्यदेवोऽसिः} इति क. पाठः~॥ 

९ {\qt अन्तस्थाभिश्च संयुतम्} इति शिक्षोक्तेः~। 

५५० 

उद्दयोतपरिवृतप्रदीपप्रकाशितमहाभाष्ये \textendash\ 

[ १ अ. १ पा. १ यपस्पशाह्निके 

सूर्मि$=$शोभनामयःप्रतिमाम्, सुषिरां \textendash\ {\qt ऊषसुधि \textendash\ इति रप्रत्य \textendash\ 

येन} सच्छिद्रां प्रविश्याभ्निर्यथा तत्रत्यं मलं भस्मीकृत्य प्रतिमां
शुद्धां 

करोति, एवं तालुदेशे प्रकाशं प्राप्य विभक्तयः विभक्त्यन्ताः शब्दाः 

शारीरं पापमपाकुर्वन्तीत्यर्थः~॥ अनेन स्वर्गप्राप्तिः फलमित्युक्तम्~॥


( उक्तप्रयोजनग्रन्थोपपत्तिप्रकरणभ् ) 

( आक्षेपभाष्यम् ) 

किं पुनरिदं व्याकरंणमेवाधिजिगांसमानेभ्यः 

प्रयोजनमन्वाख्यायते, न पुनरन्यदपि किंचित् 

इत्युकत्वा वृत्तान्तशः {\qt शैम्} इत्येवमादीन् शब्दा \textendash\ 

न्पठन्ति ? 

(प्रदीपः ) किं पुनरिति~। ननु {\qt कानि पुनरस्य} इति येन 

पृष्टं स एव कथं पृच्छति \textendash\ {\qt किं पुनरइति~।} एवं तर्हि 

भाष्यकारः प्रयोजनान्वाख्यास्य विषयविभागं दर्शयति~। पुरा 

[उद्दयोते तदाह \textendash\ उ० ७ मप० ] सूर्मिमिति~॥ को \textendash\ 

शादिप्रामाष्याद्वीर्वान्तोऽप्ययमित्यन्य्देतत्~॥ सामर्थ्यलभ्यमर्थं
भाष्योक्त \textendash\ 

माह \textendash\ [ ८ मप० ] प्रविश्येत्यादि~॥ 

[उद्दयोते ] तत्तात्पर्थयमाह \textendash\ [ उ० मप० ] \textendash\ एवं ता \textendash\ 

ल्विति~॥ वरुणस्येति शेपः~॥ 

[ उद्दयोते ] तात्पर्यार्थमाह \textendash\ उद्दयोते \textendash\ अनेनेति~। ? वरुण \textendash\ 

तुस्यत्वप्रतिपादनेनेत्यधैः~॥ स यथा स्वर्गवासी एवमन्योऽपि तादृश 

इति भावः~॥ 

[भाष्ये ] \textendash\ भाष्ये किं पुनरिति~। पुनःशब्दो हेत्वर्थे वाक्या \textendash\ 

लंकारे वा~॥ एवमग्रेऽपि~॥ इदं प्रयोजनमित्यन्वयः~॥ 

[ प्रदीपे ] प्रयोजनान्वाख्यातारं प्रत्ययं प्रश्नः~। अत एव 

सस्या्यं प्रश्नो न, तत्प्रष्टुरपि नेत्याह \textendash\ कैयटे \textendash\ ननु कानीति~॥ 

तदाबश्यकत्वशानादानर्थक्यशङ्कायास्तस्यासंभवादिति भावः~॥ 

प्रश्नाकारमाह \textendash\ किमिति~॥ पूर्वतनं तु प्रतीकम्~॥ यत्तु \textendash\ 

फल्गन्युकत्वा तदन्वाख्यानस्यावश्यकतां वक्तुं तस्यैवायमाक्षेपः \textendash\ 

इति केचित्~। तन्न, अग्रे तथाऽनुक्तेः~। अत एव तद्विरोधापत्तेश्च~॥ 

तदेतद् ध्वनयन्नह \textendash\ कैयटे \textendash\ एवं तर्हीति~॥ भाष्यकार एव~॥ 

[ प्रदीपे ] कैयटे \textendash\ दर्शयतीति~। स्वयमेव प्रश्नपूर्वकं दर्श \textendash\ 

यतीत्यर्थः~॥ 

[प्रदीपे उत्तरग्रन्थे भाष्यकृतोक्तं विषयविभागमत्रैवाह \textendash\ कैयटे \textendash\ 

पुरेति~॥ युगान्तर इत्यर्थः~॥ व्याकरणमिति~। सदाद्यङ्गानीत्यर्थः~॥ 

तदध्ययने हेतुगर्भ विशेषणमाह \textendash\ [ ६ ष्प० ] प्रधानमिति~। 

[ प्रदीपे ] वीप्सायां शसित्याशयेनाः \textendash\ कैयाटे \textendash\ वृतान्त 

मिति~॥ प्रतीत्येति~। अधिङ्त्येत्यर्धः~। वीष्साद्योतको बा 

प्रतिः~॥ {\qt पटन्तीत्यर्थः} इति पाठरतु सुगम एव~॥ {\qt वृत्ताम्ततः}

१ पुनरिति~। व्याकरणमशिजिगांसमानेभ्यः प्रयोजनमन्वा \textendash\ 

ख्यायते, वेदमरधीजिगांसमानेभ्यः प्रयोजनं किं नान्वास्थायत इति 

भाष्याशयः~। तदनुरोधेनैवाग्रे उत्तरं पुरा कल्ये इत्यादि~। किंचि \textendash\ 

त्पदार्धमाह \textendash\ ॐ इत्युकस्वेति~। येऽन्यदपि किञ्चित् ओमित्युच्चार्य \textendash\ 

पाठमभ्युपगम्य विनैव प्रयोजनशार्न शमित्येबमादीन् पठन्ति तेभ्यः 

वेदाध्ययनात्पूर्वं व्याकरणमधीयते ते बाल्यात्प्रष्टुमसमर्था इति 

न प्रयोजनमन्वाख्येयम्~। अद्यत्वे तु स्वल्पायुष्ट्वात्पूर्वमेव वेदं 

प्रधानमधीयते, अतः प्रष्टुं समर्थत्वाद्दयाकरणाध्ययनस्य प्रयोजनं 

पृच्छन्तीत्यवश्यान्वाख्येयं प्रयोजनम्~॥ न पुनरन्यदिति~। 

वेदमप्यधिजिगांसमानेभ्य इत्यर्थः~॥ इत्युक्त्वेति~। 

अभ्युपगम्येत्यर्थः~। वृत्तान्तश इति~। वृत्तान्तः प्रपाठक 

उच्यते~। वृत्तान्तं र्वृत्तान्तं प्रति \textendash\ इत्यर्थः~॥ 

( उद्दयोतः ) एवं तर्हीति~। {\qt वेदातिरिक्तविषय एव प्रयोजना \textendash\ 

न्वाख्यानं, न तु वेदविषये} इत्येवरूपं \textendash\ विषयविभागम्~॥ तृतीयस्य 

प्रश्न इत्यन्ये~॥ किंचित्पदार्थमाह \textendash\ वेदमपीति~। शेषपूरणमन्यत्~। 

किं तत् किञ्चिदित्याकाङ्कीयां \textendash\ भाष्ये \textendash\ ओमित्युक्त्वेति~॥ 

{\qt पठन्ति} इत्यस्यानन्तरं तेभ्योऽपि इति शेषः~॥ एवञ्ज वेदमधि \textendash\ 

जिगांसमानेभ्यो वेदाध्ययनप्रयोजनं किमिति नान्वाख्यायत इति 

प्रश्नः~॥ वेदाध्ययनवत् व्याकरणाध्ययनमपि प्रयोजनमन्तरेव करि \textendash\ 

इति पाठे तु तृतीयान्ताद् द्वितीयान्ताद्वाऽऽधादित्वातसिः~। 

वीप्सान्तर्भावो बोध्यः~॥ 

[उद्दयोते] विषयविभारामित्यस्य {\qt भिन्नं भिन्नं विषयम्} इत्यथ \textendash\ 

निरासाथाह \textendash\ उद्दयोते \textendash\ वेदातीति~॥ 

[उद्द्योते ] उक्तरीत्या प्रष्टावक्त्रोः प्रश्नासंभवेऽपि
प्रकारान्तरेण 

भाष्यस्य समञ्जसत्वादगतिकगतिरूपकैयटोक्तिरयुक्तेत्याशयेनाह \textendash\ 

उक्ष्योते \textendash\ तृतीयस्येति~॥ उक्तद्वयभिन्नस्येत्यः~॥ 

[ उद्दयोते ] कैयटात् सर्वस्तदर्थ इति प्रतीयते, तन्निरासा \textendash\ 

याह \textendash\ उद्द्योते \textendash\ किंचिदित्यादि~॥ शेषेति~। अनुपङ्गेणेति 

भावः~॥ वैययेक्तकिंचित्पदार्धत्य भाष्यारूढदत्वायाह \textendash\ [ उ० 

३ यप० ] र्किं तविति~॥ [ ४ धंप० ] माहेति~। विशिष्याह \textendash\ 

त्यर्थः~॥ अत एवानन्वयवारणायाह \textendash\ पठन्तीत्येति~॥ रो 

इति~। अध्याहार इत्यर्थः~॥ प्रयोजनमन्वाख्यायत इत्यस्यानु \textendash\ 

षङ्गः~॥ शमिस्यस्यादावुक्तिस्तु प्राग्व्ज्ञेया~॥ एतावद्धन्धस्य सात्प \textendash\ 


यर्माह \textendash\ [ ५ मप० ] एवं चेति~॥ खत्र तदन्वाख्याने चेत्यर्थः~॥ 

प्रश्नस्यापि परमतात्पर्यमाह \textendash\ [ ७ मप० ] वेदेति~। एवं 

चेत्यादिः~॥ तदनन्वाख्याने चेति तदर्धः~॥ अथीष्वेत्याचार्यवचना \textendash\ 

देवेति भावः~॥ [ ७ मप० ] प्रयोजनं \textendash\ तज्शानम्~॥ कचित्तयैव 

पाठः~॥ न चादौ विप्रलम्भशङ्ककान् प्रति तदुक्तिरिति वाच्यम् ; 

तरहिं शतशस्तदुक्तावपि सच्छङ्कया प्रवृत्त्यनुपपत्तेः~॥ नचार्थात्प्र \textendash\ 

वृत्तान् प्रति प्रोत्साहनार्था तदुक्तिः, तर्ष्यन्यत्रापि तदर्थ तदुक्रा \textendash\ 


वश्यकत्वात्~॥ अथ तत्र तद्विनैव निर्वाहः, सर्ह्यत्रापि तथेति 

भावः~॥ यत्तु वेदाध्ययनप्रकारं तृतीय एवाह \textendash\ भाष्ये \textendash\ ओमिति~। 

आचार्येण {\qt लघीष्वः इत्युक्ते ओमभ्यादाने} इति विहितो \textendash\ 

दाचतविशिष्टणवभादायुञ्चायेत्यर्थः इत्याह् रलकृदादयः~। तज , 

प्रयोजनं किमु नानवाख्यात इत्यन्वयः २ रणमनिः इति एद \textendash\ 

शब्दरहितः फ. पाठः~॥ ३ {\qt शमित्येव शमादीन्} एति क. पाठः 

४ {\qt वृत्तान्तं पठन्तील्यर्थः} इति क. पाठः~॥ ५ {\qt काङ्षायामाह 

भाष्ये} इति च पाठः~। 

प्रयोजनग्रन्थोपपत्तिप्रकरणम् ] महाभाष्यप्रदीद्देतव्याख्या छाया~। 

ष्यन्तीति प्रयोजनविचारो व्यर्थ इति तात्पर्यम्~॥ नन्वोमित्यधिकमत~। 

आह \textendash\ अभ्युपगम्येति~। विनैव प्रयोजनज्ञानमिति शेषः~॥ 

(समाधानभाष्यम् ) 

पुराकल्पएतदासीत् \textendash\ संस्कारोत्तरकालं ब्राह्मणा 

व्याकरणं स्माधीयते, तेभ्यस्तत्तत्स्थानकरणनादानु \textendash\ 

प्रद \textendash\ वज्ञेभ्यो वैदिकाः शब्दा उपदिश्यन्ते~। अद्यत्वै 

न तथा~। वेदमधीत्य त्वरिता वक्तारो भवन्ति~। 

{\qt वेदान्नो वैदिकाः शंब्दाः सिद्धाः}, 

लोकाच्च लीकिकाः, 

{\qt अनर्थकं व्याकरणम्} इति~। तेभ्य एवं विप्रति \textendash\ 

पन्नवुद्धिभ्योऽयेतृभ्य आचार्य इदं शास्त्रमन्वाचष्टे \textendash\ 

इमानि प्रयोजनान्यध्येयं व्याकरणम् \textendash\ इति~॥ 

( प्रदीपः ) अद्यत्वेशव्दो निपातः \textendash\ अस्मिन् काल इत्यै \textendash\ 

त्रार्थे वर्तते~॥ त्वरिता इति~। विबाहादौ~॥ 

{\qt एवं सति ओ३म्} इत्येव वाच्ये इत्युक्त्वैस्यत्य बैयर्थ्वस्य दुष्परि \textendash\ 

हरत्वात्~। तत्र तथा प्रकारसत्त्येऽपि प्रकृते तत्कथनानुपयोगाच्च~। 

तदेतद् ध्वनयन्नाह \textendash\ [ उ० ८ मप० ] नन्वोमित्यधिकमिति~॥ 

ओमित्युक्त्वेत्यधिकमित्यर्थः~॥ क्वचित्तयैव पाठः~॥ अभ्युपगम्ये \textendash\ 

तीति~। विशिष्टस्यायमर्थ इति भावः~॥ अत एव कैयटेन तथैवो \textendash\ 

क्तम्~॥ विनैवेति~। अयीष्तेत्युक्ते तदाश्नामित्यादिः~॥ 

[ भाष्ये ] एतत्पदार्थमाह \textendash\ भाष्ये \textendash\ संस्कारोत्तरेत्यादि 

दिश्यन्त इत्यन्तेन~॥ क्रियाविशेषणमैतम्~॥ 

[भाष्ये] \textendash\ भाष्ये \textendash\ नब्राह्मणा इति~। प्राग्वत्~॥ व्याकरण \textendash\ 

मित्पङ्गानामुपलक्षणम्~। प्राधान्यादस्योल्लेखः~॥ ततः किमित्याशङ्का \textendash\ 

याभत एवाह \textendash\ [ २ यप० ] तेभ्य इत्यादि~। अङ्गाध्येतृभ्य

इत्यर्थः~। स्थानं \textendash\ प्रसिद्धं कण्ठादि~॥ 

[भाष्ये] नन्येवं त्रिदिष्टवेदपाठसिद्धिरेवेत्येतावता किमत आह \textendash\ 

भाष्ये \textendash\ वेदमिति~॥ 

[ भाषयै ] प्रवक्तार इत्यत्र कर्माह \textendash\ भाष्ये \textendash\ वेदान्न इत्यादि \textendash\ 

मितीत्यन्तेन~॥ नः \textendash\ अस्माकम्~। कर्तरि शेपत्वविवक्षया पष्ठी~॥ 

सिद्धा लोकाच्च \textendash\ त्चवहाराच्च~॥ अध्यापकशिक्षयैव स्थानादिज्ञान \textendash\ 

मक्षरग्रहणवदिति एतदनर्थकमित्याह \textendash\ अनर्थेति~॥ विप्रतीति~। 

विरुद्धमर्थं प्रतिपन्ना \textendash\ विप्रतिपन्ना, सा वुद्धिर्थेषां तेभ्य इत्यर्थः
~॥ 

[भाष्यै ] तदेवाह \textendash\ भाष्ये \textendash\ इमानीति~॥ यद्यपि मन्वादि \textendash\ 

स्मृत्या वेदवेदाक्षाध्यभनयोः सकारणता लभ्यते तथाप्यार्षत्वाविशेषाद् 

ब्रीहियववद् विकल्प इति बोध्यम्~॥ व्याकरर्णं \textendash\ प्रागुक्तम्~॥ 

[प्रदीपे] त्वेशब्दानर्थक्यं परिहरति \textendash\ कैयटे \textendash\ अद्यत्वे इति~॥ 

[ प्रदीपे ] शेधं पूरयति \textendash\ कैयटे \textendash\ विवाहादाविति~॥ तत्रो \textendash\ 

युक्ताः सन्त इत्यर्थः~॥ 

[ उद्दयोते ] यत्तु \textendash\ किंचिदित्यन्त एव पूर्वपक्षः~। ओमित्यादि 

परिहारः \textendash\ इति~। तन्न समञ्समित्याश्ययेनाहे \textendash\ उद्द्योते \textendash\ भाष्ये 

उतरमिति~॥ 

१ {\qt तदद्यत्वे} इति क. च. फ. पाठः~॥ २ {\qt शब्दाः इत्यस्य} फ.

पुस्तकेन पाठः~॥ ३ {\qt ध्येतृभ्यः सुहृद्भूत्वाऽऽचार्य?} इति च. पाठः~॥ 

महाभाष्यप्रदीपोद्दयोतव्याख्या छाया 

(उद्दयोतः) भाष्ये \textendash\ उत्तरमाह \textendash\ पुराकल्पे इति~। युगान्तरे 

इत्यर्थः~॥ संस्कारः \textendash\ उपनयनम्~॥ करणं \textendash\ आभ्यन्तरप्रयत्नः~॥ अनु \textendash\ 

प्रदानं \textendash\ नादादिबाह्यप्रयत्नः~। तेनाधीतव्याकरणशिक्षेभ्य इत्यर्थः~। ते 

तदानीमपि कमानुष्टानफलवत्वं वेदस्य जानन्तो नित्यत्वं वा तदध्यय \textendash\ 

नस्य जानन्तो न तदध्ययनप्रवोजनं पृच्छन्तीति न कदाचिदपि वेदा \textendash\ 

ध्ययनप्रयोजनोपदेश इति तात्पर्यम्~॥ (भाष्ये ) \textendash\ न तथेति~। किं तु 

विपरीतमित्यर्थः~। भाष्ये \textendash\ आचार्थपदेन शास्त्राध्यापको भाष्यकृदेव 

विवक्षितः~॥ इदं शास्त्रमिति~। प्रथोजनान्बीख्यानमित्यर्थः~॥ 

(अनुबन्धचतुष्टयोपसंहारभाष्यम् ) 

उक्तः शब्दः~। स्वरूपमप्युक्तम्~। प्रयोजनान्य \textendash\ 

प्युक्तानि~॥ 

( उद्दयोतः ) भाष्ये \textendash\ उक्तः शब्द इति~। {\qt लौकिकानां 

वैदिकानां च} इत्यमेन विपयभूतः शव्द उक्त इत्यर्थः~॥ स्वरूप \textendash\ 

[उद्दयोते ] तत्र चौलान्तसंस्कारोत्तरं तदभावादाह \textendash\ [ उ०. 

२ यप०~। ] उपेति~॥ 

[उद्दधोते]उद्दयोते \textendash\ नादादीसि~। भाष्यस्थनादपदमुपलक्षणं 

विवारादीनामिति भावः~॥ यद्यपि सर्वाङ्गाध्ययनं पूर्वम्, तथापि 

वैदिकशब्दोपदेशे व्याकरणशिक्षाऽन्योपयोगाभावेन फल्तार्थमाह \textendash\ 

[ ३ यप० ] तेनाधीति~॥ वर्णोच्चारणप्रकारस्य स्थानादेः प्रकृत्यादि \textendash\ 

विभागेन तदर्थरूपकथनस्य च तयोः संत्त्वादिति भावः~॥ नन्वङ्गा \textendash\ 

ध्ययनक्राले वाल्यात् प्रयोजनप्रश्चासामर्थयाद् वेदाध्ययनकाले प्रौढत्वात्


तत्सामर्थ्येऽपि कुतो न तदुपदेश हत्यत आह \textendash\ ते तदानीमपीति~॥ 

वेदाध्ययनकाले प्रौढतायामपीत्यर्थः~॥ अपिनेदा्नीं बाल्यकाल \textendash\ 

समुच्चयः~॥अत एवात्रे न कदाचिदरपीत्युक्तिः~॥ [ धप० ] 

फलकत्बम् \textendash\ अर्थज्ञानद्वारा~॥ 

[ उद्दयोते ] {\qt तदद्यत्वेः इति रलकृद्धृतभाष्यपाठा 

ध्वनयन्नाह} \textendash\ उद्दयोतैे \textendash\ किंत्विति~। अस्पायुद्ट्वात्प्रवानीभूतवेदा \textendash\ 

ध्ययनोत्तरमङ्गाध्य्रयनमित्यर्थः~॥ 

[ उद्दयोते ] अन्यस्यासंभवादाह \textendash\ [ उ० ७ मप० ] \textendash\ भाष्ये 

आचार्येति~। तत्त्वे बीजमाह \textendash\ शासख्त्रेति~॥ इदं शास्त्र \textendash\ 

मित्यनेन मप्रागुक्तव्याकरणपरामर्शै तदन्वाख्यानस्यानुत्तरत्वादुत्तर \textendash\ 

ग्रन्थानुरोधाच्च तस्यार्थमाह \textendash\ [ उ० ८ मप० ] प्रयोजनान्वा \textendash\ 

ख्यानमिति~। शास्त्रं ध्यावरणमिदं सप्रयोजनाद्ध्येयमिति 

कथयतीति भाष्यार्थं इत्यर्थः~॥ 

[ भाष्ये ] अथ पाणिनीयशास्त्रसातइयकत्वपिपादनाय परमतं 

निरसितुमाक्षेपपूर्वकतात्पर्य वक्तुमुक्तं संगृह्यति \textendash\ भाष्ये \textendash\ उक्ता 

शब्द इत्यादि~॥ 

[भाष्ये] भाष्ये \textendash\ प्रयोजनेति~। {\qt कानि} इत्यादिना उक्तः \textendash\ 

इत्यतः प्राक्तनग्रन्थेनेति भावः~॥ 

[उद्द्योते] \textendash\ उद्द्योते \textendash\ विषयेति~। लौकिको वैदिकश्च सर्वः 

४ {\qt अद्यत्व इति~। अद्यत्वशब्दो इति क. च. पाठः~॥ ५ इत्यस्मि \textendash\ 

न्थे} इति कः पाठः~॥ ६ {\qt जनान्युक्ता \textendash\ } इति फ. अपिशब्दरहितः~॥ 

५३ 

उद्दयोतपरिवृतप्रदीपप्रकाशितमहाभाष्यम्~। 

[१ अ. १ पा. १ पस्पशाहिकै 

मपीति~। {\qt अथ गौः इत्यादिना~॥} अयमुपसंहारो ग्रन्थस्य, विषय \textendash\ 

प्रयोजननिरूपणमेतावता कृतमिति बोधयितुम्~। तेनैव संबन्धाधि \textendash\ 

कारिणाबुक्ताविति तौ पृथङ्गोक्तौ~॥ 

( इत्यनुबन्धच्चतुष्टयनिरूपणम् 

(अथ शास्त्रनिर्माणरीतिनिरूपणाधिकरणम् ) 

( ग्रकारप्रदर्शकभाष्यम् ) 

शब्दानुशासनमिदानीं कर्तव्यम्~। किं शब्दोप \textendash\ 

देशः कर्तैव्यः, आहोस्विदपशब्दोपदेशः, आहो \textendash\ 

स्विदुभयोपदेश इति ? 

(प्रदीपः) उभयोपदेश इति~। हेयोपादेयोपदेशे स्पष्टा 

प्रतिपत्तिर्भवति \textendash\ इत्युभयोपदेश उद्धावितः~॥ 

( प्रकारनिद्धारकभाष्यम् ) 

अन्यतरोपदेशेन कृतं स्यात्~। तद्यथा \textendash\ भक्ष्यनिय \textendash\ 

मेनाभक्ष्यप्रतिषेधो गम्यते~। {\qt पञ्च पञ्चनखा}

भक्ष्याः इत्युक्ते गम्यत एतत् \textendash\ अतोऽन्येऽभक्ष्या 

इति~॥ 

शब्दः शास्त्रविषय इत्युक्तंमिति भावः~। त्यादिनेति~। ध्वनिः 

शब्द इत्यन्तेनेति भावः~॥ 

[ उद्दयोते ] ननु शब्दानुशासनमित्येव सिद्धे निष्प्रयोज \textendash\ 

नोऽयं संग्रहोऽत आह \textendash\ उद्दयोते \textendash\ अयमुयेति~॥ ग्रम्भस्य \textendash\ 

पूर्वोक्तस्य~। पूर्वान्वयीदम्~॥ विषयेति~। {\qt अथ शब्दानुशास \textendash\ 

नम्} इति शास्त्रारम्भप्रतिज्ञापरवाक्यसूरचितमित्यर्थः~। शिष्य \textendash\ 

कर्यार्थमिति भावः~॥ तेनैवतयोरभिधानेनैव~॥ वुक्तावीति~। 

तप्रायावित्यर्थः~॥ पृथगिति~। साक्षादित्यर्थः~॥ वक्ष्यमाणप्रविनेकाय 

अक्तः इत्यादिनोक्तसंकीर्तनमिति वोध्यम्~॥ 

[भाष्ये] तदेतद्धवनयन्नबन्धचतुष्टसत्तवादार शास्त्रे 

तत्प्रकारनिरूपणायाह \textendash\ भाष्ये \textendash\ शब्देति~॥ इदानीं \textendash\ तञ्चतुष्टय \textendash\ 

निश्चयकाले~॥ कर्तव्यमिति~। अस्य {\qt द्ति सिद्धम्} इति शेषः~॥ 

तत्र साभान्येन प्रश्नं प्रागुक्ते करोति \textendash\ कथमिरसि~॥ एकैकोपदेशे 

न कृतकृत्यधीः, उभयोपदेशे गौरवमित्यशक्यमनुशासनमिति भावः~॥ 

स एव विशिष्याह \textendash\ किं शब्दोषेति~॥ आहोस्वित् \textendash\ अथवा~॥ 

[ प्रदीपे ] ननु गौरवात्तृतीयपक्षो दुर्वचोऽत आह \textendash\ कैयंटं \textendash\ 

हेयोपेति~॥ 

[ भाष्ये ] \textendash\ भाष्ये \textendash\ कृतं स्यादिति~। विनिगञनाविरहात् 

तेन सर्वेष्टं सिद्धं स्यादित्यर्थः~॥ 

[ भाष्ये ] \textendash\ द्ृष्टान्तेनैतदुपपादयति \textendash\ भाष्ये \textendash\ तद्यथेति~॥ 

यथेत्यर्थः~। यद्वा तेन स गम्यते यथा तथा तदित्यर्थः~॥ यद्धा 

तत् तादृशं यथेत्यर्थः~॥ नियमेनेति~। रागतः प्राप्तस्य भक्ष्यस्य 

१ {\qt कतव्यम्~। तत्कथं कर्तव्यम्~। किं श} इति~। फ. पाठः~। 

{\qt कर्तव्यं~। कथं कर्तव्यं~। किं इति ख. क. पाठः~॥ २ }प्रतिषेधेन वा 

मक्ष्य इति कः ख. च फ. पाठः~॥ ३ {\qt क्रियते} इति क् ख. फ. 

अभक्ष्यप्रतिषेधेन च भक्ष्यनियमः~। तद्यथा \textendash\ 

{\qt अभक्ष्यो ग्राम्यकुक्कुटः \textendash\ अभक्ष्यो ग्राम्यसूकरः} 

इत्युक्ते गम्यत एतत् \textendash\ आरण्यो भक्ष्य इति~॥ 

एवमिहापि~। 

यदि तावच्छब्दोपदेशः क्ियते, गौरित्येतस्मिन्नु \textendash\ 

पदिष्टे गम्यत एतत् \textendash\ गाव्यादयोऽपशब्दा इति~। 

अथाधप्यपशब्दोपदेशः क्रियेत, गाव्यादिषूपदि \textendash\ 

ष्टेषु गम्यत एतत् \textendash\ गौरित्येष शब्द इति~॥ 

( प्रदीषः ) यर्द्यपि प्रतिपत्तिः स्पष्टा, गौरवं तु भवतीत्याह \textendash\ 

अन्यतरेति~। शब्दापशब्दयोरित्यर्थः~। अन्यतरान्यतमशब्दा \textendash\ 

वव्युत्पन्नौ स्वभावात् द्विबहुविषये निर्थारणे वर्तेते~॥ पञ्चेति~। 

अर्थित्वाद्भक्षणं प्राप्तं पञ्चसु पञ्चनखेषु निगम्यमानं सामर्थ्याद \textendash\ 

न्येभ्यो निर्वर्तते~। न त्वयं विधिः, अप्राप्तेरभावात्~॥ 

( उद्दयोतः ) किंययत्तऽन्यत्र डतराभावादाह \textendash\ अन्यतरा \textendash\ 

स्येति~। तत्र सुर्तैब्दोपदेशः प्रयोगविध्यर्थोऽनुवादः~। अपशब्दोप \textendash\ 

देशस्तु प्रयोगप्रतिषेधार्थः~॥ पञ्चनखा इत्वस्य नियमत्वं दर्शयति \textendash\ 

अर्थित्वादिति~। नियम्यमानं \textendash\ बोध्यमानमित्यर्थः~॥ ननु अस्य 

परिसङ्ख्यात्वात्कथं नियमत्वेन व्यवहारः ? अस्ति च नियरुधरि \textendash\ 

पुनर्विधिनेत्यर्थः~॥ तस्यार्थिकत्वादाह \textendash\ गम्यत इति~॥ एवम \textendash\ 

ग्रेऽपि~॥ एतदेव स्फुट्यति \textendash\ पञ्चेति~॥ 

[भाष्ये ] \textendash\ एवमुपादेयनियमेनार्थात्तदन्यनिषेधप्रतीतौ दृष्टान्त \textendash\ 

मुक्त्वा निषेध्यनियमेनार्थात्तदम्योपादेयप्रतीतौ दृष्टान्तमाह \textendash\ 

भाष्ये \textendash\ अभक्ष्येति~। तदुपदेशेनेत्यर्थः~॥ 

[भाष्ये ] \textendash\ दार्ध्टन्तिके आह \textendash\ भाष्ये \textendash\ एवमिति~॥ गोश \textendash\ 

व्दादिः साधूत्वोत्तंयन्यथानुधपतया तदिरुद्धमसाधुत्बमन्वेषामर्था \textendash\ 

वुक्तं भवतीति भावः~॥ एवमग्रेऽपि~॥ 

[ प्रदीपे ] कैयटे \textendash\ तीत्याहेति~। इत्यतः सिद्धान्तेकदे \textendash\ 

श्याहेत्यर्थः~॥ 

[ भाष्ये ] तदेतद्ध्वनयश्ननुबन्धचतुष्टयसत्त्वादारब्धव्ये शास्ल्ने 

[ प्रदीपे ] यत्तु \textendash\ अत्र {\qt शक्येन} इति विशेषणं देयम्~। 

अपशब्दोषदेशस्त्वभंभावित एव अगशक्यत्वादिति स्वयमेवाग्रे उक्त \textendash\ 

त्वात् \textendash\ इति कृष्णः~। तन्न, तद्वक्तत्रैव तस्याप्यशक्यत्वस्योक्त \textendash\ 

त्वात्~। तस्याग्रे प्रतिपाद्यत्वेनात्राविवक्षितत्वाच्च~॥
द्वितीयादिद्कया \textendash\ 

न्यतरत्वनिरासायात \textendash\ कैयटे \textendash\ शब्देसि~॥ 

[ प्रदीपे ] \textendash\ कैयटे \textendash\ द्विबह्निति~। यथासंख्यमन्वयः~॥ 

[प्रदीपे] \textendash\ कैयटे \textendash\ भक्षणमिति~। युगपद् सर्वेषामित्यादिः~। . 

[प्रदीषे] \textendash\ अत एवाह \textendash\ कैयटे सामर्थ्यादिति~॥ 

[ उद्दयोते ]उद्वयोते \textendash\ डतराभावादिति~। सूत्रेण तेभ्य एव 

विधानादिति भावः~॥ 

[ उद्दयोते ] \textendash\ नियमनस्यूप्राप्तांशपरिपूरणफलकत्वेन विधिरू \textendash\ 

पेण प्रबृत्तिः कैयटे पुनरुक्त्यापत्तेश्चाह \textendash\ उद्द्योते \textendash\ पुनरिति~॥ तत्र 

शङ्कते \textendash\ [मप० ] नन्वस्येति~॥ अस्ति चेति~। अस्ति ह्रीत्यर्थः~॥ 

पाठः~॥ ४ ननु शब्दाशब्दप्रतिपत्तिरभयोपदेशेनामन्दिग्धा स्यात्त \textendash\ 

दाह \textendash\ यद्यपीति~॥ ५ सुशब्दः \textendash\ साधुशब्दः~॥ ६ {\qt पञ्च पञ्चनखा} 

इति घ, पाठः~॥ ७ {\qt नियम्यमानं पुनः शब्देन बोध्यः} इति ध. च.~॥ 

शास्त्रनिर्माणरीतिनिरूपणाधिकरणम् ] महाभाष्यप्रदीपौद्दयोतव्याख्या छाया
~। 

संख्ययोर्भेदः~। पाक्षिकाप्राप्तिकात्राप्नांशपरिपूरणफलो नियमः~। 

अन्यनिवृत्तिफला च परिसंख्या \textendash\ इति चेत्, न~। नियमेऽप्यप्राप्तां \textendash\ 

शपरिपूरणरूपफलवोधनद्वाराऽर्थान्यनिवृत्तेः सत्त्येन अभेदमाश्रि \textendash\ 

त्योक्तेः~॥ विधिरिति~। अपूर्वविधिरित्यर्थः~॥ ये तु पञ्चपदस्य तद \textendash\ 

तिरिक्ते लक्षणा, भक्ष्यपदस्य चाभक्ष्ये सा \textendash\ इति वदन्ति तेषां पदद्वये 

लक्षणागौरवम्~। अस्यैव नियमत्वेन {\qt भक्ष्यनियमेनाभक्ष्यप्रतिषेधो 

गम्यते} इति भाष्यविरोधश्च~। नियमपदस्य तद्वोधके लक्षणायां सैव 

दोषः, अक्षरास्वारस्यं च~। {\qt नियम्यमानं भक्षणं सामर्थ्यादन्येभ्यो 

निवर्तते} इति केयटासंगतिश्च~। दुष्टान्तदार्ष्टान्तिकथोरत्यन्तनैपम्याप \textendash\ 


त्तिश्च~। {\qt अन्यनिवृत्तिफला परिसंख्या} इति अभियुक्तोक्तिविरोधश्च~। 

सिद्धस्य पुनर्विनात्प्रापकप्रमाणस्य तदितराविषयत्वरूपसद्कोच्चक \textendash\ 

ल्पनभ्~। तेनान्येषां भक्षणाभावस्तद्भक्षणे प्रायश्चित्तं च सिध्यति~। 

एवसमभिव्याहारे सोऽपि स्फुटमस्यार्थस्यैव द्योतकः~। अन्ययोगव्य \textendash\ 

मच्छेद एवकारार्थ इत्ति तु सामान्यप्रमाणसंकोचफरलितमर्थमादाय 

नेयम्~॥ भक्ष्यनियम इति~। भक्ष्यानुमतिरूप उपदेश इत्यर्थः~॥ 

( प्रकारविशेषजिज्ञासाभाष्यम् ) 

किं पुनरत्र ज्यायः ? 

[६ ष्ठप० ] पक्षिकेति बहुव्रीहि~। अप्राप्तांशविशोषणम्~॥ 

फलाच \textendash\ फला तु~॥ अपिः \textendash\ परिसंख्यासमुच्चायकः~॥ अभेद \textendash\ 

मिति~। नियमपरिसंख्ययोरिति भावः~॥ 

[ उद्दयोते ] \textendash\ निवनविधेः सत्त्वाद् उत्तरानुरोधाच्चाद \textendash\ 

उद्दयोते \textendash\ [ ९ मप० ] \textendash\ अपूर्वेति~॥ ननु परिसंख्यायां शाब्दा \textendash\ 

न्यनिवृत्तिः सत्त्वेन कथं तत्रार्थान्यनिवृत्तिरुक्ताऽव आह \textendash\ ये 

त्विति~। तत्रैतरनिवृत्तिः शाब्दत्वात्पञ्चपत्नखातिरिक्ता भक्ष्य \textendash\ 

त्वाभाववन्त इति बोधादिति भावः~॥ अस्यैव \textendash\ उक्तस्यार्धख्यैव~॥ 

पत्तिश्चेति~। नियगेनेत्यस्योपदेशेनेत्येवार्थः संदर्मबलाल्लभ्यते~॥ 

शब्दान्तरेणानुपदेशादर्थतो लाभाच्च नायं मुख्य उपदेश इति 

नियमपदप्रयोगः~॥ इदृश इतरतिवृत्तिफल उपदेश एव च नियमप \textendash\ 

देनोच्यप्ष इति भावः~॥ [ १५ शप० ] युक्तोक्तिविरोधश्चेति~। 

तस्यास्तद्रीत्यः शाब्दत्पेन फलत्वोक्तेरयुक्तत्वादिति भावः~॥ नन्त्रे \textendash\ 


वमन्येषां भक्षणे दोषो न स्यात् तद्भक्षणे प्रायश्चित्ताम्नानानर्थक्यं 

च स्यादतः स्वमते सर्वेष्टसिद्धिमाह \textendash\ सिद्धस्येति~॥ प्रमाणेति~। 

रागरूपेत्यधः~॥ अयं भावः \textendash\ पंञ्चत्याद्राववधारणविषयवुद्धिस्थ \textendash\ 

वृत्तिपञ्चत्वसंख्यतति पञ्चपदस्य लक्षणा~॥ अवधारणं चात्र 

ष्वान्यसंख्याधिकरणासंबन्धिसमभिव्याहृतप्रकारवश्ञानरूपम्~। तेन 

 \textendash\ अन्येऽभक्ष्याः \textendash\ इति फलतीति~॥ एवं चैवकारसमभिव्याहा \textendash\ 

रेऽप्यार्थ्थैव परिसंख्या~। सा चोक्तान्यत्र समभिव्याहृतपदार्था \textendash\ 

भावरूपा~॥ शाब्दी त्वन्यत्र स्पष्टा~॥ तदाह \textendash\ एवेति~॥ सोऽपि 

एवकारोऽपि~॥ विरोधं परिहरति \textendash\ [१८ शप०] अन्ययोगेति~॥ 

[ उद्दयोते ] तत्र नियमत्वाभावादाह \textendash\ उद्द्योते [२ यप० ] 

भक्ष्यानुमतिरूप उपदेश इत्यर्थ इति~। 

१ {\qt लिकादयोऽपि} इति फ. पाठः~। 

५३ 

( प्रदीपः ) कि पुनरिति~। उभयोपदेशाद्युरोर्द्वावपि 

प्रशस्यौ, तयोः को ज्यायानित्यर्थः~॥ 

( निश्चितपक्षदर्शकभाष्यम् ) 

लघुत्वाच्छब्दोपदेशः~। लघीयाञ्छव्दोपदेशः~। 

गरीयानपशव्दोपदेशः~। एकैकस्य शव्दस्य वह \textendash\ 

वोऽपभ्रंशाः~। तद्यथा \textendash\ गौरित्यस्य शब्दस्य गावी 

गोणी गोता गोपोतलिका \textendash\ इत्येवमादयोऽपभ्रंशाः~। 

इष्टान्वाख्यानं खल्वपि भवति~॥ 

( प्रदीपः ) इष्टेति~। साधूप्रयोगाद्धर्मावाप्तेरित्यर्थः 

अथवा \textendash\ उपादेयौपदेशात्साक्षात्प्रतिपत्तिर्भवतीति भावः~॥ 

( आक्षेपभाष्यम् )

अथैतर्मिऽशव्दीपदेशे सति कि शब्दानां प्रति \textendash\ 

पक्तौ प्रतिपदपाठः कर्तव्यः \textendash\ गौरश्वः पुरुषो 

हस्ती शकुनिर्मृगो ब्राह्मण इत्येवमादयः शब्दाः 

पठितव्याः ? 

( उद्दयोतः ) भाष्ये {\qt प्रतिपत्तौ}इत्यस्योपायभूत इति शेषः~॥ 

(समाधानभाष्यम् ) 

नेत्याह~। अनभ्युपाय एष शब्दानां प्रतिपत्तौ 

[ भाष्ये ] अपशब्दोपदेशस्य गुरुत्वं व्यक्तीकर्तु पृच्छति \textendash\ 

भाष्ये \textendash\ किं पुनरिति~॥ ज्याय इति~। उपदिष्टमित्यर्थः~॥ एवं 

च कैयटोक्तार्धः फलित इति भावः~॥ 

[ प्रदीपे ] \textendash\ प्रकर्ममुपपादयन्नाह कैयटे \textendash\ उभयोषेति~॥ 

द्वावपीति~। प्रत्येकमिति भावः~॥ 

[ भाष्ये ] पदेशं इसि~। ज्यायानिति शेषः~। तेषामल्पत्वा \textendash\ 

न्नियतत्वेन प्रयुक्तत्वाच्चेति भावः~॥ एतदेवोपपादयति \textendash\ लघी \textendash\ 

यानित्यादिना~। अत्रोभयत्रेयसुन् प्राग्वत्~॥ न केवलं लाघव \textendash\ 

मेव शब्दोपदेशे, भिंत्वावश्थकताऽपीत्याशयेनाह \textendash\ इष्टान्वेति~। 

इष्टजनकान्वाख्यानमपि निश्चयेन भवतीत्यर्थः~॥ 

[प्रदीपे) \textendash\ तदाह \textendash\ कैयट \textendash\ साधुशब्देति~। सर्वत्रेत्यादिः~॥ 

धर्मेति~। अध्धत्रध्यापकयाभूयसा धमेण योग इत्यर्थः~॥ तथाच 

तेऽवश्यं ज्ञेया इति तदुपदेशे इष्टसिद्धिरति भानः~॥ इ्ष्टानां 

सुशब्दानां साक्षाज्ज्ञानं निश्चयेन भवतीति भाष्यार्थाशयेनाह \textendash\ 

[ २ यप० ] अथवोषेति~। 

[भाष्ये ] \textendash\ एवं तेनः शब्दोपदेशे कर्तव्यत्वेन व्यवस्थापिते 

तदुपदेशेऽप्यशक्यत्वं प्रतिपादयितुं प्रकारविशेषजिशासया पृच्छति \textendash\ 

भाष्ये \textendash\ अथैतस्मिन्निति~। लबुभूते आवश्यके चेत्यर्थः~॥ 

शब्दोपदेशे सतीति~। कतंव्यस्वेन व्यवस्थापिते इत्यादिः~॥ 

[भाष्ये ] \textendash\ प्रतिपदेति~। पदस्य पदस्य पाठ इत्यर्थः~। अत 

एव प्रष्टैव तस्स्वरूपमाह \textendash\ गौरश्च इति~॥ 

[उद्द्योते \textendash\ अनन्वयवारणायाह \textendash\ उद्दोते \textendash\ प्रतीति~॥ 

[ भाष्ये ] \textendash\ नेत्याहेति~। प्राग्वत्~॥ ({\qt नेत्याह~। द्रव्यं नाम 

तत्} इति भाष्यव्याख्यायामुक्तवत् ) 

२ {\qt साधुशब्दप्रयोगात्} इति क. घ. च. पाठः~। 

५४ उद्दयोत्परिवृतप्रदीपप्रकाशितमहाभाष्यम् [१ अ.१ पा.१ पस्पशाहिके 

~। प्रतिपदपाठः~॥ एवं हि श्रूयते \textendash\ {\qt बृहस्पतिरिन्द्राय 

दिव्यं वर्षसहस्रं प्रतिपदोक्तानां शब्दानां शब्दपा \textendash\ 

रायणं प्रोवाच नीन्तं जगाम~॥} बृहस्पतिश्च

प्रवक्ता, इन्द्रश्चाध्येता, दिव्यं वर्षसहस्रमध्ययन \textendash\ 

कालः, न चान्तं जगाम~। 

किं पुनरद्यत्वे ? यः सर्वथा चिरं जीवति \textendash\ वर्ष \textendash\ 

शतं जीवति~। 

चतुर्भिश्च प्रकारैर्विद्योपयुक्ता भवति \textendash\ आगम \textendash\ 

कालेन, स्वाध्यायकुालेन, प्रवचन्कालेन, व्यवहार \textendash\ 

कालेनेति~। तत्र चीस्यागमकालेनैर्वायुः पर्युपयुक्तं 

स्यात्~। तस्मादनभ्युपायः शब्दानां प्रतिपत्तौ प्रति \textendash\ 

पदपाठः~॥ 

[भाष्ये ] \textendash\ इदमेव विशदयितुं श्रुतिमूदाहरति भाष्ये \textendash\ 

[ ५ भप० ] एवंहीति~॥ 

[ भाष्ये] \textendash\ भाष्ये द्विव्यमिति~। मामुपं वर्षं दिव्यमहोरात्रम् ,

तैन मानेन दिव्यं वर्षसहस्रं वोध्यम्~। {\qt कालाध्वनोः} \textendash\ इति 

द्वितीया~॥ 

[भाष्ये ] भाष्ये \textendash\ {\qt जगामेतिः इति पाठः~॥} विशिष्टवक्तृश्नातृ \textendash\ 

कालानां समवधानेऽपि पुरा समाप्तिर्नाभूत् किं पुनर्वाच्यम् {\qt संप्रति 

तपां हीनत्वेऽपि सा} इत्यभिप्रेत्य व्याचष्ठे \textendash\ शृहस्पतिश्चेति~। 

चकारौ \textendash\ साधनसाहित्यवाच्चिनायाधुनिकप्रवक्ताध्यैतारो व्यतिरेच \textendash\ 

धतः~। किं पुनरद्यत्वे इति~। अस्मिन्कालेऽन्तं गमिष्यतीति 

किंपुनर्वक्तव्यमित्यर्थः , पूर्वान्वयीदम्~॥ ननु रसायनादिनेदानीमपि 

मनुष्याणां वडुकालजीबित्वात्सा सुलभेत्यत आह \textendash\ यः सेति~॥ 

अद्यत्त्र इत्यस्य मध्यमणिन्यायेनात्राप्यन्वयः~॥ सर्वथा \textendash\ अत्य \textendash\ 

न्तन्~॥ सर्वचिरमिति पाटे सर्वेभ्यश्चिरमिति {\qt सुप्सुपा} इति समासः~। 

वर्षशतमिति~। पषष्ठीतत्पुरुषः~॥ द्वितीया प्राग्वत्~॥ इदानीं 

रसायनादिसिद्धेरभावादिति भावः~॥ 

[भाष्ये \textendash\ [७मप०] प्रकारैः \textendash\ प्रकरियते व्यवच्छिद्यतेऽनेनेति \textendash\ 

प्रकारः \textendash\ सामान्यस्य भेदको विशेषः~॥ विद्यसेऽनयेति \textendash\ विद्या , 

उपयुक्ता \textendash\ फलवती~॥ क्वचिष् {\qt उपयुज्यते} इतिं पाठः~॥ अत 

एवोद्दथोतेऽग्रे तथोक्तम्~॥ 

[भाष्ये] \textendash\ तानेवाह \textendash\ भाष्ये \textendash\ [८ मप०] आगमेत्यादि~॥ 

[भाष्ये] व्यधहारेत्यस्य यज्ञाद्यनुष्ठानाध्यापनकालेनेत्यर्थः~॥ 

[भाष्ये) \textendash\ ततः किमत आह \textendash\ भाष्ये [ ९ मप० ] तत्र 

चेति~। तेषां कालानां मध्ये~॥ अस्य \textendash\ वक्तुः श्रोतुश्च~॥ आगम \textendash\ 

कालेनैव \textendash\ कृत्वेत्याद्यर्थः~॥ 

[भाष्य \textendash\ उपसंहरति भाष्ये \textendash\ [ ११ शप० ] तस्मादिति~। 

यतः कांतपयशब्दज्ञाने न सर्वक्ञानादिसिद्धिस्तत इत्यर्थः~॥ 

[प्रदीपै \textendash\ ननु कथमर्थवादेनानभ्युपायत्वमत आह कैयटे 

१ {\qt न चान्तं इति क. पाठः~॥} २ जगामेति? इति इतिशब्द \textendash\ 

सहितश्छातागृहीतः पाठ इदानीं नोपलभ्यते~॥ 

श्ुतिमुदाहरति भाष्ये \textendash\ 

उह्च्योतपरिवृतग्रदीपप्रकाशितमहाभाष्यम्~। 

( प्रदीपः) बृहस्पतिरिन्द्रायेति प्रतिपदपाठस्याश \textendash\ 

क्यत्वं प्रतिपादयितुमयमर्थवादः~॥ शब्दानामिति~। {\qt शब्द \textendash\ 

पारायण}शब्दो योगरूढः शास्त्रविशेषस्य, तत्र {\qt प्रतिपदोक्तानां इति 

विशेषणाभिधानाय गम्यमानार्थस्याप्नि शब्दानां}इत्यस्य प्रयोगः~॥ 

एकदेशोपयोगादपि लोके {\qt उपयुक्तं} इत्युच्यते~। यथा \textendash\ औषध \textendash\ 

संस्कृतवृतमात्रैकदेशोपयोगे {\qt उपयुक्तं वृतं} इति व्यवहारः, 

तथेह न \textendash\ इति प्रतिपादयति \textendash\ चतुर्भिरिति~। आगमकालः \textendash\ 

ग्रहणकालः~। स्वाध्यायकालः \textendash\ अभ्यासकालः~। प्रवचन \textendash\ 

काण्ठः \textendash\ अध्याप्नकालः व्यवहारो याज्ञे कर्मणि~॥ 

( उद्दयोतः) (भाष्ये) \textendash\ अनभ्युपाय इति~। तत्त्वं चाशक्यत्वेन; 

कृत्ज्लाभिमतानभ्युपायत्वात्~। तत्रार्थवादमाह \textendash\ भाष्ये \textendash\ बृहस्पति \textendash\ 

रिति~॥ननु {\qt शब्दानां इति पुनरुक्तमत आह} \textendash\ योगरूढ इति~॥ 

अर्थपौनरुक्त्यमाशङ्क्य निराकरोति \textendash\ तत्रेति~॥ ननु या्देवाध्येष्यते 

प्रतिपदेति~। अशक्यत्वद्वारेति भावः~॥ प्रतिपादयितुं \textendash\ प्रति \textendash\ 

पादयतः~। 

[प्रदीपे ] शब्दानां पारमन्तमयते गच्छतिं येनेसि व्युत्पत्ति \textendash\ 

बोध्या~। रूढिस्तु प्रतिपदोक्तशब्दप्रतिपादकस्य~॥ तदाह \textendash\ कैयटे 

शास्त्रेति~॥ 

[प्रदीपे \textendash\ कैयटटे [५ मप० ] अपिना समुदायपरेग्रहः~॥ 

[६ ष्टप०] घृतमाञ्रैकेति~। ओऔषधिसंबन्धिसारांशादिना संस्कृता 

घृतस्य या मात्रा परिमितिस्तदेकेत्याद्यथः~॥ [ ७ मप० ] तथे \textendash\ 

हेति~। शास्त्र इत्यर्थः~॥ 

[प्रदीपे) \textendash\ कैयटे \textendash\ [ ८ मप० ] ग्रहणेतिं~। अध्ययनेत्यर्थः~॥ 

स्याधीनाक्षरावाप्तीतियावत्~॥ अभ्यासेति~। गुणनेत्यर्थः~॥ 

मननैनैति यावत्~॥ 

[प्रदीपे] \textendash\ तथाच व्यवहरशब्दस्य कत्वर्धयुरुषार्थकर्मादौ 

यथाहंमुपकार इत्यर्थः~॥ तदेतदभिप्रेत्याहृ \textendash\ कैयटे [ ९ मप० ] 

व्ययहारो याज्ञे कर्मणीति~॥ 

[उद्दयोते] अत एत्राग्रे \textendash\ अनभ्युपतय इत्युक्तं भगवता~॥ 

[उद्द्योते \textendash\ ननु लक्षणेन प्रकृत्यादिविभागेन तद्धेतुसंशादिधी \textendash\ 

पूर्वकमेवत्व शानाह्गौरवम् , प्रतिपदपाठे तु तत एव तत्तच्छब्दोच्चार \textendash\ 

णेनैवायासैन तत्प्रतिपत्तेलीघवमिति कथं तस्यानभ्युपायत्वं भगवतोक्त \textendash\ 

मत आह \textendash\ [ उ० २ यप० ] तत्त्वं चेति~। अशक्येति~। 

आनन्त्यादिति भावः~॥ 

[उद्दयोेते ] तन्नेति~। अनभ्युपावत्वे इत्यर्थः~॥ 

[उद्द्धोते\ \textendash\ [ ३ यप० ] पुनरुक्तमिति~॥ शब्दानामित्यनेन 

शब्दपारायणमित्य्रत्यशब्दपदं पुनरुक्तमित्पर्थः~॥ शब्दपारायण \textendash\ 

मित्यत्रत्यशब्दशब्दस्याफलत्वमिति यावत्~॥ 

[ उद्दयोते] \textendash\ भर्थपौनरुक्त्यमिति~। योगरूढयैव शब्दानां 

विषयत्वावगमादिति भावः~॥ अत एव गभ्यमानेति कैयटेनो \textendash\ 

क्तम्~॥ [४ र्थप०] आशङ्खयति~। तदाशङ्कां ह्वदि निधायेत्यर्थः~॥ 

तन्रेतीति~। श्रुतावित्यर्थः~॥ 

[उद्द्योते) \textendash\ [ ४ र्थप० ] यावदेवेति~। यावता कालेनेति 

३ {\qt चागमकालेनैव कृत्स्मायुः} इति क. पाठः~। 

४ {\qt आयुः कृत्स्यं पर्युपः इति मुद्रितपाठः~।} 

शास्त्रनिर्माणरीतिनिरूपेणाधिकरणम् ] महाभाष्यप्रदीपोद्दयोतव्याख्या
छाया~। 



त्तावदेवाभ्युपयुक्तं स्यादत आह \textendash\ भाष्ये \textendash\ चतुर्भिरिति~॥ 

उक्तमेव तदाशयमाह \textendash\ एकदेशेत्यादि~। विद्योपयोगाधारभूतानां 

कालानां विद्यां प्रत्यप्याधारत्वात् तस्य च करणत्वविवक्षा 

बोध्या~। चतुर्षु कालेषु विद्योपयुज्यते \textendash\ इति फलितोऽर्थः~। तत्राद्ययोः 

{\qt विद्यार्थ्ययं वुद्धिपान्} इत्यादरपूर्वकमन्नवस्त्रादिलाभरूप उपयोगः~। 

तृतीये \textendash\ प्रतिष्ठ, सचध्छिष्पलाभद्वाराऽर्थप्राप्तिः, सत्कारविेषश्च~। 

चतु्र्थे \textendash\ यद्याद्यनुष्ठानकालेऽपशब्दप्रथोगप्रयुक्तप्रायम्चित्ताभावः,
कर्म \textendash\ 

साङ्गता, दक्षिणालाभः, प्रतिष्ठा चेत्युपयोगः~। तादृशश्चोपयोगः 

सर्वाध्ययन एत्र~॥ भाष्ये \textendash\ पर्युपयुक्तमिति~। समाप्तमित्यर्थः~। 

{\qt पर्यवसन्नं}इति पाठान्तरम्~। न चेष्टापत्तिः, विद्योच्छेदापत्तेः~। 

कैर्मसु प्राथश्चित्तविधियैयथ्थापत्तेश्च~॥ 

(आक्षेपभाष्यं ) 

कथं तर्हिने शब्दाः प्रतिपत्तव्याः ? 

(समाधानभाष्यम् ) 

किंचित्सामान्यविशेषवल्लक्षणं प्रवर्त्यम्~। येना \textendash\ 

भावः~॥ \textendash\ युक्त सादिति~। तावानेवाभ्युपयोगो भविष्यतीत्यर्थः~॥ 

तथाच किं समाप्येति भावः~॥ स्वेन विरोधं परिहरति [ द ष्टप० ] 

उक्तमेवेति~॥ तदाशयं \textendash\ तथा भाष्योक्तेहेतुभूतं सदृष्टान्तमाशयम्~॥ 

[उद्द्योते] \textendash\ नन्वेवमि कथं विद्यां प्रति कालस्य प्रकारत्वम् , 

तस्योपयोगाधारत्वेऽपि विद्याऽनाधारत्वादत आह \textendash\ उद्दयोते [ ६ 

ष्ठप० ] विद्योषेति~। प्याधारत्वादिति~। तां प्रति प्रकारता \textendash\ 

श्रयणम् \textendash\ इति शेषः~॥ क्वचित्तधा पाठ एव~॥ नन्वेवं तृतीयानुपप \textendash\ 

त्तिरत आह \textendash\ [ ७ मप० ] तस्य चेति~। आधारस्य चेत्यर्थः~॥ 

तत्फलितमाह \textendash\ [ ८ मप० ] चतुर्ष्विति~॥ 

[उद्दयोते] \textendash\ प्रवचनस्य व्यवहारत्वेऽपि अर्थानुष्ठानवम्नोभयार्थ \textendash\ 

त्त्वम् , किंतु पुरुषार्धत्वमेवेति पृथगुक्तिः~॥ {\qt अत्राद्ययो} \textendash\ 

नत्वम् , अन्त्ययोः फलार्धत्वेनेति विवेकः~॥ कालग्रहणं कालस्या \textendash\ 

न्यत्वोक्तिप्रस्तावात्~॥ तदेतदभिप्रेत्वाह \textendash\ [ उ० ८ मप० ] तत्रा \textendash\ 

द्ययोरिति~। चतुर्णा मध्ये, आगमस्वाध्यायकालयोरित्यर्थः~॥ 

पूर्वकमिति~। एरैर्दानात्तथेति इति शेषः~॥ यद्वा तत्पूर्वकत्वं 

दानद्वारेति यथाश्रुतमेवारवु उपयोग इति भावः~॥ एवमग्रेऽपि~॥ 

[ १० मप० ] विशेषश्चेति~। सुकृतविशेषोऽपि वोध्यः~॥ {\qt विद्याया 

उपयोगः} इत्यस्यानुषङ्गः~॥ 

[उद्द्योते) \textendash\ {\qt आगमकाले इति सप्तम्यन्तम्~।} तेषां मध्ये 

आगमकालादारभ्यासमाप्तिपर्यन्तं अस्य नैव आयुः पर्थुपयुक्तं 

प्याप्तं स्यादित्य्ः इति कश्चिव्~। पाठान्तरानुरोधेन समाप्तमित्ये \textendash\ 

वार्थ शति ध्वनयन्नाह \textendash\ [उ० १४ शप०] पर्यवेति~॥ विद्यो \textendash\ 

स्छेदेति~। फलाभावादिति भावः~॥ अनुष्ठानाभावाशह \textendash\ 

कर्भस्विति~॥ 

[भाष्ये] \textendash\ पुनः स एव पृच्छति \textendash\ भाष्ये \textendash\ कथं सर्हाति~॥ 

गौरश्च इत्यादयः स्वरूपेणापि ये थे दु्शानास्ते कथं लक्षणेनानुगन्तुं 

शक्या श्त्यथः~॥ 

कर्मविधियैयर्भ्वापतेश्च इति ज, पाठः~। 

५५ 





ल्पेन यल्लेन महतो महतः शब्दौघान् प्रतिपद्येरन्~॥ 

किं पुनस्तत् ? 

उत्सर्गापवादौ~। कश्चिदुत्सर्गः कर्तव्यः, कश्चिद \textendash\ 

पचवादः~॥ 

( प्रदीपः ) किंचिदिति~। सामान्यविशेषौ यस्मिस्तत् \textendash\ सा \textendash\ 

मान्यविशेषवत्~। {\qt कर्मण्यण, आतोऽनुपसर्गे कः}इत्यादि~॥ 

( उद्दयोतः ) ननु तदात्मकत्वाल्लक्षणस्य मतोरनुपपत्तिरत 

आह \textendash\ सामान्यविशेषाबिति~। लक्षणं \textendash\ शास्त्रमित्य्थः~॥ 

(आक्षेपभाष्यम् ) 

कथंजातीयकः पुनरुत्सर्गः कर्तव्यः, कथंजाती \textendash\ 

यकोऽपवादः ? 

(उद्दयोतः ) भाष्ये \textendash\ कथंजातीयक इति~। केन प्रकारेणे \textendash\ 

त्यर्थः~। एतेनैवंजातीयकेन शास्त्नेण सुशब्दोपदेशप्रवृतिः पाणिने \textendash\ 

रुच्चितेति सूरचितम्~॥ 

[ भाष्ये ] \textendash\ उत्तरमाह सिद्धान्ती \textendash\ किंचिदिति ~। लघुभूत 

मित्यर्थः~॥ 

[भाष्ये] \textendash\ ननु तत्प्रवर्तनेऽपि किं स्यादत आह भाष्ये \textendash\ येनेति~। 

व्यल्पेन \textendash\ लघुभूतेन~॥ यत्नेन \textendash\ उपायेन~। यत्रसाध्यत्वाल्लक्षणं \textendash\ शास्त्रं 

यत्रशब्देनोच्यते येन शास्त्रेण~। महतो महत इति~। वीप्सार्था 

द्विरुक्तिः, द्वितीयावहुवचनान्तावेतौ शब्दसमूहविशेषणे~॥ यद्वा \textendash\ 

आद्यः पञ्चम्यन्तः, विशालादपि विशालाञ्शव्दसमूहानित्यर्थः~॥ 

नन्तेवसस्यतौ सिजङोः सामान्यविशेपलक्षणाभ्यामनिष्टरूपद्वया \textendash\ 

पत्तिः , {\qt इरितो वा} इलि वापदानर्थक्यं चेत्याशयेन पृच्छति \textendash\ किं 

पुनस्तदिति~। शास्त्रीयसामान्यविशेषयोः स्वरूपं किं पुनरित्यर्थः~॥ 

तौ कीदृशाविति यावत्~॥ यथा विशेषेण परामर्शस्तथा प्रतिपाद \textendash\ 

यितव्यम्~॥ 

उत्तरमाह \textendash\ उत्सर्गेति~। तथाचापवादविषयपरिहारेण कौण्डिन्ये 

दधितक्रयोरिव नास्यतौ रूपद्वयम् , नापि तद्वैयर्थ्यमिति भावः~॥ 

तत्फलितमाह \textendash\ कश्चिदिति~। योग इत्यर्थः~॥ शास्त्रे इति भावः~॥ 

कर्तव्य इत्यस्याग्रेऽनुषङ्गः~॥ 

[ प्रदीपे ] \textendash\ सामान्थविशेषशब्दौ सूत्रपराविति तयोः क्रमेण 

स्वरूपे भाष्योक्ते एवाह कैयटे \textendash\ कर्मेति~। भरुदिना {\qt कर्तरि शप्

रुधादिभ्यः श्म्} इत्यादिपरिग्रहः~॥ 

[उद्दयोतेः] \textendash\ तदात्मकत्वादिति~। सामान्यविशेषस्वरूपत्वा \textendash\ 

दित्यर्थः~॥ नन्वेवं लक्षणमित्यनुपपल्मत iSiii 

अष्टाध्यायीरूपमित्यर्थः~॥ 

[भाष्ये] तयोरुदाहरणे प्रतिपादयितुं विशेषेण पृच्छति \textendash\ 

कथमिति~॥ प्रकारमात्रे थमुः, तद्वति जातीयर इति \textendash\ अत्र जातीयरैव 

निर्वाहेऽपि सामान्योपकमाच्छिशपा वृक्ष इतिवत्प्रयोगः~॥ 

[उद्दयोते] \textendash\ उक्ताथांनामित्यस्यानित्यत्व सूत्रयितुमेवं प्रयोग इति 

कश्चित्~। तदेतदभिभ्रेत्याह \textendash\ उद्दयोते \textendash\ केनेति~॥ शास्त्रेण \textendash\ अनेन~॥ 







२ {\qt येन येनाल्वेन} इति फ. पाठः~। 

५६ 

उद्दयोतपरिवृतग्रदीपप्रकाशितमहाभाष्ये~। 

[१ अ. १ पा. १ पस्पशाह्निके 

(समाधानभाष्यम् ) 

सामान्येनोत्सर्गः कर्तव्यः~। तद्यथा \textendash\ {\qt कर्म णयण्~।}

तस्य विशेषेणापवादः~। तद्यथा \textendash\ {\qt आतोऽनुप \textendash\ 

सर्गे कः}~॥ 

( जातिव्यक्तिपदार्थनिर्णयाधिकरणम् ) 

( आक्षेपभाष्यम् ) 

किं पुनराकृतिः पदार्थः, आहोस्विद् द्रव्यम्? 

(समाधानभाष्यम् ) 

उभयमित्याह~॥ 

कथं ज्ञायते ? 

उभयथा ह्याचार्येण सूत्राणि पठितानि~। आकृतिं 

पदार्थ मत्वा \textendash\ {\qt जात्याख्यायामेकसिमिन्वहुवचनम \textendash\ 

न्यतरस्याम्} इत्युच्यते~। 



[भाष्ये] \textendash\ उत्तरमाह \textendash\ माध्ये \textendash\ सामेति ~॥ कचित् {\qt सा 

मान्ये} इति पाटः~॥ तत्र विपथसप्तमीति कृष्णः~। तृतीयार्थ 

सप्तमीत्युचितम्~। करणस्याधिकरणत्वरिवक्षया मेति यावत्~॥ 

[भाष्ये) \textendash\ आकृतिः पदार्थ इति~। उत्मर्गापवादलक्षणगतानां 

पदानामिति भावः~॥ 

[भाष्ये] \textendash\ तदाह सिद्धान्ती भाष्ये \textendash\ उभयमिति~। इत्या \textendash\ 

हेति पूर्ववत्~॥ तथाच कचिदुत्सर्गापवादलक्षणयोस्तत्तत्पदानां 

जातिरथः कचित्तु व्यक्तिरिति लक्ष्यानुरोधेनोभयाश्रथेण शास्त्रं 

प्रवृत्तिमिति भावः~॥ 

उक्ताशयानभिज्ञो लक्षगकचक्षुष्कः शङ्क \textendash\ कथमिति 

सूत्रकृता तथा शास्तं प्रणीतमिति केन प्रकारेण निश्चीयत इत्यर्थः~॥ 

तत्र प्रमाणं किमिति यावत्~॥ 

एकदेशी तदनुरोधेनैवोत्तरमाह \textendash\ उभयथेति~॥ हि \textendash\ यतः~॥ 

अन्यथानुपपत्त्या लब्धजातिपक्षसाधकव्यक्तिपक्षसाधकस्वरूप्रकारद्वये \textendash\ 

नेत्यर्थः~॥ सूत्राणीति~। सूत्रे पटिते इत्व्थः~॥ आर्प बदहुवचनम्~॥ 

[ भाष्ये ] \textendash\ तदेवाह \textendash\ भाष्ये \textendash\ आकृतिमिति~। यदि व्यक्तिरे \textendash\ 

वेत्येतत्सार्वत्रिकमिष्टं स्यात् तदा संपन्ना व्रीहय इत्यत्र
व्यक्तिबद्दु \textendash\ 

त्वाद्बहुवचनं सिद्धमेवेति {\qt जात्याख्यायाम्} इति नारभेत, तदा \textendash\ 

रम्भात्तु जातेस्तत्त्वमिति निश्चय इति भावः~॥ 

[ भाष्ये \textendash\ इत्येकशेष इति~। एकस्य शेषो येनेति बहु \textendash\ 

ब्रीहिः~। सरूपाणामिति रूप एकशेपविधायको योग इत्यर्थः~॥ 

यदि जातिरेवेति सार्वत्रिकमिष्टं तर्हि सरूपसूत्रं नारभेत, यत 

आकृतेरेकत्वादेवैकशब्दप्रयोगः, तेनैव च जात्यवच्छिन्नसकलव्य \textendash\ 

क्त्युपस्थितिसंभव इति नानेकव्यक्त्यभिधानायानेकशब्दप्रसङ्गः~। 

तदारम्भात्तु व्यक्तेरपि तत्त्वमिति निश्चय इति भावः~॥ 

[ प्रदीवे ] \textendash\ कैयटे \textendash\ [१मप०] श्रयण इति~। सप्तम्यन्तम्~॥ 

पक्षद्वयेति~। लक्ष्पानुरोधेन व्यवस्थिवत्यादिः~। प्रश्नति~। 



द्रव्यं पदार्थं मत्वा \textendash\ {\qt सरूपाणाम् \textendash\ } इत्येकशेष 

आरभ्यते~॥ 

( प्रदीपः ) सकलशास्त्रव्यवस्था \textendash\ एकतरपक्षाश्रयणे न सिध्य \textendash\ 

तीति पद्वाश्रयणं प्रश्नपुर्वकं करोति \textendash\ किं पुनरिति~। आ \textendash\ 

कृतिपक्षे केवल आध्रीयमाणे {\qt सकृद्गतौ विप्रतिषेधे \textendash\ } इत्यादि 

नोपपद्यते, केवलेऽपि व्यक्तिपक्षे {\qt पुनःप्रसङ्गविज्ञानात्}इत्यादि 

न घटेत~। तम्माह्वक्ष्यसिद्धये क्वचित्प्रदेशे कश्चित्पक्षः परिगृह्यते~। 


तत्र जातिवादिन आहुः \textendash\ जातिरेव शब्देन प्रतिपाद्यते, 

व्यक्तीनामानन्त्यात्संवन्धग्रहणासंभवात्~। सा च जातिः सर्व \textendash\ 

व्यक्तिष्येकाकारप्रत्ययदर्शनादस्तीत्यवसीयते~। तत्र गवादयः 

शब्दा भिन्नद्रव्यसमवेतां जातिमभिदधति~। तस्यां प्रतीतायां 

तदावेशात्तदवच्छिन्नं द्रव्यं प्रतीयते~। शुक़ादयः शब्दा गुणस \textendash\ 

मवेतां जातिमाचक्षते~। गुणे तु तत्संषन्धात्प्रत्ययः, द्रव्ये 

सम्बन्धिसंबन्धात~। संज्ञाशब्दानामप्युत्पत्तिप्रभृत्याविनाशात् 

पिण्डस्य कौम मार्यीवनायवस्थाभेदेऽपि स एवायमिल्यभिन्नप्रत्य \textendash\ 



अन्यदोषेत्यादिः~॥ प्रसङ्गदिति भावः~॥ करोतीति~। सिद्धन्ति 

त्यादिः~॥ अर्थविशेपान्तर्भावेणैव साधुत्वस्य वक्ष्यमाणत्वात्तद \textendash\ 

थमिति भावः~॥ 

[ प्रदीपे ]सकलशास्त्रव्यवस्थामाह कैयटे [ ३ यप० ] केव \textendash\ 

लेति~। अनेन व्यक्तिपक्षव्यावृत्तिः~॥ एवमभग्रेऽपि~॥ आदिना 

{\qt यद्वाधितं तद्वाधितमेव} इत्यस्य परिग्रहः~॥ [ ४ थप० ] केव \textendash\ 

लेऽपीति~। केवले व्यक्तिपक्षेऽपीतल्यथः~॥ आदिना {\qt सिद्धम्} 

इत्यस्य परिग्रहः~॥ 

[ प्रदीपे ]उत्तरार्थं वदम्नुपसंहरति \textendash\ कैयटे \textendash\ [५मप०] त \textendash\ 

स्मादिति~। वुद्धिस्थाद्धेतोरित्यर्थः~। सकलशास्त्रव्यवस्थाया अन्य \textendash\ 

तरपक्षाश्रयणेनासिद्धत्वादिति यावत्~॥ शापकपरवक्ष्यमाणभाष्य \textendash\ 

स्यायुक्तत्वं ध्वनयन्नाह \textendash\ लक्ष्येति~॥ लक्ष्यानुरोध एव तदाश्र \textendash\ 

यणे शरणमिति भावः~॥ f 

[ प्रदीपे ] प्रसङ्गादनयोः पक्षयोरुपपत्ती क्रमेणाह \textendash\ कैयटे \textendash\ 

[ ५ मप० ] तत्रेत्यादिना~॥ तत्र \textendash\ तयोः पक्षयोर्मध्ये~॥ पाद्यते 

इति~। लाघवादिति भावः~॥ हेत्वन्तरमाह \textendash\ व्यक्तीनामिति~। 

व्यभिचाराच्चत्यपि बोध्यम्~॥ तत्सत्त्वे मानमाह \textendash\ साचेति~। प्रत्ययो 

ज्ञानम्~॥ सामान्येनोक्तं स्पष्टार्थ विशिष्य चतुर्विधशब्देषु क्रमेणोप \textendash\ 


पादयति \textendash\ [८मप०] तन्नेति~। चतुर्णा शब्दानां मध्य इत्यर्थः~॥ 

[ प्रदीपे ] केयदे [१२शप०] संबन्धादिति~। प्रत्यय इत्यस्या \textendash\ 

नुपङ्गः~। परम्परयेति भावः~॥ संज्ञास्थले पिण्डस्यैकत्वेन जात्य \textendash\ 

संभवात्तत्र कथं निर्वाह इति मध्ये एव विमताशकङ्कायां व्युत्क्रमे \textendash\ 

णादौ तत्रैवह \textendash\ १२ शप० ] संज्ञेति~॥ अत एवाह \textendash\ 

\ldots~॥ यद्वा \textendash\ कमप्रतिपादने तात्पर्यादेव \textendash\ 

मुक्तिः~। यद्वा \textendash\ त्रयीपक्षे संशाशब्दानां जातिशब्दे५न्तर्भावेन तत्र 

जातेः प्रसिद्धत्वात्क्रियामु तस्याः प्रतिपादनीयत्वादेवमुक्तिः~॥ 

जातिव्यक्तिपदार्थनिरूपणाधिकरणम् ] महाभाष्यप्रदीपोद्द्योतव्याख्या छाया
~। 

५७ 

यनिमित्ता डित्थत्वादिका जातिर्वाच्या~। क्रियास्वपि जातिर्वि \textendash\ 

द्यते, सैव धातुवाच्या~। पठति \textendash\ पठतः \textendash\ पठन्ति \textendash\ इत्यादेरभिन्नस्य 

प्रत्ययस्य सद्भावात्तजिमित्तजात्यभ्युपगमः~॥ व्यक्तिवादिन \textendash\ 

स्त्वाहुः \textendash\ शब्दस्य व्यक्तिरेवाभिधेया, जातेस्तूपलक्षणभावेना \textendash\ 

श्रयणादानन्त्यादिदोषानवकाशः~॥ 

(उद्दयोतः )आकृतिपक्षे इति~। तत्र उद्देश्यतावच्छेदकजा \textendash\ 

त्याक्रान्ते क्वचिच्चरितार्थयोर्द्वयोः सत्प्रतिपक्षन्यायेन विरोधस्थले 

{\qt उभयोरप्यग्राप्तौ विप्रतिषेधे परम} इति विध्यर्थम्, न तु नियमार्थ \textendash\ 

मिति सकृद्गतौ शति न सिथति व्यक्तौ तु तत्तमृतिविषययोरि 

शास्त्रयोरचारितार्थ्यात्पर्ययेणोभयप्राप्तौ नियमार्थ तदिति {\qt पुनः प्रस \textendash\ 

ङ्गेन} इत्यादि न सिद्धतीत्यर्थः~। इदं च विप्रतिषेधसूत्रे दूषयिष्यते~॥


तस्यां प्रतीतायामिति~। शब्दशक्यत्वेन गृहीतायामिल्यर्थः~। एवं 

{\qt अभिदधति} इत्यस्याभिधाविषयत्वेन गृह्णन्तीत्यर्थः~॥ तदावेशात् \textendash\ 

तत्समैवायात्~॥ प्रतीयत इति~। अविनाभावाच्छब्दजन्यबोधविषयो 



अस्य जातिर्वाच्येत्यत्रान्वयः~। स्पिः पिण्डस्येत्यत्र योज्यः~। 

एकस्यापि पिण्डस्य तस्माद्भेदे सतीत्यर्थः~। प्रत्ययेति बहुब्रीहिः~॥ 

तत्र तत्सत्त्वे मानमाह \textendash\ पठतीति~। घ्रत इत्यादिः~॥ [ १७ \textendash\ 

शप० ] व्यक्तिरेवेति~। आनुभविकत्वादनुपपत्तिप्रतिसंधानं विना 

प्रतीयमानत्वाद्वाहाद्यन्वयस्य तत्रैव संभवाद्भमकं विना ततस्याः 

प्राधान्येनाप्रतीयमानत्वच्च नानन्त्यादिदोषोऽत आह \textendash\ जाते \textendash\ 

स्स्विति~। अनुगताया इति भावः~॥ 

[ प्रदीषे ] [कै० १८ शप० ] आनन्त्यादीति~। आदिना 

गौरवब्यभिचारपरिग्रहः~॥ 

[ उद्द्योते ]अनुपपत्तिप्रकारं क्रमेणाह \textendash\ उद्दयौते \textendash\ सतश्रेत्या \textendash\ 

दिना~॥~। तत्र \textendash\ जातिपक्षे~॥ तत्र कार्यस्याबाधादाह \textendash\ उदेश्येति~। 

अपिर्मिथः समुच्चायकः~॥ यद्वा विप्रतिषेधे इति नियमसमुच्चयः~॥ 

अप्राप्तिविध्यर्थत्वे निमित्तम्~॥ स्पष्टार्थमाह \textendash\ [ ३ यप० ] नत्वि \textendash\ 


ति~। तुरुक्तवैलक्षण्ये~॥ [ ४ र्थप० ] विषययोरपीति~। सप्त \textendash\ 

स्यन्तम्~। तयोरपि सतोरित्यर्थः~॥ [ ५ मप० ] शास्त्रयोः \textendash\ विरो \textendash\ 

{\qt धाधारलक्ष्यविषययोः~।} यद्वा हेतुगर्भविशेषणम्~। सकलव्यक्ति \textendash\ 

विषययोरपि शास्त्रयोरित्यर्थः~॥ निरपिपाठस्तु सुगम एव~॥ 

[ ५ मप० ] तदिति~। {\qt विप्रतिषेधे \textendash\ } इति सूत्रमित्यर्थः~॥
प्रसङ्गेत्या \textendash\ 

दीति~। आादिना {\qt विज्ञानात्सिद्धम्} इत्यस्य परिग्रहः~॥ {\qt आकृ \textendash\ 

तीत्यादि \textendash\ घटेत} इत्यन्तकैयटासंगतिमाह \textendash\ इदं चेति~॥ 

[ उद्द्योते ] विशिष्टङ्ञानजनकोपस्थितिरूपनिर्विकल्पस्मरणायाह \textendash\ 

[उ० ७ मप० ] शब्दशक्येति~॥ सिंहावलोकनन्यायेनैतदेकवा \textendash\ 

क्यतायै आह \textendash\ एवमिति~। एवमग्रेऽपि~॥ तद्वैशिष्ट्यादित्यर्थेन 

पौनरुक्त्यवारणायाह \textendash\ [ ८ मप० ] तत्समेति~॥ अत एव {\qt तत्सं} \textendash\ 



१ {\qt व्यक्तिरेन वाच्या} शति मुद्रितपाठः~। 

२ {\qt पुनः प्रसङ्गेत्यादि} इति मुद्रितपाठः~। 

३ जातिपक्षे व्यक्तिप्रतीतिमुपपादयति \textendash\ तस्समेति~॥ 

४ कोष्टकान्तर्गतो५यं ग्नन्थो दाधिमथैः {\qt इदञ्च विप्रतिषेधसूत्रे} 

५ प्र०





भवतीत्यर्थः~॥ प्रत्ययः \textendash\ ज्ञानम्~॥ उपलक्षणभावेनेति~। शक्यताव \textendash\ 

च्छेदकत्वेनेत्यर्थः~। तत्र तु न शक्तिरिति मञ्जूषायां प्रतिपादितम्~॥ 

[भाष्ये \textendash\ जात्याख्यायामिति~। व्यक्तौ पदार्थं {\qt संपन्ना ब्रीहयः} 

इति व्यक्तिबहुत्वाद्वहुवचनं सिद्धमेवेति व्यर्थ तत्स्यात् \textendash\ इति भावः~॥ 

सरूपाणामिति~। आकृतेरेकत्वादेवैकशब्दप्रयोगोपपत्तौ तवुक्तिसंभवो 

व्यक्तिपक्ष एवेति भावः~। न चाकृतिपक्षेऽपि {\qt अक्षाः पादाः} इत्या \textendash\ 

दर्थ स आवश्यक इति वाच्यम्~। तावतीषु कांचिदेकां जातिं स्वीकृत्यं 

तस्या एव शब्दवाच्यत्वम् , परम्परयाऽत्र व्यक्तिप्रत्यय इति सरूप \textendash\ 

सूत्रे कैयटेनैव वक्ष्यमाणत्वात्~। जातौ न जातिरिति परेषां दर्शनं, 

नास्माकमिति च तत्रोक्तम्~॥] 

(शब्दनित्यत्वानित्यत्वविचारः ) 

( सन्देहदर्शकभाष्यम्) 

र्किं पुनर्नित्यः शब्दः, आहोस्वित्कार्यः ? 

( प्रदीपः) किं पुनरिति~। विप्रतिपत्त्या संशयः~। केचित् 



{\qt बन्धात्} इत्यग्रे वक्ष्यति कैयटः~॥ लक्षणादिवारणायाह \textendash\ [ ९ 

भप० ] अविनेति~। व्याप्तेरित्यर्थः~॥ 

[ उद्दयोते ] शब्दोपात्तस्याप्युपलक्षणत्वस्य काकवदित्यादौ सत्वा \textendash\ 

त्तन्मतग्रवेशसत्त्वादाह [ उ० १० मप० ] शक्येति~॥ नन्वेतावतैव 

तस्या वाच्यत्वापत्तिरिति स दोष एवात आह \textendash\ तत्र स्विति~। 

शक्यतावच्छेदके त्वित्यर्थः~॥ अकार्यत्वेऽपि कार्यतावच्छेदकत्वव \textendash\ 

दशक्यत्वेऽपि शक्यतावच्छेदकत्वसंभवादिति भावः~॥ 

[ उद्दयोते ) \textendash\ तदाह \textendash\ उद्दयोते [१२ शप०] व्यक्ताविति~॥ 

इतीति~। इत्यत्रेत्यर्थः~॥ 

[ उद्दयोते ] तदाह \textendash\ [उ० १४ शप०] आकृतेरिति~॥ नारा \textendash\ 

यणकौस्तुभाद्यमुरोधेन शङ्कते [ १५ शप० ] नचेति~॥ स इति~। 

सरूपेतियोग इत्यर्थः~॥ तावतीु \textendash\ जातिषु~॥ वाच्यत्वमिति~। 

{\qt स्वीक्रियतामिति शेषः~॥} नन्वेवं तदाश्रयजातिबोधेऽपि व्यक्तिबोधो 

न स्यादत आह \textendash\ [ १७ शप० ] परम्परेति~॥ ननु जातौ न 

जातिरनवस्थापत्तेरत आह \textendash\ [ १८ शप० ] जाताविति~॥ परेषां 

नैयायिकानाभ्~॥ तत्रोक्तम् \textendash\ वैयटेनैव~॥ अग्रे अग्रे तदनङ्गीकारेण 

नानवस्येति भावः~॥ अत्र {\qt कैयटेनैवः इत्यनेन} सूचितारुचिबीजं तु 

तत्रेणैव तत्र निर्वाह इति~। वस्तुतस्तु लक्ष्यानुरोध एवात्र शरणम्~। 

अत एवानयोः प्रत्याख्यानसंगतिः~। अन्यथा भाष्ययोर्विरोधः 

स्पष्ट एव~। एवं च सूत्ररीत्येदं भाष्यम्~॥ सिद्धान्तरीत्या न भाष्यं~। 

किंच यत्र वने वृक्षान्तरसत्त्वेऽपि पनसव्यक्तिरेकैव तत्र लक्षणया 

जातिपरत्वे पनसाः सन्तीति प्रयोगसिद्धर्थ जात्याख्याया \textendash\  इति 

सुत्रमावश्यकम्~। अत एव वक्ष्यति सरूपसूत्रे वातिककृत् \textendash\ 

*ब्यर्थेषु च मुक्तसंशयम् इति, इति बोध्यम्~॥ 

{\qt निरूपयिष्यते} इत्युत्तरं निवेशित आसीत्~। आदर्शपुस्तकेषु तादृश \textendash\ 

पाठानुपलम्भात् समुपक्रान्तप्रदीपव्याख्यानपरिसमाप्तावेव भाष्यव्या \textendash\ 

ख्यानस्य श्रीमन्नागोजीभद्वशैलीसिद्धत्वात् टिप्पणस्वरूपच्छायानुरोेधे \textendash\ 

नाखण्डप्रवाहस्वरूपग्रन्थविपरिवर्तनस्याक्षमत्वाच्च तन्नाद्रियते~॥ 

५८ 

उद्दयोतपरिवृतप्रदीपप्रकाशितमहाभाष्यै \textendash\ 

[ १ अ. १ पा. १ पस्पशाह्रिके 



ध्वनिव्यङ्ग्यं वर्णात्मकं नित्यं शब्दमाहुः~। अन्ये \textendash\ वर्णव्यति \textendash\ 

रिक्तं पदस्फोटमिच्छन्ति~। वाक्यस्फोटम्परे संगिरन्ते~। अन्ये 

तु \textendash\ ध्वनिरेव शब्दः स च कार्यः, तद्यतिरेकेणान्यस्यानुपलम्भात् \textendash\ 

इत्याचक्षते~॥ 

(उद्दयोतः ) विप्रतिपत्त्येति~। शब्दानां नित्यत्वे व्यर्थ 

व्याकरणशास्त्रमिति प्रश्नाशयः~॥ केचित् \textendash\ मीमांसकाः~॥ वर्णा \textendash\ 

त्मकमिति~। पदवाक्ये च वर्णसमूहरूपे एवेति तदाशयः~। अन्र 

{\qt मते एकं पटपदं}इत्याद्यनुपपत्तेराह \textendash\ अन्ये \textendash\ वैयाकरणाः~॥ अपरे \textendash\ 

त एव मुख्याः~। पदे वर्णानामिव वाक्ये पदानां कल्पितत्वात् 

तेपामर्थवत्त्वमपि काल्पनिकमितिं वाक्यस्यैव शब्दत्वमिति तद्भावः~॥ 

अन्ये त्विति~। वैशेषिकादयः~॥ ध्वनिरिति~। स च वर्णावर्णा \textendash\ 

त्मकत्वेन द्विविधः~। एवं च कण्ठादिवच्छास्त्रमपि तस्य कारण \textendash\ 

मित्याशयः~॥ 

( पक्षनिर्द्धारकभाष्यम् ) 

संग्रह एतत्प्राधान्येन परीक्षितम् \textendash\ नित्यो वा 

स्यात्कार्यो वेति~। तत्रोक्ता दोषाः, प्रयोजनान्यप्यु \textendash\ 

. \textendash\ 

[ प्रदीपे ] संशयवीजमाह \textendash\ कैयटे विप्रेति~। विरुद्धकोटिद्वयो \textendash\ 

पस्थित्येत्यर्थः~॥ 

[प्रदीपे] विप्रतिपत्तिमेवाह \textendash\ कैयटे केचिदित्यादिना~॥ वर्णा \textendash\ 

त्म्कमिति~। पदवाक्ययोस्तदतिरिक्तत्वं नेत्यर्थः~॥ 

[प्रदीपे ] कैयटे पदस्फोटमिति~। नित्यं शब्दमित्यस्यानुषङ्गः~॥ 

एवमग्रेऽपि~॥ 

[उद्दयोते] प्रश्नबीजयोस्तात्पर्यमाह उद्दयोते \textendash\ शब्देति~॥ शय 

इति~। इति भाव इति शेषैः~॥ 

[उद्दयोते] तदाह \textendash\ उद्दयोते \textendash\ पदेति~॥ अन्ये इति~। अस्य 

{\qt इति} इति शेषः~॥ रणा इत्यस्य {\qt इत्यर्थः} इति शेषः~॥ एवमग्रेऽपि~॥ 

[उद्दयोते ] त एव \textendash\ वैयाकरणा एव~॥ तेषां \textendash\ पदानाम्~॥ 

वाक्यस्यैवेति~। असत्यपदपदार्थयिभागकस्येत्यादिः~॥ न्यूनतां 

परिहरति \textendash\ सचेति~॥ एवञ्च \textendash\ जन्यस्य तस्योक्तरीत्या द्वैविध्ये 

च्~॥ कारणमिति~। तथाचात्र पक्षे तदावश्यकत्वमिति भावः~॥ 

[ भाष्ये ] निष्फलोऽयं विचार इत्याशयेन सिद्धान्ती उत्तरमाह \textendash\ 

भाष्ये \textendash\ संग्रह इति~। कैयटोक्तस्तदर्थः~॥ 

[ भाष्ये ] तत्र बहुधा परीक्षा कृता, अत्र तु शब्दानुशासनप्र \textendash\ 

स्तावात्तदुपजीवनमित्याशयेनाह भाष्ये \textendash\ प्राधान्येनेति~॥ एतत्प \textendash\ 

दार्थमाह \textendash\ नित्यो वेति~॥ प्राधान्येनेत्यस्यार्थमाह \textendash\ तत्रोक्ता 

इति~। उभयत्रेत्यर्थः~॥ वर्णानित्यत्वे व्यापकस्य विकाराभावात् इकः 

१ शब्द इति~। स एव वर्णं इति तार्किकाः~। 

२ इत आरभ्य छायायाः सारो मुद्रणीयः, अतः परं समग्राया 

अनुपलम्भात्~॥ 

३ अयं प्रमादः~। आशयपदस्य भावार्थकत्वात्~। (र. ना. ) 

वस्तुतस्तु {\qt विप्रतिपत्त्या संशयः} इति प्रदीपव्याख्यानावसरे शब्दानां 

नित्यत्वे व्यर्थ व्याकरणशास्त्रमिति प्रश्नाशयः इत्युक्तमुद्दयोते~। 
तदेत \textendash\ 





क्तानि~। तत्र त्वेष निर्णयः \textendash\ यद्येव नित्यः, अथापि 

कार्यः, उभयथाऽपि लक्षणं प्रवर्त्यमिति~॥ 

( प्रदीपः ) संग्रह इति~। ग्रन्थविशेषे~॥ 

( उद्दयोतः ) संग्रहः \textendash\ व्याडिकृतो लक्षश्लोकसंख्यो ग्रन्थ इति 

प्रसिद्धिः~॥ भाष्ये \textendash\ उभयथाऽपीति~। एवं च निष्फलोऽयं विचार 

इति भावः~॥ साधुत्वज्ञानायोभयथाऽपि शास्त्रमावश्यकमिति 

तात्पर्यम्~॥ 

(नित्यशब्दवादेऽपि शास्त्रस्य धर्मजनकताधिकरणम् ) 

( वार्तिकावतरणभाष्यम् ) 

कथं पुनरिदं भगवतः पाणिनेराचार्यस्य लक्षणं 

प्रवृत्तम्? 

( प्रदीपः ) कथं पुनरिति~। किमाचार्य एव श्रष्टः शब्दा \textendash\ 

र्थसंबन्धानाम् , अथ स्मर्ता \textendash\ इति प्रश्नः~॥ 

( उद्दयोतः ) अथ पाणिनिसूत्रव्याख्यानभूतं वार्तिकमवतार \textendash\ 

यति \textendash\ कथं पुनरिति~। अत एवात्रैव पाणिनिनामग्रहणम्~। 

इतः पूर्वं तु सवातिकशास्त्रान्वाख्यानप्रयोजनविषयप्रदशनपरो 

भाष्यकारस्यैव अन्थ इति बोध्यम्~॥ शब्दार्थसंबन्धानामिति~। 



स्थाने यण् न स्यात्~। कार्यत्वे तच्छब्दानन्तर्यरूपसंहिताया अभावात् 

स न स्यादित्यादयो दोषा वोध्याः~॥ प्रयोजनेति~। अनवयवस्य 

स्फोटस्य व्यञ्जकध्वनिधर्मेणान्वाख्यानम्~। {\qt इको यणचिः} इत्या \textendash\ 

दिना साधुत्वमात्रं क्रियत इत्यादिप्रयोजनानीत्यर्थः~॥ ननूभयत्र 

कारणोक्तौ तयोरन्यतरानिश्चये शास्त्रं कर्तव्यमेवेति दं निश्चयः ? 

किंचैवं सति कथमस्य बिचारस्य निष्फलत्वमत आह \textendash\ तत्र त्विति~॥ 

तत्रैवेत्यर्ः~॥ तत्रापीति पाटेऽप्येवमेव~॥ यद्येव \textendash\ यद्यपि~॥ उ \textendash\ 

यथेति~। पक्षद्वयेऽपि दोषाणां सुनिरसत्वादिति भावः~॥ 

[उद्दयोते] तस्यार्थमाह उद्दयोते \textendash\ व्याछीति~॥ संख्य इति~। 

लक्षश्लोकपरिमित इत्यर्थः~॥ 

[उद्दयोते] एवं च \textendash\ तत्रोभयथा शास्त्रारम्भस्यावश्यकत्वोक्तौ 

च~॥ ननु तदावश्यकत्वरूपनिश्चयोक्तिरेवादौ कथमत अह \textendash\ 

साधुत्वेति~॥ तात्पर्यमिति~। व्याडेरिति भावः~॥ 

{\qt अत्र भगवतः} इत्युक्त्या पूज्यत्वप्रतिपादनद्वारा वार्तिककृद \textendash\ 

पेक्षया सूत्रकारे श्रद्धातिशयः सूचितः~। अत एव सूत्रसमथनपूर्वकं 

वार्तिकखण्डनं भगवान् करिष्यति~॥ 

[उद्दयोते] अथेति~। सूत्रकृदाद्यनुक्तविषयाद्यौषोद्धातिकनिरूपणा \textendash\ 

नन्तरमित्यर्थः~॥ पाणिनिसूत्रसमूहव्याकरणशास्त्रस्य नियमार्थताप्रति \textendash\ 

पादकमाद्यवार्तिकमित्यर्थः~॥ r 

दसङ्गतम् , यतः प्रदीपे प्रश्न एव नास्तीति कुतः प्रश्लाशयः ? तत्सा \textendash\ 

ज्त्यायाह छायायां \textendash\ इति भाव.इति शेष इति~। एवञ्च व्याकरण \textendash\ 

शास्त्रप्रारम्भावसरे शब्दस्य नित्यत्वानित्यत्यमाविष्कुर्वताम्भाष्यकृतां


शब्दानां नित्यत्वे व्यर्थं व्याकरणमिति प्रश्नस्याशयः, इतिभावस्तु 

{\qt विप्रतिपत्त्या संशयः} इति लेखकस्य प्रदीपकारस्येति नात्र कश्चित्प्रमाद


इति {\qt अयं प्रमादः} इति लेख एव प्रामादिकः~॥ 

शास्त्रस्य धर्मजनकसाधिकरणम् ] 



शब्दाश्चार्थाश्च संबन्धाश्चति द्वन्द्वः~। एवं च किमपूर्वशब्देष्पादन \textendash\ 

द्वाराऽर्थेविशेषसंबन्धनिष्पादकत्वं शास्त्रस्य, किं वा सिद्धशब्दैबन्ध \textendash\ 

बोधकत्वमिति प्रश्नार्थ इति तात्पर्यम्~॥ 

(१ आक्षेपसाधकवातिंकप्रथमखण्डम्~॥ १~॥ ) 

~॥~॥ सिद्धे शव्दार्थसंबन्धे~॥~॥ 

(व्याख्याभाष्यम् ) 

सिद्धेः शब्दऽर्थं संबन्धे चेति~॥ 

( प्रदीपः ) सिद्ध इति~। तत्र नित्यः श्ब्दोः जातिस्फो \textendash\ 

टलक्षणो व्यक्तिस्फोटलक्षणो बा~। कार्यशद्दिकानार्मपि प्रवाह \textendash\ 

नित्यतया~। अर्थस्यापि जातिलक्षणस्य नित्यत्वम्~। द्रैव्यपक्षेऽपि 

सर्वशब्दानामसत्योपाध्यवच्छिन्नं ब्रह्मतत्त्वं वाच्यमिति \textendash\ नित्यता, 

प्रवाहनित्यतया वा~। संबन्धस्यापि व्यवहारपरम्परयाऽनादित्वा \textendash\ 

न्नित्यता~॥ 

( उद्दयोतः ) {\qt सिद्धे शब्दार्थसंबन्धे} इति वार्तिके समा \textendash\ 

हारद्वन्द्वात्सप्तमी~। संबन्धश्च प्रत्यासत्त्या शब्दार्थयोरेव~। 
त्रयाणामा \textendash\ 

त्यन्तिकाविक्षप्रददीनाय समाहारनिर्देश इति बोध्यम्~॥ व्यक्ति \textendash\ 

स्फोटेलि~। सीं पदवाक्यरूपा नित्यैति तद्भावः~। एतच्चोपपादितं 

भ्राक्~॥ कार्यशादिदकानामिति~। {\qt कार्यशाब्दिकानाम् इति त्वप} \textendash\ 



ननु शब्दार्थसंबन्धस्रष्टत्वं कथमाचार्यस्य ? किं च ब्रह्माण्डस्य 

घ्रयानन्थत्येन पाणिनेस्तदन्तर्गतस्य तत्स्रषटृत्वं कथम् ? अतः वैयट \textendash\ 

तात्पर्यमाह \textendash\ एवं चेति~॥ 

यद्यपि द्विवीयपक्षाश्रयेणोत्तरत्वेन प्रवृत्तं नवपदात्मकमेकं वार्तिकम् 

{\qt सिद्धे शब्दार्थसंबन्धे लोकतोऽर्थप्रयुक्ते शब्दप्रयोगे शास्त्रेण 

धर्मनियमो यथा लौकिकवेदिकेषु} इति~। क्रियत इति त्वध्या \textendash\ 

१ केचित्तु \textendash\ {\qt अथ शब्दानुशासनम्}? रक्षोहागमलध्व \textendash\ 

संदेहाः प्रयोजनम् इत्यादीमि व्याख्येयवक्यानि वार्तिककारस्यैव, 

व्याख्यानवाक्यानि भाष्यकारस्य सन्ति~। अत एव सायणाचार्येण 

कऋक्संहिताभाष्योपोद्धाते \textendash\ तस्यैतस्थ व्याकरणस्य प्रयोजन \textendash\ 

विशेषो वररुचिना वार्तिके दर्शितः \textendash\ {\qt रक्षोहालमलध्वसंदेहाः 

प्रयोजनम्} इति~। एदानि रक्षादिप्रयोजनानि प्रयोजनान्त \textendash\ 

राणि च महाभाष्ये पतञ्जलिना स्पष्टीकृतानि इति स्पष्टमेव 

वर्णितम् \textendash\ इति वदन्ति~॥ तन्न सम्यक्, {\qt सिद्धे शब्दार्थसंबन्धे 

इति प्रकृतवार्तिकस्थपिद्धपदस्य मङ्गलार्थत्वनर्णनप्रस्तावे माङ्गलिक 

आचार्या महतः शास्तौघस्य मङ्गलार्थं सिद्धशब्दमादितः 

प्रगुङ्के इति भाष्यग्रन्थे सिद्धशब्दमादितः} इति लेखस्य 

तथा सत्यसामञ्जस्यात्~। {\qt किंत् अथ शब्दाजुशासनम्} इति 

वाक्यस्यैव प्रथमोच्चारितत्वेन तत्स्थस्याथशब्दग्यैव मङ्गलार्थत्वं
व्याख्या \textendash\ 

येत~। तस्मात्सर्वोऽपि प्राक्तनग्रन्धो भाष्यकारस्यैव~। न चैकस्मिश्नेव 

मूलब्याख्योभयाजीकारे पौनरुक्त्यापत्तिरिति वाच्यम्~। यतः स्वपद \textendash\ 

व्याख्यानत्वमेव भाष्यलक्षणम्~। अत एव \textendash\ 

सू्त्रार्थो वर्ण्यते यत्र वाक्यैः सूत्राजुसारिभिः~। 

स्वपदानि च वर्ण्यन्ते भाष्यं भाष्यविदो विदुः~॥ 

इत्यभियुक्तोक्तं संगच्छते~॥ सायणभाष्ये तु केनचिदूर्वनधीत \textendash\ 



महाभाष्यप्रदीपोद्व्योतव्याख्या छाया~। 

पाठः , उत्तरपदबृद्धेलभत्वात्~॥ असत्योषाधीति~। जातिरूपेत्यर्थः~। 

जातिहि तन्मते आविद्यको धर्यविशेषः~। यद्वा [ लैवेन धर्माणा \textendash\ 

मेवाविद्याकार्यत्वेन ] सर्वेषां धर्माणां ब्रह्मण्यध्यस्तत्वेन
शुक्तिरजतादि \textendash\ 

ज्ञाने शुक्त्यादीनां विशेष्यत्ववत् सर्वत्राधिष्ठानभूतं ब्रह्मैव
विशेष्यम्, 

विशेष्यमेव न्च द्रव्यमिति भावः~। 

[यैतु \textendash\ व्यक्तीनामाविद्यकत्त्रेन तदवच्छिन्नं ब्रह्मैव जातिरिति, तन्न~। 


अनेकव्यक्तीनामाविद्यकानां कल्पने गौरवात्~। अपागादग्नेरग्नि \textendash\ 

त्वभ् इत्यादि श्रुत्या जातेरेवानित्यत्ववोधनाच्च~। अत एव जातेर्व्यव \textendash\ 

हारनित्यताम् \textendash\ आकृतेः प्रवाहनित्यतामेवाग्र वक्ष्यति भाष्यकृत्~। ; 

शब्दार्धयोसित्यत्वपक्षे तत्संवन्धस्य नित्यत्यैऽप्यनित्यत्वपक्षे कथं 

संबन्धस्य नित्यत्वमत आह \textendash\ संबन्धस्येति~। तयोरिव तस्यापि 

नित्यत्वमिति भावः~॥ शव्दार्थयोः संवन्धश्च शक्तिरूपं तादात्थमेवे \textendash\ 

त्यन्यत्र प्रपश्चितम्~॥ 

(वार्तिकघटकसिद्धशब्दार्थनिरूपणभाष्यम् ) 

(आक्षेपभाष्यम् ) 

अथ सिद्धशब्दस्य कः पंदार्थः ? 

( प्रदीपः ) सिद्धशब्दस्य निल्यानित्ययोर्दर्शनात्पृच्छति \textendash\ 

अथेति~॥ 

हारः~। शब्दार्थसंवन्धे नित्ये सति शब्दप्रयोगे न लोकतोऽर्थप्रयुक्ते \textendash\ 

अर्थबोधनाय प्रसक्ते सति शास्त्रेण धर्मनियमः क्रियते, यथा लौकिक \textendash\ 

वैदिकेष्यिति तदर्थः~। तथापि {\qt लोकतः? इत्यस्य मध्यमणिन्यायेनो \textendash\ 

भयञान्वयेन खण्डशो वार्तिकं यथेष्टं व्याचक्षाणो द्वितीयपक्षाश्रयेणो \textendash\ 

त्तरत्यं प्रतिपादयन्ना्खखण्डमाह \textendash\ सिद्धे इति~॥ 

व्याकरणभाष्येण वैदिकविदुपा अयोजनविशेषो वररुचिना वार्तिके 

दर्शितः} इत्येतावान्ग्रन्थो वृथैव ग्रन्थमध्यै प्रक्षिप्त इति प्रतीयते 

 \textendash\ व्याकरणस्य {\qt रक्षोहागमलघ्तसंदेहाः प्रयोजनम्} इत्ये \textendash\ 

तानि \textendash\ {\qt  इत्येवं पाठेऽपि ग्रन्थसमन्वयाद्} \textendash\ इति बोध्यम्~॥ इति 

दाधिमथाः~॥ 

२ मपि मते प्रवा \textendash\ इति क. पाठः~॥ 

३ प्रवाहनित्यतयेति~। कार्यशब्दीकानामपि मते प्रवाहनित्य \textendash\ 

तया नित्यत्वमुपपद्यत इति भावः~॥ 

४ व्यक्तिवादेऽपि अर्थस्य नित्यत्वमुपपादयति \textendash\ द्रव्यपक्षे \textendash\ 

ऽपीति~॥ 

५ सा \textendash\ पदव्यक्तिवीक्यव्यक्तिश्च~॥ 

६ {\qt तन्मते} इत्यस्य ग. पुस्तके न पाठः~॥ 

७ [ ] अत्रत्यः पाठो ग. पुस्तके एव~॥ 

८ {\qt प्रवाहनित्यतया तयोरिव} इति घ. च. पाठः~। तयोरिव \textendash\ 

शब्दार्थयोरिव~। तस्यापि \textendash\ सम्वन्धस्यापि~। यथा प्रवाहनित्यतया 

ब्रह्मतत्त्वं वाच्यमिति प्रकारेण च शब्दा्थयोर्नित्यत्वमुपपाद्यते तथा 

सम्बन्धस्यापीत्यर्थः~॥ 

९ पदार्थ इति~। अर्थ इत्यर्थः~। नित्यः कार्या वेति भावः~॥ 

छाया~॥ 



६० 

उद्दयोतपरिवृतप्रदीपप्रकाशितमहाभाष्यम्~। 

[१ अ. १ पा. १ पस्पशाहिके 



(उद्दयोतः ) भाष्ये \textendash\ सिद्धशब्दस्य कः पदार्थ इति~। 

ऊत्र पदार्थशब्दो रूढः~। सिद्धशब्दसंबन्धी कः पदार्थः \textendash\ इत्यर्थः~॥ 

(समाधानभाष्यम् ) 

नित्यपर्यायवाची सिद्धशब्दः~॥ 

कथं ज्ञायते? 

यत्कूटस्थेष्वविचालिषु भावेषु वर्तते~। तद्यथा \textendash\ 

सिद्धा द्यौः,सिद्धा पृथिवी, सिद्धमाकाशमिति~॥ 

( प्रदीपः ) नित्येति~। नित्यलक्षणस्यार्थस्य पैर्यायेण वाच \textendash\ 

कः, तमेवार्थं कदाचिन्नित्यशब्द आह \textendash\ कदाचित्सिद्धशब्द 

इत्यर्थः~। कूटस्थेष्विति~। अविनाशिषु~॥ अविश्वालि \textendash\ 

ष्विति~। देशान्तरप्राप्तिरहितेषु~॥ 

( उद्दयोतः ) कूटस्थेष्विति~। कूटंन्अयोधनस्तद्वत्तिष्ठन्ति ये 

तेषु, संसरगिनाशेऽपि स्वयमनष्टेष्वित्यथः~॥ नन्वयोधनस्यापि तहिं 

नित्यत्वं स्यादत आह \textendash\ अविचालिष्विति~। भाष्ये द्यावापृथिव्याद्यपि 

व्यावहारिकनित्यत्वाभिप्रायेण दृष्टान्तितम्~। आकाशस्यापि व्याव \textendash\ 

हारिकनित्यत्वमेवाचार्याभिमतम्~। एवञ्च तत्र रूढत्वान्नित्यवाचकस्यैव 

ग्रहणमिति भावः~॥ 

( आक्षेपभाष्यम् ) 

ननु च भोः कार्येष्वपि वर्तते~। तद्यथा \textendash\ सिद्ध 

ओदनः, सिद्धः सूपः, सिद्धा यवागूरिति~। यावता 

कार्येष्वपि वर्तते, तत्र कुत एतत् \textendash\ नित्यपर्यायवा \textendash\ 

चिनो ग्रहणम्, न पुनः कार्ये यः सिद्धशब्द इति~। 

( प्रदीपः ) ननु चेति~। सिद्धशब्दात्क्रियानिष्पन्नोऽप्यर्थो \textendash\ 

ऽवगम्यत इत्यर्थः~॥ 



१ अत्रार्थशब्दोऽभिधेयपरो न तु प्रयोज्नपर बोधयितुं 

पदपदमिति गुरवः~। (र.ना.) वस्तुतस्तु पदशब्दाभावे {\qt सिद्धशब्दस्य 

कोऽर्थः} इत्युक्तावपि उत्तरग्रन्थानुरोधेन व्याख्यानात्प्राक्
प्रयोजनप्रश्न \textendash\ 

स्यासंभवेन च {\qt अर्थंशब्दस्य प्रयोजनवाचित्वं} न संभवतीत्यत आहो \textendash\ 

द्वयोते \textendash\ अत्र पदार्थेति~। एवं च पदशब्दस्य नान्यत्प्रयोजनमिति 

भावः~॥ 

२ पर्यायेण \textendash\ अवयवार्थानपेक्षरूढ्या~। सा च योगबाधिका, 

रथकाराधिकरणादौ ( पू. मी. ६~। १~। १२ ) स्पष्टा~॥ विशेष्यविशेषण \textendash\ 

भायन्यत्यथमात्रेणारथिकोत्तरत्वस्य सत्त्वान्न काचिदनुपषत्तिरिति तत्त्वम्
~॥ 

छाया~॥ 

३ अत्र भाष्ये {\qt अनित्येषु} इति वाच्ये {\qt कार्येषु} इत्युक्तिः कार्यत्व \textendash\ 

प्रयुक्तानित्वत्वसूचनाय~। अत एव \textendash\ कैयटे क्रियानिष्पन्न इत्युक्तम्~॥ 

छाया~॥ 

४ अत्र निर्णायकमा८ \textendash\ संग्रह इति~॥ छाया~॥ 

५ ननु यदि संग्रहसमानत्रत्वादत्रापि सिद्धशब्दस्य नित्य एवार्थः 

स्वीक्रियेत तदा संग्रहाश्रितव्यक्त्येकपदार्थवादोऽपि स्वीकर्तव्यः 

स्यात्, स च नेष्टोऽतः पक्षान्तरमाह \textendash\ अथवेति~॥ 





(समाधानभाष्यम् ) 

संग्रैहे तावत्कार्यप्रतिद्वन्द्विभावान्मन्यामहे नि \textendash\ 

पर्यायवाचिनो ग्रहणमिति~। 

इहापि तदेव~॥

( प्रदीपः ) संग्रहे तावदिति~। तत्र हि {\qt किं कार्यः 

शब्दोऽथ सिद्धः} इति पक्षद्वयविचारः कृतः~। तत्र कार्यप्रति \textendash\ 

पक्षार्थाभिधायी सामर्थ्यात्सिद्धशब्द इति स्थितम्~। तत्समान \textendash\ 

तन्त्रत्वादिहापि तथैव युक्तमित्यर्थः~॥ 

(समाधानान्तरभाष्यम् ) 

अथवा सन्त्येर्कपदान्यप्यवधारणानि~। तद्यथा \textendash\ 

अब्भक्षो वाूम ति, अप एव भक्षयति \textendash\ वायु \textendash\ 

मेव भक्षयतीति गम्यते 

एवमिहापि \textendash\ सिद्ध न साध्य इति~॥ 

(प्रदीपः) अथवेति~। एवशब्दप्रयोगे द्विपदमवधारणम् , 

द्योतकत्वेनैवशब्दस्यापेक्षणात्~। यदा तु द्योतकमन्तरेण सामर्थ्यादि \textendash\ 

वधारणं गम्यते तदा तत्एकपदमित्युच्यते~॥ तत्र {\qt सर्व एवापो 

भक्षयन्ति} इत्यब्भक्षश्रुतिः सामर्थ्यान्नियममबरामयति \textendash\ अपँ 

एवेति~॥ ईहापि निल्यानित्यव्यतिरेकेण राश्यन्तराभावात्सिद्ध \textendash\ 

शब्दोपादानान्नियमोऽवगम्यते \textendash\ सिद्ध एवेति~। कीर्याणां तु 

पदार्थानां प्राक्प्रध्वंसावस्थयोः सिद्धता नास्तीति न ते सिद्धा एव~॥ 

(उद्दयोतः) ग्रन्थान्तरानालोचनेऽपि नियामकमाह \textendash\ भाष्ये \textendash\ 

अथवेति~। एतेन तत्समानतत्रत्वाद्यक्त्येकपदार्थवादोऽपि तव 

स्यादित्यपास्तम्~॥ ननु {\qt एकपदानि} इत्युक्त्वा {\qt अब्भक्षः?} इत्यस्य 

पदद्वयात्मकस्योदाहरणत्वमसङ्गतमत आह \textendash\ एवशब्देति~॥ निय \textendash\ 

मप्रत्यायकसामर्थ्य दर्शयति \textendash\ इहापीति~॥ कार्यवाचिन व 

ग्रहणाभावे मूलमाह \textendash\ कार्याणामिति~। प्राक्प्रध्वंसावस्थयोरिति 



 

६ एकपदानीति बहुब्रीहिः~॥ छाया~॥ 

७ {\qt अप यएव भक्षयन्तीतिः} इति मुद्रितेषु पाठः~॥ 

८ अनब्मक्षदृष्टान्तमुपपादयतिः \textendash\ इहापि नित्येति~। सिद्ध \textendash\ 

शब्दस्य नित्यरूपोऽर्थः प्रसिद्ध एव, {\qt सिद्ध ओदनः} इति व्यवहारा \textendash\ 

वुत्पन्नरूपोऽप्यर्थः प्रतीयते~। तृतीयस्यः तु
प्रकारस्यासम्भवात्सिद्धशब्दो \textendash\ 

पादाने सिद्ध एवेति निर्द्धारणमब्भक्षन्यायेन भविष्यति~। एवञ्च सिद्ध 

एवेत्यवधारणस्य {\qt नित्य एव \textendash\ उत्पन्न एव} इति प्रकारद्वयमेव सम्भ \textendash\ 

वति~। तत्र नित्य एवेति प्रकारो५निर्वाच्य इष्ट एव~। उत्पन्न एवेत्य \textendash\ 

स्यानुत्पन्ने विनष्टे वा पदार्थेऽसम्भवात्स ग्रहीतुं न शक्य इति प्रथमं 

एव प्रकारोऽत्रानन्यगत्या ग्राह्य इतीष्टमुपयन्नम्~॥ 

५ नन्वेवमपि सिद्धशब्दः कार्यवाची किं न स्यादित्याशङ्कय निय \textendash\ 

मेन त्रैकाल्यसिद्धताद्योत्यत इतिं परिहरति \textendash\ कार्याणामिति~॥ 

१० प्राक्प्रध्वंसाभावयोरिति काचित्कोऽपपाठः, अनन्वया \textendash\ 

पत्तेरत आह \textendash\ प्राक्प्रध्वंसेति~। आकाशादीनां कारणात्मना 

सिद्धत्वमिति भावः~॥ छाया~॥ 

शास्त्रस्य धर्मजनकताधिकरणम् ] पस्पशाह्निकम्~। ६१ 



पाठः~। भाष्ये अप एवै {\qt इति गम्यते} इति वाक्ये इव

समासेऽपि सामथर्यात् तादतिरिभक्षणाभावो गम्यतः इत्यर्थः 

(समाधानान्तरभाष्यम् ) 

अथवा पूर्वपदलोपोऽत्र द्रष्टव्यः \textendash\ अत्यन्तसिद्धः \textendash\ 

सिद्ध इति~। तद्यथा \textendash\ देवदत्तः \textendash\ दत्तः, सत्यभामा \textendash\ 

भामेति~॥ 

( प्रदीपः ) अथवेति~। कैथं पुनर्देवदत्तशब्दे संज्ञात्वेन 

विनियुक्ते एकदेशः प्रयुज्यते, न ह्यसौ संज्ञात्वेन विनियुक्तः~। 

न चैकदेशात्स्मर्यमाणस्य समुदायस्य वाचकत्वमुपपद्यते, प्रतीय \textendash\ 

मानस्य प्रत्यायकत्वासंभूवादुच्चार्यमाणस्यैव वाचकत्वात्~। एवं 

तर्ह्यनुनिष्पादिन्योऽवयवसरूपाः संज्ञा विनियोगकाले विनियुक्ता 

एव~। लोपस्तु वर्णानां साधुत्वं मा भूदित्यन्वाख्यायते~। इहापि 

निल्यानित्ययोर्निष्पन्नत्वाविशेषात्सिद्धश्रुतिरुपात्ता प्रकर्षं गमयति \textendash\ 


अत्यन्तसिद्ध इति~॥ 

(उद्दयोतः ) अथवेति~। मनुष्यवबृक्षाद्यपेक्षयाऽऽकाशादीना \textendash\ 

मप्यत्यन्तसिद्धत्वमिति भावः~॥ प्रतीयमानस्येति~। एतदणुदित्सूत्र 

उपपादयिष्यते~॥ एवं तर्हीति~। वमेव धटपदं कलशपर्यायः? 

इत्यादिप्रयोगाः~। अन्यथा {\qt पर्यायवाचि} इति प्रयोज्यं स्यात्~॥ 

विर्तापि प्रत्ययं पूर्वोत्तरपदयोर्लोप इति लोपविधायकं तहि किमर्थ \textendash\ 

मत आह \textendash\ लोपस्त्विति~। वर्णानाम् \textendash\ {\qt देइत्यादीनाम्~।} तत्रापि 

कचित्पूर्वपदस्यैव \textendash\ क्वचिदुत्तरपदस्यैव \textendash\ कृचिदुभयोरिति लोकब्यवहारेण 

निर्णयैम्~। एवं वर्णानां शक्त्यभावोऽपि तत एव निर्णैय इति तत्त्वम्~॥ 

निष्पन्नत्वेति~। विद्यमानकालसंबन्धः \textendash\ निष्पन्नत्वम्~॥ 



१ मध्यम इतिराद्यर्थकः~। अप एवेत्यादि गम्यत इत्यन्तमितीत्यर्थः~॥ 

वाक्ये$=$तिङन्तघटितसजातीयापरलौकिकवाक्य इत्यर्थः~॥ छाया~॥ 

२ आकाशादीनामपि चिरकालावस्थायित्वेन नित्यसिद्धशब्दाभ्यां 

व्यवहारसंभवान्न वास्तवं सिद्धत्वमतो दृष्टान्तासिद्धिरत आह \textendash\ अथ \textendash\ 

वेति~॥ छाया~॥ 

३ {\qt सिद्ध एव} इत्युक्ते परिनिष्पन्न एव, नानित्यः \textendash\ इत्यवधारण \textendash\ 

सम्भवास् एकपदावधारणपक्षेऽप्यभिमतासिद्धिमाशङ्क्योक्तं \textendash\ अथ \textendash\ 

वेति \textendash\ भाष्ये~। तद्बृष्टान्तमाक्षिपति \textendash\ कथमिति~। संशात्वेन देव \textendash\ 

दत्तादिशब्दानां विनियोगे दत्तादिशब्दैः कथन्तदर्थस्य प्रतीतिसम्भव 

इत्याक्षिप्य प्रतीयमानस्य प्रत्यायकत्वाभावमुपपाद्य च्च यदा देवदत्त इति


संज्ञा क्रियते तदा नान्तरीयकत्वादवयवेष्वपि संज्ञासम्प्रत्यय इति प्रति \textendash\ 


पादयति \textendash\ एवन्तर्द्यनुनिष्पादिन्य इति~॥ 

४ अयं भावः \textendash\ समुदायैकदेशः धुत एकसंबन्धिज्ञानमिति 

न्यायेन संस्कारपाटवादेव शब्दस्मारकस्तत एव बुद्धयुपारूढात्संघाता \textendash\ 

देवार्थप्रतीतिरिति~॥ छाया~॥ 

५ न केवलं संज्ञास्वेवायं व्यवहारः किन्तु अन्यत्रापीति प्रदर्श \textendash\ 

यति \textendash\ एवमेवेति~॥ 

६ (५~। ३~। ८२ ) सुत्रे अप्रत्यये तथैवेष्टः इति वातिका्थभूत \textendash\ 

मिदं {\qt समासेऽनन्पूर्वे क्त्वो ल्यप्?} इति सूत्रेण विहितस्य ल्यबा \textendash\ 

देशस्य समासमन्तरानुपपन्नस्य समासशापकपूर्वपदाप्रयोगेवऽपि स्वकीये 



(समाधानान्तरभाष्यम्)

अथवा {\qt व्याख्यान्तो विशेप्रतिपत्तिर्न हि संदे 

 हादलक्षणम्} इति नित्यपर्यायवाचिनो ग्रहणमिति 

व्याख्यास्यामः~॥ 

( प्रदीपः ) न्यीयाद्वा नित्यत्वं शब्दादीनां स्थितमित्याह \textendash\ 

अथवेति~। न हि संदेहमात्रादलक्षणता भवति, पुनः प्रमाणा \textendash\ 

न्तरेण निश्चयोत्पादात्~॥ 

( उद्दयोतः ) ननु निर्युक्तिकं व्याख्यानमयुक्तं, विपरीतस्यापि 

संभवादत आह \textendash\ न्यायाद्वेति~। वृद्धव्यवहारादेव पदार्थसंबन्धामां 

नित्यत्वं संग्रहादौ स्थितमिति व्याख्यानतः सिद्धशब्देन तदेवोपात्त \textendash\ 

मित्यर्थः~॥ 

(आक्षेपभाष्यम् ) 

किं पुनरनेन वर्ण्येन~। किं न मर्हता कण्ठेन नि \textendash\ 

त्यशब्द एवोपात्तः, यस्मिन्नुपादीयमानेऽस्ंदेहः 

स्यात्?

( प्रदीपः ) वर्ण्येनेति~। प्रयत्नव्याख्यातव्येनेत्यर्थः~॥ 

(समाधानभाष्यम् ) 

मङ्गलार्थभ्~॥ 

माङ्गलिक आचार्यो महत; शास्त्रौघस्य मङ्ग \textendash\ 

लार्थ सिद्धशब्दमादितः प्रयुङ्धे~। मङ्गलादीनि हि 

शास्त्राणि प्रथन्ते वीरपुरुषाणि च भवन्ति, आयुष्म \textendash\ 

त्पुरुषाणि चाध्येतारश्च सिद्धार्थी यथा स्युरिति~॥ 

जाम्बवतीजयापराभिधाने पातालविजयकाव्ये {\qt संध्यावधूं 

गृह्य करेण भानुः} इत्यत्र ल्यबादेशप्रयोगं कुर्वतो भगवतः सूत्रका \textendash\ 

रस्यापि संमतमेव~॥ एवं च रुद्रटालक्लारव्याख्यायामस्य प्रयोग \textendash\ 

स्यापशब्दत्वं प्रकाशयन्नमिसाधुस्तु ब्याकरणमर्मानभिश एवेति 

बोध्यम्~॥ इति दाधिमथाः~॥ 

७ अस्यार्धस्य लोकसिद्धत्वेन कृष्णसर्पादावनतिप्रसङ्गः~। अत एव 

भीमसेनादौ सेन इत्यादेर्न तत्त्वम्~। अत एवानुभवविरुद्धत्वस्याव्यव \textendash\ 

हारस्य सत्त्वाच्चेति भावः~॥ छाया~॥ 

८ न च सिद्धशब्दस्य क्रियाशब्दत्वात् संशाशब्दविषयलोपा \textendash\ 

न्वाख्यानस्यात्राप्रवृत्तिरिति वाच्यम्~। जातिपक्षे सर्वशब्दानां स्वरूप \textendash\ 


प्रवृत्तिनिमित्तकत्वेन संशात्वाङ्गीकारादिति भावः~। अयं च पक्षो व्यक्ति \textendash\ 


पक्षेऽसंभवीत्यतः पक्षद्वयसाधारणं पक्षद्वयमुपन्यस्तवान् \textendash\ अथयेति~। 

तदनुपपन्नं, व्याख्यानेन संदेहनिवृत्तावतिप्रसङ्गात् \textendash\ इत्याशक्क्य प्रमाणेन


प्रतिपादनं व्याख्यानमित्याह \textendash\ न्यायादिति~। {\qt लोकतः} श्ति वार्तिको \textendash\ 

क्तन्यायात् शब्दादीनां निल्यत्वं स्थितमिति, अतो नित्यवाची सिद्धशब्द 

इति प्रमाणेनोपपादयिष्यामीत्याहेत्यर्थः~। (अ.पुस्तके सदाशिवभद्टाः)~॥ 

९ महता कण्ठेनेति~। दोषयुक्त्वाक्यमुपांशूच्यते निर्दुष्टं तुश्चै \textendash\ 

रिति लोकप्रसिद्धमिति निर्वुष्टताऽनेन सूचिता~॥ छाया~॥ 

१० क्वचित् {\qt मङ्गलार्थं च} इति पाठः~। तत्र चेन नित्यार्थधी \textendash\ 

समुच्चयः~॥ अर्थशब्दः प्रयोजनवाची~॥ छाया~॥ 

६२ 

उक्ष्योतपरिवृतप्रदीपप्रकाशितमहाभाष्यम्~। [ १ अ. १ पा. १ पस्पशाहिके 



( प्रदीपः ) माङ्गलिक इति~। अगर्हिताभीष्टार्थसिद्धिः \textendash\ म \textendash\ 

ङ्गलं, तत्प्रयोजन आचार्यौ माङ्गलिकः~॥ प्रथन्त इति~। अध्य \textendash\ 

यनयाविच्छेदात्~॥ वीरपुरुषाणीति~। श्रोतृणां परैरषराज \textendash\ 

यात्~॥~। आयुष्मत्पुरुषाणीति शास्त्रारार्थानुष्ठाने धर्मोपचया \textendash\ 

दायुर्वर्धनात्~। सिद्धार्था इति~। अध्ययननिर्वृत्तिद्वेव तेषां 

सिद्धिः~॥ 

( उद्दयोत ) सिद्धिर्मङ्गलमिति~। तस्या मङ्गलत्वं च कीर्ये 

कारणरूपोपचारादिति वोध्यम्~॥ तत्प्रयोजनमस्य \textendash\ इति प्राग्वती \textendash\ 

यष्ठक्~॥ तत्र मङ्गलानुष्ठानप्रयोजनान्याह \textendash\ भाष्ये \textendash\ मङ्गलादीनी \textendash\ 

त्यादिना~॥ गन्थे तन्निबन्धनस्य फलमाह \textendash\ अध्येतारश्चेत्यादिना~। 

आनुषङ्ञिकमञ्जलसंपत्त्येत्यर्थः~॥ अध्ययननिर्बृत्तिः \textendash\ तत्समाप्तिः~॥ 

सिद्धिः \textendash\ सिद्धिविषयोऽर्थः~॥ मङ्गलार्थमित्यनेनेत एव वार्तिकप्रवृत्तिः 

सूचिता~॥ 

(समाधानभाष्यशेषभाष्यम् ) 

अयं खलु नित्यशब्दो नावश्यं कूटस्थेष्वविचा \textendash\ 

लिषु भावेषु वर्तते~। 

किं भावेषु वर्तते 

किं तर्हि 

वि् वर्तते~। 

आभीक्षणयेऽपि वर्तते~। तद्याथा \textendash\ नित्यप्रहसितो 

 नित्यप्रजल्पित इति~। यावता आभीक्षण्येऽपि, तत्रा \textendash\ 

प्यनेनैवार्थः स्यात् \textendash\ {\qt व्याख्यानतोविशेषप्रतिप

त्तिर्न हि संदेहाद्लक्षणम्} इति~। पश्यति त्वा \textendash\ 

चार्यो मङ्गलीर्थश्वैव सिद्धशब्द आदितः प्रयुक्तो 

भविष्यति, शक्ष्यामि चैनं नित्यपर्यायवाचिनं वर्ण \textendash\ 

यितुमिति~। 

अतः सिद्धशब्द एवोपाक्तो न नित्यशब्दः~॥ 

( प्रदीपः ) नावश्यमिति~। ततश्चाभीर्ण्येन ये शब्दाः 

युज्यन्ते आगोपालाङ्गनं तेषामेवान्वाण्यानं स्यात्, न विरल \textendash\ 

प्रयोगाणाम्~। विनाऽपि च क्रियापदप्रयोगेणाभीक्ष्ण्यवृत्तिर्नित्य \textendash\ 



१ कार्ये कारणेति~॥ अन्नाशयेन तण्डुलमानयेति प्रयोगे पच्यते 

ओदन इत्यत्र च ओदने तण्डुलत्वोपच्चारवत् आयु्वै घ्रतमितिवच्च 

समाप्तौ नत्यादित्वोपचार इत्यर्थः~॥ समाप्तावगहितत्वं च विघ्नाभावा \textendash\ 

नुगृह्वीतत्वम्~। तथा च यागेनापूर्वद्वारा स्वर्ग इव नत्यादिना प्रतिवन्ध \textendash\ 


कदुरितनिवृत्तिद्वाराऽभीष्टसमाप्तिः साध्वेत्युक्तं भवति~। तथाच तत्र 

सिद्धशब्दप्रयोगस्य हेतुताऽस्तीति तदर्थ शास्त्रारम्भे तत्प्रयोग इति 

भावः~॥ छाया~॥ 

२ खलुरप्यर्थै~। अयं नित्यशब्दोऽपीत्यर्थः~॥ क्वचित \textendash\ स्वल्व \textendash\ 

पीति पाठः~। तत्र खलु \textendash\ निश्चये~॥ छाया~॥ 

३ यावतेति~। यत इत्यर्थः~॥ छाया~॥ 

{\qt क्षण्येऽपि वर्तते} इति क. पाठः~॥ 

५ तत्रेति~। अत इत्यादिः~॥ छाया~॥ 

६ स्वरूपेण तस्य मङ्गलार्धत्वमिति सूच्चयन्नाह \textendash\ मङ्गलार्थश्चै \textendash\ 

वेति~। सिद्धशब्द एकवदौ प्रयुक्तो मङ्गलफलकोऽपीत्यर्थः मङ्गलाति 



शब्दः प्रयुज्यते~। यथा \textendash\ {\qt आश्चर्यमनित्ये निल्यवीप्सयोः} इति~॥ 

( उद्दयोतः ) असंदिग्धो नित्यशब्दः किं न प्रयुक्त इत्युक्तम्, 

तत्रापि संदेहमाह \textendash\ भाष्ये \textendash\ अयं खल्विति~॥ आभीक्ष्ण्येन \textendash\ बाहु \textendash\ 

ल्येन~। {\qt अभीक्ष्णे} इति पाठेऽप्ययमेवार्थः~॥ प्रयुज्यन्त इति~। 

साक्षात् शब्दे आभीक्ष्ण्यासंभवात्प्रयोगद्वारा तत् ग्राह्यम्~। एवश्च 

नित्यशब्दोपादानेऽपि सप्रयोजनकोस्यन्तरसंभवेन तत्रापि संदेह 

एवेति भावः~॥ {\qt नित्ये शब्दार्थसंबन्धे} इत्यस्य शब्दार्थसंबन्धे ज्ञाते 

सति नित्ये \textendash\ नित्यप्रयोगविषये शब्दे शास्त्रं प्रवृत्तमित्यर्थः स्यादिति 

तात्पर्यम्~॥ नन्वाभीक्ष्ण्यवाचिनो नित्यं क्रियापदसाकाङ्कृतया कथं 

केचजस्य प्रयोगोऽत आह \textendash\ विनाऽपि चेति~॥ 

(नित्यतालाधकपक्षनिर्णयाधिकरणम् ) 

( आक्षेपभाष्यम्) 

अंथ कं पुनः पदार्थं मत्वा एष विग्रहः कियते \textendash\ 

सिद्धे \textendash\ शब्दे \textendash\ अर्थ \textendash\ संबन्धे चेति ? 

(समाधानभाष्यम् ) 

आकृतिमितयाह~॥ 

कुत एतत् ?

आकृतिर्हि नित्या द्रव्यमनित्यम्~॥ 

( आक्षेपभाष्यम् ) 

अथ द्रव्ये पदार्थे कथं विग्रहः कर्तव्यः ? 

(समाधानभाष्यम् ) 

सिद्धे \textendash\ शब्दे \textendash\ अर्थसंबन्धे चेति~। नित्यो ह्यर्थवता \textendash\ 

मर्थैरभिस्ंबन्धः~॥ f 

( प्रदीपः ) अर्थसंबन्धे चेति~। दव्यपक्षे द्रव्यस्यानित्य \textendash\ 

त्वादर्थग्रहनं सबन्धन्धविशेषणार्शमुपात्तम्~। अनित्येऽर्थे कथं 

संबन्धन्य नित्यता \textendash\ इति चेत्, योयतालक्षणत्यात्संबन्धस्य~। 

तस्याश्च कृव्दाश्रयत्वात् \textendash\ शब्दस्य च भित्यत्वाददोषः~॥ 

शये रलसयतन्वन मङ्गलार्थत्वमूपेद्धत्वकथनं तस्या 

युक्तमिति भावः~॥ छाया~॥ 

७ भीक्षणं इति अ. पाठः~॥ 

८ एतन्नित्यशब्दस्य वाहुल्यसमानार्थकत्वं {\qt ङछृमो हस्वादन्ति 

ङमुण्नित्यम्} इति निल्यपदधटितसूत्रेण ख्मुडागमं विधायापि 

{\qt इको यणचि}? इत्यादौ ङमुडभावं कुर्वतः {\qt शप्श्यनोर्नित्यम्} 

इति नित्यं नुमं विवायापि 

{\qt अपश्यती वत्समिवेन्दुबिम्बं तच्छर्वरी गौरिव हुक \textendash\ 

रोति~।} इति 

स्वोपशपातालविजयकाव्ये {\qt अपश्यती} इत्यत्र नुममकुर्वतः 

सू्त्रकृतोऽपि संमतमेव~॥ एतेनास्य प्रयोगस्यापशब्दत्वं प्रकाशयन् 

रुद्रदालंकारव्याख्याता नमिसाधुरपास्तः~॥ इति दाधिमथाः~॥ 

९ भगवदुक्तौ शिष्यः शङ्कते \textendash\ अथेति~। पदार्थद्वयस्य प्राक् 

प्रकान्तत्वादिति भावः~॥ छाया~॥ 

१० योग्यता च \textendash\ अर्थप्रतिपादनशक्तिः~। सा च दाह्याभावेऽपि 

 दहनशक्तेदैहनाश्रयत्ववत् अर्थाभावेऽपि शब्दाश्रया~॥ 

नित्यतासाधकपक्षनिर्णयाधिकरणम् ] 

पस्पशाद्निकम्~। 

६३ 



(उद्दयोतः ) नन्वेवमर्थग्रहणं व्यर्थमत आह \textendash\ द्रव्यपक्षे 

इति~॥ योग्यता$=$बोधजनकत्वयोग्यता \textendash\ तादात्म्यम्~। तदुक्तं भाष्ये \textendash\ 

नित्यो ह्यर्थवतामिति~। शब्दानामित्यर्थः~॥ ननु तादात्म्यस्य 

संबन्धत्वे कैथं तस्य नित्यत्वमिति चेन्न~। नष्टभाविवस्तुनोऽपि 

शब्देन बोधरद्वौद्धार्थेन तस्य तादात्म्यं नित्यमित्याशयात्~। शब्द \textendash\ 

वृत्तिधर्मस्थैवार्भवृत्तिधर्माभेदमापन्नस्य तादात्म्यत्वेनादोषाच्च~॥ श \textendash\ 


ब्दस्य च नित्यत्वादिति~। आकाशवकत्तन्निष्ठः शब्दोऽपि नित्यः~। 

व्यञ्जकाभावात्तु न सर्वदोपलम्भ इति भावः~॥ 

( द्वव्यएदार्थाभ्युपगमभाष्यम् ) 

अथवा द्रव्य एव पदार्थ एष विग्रहो न्याय्यः \textendash\ 

सिद्धे \textendash\ शव्दे \textendash\ अर्थे \textendash\ संवन्धे चेति~। द्रव्यं हि नित्यैम्, 

आकृतिरनित्या~॥ 

कथं ज्ञायते ? 

एवं हि दृश्यते लोके \textendash\ मृत्कयाचिदाकृत्या युक्ता 

पिण्डो भवति, पिण्डाकृतिमुपमृद्य घटिकाः क्रि \textendash\ 

यन्ते, घटिकाकृतिमुपमृद्य कुण्डिकाः क्रियन्ते~। 

तथा \textendash\ सुवर्णं कयाचिदाकृत्या युक्तं पिण्डी भ \textendash\ 

वति, पिण्डाकृतिमुपसृद्य रुचकाः क्रियन्ते, रुचका \textendash\ 

कृतिमुपमृद्य कटकाः क्रियन्ते, करकाकृतिमुपमृद्य 

स्वस्तिकाः क्रियन्ते~। पुनरावृत्तः सुवर्णपिण्डः पुनर \textendash\ 

परयाऽऽकृत्या युक्तः खदिराङ्गारसवर्णे कुण्डले भ \textendash\ 

वतः~। आकृतिरन्या चान्या च भवति, द्रव्यं पुनः \textendash\ 

स्तदेव~। आकृत्युपमर्देन द्रव्यमेवावशिष्यते~॥ 

( प्रदीपः ) द्रव्यं हि नित्यमिति~। असत्योपाध्यवच्छिन्नं 

ब्रह्यतत्त्वं द्रव्यशब्दवाच्यमित्यर्थः~॥ आकृतिरिति~। संस्था \textendash\ 

नम्~। ब्रह्मदर्शने च गोत्वादिजातेरप्यसत्यत्वादनित्यत्वम् , 

{\qt आत्मैवेदं सर्वम्} इति श्रुतिवचनात्~॥



१ अर्थवतामित्यस्य जातिमतामिति एकार्थविशेषाणामन्धार्थानामिति 

चार्थनिरासायाह \textendash\ शब्दानामिति~॥ छाया~॥ 

२ कथमिति~। तस्योभयरूपत्वादर्थस्य चानित्यत्वादिति भावः~॥ 

छाया~॥ 

३ नित्यमिति~। अपागादग्नेरन्नित्वं इत्यादिश्रुतिप्रामाण्यात् व्य \textendash\ 

क्तीनामानन्त्यात्तासाभाविद्यकत्वकल्पनापेक्षया लाघवाच्च जातेरेव तत्क \textendash\ 

र्प्यते~। तथाच जातिरनित्या, तदवच्छिन्नं ब्रहमैव द्रव्यमिति हि तन्नित्य \textendash\ 


मेेत्यभिप्रायः~। अथवा \textendash\ अवयवसंस्थानमाकृतिः~॥ 

४.पुर्वावयवसंयोगस्य द्रव्यारम्भकस्य विनाशेन पूर्वद्रव्यनाशे 

तेषामेव संयोगान्तराद्व्व्यान्तरमुत्पद्यते इति वैशेषिकमतनिरासायाह \textendash\ 

आकृतिरन्या चान्या च भवति~। द्रव्यं पुनस्तदेवेति~॥ छाया~। 

५ स च \textendash\ उपाधिश्च~। 

६ {\qt जातिरूपो वा} इति ध. पाठः~॥ 

७ ननु प्रदीपे आकृतिपदस्य संस्थानमित्यर्थमुकत्वा ब्रह्मदशने च~। 

इत्या्युक्तिः किमर्थन्तदाह \textendash\ भाष्ये यद्याकृतिशब्देनेति~। एवञ्च 

भाष्यस्थाकृतिशब्देन कम्बुग्रीवादिमदाकारस्तद्यङ्ग्या
घटत्वादिजातिश्चत्यु \textendash\ 





( उद्दयोतः ) ननु धटादिद्रव्यस्य कथं नित्यतेत्यत आह \textendash\ 

 असत्येति~। स च कम्बुग्रीवादिमदाकारादिरूपः, तद्यङ्गयजीतिरूपश्च~। 

भीष्ये यद्याकृतिशब्देन तद्यङ्ग्यमुच्यते तर्हि तदप्याविद्यकत्वेन व्यञ्ज \textendash\ 


कानित्यत्वेन चानित्यमित्याह \textendash\ जातेरपीति~। अनेनैतत्प्रकरणस्था \textendash\ 

कृतिपदस्य व्यङ्गयव्यञ्जकोभयपरतां सूचयति~। गौतमेनाप्युक्तं {\qt जा \textendash\ 

त्याकृतिव्यक्तयः पदार्थः} इति~॥ भीष्ये खदिराङ्गारसवर्ण 

कुण्डले भवतः इति प्रनोगादच्व्र्यन्ते विकृतेः कर्तृत्वं बोध्यम्~॥ 

( आकृतिपदार्थाभ्युपगमभाष्यम् ) 

आकृतावपि पदार्थ एष विग्रहो न्याय्यः \textendash\ सिद्धे 

शब्दे अर्थे संवन्धे चेति~॥ 

ननु चोक्तम् \textendash\ आकृतिरनित्या \textendash\ इति~॥ 

नैतदस्ति~। नित्याऽऽकृतिः~॥ 

कथम्? 

न क्चिदुपरतेति कृत्वा सर्वत्रोपरता भवति~। 

द्रव्यान्तरस्था तूपलभ्यते~॥ 

प्रदीपः ) न क्वचिदुपरतेति~। अनभिव्यक्तेत्यर्थः~। 

अद्रैतेन लोके व्यवहाराभावात् \textendash\ व्यवहारे चाकृतेरेकाकारपराम \textendash\ 

शहेतुत्वान्नित्यत्वम्~॥ 

( उद्दयोतः ) आकृतिपदेन भौतिरित्यभिप्रायेण \textendash\ भाष्ये \textendash\ आकृ \textendash\ 

तावपि पदार्थ इति~॥ क्कैचिदुपरतेति कृत्वा न सर्वत्रोपरतेत्य \textendash\ 

न्वयः~॥ ननूपरमो नाशश्चेत्तत्काले एवान्यत्र सत्त्वं विरुद्धमन अह \textendash\ 

अनभीति~॥ ननु परमार्थदृष्ट्या सर्वमनित्यमतआह \textendash\ अद्वैतेनेति~॥ 

नित्यत्वमिति~। सर्वदा एकाकारपरामश्दर्शनेन यावद्यवहारकालं 

तस्या अपि ध्रुवादित्वेन नित्यत्वमिति भावः~॥ 

(समाधानसाधकनित्यलक्षणभाष्यम् ) 

अथवा नेद्मेव नित्यलक्षणम् \textendash\ धुवं कूटस्थम \textendash\ 

विचाल्यनपायोपजनविकार्यनुत्पत्त्यवृद्ध्यव्यययोगि 

यत् तन्नित्यमिति~। 

भयमपि वाच्यम् , तत्राकारस्यानित्यत्वं भाष्यकृतैवोक्तम्, तघगयजा \textendash\ 

तेश्चानित्यत्वं ब्रह्मदर्शने सतीत्यादिनोपपादयति~। आकारस्य तद्यङ्ग्य \textendash\ 

स्योभयस्यापि पदार्थत्वै दशनान्तरसम्मतिमाह \textendash\ गौतमेनापीति~॥ 

८ {\qt ननु कुण्डले भवतः इत्यत्र भवतः} इति द्विवचनमनुपपन्नम् 

सुवर्णपिण्डस्यैवात्र कर्तृत्वावतस्तं प्रयोगमुपपादयति \textendash\ भाष्ये खदि \textendash\ 

रेति~॥ 

९ च्व्यन्ते प्रकृतेः कर्तृत्वादाह \textendash\ अच्ष्यम्ते विकृतेरिति~॥ छाया~॥ 

१० मतान्तरमाह \textendash\ आकृतावपीति~। अपिना द्रव्यपरिग्रहः~। 

अत एव मतान्तरत्वम्~॥ अथवेत्यादिः~॥ छाया~॥ 

११ नित्येति~। एकाकारानुगत्धीषेतुत्वाज्जातेर्व्ववहारदशायां 

सर्वदा नित्यत्वमित्यन्ये~॥ छाया~॥ 

१२ जातिरितीति~। न तु अवयवसंस्थानमाकृतिशब्दार्थोऽत्र~॥ 

१३ संशयनिवृत्त्यर्थ भाष्यान्वयमुपपादयति \textendash\ क्वचिदित्यादिना~॥ 

१४ दर्नेन तस्यास्त्चञ्कसंस्थानस्य च निल्यत्वम् इति पाठा \textendash\ 

न्तरम्~॥ 

६४ 

उद्दयोतपरिवृतप्रदीपप्रकाशितमहाभाष्ये~। [१ अ. १ पा. १ पस्पशाह्निके 



तदपि नित्यं \textendash\ यस्मिंत्त्वं न विहन्यते~॥ 

किं पुनस्तत्त्वम् ? 

तंद्भावस्तत्त्वम्~॥ 

आकृतावपि तत्त्वं न विहन्यते~॥ 

( प्रदीपः ) अथवेति~। असत्यत्वेऽपि तत्त्वतो लोकव्य \textendash\ 

बहाराश्रयेण जतेर्नित्यत्वं साध्यते~। त्रिविधा चानित्यता, 

संसर्गानित्यता यथा \textendash\ स्फटिकस्य लाक्षाद्युपधाने स्वरूपति \textendash\ 

रोधानेन पररूपप्रतिभासः~। उपधानापगमे स्वरूपप्रतिभासात्तु 

परिणामाभावः~॥ परिणामानित्यता यथा \textendash\ बदरफलस्य 

श्यामतातिरोभावे लौहित्यस्याविर्भावः~॥ प्रध्वंसानित्यता \textendash\ 

सर्वात्मना विनाशः~। एतत्रिविधानित्यताप्रतिक्षेपेण नित्यतां 

प्रतिपादयितुमुक्तं \textendash\ ध्रुवमित्यादि~। तत्र ध्रुवं कूटस्थमिति 

संसर्गानित्यता परिहृता, अविचालीति परिणामानित्यता, 

भनपायेत्यादिना प्रध्वंसानित्यता~॥ 

( उद्दयोतः ) अथार्वैयवसंस्थानरूपाया जातिव्यञ्जिकाया आकृ \textendash\ 

तेर्यावद्यवहारकालं मध्ये मध्ये उत्पत्तौ नाशेऽपि प्रकारान्तरेण 

नित्यत्वमाह \textendash\ भाष्ये \textendash\ अथवेति~॥ नित्यत्वलक्षणे ध्रवपदस्यैव 

व्याख्यानं \textendash\ कूटस्थमिति~। रूपान्तरापत्तिः \textendash\ विचालः~। यथा पयसो 

दध्यादिरूपता~। अनेन परिणामानित्यता परास्ता~। उत्पत्तेः 

सत्तापर्यन्तत्वादनुस्पत्तीत्यनेन जन्मसत्तारूपौ भावविकारौ निरस्तौ~। 

अवृद्धीत्यनेन तृतीयो बृद्धिलक्षणः~। अनुपजनेति चतुर्थः परि \textendash\ 

णामः~। नपँपयेति पञ्चमोऽपचयः~। एतद्रूपविकाररहितमिति 

तदर्थः~॥ अ्ययेति षष्ठो विनाशः~। दं च ब्रह्मविषयं नित्यत्वं, 

यावद्यवहारमेकरूपस्थितपदाविषयं च~॥ अयमेव न नित्यशब्दार्थः, 

प्रवाहाविच्छेदेऽतादृश्यपि नित्यत्वव्यवहारादित्याह \textendash\ भाष्ये \textendash\ तदपी \textendash\ 

ति~॥ यस्मिस्तत्त्वमिति~। यस्मिग्विहतेऽपि तद्वृत्तिधर्मो न विह \textendash\ 

न्यते तदित्यर्थः~॥ प्रवाहनित्यता चानेनोक्ता~। तन्नाशेऽपि तद्धमो न 

नश्यति~। आश्रयप्रवाहाविच्छेदादिति भावः~॥ 

१ तद्भाव इति~। तस्याः$=$आकृतेर्भावः \textendash\ ्तन्निष्ठो धर्मः, सर्वदा \textendash\ 

प्रत्यभिज्ञायोग्यत्वादि~। योग्यतावच्छेदकञ्च \textendash\ त्रिकालवृत्तित्वम्~॥ {\qt तस्य


भावस्तत्त्वम्} इति मुद्रितपाठः~॥ 

२ उपसंहरति \textendash\ आकृताविति~। अवयवसंस्थानरूपायामाकृती 

नष्टायामपि तत्त्वं तद्वयङ्गयस्तद्बृत्तिधर्मो न बिहन्यते,
आश्रयप्रवाहावि \textendash\ 

च्छेदादिति साऽपि नित्येत्यर्थः~॥ छाया~॥ 

३ जातेरिति~। तद्वयजकावयवसंस्थानरूपाकृतेरित्यर्थः~॥ छाया~॥ 

४ ननु लक्षणे किमर्थवहानि विशेषणानीत्यत आह \textendash\ त्रिविधा 

चेति~॥ छाया~॥ 

५ परिणामाभाव इति~। रूपान्तरप्रतिभासाभाव इत्यर्थः~॥ 

छाया~॥ 

६ प्रतिक्षेपेण~। निरसनेन~॥ छाया~॥ 

७ यथाश्चुतकैयदोक्तार्थाङ्गीकारे पौनरुक्त्यापश्तिरतीऽ५न्यथाव्याख्या. 

नमाह \textendash\ थथेति~॥ छाया~॥ 

८ यद्यप्यनपायेत्यादिना कैययेक्तं प्रतिपाद्यं स्फुटमेव तथापि 

सावन्मात्रप्रतिषादनेऽन्तिमविशेषणेनैव सिद्धेऽनेकविशेषणोपादानं व्यर्थ \textendash\ 





( नित्यानित्यत्वविचारस्याप्रकृतत्वबोधकभाष्यम् ) 

अथवा किं न एतेन \textendash\ इदं नित्यम् \textendash\ इदमनित्यमिति~। 

यन्नित्यं तं पदार्थ मत्वैष विग्रहः क्रियते \textendash\ सिद्धे 

शब्देऽर्थे संबन्धे चेति~॥ 

( प्रदीपः ) यन्नित्यमिति~। बुद्धिप्रतिभासः शब्दार्थः, 

यदा यदा शब्द उच्चारितस्तैदाऽर्थाकारा बुद्धिरुपजायते \textendash\ इति 

प्रवाहनित्यत्वादर्थस्य नित्यत्वमित्यर्थः~॥ 

( उद्दयोतः ) यन्नित्यमिति~। व्यक्तिजात्याकृतीनां मध्ये यन्नि \textendash\ 

त्यमित्यर्थः~॥ ननु शशश्वृङ्गादिपदार्थानां कथं नित्यत्वं, तेषां स्वरूप \textendash\ 


स्यैवाभावादत आह \textendash\ बुद्विप्रतिभास इति~। बाह्यः पदार्थो न 

शाब्दबोधे विषयः, किंतु बौद्धः~। स च प्रवाहनित्य इति भावः~॥ 

एतञच्च मञ्जृषायां निर्र निरूपितम्~॥ 

( वार्तिकद्वितीयखण्डावसरणभाष्यम् ) 

कथं पुनरज्ञायते \textendash\ सिद्धः शब्दोऽर्थः संबन्धश्चेति? 

( १ शब्दार्थसंबन्धानां प्रमाणबोधकवार्तिक \textendash\ 

द्वितीयखण्डम्~॥ २~॥ ) 

~॥~॥ लोकतः~॥~॥ 

(भाष्यम्) 

यल्लोकेऽर्थमुपादाय शब्दान्प्रयुञ्जते, नैषां 

निर्वृत्तौ यत्नं कुर्वन्ति~। ये पुनः कार्या भावा 

निर्वृत्तौ तावत्तेषां यत्नः क्रियतै~। तद्यथा \textendash\ घटेन् 

कार्य करिष्यन् कुम्भकारकुलं गत्वाऽऽह \textendash\ कुरु घटं 

कार्यमनेन् करिष्यामीति~। न तद्वच्छब्दान्प्रयुयुक्ष \textendash\ 

माणो वैयाकरणकुलं गत्वाऽऽह \textendash\ कुरु शब्दा \textendash\ 

न्प्रयोक्ष्य इति~। 

तावत्येवार्थमुपादाय शब्दान्प्रयुञ्जते~॥ 

( प्रदीपः ) लोकत इति~। अन्यथा कार्येषु वस्तुषु लोक \textendash\ 

व्यवहारः, अन्यथा नित्येषु~। शाब्दश्च व्यवहारोऽनादिवृद्ध \textendash\ 



मतस्तस्य तात्पर्यान्तरमप्यस्तीत्याशयेन भाष्यीयक्रममुल्लिख्य षड्भाव \textendash\ 

विकारक्रमेणाह \textendash\ उत्पत्तेरिति~॥ छाया~॥ 

९ स्वनिष्ठधर्माणामन्यथात्वं \textendash\ परिणामानित्यता~। अनपायेत्यादिना 

षड्भावविकारशूत्यत्वं दशितभ् , तश्च प्रध्वंसानित्यताशून्यत्वव्याप्यम्~॥


१० अत्र भगवता द्विविधं नित्यत्वं प्रतिपादितम्~। कैयटोक्तमेकम्~। 

जायतेऽस्ति वर्धते विपरिणमतेऽपक्षीयते नश्यतीत्येवरूपपड्भावविकार \textendash\ 

शून्यं त्वपरम्~। तदाह \textendash\ इदं चेति~। द्विविधमुक्तमित्यर्थः~॥ छाया~॥ 

१ १ नित्ययत्वानित्यत्वसाधनाग्रहे नास्माकं प्रयोजनमित्याशयेनाह \textendash\ 

अथवेति~॥ छाया~॥ 

१२ {\qt त्तदा तदार्थाकारा} इति ख. 

पाठः~॥ 

१३ अर्थसंबन्धयोव्यीकरणानिष्पाद्यत्वेन नित्यत्ववच्छव्दस्तैन सथेति 

तु केन प्रकारेण निश्चीयते इत्यर्थः~। शास्त्रस्य निष्पादकत्वाच्छब्दस्था \textendash\ 


नादिपरम्परयासिद्धत्वात्~। अन्यथा इन्द्रादीनां
तत्करणपघ्रवृत्त्यानर्थक्या \textendash\ 

पत्तेरिति भावः~॥ छाया~॥ 



घर्मनियमाधिकरणम् ] 

पस्पशाह्निकम्~। 

५ 

व्यवहारपरम्पराव्युत्पत्तिपूर्वक इति शब्दादीनां नित्यत्वम्~। घ \textendash\ 

टादयस्त्वर्थकियार्थिभिरन्यत आनीयन्ते, उत्पादविनाशयुक्ता \textendash\ 

श्लोपलभ्यन्ते~। नैवं शब्दादयः~॥ तावत्येवार्थमिति~। 

बुद्ध्या वस्तु निरूप्येत्यर्थः~॥ 

( उद्दयोतः ) शब्दादीनां नित्यत्वमिति~। व्याकरणानिष्पा \textendash\ 

ब्य्मेत्यर्थः~॥ ताव्रत्येेति~। वैयाकरणकुलमगत्वैवेत्यर्थः~॥ 

( इति नित्यतासाधकपक्षनिर्णयाधिकरणम् ) 

( धर्मतियमाधिकरणम् ) 

( वार्तिकतृतीयखण्डावतरणभाष्यम् ) 

यदि तर्हि लोक एषु प्रमाणम्, किं शास्त्रेण 

क्रियतै? 

(१ शास्त्रोपयोगप्रकथने वार्तिकतृतीयखण्डम्~॥ १~॥ ) 

~॥ लोकतोऽर्थप्रयुक्ते शब्दप्रयोगे 

शास्त्रेण धर्मनियमः~॥~॥ 

(भाष्यम् ) 

लोकतोर्थप्रयुक्ते शब्दप्रयोगे शास्त्रेण धर्मनि \textendash\ 

यमः क्रियते~॥ 

किमिदं धर्मनियम इति ? 

धर्माय नियमुः \textendash\ धर्मनियमः, धर्मार्थो वा नियमः \textendash\ 

धर्मनियमः धर्मप्रयोजनो वा नियमः \textendash\ धर्मनियमः~॥ 

( प्रद्वीपः ) अत्र भाष्यकारेण संभवन्तीमप्येकवाक्यतामना \textendash\ 

श्रित्य वाक्यत्रयं स्थपितम्~। सिद्धे शब्दार्थसंबन्धे 

शास्त्रं प्रवृत्तमित्येकं वाक्यम्~। कथं ज्ञायत इति प्रश्ने लोकतो 

ज्ञायते इति द्वितीयम्~। लोकत इत्ययावत्त्या लोकतोऽर्थ \textendash\ 

प्रयुक्त इत्यादि तृतीर्यम्~॥ शब्दप्रयोग इति~। प्रयोगग्रहणेन 

१ नन्वेवं शस्त्रनर्थक्यम् {\qt अप्राप्ते शास्त्रमर्थवत्} इति न्यायादि \textendash\ 

त्याशयेन तृतीयं खण्डं व्याख्यातु पृच्छति \textendash\ यदि तर्हीति~॥ छाया~॥ 

२ धर्मनियम हृति~। प्रथमपक्षे धर्मशब्दस्य प्रत्यवायपरिहारोऽर्थः~। 

द्वितीये धर्मपदेन नियम एव वाच्यः, साधुभिर्भाषितव्यमिति नियमस्य 

धर्मार्थत्वाश्नियमोऽपि धर्मपदेन व्यव्हियते~। तृतीये धर्मपदवाच्यमपूर्व \textendash\ 


मिति पक्षत्रयोपपत्तिः~॥ 

३ {\qt व्यवस्थापितम्} इति मुद्रितपाठः~॥ 

४ तृतीयमिति~। अत एव प्रश्नवाक्येषुक्रमेण {\qt प्रवृत्तम् ज्ञायते क्रियते} इति क्रियाभेदोक्तिरिति भावः~॥ छाया~॥ 

५ {\qt लोकत एव शब्दार्थसंवन्धे सिद्धे अथज्ञानप्रयोजनकृते शब्द \textendash\ 

प्रयोगेऽपि च सिद्धे शास्त्रेण गवादय एव प्रयुक्ता धर्मजनका न 

गाव्यादय इत्येवं धर्मनियमः क्रियते} इत्येकवाक्यतेत्यर्थः~॥ छाया~॥ 

६ ननु कथं प्रयोगस्यार्थज्ञानजनकत्वमिति चेत्~। शृणु \textendash\ शब्द \textendash\ 

प्रयोगेण हि शब्दाविर्भावः~। आविर्भूतश्च शब्दः परेण श्रूयते~। श्रवणेन 

च तदर्धस्मरणम्~। ततो बोधो इत्येवं शब ब्दप्रयोगस्यार्थधीजन कत्वमिति 

तदाह \textendash\ प्रयोगेणेति~॥ छाया~॥ 

७ {\qt एवं प्राप्ने शब्दप्र इति मुद्रितपुस्तृकपाठः~। तथा पाठे 

हि एवं शब्दप्रयोगे प्राप्ते सति शास्त्रेण धर्म नियमः क्रियते} इति 

सति सप्तम्यर्थः प्रतिभासेत~। वार्तिकप्रतिपादितश्च रोकत्तो५र्थज्ञान \textendash\ 

५ प्र० पा० 



{\qt प्रयोगाद्धर्मो न तु ज्ञानमात्रात्} इत्युक्तं भवति~। अर्थेनात्मप्र \textendash\ 

त्यायनाय् श्रयुक्तः \textendash\ अर्थप्रयुक्तः~॥ 

धर्मीय नियम इति~। चतुर्थ्या तादर्थ्य प्रतिपाद्यते~। 

संबन्धसामान्ये तु षष्टीं विधाय समासः कर्तव्यः, चतुर्थीसमा \textendash\ 

सस्य प्रकृतिविकारभाव एव विधानात्~॥ धर्भार्थं इति~। 

धर्मार्थत्वान्नियम् एव धर्मशब्देनाभिधीयते \textendash\ इति कर्मधारयः 

समासः~॥ धर्मप्रयोजन इति~। लिङादिविषयेण नियोगाख्येन 

धर्मेण प्रयुक्त इत्यर्थः~॥ f 

( उद्दयोतः ) ननु सकृदन्वितस्य {\qt लोकतः} इत्यस्य पुनः 

{\qt अर्थप्रयुक्ते} इत्यनेनान्वयः कथम् ? वातिके चैर्कवाक्यतयैवान्वयः 

प्रतीयतेऽ? आह \textendash\ अत्रेति~॥ अनाश्रित्येति~। न्यूनतापरिहारा \textendash\ 

येत्यर्थः~। {\qt लोकतः} इत्यस्य च लोकव्यवहारत इत्यर्थः~॥ भाष्ये \textendash\ 

अर्थप्रयुक्ते इति~। अर्थज्ञानप्रयोजनेन कृत इत्यर्थः~। परस्य 

विशिष्टार्थबोधो भवत्विति शब्दः प्रयुज्यते, प्रयोगेर्णविर्भूतशब्दज्ञाने \textendash\ 


नार्थश्ञानादिति भावः~॥ एवम्प्राप्तशब्दप्रयोगे शास्त्रेणास्येयं
प्रकृत्निरयं 

प्रत्यय इत्यादिप्रकृत्यादिविभागज्ञानद्वारा गवादय एव प्रयुक्ता धर्म \textendash\ 

जनकाः, न गाव्यादय इति नियमः क्रियत इति तात्पर्यम्~॥ 

विधानादिति~। तादर्थ्यस्य षष्ठयर्थत्वबोधनाय {\qt त्वेवं} प्रयोग इति 

भावः~॥ {\qt चतुर्थी} इति योगविभागो न भाष्यारूढः, {\qt सुप्सुपा} 

इति समास इत्यप्यगतिकगतिरित्येवं व्याख्यातम्~॥ द्वितीयपक्षात्तुतीये 

विशेषं दर्शयति \textendash\ लिङादिति~। प्रभाकराङ्गीकृतमतेनैदम्~। 

तन्मते हि लिङादीनार्मपूर्वरूपं कार्य वाच्यम्~। तदेव च स्वस्मिन् 

पुरुषं प्रयुजानं नियोग इत्युच्यते~। स एव धर्मः, तेन प्रयुज्यते \textendash\ 

आक्षिप्यते इति कर्मल्युडन्तः प्रयोजनशब्दः~। स चासाधुनि \textendash\ 

बृत्तिरूपो नियमः~। {\qt धर्मप्रयोजनः} इति षष्ठीसमासः~। एवं च 

द्वितीये धर्मफलको नियम इत्यर्थः~। तृतीये धर्मप्रयोज्य इत्यर्थं इति 

भेदः~॥ {\qt केचित्त धर्माय नियमः} इत्यनेन प्रत्यवायपरिहाररूपधर्मा \textendash\ 



प्रयोजनेन कृते शब्दप्रयोगे {\qt शास्त्रेण धर्मनियमः क्रियते} इति सर्व \textendash\ 

सम्मतोऽर्थः, तत्सम्पत्तये च एवम्प्राप्तो यः शब्दस्तस्य प्रयोगेः 

इत्यर्थकः {\qt एवम्प्राप्तशब्दप्रयोगे} इति अन्थपाठ इति प्रतिभाति~॥ 

८ {\qt अपूर्वसंज्ञकंकार्य इति मुद्रितपाठः~॥} 

९ केचित्तु इति~। धर्मनियम इत्यत्र भाष्यकृता त्रिविधा 

व्युत्पत्तिः प्रदर्शिता~। प्रदीपे त्रिविधायास्तस्याः प्रथमे
धर्मशब्दस्या \textendash\ 

पूर्वमर्थः, द्वितीये धर्मः \textendash\ यागादिः, तृतीये धमः \textendash\ नियोग इत्येवम्र्थ 

उक्तः~। अत्र चान्तिमः पक्षः प्रभाकरमतेन प्रसाध्यते, मतान्तरेनियो \textendash\ 

गासम्भवात्~। भाष्यत्चैकमतेन प्रवृत्तं, त्रिविधव्युत्पत्त्या
सम्पन्नोऽर्थश्च 

सैद्धान्तिक इत्यभिप्रायेण केचित्त्वित्यादिनोद्दयोते प्रकारान्तरं
प्रदर्श \textendash\ 

यति~। तदभिप्रेताश्च \textendash\ प्रत्यवायपरिहारः, यागादिः, अपूर्वमिति क्रमे \textendash\ 

णार्थाः~। एकविधेस्तदितरनिषेधफलकत्वात् {\qt एकः शब्दः \textendash\ } इत्यनेन 

श्रौतवचनेनासाधुशब्दप्रयोगे प्रत्यवायो बोध्यते~। तथाच धर्मीय \textendash\ 

प्रत्यवायपरिहाराय यो नियम इति प्रथमोऽर्थः सम्पन्नः~। द्वितीयश्च 

मतद्व्येऽपि समानः, {\qt नानृतं वदेत्} इति कर्माङ्गनिषेधो यागाङ्गत्वा \textendash\ 

द्याग एव, स च धर्मपदवाच्यः~। तृतीये धर्मस्य \textendash\ अपूर्वस्य प्रयोजन 

यो नियमः साधुभिर्भाषितव्यमिति~। उद्दयोतप्रदर्शिते मतभेदप्रसक्किर्नी 

स्तीति विशेषः~॥ 

६६ 

उद्दयोतपरिवृतप्रदीपप्रकाशितमहाभाष्यम्~। [१अ.१ पा. १ आ. पस्पशाहिके 



याय नियमः, असाधुप्रयोगेऽधर्मोत्पत्तिः~॥ {\qt धर्मार्थो नियमः} इत्य \textendash\ 

नेन धर्मस्य यशादेरङ्गभूत इत्यर्थः, {\qt नानृतं वदेत} इति ऋत्वङ्ग \textendash\ 

बावंयैन क्रतुवैगुण्यस्य बोधनात्~॥ {\qt धर्मप्रयोजनो नियमः} इत्यनेन 

{\qt एकः शब्दः} इति श्रुतेमियमादृष्टं पुरुषार्थर्करं सूचितमित्याहुः~॥ 

(१ वार्तिकचतुर्थखण्डम्~॥ २~॥ ) 

यथा लौकिकवैदिकेषु~॥ 

(भाष्यम् ) 

प्रियतद्धिता दाक्षिणात्याः~। {\qt यथा लोके वेदे 

च} इति प्रयोक्तव्ये यथा {\qt लौकिकवैदिकेषु} इति 

प्रयुञ्जते~॥ 

अथवा \textendash\ युक्त एंव तद्धितार्थः~। यथा लौकिक \textendash\ 

वैदिकेषु च कृतान्तेषु~॥ 

लोके तावत् \textendash\ {\qt अभक्ष्यो आम्यकुक्कुटः, अभक्ष्यो 

ग्राम्यसूकरः} इत्युच्यते~। भक्ष्यं चै नाम क्षुत्प्रतीघा \textendash\ 

तार्थमुपादीयते~। शक्यं चानेन श्वमांसादिभिरपि 

क्षुत्प्रतिहन्तुम्~। तत्र नियमः क्रियते \textendash\ इदं भक्ष्यम्, 

इदमभक्ष्यमिति~॥ तथा \textendash\ खेदात्ख्रीषु प्रवृत्तिर्भवति~। 

समानश्च खेदविगमो गम्यायां चागम्यायां च~। 

तत्र नियमः क्रीयते \textendash\ इयं गम्या \textendash\ इयमगम्येति~॥ 

( प्रदीपः ) प्रियतद्धिता इत्ति~। नायमपशब्दः, किंतु ये 

लोकवैदयोर्सवा अवयवास्ते लोकवेदशब्दाभ्यामभिधातुं शक्य \textendash\ 

न्ते~। आधाराधयभावकल्पनया तु तद्धितप्रयोगः प्रियतद्वित \textendash\ 

त्वनिमित्तः~। यथा \textendash\ कश्चिद्धनस्पतय इति प्रयुङ्क्ते, कश्चिद्धान \textendash\ 



१ प्रियतद्धिता इति~। लोकवेदशब्दयोः समुदायेऽवयवे च 

समानैव प्रबृत्तिरिति लोकवेदावयवबोधनेच्छायान्तद्धितप्रत्ययो नाव \textendash\ 

श्यक इत्याशयेन तस्य नैरथैक्यं प्रतिपादयन् भाष्यकारः {\qt अथवा युक्त 

एव} इत्यादिना तस्य सार्थक्यमुपपादयति~। प्रथमपक्षे लोकवेदश \textendash\ 

ष्दाभ्यां समुदायः, तदुत्तरप्रत्ययेन च लोकवेदावयवः \textendash\ अभक्ष्यो 

ग्राम्य \textendash\ पयोन्रत इत्यादिर्बोध्यः~। एवञ्च प्रभमपक्षे \textendash\ अवयवावयविनोरा \textendash\ 

धराधेयकल्पना, उभयबोधकत्वेन प्रसिद्धयोर्लोकवेदशब्दयोश्च समुदा \textendash\ 

यमात्रबोधकत्वम्~। द्वितीये च पक्षे तद्धितेन सिद्धान्तार्थ एवोच्यते \textendash\ 

सोकवेदयोर्भवा णे सिद्धान्तास्तेषु यथेति~। एवञ्चात्र पक्षे
पूर्वोक्तदोषरा \textendash\ 

हित्य, प्रत्ययश्च सार्थकः~। यथा लोकवेदसिद्धान्तेषु नियमो धर्मप्रयो \textendash\ 

ज्नस्तथा शब्दप्रयोगे शास्त्रेण धर्मनियमः क्रियते \textendash\ साधुभिरेव भाषि \textendash\ 

त्वयं नासाधुभिरिति~। अभक्ष्यो ग्राम्यकुकुटः पयोत्रतो ब्राह्मण इत्या \textendash\ 

दिषु लौकवेदावयवेषु शब्देन नियमो न प्रदर्शितः किन्तु लोकवेदसि \textendash\ 

शान्ते एव स इति तद्धितप्रत्ययः सूपपादो भवतीत्याशयः~॥ 

२ वार्तिककृत इति शेषः~। अनेन वार्तिककारस्य दाक्षिणात्यत्वं 

सूचितम्~। निरर्थकशब्दाङम्बरकारित्वस्वभावेनोपहसनीयता तेषां 

तदन्तर्गतत्वादुपहासश्च कृत इति बोध्यम्~। छाया~॥ 

३ {\qt एवात्र} इति क. ख. पाठः~। वैदिकेषु च कृतान्तेष्वित्यत्र 

कृतान्तशब्दस्य दृष्टान्तमर्थ ब्रुवन् गुरुप्रसादो ननु बभ्राम,
{\qt सिद्धान्तः 

शब्दार्थो भवरूपः इति प्रदीपं १कृतान्तः सिद्धान्तैः} इत्युद्द्योतञ्च 

श्ययतोऽप्यस्य निराकुलब्चक्षुरिति~॥ 





अथवेति~। नात्रावयवावयविविभागः, किं तर्हि ? वेदलोर्क \textendash\ 

व्यतिरिक्तः सिद्धान्तः शब्दीर्थो भवरूप इत्यर्थः~॥ लौकिकः \textendash\ 

स्मृत्युपनिबद्धः~। वैदिकः \textendash\ श्रुत्युपनिबद्धः~॥ शक्यं चाने \textendash\ 

नेति~। शकेः कमसामान्ये लिङ्गसर्वनामनपुंसकयुक्ते कृत्यप्रत्ययः~। 

तंतः शब्दान्तरसंबन्धादुपजायमानमपि स्त्रीत्वं वहिरङ्गत्वादन्तर \textendash\ 

ङ्गलसंस्कारं न बाधते इति {\qt शक्यं क्षुत्} इत्युक्तम्~। यदा तु 

पूर्वमेव विशेषविवक्षा तदा {\qt शक्या क्षुत्} इति भवत्येव~। यदा तु 

प्रतिघातस्यैव क्षुत्कर्म, शकेस्तु प्रतिघातः, तदा {\qt क्षुधं प्रतिहन्तुं 

शक्यम्} इति भवति~॥ खेदादिति~। खेदयतीति \textendash\ खेदःरागः, 

इन्द्रिरयनियमासामर्थ्यं वा खेदः~॥ 

(उद्दयोतः ) नन्वीदृशो धर्मनियमः क दृष्टोऽत आह भाष्ये \textendash\ 

यथेति~॥ ननु {\qt लोकवेदयोः} इत्येव सिद्धे तद्धितनिर्देशोऽयुक्तोऽत 

आह \textendash\ प्रियतद्धिताः इति~॥ नन्वन्यार्थेऽन्यशब्दप्रयोगेऽपशब्दत्वं 

स्यादत आह \textendash\ नायमिति~॥ ते वेदैलोकेति~। अवयवं समु \textendash\ 

दायशब्दप्रयोगदर्शनादिति भावः~। आधाराधेयकल्पनया 

त्विति~। लोकरूपसमुदायतदर्वयवयोरित्यादिः~॥ नात्रावयवा \textendash\ 

वयविविभाग हति~। {\qt तत्र भवः} हत्यादिव्यवहारप्रयोजको विभा \textendash\ 

गारोपोऽबयवावयविनोर्नेत्यर्थः~॥ कृतान्तेषु~। कृतान्तः \textendash\ सिद्धान्तः, 

तत्प्रतिपादकवाक्येष्वित्यर्थः~॥ {\qt भक्ष्यं च नाम} इत्यस्य तद्भवति 

यदिति शेषः~॥ {\qt शक्यं क्षुत्} इति सामानाधिकरण्यं कथं भिन्नलिङ्गत्वादत 

आह \textendash\ शकेरिति~॥ तदा क्षुधमिति~। ननु प्रैधीनकियानेरूपित \textendash\ 

शक्तेरभिधाने गुणक्रियानिरूपितशक्तेरनभिहिताया अभिहितवत्प्रकाशस्य 

{\qt स्वादुमि णमुल्} इत्यत्र वक्ष्यमाणत्वेन कथमेतत् ? इति चेत्, न~। 

प्रधानतिङश्तीर्थक्रियानिरूपितशक्तेरेवायं स्वभावो यत् स्वसमानाधिकर \textendash\ 



४ लौकिकसिद्धान्तमुपपादयति \textendash\ लोके सावदित्यादिना~॥ 

५ भक्ष्यं च श्चुत्प्रतिघातार्थः इति अ. पाठः~॥ 

द {\qt लोकवेव्व्यतिरिक्तसिद्धान्तशब्दार्थोभयरूप} इति मुद्रितपाठः 

क्वचिदेवोपलब्धः~॥ 

७ भावनाद्वैविध्येन तत्स्वरूपमाह \textendash\ शब्दार्थो भवेति~॥ तद्धि \textendash\ 

तार्थ इति शेषः~॥ इति छाया~॥ 

८ शकेरिति~। क्षुत्सामानाधिकरण्यात्पूर्वमिति शेषः~॥ छाया~॥ 

९ {\qt तत्ः पदान्तर} इति ख. पाठः~॥ 

१० इन्द्रियेति~। तन्निग्रहाक्तेजनितपीडेति तु तत्त्वम्~॥ 

छाया~॥ 

११ {\qt ते लोकवेदशब्दाभ्यां इति प्रदीपग्रन्थस्य प्रतीक ते वेद \textendash\ 

लोकेति} इत्येव सर्वत्रोपलभ्यते~॥ 

१२ प्रधानेतरयोर्यत्रं द्रव्यस्य क्रिययोः पृथक्~। 

शक्तिर्गुणाश्रया तत्र प्रधानमनुरुध्यते~॥ 

प्रधानविषया शक्तिः प्रत्ययेनाभिधीयते~। 

यदा गुणे तदा तद्वदनुक्ताऽपि प्रतीयते~॥ 

इति हर्युक्तेः~॥ 

१३ तिङ्न्तेति~। प्रकृते तु कृदन्तमिति न तस्य विषय. इति. 

भावः~॥ छाया.~॥ 

धर्मनियमाधिकरणम् ] पस्पशह्निकम् 



णगुणशक्तेरभिहितवत्प्रकाश इत्यभिमानः~। वस्तुत इदमयुक्तमेवेति 

{\qt स्वाहुमि \textendash\ } इत्यत्र निरूपयिष्यामः~॥ खेद इति~। एवञ्च ग्राम्यकु \textendash\ 

कुटपरदारादौ विशेषनिषेधस्येतराभ्यनुज्ञाफलकतया आरण्यकतद्भक्षण \textendash\ 

स्वदारगमनयोर्यथा दोषामावः, यथा च ग्राम्यकुक्कुटभक्षणपरदारगम \textendash\ 

नयरैषर्मः, तथा शास्त्रजनितशानपूर्वके गवादिभ्रयोगेऽपूर्वोत्पत्तिरूपं 

फलं शास्त्रेण बोध्यतेः तेषां साधुत्वबोधनात्~। अर्थादपशब्दानामधर्म \textendash\ 

जनकत्वं बोध्यते, एकविधेस्तदितरनिषेधफलकत्वात्~। यथैकनिषेधस्या \textendash\ 

प्रराभ्यनुञ्ञाफलकत्वमिति भावः~। दृष्टान्तता र्ववंशेन, वक्ष्यमाणवैदि \textendash\ 

कानां तु सर्वाशेनेति बोध्यम्~॥ r 

( वैदिकसिद्धान्तोपपादकभाष्यम् ) 

वेदे खल्वपि \textendash\ {\qt पयोव्रतो ब्राह्मणः \textendash\ यवागूव्रतो 

राजन्यः \textendash\ आमिक्षाव्रसो वैश्यः} हत्युच्यते~। व्रतं च 

नामाभ्यवहारार्थमुपादीयते~। शक्यं चानेन शालि \textendash\ 

मांसादीन्यपि ब्रतयितुम्~। तत्र नियमः कियते~। 

तथा \textendash\ {\qt बैल्वः खादिरो वा यूपः स्यात्} इत्यु \textendash\ 

च्यते~। यूपश्च नाम पश्वनुबन्धार्थमुपादीयते~। 

शाक्यं चानैम र्यत्किचिदेव काष्ठमुच्छ्रित्यानुच्छित्य 

वा पशुरनुबन्धुम्~। तत्र नियमः क्रियतै~। 

तथा \textendash\ अग्नौ कपालान्यधिश्रित्याभिमन्त्रयते \textendash\ 

भृगूणामङ्गिरसां धर्मस्य तपसा तप्यध्वम् इति~। 

अन्तरेणापि मन्त्रमग्निर्हनकर्मा कपालानि सन्ता \textendash\ 

पयति~। तत्र च नियमः क्रियते~। 

एवं फ्रियमाणमभ्युदयकारि भवतीति~॥ 

( प्रदीपः ) पयोव्रत इति~। सत्यामर्थितायां {\qt पय एव 

व्रतयति} इति नियमोऽयं न तु विधिः, अर्थित्वाभावे कारणाभावात्~॥ 

(उद्दयोतः ) वतयतीति~। अभ्यवहार्यत्वेनोपादत्ते \textendash\ इत्यर्थः~॥ 

भाष्ये \textendash\ उच्छ्रित्यानुच्छ्रित्य वेति~। संतक्ष्यासंतक्ष्य वेत्यर्थः~। 

निखन्यानिखन्य वेत्यर्थ इत्यन्ये~॥ 



१ अभिमानपदसूचितामरुचिमाह \textendash\ वस्तुत इति~॥ छाया~॥ 

२ धर्मोत्पत्तिश्चति बोध्यम्~॥ छाया~॥ इदं चिन्त्यम्~। आरण्य \textendash\ 

कत \textendash\ द्भक्षणस्य रागतः प्राप्तत्वेन धर्माहेतुत्वात्~। स्वदारगमनं तु ऋतौ 

धर्मजनकमन्यदा तु तत्रापि दोषाभावमात्रमिति (र. ना.)~। सुप्रसिद्ध \textendash\ 

मेव मीमांसापाण्डित्यं र.नापण्डितानाम्~। मीमांसाऽभिशस्य छाया \textendash\ 

कारस्यायमभिप्रायः \textendash\ अभक्ष्यो ग्राम्यकुक्कुट इति पुरुषार्थनिषेधस्यारण्यक \textendash\ 


भक्षणेऽव्यापाराद्वोषोऽपि नास्ति धर्मोऽपि न भवति \textendash\ इति तत्र दोषा \textendash\ 

भाव शत्येव~। स्वस्त्रीगमने तु कतौ भार्यामुपेयादिति वचनस्य पुरुषा \textendash\ 

र्धस्य व्यापारेण धर्मोत्पत्तिरिति~॥ 

३ दोषश्चत्यपि बोध्यम्~॥ छाया~॥ \textendash\ इदमपि चिन्त्यम्~। अधर्म \textendash\ 

स्यैव तत्र दोषत्वात् (र. ना.)~। यत्र प्राप्तिश्च रागतो निषेधश्च
पुरुषार्थ \textendash\ 

स्तत्र प्रतिषिध्यमानस्य ग्राम्यकुक्कुटभक्षणस्यानर्थहेतुत्वादधधर्म
इत्युक्तम्~। 

ऋतौ भार्यामिति वचनेन फलितार्धस्य निषेधस्य पुरुषार्थत्वाभावेन पर \textendash\ 

स्त्रीगमनाच्च दोष इति सूपपादितं छायाकारेणं~। अभक्ष्यो ग्राम्य इत्यत्र 

निषेधस्य पुरुषार्धत्वात् ऋतौ भार्यामित्यत्र च परदारगमनस्य फलि \textendash\ 





५७ 

( दार्ष्टान्तिक उपसंहारभाष्यम् ) 

एर्वैमिहापि समार्नायामर्थगतौ शब्देनापशब्देश 

च धर्मनियमः क्रियते \textendash\ शब्देनैवार्थोऽभिधेयो नाप \textendash\ 

शब्देनेति~। एवं क्रियमाणमभ्युदयकारि भवतीति~॥ 

( प्रदीपः ) समानायामिति~। यद्यपि साक्षादपभ्रंशा म 

वाचकास्तथापि स्मर्यमाणसाधुशब्दव्यवधानेनार्थं प्रत्याययन्ति~। 

केचिच्चापभ्रंशाः परम्परया निरूढिमागताः साधुशब्दानस्मारयन्त 

एवार्थ प्रत्याययन्ति~॥ अन्ये तु मन्यन्तै \textendash\ साधुशबदवंदपभ्रंशा 

अपि साक्षादर्थस्य वाचका इति~॥ 

( उद्दयोतः ) ननु अपभ्रशा न वाचकाः, वाच्यप्रतीत्यन्यथाऽनु \textendash\ 

पपस्या कल्प्यमानशक्तेः शिष्टप्रयुक्तसंस्कृतेष्वेव कल्पनात्~। तदनु \textendash\ 

सारेणैव चापभ्रंशानां साधुस्मरणेन बोधकत्वोपपत्तेः {\qt समानायां} 

इत्यनुपपन्नमिति शङ्कते \textendash\ यद्यपीति~॥ तथापीति~। एवं च 

वाचकत्वाभावेऽप्यर्थप्रत्यायकत्वाविशेषेण {\qt समानायां} इत्युक्तम्~॥ 

निरूढिमागता इति~। ते च शक्तिभ्रमेण बोधका इति भावः~॥ 

शक्तिभ्रमश्चेत्थम् \textendash\ केनचित् {\qt गाबी} इति प्रयुक्ते {\qt गौः इति साधु} \textendash\ 

स्मरणात् प्रयोज्यस्य वोधेऽपि तदस्थस्य गावीशब्दादेवास्य गोबोध इति 

भ्रमेण तन्मूलकोऽन्येषामपि भ्रम इति~॥ वस्तुतो विनिगमनाविरहात् 

भाषाशब्देष्वपि शक्तिरेवेत्यर्ह \textendash\ अन्ये त्विति~। साधुत्वं चैत \textendash\ 

न्मते शब्दगतधर्मसाधनतावच्छेदकवैजात्यमेव~। यद्यपि {\qt साधूनेव 

भाषेत} इति न श्रूयते, तथापि फलनिर्देशवाक्ये तन्निर्देशादेवेदश \textendash\ 

विधिः कल्प्यते~। तत्र {\qt के साधवः?} इत्याकाङ्कायामेत एव साधव इति 

व्याकरणेन निष्पादनात्तद्वारा शास्त्रेण धर्मनियमो विधीयत इति 

बोध्यम्~॥ भाष्ये \textendash\ शब्देनैवेति~। साधुनेत्यर्थः~॥ एवं क्रियमा \textendash\ 

णमिति~। शास्त्रशानपूर्वकमुच्चार्यमाणमित्यर्थः~॥ 

इति धर्मनियमाधिकरणम्~॥ 

तार्थत्वेन पुरुषार्थत्वाभावाद्भाष्यकृत्प्रदशितस्तयोभेदेन व्यवहारोऽपि 

संगच्छते~। अत एव {\qt इदमपि चिन्त्यं} इत्युक्तिः साहसमात्रमेवेति~॥ 

४ लौकिकदृष्टान्तवाक्ये आाम्यकुक्कुटाभक्षणेन धर्मं आरण्यकभक्षणे 

दोषाभावः, ऋतौ भार्यागमनेन धर्मः परदारगमने दोषक्ष~। दाष्टा \textendash\ 

न्तिके साधुशब्दप्रयोगेण धर्मः, असाध्रुप्रयोगे {\qt तेऽसुराहेलय} इत्य \textendash\ 

नेनाधर्मश्चेत्यत आह \textendash\ त्वंशेनेति~॥ 

५ शक्यमिति~। भावेऽत्र प्रत्ययः~। व्रतयितुमित्यत्रास्यान्वयः~। 

शालिमांसादेस्तत्र~। अत एव न काचिदनुपपत्तिर्न वा {\qt स्वादुमि 

णमुल्} इत्यत्रत्यसिद्धान्तविरोध इति बोध्यम्~॥ छाया~। 

६ {\qt किञ्चिदेव} इत्येव प. पाठः~॥ 

७ दृष्टान्त उपपादितं दाष्र्टान्तिकेयोजयति \textendash\ एवमिति~॥ छा.~॥ 

८ {\qt नायामर्थावगतौ शब्देन} चापशब्देन इति क. ख. च. पाठः~॥ 

९ इति भ्रमस्तन्मू \textendash\ इति घ. पाठः~॥ 

१० अत एव प्रकृतभाष्यसंगतिः~। अनुभवोऽय्येवम्ं~। सुशब्दाज्ञानां 

पामराणामपशब्दैर्जायमानस्यार्थबोधस्य सुशब्दस्मारणद्वारकत्वासंभ \textendash\ 

वश्च~॥ छाया~॥ 

६८ उद्दयोतपरिवृतप्रदीपप्रकाशितमहाभाष्यम्~। [१ अ. १ पा. १ आ.
पत्पशाहिके 

(शब्दविषयप्रदर्शनाधिकरणम् ) ( आक्षेपासंगतिबोधकभाष्यम् ) 

( नियमौक्षेपे वार्तिकांशानुवादभाष्यम्~॥ )

~॥ अस्त्यप्रयुक्तः~॥ 

संन्ति वै शब्दा अप्रयुक्ताः~। तद्यथा \textendash\ ऊष, तेर, 

चक, पेच \textendash\ इति~॥ 

किमैतो यत्सन्त्यप्रयुक्ताः ? 

प्रयोगाद्धि भवाञ्छब्दानां साधुत्वमध्यवस्यति~। 

य इदानीमप्रयुक्ताः, नामी साधैवः स्युः~॥ 

( प्रदीपः ) अर्रत्यप्रयुक्त इति~। प्रयोगमूलत्वादस्याः 

स्मृतेरप्रयुक्तानामप्यन्वाख्यानादप्रामाण्यमाशङ्कते~॥ 

( उद्दयोतः ) {\qt अस्त्यप्रयुक्तः} इत्यादेः प्रकृतोपयोगमाह \textendash\ 

प्रयोगमूलत्वादिति~। लोकतोऽर्थप्रयुके शब्दप्रयोगे शास्त्रेण 

धर्मनियमः इत्यनेन प्रयुक्तेषु साध्वसाधुषु प्रयुक्तसाध्वन्वाख्यानेन 

तेषु धर्मः प्रयुक्तासाधुष्वधर्म इत्यर्थकेनेदं सूत्षितम्~। 
एवंचाप्रयुक्ता \textendash\ 

नामप्रयोगेणैवासाधुत्वानुमानादसाधूनामप्यनेनान्वाख्यानादप्रामाण्य \textendash\ 

मिति भावः~॥ वातिकोक्तनिर्येमानुपपत्तिश्चेत्यपि बोध्यम्~॥ 

भाष्ये \textendash\ किमत इति~। अप्रयुक्तसत्त्वं नास्माकमनिष्टं, साधुमात्रार्थ \textendash\ 

त्वाद्वयाकरणस्येत्यर्थः~॥ 

उत्तरम् \textendash\ प्रयोगादिति~। तथा चाप्रयोगादसाधुत्वानुमानमिति 

भावः~॥ 



१ {\qt अस्त्यप्रयुक्तः} इत्यस्य भागस्याग्निमवार्तिकेऽपि सत्त्वेन पौनरु \textendash\ 

कत्यापत्त्या न वार्तिकत्वस्~। किंतु आक्षेपबाधकवातिके प्रतिषेध्यनिरू \textendash\ 

पणाय भाष्यकृदुत्प्रेक्षितत्वमिति वदन्ति~॥ यथा पूर्वमीमांसायां 

(१~। २~। २० ) {\qt लोकवदिति चेत्} इति शङ्कासूत्रस्य {\qt न पूर्वत्वास्}

इति दूषणसूत्रमुपलभ्यते, तथैवात्रापि पूर्वमाक्षेपवातिकमावश्यकम्~। 

अत एव छायाकारेणाप्यस्याक्षेपवार्तिकस्योद्द्योतव्याख्यायां वार्तिकत्व \textendash\ 

मुपदर्शितम्~। आक्षेपबाधकवार्तिके {\qt अस्त्यप्रयुक्तः} इत्यंशस्य स्पष्टा \textendash\ 

र्थं वार्तिककारस्यैव पौनरुक्त्यमस्तु~। भाष्यकारस्य वा प्रतिषेध्यग्रन्थ \textendash\ 


पूरणम्~॥ आक्षेपवार्तिकमन्तरा आक्षेपसाधकवार्तिकेन साधनीयमेव 

न स्यादिति तु दाधिमथा वदन्ति~। वस्तुतस्तु नेदं वार्तिकम्, 

भाष्यकृत्कृतोऽरयं वार्तिकांशानुवाइ एव~। अत एव तद्याख्यानभाष्ये 

बुहुवचनान्तशब्दप्रयोग उपपन्नः~। यत्र च वार्तिकमेव व्याख्यायते 

तत्र वार्तिकगतशब्दैरेव ब्याख्यानं भाष्यकृत्सम्प्रदायसिद्धम्~॥ 

रे वार्तिके जात्यभिप्रायमेकवचनमिति सूचयन्व्याचष्टे \textendash\ सन्ती \textendash\ 

तिं~॥ छाया~॥ 

३ दोषदातुराशयं तन्मुखादेवावगन्तुं तटस्थः शङ्कते \textendash\ किमत 

हृति~॥ छाया~॥ 

४ यद् शब्दा अप्रयुक्ताः सन्तीत्यतः किमित्यन्वयः~। अप्रयुक्ताः 

शब्दाः सन्तीत्यनेन प्रस्तुते किमायातमित्याशयः~॥ 

५ साधकः स्युरिति~। पूर्ववार्तिके साधुशब्दानां प्रयोगेण धर्म \textendash\ 

माह भगवान्, न तु ज्ञानमात्रेण धर्मः~। तथाऽपशब्दानां प्रयोगा \textendash\ 

देवाधर्मः, न वु ज्ञानमात्रात्~। तथा च ये इदानीभप्रयुक्तास्ते साधवो 





( आक्षेपासंगतिबोधकभाष्यम् )

इदं तावद्विप्रतिषिंद्म् \textendash\ यदुच्यते \textendash\ {\qt सन्ति वै 

शब्दाः, अप्रयुक्ताः} इति~। 

यदि सन्ति, नाप्रयुक्ताः~। 

 अथाप्रयुक्ताः, न सन्ति~। सन्ति चाप्रयुक्ताश्चेति 

विप्रतिषिद्धम्~। प्रयुभ्रान एव खलु भवानाह \textendash\ सन्ति 

शब्दा अप्रयुक्ता इति~। कश्चेदानीमन्यो भवजाती \textendash\ 

यकः पुरुषः शब्दानां प्रयोगे सौधुः स्यात् ? 

( प्रदीपः ) यथा घटादीनां विनाऽप्यर्थक्रियया सत्त्वं गम्यते 

नैवं शव्दानां, ते हि सर्वदा व्यवहाराय प्रयुज्यमानाः सन्तः 

सर्त्वेनावसीयन्ते \textendash\ इत्याह \textendash\ इदमिति~। कश्चेदानीमिति 

उपहासपरम्~॥ 

(उद्दयोतः ) विप्रतिषिद्धत्वमुपपादयति \textendash\ यथेति~। एवं र्वै 

सत्त्वे प्रयोग आवश्यक एव, अप्रयोगे च सत्त्वमेव न स्यादिति 

भावः~॥ ननु लक्षणवशेन सत्त्वं, केनाप्यप्रयोगाच्चाप्रयुक्तत्वमिति 

न विरोधोऽतो भाष्ये \textendash\ प्रयुञ्चन एवेति~॥ ननु मया प्रयुक्त \textendash\ 

श्रेत्किम् ? तश्र सोपहासमाह \textendash\ कश्चेदानीमिति~। स्वयमेव प्रयोगं 

कृत्वा {\qt सन्ति चाप्रयुक्ताः} इत्यादीनां शब्दानां प्रयोगे साधुर्योग्यः 

स्यादित्यर्थः~। त्वत्प्रयोगेणैव
अ्रयुक्तत्वावगमे५प्रयुक्तत्वोक्तिविरूद्धेति 

भावः~॥ 



न भवेयुस्तेषां साधुत्वसस्पादनमेव शास्तरेण प्रसाध्यत इति अप्रयुक्त \textendash\ 

शब्दसाधकशास्त्रस्य नियमार्थत्वानुपपत्तिरित्याशयः~॥ 

६ ननु {\qt अस्त्यप्रयुक्तः} इति वार्तिकादिः \textendash\ {\qt तनूनाम्} इत्यन्तो 

ग्रन्थः प्रकृतानुपयोगादसंबद्धोऽत आह \textendash\ अस्त्येति~॥ छाया~॥ 

७ वार्तिकोक्तेति~। लोकतः प्रयोगे सिद्धे वैयर्थ्याच्छास्त्रं निया \textendash\ 

मकमिति तेनोक्तम्~। यदि तु तेषामप्यन्वाख्यानं तर्हि शास्त्रस्यापूर्व \textendash\ 

विधित्वेन तस्य नोपपत्तिरिति भावः~॥ एवं सत्याक्षेपिकी संगतिः 

स्फुटतरेति वोध्यम्~॥ छाया~॥ 

८ पूर्वपक्षे तटस्थो दोषमाह \textendash\ इदमिति~॥ छाया~॥ इदं 

विप्रति श्त्येव प. पाठः~॥ 

९ विप्रतिषिद्धमिति~। शब्दानां सत्ता प्रयोगादेवावम्यते, 

भवता चोच्यते \textendash\ अप्रयुक्ताः शब्दाः सन्तीति~। अत्र यदि सन्ति \textendash\ इति 

भवताऽवगम्यते तत् शब्दानां प्रयोगादेवावगम्यमिति तेषां प्रयोगोऽ \textendash\ 

स्त्येव~। यदि अप्रयुक्तास्तर्हि ते न सन्तीति विप्रतिषिद्धमित्यर्थः~॥ 

१० इतीति~। इदं मिथ इति शेषः~॥ छाया~॥ 

११ शब्दानां प्रयोगे \textendash\ {\qt सन्ति चाप्रयुक्ताः} इति शब्दानं प्रयोगे~। 

एतादृशं विरुद्धं भाषणं त्वदन्यः कः कुर्यादित्याशयः~॥ 

१२ {\qt साधु स्यात्} इति प. पाठः~॥ 

१३. यथा घटेति~। घटादयो हि जलाहरणादिक्रियाशून्या अपि 

प्रसिद्धा दृश्यन्ते, शब्दस्तु न तथा, तस्यं हि सत्ता प्रयोगमात्रैक \textendash\ 

गम्येति भावः~॥ 

१४ शब्दानां सत्त्वस्य प्रयोगैकनिबन्धनत्वात्तदभावेऽपि तत्सस्ताङ्गी \textendash\ 

करण विरुद्धं प्रमाणाभावादित्याह \textendash\ एवं चेति~॥ छायाः~॥ 

शब्दविषयप्रदर्शनाधिकरणम् ] पस्पशाह्निकम्~। ९ 

( आक्षेपसंगतिसाधकभाष्यम् ) 

नैतेद्विप्रसिषिद्धम्~। सन्तीति तावद्कूमः, यदेता \textendash\ शास्त्रविदः
शास्त्रेणानुविद्धते 

ऊशास्त्रविदः शास्त्रेणानुविदधते~॥

अप्रैयुक्ता इति ब्रूमः, यल्लोकेऽप्रयुक्ता इति~। 

यैदप्युच्यते \textendash\ {\qt कश्चेदानीमन्यो भवज्जातीयकः 

पुरुषः शब्दानां प्रयोगे साधुः स्यात्} इति~। 

र्नं ब्रूमोऽस्माभिरप्रयुक्ता इति~। 

किं तर्हि ? 

लोकेऽप्रयुक्ता इति~॥ 

( प्रदीपः ) उत्तरं तु शास्त्रदृष्ट्या प्रकृतिप्रत्ययादिसद्भावाद \textendash\ 

जनुमितसत्त्वाः, व्यवहारे तु न दृश्यन्त इत्युक्तम्~॥ 

(आक्षेपभाष्यम् ) 

ननु च भवानप्यभ्यन्तरो लोके~॥ 

(समाधानभाष्यम् ) 

अभ्यन्तरोऽहं लोके, नं त्वहं लोकः~॥ 

( प्रदीपः ) न त्वहं लोक इति~। यथा लोकोऽर्थावग \textendash\ 

माय शब्दान् प्रयुङ्क्ते नैवं मयैतेऽर्थे प्रयूक्ताः, अपि तु स्वरूपप \textendash\ 

दार्थका इत्यर्थः~॥ 

(उंद्दयोतः) {\qt ननू लोकान्तर्भूतस्य न स्वह्ं लोकः} इति वचो 

विरुद्ममत आह \textendash\ यथेति~। अर्थबोधाय शब्दान्प्रयुञ्जनो हि मम 

लोकत्वेनाभिमत इत्यर्थः~॥ 

( नियमाक्षेपबाधर्क् वार्तिकम्~॥ १~॥ ) 

~॥~॥ अस्त्यप्रयुक्त इति चेत् नार्थे 

शब्दप्रयोगात्~॥~॥ 

( भाष्यम् ) अस्त्यप्रयुक्त इति चेत्, तन्न



१ इदं तावदित्याशङ्कां समाधत्ते \textendash\ नैतदिति~॥ छाया~॥ 

२ अनुविधानं \textendash\ संस्कारः~॥ छाया~॥ 

३ {\qt अप्रयुक्ता इत्यपि ब्रूमः} इति प. पाठः~॥ 

४ लोके इति~। अर्थबोधायेति शेषः~॥ छाया~॥ 

५ एवं विरोधं परिहृत्योपहासं परिहर्तुमनुवदति \textendash\ यदुपीति~॥ 

छाया~॥ 

६ मत्प्रयोगो न लोकप्रयोग इत्याशयेन परिहरति \textendash\ न बृम 

इत्ति~॥ छाया~॥ 

७ {\qt यल्लोकेऽप्र} इति प. पाठः~॥ 

८ {\qt शास्त्रं दृष्ट्वा प्रकृतिप्रत्ययादिसद्भावादनुमितसत्त्वात्} इति 

अ. पाठः~। 

५९ {\qt न चाहं} इति प. पाठः~॥ 

१० {\qt अस्त्यप्रयुक्त} इत्याशङ्कां दूषयति \textendash\ अस्त्यप्रेति~॥ छाया~॥ 

११ शब्दप्रयोगादिति~। {\qt अर्त्यप्रयुक्तः} इंति वार्तिकानुवादेना \textendash\ 

प्रयुक्तशब्दानां साधुत्वबोधनार्थं शास्त्रमावश्यकमिति न तस्य नियाम \textendash\ 

 \textendash\ कस्वमिस्युच्यते~। अनेन च तन्निसक्रियते \textendash\ अर्थज्ञानाय शब्दप्रयोगस्या \textendash\ 

व्यकत्वेन भवताऽअयुक्तत्वेन प्रदार्शितानां शब्दानामर्थस्य सङ्भावात्त \textendash\ 





किं कारणम् ? 

{\qt अर्थ् शब्दप्रयोगात्} अर्थ शब्दाः प्रयुज्यन्ते~। 

सन्ति चैषां शब्दानामर्था येष्वर्थषु प्रयुज्यन्ते~॥ 

( प्रदीपः ) अर्थ शब्दप्रयोगादिति~। अर्थसद्भावः 

शब्दसद्भावे लिङ्गम्~। न हि विना शब्देनार्धप्रत्यायनमुपपद्यते~॥ 

( उद्दयोतः ) भाष्ये \textendash\ अर्थे शब्दप्रयोगादिति~। अर्थविषयक \textendash\ 

ज्ञानाय शब्दप्रयोगादित्यर्थः~॥ तदाशयमाह \textendash\ अर्थसद्भाव इतिं~॥ 

(३ बाधकांशमिरासवार्तिकम्~॥ २~॥ ) 

~॥ \#~॥ अप्रयोगः प्रैयोगान्यत्वात्~॥ \#~॥ 

( भाष्यम्) 

अप्रयोगः खल्वप्येषां शब्दानां न्याय्यः~। 

कुतः ? 

{\qt प्रयोगान्यत्वात्}~। यदेतैषां शब्दानामर्थेऽन्याञ्छ \textendash\ 

ब्दान्प्रयुञ्जते~। तद्यथा \textendash\ {\qt ऊष} इत्यस्य शब्दस्यार्थे \textendash\ 

क्व यूयमुषिताः, {\qt तेर} इत्यस्यार्थ \textendash\ क्व यूयं तीर्णाः, 

{\qt चक्र इत्यस्यार्थे \textendash\ क्व यूयं कृतवन्तः~। पेच} इत्य \textendash\ 

स्यार्थे \textendash\ क्व यूयं पक्कवन्त इति~॥ 

( प्रदीपः ) इतरोऽन्यथासिद्धताआह \textendash\ अप्रयोग इति~। 

यतोऽन्ये तेषामर्थीनां सन्ति वाचकास्ततो नैषामनुमानमुपप \textendash\ 

द्यते~। यद्यपि {\qt अष} इत्यस्य {\qt उषिताः} इति समानार्थो न भवति, 

परोक्षतादेर्विशेषस्यानवगमात्, तथापि तत्प्रत्यायनाय पदान्तर \textendash\ 

सहितः प्रयुज्यते~॥ 

( उद्दयोतः) भाष्ये \textendash\ प्रयोगान्यत्वादिति~। प्रयुज्यत इति \textendash\ 

प्रयोगः$=$शब्दः, सोऽन्यो यस्यार्थस्पास्ति तत्त्वादित्यर्थः~। सर्वनाम्नः 

परनिपातः, पूर्वनिपातप्रकरणानित्यत्वात्~। पदान्तरसहित इति~। 

पदान्तरं \textendash\ क्व यूयमिति~। तत्र {\qt क्व} इत्यनेन साधनपारोक्ष्यं, निष्ठया 



द्वाचकाः शब्दा अपि सन्स्येवेति न तदर्थ शास्त्रमावश्यकमिति शास्त्ररस्य 

नियमधरत्वोपपत्तिरिति~॥ 

१२ सन्तीति~। सर्वथा मूरूप्रयोगं विना न शास्त्रोपदेशः~। 

यद्येषामसन्तोर्थास्तहिं शम्दार्थयोरप्रसिद्धा शशशुङ्गादितोऽपि
विप्रकृष्ट \textendash\ 

तरत्वं स्यात्~। तस्मादर्थाः कालादयः सन्ति~। न च तेषां प्रतीतिः 

शब्दं विना संभवति~। अत दव वक्ष्यते \textendash\ शष्दपूर्वको ह्यर्थे संप्र \textendash\ 

त्ययः इति~। तस्मादर्थं सति शब्दस्याप्रयोगो दुरवधारण इत्यर्थः~॥ 

छाया~॥ 

१३ {\qt शब्दप्रयोगे लिङ्गम्} इति मुद्रितपाठः~। 

१४ प्रयोगान्यत्वादिति~। ऊषतेरेत्यादिशब्दवाच्यस्यार्थस्य 

सद्भावादूषादीनां प्रयोगोऽनुमीयत इत्यपि न, तेषां पर्यायशब्दसद्भावेन 

तैरेव व्यवहारे तथानुमानासंभवात्~। एवचाप्रयुक्तानां सिद्धर्य शास्त्र \textendash\ 

मावशयकमिति न नियमार्थतोपपत्तिरित्यर्थः~॥ 

१५ अन्यथासिद्धतामिंति~॥ अर्थप्रतीतेरिति भावः~॥ एवं चो \textendash\ 

षादीनामर्थसद्भावात्प्रयोगसद्भावानुमानं न सिध्यति, तत्प्रतीतेरन्यथा \textendash\ 

प्युपपत्तेरिति भावः~॥ छाया~॥ 

१६ {\qt सर्वनाम्नां इति मुद्रितपाठः~।}

७० 

उद्दयोतपरिवृतप्रदीपप्रकाशितमहाभाष्यम्~। [१अ. १पा. १आ. पस्पशाह्रिके 



भूतत्वं कर्तृत्वं च, यूर्य इत्यनेन मध्यमपुरुषबोध्यमौभिमुख्यं बहु \textendash\ 

त्वञ्च, चेतनत्वं वा बहुत्वं च~। वसेः साक्षात् {\qt गत्यर्थाकर्मक \textendash\ }

इत्यत्र निर्देशात् , तरतेर्गत्यर्थत्वात् कर्तरि क्तः~॥ नैन्वाख्याते
क्रिया \textendash\ 

विशेष्यको बोधः, अत्र कर्तृविशेष्यकः \textendash\ इति कथं लिडन्तसमानार्थत्वम् ? 

इति चेत् , न~। विषयताविशेषैनादरेण तथोक्तेः~। न चाख्याते 

क्रियाविशेष्यकबोधे पक्ष्यति भवति  {\qt अपाक्षीद्भवति} इति भूवादि \textendash\ 

सूत्रस्थभाष्यप्रयोगविरोधः, अतीतानागतपाककर्तृकसत्ताया वर्तमान \textendash\ 

त्वासङ्गतेरिति वाच्यम्~। वर्तमानसामीप्ये तत्र लटः सत्त्वात्~। 

{\qt तरति ब्रह्महत्यां योऽश्वमेधेन यजते} इत्यादौ यत्कर्तृको याग \textendash\ 

स्तत्कर्तृकं तरणमिति बोधः~। अत एव {\qt हस्वो नपुंसके \textendash\ } इते सूत्र 

{\qt रमते ब्राह्मणकुलमिति क्रियाप्रधानं} इति भाष्ये उक्तम्~॥ 

(४ बाधकोपपत्तिवार्तिकम्~॥ ३~। ) 

॥~॥~अप्रयुक्ते दीर्घसत्रवत्~॥~॥

(भाष्यम् ) 

यद्यप्यप्रयुक्ताः, अवर्व्यं दीर्धसत्रवल्लक्षणेनानु \textendash\ 

विधेयाः~। तंद्यथा \textendash\ दीर्धसतत्राणि वार्षशतिकानि वार्ष \textendash\ 

सहंस्त्रिकाणि च, न चाद्यत्वे कश्चिदप्याहरति~। केव \textendash\ 

लमृषिसंप्रदायो धर्म इति कृत्वा याशिकाः शास्त्रे \textendash\ 

णाजुविद्धते~॥ 

(प्रदीपः ) संप्रत्यप्रयुज्यमानानामपि पूर्वं प्रयुक्तत्वादनुशासनं 

कर्तव्यमित्याह \textendash\ अप्रयुक्त इति~॥ ऋषिसंप्रदाय इति~। 

वेदाध्ययनमित्यर्थः~॥ 

( उद्दयोतः ) नन्वप्रयुक्तानुशासने निर्मूलत्वाच्छास्त्रस्याप्रामाण्यं 



१ {\qt माभिमुख्यं चेतनत्वं वा बहुत्वम्} इति घ. पाठः~। 

२ एतद्वार्तिकभाष्यसंमतत्वादाख्यातार्थविशेष्यक एव शाब्दबोधो 

न धात्वर्थविशेष्यक इति तथा वदन्तो वैयाकरणमन्या अयुक्ता एवेति 

रत्नोक्त्यसांगत्यं ध्वन्यन्नाह \textendash\ नन्वाख्यात इति~॥ छाया~॥ 

३ नादरेणेति~। तथा च तत्समानविषयकत्वात्तत्त्वम् , न तु 

तत्समानविशेष्यकत्वादिनेति भावः~॥ छाया~॥ 

४ यथा दीर्घसत्राणि कश्चिदपि नेदानीमाहरति, केवलं वेद \textendash\ 

प्रोक्तानि तान्यध्ययनविषयाणि, तथा ऊषतेरेत्यादीनामिदानीं प्रयोगा \textendash\ 

भावेऽपि पूर्वं प्रयोगविधयास्ते साधवः शास्त्रेण साधनीया इति तदर्थ 

शास्त्रमावश्यकमिति न शास्त्रस्य नियमार्थतोपपत्तिरित्यर्थः~॥ 

५ अवश्यमिति~। तथापीत्यादिः~॥ छाया~॥ 

६ {\qt समा मासा अहोरात्रास्तुल्या ब्राह्मणचोदिताः} 

इति गृह्यसंग्रहोक्तिः आदित्यो वा सर्व ऋतवः स यदैवो \textendash\ 

देत्यथ वसन्तो यदा सङ्गवोथ आओष्मो यदा मध्यन्दिनोथ 

वर्षा यदाऽपराह्णोथ शरत् , यदाऽस्तमेत्यथ हेमन्तशिशिरौः 

इति ब्राह्मणमूलिकान्न प्रमाणम्~। अत एव सहस्संवत्सरं तदायु \textendash\ 

षामसंभवान्मनुष्येषु? इति पूर्वपक्षे {\qt अहानि वाभिसंख्य} \textendash\ 

त्चात् ( पू. मी. ६~। ७~। ३ १ \textendash\ ४० ) इति जैमिनिनापि सिद्धान्तितं 

संगच्छते~॥ इति दाधिमथाः~॥ 

७ {\qt भाष्यस्थःकषि}पदस्य वेदपरत्वे प्रमाणमुपपादयति \textendash\ ननु 

सश्नाणामिति~॥ 





स्यादत आह \textendash\ संप्रतीति~। पाणिनेर्व्याकरणप्रणयनकाले इ \textendash\ 

त्यर्थः~। पूर्वप्रयोगसत्ता च व्याकरणप्रणयनादेवानुमीयत इति 

भावः~। इदमपि यत्र भाष्यवारतिककाराभ्यामप्रयुक्तत्वं नोक्तं तदभि \textendash\ 

प्रायकमनुमानं बोध्यम्~॥ (भाष्ये ) वार्षशतिकानीति~। वर्षशब्दो 

दिर्वसपरः, यथाश्चुत एव वा~॥ आहरति~। सामग्र्यभावादिति 

भावः~॥ ननु सत्राणां बहुयजमानकानामृषिभिर्वसिंष्ठादिभिः 

संप्रदीयमानत्वाभावादृषिसंप्रदाय इत्ययुक्तमत आह \textendash\ वेदेति~॥ 

भाष्ये {\qt कृत्वा} इत्यनन्तरं अधीयते इति शेषः~। याशिकाश्च कल्प \textendash\ 

सूत्रेणान्वाचक्षत इत्यर्थः~। अत्रं सत्राणां वर्मत्वम्~॥ 

(५ सिद्धान्तसमाधानवर्तिकम् ~॥ ४ ) 

~॥ *~॥ संर्वे देशान्तरे~॥~॥ 

(भाष्यम्) 

सर्वे खल्वप्येते शब्दा देशान्तरेषु प्रयुज्यन्ते~॥ 

( प्रदीपः ) सर्वं इति~। इदमत्र तात्पर्यम् \textendash\ यस्य कस्य \textendash\ 

चिद्वचनात्प्रयोगाप्रयोगौ न व्यवतिष्वेतैे, अपि तु शिष्टानामेव 

वचनात्~॥ 

(आक्षेपभाष्यम् ) 

न चैवोपलभ्यन्ते~॥ 

( अद्दयोतः ) अनुपैलब्ध्याऽप्रयुक्तत्वं शङ्केते \textendash\ भाष्ये \textendash\ न च \textendash\ 

वेति~॥ 

(समाधानभाष्यम् ) 

उपलब्धौ यत्नः कियताम्~। महान शब्दस्य 

प्रयोगविषयः~। सप्तद्वीपा वसुमती, अत्रो लोकाः, 

८ अत्र \textendash\ अनुविदधनक्रियायाम् 

९ इदानीं समयेऽपि प्राक्प्रयोगा लक्षणेनानुमेया इत्युक्तं प्राक्~। 

संप्रति {\qt प्रयोगोऽपि तेषां कचिदस्ति} इत्याह \textendash\ सर्वै इति~॥ छाया~॥ 

१० देशान्तर इति~। अनेन वार्तिकेन ये भवताऽप्रयुक्ता इत्यु \textendash\ 

च्यन्ते तेषामपि देशान्तरे प्रयोगसद्भावेन प्रयोगबिषयत्वात्साधुत्वं तेषां


स्यादिति शास्त्रेण धर्मनियम एव क्रियत इति सिद्धान्तः प्रतिपाद्यते~। 

एवञ्च शास्त्रेण प्रयोगनियमेऽपि तेनैव हेतुना साधुत्वमप्येषामनुमेय \textendash\ 

मिति साधुत्वसम्पादकत्वमपि शास्त्रस्योपपद्यत इति भावः~॥ 

११ एवं चैकत्र देशेऽप्रयुक्तत्वमात्रेण शास्त्राविषयत्वं नेति भावः~॥ 

छाया~॥ 

१२ अनुपलब्ध्येति~। योग्यानुपलब्ध्येत्यर्थः यदि ते प्रयुज्यन्ते 

तह्र्ययवश्यं तेषामुपलब्ध्या भवितव्यं योग्यत्वात्, यतो नोपलभ्यन्तेऽ \textendash\ 

स्माभिरतस्तेऽप्रयुक्ता इति भावः~॥ छाया~॥ 

१३ शङ्कत इति~। आक्षेपक एवेति भावः~॥ छाया~॥ 

१४ समाधत्ते \textendash\ उपेति~। योगाभ्यासः कर्तव्य इति भावः~॥ 

यद्वा अन्यदेशस्थप्रयोगस्य
तत्त्वेनायोग्यत्वादनुपलब्धेरयोग्यानुपलब्धेश्चा \textendash\ 

भावासाधकत्वान्न तत्त्वम्~। किं त्वन्यदेशस्थेनापि तद्देशऽगमनं तत 

आगतेभ्यः श्रवणं च कार्यमिल्यर्थः~॥ छाया~॥ 

१५ सप्तद्वीपात्मकवसुमत्येव पारं गन्तुमशक्या का कथा लोका \textendash\ 

न्तरपारगमनस्येत्याह \textendash\ त्रयो लोका इति~॥ छाया~॥ 

शब्दविषयप्रदर्शनाधिकरणम् ] 

पस्पशाह्निकम्~। 



चत्वारो वेदाः सङ्गः सरहस्या बहुधा भिन्नाः

एकशतमध्वर्युशाखाः, सहस्रवर्त्मा सामवेदः, 

एकर्विशतिधा वाह्णच्यं, नवधाऽऽर्थर्वणो वेदः, 

बाकोवाक्यम्, इतिहासः, पुराणं, वैर्द्यकमित्येतावा \textendash\ 

अछध्दस्य प्रयोगविषयः~। एतावन्तं शब्दस्य प्रयोग \textendash\ 

विषयमननुनिशम्य {\qt सन्त्यप्रयुक्ता }इति वचनं 

केवलं साहसंमात्रमेव~॥ 

( प्रदीपः ) वाकोवाक्यशब्देनोक्तिप्रत्युक्तिरूपो ग्रन्थ 

उच्यते~। यथा \textendash\ {\qt किंखिदावपनं महत् भूमिरावपनं 

महत} इति~। पूर्वचरितसंकीर्तनं \textendash\ इतिहासः~। वंशाद्यनु \textendash\ 

कीर्तनं \textendash\ पुराणम्~॥ 

(उद्दयोतः ) इतरलोकापेक्षया वसुमत्यां शब्दबाहुल्याच्तस्या: 

पुथग्ग्रहणम्~। अनेनैन्त्यपातालपर्यन्तसंग्रहः~। त्रयो लोकाः \textendash\ स्वस्त्वे \textendash\ 

नोपरितनसकलसंग्रहः~। रहस्यम् \textendash\ उपनिषत्~। मन्वादिस्मृतयो वा, 

वेदनिर्गूढार्थप्रकाशकत्वात्~। एवं च देशान्तरशब्दः शात्लान्तरवेदा \textendash\ 

न्तरोपलक्षक इति बोध्यम्~॥ 

१ अत्र वेदशब्देनैव मत्रभागस्येव बाह्मणभागस्यापि ग्रहणम् 

{\qt मन्रब्राह्मणयोर्वेदनामधेयम्} इति बहुभिरापस्तम्बादिभिरुक्तत्वात्~। 

अत एव {\qt आास्नायस्य क्रियार्थत्वादानर्थक्यमतदर्थानाम्} इति 

सुज्रे जैमानिमुनेरपि ब्राह्मणभागस्याम्नायपदेनाभिधानं संगच्छते~। न्याय \textendash\ 


भाष्यकर्तुर्वात्स्यायनस्यापि विभागश्च व्राह्मणवाक्यानां न्निविधः 

इत्युपक्रम्य \textendash\ {\qt एवं वेदवाक्यानामपि विभागेनार्थग्रहणात्प्रमा \textendash\ 

णत्वं भषितुमर्हति} इत्युपसंहारात्स्पष्टमेव मन्रब्राह्मणयोर्येदशब्दा \textendash\ 

भिधेयत्वं संमतम्~। {\qt उदितेऽनुदिते चैव} इति भनुस्मृतावपि ब्राह्मण \textendash\ 

भागस्य वेदत्वोक्तिः स्पष्टा~। छन्दोब्राह्मणानि च तद्विषयाणि? 

इत्यत्र तु {\qt नामाख्यातोपसर्गनिपाताश्व}? इति भाष्यनिरुक्तवाक्ये 

गोबलीवर्दन्यायेन विशेषसमवधाने सामान्यपदस्य तद्विशेषातिरिक्तपर \textendash\ 

त्वाङ्गीकारेण निपोतपदस्योपसर्गातिरिक्तपरत्वस्येव छन्दःपदस्य ब्राह्मणा \textendash\ 


तिरिक्तपरत्वाङ्गीकारे न कोपि दोष इति दिक्~॥ इति दाधिमथाः~॥ 

२ बह्नुचानामाम्नायो बाह्नच्यम्~। छन्दोगौक्थिकयाज्ञिकबह्य \textendash\ 

धनटाव्यः इति ज्यः प्रत्पय इति~॥ श० कौ०~॥ 

३ अथर्वणा ऋषिणा प्रोक्तो वेद आथर्वणः {\qt अन्} इति प्रकृति. 

भावः~। तमधीयते आधथर्वणिकाः~। वसन्तादित्वाट्टक्~। तेषामास्चाय 

आथर्वणः {\qt आथर्वणिकस्य \textendash\ } इत्यण् इकलोपक्चेत्यर्थः~॥ छाया~॥ 

४ वैद्यकमिति~। चरकादीत्यर्थः~। वसन्तराजादिप्रणीताः शकुना \textendash\ 

यागमा अपि ज्योतिःशास्त्रैव वेदाङ्गस्य शेषभूताः~। कल्पसूत्रादयो 

माथाधनुर्वेदगान्धर्ववेदार्थशास्त्राद्य उपवेदाश्च यथायथं चतुर्दशविद्या \textendash\ 


स्थानेष्वेवान्तर्मवन्तीति न पृथगुक्ताः~। पाखण्डाद्यागमाः काव्यनाढ \textendash\ 

कादयश्च नात्मन्तं वर्जितापशब्दा श्ति न पृथर्गणिताः~॥ वस्तु \textendash\ 

तस्तु \textendash\ भाष्ये न परिगणनं किंतु दिक्प्रदर्शनम्~। प्राधान्याद्वेदप्रपञ्च \textendash\ 

स्पोरक्तिरिति. न कोऽपि दोष इति बोध्यम्~॥ छाया~॥ अत्र वैद्यकम्~। 

इत्युक्त्या मन्नायुेदप्रामाण्यवश्च तत्प्रामाण्यमाप्तप्रामा \textendash\ 
 ~। 





(शब्दानां नियतविषयकत्वदर्शकभाष्यम् ) 

एतरस्मिश्चातिमहति शब्दस्य प्रयोगविर्षयेते ते 

शब्दास्तत्र तत्र नियतविषया दृश्यन्ते~। तद्यथा \textendash\ 

शवतिर्गतिकर्मा कम्बोजेष्वेव भौषितो भवति, 

विकार एनमार्या भाषन्ते \textendash\ शव इति~। हम्मतिः 

सुराष्ट्रेषु, रहतिः प्राच्यमध्येषु, गमिमेव त्वार्याः 

प्रयुञ्जते~। दातिर्लवनार्थ प्राच्येषु, दात्रमुदीच्येषु~॥ 

(प्रदीपः ) विकार इति~। जीवतो मृतावस्था \textendash\ विकारः, 

तत्रेत्यर्थः~॥ 

(उद्दयोतः ) भाष्ये \textendash\ हम्मतिरिष्यत्र रंहतिरित्यत्र च {\qt गति \textendash\ 

कर्मा} इत्यनुकर्षः~॥ दातिर्ल्लवनार्थ इति~। क्तिन्नन्तं क्तिजन्तं वा~। 


लवनेऽर्थं दातिशब्दं प्राच्याः प्रयुञ्जते, तत्राथ दात्रशब्दमुदीच्या 

इत्यर्थः~॥ 

( अप्रयुक्तशब्दीपलम्भकभाष्यम्) 

ये चाप्येते भवतोऽप्रयुक्ता अभिमताः शब्दा 

एतेषामपि प्रयोगो दृशयते~॥ 



ण्यात् हइति (२~। १~। ६७ ) सूत्रे {\qt दृष्टार्थेनाप्तोपदेशेनायुर्वेदे \textendash\ 

नाटृष्टार्थो वेदभागोऽनुमातव्यः प्रमाणम् इति} य एवाप्ता 

चेदार्थानां द्रष्टारः प्रवक्तारश्च त एवायुर्वेदप्रभृतीनाभ् \textendash\ 

इति न्यायभाष्यकारोक्तदिशा वेदैकगम्ये धरमे दृष्टबिरोधकुतर्कानवकाश 

इति ध्वनितम्~॥ इति दाधिमथाः~॥ 

५ {\qt साहसमात्रं इत्येव प. पाठः~॥} 

६ तस्यार्थमाह \textendash\ भूरिति~॥ नन्वनेनैव वसुमत्या ग्रहणं सिद्ध \textendash\ 

मत आह \textendash\ इतरेति~॥ वस्त्विति~। सप्तद्वीपवत्यामित्यर्थः~। तथा च 

तत्र तद्वाहुल्यसूचननायैव पृथग्ग्रहणमिति भावः~। भाष्यीयन्यूनतां परि \textendash\ 

हरति \textendash\ भुवरित्येत्यादिना~॥ स्वसिति~। स्वरित्यनेनेत्यर्थः~॥ 

छाया~॥ 

७ अनेन \textendash\ सप्तद्वीपसहितवसुमत्याः पृथग्ग्रहणेन~॥ 

८ त्रयो लोकाः \textendash\ अनेन प्रसिद्धानां भूर्भुवस्वरित्येषां ग्रहणम्~। अत्रे 

छायादृष्टः पाठोऽन्यविध इति तत्रत्यप्रतीकदर्शनेनावसीयते~। तदर्थ 

दाधिमथैरीदृशः पाठः कल्पितः \textendash\ {\qt त्रयो लोका इति~। भूर्भुवःस्वरा \textendash\ 

त्मका इत्यथैः~। इतरलोकापेक्षया वसुमत्या शब्दबाहुल्यात्तस्या: 

पृथग्ग्रहणम्~। भुवारत्यननान्त्यपातालपर्यन्तसंग्रहः~। स्वस्त्वेनोपरि \textendash\ 

तन \textendash\ } इति~। वस्तुतः प्रामाणिकप्राचीनपुस्तकेषूपलब्धः पाठों 

नासङ्गत इति त्रुटितछायाग्रन्थानुरोधेन पाठकल्पनानुसन्धानं नोचि \textendash\ 

तम्~। छायादृष्टपाठोपलब्धौ च यत्नः क्रियताम्~॥ 

१ वेदनिगृढेति~। स्पष्टं चेदं स्मृतिप्रामाण्याधिकरणे ( पू. मी. 

१~। ३~। ६ ) इति भावः~॥ छाया~॥ 

१० संर्बे देशान्तरे इति वार्तिकघटकदेशान्तरशब्दार्थमाह \textendash\ ए्वं 

श्वेति~॥ 

११ विषये ते शब्दाः इत्येव प. पाठः~॥ 

१२ भाषिष्यते इति प. पाठः~॥ 

७२ 

उद्दयोतपरिवृतप्रदीपप्रकाशितमहाभाष्यम्~। [ १अ. १ पा. १ आ. पस्पशाह्निके




क्व ? 

वेदे~। तद्यथा \textendash\ {\qt यद्वो रेवर्तीरेवत्यन्तमूप, यन्मे 

नरः श्रुत्यं ब्रह्म चक्र, यत्रा नश्चक्रां जरसं तनूनाम्} 

इति शब्दविषयप्रदर्शनाविकरणम्~॥ 

(अथ शब्दज्ञानस्य धर्मजनकताधिकरणम् ) 

( आक्षेपभाष्यम् ) 

किं पुनः \textendash\ शब्दस्य ज्ञाने धर्मः, आहोस्वित्प्रयोगे ? 

(प्रदीपः ) किं पुनरिति~। एकः शब्दः सम्यक् 

ज्ञातः सुप्रयुक्तः स्वर्गे लोके कामधुग्भवति(९~। १~। ८४) 

इति श्रुतिः~। तत्र किं सम्यक् ज्ञातः कामधुर्भवति, सुप्रयोगात्तु 

सम्यग्ज्ञातत्वानुभानमित्यर्थः; आहोस्वित्सुप्रयुक्तः कामधुक् भवति, 

सुप्रयुक्तत्वं सम्यग्ज्ञानादित्यर्थ इति प्रश्नः~॥ 

(उद्दयोतः ) वैर्तिकोक्तो धर्मनियमः श्रुतिसिद्धस्तस्यामुभयो: 

श्रवणात्प्रश्न इत्याह \textendash\ एकः शब्द इति~। तैत्र {\qt यदैकस्मादपूर्वं 

तदितरत्तदर्थम्} इति न्यायात् द्वयोः साम्येन फलसंबन्धाभावा \textendash\ 

द्विकल्पायोगेनैकस्य प्राधान्यम् , अपरस्याङ्गत्वं कल्प्यम्~॥ तत्र 

{\qt प्रथमं वा नियम्येत \textendash\ }? (पू० मी० ११~। १~। ४३ ) इति न्याया \textendash\ 

श्रयेणाह \textendash\ तन्र किमिति~॥ 

संर्वत्रैव हि विज्ञानं संस्कारत्येन गम्यते~। 

पराङ्गं चात्मविज्ञानादन्यन्रेत्यवधार्यताम्~॥ इति \textendash\ 

न्यायेन ज्ञानस्य शेयषतया धर्महेतुत्वाभावात् , प्रयोर्गङ्गतया दृष्टा \textendash\ 

थत्वाच्च, प्रयोगस्य फलं प्रति संनिहितत्वाच्चाह \textendash\ ाहो स्बिदिति~॥ 

( प्रत्याक्षेपभाष्यम् ) 

कश्चात्र विशेषः ? 



१ अस्मात् पूर्वं सप्तास्येरेवतीरेवदूष इति मुद्रितपाठः~। 

२ ज्ञाने धर्मः प्रयोगे वेतिपक्षद्वयोपस्थितौ कारणमाह \textendash\ वार्ति \textendash\ 

कोक्त इति~। तस्यामुभयोः \textendash\ एकः शन्देत्यादिश्रुतौ {\qt सम्यग्ज्ञातः 

सुप्रयुक्तः} इत्युभयोरित्यर्थः~॥ 

३ ननु श्रुतौ ज्ञानप्रयोगयोरुभयोः श्रवणात्तयोरुभयोरैव धर्मत्वं 

स्यान्न त्वेकैकस्येति पक्षद्वयानुपपत्तिरत आह \textendash\ सन्र यदैकस्मादिति~॥ 

द्वितीयपक्षाश्रयणे बीजमाह \textendash\ सर्वत्रैव द्वीत्यादिना~। {\qt संस्का \textendash\ 

रित्वेन रम्यते} इति मुद्रितमुस्तकेषु छायायाञ्च पाठ उपलभ्यते~। 

एतञ्च तत्रवार्तिकं प्रथमाध्यायतृतीयपादस्याष्टमाधिकरणे उक्तम्~। 

तत्रवार्तिके प्रामाणिकोद्दयोतपुस्तकेषु च {\qt संस्कारत्वेन} इत्येव पाठ उप \textendash\ 


लभ्यते~। संस्कारित्वेनेति पाठे मत्वर्थोत्तरभावप्रत्ययेन सम्बन्धाभि \textendash\ 

धानमि्यभियुक्तव्यवहारात् संस्कारसम्बन्धेन ज्ञायत इत्येवार्धः, न तु 

संस्कारजनकत्वेनेत्यर्थः~। शब्दस्य ज्ञानमेव संस्कारो न तु तदति \textendash\ 

रिक्तः क्रश्चिज्शानजन्यः संस्कार उपलभ्यते~॥ 





(६ ज्ञानपक्षदूषणवार्तिकम्~॥ १~। ) 

~॥ \#~॥ ज्ञाने धर्म इति चेत्तथाऽधर्मः~॥~॥ 

(भाष्यम्) 

ज्ञाने धर्म इति चेत्तथाऽर्धर्मः प्राप्नोति~। यो 

हि शब्दान् जानाति, अपशब्दानप्यसौ जानाति~। 

यथैव शब्दज्ञाने धर्मः, एवमपशब्दज्ञानेऽप्यधर्मः~॥ 

( प्रदीपः ) ज्ञाने धर्म इति चेदिति~। यथा श्लेष्मणः 

प्रकं पनं स्नेहद्रव्यं, रूक्षं वायोः, तथेहापि प्राप्तमिति भावः~॥ 

( उद्दयोतः ) यथा श्लेष्मण इति~। तथा धर्मजनकशानविप \textendash\ 

रीतत्वात् तैद्विषरीतजनकत्वेनापशब्दशानादधर्मप्राप्तिरित्यथः~॥ 

(अधर्माधिक्थदर्शनभाष्यम् ) 

अर्थवा भूयानधर्मः प्राप्नोति~। भूयांसौ ह्यप \textendash\ 

शब्दाः, अल्पीयांसः शब्दाः~। एकैकस्य शब्दस्य 

वहवीऽपभ्रंशाः~। तद्यथा \textendash\ गौरित्यस्य गावी \textendash\ गोणी \textendash\ 

गोत्ताःगोपोतलिकेस्येवमादयोऽपभ्रंशाः~॥ 

(७ ज्ञानपक्षाभावबोधकं वार्तिकम्~। २~॥ ) 

~॥ *~॥ आचारे नियमः~॥~॥ 

(भाष्यम् ) 

आचारे पुनर्वषिर्नियमं वेदयते \textendash\ {\qt तेऽसुरा हे \textendash\ 

लयो हेलय इति कुर्वन्तः पराबभूवुः} इति~॥ 

( प्रदीपः ) आचार इति~। प्रयोगे~॥ ऋषिः \textendash\ वेदः~॥ 

(उद्दयोतः ) प्रयोगे इति~। एवं च प्रयोगादेवाधर्मवद्धर्मोऽपीति 

ज्ञानाद्धर्मः इति वेदविरुद्धमिति भावः~॥ 

इति ज्ञानपक्षनिराकरणम्~॥ 

( अथ प्रयोगपक्षाङ्गीकारभाष्यम् ) 

अस्तु तर्हि प्रयोगे~॥ 

५ {\qt सम्भवतिवृष्टफलकत्वे} इति न्यायेनाह \textendash\ प्रयोगाङ्गेसि~। 

ज्ञानस्य प्रयोग सव दृष्टम्फलं सम्भवति, अतोऽवृष्टम्फलं {\qt स्वर्गे लोके} 

इत्यादिकं कल्पयितुभशक्यमिति भावः~। अत्र गुरुप्रसादो राजलक्ष्मीं 

प्रयोगाङ्गतयेति \textendash\ प्रयोगाङ्गतयैवाद्वष्टजनकतयेत्यथः? इति व्याख्या \textendash\ 

नेन चिभूषथन् मीमांसाकौशलम्प्रदर्शयति~॥ 

६ सम्निधिपारूपस्थानप्रमाणेनापि प्रयोगस्यैव फलेऽन्वयो न 

ज्ञानस्मेत्याह \textendash\ प्रयोगस्य फलमिति~॥ 

७ तरद्विषेति~। धर्मविषेत्यर्थः~। तथा च शास्त्रादेव शब्दज्ञान \textendash\ 

नान्तरीयकतया जायमानापशब्दज्ञानस्य तुल्यन्यायेनाधर्महेतुतायत्त्या \textendash\ 

ऽनुपादेयत्वं शास्त्रस्य स्यादिति भावः~॥~। छया~॥ 

८ अनर्थभूयस्त्वाद्विषसंप्रयुक्तमध्वन्नवद्धर्ज्यमेव व्याकरणमित्याह \textendash\ 

अभवेति~॥ अयं साधुरयमसाधुरिति शानं स्वरूपनिर्देशं विना म 

संभवतीति सम्यक्प्रयुक्त इत्येतदत्र मते निर्देशार्थं स्यादिति भारः 

~॥ छाया~। 

शब्दज्ञानस्य धर्मजनकताधिकरणम् ] महाभाष्यप्रदीपोद्दयोतव्याख्या छाया~। 


(८ प्रयोगपक्षे दूषणवार्तिकम्~॥ ३~॥ ) 

~॥~॥ प्रयोगे सर्वलोकस्य~॥~॥ 

(भाष्यम् ) 

यदि प्रयोगे धर्मः, सँवी लोकोऽभ्युदयेन 

युज्येत~॥ 

(आक्षेपभाष्यम् ) 

कैश्चेदानीं भवतो मत्सरः, यदि सर्वो लोको \textendash\ 

ऽभ्युदयेन युज्येत ? 

(समाधानभाष्यम् ) 

र्ने खल कश्चिन्मत्सरः~। प्रयैल्लानर्थक्यं तु 

भवति~। फलवता च नाम यत्लेन भवितव्यम् 

न च प्रयत्नः फलाद्यतिरेच्यः~॥ 

( प्रदीपः ) न च प्रयल्ल इति~। यदि प्रयत्नेन विना फलं 

स्यात् प्रंयत्नवैयर्थ्यमापद्येतेत्यर्थः~॥ 

( उद्दयोतः ) भाष्ये \textendash\ फलाद्ययतिरेच्य इति~। फलवद्वृत्त्यभाव \textendash\ 

प्रतियौगी न कार्य इत्यर्थः~॥ प्रय्लवैयर्थ्यमिति~। व्याकरणाध्यय \textendash\ 

नविषयकप्रमलवैयर्थ्यमित्यर्थः~॥ 

(समाधानबाधकभाष्यम् ) 

ननु च ये कृतप्रयत्नास्ते साधीयः शब्दान्प्रयो \textendash\ 

क्ष्यन्ते, जत एव साधीयोऽभ्युदयेन योक्ष्यन्ते~॥ 



१ {\qt प्रयोगे सर्वलोकस्य} इति वार्तिके धर्मं इति चेत् इत्यस्या \textendash\ 

नुषङ्गः~। {\qt धर्मः स्यात्} इति शेषः~॥ छाया~॥ 

२ सर्वो लोक इति~। अयं भावः \textendash\ {\qt सम्यग्ज्ञातः} इति नान्त \textendash\ 

रोयकशनानुवादः~। शास्त्रान्वित इत्यस्य च न शास्त्रशानपूर्वक इत्य \textendash\ 

र्थः~। किंतु शास्त्रव्युत्पादिसरूपमनतिक्रान्त इति~। एवं च शास्त्रकरण \textendash\ 

कृतश्रमोऽकृतश्रमोऽनधीतव्याकरणः सर्वोऽपि प्रयोगमात्राद्धमैण युज्ये \textendash\ 

तेति~॥ छाया~॥ 

३ ननु प्रमाणबलप्राप्ते किमस्माभिः कर्तु शक्यम्~। नहि परीक्षका \textendash\ 

पाम्प्रामाणिकार्थत्यागात् किंच्चिदनिष्टमित्याक्षिपति \textendash\ कश्चेदानी 

मिति~॥ छाया~॥

४ उत्तरयत्ति \textendash\ न खल्विति~॥ छाया~॥ 

५ प्रयत्नानेति~। वैयाकरणावैयाकरणकृतप्रयोगयोः समामफल. 

कत्वे ब्याकरणाध्ययनप्रयल्लानर्थक्यं स्यात्, अतः प्रयत्नस्यापि किश्चित्


फलमेषितव्यमित्यर्थकं फलवता च नाम यत्रेन इत्युक्तम्~। ननु 

{\qt एकः शब्दः सम्यक्} इति श्रुत्या प्रयोगस्यैव फलवत्तोच्यते, न 

प्रयत्नस्य फलसम्बन्धसम्भावनाऽपि कल्पयितुं शक्येत्यत आह \textendash\ न च 

प्रयत्न इति~। व्यतिरेच्य इति {\qt रिच \textendash\ वियोजने} कर्मणि ण्यत्, तथा 

च त्वया फलावधिकं प्रयलस्य वियोजनं न कार्यमित्यर्थः सम्पंद्यते~। 

फलप्रयलयोश्च सम्बन्धः श्रुत्या नोक्तो नापि लोके प्रसिद्ध इति तयो \textendash\ 

र्वियोजनं सिद्धमेवेति पुनः प्रतिपादनात्फलपदं फलवति लाक्षणिकम् ~॥ 

तथाच फलवदवधिकं$=$प्रयोगावधिकं प्रयलस्य वियोजनं न कार्यमित्यर्थः~। 

तदेतत् भाष्याक्षरानुगुणमेव {\qt फलवदब्बृत्त्यभावप्रतियोगी न कार्यः} इति 

ब्याख्यातमुद्दयोते~। प्रयोगस्य फल्वत्वं श्रुत्या प्रतिपाद्यते, तत्र च 

व्याकरणाध्ययनप्रयल्लविशिष्टः प्रयोग एव फलवान् भवतीति भाष्याभि \textendash\ 

प्रायः एवान्न पूर्वोत्तरवाक्ययोः पृथगर्थत्वातपुनृक्तिर्नेत्यवधेयम् 

१ प्र०पा० 





७३ 

( उद्दयोतः ) साधीय इति~। क्रियाविशेषणम्~। शास्त्रसंस्कार \textendash\ 

वतां प्रभूतश्नब्दप्रवोगजन्यबह्रपूर्वद्वारा
बहुतरार्भ्युदययुक्तत्वमित्यर्थः~॥ 

( समाधानसाधकभाष्यम्ं ) 

व्यतिरेकोऽपि वै लक्ष्यते~। दृश्यन्ते हि कृतंप्रय \textendash\ 

ताश्चाप्रवीणाः, अकृतप्रयत्लाश्च प्रवीणाः~। तत्र 

फलब्यतिरेकोऽपि स्यात्~॥ 

( प्रदीपः ) व्यतिरेक इति~। परिहासः~॥ 

(उद्दयोतः ) भाष्ये \textendash\ तन्न फलब्यतिरेक इति~। कौरल \textendash\ 

ब्यतिरेकवत् फलव्यतिरेकोऽपि स्यादितिं तस्य व्याकरणाध्ययनं 

व्यर्थमेव स्यादिति भावः~॥~। 

( पक्षद्वयनिराकरणोपसंहारवार्तिकावतरणभाष्यम् ) 

एवं तर्हि \textendash\ नापि ज्ञान एव धर्मः, नापि 

प्रयोग एव~। 

किं तर्हि 

(९ ध्वनिः शब्दुपक्षे सिद्धान्तवार्तिंकम्~। ४~॥ ) 

~। ~॥ शास्त्रपूर्वके प्रथोगेऽभ्युदयस्तत्तुल्यं 

वेदशाब्देन~॥~॥ 

(भाष्यम्) 

शस्त्रपूर्वकं यः शब्दान्प्रयुङ्क्ते सोऽभ्युदयेन 



६ {\qt प्रयल्स्य वैय \textendash\ } इति ख. पाठः~। 

1७७ {\qt त एव} इत्येव प. पाठः~॥ 

८ {\qt भ्युदयफलत्यं} इति पाठो मुद्रितपुस्तकेषु दृश्यते~। 

९ तथाहि \textendash\ केचदिप्रतिभयाः सत्त्वं निश्चेतुं न शक्ताः, निश्चितेऽपि 

सत्त्वे केचिन्च्यवहारप्रत्युत्पन्नमतित्वाभावान्न प्रयोगप्रौढाः~। अनधीत \textendash\ 


व्याकरणा अप्यन्ये परप्रत्ययावगतशब्दाः प्रयोगकुशला भवन्ति~। 

तथाच मुग्थवैयाकरणापेक्षया प्रौढोऽवैयाकरणः साधुशब्दान्प्रयुङ्के 

इति भावः~॥ छाया~॥ 

१ ० {\qt फलब्यतिरेकोऽपि} हति भाष्यस्थःअपि शदबललभ्यमर्थमाह \textendash\ 

कौशलेति~॥ 

११ एवं पक्षद्वये दोष उक्ते सिद्धान्तबातिकमवतारयति \textendash\ एवं 

तर्हिति~॥ नापीति~। नैवेत्यर्थः~। ततस्यैव समुच्चायको वा~॥ सर्वं 

वाक्यमितिन्यायेन प्रकृते तथेष्टमित्यह \textendash\ ज्ञान एवेति~॥ एवम \textendash\ 

ग्रेऽपि~॥ छाया~॥ ु ु 

१२ किं तर्हीति~। तदि कः सिद्धान्त इत्यर्थः शास्त्रान्धित 

इत्यस्यावलम्बेन~॥ छाया~॥ 

१३ तत्राह \textendash\ शास्त्रेति~। एकः शब्द इत्यादिश्रुतिप्रामाण्यादि \textendash\ 

त्यर्थः~॥ तत्रे सम्यग्ज्ञासः \textendash\ सम्यक्त्वेन साधुत्वेन शातः, शास्ना \textendash\ 

न्वितः \textendash\ शास्त्रब्युत्पादनमार्गेणानुसंहितः, सुप्रयुक्तः \textendash\ शिक्षोक्तरीत्या 

प्रयुक्त इत्यर्थो वोध्यः~॥ वेदेऽपि शास्त्रतो शानपूर्वकानुष्ठाने फलाति \textendash\ 


शयव्यबहारात् तच्छायानुकारिण्या अस्या अपि स्मृतेरेवं व्याख्यैव 

युक्तेत्याह \textendash\ तत्तुल्यमिति~॥ तदिति~। प्रयोगरूपमित्यर्थः~॥ छाया~॥ 

१ वार्तिकं व्याचष्टे \textendash\ शास्त्रेत्यादि~॥ ननु शानपूर्वकानुष्ठितेना \textendash\ 

न्निष्टोमादिना फलं भवतीत्यत्रैव किं मानमिति चेत्, वेद एवेलाशये \textendash\ 

नाह \textendash\ वेदशब्दा अप्येवमिति~। षष्ठीतत्पुरुषः कर्मधारयो वा~। 

अपिरेवार्थे~॥ एवमिति~। यथा मयोक्तं तथेत्यर्थः~॥ छाया~॥ 

७४ 

उद्दयोतपरिवृतप्रदीपप्रकाशितमहाभाष्ये \textendash\ 

[ १ अ. १ पा \textendash\ १ पस्पशाह्रिके 



युज्यते~। {\qt  तत्तुल्यं वेदशब्देन}~। वेदशब्दा अप्येवमभि \textendash\ 

वदन्ति \textendash\ {\qt यो५ग्निष्टोभेन यजते य उ चैनमेवं वेद},

योऽस्निं नाचिकेतं चिनुते य उ चैनमेवं वेद~॥ 

( प्रदीपः ) तत्तुल्यमिति~। वेदः शब्दो यस्यार्थस्य सः \textendash\ 

वेदशंब्दः, तस्य् यथो ज्ञात्वाऽनुष्टानं तथा शब्दानामपि प्रकृत्या \textendash\ 

दिविभागज्ञानपूर्वकः प्रयोग इत्यर्थः~॥ 

( उद्दयोतः ) शास्त्रपूर्वके इति~। व्याकरणाध्ययनपूर्वके 

इत्यर्थः~॥ तत्र वेदाङ्गत्वादेतद्व्याकरणाध्ययनपूर्वकप्रयोगे एव धर्म 

इति प्रामाणिकाः~॥ वेदः शब्द इति~। बोधक इत्यर्थः~। प्रमाण \textendash\ 

मित्यर्थो वा~॥ तस्य यथेति~। तस्य यथा {\qt य उ चैनमेवं वेद} इति 

ग्रन्थेन वेदतो ज्ञात्वाऽनुष्ठानं फंलायोक्तम् , तथाऽस्यापि शास्त्रज्ञान \textendash\ 


{\qt पूर्वकप्रयोगः फलायेति एकः शब्दः} इत्यादिनोक्तमित्यर्थः~॥ 



१शास्त्रान्तरनिरासायाह \textendash\ व्याकरणाध्ययनेति~। अयमा \textendash\ 

शयः \textendash\ द्वयोरन्यतरस्योक्तत्या तादथ्ये सति संदेहशानस्य ज्ञेयसंस्कार \textendash\ 

कत्वेनापि निराकाङ्कृस्य साध्यान्तराभावेनेतिकर्तव्यताकाङ्काविरहः

प्रयोगस्तु {\qt सुप्रयुक्तः स्वर्गे लोके} इत्यादिवाक्येन फलार्थतयाऽ \textendash\ 

वगतः प्रयोज्यकर्मीभूतशब्दसंस्कारमपेक्षमाणस्तत्र समभिव्याह्वतेन 

{\qt सम्यग्ज्ञातः} इत्यनेन तदङ्गतया ज्ञानस्य विनियोगे सति तस्य 

चान्यथाप्युपपत्तौ प्रागुक्तदोषापातात् शास्त्रान्वित इति विशेषण \textendash\ 

सामर्थ्याच्छास्त्रानुगतज्ञानस्य विनियोगे तादृशप्रयोगाद्धमो भवतीति न 

कोऽपि दोष इति~॥ तस्येति~। अग्निष्टोमादरित्यथः~॥ फलायोक्त \textendash\ 

मिति~। फलायेत्युक्तमित्यर्ः~॥ तथाऽस्यापि \textendash\ सुशब्दस्यापि~॥ ननु 

{\qt य उ चैनम्} इत्यत्रैव चकारादुभयोः फलसंबन्धो पुक्त इति प्रकृते \textendash\ 

ऽपि तथैवास्तु~। किं च {\qt तरति शोकमात्मवित्}  इत्यात्मशानविनि \textendash\ ~। 

युक्ततया शब्दज्ञानस्यैव फलसाधनताया युक्त वामेति चेत्, न~। तस्य 

समुच्चयार्थकत्वेऽपि गुणप्रधानभावपूर्वकसुवचतेत्याशयेन पक्षान्तर \textendash\ 

त्वात्~॥ ज्ञानेन चानुष्ठानस्य प्रत्यक्षमुपकारदर्शनात्~। ज्ञानस्य च 

{\qt यदेव विद्यया करोति तदेव वीर्यवत्तरं भवति} इति श्रुत्या 

सामर्थ्येन च पाराथ्यीवगमात्~। ज्ञानेनैव सिद्धेऽर्थं बह्वायाससाध्यानु \textendash\ 

ष्ठानवैयर्थ्यापत्तिः~॥ अत्यन्ताशेषभूते तु शाने स्वतद्रकल्पसाधनता, 

यथाऽऽत्मशानस्य शोकतरणोपलक्षितात्मस्वरूपं प्रति~। तदुक्तम् \textendash\ 

सर्वत्रैव हि विज्ञानं संस्कारित्वेन गम्यते~। 

पराङ्गं चात्मविशानादन्यत्रेत्यवधार्यताम्~। ? इति~॥ छाया~॥ 

२ वेदशब्देनेत्यत्र वेदशब्इतुल्यत्वस्यैव प्रतीतेरर्थतुल्यत्वपरत्वे 

लक्षणापत्तेर्वहुब्रीह्यङ्गीकारे पुरःस्फूर्तिकार्थत्यागः प्रतिपत्तिगौरवं
चेत्यस्व \textendash\ 

रसादाह \textendash\ अपर आहेति~। अस्य व्याख्या प्राग्वत्~। ब्दा निय \textendash\ 

मेति~। {\qt आहू तश्चाप्यधीर्यीत \textendash\ } इत्यादि धर्मशास्ोक्तनियमपूर्वक \textendash\ 

मधीताः सन्तः फलवन्तो ज्योतिष्टोमादिकर्मानुष्ठानाङ्गविशिष्टवोधेन फल \textendash\ 

वन्तो भवन्तीत्यर्थः~। एवमिति~। अर्थबोधनाय प्रयुक्ताः \textendash\ पचति देवदत्त 

इत्यादयो लौकिकशब्द~। व्याकरणानुगतप्रकृत्यादिविभागधर्मवत्तया 

तस्मादेवविदिताः सन्तः पुरुषैरुच्चारितास्तद्गतनियमपूर्वकहेतुकाभ्युदयेन 

फलवन्तो भवन्तीत्यर्थः~॥ शास्त्रेति~। क्रियाविशेषणम्~॥ एवं प्रागपि~॥


छाया~॥ 

३ एवं विशिष्टपक्षादरेण समाधौ विनिगमनाविरहाज्ज्ञानस्यान्या \textendash\ 

विशिष्टस्य तद्धेतुत्वमाश्रित्यापि समाधिः सुवच इत्याशयेन पक्षान्तर \textendash\ 

माह \textendash\ थथवा पुनरिति~॥ एवेनान्यप्राधान्यव्यावृत्तिः~। अप्यर्थो 

वा सः~॥ शब्देति~। शब्द एव प्रमाणं येषामित्यर्थः~। फलसाधम \textendash\ 





(व्याख्यान्तरभाष्यम् ) 

अपर आह \textendash\ . . 

तत्नुल्यं वेदशब्देन \textendash\ इति~। 

यथा \textendash\ वेदशब्दा नियमपूर्वमधीताः फलवन्तो 

भवन्ति, एवं यः शास्त्रपूर्वकं शब्दान्प्रयु्क्ते सोऽभ्यु \textendash\ 

दयेन युज्यत इति~॥ 

( प्रदीपः ) अपर आहेति~। वेदश्चासौ शब्वश्च \textendash\ वेदशब्द 

इति कर्मधारयः~॥ 

( सिद्धान्तभाष्यम् ) 

अथ वा पुनरस्तु \textendash\ 

ज्ञान एव धर्म इति~॥ 

शक्तेत्यक्षत्वात्~। संगतेरग्रहादनुमानाद्यापि~॥ यद्यपि न शब्दोऽपि 

संबन्धग्रहसापेक्षः, तथापि स्वर्गकामो यजेतेति वाक्यं गृहीतंसंबन्ध \textendash\ 

पदार्थं बोधयन्संसर्गमलैकिकमप्याकाङ्कादिमहिम्ना बोधयति~। मन्वा \textendash\ 

दीनां धर्मसाक्षात्कारोऽपि यथाशास्त्रमेव~। अस्माकं तु नित्यपरोक्ष इति 

यावत्~॥ तदाह \textendash\ यच्छब्द इत्यादि~॥ अविहितानिपिद्धव्यवहारे 

दोषगुणौ नेल्यत्र दृष्टान्तमाह \textendash\ तद्यथेति~॥ नाभ्युदयायेति~। 

नाप्यभ्युदयायेत्यर्थः~॥ एवं प्रागपि~॥ छाया~॥ 

४ {\qt ज्ञान एव} इत्येव धर्म इतिइत्यंशरहितः प. पाठः~॥ ज्ञान 

एव धर्म इति~। एतच्चाभ्युपेत्यवादमात्रं पूर्वोक्तदोषपरिहारसामूर्थ्य \textendash\ 

प्रदर्शनार्थं कृत्वाच्चिन्तान्यायेनोक्तम्~। 
परमार्थतस्तु \textendash\ आनर्थक्यप्रसङ्ग \textendash\ 

विशातपाराथयापादितार्थवादत्वात्फलश्रुतिर्तं फलप्रतिपत्तिक्षमा विज्ञायते
~। 

यथा योऽश्वमेवेन यजते य उ चैनमेवं वेदेति ज्ञानमात्रादेव बह्महत्या \textendash\ 

तरणं यदि सिद्धयेत्को जातुच्चिद्वदुद्रव्यव्ययायाससाध्यमश्वमेधं कुर्यात्
! 

तद्विधानञ्चानर्थर्क स्यात्~। एवं शब्दज्ञानाच्चेद्धर्मः
सिद्धयेत्कोनामानेक \textendash\ 

ताल्वादिव्यापारायासखेदमनुभवेत्~। तस्मात्क्रतुबदेव ज्ञानपूर्वप्रयोगस्यैत


फलम्~। कारणे कार्यवदुपत्तारात् {\qt देवो वर्षति तण्डुलान्} इत्यत्र 

तण्डुले देववर्यणवत् ज्ञाने धर्मवचनमापादिताधर्मपरिहाराभिवानशक्ति \textendash\ 

मात्रप्रदर्शनार्थमेवोपन्यस्तम् न फलवत्त्वप्रतिपादनाय~। द्रव्यसंस्कार \textendash\ 


कर्मसु परार्थत्वात् \textendash\ इति न्यायेन ज्ञानस्य पुरुषशब्दसंस्कारत्वेन निरा \textendash\ 

काङ्क्स्य फलसम्बन्धासम्भवादिति प्रथमस्य तृतीये पादे मीमांसवास \textendash\ 

नावासितान्तःकरणानान्तन्रवार्तिककाराणां मतं व्याकरणसिद्धान्ता \textendash\ 

ननुगतमिति भाष्याभिप्रायावगतये नालम्~। वस्तुतस्तु ज्ञाने धर्म इत्येव 

सिद्धान्तः~। तथा हि भाष्ये शब्दस्वरूपविवरणावसरे येनोच्चारितेन 

{\qt सास्ना} इति अथवा {\qt प्रतीतपदार्थको लोके ध्वनिः} इति द्विधाशब्द \textendash\ 

स्वरूपमुक्तम्~। तदनुसारेण ध्वनिरूपशब्दमाश्रित्य शस्त्रपूर्वके प्रयोगे \textendash\ 


ऽभ्युदय इति समाहितम्~। स्फोटः शब्द इति सिद्धान्ते च जाने एव 

धर्म इति~। स च स्फोटो नित्यः श्रुतिप्रतिपाद्यो ब्रह्मशब्दापरपर्यायः~। 

स्फोटशानञ्च ब्रह्मज्ञानवन्मुख्यत्वात्पराङ्गं न भवतीति {\qt सर्वत्रैवहि 

विज्ञानं} इति न्यायविरोधो न~। संस्काररूपत्वाभावाच्च {\qt द्रव्यसंस्कार \textendash\ 

कर्मसु परार्थत्वात्} इति सूत्रेविरोधोऽपि न~। शब्दविवरणावसरेऽभ्य \textendash\ 

हितत्वात्स्फोटसिद्धान्तं पूर्वमुपपाद्य स्फोटाभिव्यक्त्युपायभूतं
ध्वनिपक्षं 

सककललोकसाधारण्येन घश्चादुपन्यस्यति~। अत्र तु प्रयोगपूर्वके ज्ञाने 

धर्म इति सिद्धान्तपक्षे विशेषणांशं प्रयोगपक्षे पूर्वोपस्थितमुपपाद्य 

सिद्धान्तः समर्थित इति न काऽप्यनुपपत्तिः~। यवञ्च शब्दज्ञान एव 

धर्मं इत्येव व्याकरणसिद्धान्तः~॥ 

शब्दनस्य धर्मजनकताधिकरणम् ] महाभाष्यप्रदीपोद्दयोतव्याख्या छाया~। 



( आक्षेपस्मारणभाष्यम् ) 

ननु चोक्तम् \textendash\ ज्ञाने धर्म इति चेत्तथाऽधर्मः

इति~॥ 

(आक्षेपनिराकरणभाष्यम् ) 

नैष दोषः, शब्दप्रमाणका वयम्, यच्छब्द आह 

तदस्माकं प्रमाणम्~। शब्दश्च शब्दज्ञाने धर्ममाह, 

नापशब्दज्ञानेऽर्धर्मम्~॥ यच्च पुनरशिष्टाप्रतिषिद्धम्, 

तैव तद्दोषाय भवति, नाभ्युदयाय~। तद्यथा \textendash\ हि \textendash\ 

क्वितहसितकण्डूयितानि नैव दोषाय भवन्ति, ना \textendash\ 

भ्युदयाय~॥ 

( उद्दयोतः ) भाष्ये \textendash\ प्रमाणमिति~। प्रामाणिकमित्यर्थः~॥ 

शब्द शब्दज्ञाने इति~। {\qt एकः शब्दः सम्यक् ज्ञातः} 

इत्यादिः~॥ {\qt हेलयो हेलयः} इत्यादि तु अपशब्दप्रयोगेऽधर्ममाह, 

न तु ज्ञाने इति भावः~॥ यद्यपि तत्र {\qt प्रयुक्तः} इत्यप्युक्तं, तथाप्य \textendash\ 

वैयाकरणस्य सम्यक्प्रयोगादपि धर्माभावेन ज्ञानमेव तद्धेतुः~। 

प्रयोगस्तु तस्य द्वारमिति भावः~॥ 

(समाधानान्तरभाष्यम् ) 

अथवाऽभ्युपाय एवापशब्दज्ञानं शब्दज्ञाने~। यो 

ह्यपशब्दाञ्जानाति शब्दानप्यसौ जानातिः~। तदेवं 

{\qt ज्ञाने धर्मः इति ब्रुवतोऽर्थादापन्नं भवति \textendash\ अप \textendash\ 

शब्दज्ञानपूर्वके शब्दज्ञाने धर्मः} इति~॥ 

( प्रदीपः ) अथवेति~। अपशब्दज्ञाननान्तरीयकत्वाच्छ \textendash\ 

ब्दज्ञानस्य पृथक्फलमपशब्दज्ञानस्य नास्तील्यर्थः~॥ 



१ ननु तेऽसुरा हेलयः इत्यादिना तदप्युक्तमत आह \textendash\ हेऽ \textendash\ 

लय इति~॥ इत्यप्युक्तमिति~। एवं चोभयोः स्वर्गसाधनत्वं वचने \textendash\ 

नैव बोध्यते इति {\qt शब्दश्च ज्ञाने धर्ममाह} इत्ययुक्तमिति भावः~॥ 

धर्मेति~। स्वर्गसंपादकेत्यादिः~॥ छाया~॥ 

२ ननु शास्त्रविहितामिक्षाव्यापारनान्तरीयकतयोत्पन्नवाजिनस्य 

{\qt कार्यविशेषे वाजिभ्यो वाजिनम्} इति शास्त्रेण विनियोगदर्शनाद \textendash\ 

त्रापि शब्दशाननान्तरीयकतयोत्पन्नस्यापशब्दशानस्य साक्षाद्विनियोजक \textendash\ 

प्रमाणाभावेऽपि तस्य कत्पना स्यादत आह \textendash\ अथवेति~। उपसंह \textendash\ 

रति \textendash\ तदेवमिति~। आपन्नं \textendash\ प्राप्तम्~॥ छाया~॥ 

३ यद्यपिं द्वयोरेकशास्त्रविहितव्यापारजत्वम्, तथाप्येकस्य श्रुत्या 

साक्षात्फले विनियोगकल्पनापेक्षया फल्वदङ्गत्वविनियोगकत्वमेव कल्प \textendash\ 

यितुं युक्त {\qt फलवत्संनिधौ \textendash\ } इति न्यायात्~। तत्र तु तथा 

शास्त्रमस्तीति वैषम्यमिति भावमभिप्रेत्याह \textendash\ अपशब्देति~। बहु \textendash\ 

ब्रीहिः~॥ गाब्याद्यपशब्दस्य यथाकथंचिदज्ञानस्य त्वनभ्युपायत्वा \textendash\ 

दाह \textendash\ अपशब्दत्वेनेति~॥ तज्ज्ञानमिति~। अपशब्दशानमित्यर्थः~॥ 

उक्तहेतोरेवाह \textendash\ एवमिति~॥ तदुपपादयति \textendash\ अपेति~॥ नन्वेवं 

भाष्यविरोधोऽत आह \textendash\ यो हीतीति~। इत्यादिभाष्यस्येत्यर्थः~। भाव 

इत्यत्रान्वयः~॥ हिरपिश्चार्थे इत्याह \textendash\ य एवेत्यगदि~॥ तज्ज्ञानेति~। 



(उद्दयोतः ) भाष्ये \textendash\ अभ्युपाय एवेति~। अपशब्दज्ञाननान्त \textendash\ 

रीयकं शब्दज्ञानमिति शब्दज्ञानेन फले जननीये सहयकारितैवाप \textendash\ 

शब्दज्ञानस्य, न पृथक्फलतेति भावः~॥ अपशब्दज्ञानम् \textendash\ अप \textendash\ 

शब्दत्वेन तज्ज्ञानम्~॥ एवं {\qt शब्दशाने} इत्यपि~॥ अपशब्दत्येन 

तज्ज्ञाने हि तद्भिन्नत्वेन शब्दज्ञानं भवति~॥ {\qt यो हि} इति भाष्यस्य 

य एव अपशब्दान् जानाति असावेव शब्दान् जानातीति तज्ज्ञा \textendash\ 

नसमानाधिकरणशब्दज्ञानादेव धर्म इति न ततो विपरीतफल \textendash\ 

कल्पना युक्तेति भावः~॥ तदाह \textendash\ अपशब्दज्ञाननान्तरीय \textendash\ 

कत्वादिति~। बहुब्रीहिः~। न दि वह्लयानयननान्तरगयकं पात्रा \textendash\ 

नयनं ततः पृथक्फलं भवति~॥ 

( समाधानान्तरे न्यायदर्शकभाष्यम् ) 

अथवा कृपखानकवदेतैद्भविष्यति~। तद्यथा \textendash\ 

कूपखानकः कूपं खनन्यद्यपि सृदा पांशुभिश्चाव \textendash\ 

कीर्णो भवति सोऽप्सु संजातासु तत एव 

तं गुणमासादयति, येन स च दोषो निर्हण्यते 

भूयसा चाभ्युदयेन योगो भवति~॥ एवमिहापि 

यद्यंपशब्दज्ञानेऽधर्मः, तथापि यस्त्वसौ शब्दज्ञाने 

धर्मस्तेन स च दोषो निर्धानिष्यते, भूयसा 

चाभ्युदयेन योगो भविष्यति~॥ 

( प्रदीपः ) दोष इति~। उत्कृष्टधर्मफलावापौ स्वल्पमध \textendash\ 

मफलमुत्पन्नमप्यनुत्पन्नकल्पं भवतीत्यर्थः~॥ 

अपशब्देत्यर्थः~॥ ततः \textendash\ अपशब्दशानात्~॥ तत्पुरुषव्यावृत्तय
आह \textendash\ \textasciitilde 

बह्नीति~॥ वह्वयानयनेति~। तत्पुरुषः~। ततः \textendash\ अग्यानयनात्~॥ 

छाया~॥ 

४ अथवेति~। नैतत्समाध्यन्तरं, पूर्वोक्तसमाधावेव न्यायप्रदर्शन \textendash\ 

मेतत्~। अत एव {\qt समाध्यन्तरं} इत्युद्दयोतः {\qt समाधिविशेषमित्यर्थः} 

इति व्याख्यातश्छायायाम्~॥ 

५ एतत् \textendash\ अपशब्दज्ञानरूपं वस्तु कूपखानकेन तुल्यं भविष्यती \textendash\ 

त्यर्थः~॥ अवकीर्णो मिलितः~॥ तत एव \textendash\ अद्भय एव~। तं गुणं \textendash\ 

स्नानपानादिरूपेष्टसाधनातिशयमित्यर्थः~॥ येनेति~। कर्तरि तृतीया~॥ 

कर्मणि लः~॥ चशब्दौ मिथः समुच्चायौ~॥ इहापि \textendash\ शब्दशान \textendash\ 

स्थलेऽपि~॥ यदि \textendash\ यद्यपि~॥ छाया~॥ 

६ {\qt पांसुभिः} इति च. क. पाठः~॥ 

७ {\qt यद्यप्यपशब्द} इति क. ख. पाठः~। 

८ {\qt निर्धेति णिजन्तरूपम्} इति छाया~॥ प. पुस्तके {\qt निर्धा \textendash\ 

तिष्यते} इति पाठः~॥ अत्र {\qt येन स च दोषो निर्हण्यते}इत्यादि \textendash\ 

कर्मप्रत्ययान्तधटितपूर्ववाक्यानुरोधेन {\qt निर्घानिष्यते} इति कर्मणि
लृडन्त \textendash\ 

पाठ एवानुकूंलः प्रतिभाति~। निर्घातिष्यत इति पाठे स्वार्थणिच्कल्षनं, 

प्रक्रमभङ्गः, {\qt दोषः} इति कर्मणि प्रथमाया अनुपपत्तिश्च~॥ 

७६ 

उद्दयोतपरिवृतप्रदीपप्रकाशितमहाभाष्यम्~। 

[१ अ. १ पा. १ पस्पशाह्निके 

(उद्दयोतः ) {\qt तुष्यतु दुर्जनः?} इति न्यायेनापशब्दानस्य विप \textendash\ 

रीतत्वमात्रेणाधर्ससाधनत्वमभ्युपगम्यापि भाष्ये समाध्यन्तरभाह \textendash\ 

अथवा कूपेति~। मृत् \textendash\ आर्द्रा~। पांसुः \textendash\ शुष्का~। यत्कर्मना \textendash\ 

न्तरीयककर्मजन्यो यो विपरीतत्वमात्रेण कल्प्यो्वैषो दोषः स 

तत्कर्मजन्यफलनाश्यः, यथा कुपखनननान्तरीयकशरीररव्यापारजन्यो 

मृल्लेपस्तत्फलजलनाश्य इति सामान्यती दृष्टानुमानात्तन्नाशानु \textendash\ 

मानमिति बोध्यम्~॥ उत्कृष्टधर्मेति~। यैर्मस्य बिधिोध्यत्वादुत्कृ \textendash\ 

ष्टत्वम्~। अधर्मस्य तु वचनाबोध्यत्वेन विपरीतत्वमात्रेण कल्प्य \textendash\ 

स्याल्पत्वेन र्तत्फलधर्मनाश्यत्वम्~॥ यागीयहिसायाः {\qt न हिंस्यात्ः 

इतिवचनेन दुष्टताबोधनाद्युष्टह्वमेवेत्यन्यत्र विस्तरः~॥}

( द्वितीयदूषणनिरासभाष्यम् ) 

यदप्युच्यते \textendash\ आचारे नियमः इति~। 

याज्ञे कर्मणि स्स नियमः, अर्व्यत्रानियमः~। एवं हि 

श्रूयते \textendash\ {\qt यर्वाणस्तर्वाणो नाम ऋषयो वभूवुः प्रत्य} \textendash\ 

क्षधर्माणः परापरशा विदितवेदितव्या अधिगत \textendash\ 

याथातथ्याः~। ते तत्रभवन्तः {\qt यद्वा नः तद्वा नः}

इति प्रयोक्तव्ये यर्वाणः \textendash\ तर्वाणः इति प्रयुञ्जते, याज्ञे 

पुमः कर्मणि नापभाषन्ते~॥ तैः पुनरसुरैर्याज्ञे कर्म \textendash\ 

ण्यपभाषिसम्, ततस्ते पराभूताः~॥ 

( प्रदीपः ) प्रत्यक्षधर्माण इति~। योगिप्ररत्यक्षेण सर्व 

विदितवन्तः~॥ परापरज्ञाः \textendash\ विद्याऽविद्याप्रविभागज्ञाः~॥ 

१ नापशब्दज्ञानस्येति~। तन्नान्तरीयकेत्यादिः~॥ समाध्यन्तर \textendash\ 

ति~। समाधिविशेषमित्यर्थः~। यथाश्चुतेऽनुपपत्तिः स्पष्टैव~॥ कूपे \textendash\ 

~। कूपस्य खानक इत्यनित्यसमासत्वात्स्वपदविग्रहः~॥ खनन \textendash\ 

वमात्रमिह विवक्षितं, नतु शिल्पित्वम्~। तेन स्वरूपेणापि शिल्पिनि 

ल्पिनि ष्वुन् इति ष्बुना ण्बुलो न बाधः~। अत्र कर्मण्यण् तु न, 

{\qt वासरूपः} इत्युक्तेः~॥ तृतीयासमर्थाद्वतिः~॥ शुष्केति~। ब्राह्मण \textendash\ 

वसिष्ठन्यायेन विर्देश इति भावः~॥ छाया~॥ 

२ यत्तु \textendash\ {\qt इदमयुक्तम् , शब्दशानस्य धर्मसाधनत्ववदधर्मघातने 

मानाभावात्~। विषमश्च दृष्टान्तः, खननजातेन जलेन रजोनिर्हणन \textendash\ 

द्वारा पूर्वस्थितनैर्मल्यलक्षणगुणप्राप्तेरन्वयव्यतिरेकसिद्धत्वात्~। 
प्रयुक्त \textendash\ 

शब्दज्ञानजनितपूर्वहेतुकाभ्युदयमात्रस्य शास्त्रसिद्धत्वात्} इति कृष्णः~। 


तदयुक्तमिति ध्वनयन्नाह \textendash\ यत्कर्मेति~॥ छया~॥ 

३ अधर्मभूयस्त्वस्यैव प्रागुक्तत्वेन तदल्पत्वोक्तिरयुक्तेत्यत आह \textendash\ 

धर्मस्येति~॥ छाया~॥ 

४ तत्फलेति~। शब्दज्ञानफलो यो धर्मस्तेन नाश्यत्वम्~॥ 

५ {\qt यदप्युच्यते} इत्यतः पूर्व स्वकपोलकल्पनाकलितानि वार्ति \textendash\ 

कानि राजलक्ष्मीटिप्पणकौरः कल्प्यन्ते, व्याख्याभाष्यबलादनुमीयन्त 

इति चोच्यते~। तदेतत्साहसमात्रम्, स्वकत्यनातिरिक्तप्रमाणानुपल \textendash\ 

म्भात् , असद्धेतुकानुमानस्याप्रमा त्वाच्च~॥ 

६ {\qt अन्यन्रानियमः} इत्यस्य प. पुस्तके न पाठः~॥ 

७ यद्यपि {\qt ब्राह्मणेन न म्लेच्छितवै इति निषेधवाक्ये}





( उद्दयोतः ) भाष्ये \textendash\ अन्यत्रानिथम इति~। {\qt यज्ञे सुशब्द} \textendash\ 

प्रथोगाद्धर्मः \textendash\ अपश्ब्दप्रयोगादधर्म इति तत्रैव तयोः प्रयोगमियमः~। 

तदतिरिक्तस्थले तु सुशब्दापशब्दयोः प्रयोगेऽनियमः~। योऽपि यज्ञे 

दोषः सोऽपि तदङ्गभूतसंकल्पोहादिविषये एव~। श्रुतौ {\qt हेल्यो 

हेलयो वश्यध्व इत्यूह एव~॥ विदितवेदितव्याः \textendash\ इत्यनेन 

श्रवणमनननिदिध्यासनसंपदुक्ता~। अधिगतयाथातथ्याः \textendash\ इत्यनेन 

साक्षात्कार उक्तः~। यार्थीतथ्यं प्रयुक्ते \textendash\ इति वर्तमानसमीपे भूते 

लट्ट~। एवं नापभाषन्ते} इत्यपि~। अत एव तन्नामानो बभूवुरित्यर्थः~। 

स्वार्थ ष्यम्~। वेदितव्यस्य यद्यद्यथा तथा स्वरूपं शुद्धचिदानन्दरूं 

तदधिगतं \textendash\ प्रत्यक्षेण प्राप्तं यैरित्यर्थः~॥ याज्ञे पुनरिति~। अनेन 

तत्त्वशानामपि कर्माधिकारं सूच्यति~॥ 

इति शब्दज्ञानस्य धर्मजनकताधिकरणम्~। 

(अथ व्याकरणाधिकरणम् ) 

(व्याकरणपदार्थोपन्यासभाष्यम् ) 

अथ व्याकरणमित्यस्य शब्दस्य कः पदार्थः ? 

सूत्रम्~॥ 

( प्रदीपः ) अथेति~। उक्तमिद \textendash\ न चान्तरेण व्याक \textendash\ 

रणं \textendash\ इत्यादि~। तत्र पक्षद्वयेऽपि दोषदर्शनात् पदार्थप्रश्नः~॥ 

( उद्दयोतः ) व्याकरणपदार्थविचारस्य सङ्गाति दर्शयितुमाह \textendash\ 

उक्तमिदमिति~। आदिना {\qt तस्मादुध्येयं व्याकरणं} इत्यादि 

पुनः पुनरुक्तिर्गृह्यते~॥ 

ब्राह्मणमात्रसंयोगः श्रुतः, तथापि यज्ञसंवद्धब्राह्मणसंस्कारत्वे
मानमाह \textendash\ 

एवं हीति~॥ छाया~॥ 

८ यद्वा \textendash\ यद्दस्तु तद्वा \textendash\ तद्वस्तु वर्ततां नोऽस्माकं किम् \textendash\ ३त्यर्थप्रतिपि \textendash\ 


पादयिषया {\qt यद्वा नस्तद्वा न} इति वाच्ये {\qt यर्वाणस्तर्वीण इति प्रयुक्त} \textendash\ 

वन्तस्ततस्तन्नामका एव च कषयः संपन्ना इत्यर्थः~॥ छाया~॥ 

९ {\qt योगजप्रत्य इति मुद्रितपाठः,} योगिप्रत्यक्षे स्व इति क \textendash\ पाठः~॥ 

१० यज्ञ इति~। {\qt एकः शब्दः} इति शब्दमात्रसंयोगात्पुरु \textendash\ 

पार्थोऽयं विधिः~। {\qt न म्लेच्छितवै} आहिताश्निः \textendash\  इति निषेध \textendash\ 

प्रायश्चित्तयोस्तु ज्योतिष्टोमप्रकरणात्रत्वर्थश्चुत्योः पुरुषार्थमूलता


संभवति यद्विरोधो ज्ञानधर्मपक्षस्य स्यात् इति भावः~॥ इति छाया~॥ 

११ {\qt याथातथ्ये स्वार्थं ष्यम्~। वेदितव्यस्य यद्यद्यथातथास्वरूपं 

शुद्धच्विदानन्दरूपं तदधिगतं \textendash\ प्रत्यक्षेण प्राप्तं यैरित्यर्थः~। याज्ञे
पुन \textendash\ 

रिति~। अनेन तत्वशानामपि कर्माधिकारं सूचयति~। प्रयुञ्जते \textendash\ इति 

वर्तमानसमीपे भूते लट्~। एवं नापभाषन्त इत्यपि~। अत एव तन्ना \textendash\ 

मानो बभूवुरित्यर्थः} इति व्यत्यस्तो मुद्रितपाठः~। 

१२ {\qt यद्यथां सथा स्वरूपं} \textendash\ समे स्वरूपं इति घ. पाठः~। यथा 

तथा स्वरूपं सत्यं स्त्रूपं इति मुद्रितपाठः~॥ 

१ वर्णोपदेशस्य शास्त्रोपोद्धातत्वसिद्धये व्याकरणपदार्थ एव 

वार्तिककृता निरूपितः~। तत्र निरूपणपूर्वपक्षः सूत्रशब्दयोर्व्याकरणपदस्य 

शिष्टप्रयोगप्रसिद्धिमन्तरेण न संगच्छत इत्वतः आह \textendash\ उक्तमिति~॥ 

व्याकरणाधिकरणम् ] 

पस्पशाह्निकम्~। 



(१० आक्षेपवार्तिकम्~॥ १~॥ ) 

~॥\#~। सूत्रे व्याकरणे षष्ठ्यर्थोऽनुपपन्नः~॥~। 

(भाष्यम् ) 

सूत्रे व्याकरणे षष्ठयर्थो नोपपद्यते \textendash\ {\qt व्याकर \textendash\ 

णस्य सूत्रम्} इति~॥ किं हि तदन्यत्सूत्रात् व्याकरणं 

यस्यादः सूत्रं स्यात् ? 

( प्रदीपः ) षष्ठयर्थ इति~। द्वाभ्यामपि शब्दाभ्यामषटा \textendash\ 

ध्याय्याः प्रतिपादनाद्व्यतिरेकाभावः~। सीमान्यविशेषशब्दतया 

तु द्वयोः व्रैयोगो न विरुध्यते~। र्येदा त्वष्ठाध्याय्येकदेशः 

सूत्रशब्देनोच्यते, तदा पष्ट्यर्थोऽप्युपपद्यते~॥ 

(उद्दयोतः ) ननु सू्रसमुदायस्य व्याकरणस्येदं सूत्रमित्युपप \textendash\ 

चतेऽते ओह \textendash\ द्वीभ्यामिति~। सूत्रपदेनाप्यष्टाध्याम्येव यदोच्यते 

तदापीष्यतेऽरं प्रयोगः, स न सिध्येदित्यर्थः~॥ ननु किमुष्यते \textendash\ 

{\qt षष्ठयर्थोऽनुपपन्नः}इति, पर्यायतया सहप्रयोगोऽपि न स्यादत 

आह \textendash\ सामान्यविशेषेति~। सूत्रं साभान्यं, व्याकरणं विशेषः~। 

सूत्रशब्देनाष्टाध्याय्येव, तदेकदेशे तु योगव्यवहार एव {\qt योगे 

योगे उपतिष्टते} इत्यादौ~॥ यदा त्विति~। {\qt सूत्राणि चाप्यधी \textendash\ 

याने} इति भाष्ये वक्ष्यमाणत्वादिति भावः~। वस्तुत एकदेशस्य 

सूत्रत्वेऽपि तस्यापि साक्षात्परम्परया वा व्याकरणत्वात्पष्ठयर्थानु \textendash\ 

पपत्तिरेवेति तत्त्वम्~॥ 

(११ आक्षेपान्तरवार्तिकम्~॥ २~॥ ) 

॥~॥~शब्दाप्रतिपत्तिः~॥~॥ 

( भाष्यम्) 

शब्दानां चाप्रतिपत्तिः प्राप्नोति \textendash\ {\qt व्याकरणा \textendash\ 

च्छब्दान्प्रतिपद्यामहे} इति~। न हि सूत्रत एव श \textendash\ 

व्दान्प्रतिपद्यन्ते~। 

किं तर्हि ? 

व्याख्यातश्च~॥ 



१ षष्ठर्थं इति~। सूत्रव्याकरणयोर्मदाभावेन भेदसम्वन्ध \textendash\ 

बोधिका षष्ठी न स्यादितिभावः~॥

२ सामान्येति~। सूत्रशब्दस्य व्याकरणसूत्रेषु प्रयोगवत् मीमां \textendash\ 

सादिसूत्रेष्वपि प्रयोगात्स सामान्यार्थस्य वाचकः, व्याकरणशब्दस्तु 

विशेषार्थस्येति सामान्यविशेषशब्दतेति भावः~॥ 

३ {\qt सहप्रयोगो} इति मुद्रितपाठः~॥ 

४ ननु सूत्रशब्दो न समुदायवाची किन्त्ववयवपर्यीप्त एवेति 

पक्षस्तदाऽऽह \textendash\ यदा त्वेति~॥ 

५ ब्याकरणस्य सूंत्रं इति प्रयोगासंभवं प्रदर्श्य लोकव्यवहारा \textendash\ 

नुपपत्तिं प्रदशियति \textendash\ शब्दाप्रतीति~॥ 

६ आदर्शपुस्तकैषु एवं तर्हि शब्दः इति भाष्यव्याख्यानोद्दयोते 

{\qt शष्दाप्रतिपत्तिहेतुतातदवस्था} इत्यनन्तरमयं ग्रन्थो दृश्यते~। खण्डशो 

विन्यासे पूर्वमुपन्यस्तः~॥ 

७ वाक्यैकदेशेति~। यथा {\qt भनाम्नवतिनगरीणामुपसंख्या \textendash\ 

नम्} इति वार्तिकस्य {\qt न पदान्ताद्वोरनाम्} इति सूत्रस्थनिर्वि \textendash\ 

भक्तिकेनानाम्पदेन शेषपूर्तित्वसूचनमिति भावः~॥ इति दाधिमथाः~॥ 

८ यक्षानुरूपो बलिरिति न्यायेनैकदेशी व्याक्रियत इति कर्म \textendash\ 





मानानीत्यथः~॥ 

( प्रदीपः ) शब्दाप्रतिपत्तिरिति~। न हि व्याख्यान \textendash\ 

रहितसूत्रमात्रश्रवणाच्छब्दाः प्रतीयन्ते~॥ 

(उद्दयोतः ) समुदायस्यातिरिक्तत्वेऽपि दोषमाह \textendash\ भाष्ये \textendash\ 

शब्दाप्रतिपत्तिरिति~। व्याकरणाच्छब्दान्प्रतिपद्यामह इति शब्द \textendash\ 

विषयव्यवहारासिद्धिः प्राप्नोतीत्यर्थः~। यतः केवलसूत्रेभ्यः शब्दप्र \textendash\ 

तिपत्तिर्त दृश्यते~। व्याकरणस्य शब्दप्रतिपत्तिहेतुत्वं च लोकसि \textendash\ 

द्धम्, व्याकरणाच्छब्दान्प्रतिपद्यामह शति व्यवहारात्~। सेयं 

प्रतिपत्तिसाधनता सूत्राद्व्थावर्तमाना तद्व्थाप्यां व्याकरणशब्द \textendash\ 

वाच्यतां व्यावर्तयतीति भावः~। तदाह \textendash\ न हीति~॥ 

( आक्षेपबाधकभाष्यम् ) 

ननु च तदेव सूत्रं विगृहीतं व्याख्यानं भवति~॥ 

( आक्षेपसाधकभाष्यम् ) 

न केवलं चर्चापदानि व्याख्यानम् \textendash\ {\qt वृद्धिः, 

आत् , ऐच्} इति~। 

रकिं तर्हि ? 

उदाहरणं \textendash\ प्रत्युदाहरणं \textendash\ वाक्याध्याहारः \textendash\ इत्येत \textendash\ 

त्समुदितं व्याख्यानं भवति~॥ 

(प्रदीपः ) समुदितमिति~। समुदायादेवार्थावसायो \textendash\ 

त्पादादित्यर्थः~॥ \textendash\ 

( उद्दयोतः ) भाष्ये \textendash\ चर्चापदानि \textendash\ चर्च्यमानानि \textendash\ विभज्य \textendash\ 

 मानानीत्यर्थः वाक्याध्याहार इति~। वाक्यघट्क्पदानां 

सृत्रान्तरे श्रुतानां स्वरितत्वप्रतिङ्ञयाऽध्याहारः \textendash\ कल्पनमित्यर्थः~। 

यद्धा वाक्याध्याहार इत्यनेन वार्तिककृद्वाक्यानां सून्नतात्पर्यवि \textendash\ 

षयता वैीक्यैकदेशन्यायेन सूच्िता~॥ 

( पक्षान्तरोपस्थापर्क भाष्यम्) 

एवं तर्हि शब्दः~। 

( उद्दयोतः ) शब्द इति~। बक्ष्यमित्यर्थः~। अत्र पक्षे षष्ठयर्थ 

उपपद्यते~। शब्दीप्रतिपत्तिहेतुता तु तदवस्था~॥ 

ब्युत्पत्त्या शब्दो व्याकरणमित्याह \textendash\ एवं तर्हीति~॥ छाया~॥ 

९ लक्ष्यमिति~। अष्टाध्याय्या इति शेषः~। तथा चानया प्रतिपाद्यः 

शब्दो व्याकरणपदवाच्य इत्यर्थो बोध्यः~। एवं च नातिप्रसङ्ग इति 

भावः~॥ सूत्रं त्वत्र पक्षे तत्संबन्धादुपचारादुच्यते~॥ छाया~॥ 

१० अत्र पक्षे प्रागुक्तदोषद्वयोद्धार इति भ्रमनिरासायाह \textendash\ 

अन्नेति~॥ शब्दाप्रेति~। {\qt व्याकरणाच्छब्दान् प्रतिपद्यामहे} इति 

शब्दविषयकव्यवहारासिद्धिहेतुतेत्यर्थः~। शब्देन शब्दप्रतीतेरसंभवादिति 

भावः~॥ छाया~॥ 

११ शब्दाप्रतिपत्तिहेतुतेति~। व्याकरणपदार्थः शब्दः स च 

सूत्रस्य लक्ष्यम्~। एवञ्च लक्ष्यलक्षणभावरूपसम्बन्धे षष्ठ्यपि साधुरिति 

{\qt व्याकरणस्य सूत्रं} इति प्रयोग उपपद्यते~। सर्वस्यापि साधुशब्दमात्रस्य 

व्याकरणपदार्थत्वेन {\qt व्याकरणाच्छब्दान् प्रतिपद्यामहे} इति व्यवहारा \textendash\ 

संभवः, व्याकरणातिरिक्तस्य प्रतिपत्तिविषयशब्दस्याभावात्~। एतेन 

सुध्युपास्य इत्यादिलक्ष्यैर्नद्युपास्य इत्यादीनां प्रतिपत्तिः संभवतीति


{\qt शब्दाप्रतिपत्तिहेतुता तदवस्थेत्युद्दयोतश्चिन्त्य} इति गुरुप्रसादोक्तिः


प्रामादिकी~। सुध्युपास्य इतिवन्नद्युपास्य इत्यस्यापि व्याकरणत्वेन 

तदन्यस्य व्याकरणपदावाच्यस्य शब्दस्य प्रतिपत्त्यसंभवात्~॥ 

७८ उद्दयोतपरिवृतप्रदीपप्रकाशितमहाभाष्यम् [१ अ.१ पा.१ पस्पशाह्निके



(१२ शब्दो व्याकरणमिति पक्षे आक्षेपवार्तिकम्~॥ ३~॥ ) 

~॥~॥ शब्दे ल्युडर्थः~॥~॥ 

(भाष्यम् ) 

यदि शब्दो व्याकरणं ल्युडर्थो नोपपद्यते \textendash\ व्या \textendash\ 

क्रियन्ते शब्दा अनेनेति व्याकरणम्~। न हि शब्देन 

किंचित् व्याक्रियते 

केन तर्हि ? 

सूत्रेण~॥ 

( प्रदीपः ) शब्द् इति~। करणे ल्युद्विधीयते~। शब्दश्च 

व्याकियमाणत्वात्कर्म, न तु करणमिति भावः~॥ 

(उद्दयोतः ) ननु राजभोजना इतिवत् कर्मतैल्युटि न दोषोऽत 

आह \textendash\ करणे इति~। कर्मणि स तु क्वाचित्क इति भावः~॥ 

(१३ आक्षेपान्तरवार्तिकम्~॥ ४~॥ ) 

~॥~॥ भवे च तद्धितः~॥~॥ 

(भाष्यम् ) 

भवे च तद्धितो नोपपद्यते \textendash\ व्याकरणे भवो 

योगो वैयाकरणः \textendash\ इति~। नहि शब्दे भवो योगः~। 

क्व तर्हि ? 

सूत्रे~॥ 

( प्रदीपः ) भवे चेति~। शब्दे५प्यन्वीख्यायकत्वेन भवो 

योग इति चेत्, मीमांसकादियोगस्यापि शब्दं प्रति विचारकत्वात् 

वैयाकरणत्वप्रसङ्गः~॥ 

( ५४ आक्षेपान्तरवार्तिकम्~॥ ५~॥ ) 

~॥~॥ प्रोक्तादधश्च तद्धिताः~॥~॥ 

(भाष्यम् ) 

प्रोक्तादयश्च तद्धिता नोपपद्यन्ते~। पाणिनिना 

प्रोक्तं पाणिनीयम्, आपिशलं, काशकृत्स्वमिति~। 

न हि पाणिनिना शब्दाः प्रोक्ताः~। 



१ कर्मेति~। कर्मत्वकरणत्वयोर्गुणप्रधानरूपत्वेन तत्कर्मणां तेषां 

तत्करणत्वासंभवादिति भावः~॥ छाया~॥ 

२ {\qt कर्मणि ल्युटि} इति कवचित्पाठः~। तस्मिन् पाटे {\qt कर्मण्युप \textendash\ 

पदे ल्युटि} इत्याशङ्का सम्भाव्येत~। अतः {\qt कर्मत्युटिः इत्येव पाठ 

उकत्वितः प्रतिभाति~॥}

३ दोषान्तरमाह \textendash\ भवे च तद्धित हति~। एवमेव वार्तिकपाठः~। 

उत्तरराष्यम्गरस्यात्~। अत एव च परिहारग्न्थे {\qt भवे च तद्धितः}

इत्येव पठितम्~॥ छाया~॥ 

४ अन्वाख्यापकत्देन \textendash\ संस्कारकत्वेन~॥ छाया~॥ {\qt अन्वाख्या \textendash\ 

पकत्वेन} इति ख. पाठः~॥ 

५ {\qt मीमांसादियोग} इति क. पाठः~। शब्दं प्रति \textendash\ वेदशब्दं 

प्रति~॥ आदिनैतत्सूचितम्कथंचिच्छब्दसंबन्धानपायात्सर्वमपि वाक्यं 

तादृशं स्यादतो व्याकरणान्यग्रन्थेष्वभवत्वेन स्थित इति वाच्यम्~। 

तच्च न संभवतीति भावः~॥ छाया~॥ 

६ प्रोक्तादिमध्ये भवार्थस्याय्यन्तर्भावात्पौनरुक्त्यमाशङ्कते \textendash\ 



किं तर्हि ? 

सूत्रम्~॥ 

( वार्तिकप्रणयनाक्षेपभाष्यम् ) 

किर्मर्थमिदमुभयमुच्यते \textendash\ भवे \textendash\ प्रोक्तादयश्च 

तद्धिताः इति~। न प्रोक्तादयश्च तद्धिताः 

इत्येव भवेऽपि तद्धितश्चोदितः स्याल्? 

(समाधानभाष्यम् ) 

पुरस्तादिदमाचार्येण दृष्टम् \textendash\ *भवे च तद्धितः* 

इति, तत्पठितम्~। तत उत्तरकालमिदं दृष्टम् \textendash\ *प्रो \textendash\ 

त्ताद्यश्च तद्धिताः* इति, तदयि पठितम्~। न चे \textendash\ 

दानीमाचार्याः सूत्राणि कृत्वा निवर्तयन्ति~॥ 

( प्रदीपः ) न चेदानीमिति~। लक्षणप्रपञ्चाभ्यां मूलसूत्र \textendash\ 

वत् वार्तिकानामुपपत्त्या दोषाभावः~॥ 

( उद्दयोतः ) लक्षणप्रपञ्चाभ्यामिति~। यैर्या \textendash\ कर्मधारय \textendash\ 

प्रकरणे, अलुक्प्रकरणे च~। विजिगीषुकथायां हि अभिहिताभिधानम \textendash\ 

शक्तिसूचचकत्वाद्दोषाय~। इह तु व्युत्पादनार्थत्वान्न दोषकृदिति भावः~॥ 

पुर्वं सामान्ये उक्ते पश्चाद्विशेषकथनं प्रपञ्चार्थमिति युज्यते, अन्न
तु 

विपरीतमिति न तत्साम्यमित्यन्ये~॥

( प्रथमाक्षेपबाधकभाष्यम् ) 

अयं तावददोषः \textendash\ यदुच्यते शब्दे ल्युडर्थः

इति~। नावश्यं करणाधिकरणयोरेव ल्युङ्क्धीयते~। 

किं तर्हि ? 

अन्येष्वपि कारकेषु \textendash\ {\qt कृत्यल्युटो बहुलम्} 

इति~। तद्यथा \textendash\ प्रस्कन्दनं प्रपतनमिति~॥ 

( प्रदीपः ) प्रस्कन्दनमिति~। यद्यप्ययं भीमादिः, 

तथापि कृत्यल्युटो बहुलम् {\qt इत्यस्यैव भीमादयोऽपादाने} इत्ययं 

प्रपञ्च इति भावः~॥ 

(करणल्युद्समर्थनभाष्यम् ) 

अथर्वा शब्दैरपि शब्दा व्याकियन्ते~। तद्यथा \textendash\ 

किमर्थमिति~। भवे च तद्भितः {\qt प्रोक्तादयश्च} इतीति पाठः~। 

इत्येव \textendash\ इत्यनेनैव~। तथापि स किं न चोदितः स्यादिति काकुः~॥ छा.~॥ 

७ {\qt आदिपदेन भवे च तद्धितः} इत्यस्यापि संग्रहः~॥ 

८ परिहरति \textendash\ पुरस्तादिति~॥ आचार्येण \textendash\ वार्तिककृता~॥ 

इदानीं \textendash\ द्वितीयप्रणयनकाले~। अनेन वातिंकादावपि सूत्वब्यवहारः 

सूचितः~॥ छाया~॥ 

९ लक्षणेति~। सामान्यसूत्रत्वविशेषसूत्रत्वाभ्यामित्यर्थः~॥ 

मूलसूत्रेति~। पाणिनीयसूज्रेत्यर्थः~॥ छाया~॥ 

१० यथेति~। विशेषणमित्यस्याग्रिमसूत्राणि प्रपञ्चभूतानि~। 

तथा {\qt अलुगुत्तरपदे} इत्यस्याग्रिमाणि~॥ छाया~॥ 

११ आचार्या महर्षयो वेदसंमिता आत्मोक्तं न निर्वर्तयन्ति हि 

यतो व्युत्पादनाय व्याख्येयेऽर्थे तेषां तात्पर्यात्~। आत्यन्तिकदोषासंभ \textendash\ 


वान्नेदृशलाघवादरस्तेषामित्येवं भाष्यस्य युक्तत्वेऽपि कैयटोक्ता
दृष्टान्ते \textendash\ 

नोपपत्तिरयुक्तेत्याशयेनाह \textendash\ पूर्वमिति~॥ छाया~॥ 

१२ करणल्युटमप्यत्रपक्षे समर्थयते भाष्ये \textendash\ अथचेति~॥ छया~॥ 

व्याकरणाधिकरणंम् ] 

पस्पशाह्रिकम्~। 



 

गौरित्युक्ते सर्वं संदेहा निवर्तन्ते, नाश्वो न गर्दभ 

इति~॥ 

( प्रदीपः ) गौरित्युक्त इति~। साल्लादिमति यदा कश्चि \textendash\ 

{\qt त्प्रति अयं गौः} इत्युच्यते तदाऽत्र वाचकान्तराणां निवृत्तिः 

कृता भवति, एवमेकस्मिनुदाहरणे उपन्यस्ते सर्वाणि तत्सदृ \textendash\ 

शानि शब्दान्तराणि प्रतीयन्ते~॥ 

(उद्दयोतः ) प्रतीयन्त इति~। एवं च विपरीतव्याबृत्तिः 

सदृशसंग्रहश्च व्याकृतिरिति भावः~॥ एतेन शैब्दाप्रतिपत्षिरप्यत्र 

पक्षे उद्धृता~॥ 

( अनुद्धतदोषप्रदर्शकभाष्यम् ) 

अयं तर्हि दोषः \textendash\ भेवे \textendash\ प्रोक्तादयश्च तद्धिताः 

इति~॥ 

(समाधानोपक्रमभाष्यम् ) 

एवं तर्हि \textendash\ 

(१५ सिद्धान्तवार्तिकम्~॥ ६~॥ ) 

~॥~॥ लक्ष्यलक्षणे व्याकरणम्~॥~॥ 

(भाष्यम्)

लक्ष्यं च लक्षणं चैतत्समुदितं व्याकरणं भवति~॥ 

किं पुनर्छक्ष्यम्, किं लक्षणम् ? 

शब्दो लक्ष्यः,सूत्रं लक्षणम्~॥ 

(उद्दयोतः ) व्यावर्त्यव्यावतकरूंपलक्ष्यलक्षणस्यातिप्रसक्तत्वात्पृ \textendash\ ~। 

च्छति \textendash\ भाष्ये \textendash\ किं पुनरिति~॥ 

(समाधानबाधकभाष्यम् ) 

एवमप्ययं दोषः \textendash\ समुदाये व्याकरणशब्दः प्र \textendash\ 

वृत्तोऽवयवे नोपपद्यते सूत्राणि चाप्यधीयान 

इष्यते \textendash\ वैयाकरण इति~॥ 

(समाधानसाधकभाष्यम् ) 

नैष दोषः~। समुदायेषु हि शब्दाः प्रबृत्ता 

अवयवेष्वपि वर्तन्ते~। तद्यथा \textendash\ पूर्वं पञ्चालाः, उत्तरे 



१ शब्दाप्रतिपत्तिरिति~। शव्दो व्याकरणमिति पक्षे करणल्यु \textendash\ 

डन्तव्याकरणशब्देन व्याक्रियन्ते \textendash\ विभज्यन्तेऽपशब्दा येनेत्यर्थे शब्दा \textendash\ 

{\qt प्रतिपत्तिः} इति वार्तिकोक्तो दोषोऽपि निराकृत इत्यर्थः~॥ 

२ {\qt भवे च \textendash\ } इति~। प्राग्वत् पाठः~। दोपैक्यसूचनाय तथानुवादो 

वा~॥ छाया~॥ अत्र {\qt भवे च तद्धितः \textendash\ प्रोक्ता} इति पाठश्छायादृष्टः~॥ 

३ सिद्धान्तभूततृतीयपक्षमाह \textendash\ एवं तर्हिति~। पक्षद्वयस्योक्त \textendash\ 

रीत्या दुष्टत्वे इत्यर्थः~॥ वार्तिके लक्ष्यलक्षणे इति द्विवचनेन
तत्त्वस्य 

व्यासज्न्यवृत्तिता सूचिता~॥ छाया~॥ 

४ व्याकरणशब्दो योगरूढया तत्पर इति भावः~। अत्र पक्षे 

करणत्वं समुदायस्यापि~। अत एव प्रतिपत्तिरपि~। समुदायावयवयो \textendash\ 

रमेदांत्षष्ट्यर्थोपपत्तिर्वृक्षस्य शाखेति्वत्~। भवे तद्धितस्यासंदेह एव,


प्रोक्ततद्वितोऽपि समुदायस्यानन्यत्वान्नेतुं शक्य इति न प्रागुक्तदोष 

इति बोध्यम्~॥ छाया~॥ 

५ {\qt रूपलक्षणस्या} इति मुद्रितपाठः~॥ 





पञ्चालाः, तैलं भुक्तम्, घृतं भुक्तम्, शुक्लो नीर्लः 

कृष्ण इति~॥ 

एवमयं समुदाये व्याकरणशब्दः प्रवृत्तोवऽय \textendash\ 

वेऽपि वर्तते~॥ 

( प्रदीप ) पूर्वं पञ्चाला इति~। जनपदान्तरनिवृत्तिवि \textendash\ 

वक्षायामेकदेशेऽपि समुदायरूपारोपात्प्रयोगः~॥ तैलमिति~। 

यदौषधसंस्कृता घृततैलमात्रा भवति तदैतदुदाहरणम्~। आकृ \textendash\ 

तिवाचित्वे तु घृततैलशब्दयोः संस्थानप्रमाणनिरपेक्षा सर्वत्र 

मुख्या वृत्तिः~॥ शुक्ल इति~। अशुक्लेऽप्यवयवेऽवयवांन्तरस्य 

शौक्लयात्समुदायस्य शुक्लत्वे सति आरोपात्प्रयोगः~॥ 

( उद्दयोतः ) जनपदान्तरेति~। भावप्रधानो निर्देशः~॥ 

यदेति~। मात्रा \textendash\ परिच्छिन्नोऽशः~। औषधसंस्कृते तत्र समुदाये 

घृतादिशब्दस्य रूढिरिति भावः~॥ भशुक्लेऽप्यवयवे इति~। {\qt प्रयोगः} 

इत्यनेनान्वेति~॥ आरोपादिति~। समुदायस्य शुक्लत्वे अशुक्लावयवै 

प्रयोगे च हेतुः~। तत्र समुदाये शुक्लत्वारोपे हेतुः अवयघान्तरस्य 

{\qt शौक्लयात्} इति~। वरनैर्कदेशे {\qt पुष्पितं वनम्} इतिवदेकदेशेऽपि 

शुक्के शुक्लः पदः इति व्यवह्ार इति वोध्यम्~॥ 

( प्रथमपक्षाभ्युपगमभाष्यम् ) 

अथ वा पुनरस्तु सूत्रम्~॥ 

( प्रथमाक्षेपस्मारणभाष्यम् ) 

ननु चोक्तम् \textendash\ सूत्रे व्याकरणे षष्ट्यर्थोऽनुपपन्नः 

इति~॥ 

( प्रथमाक्षेपनिरासभाष्यम् ) 

नैष दोषः~। द्यपदेशिवद्भावेन भविष्यति~॥ 

( प्रदीपः ) व्यपदेशिवद्भावेनेति~। यथा {\qt राहोः शिरः} 

इत्येकस्मिन्नपि वस्तुनि शै्दार्थभेदाद्भेदव्यवहारः, एवमिहापि 

व्याकरणशब्देन शास्त्रस्य व्याकृतिक्रियायां करणरूपत्वमुच्यते~। 

सूत्रशब्देन तु समुदायरूपता \textendash\ इति भेदव्यवहार उपपद्यते~॥ 

( उद्व्योतः ) शब्दार्थभेदादिति~। अनेकावस्यायुक्तं शिरो 



६ {\qt नीलः कपिलः कृष्णः इति कपिलशब्दमधिकं क्वच्वित्पठन्ति~।}

७ नन्वारोपितस्यान्यत्रारोपाभावेन कथमेतदिति चेत्~। न~। 

भाष्यप्रामाण्यादत्र तथाङ्गीकारात्~। वस्तुत इत एवारुचेः प्रकारान्त \textendash\ 

रेण भाष्यं योजयन्नाह \textendash\ वनैकदेश इति~। पुष्पिते सतीति शेष:~॥ 

शुक्ले वास्तवे~॥ पट इति~। पटत्वारोपादिति भावः~॥ छाया~॥ 

८ लोकव्यवहारानुरोधेनोक्तमाद्यपक्षं समर्थयते \textendash\ अथवेति~॥ 

छाया~॥ 

५ व्यपेति~। मुख्यव्यवहारतुल्यत्वकरणेन तदुपपत्तिर्भविष्यती \textendash\ 

त्यर्थः~॥ छाया~॥ 

१० शब्दार्थेति~। शब्देर्थे च राहुत्वारोषे सति राहुशब्दः 

समुदायशब्दः सम्पद्यते~। एवं व्याकृतिकरणत्वपरे व्याकरणपदे सूत्र \textendash\ 

शब्दसमुदायो ब्याकरणम्~। तथा च समुदायतदवयवथोभेदे व्याक 

रणस्य सूत्रमिति व्यवहारः~। पूर्व समुदाे रूढिर्व्याकरणशब्दस्योक्ता, 

इदानीं तु आरोप इति पक्षयोभेदः~॥ 

८० 

उद्दयोतपरिवृतप्रदीपप्रकाशितमहाभाष्ये~। [१ अ. १ पा. १ पस्पशाह्रिके 



राहुशब्दार्थः, यात्कैचिदेकावस्थायुक्तं तत् शिरःशब्दार्थः~। तादृ \textendash\ 

आराहुशब्दार्थस्य तादृशशिरःपदार्थोऽवयव इति पषष्ठ्यथौपपत्तिरिति 

भावः~॥ सूत्रशब्देन तु समुदायरूपतेति~। अनेकावस्थायुक्तस्य 

तस्य व्याकृतिक्रियाकरणावस्थाविशिष्टमवयव इति तद्रूपावयवस्यायं 

समुदाय इत्यर्थः~॥ ईर्दृशे स्थले विकल्पात्मकं ज्ञानं वस्तुशून्यमेव~। 

व्यपदेशिवद्भावविषयेऽप्येवमेव, {\qt शब्दज्ञानानुपाती वस्तुशून्यो 

विकल्पः} इति योगसूृत्रे पतञ्जल्युक्तेरित्यन्ये~॥ 

( द्वितीयाक्षेपस्मारणंभाष्यम् ) 

यदप्युच्यते \textendash\ {\qt *शब्दाप्रतिपत्तिः}*इति~। 

न हि सूत्रत एव शब्दान्प्रतिपद्यन्ते~। 

किं तर्हि ? व्याख्यानतश्च \textendash\ इति~। 

परिहृतमेतत् \textendash\ तदेवं सूत्रं विगृहीतं व्याख्यानं 

भवतीति~॥ 

(परिहारबीधकस्मारणभांष्यम् ) 

ननु चोक्तम् \textendash\ न केवलनि चर्चापदानि व्या \textendash\ 

ख्यानम् \textendash\ वृद्धिः \textendash\ आत्{\qt ऐजिति~। किं तर्हि ? 

उदाहरणं \textendash\ प्रत्युदाहरणं \textendash\ वाक्थाध्याहारः \textendash\ इत्येत \textendash\ 

त्संमुदितं व्याख्यानं भवति} इति~॥ 

(परिहारसाधकभाष्यम् ) ढ 

अविजानत एतदेवं भवति~। सूत्रत एव हि~। 

शब्दान् प्रतिपद्यन्ते~॥ 

आतश्च सूत्रत एव~। यो ह्युत्सूत्रं कथयेन्नादो~। 

गृह्यत~॥ 

( प्रदीपः ) सूत्रत एवेति~। पदच्छेदादिभिः सूत्रार्थस्यै \textendash\ 

वाभिव्यञ्जनात्~॥~। 

आत इति~। निपातः~। अतश्च हेतोरित्यर्थः~॥ नाद 

इति~। नैतदित्यर्थः~॥ अथवा नादोऽर्थरहितत्वात् घोषमात्रमेव 

गृह्येतेत्यर्थः~॥ 



१ ननूभाभ्यामपि शब्दाभ्यामष्टाध्यायी प्रतिपाद्यते इति प्रागुक्त \textendash\ 

स्बात् {\qt सूत्रशब्देन तु समुदायरूपता}इत्यसंगतमत आह \textendash\ अनेकेति~॥ 

अवयवावयविभावस्य तुल्यत्वेन समानविषयतथा दुष्टान्तता नतु 

समानविशेष्यकत्वादिनेति ध्वनयत्रुदाहरणार्थमाह \textendash\ तद्रूपेति~। अमु \textendash\ 

रवयेऽपि भेदे औपाधिकभेदकल्पनमुभयत्र तुल्यमिति यावत्~॥ छाया~॥ 

२ एवमन्वयमुखेनोकत्वा व्यतिरेकमुखेनाह \textendash\ आतश्चेति~॥ छाया~॥ 

३ {\qt धोषमात्रमेवेत्यथः} इति मुद्रितपाठः~॥ 

४ ननु मर्णज्ञानमेव प्रयोजनमिति कथं शङ्कावसरोऽत आह \textendash\ 

लोकेति~॥ 

५ ननु गुरुशिष्यव्यवहारविशेषे उपदेशशब्दप्रवृत्तेरत्र कथं तत्प्र \textendash\ 

योगः सर्वस्या अपि अष्टाध्याय्याः पाणितिप्रोक्तत्वादत आह \textendash\ पाणि \textendash\ 

नय इति~॥ छाया~॥ 

६ {\qt वृत्तिसमवाय इति~। अत्रै धर्मनियमः} इत्यत्रेव षष्ठी \textendash\ 

संमासः~। योगविभागस्यागतिकगतित्वाच्चतुर्वीसमास्तो न, तत्तुल्यत्बा \textendash\ ~। 





धुःब्दस्यानुशासनमिति भावः~॥ 

(उद्दयोतः ) भाष्ये \textendash\ अविजानत इति~। मन्दबुद्धेरित्यर्थः~॥ 

सून्नत एवेति~। एतन्मूलकमेव पठ्यते \textendash\ 

{\qt सूत्रेष्वेव हि तत्सर्वं यद्वृत्तौ यच्च वार्तिके~। इति~॥}

(इति व्याकरणाधिकरणम्~॥ ) 

 \textendash\ कः 

(अथ वर्णोपदेशाधिकरणम् ) 

(आक्षेपभाष्यम् ) 

अथ किमर्थो वर्णानामुपदेशः] 

(प्रदीपः) किमर्थ इति~। न हि वर्णोपदेशेन कस्यचित्सा \textendash\ 

( उद्दयोतः ) ननु लोर्बैप्रसिद्धमातृकापाठेनैव वर्णश्ञानसंभवा \textendash\ 

न्माहेश्वरो वर्णसमाम्नायः किमर्थ इति पृच्छति \textendash\ भाष्ये \textendash\ किमर्थ 

इति~। पीणिनये महादेवकृत इत्यर्थः~॥ नमु साधुत्वान्वाख्यानं 

फलमत आह \textendash\ न हीति~॥ 

(१६ समाधानवार्तिकरम्~॥ १~॥ ) 

~॥ \#~॥ वृत्तिसमवायार्थ उपदेशः~॥~॥ 

( भाष्यम् ) 

वृत्तिसमवायार्थौ वर्णानामुपदेशः कर्तव्यः~। 

किमिदं वृत्तिस्समवायार्थ इति ? 

वृत्तये समवायः \textendash\ वृर्त्तिसमवायः~। वृत्त्यर्थौ वा 

समवायः \textendash\ बृत्तिसमवायः~। वृत्तिप्रयोजनो वा सम \textendash\ 

वायः \textendash\ वृत्तिसमवायः~॥ 

का पुनर्वृत्तिः ? शास्त्रप्रवृत्तिः~॥ 

अथ कः समवायः ? वर्णानामानुपूर्व्यैण संनिवेशः~॥ 

अथ क उपदेशः ? उच्चारणम्~॥ 

कुत एतत् ? दिशिरुच्चारणक्रियः~। उच्चार्य हि 

वर्णानाह \textendash\ उपदिष्टा इमे वर्णा इति~॥ 

( प्रदीपः ) वृत्तिसमवायार्थ इति~। लाघवेन शास्त्र \textendash\ 



देव सुप्सुपेत्यपि न~। भाष्ये त्रिविधविग्रहप्रदर्शनोपयोगस्त्वत्थम् \textendash\ 

वृत्तये \textendash\ लाघवेन शस्त्रप्रवृत्तये समवायः \textendash\ वर्णानामानुपू्यैण सन्नि \textendash\ 

वेशः~। यथा \textendash\ वर्णसमवायादेवाजादिसंज्ञा सिध्यन्ति, तेन च {\qt इको 

यणचि} इति शास्त्रं लघुभूतं सम्पद्यते, अन्यथा {\qt इउकलवर्णानां यव} \textendash\ 

रलाः इत्येवं गुरुतरं सूत्रं स्यादिति प्रथमविग्रहप्रदर्शनोपयोगः~। 

वृत्त्यथौं वा समवाय इति कर्मधारयः~। अस्मिन् पक्षे लक्षणया वृत्ति \textendash\ 

शब्दार्थः साधुत्वोपयोगिशास्प्रवृत्त्यनुकूलसमवायः~। यथा \textendash\ इउक्रल 

इति यवरल इत्येतेषाञ्च वर्णसमाश्नाये क्रमेण सन्निवेशात् {\qt इग्यणः सम्प्र \textendash\ }


इति सूत्रे यथासंख्यमनु \textendash\ इत्यस्य प्रवृत्तिः~। वर्णसश्निवेशसत्वादेव 

समसंख्यत्वोपपत्तौ यथासंख्यावतारं इति द्वितीयविग्रहप्रदर्शनोपयोगः~। 

वृत्तिप्रयोजनो वा समवाय इत्यस्य तु वर्णसन्निवेशादेव णादीनामित्संशा \textendash\ 

प्रवृत्तिः फलम्~। तेन प्रत्याहारादिसिद्धिः, ततश्च {\qt ढुलोपे \textendash\ } इत्यादि \textendash\ 

शास्त्रप्रवृत्तिरिति न साक्षाञ्जाघवेन शास्त्रप्रवृत्तिः फलमिति
प्रथमतोऽस्य 

भदः~। एवञ्च प्रथमपक्षे वृत्तिशब्दस्य शास्तप्रवृत्तिरथः, द्वितीये \textendash\ सम \textendash\ 

वायः, तृतीये च इत्संशादिसंशाशास्प्रवृत्तिरथं इत्युपपादितं भवति~॥

\end{document}