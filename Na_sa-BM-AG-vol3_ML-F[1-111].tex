\documentclass[11pt, openany]{book}
\usepackage[text={4.65in,7.45in}, centering, includefoot]{geometry}
\usepackage[table, x11names]{xcolor}
\usepackage{fontspec,realscripts}
\usepackage{polyglossia}
\setdefaultlanguage{sanskrit}
\setotherlanguage{english}
\setmainfont[Scale=1]{Shobhika}
% \defaultfontfeatures[Scale=MatchUppercase]{Ligatures=TeX} 
% \newfontfamily\sanskritfont[Script=Devanagari,Mapping=devanagarinumerals]{Shobhika}
\newfontfamily\s[Script=Devanagari, Scale=0.8]{Shobhika}
\newfontfamily\regular{Linux Libertine O}
\newfontfamily\en[Language=English, Script=Latin]{Linux Libertine O}
\newfontfamily\na[Script=Devanagari, Scale=1.1, Color=purple]{Shobhika-Bold}
\newfontfamily\qt[Script=Devanagari, Scale=1, Color=violet]{Shobhika-Regular}
\newcommand{\devanagarinumeral}[1]{%
	\devanagaridigits{\number \csname c@#1\endcsname}} % for devanagari page numbers
%\usepackage[Devanagari, Latin]{ucharclasses}
%\setTransitionTo{Devanagari}{\s}
%\setTransitionFrom{Devanagari}{\regular}
\XeTeXgenerateactualtext=1 % for searchable pdf
\usepackage{enumerate}
\pagestyle{plain}
\usepackage{fancyhdr}
\pagestyle{fancy}
\renewcommand{\headrulewidth}{0pt}
\usepackage{afterpage}
\usepackage{multirow}
\usepackage{multicol}
\usepackage{mdframed,lipsum}
\usepackage{wrapfig}
\usepackage{vwcol}
\usepackage{microtype}
 \usepackage{amsmath,amsthm, amsfonts,amssymb}
\usepackage{mathtools}% <\textendash\ new package for rcases
\usepackage{graphicx}
\usepackage{longtable}
\usepackage{setspace}
\usepackage{footnote}
\usepackage{perpage}
\MakePerPage{footnote}
%\usepackage[para]{footmisc}
%\usepackage{dblfnote}
\usepackage{xspace}
\usepackage{array}
\usepackage{emptypage}
\usepackage{hyperref}% Package for hyperlinks
\hypersetup{colorlinks,
citecolor=black,
filecolor=black,
linkcolor=blue,
urlcolor=black}

\newcommand\blfootnote[1]{%
 \begingroup
 \renewcommand\thefootnote{}\footnote{#1}%
 \addtocounter{footnote}{-1}%
 \endgroup
}

\mdfdefinestyle{MyFrame}{%
 linecolor=black,
 outerlinewidth=2pt,
 %roundcorner=20pt,
 innertopmargin=4pt,
 innerbottommargin=4pt,
 innerrightmargin=4pt,
 innerleftmargin=4pt,
 leftmargin = 4pt,
 rightmargin = 4pt
 %backgroundcolor=gray!50!white}
 }

\begin{document}
\cfoot{}

\vspace*{\fill}

\begin{center}
\includegraphics[width=0.3\linewidth]{Cropped_Images/Figures/Figure-1.jpg}
\end{center}

\begin{mdframed}
\begin{center}
\textbf{\LARGE GAEKWAD'S}\\

\vspace{3mm}
\textbf{\LARGE ORIENTAL SERIES}\\

\vspace{3mm}
\rule{0.2\linewidth}{0.5pt}\\

\vspace{3mm}
{\Large No. CXXIV}
\end{center}
\end{mdframed}

\vspace*{\fill}

\newpage
\begin{center}
\textbf{\huge NĀTYAŚĀSṬRA}

\vspace{5mm}
WITH THE COMMENTARY OF

\vspace{5mm}
\textbf{\LARGE ABHINAVAGUPTA}

\vspace{3mm}
\rule{0.2\linewidth}{0.5pt}

\vfil
Edited

with an Index

\vspace{2mm}
BY

\vspace{2mm}
\textbf{M. RAMAKRISHNA KAVI, M. A.}

\rule{0.2\linewidth}{0.5pt}

VOL. III

\rule{0.2\linewidth}{0.5pt}

\vfil
1954

\vspace{2mm}
ORIENTAL INSTITUTE

BARODA
\end{center}

\newpage
\vspace*{\fill}
\begin{center}
Printed by B. Madhava Rau, M. A., B. L., at Ananda Press, Madras, and Published on behalf of the Maharaja Sayajirao University of Baroda by G. H. Bhatt, Director,\\
Oriental Institute, Baroda.

\vspace{5mm}
\textbf{Price Rs. 15}
\end{center}

\vspace*{\fill}

\newpage
\begin{center}
\textbf{\Large PREFACE}\\

\rule{0.2\linewidth}{0.5pt}
\end{center}

The third Volume of Bharata's Nāṭyaśāstra with Abhinavagupta's Commentary, comprising chapters 19 \textendash\ 27, is now published as No. CXXIV of the Gaekwad's Oriental Series. Although the Press \textendash\ Copy of the third Volume was sent to the Press as early as 1936, the work could not see the light of the day uptill now on account of many unforeseen difficulties. We are indeed very sorry that scholars in Iṇdia and abroad had to wait for such a long period. We assure them that the remaining portion of the text will be published without any delay, as the Press \textendash\ copy is now ready for printing.\\

The learned editor, Shri M. Ramakrishna Kavi, M. A. has promised to write a long introduction and a critical Note on the MSS. of the Nāṭyaśāstra and Abhinavabhāratī, which will appear in the next Volume.\\

We are thankful to Shri B. Madhava Rau, M. A., B. L., the Proprietor of the Ananda Press, Mylapore, Madras, for hearty Co \textendash\ operation.

\vspace*{\fill}

\begin{table}[h!]
 \centering
 \begin{tabular}{m{5em} m{15em} m{8em}}
 BARODA, & \multirow{2}{*}{$\Bigg\}$}& \emph{\en General Editor.}\\
 20 \textendash\ 1 \textendash\ 1954.& &
 \end{tabular}
\end{table}

\newpage
\begin{center}
\textbf{\LARGE विषयसूचि}\\

\vspace{5mm}
\textbf{\large एकोनविंशोऽध्यायः}\\

\noindent
\begin{tabular}{m{10em} m{2em} m{2em} m{2em} c c}
& & & & श्लोकसंख्या & पुटसंख्या\\
इतिवृत्तं द्विधा & . & . & . & २ & २\\
आधिकारिकम् &. &.& .& ४ &३\\
प्रासङ्गिकम्& . &. &. &४ &३\\
आनुषङ्गिकम्& .& . &. &४ &४\\
पञ्चावस्थाः& .& .& .& ८& ६\\
फलारम्भः& .& .& .& ९& ६\\
प्रयत्नः& .& .& .& १०& ७\\
प्राप्तिसम्भवम्& .& .& .& ११ &७\\
फलप्राप्तिः& .& .& .& १२ &७\\
फलयोगः& .& . &. &१३& ८\\
सन्धिभेदः &. &. &. &१७& १०\\
लोपत्रयम्& .& .& .& १८& ११\\
अर्थप्रकृतयः &. &. &. &२०& १२\\
बीजम् &. &. &. &२२ &१३\\
बिन्दुः& .& .& .& २३& १४\\
पताका &. &. &. &२४ &१५\\
प्रकरी& . &. &. &२५ &१५\\
कार्यम्& .& . &. &२६ &१५\\
अनुसन्धिः& . &. &.& २८& १६\\
मुखम्~~~~~~ \multirow{5}{*}{$\Bigg\}$} & & & & & \\
प्रतिमुखम्& & & & & \\
गर्भः & .& .& .& ३७& २३ \\
विमर्शः & & & & & \\
निर्वहणम् & & & & & \\
डिमसमवकारौ& . &. &. & ४४ &३०\\
चतुस्सन्धियुक्तौ& .& .& . &४४& ३०\\
व्यायोगेहामृगौ &. &. &. &४६& ३०\\
द्विसन्धी& .& .& .& ४७& ३०\\
सन्धीनामङ्गकल्पं& .& . &. &४९ &३१\\
अङ्गप्रयोजनम्& .& .& .& ५२& ३२\\
उपक्षेपाद्यङ्गानि& .& .& .& ५६ &३४
\end{tabular}
\end{center}

\lfoot{A}

\newpage
\lfoot{}
\fancyhead[CE,CO]{\thepage}
\setcounter{page}{6}
\renewcommand{\thepage}{\roman{page}}
% vi

\begin{center}
\begin{tabular}{m{10em} m{2em} m{2em} m{2em} c c}
& & & & श्लोकसंख्या & पुटसंख्या\\
प्रतिमुखस्य& .& .& .& ६०& ३४ \\
गर्भस्य& .& .& .& ६२& ३४\\
विमर्शस्य& .& .& .& ६४& ३५\\
काव्यसंहारः& .& .& . &६६ &३५\\
उपक्षेपः& .& .& .& ६९& ३८\\
परिकरः &.& . &. &७० &३९\\
परिन्यासः& .& .& .& ७०& ३९\\
विलोभनम्& .& .& .& ७१ &३९\\
युक्तिः& .& .& .& ७१ &३९\\
प्राप्तिः &. &. &. &७२& ३९\\
समाधानम् &.& .& .& ७२ &३९\\
विधानम्, परिभावना& .& . &. &७३ &४०\\
उद्भेदः, करणम्& .& .& . &७४ &४१\\
भेदः,& . &. &. &७५& ४१\\
विलासः &. &. &. &७६ &४२\\
परिसर्पः& .& .& .& ७६& ४३\\
विधूतम्, तापनम्& .& .& .& ७७& ४३\\
नर्मनर्मद्युती& .& .& .& ७८& ४४\\
प्रगयणम्, निरोधः &. &. &. &७९ &४५\\
पर्युपासनम्& . &. &. &८० &४५\\
पुष्पम् &. &. &. &८० &४६\\
वज्रोपन्यासौ& .& .& .& ८१ &४६\\
वर्णसंहारः& .& .& .& ८२& ४७\\
अभूताहरणम्& .& .& .& ८२ &४७\\
मार्गः& . &. &. &८२ &४७\\
रूपम्& .&. &. &८३ &४८\\
उदाहरणम् &. &. &. &८४ &४८\\
क्रमः& .& . &. &८४ &४९\\
सङ्ग्रहः, अनुमानम् &. &. &. &८५ &४९\\
प्रार्थना, साक्षेपः &. &. &. &८६ &५०\\
तोटकम्, अधिबलम्& . &. &.&८७ &५१\\
उद्वेगः& .& .& .& ८८ &५१\\
विद्रवः& .& .& .& ८८ &५२\\
अपवादः &. &. &. &८९ &५२\\
संफेट& .& .& .& ८९ &५३\\
द्रवः, शक्तिः& .& .& .& ९० &५३
\end{tabular}
\end{center}

\newpage
% vii
\begin{center}
\begin{tabular}{m{10em} m{2em} m{2em} m{2em} c c}
& & & & श्लोकसंख्या & पुटसंख्या\\
व्यवसायः प्रसङ्गः,& .& .& .& ९१ &५४\\
द्युतिः, खेदः& .& .& .& ९२& ५४\\
प्रतिषेधः, विरोधनम्& .& .& .& ९३ &५५\\
आदानम्, छादनम्& .& .& .&९४ &५५\\
प्ररोचना, व्याहारः& .& .& .&९५ &५६\\
युक्तिः, विचलना& .& .& .&९६ &५६\\
सन्धिः& .& .& .&९७ &५७\\
निरोधः, ग्रथनम्& .& .& .&९८ &५७\\
निर्णयः& .& .& .&९९ &५७\\
परिभाषणम्& .& .& .&९९ &५८\\
द्युतिः, आनन्दः& .& .& .&१०० &५८\\
समयः, प्रसादः& .& .& .&१०१ &५९\\
उपगूहनम्, भाषणम्& .& .& .&१०२ &५९\\
पूर्ववाक्यम्, काव्यसंहारः& .& .& .&१०३ &६०\\
प्रशस्तिः,& .& .& .&१०४ &६१\\
सामभेददण्डादयः& .& .& .&१०७ &६३\\
विष्कम्भः& .& .& .&११२ &६५\\
चूलिका& .& .& .&११३ &६५\\
प्रवेशकः& .& .& .&११४ &६५\\
अङ्कावतारः& .& .& .&११५ &६५\\
अङ्कमुखम्& .& .& .&११६ &६५\\
भाणलक्षणम्& .& .& .&११७ &६६\\
लास्याङ्गानि& .& .& .&११९ &६६\\
गेयपदम्& .& .& .&१२१ &६८\\
स्थितपाठ्यम्, आसीनम्& .& .& .&१२५ &६९\\
पुष्पगण्डिका& .& .& .&१२६ &६९\\
प्रच्छेदकम्& .& .& .&१२९ &७०\\
त्रिमूढकम्& .& .& .&१३० &७१\\
सैन्धवकम्& .& .& .&१३१ &७२\\
द्विमूढकम्& .& .& .&१३३ &७३\\
उत्तमोत्तमकम्& .& .& .&१३४ &७३\\
उक्तप्रत्युक्तम्& .& .& .&१३५ &७५\\
चित्रपदम्& .& .& .&१३६ &७६\\
भाविकम्& .& .& .&१३७ &७७\\
नाटकानाट्यलक्षणम्& .& .& .&१४० &७९\\
लोकानुसारि नाट्यं& .& .& .&१५१ &८१\\
& \multicolumn{3}{c}{\rule{0.2\linewidth}{0.5pt}} & 
\end{tabular}
\end{center}

\newpage
% viii

\begin{center}
\begin{tabular}{m{10em} m{2em} m{2em} m{2em} c c}
& \multicolumn{3}{c}{\textbf{\large विंशोऽध्यायः}} & \\
& & & & श्लोकसंख्या & पुटसंख्या\\
वृत्तीनां समुत्थानम्& .& .& .&१ &८३\\
वृत्तीनां समारम्भः& .& .& .&२ &८५ \\
वाग्भूयिष्ठा भारती& .& .& .&८ &८६\\
सत्त्वस्वरूपा सात्त्वती& .& .& .&१२ &८६\\
विचित्रवेषा कैशिकी& .& .& .&१३ &८७\\
नानाचारी आरभटी& .& .& .&१४ &८७\\
न्यायः& .& .& .&१९ &८९\\
भारत्यानीनां ऋगादि समुद्भवः& .& .& .&२५ &९०\\
भारती& .& .& .&२६ &९१\\
भारतीभेदाः& .& .& .&२७ &९२\\
प्ररोचना& .& .& .&२८ &९२\\
आमुखम्& .& .& .&३१ &९३\\
पञ्च आमुखाङ्गानि& .& .& .&३३ &९३\\
कथोद्घातः& .& .& .&३५ &९४\\
प्रयोगातिशयः& .& .& .&३६ &९५\\
प्रवृत्तकम्& .& .& .&३७ &९५\\
अवलगितम्& .& .& .&३८ &९५\\
सात्त्वती& .& .& .&४१ &९६\\
सात्त्वतीभेदाः& .& .& .&४४ &९७\\
उत्थापकः& .& .& .&४५ &९८\\
परिवर्तकः& .& .& .&४६ &९८\\
संलापकः& .& .& .&४८ &९८\\
सङ्घात्यकः& .& .& .&५० &९९\\
कैशिकी& .& .& .&५३& १००\\
नर्म& .& .& .&५७ &१००\\
नर्म स्फुञ्जः& .& .& .&५९ &१०१\\
नर्म स्फोषः& .& .& .&६० &१०२\\
नर्म गर्मः& .& .& .&६१ &१०२\\
आरभटी& .& .& .&६४ &१०३\\
संक्षिप्तकम्& .& .& .&६८ &१०४\\
अवपातः& .& .& .&६९ &१०४\\
वस्तूत्थापनम्& .& .& .&७० &१०४\\
सफेटः& .& .& .&७१ &१०५\\
वृत्तीनां रसप्रयोगः& .& .& .&७३ &१०५\\
& \multicolumn{3}{c}{\rule{0.2\linewidth}{0.5pt}} & 
\end{tabular}
\end{center}

\newpage
% ix

\begin{center}
\begin{tabular}{m{10em} m{2em} m{2em} m{2em} c c}
& \multicolumn{3}{c}{\textbf{\large एकविंशोऽध्यायः}} & \\
& & & & श्लोकसंख्या & पुटसंख्या\\
आहार्याभिनयः& .& .& .&१ &१०८\\
नेपथ्यम्& .& .& .&४ &१०९\\
पुस्तस्य संधिमादिः& .& .& .&६ &१०९\\
व्याजिमः, वेष्टिमः& .& .& .&८ &११०\\
माल्यं षञ्चविधम्& .& .& .&११ &११०\\
चातुर्विध्यमाभरणस्य& .& .& .&१२& १११\\
आवेद्यं& .& .& .&१३& १११\\
आरोप्यम्& .& .& .&१४ &१११\\
चूडामण्यादिः शिरसः& .& .& .&१६ &१११\\
कर्णाभरणम्& .& .& .&१६ &१११\\
कण्ठाङ्गुलिभूषणानि& .& .& .&१७ &१११\\
मणिबयन्धभूषणम्& .& .& .&१८ &११२\\
कूर्परवक्षोविभूषणानि& .& .& .&१९ &११२\\
कटिभूषणम्& .& .& .&२० &११२\\
देवतास्त्रीभूषणानि& .& .& .&२२ &११३\\
लघुभूषणम्& .& .& .&४६ &११७\\
वेष्टितादिमाल्यविधिः& .& .& .&५१& ११७\\
विद्याधरीयक्षीप्रभृतीनां भूषणानि& .& .& .&५४ &११८\\
नागस्त्रीणाम्& .& .& .&५८ &११८\\
मुनिकन्यानाम्& .& .& .&५८ &११९\\
सिद्धस्त्रीणाम्& .& .& .&६०& ११९\\
गन्धर्वीणाम्& .& .& .&६१ &११९\\
राक्षसीनाम्& .& .& .&६२ &११९\\
सुरस्त्रीणाम्& .& .& .&६३ &११९\\
नागरस्त्रीणाम्& .& .& .&६५ &११९\\
आवन्त्यादीनाम्& .& .& .&६७ &१२०\\
आभीर्यादीनाम्& .& .& .&६८ &१२०\\
वेषो मलिनः& .& .& .&७५ &१२१\\
उपवर्णाः& .& .& .&८० &१२२\\
वर्तना& .& .& .&८८ &१२३\\
परभावप्रवेशनम्& .& .& .&९१ &१२३\\
पर्वतानां स्त्रीभावः& .& .& .&९२ &१२४\\
वर्णविधिः& .& .& .&९६ &१२५\\
द्वीपान्तरवासिनः& .& .& .&१०१ &१२५\\
राजादीनाम् वर्णस्थितिः& .& .& .&१०६& १२६
\end{tabular}
\end{center}

\newpage
% x

\begin{center}
\begin{tabular}{m{10em} m{2em} m{2em} m{2em} c c}
& & & & श्लोकसंख्या & पुटसंख्या\\
क्ष्मश्रुकर्म, तत्य भेदाः& .& .& .&११४ &१२७\\
वेषस्य नानाप्रयोगविधः& .& .& .&१२१ &१२९\\
वस्त्रविधिः& .& .& .&१२७ &१३०\\
तापसवेषाः& .& .& .&१३१ &१३०\\
सांग्रामिकवेषाः& .& .& .&१३५ &१३१\\
प्रतिशिरः& .& .& .&१३९ &१३१\\
मुकुटाः& .& .& .&१४१ &१३२\\
विद्याधरादिलक्षणम्& .& .& .&१४३ &१३२\\
अर्धमुकुटविधिः& .& .& .&१४८ &१३३\\
लम्बकेशः& .& .& .&१५० &१३३\\
प्रतिवेषः& .& .& .&१५२ &१३३\\
धूर्तशृङ्गारिबालानां च मूर्धजानि& .& .& .&१४३ &१३३\\
सजीवलक्षणम्& .& .& .&१६० &१३४\\
प्रहरणानि& .& .& .&१६४ &१३५\\
जर्जरम्& .& .& .&१७४ &१३७\\
दण्डकाष्ठस्य& .& .& .&१८२ &१३८\\
घटा& .& .& .&१८६ &१३९\\
प्रतिशीर्षकम्& .& .& .&१८८ &१३९\\
अवटुः& .& .& .&१९४ &१४०\\
लोकधर्मी, नाट्यधर्मी& .& .& .&२०३ &१४१\\
रङ्गे शस्रप्रोसविधिः& .& .& .&२२५ &१४५\\
 & \multicolumn{3}{c}{\rule{0.2\linewidth}{0.5pt}} & \\
& & & & & \\
& \multicolumn{3}{c}{\textbf{\large द्वाविंशोऽध्यायः}} &\\
ज्योष्ठाभिनयः& .& .& .&२& १५०\\
अलङ्कार इति किं& .& .& .&४& १५३\\
भावहावस्वरूपम्& .& .& .&६& १५३\\
भावः& .& .& .&८ &१५५\\
हावः& .& .& .&१०& १५६\\
हेला& .& .& .&११ &१५७\\
लीलादिदशस्वभावजाः& .& .& .&१२ &१५८\\
लीला& .& .& .&१४ &१५८\\
विलासः& .& .& .&१५& १५९\\
विच्छित्तिः& .& .& .&१६& १५९\\
विभ्रमः& .& .& .&१७ &१६०\\
किलकिञ्चितम्& .& .& .&१८& १६०
\end{tabular}
\end{center}

\newpage
% xi

\begin{center}
\begin{tabular}{m{10em} m{2em} m{2em} m{2em} c c}
& & & & श्लोकसंख्या & पुटसंख्या\\
मोट्टायितम्& .& .& .&१९ &१६१\\
कुट्टमितम्& .& .& .&२० &१६१\\
बिब्बोकः& .& .& .&२१ &१६१\\
ललितम्& .& .& .&२२ &१६१\\
विहृतम्& .& .& .&२४ &१६२\\
शोभा& .& .& .&२७ &१६२\\
कान्तिः दीप्तिश्च& .& .& .&२८ &१६३\\
माधुर्यम्& .& .& .&२९ &१६३\\
धैर्यम्& .& .& .&३० &१६३\\
प्रागल्भ्यम्, औदार्यम्& .& .& .&३१ &१६३\\
पुरुषाणां शोभादयः& .& .& .&३३ &१६५\\
विलासः& .& .& .&३५ &१६५\\
माधुर्यम्& .& .& .&३६ &१६५\\
स्थैर्यम्& .& .& .&३७ &१६६\\
गाम्भीर्यम्& .& .& .&३८ &१६६\\
ललितम्& .& .& .&३९& १६६\\
औदार्यम्& .& .& .&४० &१६६\\
तेजः& .& .& .&४१& १६७\\
सूचा& .& .& .&४५ &१६९\\
अङ्कुरः& .& .& .&४६ &१७०\\
शास्त्रा& .& .& .&४७ &१७१\\
नाट्यायितम्& .& .& .&४९ &१७३\\
निवृत्यङ्कुरः& .& .& .&५० &१७४\\
अभिनयाः& .& .& .&५१ &१७५\\
आलापः, प्रलापः& .& .& .&५४ &१७५\\
विलापः, अनुलापः& .& .& .&५५& १७६\\
संलापः, अपलापः& .& .& .&५६ &१७६\\
संदेशातिदेशौ& .& .& .&५७ &१७६\\
निर्देशव्यपदेशौ& .& .& .&५८& १७७\\
उपदेशापदेशौ& .& .& .&५९ &१७७\\
प्रत्यक्षपरोक्षादि सप्तप्रकाराः& .& .& .&६०& १७७\\
तेषामभिनय& .& .& .&७१ &१७९\\
सामान्याभिनयः& .& .& .&७३& १८०\\
आभ्यन्तरम्& .& .& .&७६& १८१\\
नाट्यं बाह्यम्& .& .& .&७७& १८१\\
बाह्वः अभिनयः& .& .& .&८० १८२
\end{tabular}
\end{center}

\newpage
% xii

\begin{center}
\begin{tabular}{m{10em} m{2em} m{2em} m{2em} c c}
& & & & श्लोकसंख्या & पुटसंख्या\\
शब्दनृत्यम्& .& .& .&८२& १८२\\
स्पर्शस्य& .& .& .&८३& १८३\\
रूपस्य& .& .& .&८४& १८३\\
रसगन्धनिर्देशः& .& .& .&८५ &१८३\\
इष्टनिदर्शनम्& .& .& .&८९ &१८४\\
सौमुख्यम्& .& .& .&९०& १८४\\
अनिष्टम्& .& .& .&९१ &१८४\\
मध्यस्थाभिनयः& .& .& .&९३ &१८५\\
आत्मस्य परस्थ च वर्णना& .& .& .&९५ &१८५\\
धर्मादिकामः& .& .& .&९६ &१८६\\
शृङ्गारः& .& .& .&९७ &१८६\\
सुखमूलं स्त्रियः& .& .& .&९९ &१८७\\
देवशीलाङ्गना& .& .& .&१०३ &१८८\\
आसुरी& .& .& .&१०५ &१८८\\
गन्धर्वसत्त्वा& .& .& .&१०७ &१८८\\
राक्षसशीला& .& .& .&१०९ &१८९\\
नागसत्त्वा& .& .& .&१११ &१८९\\
शकुनसत्त्वा& .& .& .&११३ &१८९\\
पिशाचसत्त्वा& .& .& .&११५& १९०\\
यक्षशीला& .& .& .&११७ &१९०\\
व्यालशीला& .& .& .&११८ &१९०\\
मनुष्यसत्त्वा& .& .& .&१२० &१९१\\
वानरसत्त्वा& .& .& .&१२२ &१९१\\
हस्तिसत्त्वा& .& .& .&१२४ &१९१\\
मृगसत्त्वा& .& .& .&१२६& १९१\\
मत्स्यसत्त्वा& .& .& .&१२७ &१९२\\
उष्ट्रसत्त्वा& .& .& .&१२९ &१९२\\
मकरसत्त्वा& .& .& .&१३० &१९२\\
खरसत्त्वा& .& .& .&१३२ &१९२\\
सूकरसत्त्वा& .& .& .&१३४ &१९३\\
हयसत्त्वा& .& .& .&१३६ &१९३\\
महिषसत्त्वा& .& .& .&१३८ &१९३\\
अजसत्त्वा& .& .& .&१४०& १९३\\
श्वानशीला& .& .& .&१४३ &१९४\\
गोसत्त्वा& .& .& .&१४४& १९४\\
धर्मः& .& .& .&१४८ &१९५
\end{tabular}
\end{center}

\newpage
% xiii

\begin{center}
\begin{tabular}{m{10em} m{2em} m{2em} m{2em} c c}
& & & & श्लोकसंख्या & पुटसंख्या\\
बाह्याभ्यन्तरः कामः& .& .& .&१५० &१९५\\
आभ्यन्तरस्त्री& .& .& .&१५४ &१९६\\
कुलजा कन्यका च& .& .& .&१५६ &१९६\\
कामोपजातिः& .& .& .&१५८ &१९७\\
इङ्गितानि& .& .& .&१६० &१९७\\
मुखरागः& .& .& .&१६३ &१९८\\
वेश्याया मदनातुरत्वम्& .& .& .&१६५ &१९८\\
कुलजाया मदनातुरत्वम्& .& .& .&१६६& १९८\\
दशस्थानानि कामः& .& .& .&१६९& १९९\\
अभिलाषः& .& .& .&१७३&२००\\
चिन्ता& .& .& .&१७५& २००\\
अनुस्मृतिः& .& .& .&१७८& २०१\\
उद्वेगः& .& .& .&१७८& २०१\\
विलापः& .& .& .&१८३ &२०२\\
उन्मादः& .& .& .&१८५ &२०२\\
व्याधिः& .& .& .&१८७ &२०२\\
जडता& .& .& .&१९० &२०३\\
मरणम्& .& .& .&१९१ &२०३\\
विप्रलम्भसूचना& .& .& .&१९६ &२०४\\
दूतीकृत्यम्& .& .& .&२०० &२०४\\
प्रच्छन्नकामित्वम्& .& .& .&२०१ &२०५\\
राजोपचारः& .& .& .&२०३ &२०५\\
प्रच्छन्नकामितत्वम्& .& .& .&२०५ &२०६\\
दिवासम्भोगः& .& .& .&२०८ &२०६\\
वासकः& .& .& .&२०९ &२०७\\
अष्टविधनायिकाः& .& .& .&२११ &२०८\\
वासकसज्जिका& .& .& .&२१३ &२०८\\
विरहोत्कान्ठिता& .& .& .&२१४ &२०८\\
स्वाधीनभर्तृका& .& .& .&२१५ &२०९\\
कलहान्तरिता& .& .& .&२१६ &२०९\\
खण्डिता& .& .& .&२१७ &२०९\\
विप्रलब्धा& .& .& .&२१८ &२०९\\
प्रोषितभर्तृका& .& .& .&२१९ &२०९\\
अभिसारिका& .& .& .&२२० &२०९\\
तासां कामतन्त्राः& .& .& .&२२१ &२१०\\
वेश्याङ्गनालक्षणम्& .& .& .&२२७ २११
\end{tabular}
\end{center}

\lfoot{B}

\newpage
\lfoot{}
% xiv

\begin{center}
\begin{tabular}{m{10em} m{2em} m{2em} m{2em} c c}
& & & & श्लोकसंख्या & पुटसंख्या\\
कुलजाङ्गनालक्षणम्& .& .& .&२२८& २११\\
प्रेष्या& .& .& .&२२९ &२११\\
उपचारविशेषाः& .& .& .&२३१& २११\\
समागमः& .& .& .&२३६ &२१३\\
वासोपचारः& .& .& .&२३७ &२१३\\
प्रेष्यादीनां शृङ्गारः& .& .& .&२४४ &२१५\\
विषण्णास्थितिः& .& .& .&२४८ &२१६\\
दुर्निमित्तानि& .& .& .&२५५ &२१७\\
नायकागमनम्& .& .& .&२५७ &२१७\\
स्त्रीसंलापः& .& .& .&२६० &२१८\\
स्त्रीणां कोपस्य योनयः& .& .& .&२६६ &२२०\\
वैमनस्यम्& .& .& .&२६७ &२२०\\
व्यलीकः& .& .& .&२६९ &२२०\\
विप्रियम्& .& .& .&२७१& २२०\\
मन्युः& .& .& .&२७३ &२२२\\
नानास्त्रीणामभिनयः& .& .& .&२७८ &२२२\\
केशाकर्षणम्& .& .& .&२८६ &२२३\\
पुरुषस्य पादगतिः& .& .& .&२९०& २२४\\
शयनं न कार्यम्& .& .& .&२९५& २२५\\
प्रियकान्तादिशब्दाः& .& .& .&३०२ &२२६\\
नाथः& .& .& .&३०७ &२२७\\
स्वामी& .& .& .&३०९ &२२७\\
जीवितम्& .& .& .&३१० &२२७\\
नन्दनः& .& .& .&३११ &२२७\\
मानी, धृष्टः, विकत्थनः& .& .& .&३१२ &२२७\\
दुराचारः, शठ,& .& .& .&३१५ &२२७\\
वामः, निर्लज्जः& .& .& .&३१७ &२२८\\
निष्ठुरः& .& .& .&३१९ &२२८\\
आकाशपुरुषवाक्यम्& .& .& .&३२१ &२२८\\
दिव्याङ्गनाविधिः& .& .& .&३२४ &२२९\\
ईर्ष्यादयः& .& .& .&३२५& २२९\\
नायकस्य उन्मादनम्& .& .& .&३३१& २३०
\end{tabular}

\vspace{8mm}
\rule{0.2\linewidth}{0.5pt}
\end{center}

\newpage
% xv

\begin{center}
\textbf{\large त्रयोविंशोऽध्यायः}

\begin{tabular}{m{12em} m{2em} m{2em} m{2em} c c}
& & & & श्लोकसंख्या & पुटसंख्या\\
वैशिकः& .& .& .&२ &२३२\\
धात्र्यादिस्त्रीभेदः& .& .& .&९ &२३४\\
तृतीकृत्यम्& .& .& .&१४ &२३५\\
स्त्रीपुरुषयोः प्रथमसमागमः& .& .& .&१७ &२३५\\
अनुरक्ता& .& .& .&२२ &२३६\\
विरक्ता& .& .& .&२२ &२३६\\
विरागकारणानि& .& .& .&३० &२३८\\
नारीणां त्रिविधा प्रकृतिः& .& .& .&३४& २३९\\
कामोपचारकुशला& .& .& .&३९ &२३९\\
मध्यमाधमाः& .& .& .&४१ &२४०\\
प्रथमयौवनम्& .& .& .&४३ &२४०\\
द्वितीयम्& .& .& .&४४ &२४०\\
तृतीयम्& .& .& .&४५ &२४०\\
चतुर्थम्& .& .& .&४७ &२४१\\
नवयौवना& .& .& .&४८ &२४१\\
द्वितीययौवना& .& .& .&४९ &२४१\\
तृतीययौवना& .& .& .&५० &२४१\\
चतुर्थयौवना& .& .& .&५१ &२४१\\
पञ्चविधपुरुषाः& .& .& .&५२ &२४१\\
उत्तमः& .& .& .&५३ &२४२\\
मध्यमः& .& .& .&५७ &२४२\\
नारीणां विषये उपेक्षा& .& .& .&६४ &२४४\\
साम& .& .& .&६६ &२४४\\
प्रदानम्& .& .& .&६७ &२४४\\
दण्डः& .& .& .&६८ &२४५\\
मुखरागः& .& .& .&७३ &२४६\\
वेश्योपचारः& .& .& .&७७ &२४७\\
कामस्व निष्पादनम्& .& .& .&७८& २४७\\
& \multicolumn{3}{c}{\rule{0.2\linewidth}{0.5pt}} &
\end{tabular}
\end{center}

\newpage
% xvi

\begin{center}
{\small 
\begin{tabular}{m{12em} m{2em} m{2em} m{2em} c c}
& \multicolumn{3}{c}{\textbf{\large चतुर्विंशोऽध्यायः}} & \\
& & & & श्लोकसंख्या & पुटसंख्या\\
उत्तमा प्रकृतिः& .& .& .&३ &२४९\\
मध्यमा& .& .& .&४ &२४९\\
अधमा& .& .& .&७ &२५०\\
प्रमदोत्तमा& .& .& .&१० &२५०\\
मध्यमा प्रकृतिः& .& .& .&११ &२५०\\
पुरुषाणामधमा& .& .& .&१२ &२५१\\
नपुंसकः& .& .& .&१३ &२५१\\
शकारादयः& .& .& .&१४ &२५१\\
धीरोदात्तादयः& .& .& .&१८ &२५१\\
लिङ्गी, द्विजः, राजजीवी, शिष्यः& .& .& .&१९ &२५२\\
नायकः& .& .& .&२२& २५२\\
नायिका& .& .& .&२४ &२५२\\
कुलस्त्री& .& .& .&२५& २५३\\
आभ्यन्तरी& .& .& .&२७& २५३\\
आयुक्तिकाः& .& .& .&३२ &२५३\\
महादेवी& .& .& .&३५ &२५४\\
देवा& .& .& .&३७ &२५४\\
स्वामिनी& .& .& .&३९ &२५४\\
स्यापिता& .& .& .&४२ &२५५\\
भोगिनी& .& .& .&४२ &२५५\\
शिल्पकारिका& .& .& .&४५ &२५५\\
नाटकीयाः& .& .& .&४७& २५६\\
नर्तकी& .& .& .&५२& २५६\\
अनुचारिका& .& .& .&५३& २५६\\
परिचारिका& .& .& .&५५ &२५७\\
मदत्तराः& .& .& .&५९& २५७\\
कुमार्यः& .& .& .&६१& २५७\\
वृद्धाः& .& .& .&६२& २५७\\
आन्तःपुरिकाः& .& .& .&६७& २५८\\
कारुकाः& .& .& .&६८& २५८\\
कुमारीरक्षका& .& .& .&६९ &२५९\\
वर्षवराः& .& .& .&७१& २५९\\
नृपः& .& .& .&७९& २६०\\
पुरोधमन्निणः& .& .& .&८१& २६१\\
सचिवाः& .& .& .&८४& २६१\\
प्राड्विपाकाः& .& .& .&८५ &२६२\\
& \multicolumn{3}{c}{\rule{0.2\linewidth}{0.5pt}} & 
\end{tabular}}
\end{center}

\newpage
% xvii

\begin{center}
\begin{tabular}{m{10em} m{2em} m{2em} m{2em} c c}
& \multicolumn{3}{c}{\textbf{\large पञ्चविंशोऽध्यायः}} & \\
& & & & श्लोकसंख्या & पुटसंख्या\\
लघुस्वस्थानां वस्तूनामभिनयः& .& .& .&३& २६५\\
चन्द्रज्योत्स्नादीनां निर्देशः& .& .& .&६ &२६५\\
छायाभिलाषस्य निर्देशः& .& .& .&७&२६६\\
सूर्यविकाराणामभिनयः& .& .& .&९& २६६\\
हारस्रगादि निर्देशः& .& .& .&१२& २६७\\
सर्वार्थग्रहणस्य निर्देशः& .& .& .&१३& २६७\\
आत्मस्थादीनां निर्देशः& .& .& .&१४& २६७\\
विद्युदभिनयः& .& .& .&१५& २६७\\
वाटवादीनां निर्देशः& .& .& .&१७ &२६८\\
सिंहादीनां निरूपणम्& .& .& .&१८& २६८\\
गुरूणां पादवन्दनम्& .& .& .&१९ &२६८\\
गणनाभिनयः& .& .& .&२०& २६८\\
छत्रादीनां दण्डधारणम्& .& .& .&२३& २६९\\
सन्दर्शप्रकारः& .& .& .&२४ &२७०\\
दीर्घसत्त्वस्य निर्देशः& .& .& .&२६& २७०\\
शरदभिनयः& .& .& .&२८ &२७०\\
हेमन्तस्य निर्देशः& .& .& .&२९ &२७०\\
शीतस्य निर्देशः& .& .& .&३० &२७१\\
शिशिरस्य निर्देशः& .& .& .&३२ &२७१\\
वसन्तस्य निर्देशः& .& .& .&३३ &२७१\\
ग्रीष्मस्य निर्देशः& .& .& .&३४ &२७१\\
प्रावृट्कालस्य& .& .& .&३५ &२७१\\
वर्षारात्रिः& .& .& .&३६ &२७२\\
भावादीनामभिनयः& .& .& .&४० &२७२\\
विभावः& .& .& .&४२ &२७३\\
अनुभावः& .& .& .&४३ &२७३\\
वैष्णवस्थानम्& .& .& .&४७ &२७४\\
स्त्रीपुरुषविचेष्टितभेदः& .& .& .&४९ &२७५\\
स्त्रीणां ललिताभिनयः& .& .& .&५० &२७५\\
इर्षसन्दर्शनम्& .& .& .&५३ &२७५\\
क्रोधस्याभिनयः& .& .& .&५५ &२७६\\
स्त्रीणां च क्रोधाभिनयः& .& .& .&५७ &२७६\\
स्त्रीषु दुःखप्रदर्शनम्& .& .& .&५९ &२७६\\
पुरुषाणां भयकार्यम्& .& .& .&६१ &२७६\\
स्त्रीणां भयकार्यम्& .& .& .&६२ &२७७
\end{tabular}
\end{center}

\newpage
% xviii

\begin{center}
\begin{tabular}{m{10em} m{2em} m{2em} m{2em} c c}
& & & & श्लोकसंख्या & पुटसंख्या\\
मदकार्यम्& .& .& .&६५ &२७७\\
स्त्रीपुंसोर्व्यतिरेकाभिनयः& .& .& .&६६ &२७७\\
त्रिपताकाङ्गुलीप्रयोजनम्& .& .& .&६७ &२७७\\
शुकादीनामभिनयः& .& .& .&६९ &२७८\\
खरोष्ट्राणां च& .& .& .&६९ &२७८\\
यक्षपिशाचादीनां अङ्गहारनिर्देशः& .& .& .&७१ &२७८\\
अप्रत्यक्षाणां प्रयोगः& .& .& .&७३ &२७८\\
अप्रत्यक्षाभिवादनम्& .& .& .&७४ &२७८\\
पूज्यानां च& .& .& .&७५ &२७८\\
परिमण्डलाभिनयः& .& .& .&७७ &२७९\\
समूहसागरादीनां& .& .& .&७८ &२७९\\
पताकादीनां प्रयोजनम्& .& .& .&८० &२७९\\
वेलाभिनयः& .& .& .&८१ &२७९\\
डोलाभिनयनम्& .& .& .&८३ &२८०\\
दूरस्थाभाषणम्& .& .& .&८६ &२८०\\
आत्मगतम्& .& .& .&८८ &२८०\\
अपवारितकम्& .& .& .&८९& २८१\\
जनान्तिकलक्षणम्& .& .& .&९१ &२८१\\
कर्णप्रदेशवाच्यम्& .& .& .&९२ &२८१\\
जनान्तिकलक्षणम्& .& .& .&९४ &२८१\\
स्वप्नपाठ्यम्& .& .& .&९६& २८२\\
मरणकाले काकुलक्षणम्& .& .& .&९७ &२८२\\
वृद्धानां पाठ्यम्& .& .& .&९९ &२८२\\
बालानां पाठ्यम्& .& .& .&९९ &२८२\\
व्याधिप्लुतस्य पाठ्यम्& .& .& .&१०१ &२८२\\
विषपीतमरणम्& .& .& .&१०२ &२८३\\
विषवेगलक्षणानि& .& .& .&१०४ &२८३\\
प्रथमादिविषवेगलक्षणम्& .& .& .&१०८ &२८४\\
संभ्रमादिषु वाक्यानि& .& .& .&११२ &२८४\\
उत्तमादिषु भावयोजनम्& .& .& .&११४ &२८४\\
लोकप्रशंसा& .& .& .&११६ &२८४\\
अङ्गस्पर्शनम्& .& .& .&११८ &२८५\\
त्रीणि प्रमाणानि& .& .& .&१२० &२८६\\
नाट्यवस्तूनां लोकप्रमाणम्& .& .& .&१२३& २८७\\
& \multicolumn{3}{c}{\rule{0.2\linewidth}{0.5pt}} & 
\end{tabular}
\end{center}

\newpage
% xix

\begin{center}
\begin{tabular}{m{10em} m{2em} m{2em} m{2em} c c}
& \multicolumn{3}{c}{\textbf{\large षड्विंशोऽध्यायः}} & \\
& & & & श्लोकसंख्या & पुटसंख्या\\
त्रिप्रकारा प्रकृतिः& .& .& .&१ &२९१\\
पश्वादीनां प्रवेशः& .& .& .&४ &२९३\\
वचोवेषानुरूपेण प्रयोज्यानि& .& .& .&५ &२९३\\
परभावप्रवेशनम्& .& .& .&८ &२९३\\
सुकुमारप्रयोगः& .& .& .&९ &२९४\\
केवलं पुरुषप्रयोज्यम्& .& .& .&१० &२९४\\
स्त्रीभिः प्रयोगकरणम्& .& .& .&१२ &२९४\\
रूपानुरूपा प्रकृतिः& .& .& .&१५ &२९५\\
स्त्रीणां स्वभावमधुरत्वम्& .& .& .&१७ &२९५\\
अलङ्कारलक्षणम्& .& .& .&१९ &२९५\\
रम्भोर्वशीभिर्नाट्यम्& .& .& .&२० &२९६\\
स्त्रीषु पुरुषाश्रितप्रयोगः& .& .& .&२१ &२९६\\
प्रयोगस्य द्वौ भेदौ& .& .& .&२४ &२९६\\
सुकुमारप्रयोगः& .& .& .&२५& २९७\\
स्त्रीणामकर्तव्याः& .& .& .&२८ &२९७\\
डिमादयः पुरुषप्रयोज्याः& .& .& .&३१ &२९८\\
आचार्यस्य पञ्च संस्काराः& .& .& .&३६ &२९९\\
आचार्यस्य गुणाः& .& .& .&३७& ३००\\
शिष्यस्य गुणाः& .& .& .&३८ &३००\\
& \multicolumn{3}{c}{\rule{0.2\linewidth}{0.5pt}} & \\
& & & & & \\
& \multicolumn{3}{c}{\textbf{\large सप्तविंशोऽध्यायः}} & \\
प्रयोगः सिद्ध्यर्थः& .& .& .&१ &३०१\\
सिद्धिर्द्विविधा& .& .& .&२ &३०१\\
मनुष्यसिद्धिः दशाङ्गा& .& .& .&३ &३०२\\
वाङ्मयी सिद्धिः& .& .& .&४ &३०२\\
शारीरी सिद्धिः& .& .& .&५ &३०२\\
हासस्वरूपम्& .& .& .&७ &३०२\\
साधुवाक्यानि& .& .& .&१० &३०३\\
करुणविस्मयार्थेषु& .& .& .&११ &०३\\
बहुमानवार्क्यम्& .& .& .&१२ &३०३\\
दीप्तवाक्यम्& .& .& .&१३ &३०४\\
दैवी सिद्धिः& .& .& .&१६ &३०४\\
चतुर्विधा घाताः& .& .& .&१९ &३०५\\
आत्मघातः& .& .& .&२५& ३०६
\end{tabular}
\end{center}

\newpage
% xix

\begin{center}
\begin{tabular}{m{10em} m{2em} m{2em} m{2em} c c}
& & & & श्लोकसंख्या & पुटसंख्या\\
भावदूषका घाताः& .& .& .&२८ &३०७\\
स्वरदूषणम्& .& .& .&२९ &३०७\\
विषमं पुष्करगतम्& .& .& .&३१ &३०७\\
त्रिलिङ्गजदोषः& .& .& .&३२ &३०७\\
छन्दोवृत्तभेदः& .& .& .&३२ &३०८\\
यानानामारोहणाविषु अनभिज्ञत्वम्& .& .& .&३६ &३०८\\
यूपचयनादिषु अपरिग्रहः& .& .& .&३८ &३०८\\
सिद्धिमिश्चौ घातौ& .& .& .&३९ &३०९\\
प्रेक्षकाणां रसत्वम्& .& .& .&४२ &३०९\\
नान्दीश्लोकपठनम्& .& .& .&४३ &३१०\\
देशभावहीनत्वम्& .& .& .&४४ &३१०\\
नाटके योजनीयः पदबन्धः& .& .& .&४६ &३१०\\
दोषा नाट्यार्थतो ग्राह्याः& .& .& .&४७ &३११\\
गीतेषु रसभावयोर्लक्षणम्& .& .& .&४८ &३११\\
प्राश्निकलक्षणम्& .& .& .&४९& ३११\\
प्रेक्षकलक्षणम्& .& .& .&५४ &३१२\\
तरुणादीनां परितोषसमयः& .& .& .&५८ &३१२\\
बालादीनां तुष्टिः& .& .& .&६१ &३१३\\
सङ्घर्षः& .& .& .&६४ &३१३\\
अभिनयभेदः& .& .& .&६५ &३१३\\
प्राश्निकाः& .& .& .&६८ &३१४\\
व्यवहारः& .& .& .&७० &३१४\\
पताकादानम्& .& .& .&७१ &३१४\\
सिद्धिस्थानानि& .& .& .&७४ &३१५\\
घातानां लेखनम्& .& .& .&७७& ३१५\\
गीतवाद्यतालानां समत्वादिः& .& .& .&८३ &३१६\\
वारकालः& .& .& .&८७ &३१७\\
प्रदोषे नाट्यम्& .& .& .&९४ &३९८\\
संध्याभोजनकाले नाट्यप्रतिषेधः& .& .& .&९६ &३१९\\
भार्त्रज्ञा& .& .& .&९८ &३१९\\
पात्रस्य प्रयो गुणाः& .& .& .&९९& ३१९\\
समृद्धिः& .& .& .&१०३& ३२०\\
& \multicolumn{3}{c}{\rule{0.2\linewidth}{0.5pt}} & 
\end{tabular}
\end{center}

\newpage
\begin{center}
श्रीरस्तु

\vspace{1mm}
श्री गणपतये नमः

\vspace{4mm}
\textbf{\LARGE नाट्यशास्त्रम्}

\vspace{1mm}
एकोनविंशोऽध्यायः\renewcommand{\thefootnote}{1}\footnote{भ \textendash\ अत्राध्यायविभागो नास्ति ट \textendash\ विंशोऽध्यायः जादिबान्तेषु यसंज्ञके च एकविंशः}

\rule{0.2\linewidth}{0.5pt}
\end{center}

\begin{quote}
{\na इतिवृत्तं \renewcommand{\thefootnote}{2}\footnote{द \textendash\ हि}तु नाट्यस्य\renewcommand{\thefootnote}{3}\footnote{इ \textendash\ काव्यस्य ट \textendash\ कार्यस्य} शरीरं परिकीर्तितम्\renewcommand{\thefootnote}{4}\footnote{द \textendash\ इति कीर्त्यते}~।\\
पञ्चभिः सन्धिभिस्तस्य \renewcommand{\thefootnote}{5}\footnote{ड \textendash\ विभागाः परिकीर्तिताः}विभागः संप्रकल्पितः\renewcommand{\thefootnote}{6}\footnote{न \textendash\ परिकल्पितः प \textendash\ संप्रलक्षितः}~॥~१}
\end{quote}

\hrule

\begin{center}
अभिनवभारती \textendash\ एकोनविंशोऽध्यायः
\end{center}

\begin{quote}
{\qt देहे ससन्ध्यङ्गगणे समस्ते\\
यत्स्थापनं स्पर्शनवृत्तिकारि~।\\
तदिन्द्रियं यस्य वपुर्नमामि\\
तमान्तरस्पर्शमयं महेशम्~॥}
\end{quote}


{\qt पुनरस्य शरीरविधाने} त्यादिना (१८ \textendash\ १२७)\renewcommand{\thefootnote}{*}\footnote{अष्टादशेऽध्यायेऽन्तिमश्लोकस्योत्तरार्धमित्थं संस्करणीयम्\textendash

{\qt पुनरस्य शरीरविधानसान्धिविधिलक्षणं वक्ष्ये} इति~।} शरीरमितिवृत्तात्मकं विधानं च तस्य विधानरूपप्रकारात्मकं, सन्धयश्च मुखादयो विधयश्च सन्ध्यङ्गस्वभावा लक्षणीयत्वेन\renewcommand{\thefootnote}{1}\footnote{लब्धायत्त्वेन} प्रतिज्ञाताः, तत्र शरीरमादौ\renewcommand{\thefootnote}{2}\footnote{शरीरमेव.} लक्षयितव्यमिति दर्शयति \underline{इतिवृत्तं त्विति}~। तुशब्दो व्यतिरेके \textendash\ काव्यमात्रस्यानभिनेयस्य तावद् वृत्तमात्रं शरीरं, नटनीयस्य त्वभिनेयरूपस्य \underline{इति} एवंप्रकारतया यदुपस्कृतं वृत्तं, अतएवेतिवृत्तशब्दवाच्यं तद्वस्तु शरीरं, रसाः पुनरात्मा

\newpage
\fancyhead[CE]{नाट्यशास्त्रम्}
\fancyhead[CO]{एकोनविंशोऽध्यायः}
\fancyhead[LE,RO]{\thepage}
\renewcommand{\thepage}{\devanagarinumeral{page}}
\setcounter{page}{2}

%२ नाट्यशास्त्रम्

\begin{quote}
{\na इतिवृत्तं द्विधा चैव\renewcommand{\thefootnote}{1}\footnote{प \textendash\ ह्येव च द्विधाप्येव} बुधस्तु परिकल्पयेत्~।\\
आधिकारिकमेकं स्यात्\renewcommand{\thefootnote}{2}\footnote{च \textendash\ तु ट \textendash\ आद्यं स्यात् ज आद्यं च ब \textendash\ एवं स्यात्} प्रासङ्गिकमथापरम्~॥~२}
\end{quote}

\hrule

\vspace{2mm}
शरीराविर्भावकाः, अतएवार्थनिर्मापकत्वात् अर्थतादात्म्यात् अर्थरूपताध्यासात् अर्थैकज्ञाननिवेशितत्वात् अर्थोपरञ्जकत्वात् अर्थनिमित्तत्वाद्वा, इतिवृत्तार्थैकयोगक्षेमत्वं वागात्मनां शब्दानामिति~। तदाशयेन \textendash \\

वाचि यत्नस्तु कर्तव्यो नाट्यस्यैषा तनुः स्मृता~। (१४ \textendash\ २)

इति पूर्वमुक्तम्, इह वृत्तं शरीरमिति दर्शितमित्यविरोधः~। स तु कथं प्रकारवैचित्र्य इत्याशंक्याह \underline{पञ्चभिः सन्धिभिरिति}~। एतदुक्तं भवति \textendash\ प्रकारवैचित्र्यकल्पनामया एव सन्धयः~। तत्र पारम्यपरतया\renewcommand{\thefootnote}{3}\footnote{पारम्पर्यता} पञ्चसंख्येति, तेन हीनसन्धित्वेऽपि न कश्चिदत्र विरोधः~।\\

अन्ये तु सर्वत्र पञ्चैव सन्धयः, अपूर्णाङ्गत्वात्तु कस्यचित्सन्धेर्हीनसन्धित्वमुच्यत इत्याहुः~। एतच्च स्वस्थाने वितनिष्यामः~।\\

एवमितिवृत्तशब्दे \underline{इति}भागस्य योऽर्थः सोऽप्रसिद्ध इति कृत्वा द्वितीयार्धेन पञ्चभिरित्यादिना व्याख्यातः, न तु सन्धिनिरूपणमेतदुद्देशक्रमस्तस्यानेकविधत्वात्~।\\

\begin{sloppypar}
एवं शरीरमभिधाय तस्य विधानशब्देनोद्दिष्टं प्रकारवैचित्र्यं दर्शयति \underline{इतिवृत्तं द्विधा चैवेति}~। इतिवृत्तं स्थितं सत्, बुधो विवेचकः कविर्द्विधैव परिकल्पयेत्~। चकारत् प्रकरणादावितिवृत्तं च कल्पयेत्~। तच्च द्विधा~। \underline{एकमपर}मित्यनेनेदमाह \textendash\ न निसर्गतः किञ्चिदाधिकारिकम्, अन्यद्वा~। कविधिया यदेतदाधिकारिकं कृतं तदापरस्य प्रासङ्गिकतास्तीति द्विधाशब्देन सूचितं, तदेवेदं दर्शितम्~। {\qt अधिकरणविचाले च } (पा \textendash\ ५ \textendash\ ३ \textendash\ ४३) इति
\end{sloppypar}

\newpage
% एकोनविंशोऽध्यायः ३

\begin{quote}
{\na यत्कार्यं हि\renewcommand{\thefootnote}{1}\footnote{ट तु} फलप्राप्त्या \renewcommand{\thefootnote}{2}\footnote{षु \textendash\ समर्थं म \textendash\ सामर्थ्यं}सामर्थ्यात्परिकल्प्यते\renewcommand{\thefootnote}{3}\footnote{ठ \textendash\ परिकल्पते ट \textendash\ परिकल्पितम्}~।\\
तदाधिकारिकं ज्ञेयमन्यत्प्रासङ्गिकं विदुः\renewcommand{\thefootnote}{4}\footnote{अयं श्लोकः च संज्ञके न विद्यते}~॥~३

\renewcommand{\thefootnote}{5}\footnote{अयं श्लोकः {\qt कवेः} इति श्लोकानन्तरमेव जादिभान्तेष्वादर्शेषु पठितः.}कारणात्फलयोगस्य वृत्तं स्यादाधिकारिकम्~।}
\end{quote}

\hrule

\vspace{2mm}
\noindent
\underline{धा}प्रत्ययः~। एकं राशिं द्विधा कुर्वति यथा तेनैकमेवेतिवृत्तं द्विशाखमिति यावत्~।\\

तत्प्रकारद्वयं क्रमेण दर्शयति \underline{यत्कार्यं ही}ति~। प्रधानत्वेन सम्पाद्ये फले यो ज्ञानेच्छाप्रयत्नक्रियालक्षण आरम्भः \underline{तत्कार्य}मिति वक्ष्यते {\qt यदाधिकारिकं वस्तु} (१९ \textendash\ २६) इति, तथाभूतो य आरम्भो मुख्यफलप्राप्त्या परिकल्प्यते स आधिकारिकमितिवृत्तम्~। \underline{हि} यस्मात् तथैव ज्ञेयम्~। निरुक्तेनाधिकारः सर्वत्रानुयायित्वं हृदयानुयायित्वं प्रयोजनमस्य~। प्रासङ्गिकेऽपि हि तदन्तर्लीनमेव~। यथा \textendash\ आधिकारिके सहाप्तेनाचिख्यासासाधनैषाफलजिहीर्षानिष्पत्तौ यथा न शक्त्यन्तरव्यापारणं, तद्वत्पासङ्गिकेऽपि सर्वत्र शक्त्यन्तरव्यापाराभाव एव~। शक्त्यन्तरेऽपि पृथग्व्यापार्यमाणे तस्याप्याधिकारिकत्वमेव स्यात्~। प्रतिज्ञानिबर्हणं जगत्कण्टकरावणोद्धरणं शरणागतविभीषणरक्षणमित्याद्यापि हि प्रधानफले सीताप्रत्यानयनलक्षणे विवक्षिते न शक्त्यन्तरव्यापारसाध्यं , अपि तु तदुपयोगिसामाद्युपायचतुष्टयतद्द्विकत्रिकादिभेदसम्पादननान्तरीयकोपनीतमेव~। तापसवत्सराजे राज्यप्रत्यापत्तेः प्रधानफलत्वे वासवदत्तासङ्गमपद्मावतीप्राप्त्यादौ क्रियान्तरानुपयोग एव मन्तव्यः~। यदि ह्यस्य वासवदत्ताप्राप्त्युपायत्वं पद्मावतीपरिणयस्य नोच्येत न वत्सराजस्तत्र प्रवर्तेत, तदप्रवृत्तौ कुतः प्रधानफलमिति सर्वप्रासङ्गिकमेकरूपमेव~। प्रसक्तिर्हि प्रसङ्गः तत आगतं प्रासङ्गिकं, प्रसज्यते वा प्रधानफलनिष्पत्तये इति प्रसङ्गस्तत आगतमिति~। तेन शक्त्यन्तरयोगायोगाभ्यां च यत्प्रासङ्गिकस्यानेकविधत्वं टीकाकृद्भिरभ्यधाथि न तदुपाध्यायाः संमन्यन्ते~। अतएवाह \underline{कारणात्फलयोगस्येति}~। अयमर्थः \textendash\ आधिकारिकं नाम (अधिकारः)

\newpage
% ४ नाट्यशास्त्रम्

\begin{quote}
{\na \renewcommand{\thefootnote}{1}\footnote{ड \textendash\ परोप}तस्योपकरणार्थं तु कीर्त्यते \renewcommand{\thefootnote}{2}\footnote{ढ \textendash\ अप्यानुषङ्गिकम् ट \textendash\ अस्यानु \textendash\ }ह्यानुषङ्गिकम्~॥~४

कवेः प्रयत्नान्नेतॄणां युक्तानां \renewcommand{\thefootnote}{3}\footnote{ट \textendash\ विध्युपाश्रयात्}विध्यपाश्रयात्~।}
\end{quote}

\hrule

\vspace{2mm}
\noindent
यस्त्वितिवृत्तं फलसंबन्धं करोति स कविना वर्णनोपायारोहमानीतः तत्समर्थाचरणेन प्रयुज्यते~। एवमन्यत्स्यादितिवृत्तमिति पूर्वपक्षमाशंक्य तत्रोत्तरमवान्तरेणाह \underline{तस्योपकरणार्थं त्विति~। हि}रप्यर्थे भिन्नक्रमः आनुषङ्गिकमपि कीर्त्यत इति~।\\

ननु फलप्राप्तिलक्षणेन प्रयोजनेन सप्रयोजनत्वमाधिकारिकस्य लक्षणत्वमुक्तम्, फलप्राप्तिश्च प्रासङ्गिकेऽप्यस्ति सा प्रासङ्गिकीति चेत्, सिद्धे प्रासङ्गिकस्याधिकारिकाद् भेदे भवेदेतत् , तत एव तत्सिद्धौ चक्रकान्योन्याश्रयदोषः, तस्मात्फलप्राप्तिरेव विशिष्य वक्तव्येत्यभिप्रायेणाह \underline{कवेः प्रयत्नान्नेतॄणां युक्तानामिति~।} समुत्कर्षं प्राधान्यमवलम्ब्य फलप्राप्तिः कल्प्यते, प्रधानफलप्राप्तिप्रयोजनमाधिकारिकमित्यर्थः~।\\

ननु फलप्राप्तेः कथं प्राधान्यमाधिकारिकं, निर्वर्त्यत्वादिति चेत् स एव दोष इत्याशङ्क्याह \underline{कवेः प्रयत्नादिति}~। कविर्यत्फलमुत्कर्षेण विवक्षति तत्प्रधानफलम्~। ननु पुरुषेच्छा यद्यनियन्त्रिता, तदा पुनरपि स एव प्रयत्न\renewcommand{\thefootnote}{4}\footnote{प्रश्नः} इत्याह~। नेतॄणां युक्तानां विध्यपाश्रयाद् धीरोदात्तादिभेदानां नायकानां मध्ये यो यत्र नायको युक्त उचितः तस्य यो विधिः सम्पाद्यं वस्तु तदपाश्रयप्रयत्नाद्धेतोः कविफलं प्रधानमिति~। यस्मिंश्च विधौ यो नायको युक्तः उचितस्तस्य मयैतत्कर्तव्यमित्यभिसन्धानाभावेऽपि तत्सन्निधौ फलं नायकत्वं विना कर्तव्यम्, यथा तापसवत्सराजे वत्सराजस्य राज्यप्रत्यापत्तिः कर्तव्यतायाममात्याभिसंहितायाम् , अतएव ह्यस्यासौ नेता फलस्य चाक्रष्टा अमात्यसम्पादिताभिसन्धिप्रत्युपायपरम्परार्जितस्यापि~।

\newpage
% एकोनविंशोऽध्यायः ५

\begin{quote}
{\na \renewcommand{\thefootnote}{1}\footnote{ट \textendash\ कल्प्येत य \textendash\ कल्पन्ते हि फलप्राप्तिं}कल्प्यते हि\renewcommand{\thefootnote}{2}\footnote{ड \textendash\ यत्} फलप्राप्तिः \renewcommand{\thefootnote}{3}\footnote{च \textendash\ समुत्कर्षः}समुत्कर्षात्फलस्य\renewcommand{\thefootnote}{4}\footnote{ट \textendash\ फलाय} च\renewcommand{\thefootnote}{5}\footnote{ड \textendash\ तु}~॥~५

[ \renewcommand{\thefootnote}{6}\footnote{अयं ट ड ढ मातृकासूपलभ्यते}लौकिकी\renewcommand{\thefootnote}{7}\footnote{ड \textendash\ लौकिकं} सुखदुःखाख्या यथावस्था रसोद्भवा~।\\
 दशधा मन्मथावस्था व्यवस्थात्रिविधा मता~॥] ६}
\end{quote}

\hrule

\vspace{2mm}
नन्येवमपि रामस्य स्वदारप्रत्यानयनकण्टकोद्धरणभीताभयवितरणादौ सर्वत्न कर्तव्यतौचित्यस्ति , तथापि न व्यवस्थितं लक्षणपित्याह \underline{फलस्यचेति~।} चकारेण समुत्कर्षादित्यस्यावृत्तिर्द्योत्यते~। तेनायमर्थः \textendash\ यदेतत्फलं तावत्यंशे अधिकमुत्कर्षमवलम्बते तत्रैव तस्यौचित्यं कविना कल्पनीयम्~। तथा हि \textendash\ रावणोच्छेदाद्यवधि सीताप्रत्यानयनमेव समुत्कृष्टं भवति, तस्यैव सम्पादनायेतरप्रवृत्तिः, सचिवायत्तसिद्धिस्तु यतो वत्सराजस्ततो यौगन्धरायणाद्यमात्यवर्गस्तावानसाविति तदामात्याद्यभिसंहितराज्यप्राप्तिफलस्यैव तत्रोत्कर्षः~। स ह्वेवं मन्यते \textendash\ राज्यभारचिन्ता एतैर्या कृता सा मयैवेति~। एवमधिकाधिकं हृदयविपरिवर्तमानं समुचितं च नायकस्य फलं यद्यदा कविप्रयत्नेन विवक्ष्यते सम्पाद्यतया तदा तस्य प्रधानफलत्वं, रामाभ्युदयादौ सीताप्रत्यानयनादेरिव , न हि तत्राश्वमेधयागादेर्नायकोचितस्य कविविवक्षितत्वमस्ति~।\\

नन्वेवमपि कविविवक्षैव पुनरपि प्रधानीभूता तत्र चोक्तो नियमहेत्वभाव इति तत्राह \underline{विध्यपाश्रयादिति}~। विधीयत इति विधिः सव्युत्पत्तिः तस्यापाश्रयात्~। एतदुक्तं भवति \textendash\ यादृशि पुरुषार्थे व्युत्पत्तिः कर्तव्या तदुचितनायकग्रहणेन कविः प्रवर्तमानो न स्वेच्छया प्रवृत्तो भवतीति~। हिशब्देन समुच्चयाभिधायिनैतत्सूचितं \textendash\ विध्यपाश्रयाद्युक्ता ये नेतारस्तेषां यत्फलं तस्योत्कर्षाद्यः कवेः प्रयत्नः ततः फलप्राप्तिः समुत्कर्षावलम्बिनी कल्प्यत इति तात्पर्यम्~।

\newpage
% ६ नाट्यशास्त्रम्

\begin{quote}
{\na संसाध्ये फलयागे तु व्यापारः \renewcommand{\thefootnote}{1}\footnote{य \textendash\ कारकस्य ड \textendash\ साधकस्य}कारणस्य यः~।\\
तस्यानुपूर्व्या\renewcommand{\thefootnote}{2}\footnote{ड \textendash\ पूर्व्यात्} विज्ञेयाः पञ्चावस्थाः प्रयोक्तृभिः\renewcommand{\thefootnote}{3}\footnote{एतदनन्तरं टडढ मातृकासु दृश्यतेऽधिकः पाठः \textendash\ नाट्यप्रकरणाभावा द्व्यवस्था(त्र्यवस्था \textendash\ ढ) स्ता मता इह~। धर्मकामार्थसंबन्धः फलयोगस्तु कथ्यते~॥ इति~।}~॥~७

प्रारम्भश्च प्रयत्नश्च तथा प्राप्तेश्च संभवः~।\\
नियता च फलप्राप्तिः फलयोगश्च पञ्चमः~॥~८

औत्सुक्यमात्रबन्धस्तु\renewcommand{\thefootnote}{4}\footnote{य \textendash\ बन्धमात्रस्तु} \renewcommand{\thefootnote}{5}\footnote{ट \textendash\ यो}यद्बीजस्य निबध्यते~।\\
महतः फलयोगस्य \renewcommand{\thefootnote}{6}\footnote{ड \textendash\ स खल्वारम्भ, सोऽत्र प्रारम्भः}स फलारम्भ इष्यते~॥~९}
\end{quote}

\hrule

\vspace{2mm}
अथ कविप्रयत्नेन साध्ये व्यापारपरिस्पन्दो यो वाङ्मनसगतस्तस्य या अवस्था \underline{आनुपूर्व्येति} उद्देशक्रमेणैव \underline{प्रयोक्तृभिः} कविभिर्निबन्धनीयतया ज्ञातव्याः ता उद्दिशति प्रारम्भश्चेति~। \underline{चकारैस्तथाशब्देन} चावश्यंभाविक्रमत्वमासामुच्यते~। न हि प्रेक्षापूर्वकारिणोऽवस्थान्तरासम्भावनायां प्रारम्भ उचितो भवति, तत्प्रारम्भश्चेदुत्तरोत्तरावस्थाप्रसर एव~। \underline{पञ्चम} इत्यनेन क्रमो विवक्षित इति दर्शयति~। एताः क्मेण दर्शयितुमाह \underline{औत्सुक्यमात्रबन्धस्त्विति}~। महतः प्रधानभूतस्य फलस्य युज्यमानस्य तत्तन्नायकोचितस्य यद्बीजमुपायसम्पत् तस्य यदौत्सुक्यमात्रं तद्विषयस्मरणोत्कण्ठानुरूपं, अनेनोपायेनैतत् सिद्ध्यतीति, तस्य बन्धो हृदये निरूढिः प्रारम्भः, सा च नायकस्यामात्यस्य नायिकायाः प्रतिनायकस्य दैवस्य वा~। तस्या हि तथैवानुमानाद् व्यवस्था~। दैवसाध्यमपि च समुद्रदत्ताभिमतप्राप्त्यादिकं\renewcommand{\thefootnote}{*}\footnote{ब्रह्मयशस्स्वामिना कृते पुष्पदूषितके षष्ठेऽङ्के नन्दयन्तीसमुद्रदत्तयोः समागमः केवलं दैवसाधित एव न तु नीतिचक्षुषा पौरुषप्रभावेन~।} पुण्योपार्जनं प्रयत्नबहुमानसिद्धये दैवसाहाय्यस्य पुरुषकारस्य फलवर्तिता

\newpage
% एकोनविंशोऽध्यायः ७

\begin{quote}
{\na अपश्यतः फलप्राप्तिं \renewcommand{\thefootnote}{1}\footnote{च \textendash\ यो व्यापारः न \textendash\ फलाप्राप्तं यो}व्यापारो यः फलं प्रति~।\\
\renewcommand{\thefootnote}{2}\footnote{ड \textendash\ पदं}परं चौत्सुक्यगमनं \renewcommand{\thefootnote}{3}\footnote{प \textendash\ प्रयत्नः परिकीर्तितः न \textendash\ स प्रयत्न इति स्मृतः}स प्रयत्नः प्रकीर्तितः~॥~१०

ईषत्प्राप्तिर्यदा\renewcommand{\thefootnote}{4}\footnote{ड \textendash\ प्राप्तिश्चया} काचित्फलस्य\renewcommand{\thefootnote}{5}\footnote{ड \textendash\ अर्थश्च ढ \textendash\ अर्थस्य} परिकल्पते\renewcommand{\thefootnote}{6}\footnote{ड \textendash\ कल्पते द \textendash\ कीर्त्यते}~।\\
भावमात्रेण \renewcommand{\thefootnote}{7}\footnote{ढ \textendash\ स ज्ञेयो विधिज्ञैः प्राप्तिसंभवः}तं प्राहुर्विधिज्ञाः \renewcommand{\thefootnote}{8}\footnote{ट \textendash\ प्राप्त}प्राप्तिसम्भवम्~॥~११

नियतां तु\renewcommand{\thefootnote}{9}\footnote{ड \textendash\ च} फलप्राप्तिं यदा\renewcommand{\thefootnote}{10}\footnote{ड \textendash\ यत्र} भावेन पश्यति~।\\
नियतां तां फलप्राप्तिं \renewcommand{\thefootnote}{11}\footnote{ड \textendash\ सगुणं तु विनिर्दिशेत् च \textendash\ सगुणाः ट \textendash\ स्वगुणात्}सगुणां परिचक्षते~॥~१२}
\end{quote}

\hrule

\vspace{2mm}
\noindent
तद्व्युत्पत्तिलाभाय प्रदर्श्यत इति~। \underline{एवमपश्यत} इति तदुपायव्यतिरेकेण फलप्राप्तिमपश्यतः फलदर्शनमसंभाव्यमानं विवेचयतः फलमुद्दिश्य यो व्यापारः उपायविषयपरमौत्सुक्यगमनलक्षणं, तेन विनेदं फलं न भवति तस्मात् स एवेपायोऽन्वेष्यः इत्युपायविषयस्मरणेच्छासन्तानस्वभावः, स प्रयत्नः~। \underline{ईषत्प्राप्ति}रित्यादि~। भवत्यस्मादिति \underline{भावः} उपायः , तस्य सहकार्यन्तरयोगः प्रतिबन्धकवारणं च मात्रपदेनावधारितम्~। तदयमर्थः \textendash\ उपायमात्रेण लब्धेन यदा कदाचिद् विशिष्टफलप्राप्तिरीषत् कल्प्यते संभावनामात्रेण स्थाप्यते न तु निश्चीयते तदा प्राप्तेः सम्भवः~। संभावनायोग्यत्वमसंभावनाविशिष्टत्वं नाम तृतीया कर्तुरवस्था~।\\

\underline{नियतां तु फलप्राप्तिं यदेति}~। फलस्य प्रकर्षेणाप्तिर्यतः सहकारिवर्गः प्रतिबन्धकविध्वंसनसहितता च सामग्रीरूपतः , तां सामग्रीं, यदा तेन भावेन पूर्वोपात्ततया मुख्योपायेन नियतां नियन्त्रितां फलाव्यभिचारिणीं पश्यति तदा नियतफलप्राप्तिर्नामावस्था~। ननु कर्तरीत्याशङ्क्याह \underline{सगुणा}मिति \underline{गौ}णी उपचरिता तस्येयमवस्था~। नियतफलकर्तृविषयत्वेन नियतफलप्राप्तिशब्दो विषय \textendash

\newpage
% ८ नाट्यशास्त्रम्

\begin{quote}
{\na \renewcommand{\thefootnote}{1}\footnote{द \textendash\ श्लोकार्धं न वर्तते}अभिप्रेतं समग्रं च प्रतिरूपं\renewcommand{\thefootnote}{2}\footnote{ढ \textendash\ रूप} क्रियाफलम्~।\\
\renewcommand{\thefootnote}{3}\footnote{ड \textendash\ यद्दृश्यते निवृत्ते तु फलयोगः स दृष्यते (ट \textendash\ उच्यते)}इतिवृत्ते भवेद्यस्मिन्\renewcommand{\thefootnote}{4}\footnote{न \textendash\ यत्र} फलयोगः प्रकीर्तितः\renewcommand{\thefootnote}{5}\footnote{द \textendash\ स कीर्तितः}~॥~१३

\renewcommand{\thefootnote}{6}\footnote{य \textendash\ इतिवृत्तादिकाव्यस्य}सर्वस्यैव हि कार्यस्य प्रारब्धस्य\renewcommand{\thefootnote}{7}\footnote{द \textendash\ प्रारम्भस्य} फलार्थिभिः~।}
\end{quote}

\hrule

\vspace{2mm}
\noindent
विषयिणोरभेदोपचाराद् युक्त इति यावत्~। अत एव पश्यतीत्यनेन दर्शनमेवावस्थेति दर्शितम्~। यदि वा सहगुणेन दर्शनेन वर्तते, नियतफलप्राप्तिदर्शनं तन्नामावस्थेत्यर्थः~। ये त्वकारप्रश्लेषादभावेन नियतां सन्देहमयीमिति व्याचक्षते ते नियता फलप्राप्तिः संदिग्धा चेत् कथमेतद्विरुद्धं संगच्छतामिति प्रष्टव्याः~।\\

\underline{अभिप्रेतं समग्रं चे}ति~। यस्मिन्नितिवृत्ते कर्त्रवस्थात्मनि नायकस्याभिप्रेतं तादृशम्, अपि च नानुचितं, अपि तु प्रतिरूपमुचितं संभवात् पूर्णं {\qt क्रियाफल}मिति समनन्तरफलं, न च विधिफलमिव स्वर्गादि कालान्तरापेक्षि वर्ण्यते, सावस्था नायकस्य फलयोगः फलोत्पत्तिर्नाम~। तत्र सचिवामात्यादेरपि यावस्था सा वस्तुतो नायकगामिन्येव भवतीति नाटकेषु नावश्यं सर्वा नायकस्य साक्षादेवोपनिबन्धनीयाः, अपि तु सचिवादिगतत्वेनापि फलयोगस्तु साक्षादेव तद्गत इत्यभिप्रेतमित्यनेन दर्शितम्~। अवस्थान्तराणि सचिवादिगतान्यपि पर्यवस्यन्ति नायकादेरेवेत्येतदेव सुकविना रत्नावल्यां {\qt प्रारम्भेऽस्मिन् स्वामिनः सिद्धिहेतौ} (अ \textendash\ १) इति श्लोकेन प्रतिपदमुक्त्वा अस्मदभिप्रायः समुच्छ्रितेन दर्शितः~।\\

ननु मानुषव्यापारे नायकस्य तत्सचिवादेर्वा भवन्त्येता अवस्थाः, प्रतिनायकेऽप्येवं तत्र परमसदुपायापेक्षया~। यत्र तु दैवायत्तं फलं वर्ण्यते तत्र कथं न च वर्ण्यं पुरुचकारमात्राभिमानिनां दैवमवजानानां चार्वाकादिमतमेयुषां, स दैवबहुमानव्युत्पत्तये हि पुरुषकारोऽप्यफलः, तदभावोऽपि सफलः प्रदर्शनीयः, अत एव दरिद्रचारुदत्तादिरूपकाणि तद्विषयाणि~। तस्माद्दैवायत्तत्वे कथमेतदवस्थापञ्चकम्, तत्परिहर्तुमाह \underline{सर्वस्यैव ही}ति दैवदागच्छतोऽ \textendash

\newpage
% एकोनविंशोऽध्यायः ९

\begin{quote}
{\na \renewcommand{\thefootnote}{1}\footnote{ड \textendash\ यथानुक्रमशो ह्येताः ढ \textendash\ यथानुक्रममेतास्तु च \textendash\ एता अनु~।}एतास्त्वनुक्रमेणैव पञ्चावस्था भवन्ति हि~॥~१४

\renewcommand{\thefootnote}{2}\footnote{यतासां}आसां स्वभावभिन्नानां परस्परसमागमात्~।\\
\renewcommand{\thefootnote}{3}\footnote{ड \textendash\ विन्यासः फलभावेन फलाय परिकल्प्यते}विन्यास एकभावेन फलहेतुः प्रकीर्तितः~॥~१५}
\end{quote}

\hrule

\vspace{2mm}
\noindent
पीत्यर्थः~। तत्रापि हि यद्यपि नायको न यतते तथापि यत्र फलं भवति तत्रावश्यमवस्थादिभिर्भाव्यम्~। स एव च परं फलेन तदानीमर्थीभवति {\qt यमर्थमधिकृत्य प्रवर्तत} इति हि प्रयोजनलक्षणं वदन्ति~। तथा हि सेवाद्यशेषोपायप्रारम्भं विनानन्दसंपादनहृदय एव, अपरथा परतः प्राप्तमपि फलं नाङ्गीकुर्यात् , अनङ्गीकरणेऽपि वास्य फलार्थित्वमेवाधिकफलान्तरसन्तोषमनु प्रसिद्ध्यादिफलान्तराभिसन्धानादिति युक्तमुक्तं मुनिना सर्वस्यैव पञ्चावस्था इति~।\\

नन्वासां तावत् स्वरूपभेदः कालभेदश्च कालाभिन्नानां चैककालत्वाभावात् {\qt संसाध्ये फलयोगे तु व्यापारः कारणस्य } इति (१९ \textendash\ ७) यदुक्तं तत् कथम् , किं च फलयोगे साध्ये च तत्रावस्था कारणस्येति पञ्चेतीहावस्था फलयोग एव , न तु सा कदाचिदन्येत्याशङ्क्याह \underline{आसां स्वभावभिन्नाना}मिति स्वभावभेदे तु कालभेदोऽप्युपलक्ष्यते, स्वभावभेदे दिक्काले दण्डचक्रादिभिरेकफलसंपादना, तेन कालभिन्नानामपि, आसां परस्परमन्योन्यं संगत्या नान्तरीयकत्वेन यदागमनं तदवलम्ब्य यो विन्यासो यत्फलभेदः\renewcommand{\thefootnote}{4}\footnote{हेतुः} तत आद्यन्तावहर्षणं निश्चितोत्तरोत्तरकार्याणां कारणकारणानामपि हेतुत्वानपायादिति भावः~। यच्चोक्तं फलयोगे कथं फलयोगान्तरमिति तत्राप्याह \underline{एकभावेन फलहेतु}रिति~। एकभावः संबन्धः~। तेनायं भावः \textendash\ फलस्योत्पत्त्यवस्था एका नायकेन सह संबद्धा , द्वितीया येयं संसाध्ये फलयोग इत्यत्र निर्दिष्टा, पूर्वा त्ववस्था मध्यत्रयेण युज्यमाना योग्यफलोत्पत्तिदर्शना पञ्चम्यवस्थेत्यर्थः~।

\lfoot{2}

\newpage
\lfoot{}
% १० नाट्यशास्त्रम्

\begin{quote}
{\na \renewcommand{\thefootnote}{1}\footnote{य \textendash\ यद्वृत्तं हि}इतिवृत्तं \renewcommand{\thefootnote}{2}\footnote{ट \textendash\ यथाख्यातं, ड \textendash\ यदाख्यातं}समाख्यातं \renewcommand{\thefootnote}{3}\footnote{ड \textendash\ पुरस्तात्}प्रत्यगेवाधिकारिकम्~।\\
\renewcommand{\thefootnote}{4}\footnote{ट \textendash\ कविना तत्र कर्तव्यं फलान्तं च यथाक्रमम्}तदारम्भादि कर्तव्यं फलान्तं च \renewcommand{\thefootnote}{5}\footnote{प \textendash\ यदा}यथा भवेत्~॥~१६

पूर्णसन्धि च\renewcommand{\thefootnote}{6}\footnote{ड \textendash\ अपि, य \textendash\ तु} कर्तव्यं\renewcommand{\thefootnote}{7}\footnote{ड \textendash\ यत्कार्यं च \textendash\ तत्कार्यं} हीनसन्ध्यपि वा पुनः~।\\
नियमात् \renewcommand{\thefootnote}{8}\footnote{न \textendash\ पञ्च}पूर्ण सन्धि स्याद्धीनसन्ध्यथ\renewcommand{\thefootnote}{9}\footnote{ट \textendash\ सन्धि तु ड \textendash\ सन्धिस्तु न \textendash\ अपि} कारणात्~॥~१७}
\end{quote}

\hrule

\vspace{2mm}
एवमवस्थापञ्चकं प्रदर्श्य तदनुयायित्वेनेतिवृत्तस्याधिकारिकत्वं समर्थयितुमाह \underline{इतिवृत्तं समाख्यात}मिति~। यद्यस्मात्तत् कर्तव्यं कार्यं वस्त्वारम्भादि फलान्तं च तदिति तस्मात्तदवस्थानुयायित्वेनाधिकृतत्वादाधिकारिकमुच्यते~। चस्तुशब्दस्यार्थे, यथा तु तदितिवृत्तशब्दवाच्यं भवेत् तथा प्राक् सम्यगाख्यातमितिशब्दार्थमिति निरूपणेन {\qt पञ्चभिः सन्धिभिस्तस्य विभागः} इत्यनेनैतच्च तद्गतवक्तव्यान्तरोपक्षेपाय पुनरभिहितम्~।\\

ननु किं सर्वत्र पञ्चैव सन्धय इत्याह \underline{पूर्णसंन्धि चे}ति~। विकल्पः सर्वत्रेति कश्चिदाशङ्कते तं प्रत्याह \underline{नियमादि}ति~। उत्सर्गेणेति केचित्~। उपाध्यायास्त्वाहुः \textendash\ सर्वत्रेतिवृत्तं पञ्चसन्ध्येव, न हि कश्चिदपि व्यापारो प्रारम्भाद्यवस्थापञ्चकं विना सिद्ध्येत्, न शक्यमौनीकृत्यं वा~। उक्तं च \textendash\ 

\begin{quote}
{\qt सर्वस्यैव हि कार्यस्य प्रारब्धस्य फलार्थिभिः~।\\
एतास्त्वनुक्रमेणैव पञ्चावस्था भवन्ति हि~॥} इति (१९ \textendash\ १४)
\end{quote}

\noindent
अवस्थापञ्चकानुयायिना सन्धिपञ्चकेनापि भाव्यमेव, तेन सर्वं नियमात्पञ्चसन्धि , \renewcommand{\thefootnote}{1}\footnote{विधुर}हीनसन्धित्वं तु तत्र कारणादपूर्णाङ्गत्वलक्षणादुच्यते, अत एव पूर्णसन्धीति व्यपदिश्यते इत्यपिशब्देन चोक्तं {\qt हीनसन्ध्यपि वा पुनः } इति~।

\newpage
% एकोनविंशोऽध्यायः ११

\begin{quote}
{\na \renewcommand{\thefootnote}{1}\footnote{भ \textendash\ चतुर्थस्यैकलोपे तु ढ \textendash\ चतुर्थः स्यात्}एकलोपे चतुर्थस्य \renewcommand{\thefootnote}{2}\footnote{च \textendash\ लोपं}द्विलोपे त्रिचतुर्थयोः~।\\
\renewcommand{\thefootnote}{3}\footnote{ढ \textendash\ द्वितृतीय}द्वितीयत्रिचतुर्थानां त्रिलोपे लोप इष्यते~॥~१८

प्रासङ्गिके परार्थत्वान्न ह्येष नियमो भवेत्~।\\
यद्वृत्तं \renewcommand{\thefootnote}{4}\footnote{ड \textendash\ तु भवेत् किञ्चित्}सम्भवेत्तत्र\renewcommand{\thefootnote}{5}\footnote{ट \textendash\ यत्र} \renewcommand{\thefootnote}{6}\footnote{ड \textendash\ संयोज्यमविरोधि तत्}तद्योज्यमविरोधतः~॥~१९}
\end{quote}

\hrule

\vspace{2mm}
\noindent
{\qt डिमः समवकारश्च चतुस्सन्धी} इति वक्ष्यते, तत्रावमर्शस्य लोपः~। {\qt व्यायोगेहामृगौ चापि सदा कार्यौ त्रिसन्धिकौ} इत्यत्र गर्भविमर्शयोर्लोपः~। {\qt द्विसन्धि तु प्रहसनं वीथ्यङ्को भाण एव च} तत्र प्रतिमुखगर्भावमर्शानां लोपः, त्रिशब्देन (द्वितीयत्रिचतुर्थानामित्यत्र) तृतीयो लक्ष्यते~। तत्रोपक्रमोपसंहारौ तावत् सर्वत्रावश्यंभाविनौ,~। तत्र तु ये प्रेक्षापूर्वकारिणो विततं बहुफलं कर्तव्यमारभन्ते तेषां पञ्चैव सन्धयः~। आर्तिसहिष्णुत्वेन शंक्यमानविरुद्धप्रत्ययस्यापाकरणे {\qt द्वौ प्रतिषेधौ विधिं द्रढयतः} इति न्यायात् सुदृढो हि भवत्येषां फलयोगः~। डिमादिनायकास्त्वत्युद्धतप्रायत्वान्नातीव विनिपातमाशङ्कन्ते~। व्यायोगादिनायका अपि तारतम्येन फलयोगाङ्गीभावान्नाद्रियन्ते~। प्रहसनादिनायकास्त्वधर्मप्रायत्वात्तदितिवृत्तस्य चर्वितशरीरत्वादुपक्रमोपसंहारमात्रे विश्राम्यन्तीत्यपूर्णा अवमर्शादयः~।\\

एवं पञ्चभिरितीतिवृत्तशब्दे यस्य हीतिशब्दो व्याख्यातः, सोऽनेन निर्वाहितार्थः, प्रासङ्गिके तु क इतिशब्दस्यार्थ इति दर्शयति \underline{प्रासङ्गिक} इति~। नियमो य उक्तो नियमात्पूर्णसन्धि स्यादित्यादि स तत्र न भवेत् विभीषणप्रतिष्ठापनविषये रामस्य चेदौत्सुक्यबन्धादि योज्येत तर्हि तदेव यत्नसंपाद्यं भवेत्~। परामृशति \underline{यद्वृत्त}मिति तत्राधिकारिके यदविरुद्धमत्र प्रासङ्गिके सम्भवि वृत्तं प्रारम्भेष्वन्यतमं च तदेव प्रासङ्गिके योजनार्हमिति~।\\

ननूक्तं \underline{औत्सुक्यमात्रबन्धस्तु यद्बीजस्य} इत्यादि तत्र चोपायतत्सहकारिवर्गप्रतिबन्धत्वं तद्विध्वंसनं चोपक्षिप्तं तत्र तत्स्वरूपं न ज्ञातमित्युपाय \textendash

\newpage
% १२ नाट्यशास्त्रम्

\begin{quote}
{\na \renewcommand{\thefootnote}{1}\footnote{इदमर्धं टपडदनबय मातृकासु न वर्तते.}इतिवृत्ते यथावस्थाः पञ्चारम्भादिकाः स्मृताः~।\\
अर्थप्रकृतयः \renewcommand{\thefootnote}{2}\footnote{ड \textendash\ चासां पञ्च}पञ्च \renewcommand{\thefootnote}{3}\footnote{दप्रबन्धेषु प्रकीर्तिताः ट \textendash\ अन्या बीजाद्या पञ्चनाटके~। द्रष्टव्याः कविभिर्नित्यं लक्षणं च निबोधत~॥}तथा बीजादिका अपि~॥~२०

बीजं बिन्दुः पताका च प्रकरी कार्यमेव च~।\\
\renewcommand{\thefootnote}{4}\footnote{ट \textendash\ इदमर्धं टदयबादिष्वादर्शेषु न इ दृश्यते}अर्थप्रकृतयः पञ्च \renewcommand{\thefootnote}{5}\footnote{ड \textendash\ विनियोज्या}ज्ञात्वा योज्या यथाविधि~॥~२१}
\end{quote}

\hrule

\vspace{2mm}
\noindent
सामग्रीस्वरूपं दर्शयितुमाह \underline{इतिवृत्ते यथावस्था} इति, इतिवृत्तविषये यथा येन प्रकारेणाधिकारिकस्य खण्डन\renewcommand{\thefootnote}{1}\footnote{मन्थन}लक्षणेन पञ्चावस्था उक्ताः तेनैव प्रकारेणार्थप्रकृतयोऽपि पञ्चैव पठ्यन्ते~। तदनभिधाने उपायादिस्वरूपापरिज्ञानात् प्रारम्भाद्यवस्थानां परमार्थतोऽसंवेदने आधिकारिकत्वमविदितं स्यात्~। यत्रार्थः फलं तस्य प्रकृतय उपायाः फलहेतव इत्यर्थः~। तत्र जडचेतनतया द्विधाकरणं, जडश्च मुख्यकारणभूतः, गूढतरो वा, आद्यं बीजं द्वितीयं कार्यं करणीयं प्रयोक्तव्यमित्यर्थे~। चेतनोऽपि द्विधा मुख्य उपकरणभूतश्च, अन्त्योऽपि द्विधा स्वार्थसिद्धिसहिततया परार्थसिद्ध्या युक्तः शुद्धयापि च\renewcommand{\thefootnote}{*}\footnote{शुद्धयेति केवलपरार्थसिद्ध्येत्यर्थः}, तत्राद्यो बिन्दुः दितीयः पताका तृतीयः प्रकरी~। तदेतैः पञ्चभिरुपायैः पूर्णफलं निष्पाद्यते~। अत एवाह \underline{ज्ञात्वा योज्या यथाविधि} इति तासामौद्देशिकोक्तिवदुपनिबन्धक्रमनियम इत्यर्थः~। अन्ये त्वाहुः\textendash\ अर्थस्य समस्तरूपकवाच्यस्य प्रकृतयः प्रकरणान्यवयवार्थखण्डा इत्यर्थप्रकृतयः \textendash\ एतच्च व्याख्यानं नातीव प्रकृतं पोषयति~। सन्ध्यादीनामपि चार्थप्रकृतित्वमत्र व्याख्याने स्यात्, इतिवृत्तमेव च समुदायरूपम्~। अर्थ इतिवृत्ते प्रकृतय इति वक्तव्येऽर्थग्रहणमतिरिक्तं स्यात्, इत्यवस्थाभिश्च तुल्यतावर्णनं वर्णनमात्रं स्यादिति किमनेन~।

\newpage
% एकोनविंशोऽध्यायः १३

\begin{quote}
{\na \renewcommand{\thefootnote}{1}\footnote{ट \textendash\ अल्पमात्रं समुद्दिष्टं प \textendash\ अल्पमात्रमुपक्षिप्तं}स्वल्पमात्रं समुत्सृष्टं बहुधा यद्विसर्पति\renewcommand{\thefootnote}{2}\footnote{ड \textendash\ प्रसर्पति}~।\\
फलावसानं \renewcommand{\thefootnote}{3}\footnote{न \textendash\ यच्चै प \textendash\ तच्च स्यात्तद् बीजमिति}यच्चैव बीजं तत्परिकीर्तितम्\renewcommand{\thefootnote}{4}\footnote{ड \textendash\ अभिधीयते}~॥~२२

प्रयोजनानां विच्छेदे यदविच्छेदकारणम्\renewcommand{\thefootnote}{5}\footnote{न \textendash\ कारकम्}~।}
\end{quote}

\hrule

\vspace{2mm}
तदेतत्पञ्चकमुद्देशक्रमेण लक्षयति \underline{स्वल्पमात्र}मिति~। \underline{यद्वस्तु} \textendash\ सागरिकान्तःपुरनिवासेन (वसन्तोत्सव) समये गम्भीरप्रयोजनसंवेदनाभावात् स्वल्पमात्रमकिंचित्करप्रायं शंक्यते संवादेनोत्सृष्टं प्रक्षिप्तं यथावश्यं फलान्तं, यतो बहुभिः प्रकारैर्विसर्पत्येव , सर्वथा प्रसरति यत्तत् सिद्धिस्तत्फलमपि यदि निरुध्य फलत्वेन प्रवर्तते प्रथमप्रक्षेपेणैव देशकालौचित्यापेक्षैस्तद्बीजवन्न्यस्यारघट्टपरिवर्तनन्यायेन बहुतरोपायपरम्परोपरि\renewcommand{\thefootnote}{1}\footnote{क \textendash\ परसरापरि ख \textendash\ पुरस्सरापारि} कार्यमेव यस्यापेक्ष्यं तद्बीजम्~। यद्यस्मात्परितः समन्तात्कीर्तितं प्रसिद्धम्~। तच्च क्वचिदुपायमात्रं क्वचित्फलमात्रं क्वचिद्द्वयं फलं च क्वचिदुपादानं क्वचिद्धेयव्यसननिवर्तनं क्वचिदुभयमिति~। तत्रापि क्वचिन्नायकोद्देशेन क्वचित्प्रतिनायकाश्रयेणेत्यादिभेदैर्बहुधा भिद्यते~। तत्र चक्रवर्तिपुत्रलाभो मुनिजनाशीर्वचनद्वारेण फलस्वभावस्यैवाभिज्ञानशाकुन्तले~।\\

फलमपि च भविष्यदुपायाविनाभावाद् बीजमित्युच्यते~। एवमन्यत्रापि यथायथमुदाहार्यम्~। आनन्त्याद् ग्रन्थगौरवभयाच्च न प्रतिदिशं लिखितम्~।\\

अथ बिन्दुं लक्षयति \underline{प्रयोजनानां विच्छेद} इति~। प्रयुज्यते फलं यैरुपायानुष्ठानैः तेषामितिवृत्तवशादवश्यकर्तव्यतादिभिर्विच्छेदेऽपि सति यदनुसन्धानात्मकं प्रधाननायकगतं सन्धिद्रव्यज्ञानं बिन्दुः , ज्ञानविचारणं फललाभोपायत्वात्~। यावदविच्छेदः प्रत्यनुसन्धानेन (न) कृतस्तावन्न किञ्चिदपि कार्यं निर्वहति~।


\newpage
% १४ नाट्यशास्त्रम्

\begin{quote}
{\na यावत्समाप्तिर्बन्धस्य\renewcommand{\thefootnote}{1}\footnote{ढ \textendash\ कार्यस्य, न \textendash\ समाप्तिमद्बन्धः} स बिन्दुः \renewcommand{\thefootnote}{2}\footnote{ड \textendash\ इति संज्ञितः}परिकीर्तितः~॥~२३}
\end{quote}

\hrule

\vspace{2mm}
ननु बीजं तावत् फलान्तमास्ते, बिन्दोस्तु कथं स्थितिरित्याह \underline{यावत्समाप्तिरिति}~। यावत्स्वस्य बध्यमानस्य फलस्य सम्यगाप्तिस्तावत्~। एतदुक्तं भवति \textendash\ सकलोपायप्रतिजागरणनिमित्तं ह्यनुसन्धानं यावद्धि मुख्यनायकेन प्रत्यनुसन्धानेन (न) क्रियते तावद् जडाजडरूपः सर्वोऽप्युपायधर्मोऽनुपायकल्प एव~। तथा हि \textendash\ तापसवत्सराजे वासवदत्ताप्रेमानुसन्धानं राजमुखेन प्रत्यङ्कं दर्शितम् \textendash\ {\qt तद्वक्त्रेन्दुविलोकनेन दिवसो नीतः प्रदोषस्तथा}\renewcommand{\thefootnote}{*}\footnote{\begin{quote}
{\qt तद्गोष्ठ्यैव निशापि मन्मथकृतोत्साहैस्तदङ्गार्पणैः~।\\
तां संप्रत्यपि मार्गदत्तनयनां द्रष्टुं प्रवृत्तस्य मे\\
बद्धोत्कण्ठमिदं मनः किमथवा प्रेमा समाप्तोत्सवः~॥} (१ \textendash\ १५)
\end{quote}

यावदिति~। अत्र प्रत्यङ्कदर्शितप्रेमानुसन्धानं वृत्तिकारेण ध्वन्यालोकलोचने तृतीयोद्योत?? उदाहरणवबाहुल्येन विशदीकृतमेव~। अथवास्माभिः प्रकटीकृतस्य तापसवत्सराजस्योपोद्घाते तन्निरूपणं द्रष्टव्यम्~।} इति यावत् षष्ठेऽङ्के\textendash

\begin{quote}
{\qt त्वत्संप्राप्तिविलोभितेन सचिवैः प्राणा मया धारिताः\\
तन्मत्वा त्यजतः शरीरकमिदं नैवास्ति निःस्नेहता~।\\
आसन्नोऽवसरस्तवानुगपने जाता रतिः किं त्वयं\\
खेदो यच्छतधागतं न हृदयं तस्मिन् क्षणे दारुणे~॥}
\end{quote}

\noindent
तत्र प्रधानसिद्धिरायत्तसिद्धिरुभयसिद्धिः, प्रधानसिद्धावयं बिन्दुः, आयत्तसिद्धिस्तु राज्यप्राप्तिलक्षणा~। तस्याममात्यवर्गकृतमेवानुसन्धानं बिन्दुः~। उभयसिद्धौ तूभयकृतः येन यत्प्राधान्येनाभिसंहितं स एव तदनुसंधत्तो~। इत्येवं प्रधानानुसन्धानचेतनव्यापारः कारणानुग्राही स्वयं च परमकारणस्वभावस्तैलबिन्दुवत् सर्वव्यापकात्वादपि बिन्दुः~। बीजं च मुखसन्धेरेव प्रवर्त्यात्मानमुन्मेषयति बिन्दुस्तदनन्तरमिति विशेषोऽनयोः, द्वे अपि तु समस्तेतिवृत्तव्यापके~।

\newpage
% एकोनविंशोऽध्यायः १५

\begin{quote}
{\na यद्वृत्तं \renewcommand{\thefootnote}{1}\footnote{न \textendash\ हि, प यस्यावृत्तं}तु \renewcommand{\thefootnote}{2}\footnote{ट \textendash\ परार्थस्य}परार्थं स्यात् प्रधानस्योपकारकम्~।\\
\renewcommand{\thefootnote}{3}\footnote{द \textendash\ प्रधानवच्चात्मनोऽपि, प \textendash\ तत्संबन्धाच्च फुलवत् फ \textendash\ प्रधानबन्धफलवत्}प्रधानवच्च कल्प्येत सा पताकेति कीर्तिता~॥~२४

फलं \renewcommand{\thefootnote}{4}\footnote{ढ \textendash\ संकल्प्यते, ट \textendash\ स, प \textendash\ प्रकल्पते}प्रकल्प्यते \renewcommand{\thefootnote}{5}\footnote{ड \textendash\ सद्भिः, द \textendash\ यस्य}यस्याः \renewcommand{\thefootnote}{6}\footnote{न \textendash\ परार्थं केवलं बुधैः ड \textendash\ परार्थं यस्य केवलम्, फ \textendash\ परार्थं केवलं बुधाः}परार्थायैव केवलम्~।\\
अनुबन्धविहीनत्वात्\renewcommand{\thefootnote}{7}\footnote{प \textendash\ विहीनां तां फ \textendash\ विहीना च ड \textendash\ बन्धेन हीनस्य ढ \textendash\ विहीनस्य} \renewcommand{\thefootnote}{8}\footnote{य \textendash\ प्रकरीमिति निर्दि ड \textendash\ प्रकरीं तां}प्रकरीति विनिर्दिशेत्~॥~२५

यदाधिकारिकं \renewcommand{\thefootnote}{9}\footnote{प \textendash\ वृत्तं, य \textendash\ कार्यं पूर्वमेवप्रकल्पितम् (द \textendash\ कीर्तितम्)}वस्तु सम्यक् प्राज्ञैः प्रयुज्यते~।\\
\renewcommand{\thefootnote}{10}\footnote{प \textendash\ यदर्थश्च, ट \textendash\ तदर्थे यच्च संसाध्यं फ \textendash\ यदर्थस्तु}तदर्थो यः समारम्भस्तत्कार्यं परिकीर्तितम्\renewcommand{\thefootnote}{11}\footnote{न \textendash\ इति कीर्तितम्, ट \textendash\ समुपाहृतम्, य \textendash\ समुदाहृतम्}~॥~२६}
\end{quote}

\hrule

\vspace{2mm}
\underline{यद्वृत्तं तु परार्थं स्यादिति}~। यस्य संबन्धि वृत्तं संविदनुसन्धानं परस्य प्रयोजनसंपत्तये भवदपि स्वप्रयोजनं संपादयति~। अत एवाह \underline{प्रधानवच्च कल्प्येतेति}~। सचेतनानुसन्धाना पताका सिद्धिप्रधानस्योपकारिणी~। एवं सुग्रीवविभीषणप्रभृतिरपि रामादिनोपक्रियमाणे रामादेरात्मनश्चोपकाराय प्रभवमाने प्रसिद्धिप्राशस्त्ये संपादयतीति~। एवमौचित्यानौचित्यज्ञानोपयोगिन्यानयात्र पताकावदुपयोगित्वादियं पताकोति चिरन्तनाः~।\\

\underline{फलं प्रकल्प्यते यस्या} इति यतश्च ततः परार्थमेव केवलं सर्वमनुतिष्ठति सा प्रकरी~। यथा कृत्यारावणे कुलपतिः , वेणीसंहारे भगवान्वामुदेवः~। प्रकर्षेण स्वार्थानपेक्षया करोतीति~। इक् सार्वधातुभ्यः संविदपेक्षया च, स्त्रीलिङ्गत्वे कृदिकारादिति ङीष्~। फलमिति फलतीति कृत्योपायानुष्ठानमुच्यते~।\\

\underline{यदाधिकारिकमिति} प्राज्ञैः प्रधाननायकपताकानायकप्रकरीनायकैश्चेतनरूपैः, यद्वस्तु फलरूपं प्रयुज्यते संपाद्यते संपाद्यत्वेनानुसन्धीयते तत्फल \textendash\

\newpage
% १६ नाट्यशास्त्रम्

\begin{quote}
{\na एतेषां यस्य येनार्थो यतश्च गुण इष्यते~।\\
\renewcommand{\thefootnote}{1}\footnote{ड \textendash\ प्रधानं तत्प्रकर्तव्यं}तत् प्रधानं तु कर्तव्यं गुणभूतान्यतः\renewcommand{\thefootnote}{2}\footnote{ट \textendash\ भूतमतः} परम्~॥~२७

एको\renewcommand{\thefootnote}{3}\footnote{ड \textendash\ नैको}ऽनेकोऽपि वा सन्धिः पताकायां तु यो भवेत्\renewcommand{\thefootnote}{4}\footnote{ड \textendash\ च योज \textendash\ येत्, ढ \textendash\ पताकायाश्च योजयेत्}~।\\
प्रधानार्थानु\renewcommand{\thefootnote}{5}\footnote{द \textendash\ अनुवादि}यायित्वादनुसन्धिः प्रकीर्त्यते\renewcommand{\thefootnote}{6}\footnote{ढ \textendash\ स कीर्त्यते, ट \textendash\ सन्धिसती (?) भवेत् ड \textendash\ अनुबन्धः स कीर्तितः}~॥~२८}
\end{quote}

\hrule

\vspace{2mm}
\noindent
प्रयोजनो यः संपूर्णतादायी पूर्वपरिगृहीतस्य प्रधानस्य बीजाख्योपायस्य फलम्, आरभत इत्यारम्भशब्दवाच्यो द्रव्याक्रियागुणप्रभृतिः सर्वोऽर्थः (यस्य) सहकारी (तत्) कार्यमित्युच्यते , चेतनैः कार्यते फलमिति व्युत्पत्त्या~। \underline{सम्यगिति} प्रभुमन्त्रोत्साहशक्तित्रयसंपन्नैरित्यर्थः~। तेन जनपदकोशदुर्गादिकव्यापारवैचित्र्यं सामाद्युपायवर्ग इत्येतत्सर्वं कार्येऽन्तर्भवति~। तत्र परं प्रथमपरिगृहीतः प्रधानभूतोऽभ्युपायो बीजत्वेनोक्तः~।\\

ननु प्रारम्भादिवदासामर्थप्रकृतीनां किं सर्वत्र सर्वासां सम्भवस्तथार्थप्रकृतिसन्ध्यवस्थाभिः सह किं यथासंख्यं नियमस्तथा किं स्वात्मन्यासां कर्तृक्रम इति शङ्कात्रयमपाकर्तुमाह \underline{एतेषा}मिति\renewcommand{\thefootnote}{1}\footnote{एतासां} पञ्चकवर्गत्रयं परामृश्यते एकैकस्य वर्गस्यैकशेषेण~। तदयमर्थः \textendash\ न सर्वत्र प्रारम्भादिवत् सर्वा अर्थप्रकृतयोऽपि~। अपि तु यस्य नायकस्य यं येनार्थप्रकृतिविशेषेण प्रयोजनसंपत्तिरधिका तदेव प्रधानम्, अन्यत्तु भवदपि गुणभूतमसत्कल्पम्, यथा स्वपराक्रमबहुमानशालिनां पताकाप्रकर्ये विवक्षिते एव~। बीजबिन्दुकार्याणि तु सर्वत्रानपायीनि~। तत्रापि तु गुणप्रधानभावः तथा सन्ध्यवस्थार्थप्रकृतीनां यस्य \underline{येनो}चितः संबन्धः \underline{प्रधानं} नाटकादिकार्यामिति द्वितीयापि निरस्ता~। यतश्च गुण उपकारो झटिति वाच्यते तदेवार्थप्रकृतिरूपं पञ्चानामन्यतमं प्रधानत्वेन बाहुल्येन निबन्धनीयम्, अन्यद् गुणभावेन प्रधानायत्तसिद्धौ च

\newpage
% एकोनविंशोऽध्यायः १७

\noindent
निबध्यमानाया यो यत्रांशेऽप्यधिकोपकारी स तत्र प्रधानीकर्तव्यः~। यथा वासवदत्तालाभे (तापसवत्सराजे) बिन्दुः प्रधानं, कौशाम्बीराज्यलाभे तु प्रकरी पताका च प्रधाना, अमात्यस्य राज्यसिद्धौ स्वार्थसिद्धिरिति पताकात्वं केचिदाहुः~। एवं सुसङ्गतासांकृत्यायनी\renewcommand{\thefootnote}{*}\footnote{सुसङ्गता रत्नावल्यां, सांकृत्यायनी तापसवत्सराज्ञे~। त्यास्यात्}बीजधर्मादिषु वाच्यम्~। अपरे तु \textendash\ प्रथमतरमेव नायकस्य तावद्रूपत्वान्नैव प्रमाणं पृथङ्नायकत्वं, सुग्रीवादयस्तु प्रथग्भूता एव, सांप्रतिके कार्ये केवलमाश्रिता इति , एवं प्रमाणमिव पताकादिरूपत्वम्~। तेषां मते तापसवत्सराजे उभयत्रापि बिन्दुरेव प्रधानं, पञ्चानामप्यर्थप्रकृतीनामन्यतमस्य द्वयोस्तिसृणां चतसृणां सर्वासां वा प्राधान्यं यथास्वं विभजनीयम्~। पताकाप्रवृत्तस्य प्रधानवदतिदेशात् पञ्चसन्धीत्यासां1 भवेत् पृथग्गणना~। अस्त्वित्यसौ परार्थत्वादेव~। अत एव च कुतः सन्धिः सूच्यते वाभ्यूह्यते वा निबध्यते वा~। यथा मायापुष्पके \textendash

\begin{quote}
{\qt बाली यथा विनिहतः प्रथितप्रभावो\\
दग्धा यथैककपिना प्रसभं च लङ्का~।\\
तीर्णो यथा जलनिधिर्गिरिसेतुना च\\
मन्ये तथा विलसितं चपलस्य धातुः~॥}
\end{quote}

\noindent
पताकायां हि पूर्णवर्णने पताकान्तरं स्यादित्यनवस्था~। तद्युक्तमुक्तमनुसन्धित्वान्न पृथग्गणनमस्येति~।\\

तथा लोल्लटाद्यास्तु मन्यन्ते परार्थे साधयितव्ये पताकानायकस्येतिवृत्तभागा अनुसन्धयः~। यथा कृत्यारावणे \textendash

\begin{quote}
{\qt धन्यास्ते ते कृतिनः श्लाघ्या तेषां च जन्मनो वृत्तिः~।\\
यैरुज्झितात्मकार्यैर्येषामर्थाः प्रसाध्यन्ते~॥}
\end{quote}

\noindent
इति मृखानुसुन्धिः, {\qt वाली यथे} ति प्रतिमुखस्यानुसन्धिः, शक्तिहते लक्ष्मणे ओषध्यानयने गर्भस्य, अङ्गददौत्ये मन्दोदर्याक्षेपेऽवमर्शस्य, {\qt आ ! अशक्त, अनार्ये तिष्ठ तिष्ठ, अतिरथस्त्वं सा न} इति निर्वहणस्य, इति~। एतत्तु भवति सर्वस्यैव हि पञ्चावस्था भवन्तीत्युक्तम्~। किं तस्यानुसन्धिद्वित्वाभिधाने प्रयोजनं न पश्यामः~।

\lfoot{3}

\newpage
\lfoot{}
% १८ नाट्यशास्त्रम्

\begin{quote}
{\na आगर्भादाविमर्शाद्वा\renewcommand{\thefootnote}{1}\footnote{ट \textendash\ च} पताका विनिवर्तते~।\\
\renewcommand{\thefootnote}{2}\footnote{ड \textendash\ तस्मात् द \textendash\ कस्मान्निबन्धो ह्येतस्याः}कस्माद्यस्मा\renewcommand{\thefootnote}{3}\footnote{ड \textendash\ तु बन्धो}न्निबन्धोऽस्या: \renewcommand{\thefootnote}{4}\footnote{ड \textendash\ परार्थायोपकल्प्यते ट \textendash\ परार्थाय तु कल्प्यते}परार्थः परिकीर्त्यते~॥

\renewcommand{\thefootnote}{5}\footnote{न \textendash\ यत्रान्यस्मिन् चिन्त्यमाने कल्पितोऽन्यः ढ \textendash\ यत्रार्थश्चिन्त्यमानोऽपि तल्लिङ्गोऽर्थः}यत्रार्थे चिन्तितेऽन्यस्मिन्न्स्तल्लिङ्गोऽन्यः प्रयुज्यते~।}
\end{quote}

\hrule

\vspace{2mm}
स्वफलसिद्धये यतमानस्य तत्र तत्रावश्यं पृथग्गणनाशङ्केति, तत्प्रशमनप्रयोजनम्, अस्मत्पक्षे कस्मिंस्तर्हि प्रधानसन्धौ तस्यानुयायित्वमिति दर्शयितुमाह \underline{आगर्भादाविमर्शाद्वेति} प्रतिमुखे गर्भे यदि वा~। यमर्थं व्याप्य निवर्तते पताकेतिवृत्तं तावत्येव पताकानायकस्य स्वफलसिद्धिरुपनिबन्धनीया, सिद्धफलस्त्वसौ प्रधानफल एव व्याप्रियमाण आसीनोऽपि भूतपूर्वगत्या पताकाशब्दवाच्यो न मुख्यत्वेन~। अत्राह \underline{कस्माद्यस्मादिति}~। कस्मादस्याभिप्रायः, प्रधानवच्च कल्प्येतेत्युक्तत्वात् निर्वाहादापि किं तद्भवति, अत्रोत्तरं \underline{यस्मा}दिति निर्वहणपर्यन्ते तत्फले क्रियमाणे तुल्यकालयोरुपकार्योपकारकत्वाभावात् तेन प्रधानोपकारो (कारभावो ?) न भवेत्~। अभिविधावाङ्~। ये तु मर्यादायां तं व्याचक्षते ते न सम्यगमंसत \textendash\ विनिपातप्रतीकारः प्रधानविमर्शसन्धौ प्रस्तुतोपयोगः पताकायाः, यत्र कृतघ्नतादृष्टा तन्न नीत्योच्यते कृतज्ञस्तु प्राप्तफलो विनिपातान् प्रतिकुर्यादेवेति~। तत्र पताकानायको यथास्वार्थे प्रवर्तते \underline{परार्थं} च (संपद्यते), भृत्यस्यान्यस्य वा जडस्य वा स्थितिः\textendash\ स्वार्थेऽपि सति परार्थे संपद्यते~। तत्पताकास्थानकं पताका तत्र जडस्य स्वार्थपरार्थप्रवृत्तौ तावदभिसन्धानाभावात्~। अभिसन्धानवतः सुग्रीवादेः पताकानायकाद्भेदः, अजडस्यापि स्वार्थे यद्यप्यनुसन्धानमस्ति तत्रापि परार्थे नास्तीति विशेषः, प्रधानवच्च कल्प्येतेति पताकालक्षणेऽभिधानाद् बहुतरेतिवृत्ते व्यापकता नायकस्य, अस्य तु परिमितेतिवृत्तव्यापकत्वमित्यपि विशेषः~।\\

\begin{sloppypar}
तस्य पताकास्थानकस्य वक्ष्यमाणभेदैर्भेदवत्त्वात् सामान्यलक्षणं तावदाह \underline{यत्रार्थे चिन्तितेऽन्यस्मि}न्निति~। अर्थः प्रयोजनं, उपायश्च, कर्मकरण \textendash
\end{sloppypar}

\newpage
% एकोनविंशोऽध्यायः १९

\begin{quote}
{\na आगन्तुकेन भावेन पताकास्थानकं तु तत्~॥~३०

सहसैवार्थसम्पत्ति\renewcommand{\thefootnote}{1}\footnote{न \textendash\ संपत्तौ}र्गुणवत्युपकारतः\renewcommand{\thefootnote}{2}\footnote{न \textendash\ कारकः , ड \textendash\ चारतः}~।\\
पताकास्थानकमिदं प्रथमं परिकीर्तितम्\renewcommand{\thefootnote}{3}\footnote{न \textendash\ स्थानमित्याद्यमलङ्कारार्थसंयुतम्, द \textendash\ समुदाहृतम् , ढ \textendash\ परिकीर्त्यते}~॥~३१}
\end{quote}

\hrule

\vspace{2mm}
\noindent
व्युत्पत्त्या, अन्यस्मिन्नुपाये प्रयोजने वा चिन्तिते अन्यः उपायान्तरप्रयोजनान्तरलक्षणः प्रकर्षेण युज्यते संबध्यते यत्रेति तत्पताकास्थानकम् , पताकाधारत्वादुपचारादितिवृत्तमपि पताकास्थानकम्~। उपाध्यायास्त्वाहुः \textendash\ पताकायः स्थानमितिवृत्तता, तत्र चार्थः क्रियमाणोऽपि पूर्वपदार्थमुपसंक्रामति, राजवाहन इवायमुद्धुरकन्धरस्तिष्ठतीति यथा, तेन पताकास्थानकमितिवृत्त\textendash\ मेवोच्यते~। तत्र वर्ण्यमानं तु जडाजडरूपं पताकासदृशमित्यर्थादुक्तं भवति~। स चान्योर्थस्तल्लिङ्गस्तन्मुख्यमर्थं लिङ्गयति विचित्रयतीति~।\\

ननु किं तत्र पताकासादृश्यमित्याह \underline{आगन्तुकेन भावेनेति}~। भावनं भावः कारणत्वम्, तच्च द्विविधं स्वरूपकृतं सहकारिकृतं च~। सहकारिकृतमागन्तुकमुच्यते~। तेन सहकारित्वसामान्यात् तत्समर्थाचरणलक्षणात् पताकासादृश्यमिति यावत्~। अन्याभिसन्धानेऽन्यसिद्धिश्चेत् भूषणभूतापि कैश्चिद्दूषणत्वेन गृहीता, तैरर्थशब्दः उपायवाच्योपाश्रितः~। तल्लिङ्ग इति कारणत्वधर्माभावप्रवृत्तिनिमित्त उपायः~। उदाहरणं सामान्यलक्षणस्य विशेषलक्षणव्याख्याने शक्ययोजनमिति तत्रैव दर्शयिष्यामः~। तद्भेदान् क्रमेण लक्षयति \underline{सहसैवार्थसंपत्तिरिति}~। यत्रोपकारकमपेक्ष्य गुणवती उत्कृष्टा अर्थस्य फलस्य सहसैवाचिन्तितोपनतत्वेन भवति संपत्तिः तत् प्रथममिति साध्यफलयोगात्प्रधानं पताकास्थानम्~। यथा रत्नावल्यां सागरिकायां पाशावलम्बनप्रवृत्तायां वासवदत्तेयमिति मन्यमानो यदा राजा पाशं मुञ्चति तदा तदुक्त्या सागरिकां प्रत्यभिज्ञाय {\qt हा कथं प्रिया मे सागरिका, अलमलमतिमात्र} मित्यादि~। अत्रान्यत्प्रयोजनं चिन्तितं तद्वैचित्र्यकारि च प्रयोजनान्तरं संपन्नम्~। तत्र च दैवयोगः तथाभूतदेशकालयोगो नायकः स्वात्मैवान्याभिसन्धिबलात् कल्पितभेदः, सागरिकैव वा मरणमेवोचितमित्यन्याभि \textendash

\newpage
% २० नाट्यशास्त्रम्

\begin{quote}
{\na \renewcommand{\thefootnote}{1}\footnote{ड \textendash\ वचसातिशय}वचः सातिशयं श्लिष्टं काव्यबन्धसमाश्रयम्\renewcommand{\thefootnote}{2}\footnote{ट \textendash\ रसाश्रयम्}~।\\
\renewcommand{\thefootnote}{3}\footnote{ट \textendash\ पताकं}पताकास्थानकमिदं द्वितीयं परिकीर्तितम्\renewcommand{\thefootnote}{4}\footnote{य \textendash\ कल्पितम्}~॥~३२

अर्थोपक्षेपणं यत्र\renewcommand{\thefootnote}{5}\footnote{द \textendash\ क्षेपकं यत्तु, ड \textendash\ यत्तु, भ \textendash\ यत्तत्} लीनं \renewcommand{\thefootnote}{6}\footnote{ढ \textendash\ संविनयं}सविनयं भवेत्~।\\
\renewcommand{\thefootnote}{7}\footnote{भ \textendash\ श्लिष्टं}श्लिष्टप्रत्युत्तरोपेतं तृतीयमिदमिष्यते~॥~३३}
\end{quote}

\hrule

\vspace{2mm}
\noindent
सन्धानेन वदतीति पताकानायकसदृशत्वं भजते~। अन्यस्मिन्नुपाये चिन्तिते सहसोपायान्तरप्राप्तिः यथा नागानन्दे जीमूतवाहनस्य शङ्खचूडाप्राप्तवध्यपटस्य कञ्चुकिना वासोयुगलार्पणम्~।\\

अन्ये तु {\qt चतुष्पताकापरम} मितिभाविसन्धिचतुष्टयाभिप्रायेण मन्यमानाःप्रथमद्वितीयादिशब्दान् मुखादिसन्धिविषयप्रयोगाभिप्रायेण व्याचक्षते~। अत्र च युक्तिर्न लक्ष्यते, न वा चमत्कारं भजतीत्यसदेव~। एतत्तुल्यतया गणनभूतेन्द्रियैः सह पताकापञ्चकगणनमप्यापादयेदिति च वदद्भिश्चिरन्तनैरेवायमुपहासपात्रीकृतः पक्ष इत्यास्ताम्~।\\

\underline{वचः सातिशथं श्रिष्ट}मिति~। काव्यस्य प्रकृतस्य वर्णनीयस्य यो बन्धः, अतिशयोक्त्यादिना योजनं तन्निमित्तवशाद्यद्वचनं सातिशयश्लिष्टमप्रकृतं प्रत्युचितं जातं तद्वचनं तदर्थो वा तदुच्चारयिता वा यादृच्छिकं वा प्रकृतोपयोगित्वेन सहकारित्वेन गच्छद् द्वितीयं पताकास्थानमभिसन्धानापेक्षया~। यथा रामाभ्युदये तृतीयेऽङ्के सीतां प्रति सुग्रीवस्य संदेशोक्तिः \textendash

\begin{quote}
{\qt बहुनात्र किमुक्तेन पारेऽपि जलधेः स्थिताम्~।\\
अचिरादेव देवि त्वामाहरिष्यति राघवः~॥}
\end{quote}

\noindent
अत्रान्यप्रयोजनेनातिशक्त्याशयेन प्रयुक्तेऽपि वचसि पारेऽपीत्यादि प्रकृतोपयोगातिशयात्पताकास्थानकम्~।\\

\underline{अर्थापक्षेपणं यत्रे}ति~। लीनमस्फुटरूपं उत्क्षिप्यमाणमर्थजातं, श्लिष्टेन संबन्धयोग्येनाभिप्रायान्तरप्रयुक्तेनापि प्रत्युत्तुरेणोषेपेतं सद्यत्र, सविनयं विशेषेण नयनेन विशेषनिश्चयप्राप्त्या सहितं संपद्यते तत् तृतीयं पताकास्था \textendash

\newpage
% एकोनविंशोऽध्यायः २१

\begin{quote}
{\qt \renewcommand{\thefootnote}{1}\footnote{भ, ढ \textendash\ व्यर्थो}द्व्यर्थो वचनविन्यासः \renewcommand{\thefootnote}{2}\footnote{ढ \textendash\ सं, भ \textendash\ यत्र स्यात्}सुश्लिष्टः \renewcommand{\thefootnote}{3}\footnote{प \textendash\ कार्य}काव्ययोजितः~।\\
\renewcommand{\thefootnote}{4}\footnote{ढ \textendash\ उपमित्या संप्रयुक्तः, न \textendash\ उपपत्त्या युतं यच्च, द \textendash\ उपपत्त्या च \textendash\ उपन्याससंयुतश्च}उपन्याससुयुक्तश्च \renewcommand{\thefootnote}{5}\footnote{ढ \textendash\ चतुर्थमिति कीर्तितम् द \textendash\ चतुर्थं परिकीर्तितम् फ \textendash\ द्वार्थं न्यासं\ldots ष्टं \ldots तम्, युक्तं च \ldots तत्.}तच्चतुर्थमुदाहृतम्~॥~३४}
\end{quote}

\hrule

\vspace{2mm}
\noindent
नकम्~। यथा मुद्राराक्षसे चाणक्यः \textendash\ {\qt अपि नाम राक्षसो दुरात्मा गृह्येत}~। एवमस्फुटेऽर्थे उपक्षिप्ते, (प्रविश्य) सिद्धार्थकः \textendash\ {\qt अंअ गण्हिदो\renewcommand{\thefootnote}{*}\footnote{आर्य गृहीतः}}~। इत्येतत्प्रत्युत्तरं सन्देशाशयेन प्रयुक्तमौचित्याद्विशेषनिश्चयं करोति~। तथा च पुनश्चाणक्यः \textendash\ (सहर्षमात्मगतं) {\qt हन्त गृहीतो दुरात्मा राक्षसः} इति~। इदं च प्रकृतसाध्योपयोगाङ्गित्वात् पताकास्थानीयमिति वीथ्यङ्गाद् गण्डादस्य भेदः \textendash\ {\qt ऊरुयुग्मं च भग्नं} तद्धि प्रत्युत दुर्योधननाशादाशयश्च दुष्टः~। कस्तस्योपयोगः, पाण्डवानुसारेण तु भवतु~। इदं पताकास्थानकं भिन्न\textendash\ विषयत्वं कृतं ह्येतद्रूपं न क्षतिमावहति~।\\

\underline{द्व्यर्थो वचनविन्यास} इति यो वचनविन्यासः कथारूपं वा सालङ्खारत्वसंपत्त्याशयेन, शोभनः प्रसादयुक्तः, श्लेषवशात्, \underline{द्व्यर्थ} इति अनेकार्थसंप्रयुक्तः तादृशः सन्नुपन्यासे वस्त्वन्तरोपक्षेपे सुष्ठु संपद्यते , तच्चतुर्थम्~। यथा\textendash\ {\qt प्रीत्युत्कर्षकृतो दृशामुदयनस्येन्दोरिवोद्वीक्षते}\renewcommand{\thefootnote}{1}\footnote{\begin{quote}
{\qt अस्तापास्तसमसुतभासि नभसः पारं प्रयाते रवा\textendash\ 
वास्थानीं समये समं नृपजनः सायन्तने संपतन्~।

संप्रत्येष सरोरुहद्युतिमुषः पादांस्तवासेवितुं \ldots}
\end{quote}} (रत्नावली अ १ \textendash\ २३) इत्यत्र हि काव्यरूपताशयेन श्लेषः प्रयुक्तः प्रधानवस्त्वन्तरं सागरिकागतमुत्क्षिपति \textendash\ {\qt अयं सो राआ उदयणो जस्स अहं तादेण दिण्णा \textendash\ अयं स राजा उदयनो यस्याहं तातेन दत्ता} (रत्ना अ १) इति~। {\qt उद्दामोत्कलिकां}\renewcommand{\thefootnote}{2}\footnote{\begin{quote}
{\qt उद्दामोत्कलिकां विपाण्डुररुचं प्रारब्धजृम्भां क्षणा\textendash \\
दायासं श्वसनोद्गमैरविरतैरातन्वतीमात्मनः~।\\
अद्योद्यानलतामिमां समदनां नारीमिवान्यां ध्रुवं\\
पश्यन् कोपविपाटलद्युतिमुखं देव्याः करिष्याम्यहम्~॥}
\end{quote}}

\newpage
% २२ नाट्यशास्त्रम्

\begin{quote}
{\na \renewcommand{\thefootnote}{1}\footnote{ढ \textendash\ आदर्श एव}[ यत्र सातिशयं वाक्यमर्थोपक्षेपणं भवेत्~।\\
विनाशिदृष्टमन्ते च पताकार्धं त तद्भवेत्~॥ ]~३५

चतुष्पताकापरमं\renewcommand{\thefootnote}{2}\footnote{भ \textendash\ पताकमेवं हि} नाटके \renewcommand{\thefootnote}{3}\footnote{न \textendash\ काव्य}कार्यमिष्यते~।\\
पञ्चभिः सन्धिभिर्युक्तं \renewcommand{\thefootnote}{4}\footnote{द \textendash\ तान्प्रव ढ \textendash\ तत्प्रव}तांश्च वक्ष्याम्यतः परम्~॥~३६}
\end{quote}

\hrule

\vspace{2mm}
\noindent
इति तु नोदाहरणं, द्व्यर्थताप्रतिपत्तावपि हि नात्रार्थेन सहकारिता कुत्रचिदाचरिता~। तस्मादेतद्वीथ्यङ्गस्य व्याहारस्यैवोदाहरणं युक्तम्\renewcommand{\thefootnote}{1}\footnote{{\qt अन्ये तु कार्ययोजितः} इति पाठं स्वीकृत्य उद्दामेत्याद्युदाहृत्यायं वचनविन्यास उपक्षेपणार्थेन समालिङ्गितः प्रसङ्गादिति व्याचक्षते~।}~।\\

एषामुत्कर्षं दर्शयितुमाह \underline{चतुष्पताकापरम}मिति~। चतुष्पताकाशब्देन समनन्तरं पताकास्थानकमुक्तं तैश्चतुर्भिः कृतैः परममुत्कृष्टं \underline{नाटके} नाट्यविषये \underline{कार्य}मिष्यते तस्मात्तथाकर्तव्यमित्यर्थः~।\\

पताकानायकेन हि यल्लेशतः कर्तव्यं तदेकेन क्रियते चदुष्टयेन वा~। केचिदित्याहुः \textendash\ चतुर्षु सन्धिषु चत्वारः पताकानायकाः, तेषां यथाक्रमं सूचकानि पताकास्थानकानि, प्रथमं मुखसन्धौ यावच्चतुर्थमवमर्शसन्धावितितच्चासत् , पताकाया इव प्रकरीकार्यबिन्दुबीजानामपि सूचकान्तराणि वक्तव्यानि स्युः चत्वारश्च नियमेन पताकानायका भवेयुः , आगर्भादिति च पक्षे चतुष्पताकापरममित्यसङ्गतं स्यात्, मुख्यनायके चेतिवृत्तसूचकं न लक्षणतः कथ्यते पताकानायके तु कथ्यत इत्यधरोत्तरमाश्रितं स्यात् , न च मुख्यसंधावाद्यं द्वितीयं प्रतिमुखसन्धावित्यादिक्रमो न्याये लक्ष्ये वा साक्ष्यमाक्षिपति \textendash\ इत्यलमनेन~।\\

एवमितिवृत्तं व्याख्यातं, तस्य च भेदद्वयं निरूपितं, प्रसङ्गादाधिकारिकत्वसिद्धये अनुवृत्तिस्थानभूता अवस्थाः पञ्च दर्शिता अर्थप्रकृतयश्च, तत्प्रसङ्गादेव पताकास्थानानि~। अधुना त्वितिशब्दार्थं प्रयुक्तं \underline{पश्चभिः सन्धिभिर्युक्तमिति}~। तत्र प्रतिज्ञां करोति \underline{तांश्च वक्ष्यामीति}~। अथ

\newpage
% एकोनविंशोऽध्यायः २३

\begin{quote}
{\na मुखं प्रतिमुखं चैव गर्भो विमर्श एव च\renewcommand{\thefootnote}{1}\footnote{प \textendash\ गर्भो विमर्शश्च तथैव हि भ \textendash\ चैव गर्भोऽवमर्श एव च}~।\\
तथा निर्वहणं चेति नाटके पञ्च सन्धयः\renewcommand{\thefootnote}{2}\footnote{ड \textendash\ सन्धयो नाटके स्मृताः , ढ \textendash\ नाटके सन्धयः स्मृताः , प \textendash\ सन्धयः पञ्च नाटके}~॥~३७

\renewcommand{\thefootnote}{3}\footnote{अयं श्लोको डनयमभढ \textendash\ मातृकासु वर्तते}[ पञ्चभिः सन्धिभिर्युक्तं प्रधानमनु कीर्त्यते~।\\
शेषाः प्रधानसन्धीनामनु\renewcommand{\thefootnote}{4}\footnote{ढ \textendash\ ग्राह्यानु, फ \textendash\ ग्राह्यस्तु}ग्राह्यनुसन्धयः~॥ ]~३८

यत्र बीज\renewcommand{\thefootnote}{5}\footnote{द \textendash\ समाप्तिस्तु}समुत्पत्तिर्नानार्थरससम्भवा~।\\
\renewcommand{\thefootnote}{6}\footnote{न \textendash\ श्लिष्टा काव्यशरीरेण, द \textendash\ शरीरकाव्यानुगमात् , भ \textendash\ काव्यशीर्षानुगता मुखं तत्समुदाहृतम् , प \textendash\ कथाशरीरानुगता}काव्ये शरीरानुगता तन्मुखं परिकीर्तितम्~॥~३९}
\end{quote}

\hrule

\vspace{2mm}
\noindent
निर्णिनीषुरुद्देशं तावदाह \underline{मुखं प्रतिमुखं चैवेति}~। समुच्चयपदैः पञ्चानां सर्वत्रावश्यंभावित्वं द्योतितम्~। नियमवाचिभिः क्रमनियमः~। नाटक इत्यभिनेयरूपके इत्यर्थः~। महावाक्यार्थरूपस्य रूपकार्थस्य पञ्चांशा अवस्थाभेदेनकल्प्यन्ते~। तत्र मुखस्य स्वतन्त्रस्येतिवृत्ते समस्तप्रयोजनस्यात एव नायकस्य स्वमुखेन परद्वारेण वा या प्रारम्भावस्था प्रथमा व्याख्याता तदुपयोगी यावानर्थराशिः स मुखसन्धिः~। तस्यार्थराशेरवान्तरभागान्युपक्षेपाद्यानि सन्ध्यङ्गानि~। एवमन्येषु सन्धिषु वाच्यम्~। तेनार्थावयवा सन्धीयमानाः परस्परमङ्गैश्च सन्धय इति समाख्या निरुक्ता~। तदेषां सामान्यलक्षणम्~। तत्रैषां क्रमेण विशेषलक्षणमाह \underline{यत्र तीजसमुत्पत्तिरिति}~। प्रागारम्भभावित्वान्मुखमिव मुखं, यावत् क्रियावत्यर्थभागराशौ बीजस्य मुखोपायस्य सम्यगुत्पत्तिः शरीरेण प्रारम्भात्मना अनुगता भवति , नानाभूतोऽर्थवशात् प्रसङ्गायातो रससंभवो यः स्यात्~। एतदुक्तं \textendash\ प्रारम्भोपयोगी यावानर्थराशिः प्रसक्तानुप्रसक्त्या विचित्रास्वाद आपतितः तावान् मुखसन्धिः, तदभिधायी

\blfootnote{*विमर्शावमर्शशब्दयोरन्यतरस्वीकारेण बहूनां पाठानामश्राव्यता दृश्यते}

\newpage
% २४ नाट्यशास्त्रम्

\begin{quote}
{\na बीजस्योद्घाटनं यत्र\renewcommand{\thefootnote}{1}\footnote{ड \textendash\ उद्घटनं यत्तु} दृष्टनष्टमिव क्वचित्~।}
\end{quote}

\hrule

\vspace{2mm}
\noindent
च रूपकैकदेशः~। यथा रत्नावल्यां प्रथमोऽङ्कः , तथा हि , अमात्यस्य वीरो, वत्सराजस्य शृङ्गाराद्भुतौ ततः शृङ्गार इति इयानयं सागरिकाया राजदर्शनेऽ मात्यप्रारम्भविषयीकृतेऽर्थराशिरुपयोगीति मुखसन्धिः~। एवं प्रतिसन्धि वक्तव्यम्~।\\

\underline{बीजस्योद्घाटनं यत्रेति}~। कार्यतया दृष्टं कारणतया नष्टमिति केचित्, उपादेये दृष्टं हेये नष्टमित्यन्ये, नायकवृत्ते दृष्टं प्रतिनायकेतिवृत्ते नष्टमित्यपरे~। न चैतत्समञ्जसम्, एकविषयमन्तरेण सन्धानायोगात्, नाशस्यापि च हेयादिविषयस्य प्रारम्भवशेन दृष्टतयैव संग्रहसंपत्तेः~। तस्मादयमत्रार्थः\textendash\ बीजस्योद्घाटनं तावत् फलानुगुणो दशाविशेषः तद् दृष्टमपि विरोधिसंनिधेर्नष्टमिव, पांसुना पिहितस्येव बीजस्याङ्कुररूपमुद्घाटनम्~। यथा वेणीसंहारे कञ्चुकिवचनम् \textendash

\begin{quote}
{\qt आशकस्त्रग्रहणादकुण्ठपरशोस्तस्यापि जेता मुने\textendash \\
स्तापायास्य न पाण्डुसूनुभिरयं भीष्मः शरैः शायितः~।\\
प्रौढानेकधनुर्धरारिविजयश्रान्तस्य चैकाकिनो\\
बालस्यायमरातिलूनधनुषः प्रीतोऽभिमन्योर्वधात्~॥} (अ २ \textendash\ २)
\end{quote}

\noindent
अत्र पाण्डवाभ्युदयस्य मृखोपक्षिप्तस्योद्घाटनं भीष्मवधाद् दृष्टमभिमन्युवधान्नष्टम्~। अत्रापि वेदितमिति केचित्~। तदा चार्थो\renewcommand{\thefootnote}{1}\footnote{तदपार्थो} न संगमितः स्यात्~। दृष्टतैव प्रतिमुख उपयोगिनी नष्टता त्ववमर्श एवेति केचिदुत्तरोत्तरविकासतारतम्यं दृष्टनष्टत्वमाहुः~। पूर्वावस्था हि दृष्टाप्युत्तरदृष्टविकासापेक्षया नष्टा~। एवं संमृष्टोपमविकास उत्तरापेक्षयेति मन्यते, अत्रापीवार्थो न संगच्छत एव, न कार्यजननं शक्त्या~।\\

तस्मादयमत्रार्थः \textendash\ दृष्टं नष्टमिव कृत्वा तावन्मुखे न्यस्तं भूमाविव बीजं, अमात्येन सागरिकाचेष्टितं वसन्तोत्सवकामदेवपूजादिना तिरोहितं नष्टमिव\renewcommand{\thefootnote}{2}\footnote{नष्टमेव}

\newpage
% एकोनविंशोऽध्यायः २५

\begin{quote}
{\na \renewcommand{\thefootnote}{1}\footnote{भ \textendash\ उपक्षेपार्थसंयुक्तं}मुखन्यस्तस्य सर्वत्र\renewcommand{\thefootnote}{2}\footnote{प \textendash\ दृश्येत} तद्वै प्रतिमुखं स्मृतम्\renewcommand{\thefootnote}{3}\footnote{ड \textendash\ भवेत्}~॥~४०

उद्भेदस्तस्य\renewcommand{\thefootnote}{4}\footnote{प \textendash\ कार्य भउद्भेदो यत्र} बीजस्य प्राप्तिरप्राप्तिरेव वा~।\\
\renewcommand{\thefootnote}{5}\footnote{ट \textendash\ अतः}पुनश्चान्वेषणं यत्र\renewcommand{\thefootnote}{6}\footnote{भ \textendash\ तस्य द \textendash\ यत्तु य \textendash\ यच्च} स गर्भ इति संज्ञितः~॥~४१}
\end{quote}

\hrule

\vspace{2mm}
\noindent
सागरिकाचेष्टितस्य हि बीजस्येव तदाच्छादकमप्युत्सवादिरूपं भूमिरिव प्रत्युद्बोधकम्~। तस्य दृष्टनष्टतुल्यं कृत्वा न्यस्तस्य , अत एव कुङ्कुमबीजस्य यदुद्घाटनं\renewcommand{\thefootnote}{1}\footnote{यदुच्चाटनं} तत्कल्पं, यत्रोद्घाटनं \underline{सर्वत्रैव} कथाभागसमूहे\renewcommand{\thefootnote}{2}\footnote{भावसमूहो} तत्प्रतिमुखं\renewcommand{\thefootnote}{3}\footnote{स प्रतिमुखं}, प्रतिराभिमुख्येन यतोऽत्र वृत्तिः~। पराङ्मुखता हि दृष्टनष्टकल्पनानिदर्शनम्~। रत्नावल्यां \textendash\ परप्पेसत्तणदूसिदं वि मे सरीरं एदस्स दंसणेण अज्ज मे बहुमदं संपण्णम् ( परपेष्यत्वदूषितमपि मे शरीरमेतस्य दर्शनेनाद्य मे बहुमतं सम्पन्नम्) इत्यादिसागरिकोक्तेरनङ्गाङ्कात् ( प्रथमाङ्कात्) सुसङ्गतारचितराजतत्समागमपर्यन्तं काव्यं द्वितीयाङ्कगतं प्रतिमुखसन्धिः~। उद्घाटितत्वाद् बीजस्य स्तोकमात्रं तु शङ्कुकादिभिरुदाहृतं यत्तदेकदेशलक्षणमिति द्रष्टव्यम्~।\\

\underline{उद्भेद} इति~। \underline{तस्येति} उत्पत्त्युद्घाटनदशाद्वयाविष्टस्य बीजस्य यत्रोद्भेदः फलजननाभिमुख्यत्वं स गर्भः~। उद्भेदमेवं विवृणोति \underline{प्राप्ति}रित्यादिना~। प्राप्तिर्नायकविषया, अप्राप्तिः प्रतिनायकचरिते पुनश्चान्वेषणमित्युभयसाधारणम्~। अन्ये तु वीररौद्रविषय एवैतस्यार्थस्य भावादव्यापित्वादेवमाहुः~। प्राप्तिः, अप्राप्तिरन्वेषणमित्येवं भूताभिरवस्थाभिः पुनः पुनर्भवन्तीभिर्युक्तो गर्भसन्धिः , प्राप्तिसम्भवाख्ययावस्थया युक्तत्वेन फलस्य गर्भीभावात्~। तथा हि \textendash\ रत्नावल्यां द्वितीयेऽङ्के सुसङ्गता \textendash\ अदक्खिणा दाणिं तुमं, जा एव्वं भट्टिणा हत्थेण गहीदा वि कोवं ण मुंचेसि (अदक्षिणा इदानीं त्वं या एवं भर्त्रा हस्तेन गृहीतापि कोपं न मुञ्चसि) इत्यादौ प्राप्तिः~। पुनर्वासवदत्ताप्रवेशेऽप्राप्तिस्तृतीयेऽङ्के~। {\qt तद्वृत्तान्वेषणाय गतश्चिरयति वसन्तकः}

\lfoot{4}

\newpage
\lfoot{}
% २६ नाट्यशास्त्रम्

\begin{quote}
{\na \renewcommand{\thefootnote}{1}\footnote{ट \textendash\ गर्भो ढ \textendash\ गर्भात्}गर्भनिर्भिन्नबीजार्थो \renewcommand{\thefootnote}{2}\footnote{ढ \textendash\ विलोपन, प \textendash\ विप्रलम्भकृतोऽपि वा}विलोभनकृतोऽथवा\renewcommand{\thefootnote}{3}\footnote{च \textendash\ अपि वा}~।\\
\renewcommand{\thefootnote}{4}\footnote{भ \textendash\ किंचिदाश्लेषसंयुक्तोऽ\underline{वमर्श इति संज्ञितः} (ड \textendash\ विमर्श इति स स्मृतः ) प \textendash\ विमर्श इति कीर्तितः, ट \textendash\ सोऽवमर्शः प्रकीर्तितः द \textendash\ विचारो विमर्शः स्मृतः}क्रोधव्यसनजो वापि स विमर्श इति स्मृतः5~॥~४२}
\end{quote}

\hrule

\vspace{2mm}
\noindent
इत्यन्वेषणम्~। विदूषकः \textendash\ ही ही भोः कोसंबीरज्जलंभेणावि ण तारिसो पिअवअस्सस्स परितोसो जारिसो मम सआसादो पिअवअणं सुणिअ भविस्सदि\renewcommand{\thefootnote}{*}\footnote{ही ही भोः कौशाम्बीराज्यलाभेनापि न तादृशः प्रियवयस्यस्य परितोषो यादृशो मत्सकाशात् प्रियवचनं श्रुत्वा भविष्यति~।} इत्यादौ प्राप्तिः~। पुनः {\qt इह तदप्यस्त्येव बिम्बाधरे\renewcommand{\thefootnote}{$\dagger$}\footnote{\begin{quote}
{\qt किंपद्मस्य रुचिं न हन्ति नयनानन्दं विधत्ते न किं\\
वृद्धिं वा झषकेतनस्य कुरुते नालोकमात्रेण किं~।\\
वक्रेन्दौ तव सत्ययं यदपरः शीतांशुरुज्जृम्भते\\
दर्पः स्यादमृतेन चेदिह तदप्यस्त्येव बिम्बाधरे~॥}
\end{quote}}} इति, विदूषकस्य \textendash\ {\qt भो वयस्स, किं अवरं}\renewcommand{\thefootnote}{$\ddagger$}\footnote{भो वयस्य किमपरम्.} इत्यत्र वासवदत्ताप्रत्यभिज्ञानादप्राप्तिः~। पुनः सागरिकायाः सङ्केतस्थानागमने अन्वेषणम्~। पुनर्लतापाशकरणे प्राप्तिरित्येवं गर्भः~। अप्राप्त्यंशश्चात्रावश्यंभावी, अन्यथा हि सम्भावनात्मा प्राप्तिसम्भवः कथं निश्चय एव हि स्यात्~। अवमर्शे त्वप्राप्तेरेव प्रधानता प्राप्त्यंशस्य च न्यूनतेति विशेषः~।\renewcommand{\thefootnote}{$\oint$}\footnote{{\qt अवमर्शे तु प्राप्तेरेव प्रधानता, अप्राप्त्यंशस्य च न्यूनता} इति पाठः स्यात्~। यतः, गर्भसन्धावप्राप्त्यंशः प्रधानं फलसंभावनात्मकत्वात्, अन्यथा स फलनिश्चयात्मक एव स्यादित्युक्तं, तद्व्यतिरेकेऽवमर्शे प्राप्तेरेव प्रधानता~।}\\

\underline{गर्भनिर्भिन्नबीजार्थ} इति~। केचिद्विमर्श इति पठन्ति, अन्येऽवमर्श इति~। तत्र सन्देहात्मको विमर्शः~। ननु पूर्वः सम्भावनाप्रत्ययः , ततः संशय इति नेदमुचितम्, संशयनिर्णयान्तरालवर्तिनं हि तर्कं तार्किकाः प्राहुः~। किं च विमर्शसन्धिर्नियतफलप्राप्त्यवस्थया व्याप्तः, तच्च नियतत्वं सन्देहश्चेति किमेतत्~। अत्राहुः \textendash\ तर्कानन्तरमपि हेत्वन्तरवशाद् बाधच्छलरूपतापराकरणे

\newpage
% एकोनविंशोऽध्यायः २७

\noindent
संशयो भवेत्, किं न भवति~। इहापि च निमित्तबलात्कुतश्चित्संभावितमपि फलं यदा बलवता प्रत्यूह्यते कारणानि च बलवन्ति भवन्ति तदा जनकविघातकयोस्तुल्यबलत्वात् कथं न सन्देहः~। तुल्यबलविरोधकविधीयमानवैधुर्यव्याधूननसन्धीयमानस्फारफलावलोकनायां च पुरुषकारः सुतरामुद्धुरकन्धरीभवतीति तर्कानन्तरमत्र संशयः ततो निर्णय इत्येतदेवोचिततरम्~। तथा हि \textendash\ पुरुषकारशालिन एव श्लाघ्यन्ते, अद्भुतमद्भुतं प्राणसन्देहादप्यनेकात्मा समुत्तारितो यत्र संभावनाविनाभवति , यत एवात्र प्रयत्नतो विधुरप्रयत्नतो य उपनिपातः , तत एव पुरुषकारोद्यतः पुनर्नाशमपि विजिगीषागर्भत्वेन प्रोद्यमं भजतीति तदाशयेन नियता फलप्राप्तिरुच्यते~। श्रेयांसि बहुविघ्नानीति पश्यता तदत्र मया विघ्नापसारणं कर्तव्यमिति साभिमानः स्वमुद्योगसूत्रं सहस्रगुणीकुरुते , तथा हि सागरिकाबन्धनेऽपि महमात्यप्रयुक्तमैन्द्रजालिकवृत्तं सुनिपुणमुपनिबद्धं तावत्~।\\

अन्ये त्ववमर्शो विघ्न इति विदन्ति~। स च व्याख्याने बीजशब्देन तद्बीजफलं अर्थशब्देन निवृत्तिरुच्यते~। तेन गर्भनिर्भिन्नप्रदर्शितमुखं यद्बीजफलं तस्य योऽर्थो निवृत्तिः पुनस्तत्रैव संपादनं निष्प्रत्यूहप्राणतया फलप्रसूतिः, तच्छब्देन यत्रेत्याक्षिप्तम्, सा च निवृत्तिः क्रोधेन च निमित्तेन लोभेन वा व्यसनेन शापादिना वा~। अपिशब्दाद् विघ्ननिमित्तान्तराणां प्रतिपदमशक्यनिर्देशानां सङ्ग्रहः, स च देव्या वासवदत्तया सागरिकायाः कारानिक्षेपात्प्रभृति येयं तुरीयेऽङ्के राज्ञ उक्तिः\textendash

\begin{quote}
{\qt कण्ठाश्लेषं समासाद्य तस्याः प्रभ्रष्टयानया~।\\
तुल्यावस्था सखीवेयं तनुराश्वास्यते मम~॥}
\end{quote}

\noindent
अत्न विघ्ने वासवदत्ताक्रोधो निमित्तम्~। लोभस्तु निमित्तं, यथा तापसवत्सराजे \textendash\ {\qt त्वत्संप्राप्तिविलोभितेन सचिवैः प्राणा मया धारिताः\renewcommand{\thefootnote}{*}\footnote{\begin{quote}
{\qt त्वत्संप्राप्तिविलोभितेन सचिवैः प्राणा मया धारिताः\\
तन्मत्वा त्यजतः शरीरकमिदं नैवास्ति निस्स्नेहता~।\\
आसन्नोऽवसरस्तवानुगमने जाता धृतिः किं त्वियान्\\
खेदो यच्छतधा गतं न दृदयं तस्मिन् क्षणे दारुणे~॥}
\end{quote}
अयं श्लोको भिन्नपाठत्वेन कुन्तकाभिनवगुप्तभट्टनायकहेमचन्द्रादिभिरुदाहृतः~।}}(६ \textendash\ ३)\\ इति~। तपपरे न सहन्ते \textendash\ न ह्यत्र वासवदत्ताप्राप्तिलोभः प्रकृते फले विघ्न \textendash

\newpage
% २८ नाट्यशास्त्रम्

\noindent
कारीति , इदं तदोदाहरणं \textendash\ तत्रैव परिणीतायामपि पञ्चावत्यां वासवदत्तामलभमानस्य राज्ञो मरणाध्यवसायो मुमूर्षोः , तदलोभे मन्त्रिणां सुतरां राज्यप्राप्तिदीर्घलाभो निमित्तमिति~।\\

तच्च व्यसनं त्वमर्षनिमित्तमिति अभिज्ञानशाकुन्तले दर्शितम्~। एवमन्यदुत्मेक्ष्यम्~। तथा हि \textendash\ सपत्न्यां विद्याप्रभावो निमित्तमवमर्शे, क्वचिहैवं, क्चित्समयः \textendash\ यथा विक्रमोर्वश्यां पुत्रवदनावलोकनादूर्वश्याः स्वर्गगमनाध्यवसाये~।\\

\underline{अन्ये} त्वादृत्तिविमर्शशब्दं कल्पयन्त इत्थं व्याचक्षते \textendash\ गर्भान्निभिन्नो बीजार्थफलं यस्मिन् , विमर्शादिकारणत्वाद् विमर्शरूपे कथावयवे स विमर्शो नामेति~। अत्र व्याख्याने मुख्यमस्य सन्धेर्यद्रूप विदूरकारणसंपातात्मकत्वं नाम तदस्पृष्टमेव स्यात्~।\\

अन्ये तु लाभयोग्यत्वं, नाशावस्था, अन्वेषणावस्था च यथारुचि गर्भे यथारुचि निबन्धनीया, तत्र यदा लाभात्मिका प्राप्त्यवस्था प्रतिमुखेनैव बध्यते तदान्ये द्वे गर्भे सन्धौ, यदाप्यवमर्शे नाशावस्था तदा गर्भेऽन्वेषणमेव, गर्भे यदा नाशान्वेषणे तदा चावमर्शै विचारो निबन्धनीयः~। कथं मया प्राप्तप्रायमप्यपहारितं, किमत्र विगुणोपायानुष्ठानं मया कृतं , उत प्राप्तियोग्यमेवैतन्न भवतीति, ( यदाहोद्भटः \textendash\ यासावन्वेषणभूमिरवमृष्टिरवमर्श इति, तच्चेदं व्याख्यानं लक्ष्यविरुद्धं युक्त्या च पूर्वोदितप्रारम्भाद्यवस्थापञ्चकगतक्रमनियमसमर्थनप्रस्तावोक्तया विरुद्धमित्यास्ताम्~।\\

{\qt अहमनेन विफलायां क्रियायां विलोभ्य प्रवर्तित इति यत्र कर्ता विमृशति स विलोभनकृत इति, क्रोधव्यसनादेस्तु व्यापद्यमान फलव्यापत्तिविषयो यः कर्तुर्विचारः स क्रोधव्यसनजे विमर्श इत्येवं विमर्शनखभाव एव विमर्शः , कार्यविनिपातस्तू्तरनिर्वहणसन्धिनिबध्यमानाद्भुतरसपरिपोषकत्वेन निबध्यते} इति श्रीशङ्कः~। तन्मते विचारस्य सर्वसन्ध्यनुयायित्वात् पृथग्विमर्शसत्त्वेनाभिधानं स्यात्~।\\

व्यापत्तिविषयो विचार इति केचित्~। पुनरप्यस्य सरणिरेव\renewcommand{\thefootnote}{1}\footnote{अस्मत्सरणिरेव.}, सा च न व्याख्यानेन क्रमेण दर्शिता~। विलोभनकृतोदाहरणं तु न व्यापत्तिविमर्श इति सर्व त्वसमञ्जसं यथारुचि परिकल्पितमित्यलमनेन~।

\newpage
% एकोनविंशोऽध्यायः २९

\begin{quote}
{\na \renewcommand{\thefootnote}{1}\footnote{प \textendash\ यत्रानयन}समानयनमर्थानां \renewcommand{\thefootnote}{2}\footnote{ढ \textendash\ मुखार्थानां}मुखाद्यानां सबीजिनाम्\renewcommand{\thefootnote}{3}\footnote{भ \textendash\ यथातथम्, प \textendash\ महौजसाम्}~।\\
\renewcommand{\thefootnote}{4}\footnote{ड \textendash\ फलोपसङ्गतानां च प \textendash\ फलोपबृंहितानां स्याज्ज्ञेयं निर्वहणं च तत्}नानाभावोत्तराणां \renewcommand{\thefootnote}{5}\footnote{भ\textendash\ च यत्र}यद्भवेन्निर्वहणं तु तत्~॥~४३}
\end{quote}

\hrule

\vspace{2mm}
\underline{समानयन}मिति~। मुखाद्यानां चतुर्णां सन्धीनां येऽर्थाः प्रारम्भाद्याः तेषां सहबीजिभिः बीजविकारैः क्रमेणावस्थाचतुष्टयेन भवद्भिः उत्पत्त्युद्घाटनोद्भेदगर्भनिर्भेदलक्षणैः वर्तमानानां नानाविधैः सुखदुःखात्मकैः हासशोकक्रोधादिभिर्भावैरुत्तराणां चमत्कारास्पदत्वे जातोत्कर्षाणां यत्समाननयनं, यस्मिन्नर्यराशौ समानीयन्ते फलनिष्पत्तौ योज्यन्ते तन्निर्वहणं\renewcommand{\thefootnote}{1}\footnote{क \textendash\ निवर्हणं} पलयोगावस्थया व्याप्तम्~। अत्र केचित्मून् सर्वान् सन्धीनवस्थापञ्चकनिर्वहणे पृथग्वृत्त्या \renewcommand{\thefootnote}{2}\footnote{योज्यमानमिच्छन्ति~।}योज्यमानानिच्छन्ति~।\\

अन्ये तु सन्धौ सन्ध्यन्तरानुप्रवेशमिच्छन्तोऽपि प्रागवस्थाया एकोत्तरावस्थापरिणामात्मकत्वे कारणं न पश्यन्त्यपि तु (ताः) कार्यीभवन्तीति सांख्यदर्शनतच्छायाश्रयेणैकावस्थायाः फलसंबन्धसंगमनोपकरणभावप्राप्तं तदेकभावानामवस्थान्तराणां फलसंगमनमुचितमेवेति मन्यन्ते~।\\

अन्ये तु मुखसन्धौ ये अवलम्ब्यमानतया आद्याः प्रधानभूता अर्थाः उपायास्ते महौजसः फलसंपत्तौ साधकाः तेषां फलसंगत्या समाननयनमिति व्याचक्षते~। {\qt महौजसां फलोपसंगतानां च} इति पाठे\renewcommand{\thefootnote}{3}\footnote{भोजदेवोऽपि \textendash\ {\qt महौ \textendash\ जसां , फलोपसङ्गतानां } इति~।} \textendash\ यदा च सुखप्राप्तेः फलवत्त्वं तदा रतिहासादिबाहुल्यं प्रारम्भादीनां, दुःखहानेस्तु फलत्वे क्रोधशोकादिदुःखात्मकभावाद् बाहुल्यं, (उभयत्र) स्वोचितव्यभिचारिसहितं द्रष्टव्यम्~। उदाहरणं रत्नावल्यामैन्द्रजालिकप्रवेशात्प्रभृत्यासमाप्तेः~। एषामवस्था सन्ध्यादीनां नायकतदमात्यतत्परिवारनायिकादिमुखेनापि नियोजनं न त्वेकमुखेनैवेति नियम इत्युक्तं पूर्वमेव~।

\newpage
% ३० नाट्यशास्त्रम्

\begin{quote}
{\na एते तु\renewcommand{\thefootnote}{1}\footnote{ड \textendash\ हि} सन्धयो ज्ञेया \renewcommand{\thefootnote}{2}\footnote{ट \textendash\ नाटकेषु}नाटकस्य प्रयोक्तृभिः~।\\
\renewcommand{\thefootnote}{3}\footnote{ढ \textendash\ यथा}तथा प्रकरणस्यापि शेषाणां च निबोधत~॥~४४

डिमः समवकारश्च चतुःसन्धी प्रकीर्तितौ\renewcommand{\thefootnote}{4}\footnote{ट \textendash\ चतुःसन्धिः प्रकीर्तितः, ड \textendash\ प्रकीर्तितौ~। गर्भावमर्शहीनौ ( नः) तु कर्तव्यौ (व्यः) कविभिः सदा *~। कैशिकीवृत्ति हीनौ तु (नस्तु) कर्तव्यौ च ( व्यश्च ) त्रिवृत्तिकौ ( कः)~। भारत्या त्वथ सात्त्वत्या ह्यारभट्या तथैव च~॥}~।\\
\renewcommand{\thefootnote}{5}\footnote{ट \textendash\ विमर्शस्तु तयोर्न स्यान्न च वृत्तिस्तु कैशिकी , न \textendash\ न तयोर्विमर्शस्तु स्यान्न च वृत्तिस्तु कैशिकी~।}न तयोरवमर्शस्तु कर्तव्यः कविभिः सदा~॥~४५

व्यायोगेहामृगौ चापि सदा कार्यौ त्रिसन्धिकौ\renewcommand{\thefootnote}{6}\footnote{भ \textendash\ त्रिसन्धी संप्रकीर्तितौ ( ड \textendash\ परि )}~।\\
\renewcommand{\thefootnote}{7}\footnote{प \textendash\ न गर्भो न विमर्शश्च न च वृत्तिश्च कैशिकी द \textendash\ गर्भो विमर्शो न स्यातां न च वृत्तिस्तु, ड \textendash\ गर्भं चैवावमर्शं च त्यक्त्वा वृत्तिं च कैशिकीम्}गर्भावमर्शौ न स्यातां तयोर्वृत्तिश्च कौशिकी~॥~४६

द्विसन्धि तु प्रहसनं वीथ्यङ्को भाण एव च~।\\
मुखनिर्वहणे \renewcommand{\thefootnote}{8}\footnote{भ \textendash\ चैव, ब \textendash\ तेषां, ट \textendash\ तेषु ड \textendash\ स्यातां तेषां वृत्तिश्च भारती,}तत्र कर्तव्ये कविभिः सदा\renewcommand{\thefootnote}{9}\footnote{भ \textendash\ कैशिकीं विना भ \textendash\ द्विजाः}~॥~४७

\renewcommand{\thefootnote}{10}\footnote{अयं श्लोकार्थः पूर्वाध्याये (१८ \textendash\ १२) उक्त एव भ \textendash\ मातृकायामिहैव \textendash\ दृश्यते}[ वीथी चैव हि भाणश्च तथा प्रहसनं पुनः~।\\
कैशिकीवृत्तिहीनानि कार्याणि कविभिः सदा ]~॥}
\end{quote}

\hrule

\vspace{2mm}
एतेषां विनियोगं विभजति \underline{एते त्वित्या}दिना {\qt मुखनिर्वहणे तत्र \underline{कर्तव्ये कविभिः सदा}} इत्यन्तेन~। एतच्च पूर्वमेवनिर्णीतार्थं {\qt एकलोपे चतुर्थे} स्येत्यादि (१९ \textendash\ २७) व्याख्यानावसरे~।\\

कस्मात्तौ (डिमसमवकारौ) चतुःसन्धी इत्याह \underline{न तयोरि}त्यादिना~। तुर्हेतै, यतस्तयोरवमर्शं निबद्धुमशक्यमिति~। एवमुत्तरत्रापि हेतुग्रन्थान्तरत्वे \textendash

\newpage
% एकोनविंशोऽध्यायः ३१

\begin{quote}
{\na एवं \renewcommand{\thefootnote}{1}\footnote{भ \textendash\ तु}हि सन्धयः कार्या \renewcommand{\thefootnote}{2}\footnote{भ \textendash\ दशरूपेषु}दशरूपे प्रयोक्तृभिः~।\\
\renewcommand{\thefootnote}{3}\footnote{ड \textendash\ पुनः सन्ध्यन्तरं तेषां}पुनरेषां तु सन्धीनामङ्गकल्पं निबोधत\renewcommand{\thefootnote}{4}\footnote{ड \textendash\ अतः परं {\qt साम भेदः} इत्यदि श्लोकत्रयं वर्तते}~॥~४९

सन्धीनां यानि वृत्तानि \renewcommand{\thefootnote}{5}\footnote{च \textendash\ प्रदेशश्च तु पूर्वतः, ढ \textendash\ प्रवेशेषु}प्रदेशेष्वनुपूर्वशः~।\\
\renewcommand{\thefootnote}{6}\footnote{प \textendash\ सु}स्वसम्पद्गुणयुक्तानि तान्यङ्गान्युपधारयेत्\renewcommand{\thefootnote}{7}\footnote{म \textendash\ अवधारयेत्}~॥~५०}
\end{quote}

\hrule

\vspace{2mm}
\noindent
नेदं योज्यम्, न तु लोपस्थानित्वेन, तस्यैको लोप इत्यादिना पूर्वमेवोक्तत्वात्~।\\

ननु सन्धिपञ्चकात्मक इतिवृत्तशरीरारम्भे कथं दशरूपकादिभेद इत्या शंक्याह \underline{एवं ही}ति~। हिर्यस्मात् , एवमुक्तेन विनियोगप्रकारेण सन्धयो भवन्ति ततो दशरूपभेद इति केचिदाशङ्कापूर्वकं व्याचक्षते, तच्चासत्, \renewcommand{\thefootnote}{1}\footnote{पक्षभेदादेव}लक्ष्यभेदादेव दशरूपकभेदस्य दर्शितत्वात्~। अवश्यं चेतत्, अन्यद्वा डिमसमवकारयोश्चतुःसन्धिताविशेषात् कथं भेदः स्यात् , नाटकादीनां वा, तस्मादुपसंहारग्रन्थोऽयमिति हीति~। अङ्गानां कल्पं कल्पनाप्रकारो वा तेनैवंप्रायमन्यदपीतिवृत्तोपयोगि भवति~।\\

अङ्गानां सामान्यस्वरूपं प्रयोजनद्वारेण दर्शयितुं प्रथमेन स्वरूपं द्वाभ्यां प्रयोजनमेकेन द्वयं द्वयेन प्रकाशयन्नाह श्लोकषट्कं \underline{\qt सन्धीनां यानि वृत्तानी} त्यादि \underline{\qt शोभामेति न संशयः} इत्यन्तम्~।\\

अर्थभागराशिः सन्धिरित्युक्तं तत्र सन्धीनां संबन्धनीयानि वृत्तानि संविधानखण्डानि~। अनुपूर्वश इति मुख्यप्रयोजनसंपादनबलोपनतेन क्रमेण, न तु लक्षणनिरूपणप्रसङ्गपरिकल्पितेन, \underline{प्रदेशे}ष्वादिमध्यान्तभागेषु वर्तनेनाङ्गानि , कुत इत्याह स्वस्याङ्गिनः सन्धेर्या संपत्तेर्निष्पत्तिः तत्र गुणवत्वे शेषभावे यतो यतो युक्तान्युचितानि संबन्धसंपादकत्वादङ्गानीत्यर्थः~।

\newpage
% ३२ नाट्यशास्त्रम्

\begin{quote}
{\na इष्टस्यार्थस्य रचना\renewcommand{\thefootnote}{1}\footnote{ड \textendash\ वचनं य \textendash\ वचनात्} वृतान्तस्यानुपक्षयः~।\\
रागप्राप्तिः प्रयोगस्य गुह्यानां चैव गूहनम्\renewcommand{\thefootnote}{2}\footnote{च \textendash\ च निगूहनम्}~॥~५१

आश्चर्यवदभिख्यानं प्रकाश्यानां प्रकाशनम्~।\\
\renewcommand{\thefootnote}{3}\footnote{इदमर्धं च \textendash\ मातृकायां न दृस्यते}अङ्गानां षड्विधं ह्येतद् \renewcommand{\thefootnote}{4}\footnote{ड \textendash\ उक्तं शास्त्र}दृष्टं शास्त्रे प्रयोजनम्~॥~५२

अङ्गहीनो नरो \renewcommand{\thefootnote}{5}\footnote{ट \textendash\ यावत्}यद्वन्नैवारम्भं\renewcommand{\thefootnote}{6}\footnote{ड \textendash\ आरम्भे}क्षमो भवेत्~।\\
अङ्गहीनं तथा \renewcommand{\thefootnote}{7}\footnote{ट \textendash\ कार्यमप्रयोग}काव्यं न प्रयोगक्षमं भवेत्~॥~५३}
\end{quote}

\hrule

\vspace{2mm}
अन्ये त्वाहुः \textendash\ स्वसंपदो बीजोत्पत्त्युद्घाटनादिका गुणाश्च शब्दार्थवैचित्र्याणि , स्वसंपदां वा गुणाः तैरेव युक्तानीति~।

\underline{इष्टस्ये}त्यादिना योजनमाह~। अभीष्टस्य प्रयोजनस्य रसास्वादकृतो \underline{रचना} विस्तारणा~। \underline{वृत्तान्तस्यानुपक्षयः} क्रमेण स्फुटत्वादयः शलाकाकल्पत्वाभावः, एतत्प्रयोजनं सर्वसाधारणम्~। प्रयोगस्येतिवृत्तस्य स्वयं परस्परस्यापि रागप्राप्तिः रञ्जनायोग्यत्वलाभः व्युत्पत्त्यवस्थायोगात् , यदि वा पौनरुक्त्याद्याभासे गुह्याः संछादनीया अर्थाः तेषां संछादनम्~। पुनः पुनः श्रुतमपि यदभिख्यानं इतिवृत्तं तत एव नाश्चर्यकारि तदपि अङ्गयोजनायाम \textendash\ पूर्वतामिव दधदद्भुततामेति, तदाह \underline{आश्चर्य}वदिति~। यच्च व्युत्पत्तौ सातिशयोपयोगि तत एव प्रकाश्यं तस्य प्रकाशनं विस्तारणम्, आद्यन्तु प्रयोजनं चमत्कारकृतं स्मृतिदृष्टमपि प्रत्यक्षविशेषसिद्धमेव, न तु सन्ध्योपासनादिवददृष्टं, नापि पूर्वरङ्गाङ्गवदुभयरूपमित्यर्थः~। \underline{शास्त्र} इति नाट्यात्मके वेद इत्यर्थः~। एषां प्रयोजनानामङ्गलक्षणेषूदाहरणं वर्णयिष्यामः अत एव दृष्टान्तेन द्रढयति~। अङ्गकर्तव्यसंपादनं \underline{प्रयोगक्षम}मिति~। ततः प्रयोजनस्यासंपत्तेर्दृष्टस्य वा

\newpage
% एकोनविंशोऽध्यायः ३३

\begin{quote}
{\na \renewcommand{\thefootnote}{1}\footnote{अयं श्लोकः सर्वासु मातृकासु {\qt काव्यं यदपि} इत्यादि श्लोकानन्तरमेव दृश्यते~।}उदात्तमपि यत्काव्यं\renewcommand{\thefootnote}{2}\footnote{च \textendash\ कार्यं} \renewcommand{\thefootnote}{3}\footnote{ट \textendash\ तदङ्गैः}स्यादङ्गैः परिवर्जितम्~।\\
हीनत्वाद्धि\renewcommand{\thefootnote}{4}\footnote{ड \textendash\ तत्} प्रयोगस्य न सतां \renewcommand{\thefootnote}{5}\footnote{ट \textendash\ सतो}रञ्जयेन्मनः~॥~५४

काव्यं \renewcommand{\thefootnote}{6}\footnote{न \textendash\ पदविहीनार्थ}यदपि हीनार्थं सम्यगङ्गैः समन्वितम्~।\\
\renewcommand{\thefootnote}{7}\footnote{भ \textendash\ दीप्ताङ्गत्वात्, ब \textendash\ दीप्तिं गत्वा प्रयोगश्च, ढ \textendash\ दीप्ताङ्गस्य}दीप्तत्वात्तु प्रयोगस्य शोभामेति न संशयः~॥~५५

[ तस्मात् सन्धिप्रदेशेषु \renewcommand{\thefootnote}{8}\footnote{च \textendash\ प्रयोगेषु}यथायोगं\renewcommand{\thefootnote}{9}\footnote{ड \textendash\ देशं प \textendash\ काव्यं} यथारसम्~।\\
\renewcommand{\thefootnote}{10}\footnote{भ \textendash\ कार्याण्यङ्गानि तेषां तु प्रविभागः प्रदर्श्यते, ब \textendash\ कविता}कविनाङ्गानि कार्याणि सम्यक्तानि निबोधत ]\renewcommand{\thefootnote}{11}\footnote{अतः परं यमनप \textendash\ आदर्शेषु {\qt साम भेद}इत्यादि श्लोकत्रयं च, तदनु \textendash\ एते विशेषाः सन्धीनां स्युः सन्धिष्वर्थयोगतः~। एभ्योऽङ्गान्यर्थयोगेन सन्धितानि निबोधत इति श्लोको वर्तते}~॥~५६}
\end{quote}

\hrule

\vspace{2mm}
\noindent
प्रच्युतसंभावनात्~। एतद् व्यतिरेकद्वारेण स्फुटयति \underline{उदात्तमपीति} लक्षणगुणालङ्कृतियुक्तमित्यर्थः~। प्रयोगस्येति अपादानमपि संबन्धित्वेन (षष्ठी), वृक्षस्य पर्णं पततीति यथात्र~। तस्य प्रयोगस्य तस्य काव्यस्य यतो हीनत्वं यदयोग्यत्वं यस्मात् , सतां परोपकारप्रवृत्तानां कविनटानां साधुभूतानां वा सामाजिकानां मनो न रञ्जयतीति संभाव्यते~। अन्वयद्वारेणोपसंहरति \underline{यदपि} इति~। \underline{हीना}र्थमिति स्वल्पमपि प्रयोजनं प्रहसननिदर्शनकथाख्यायिदि\renewcommand{\thefootnote}{*}\footnote{प्रहसनानिदर्शितकथंचाधीक्यादि (?)}~। प्रयोगः प्रयुक्तिः तत्राङ्गं प्रयोजकं रञ्जनातिशयो व्युत्पत्त्यतिशयश्च तदुभयम्, तत्र काव्ये दीप्तं स्फुटमित्यर्थः~।

\lfoot{5}

\newpage
\lfoot{}
% ३४ नाट्यशास्त्रम्

\begin{quote}
{\na उपक्षेपः परिकर: परिन्यासो विलोभनम्~।\\
युक्तिः प्राप्तिः समाधानं विधानं परिभावना~॥~५७

उद्भेदः करणं भेद \renewcommand{\thefootnote}{1}\footnote{म \textendash\ द्वादशाङ्गानि}एतान्यङ्गानि वै मुखे~।\\
तथा प्रतिमुखे चैव शृणुताङ्गानि नामतः\renewcommand{\thefootnote}{2}\footnote{ट \textendash\ वक्ष्याम्यङ्गानि ढ \textendash\ वक्ष्याम्यङ्गान्यतः परम् ड \textendash\ अतः परम्}~॥~५८
 
विलासः परिसर्पश्च विधूतं \renewcommand{\thefootnote}{3}\footnote{ड \textendash\ शमनं}तापनं तथा~।\\
\renewcommand{\thefootnote}{4}\footnote{ट \textendash\ नर्मद्युतिः प्रगमनं विरोधः पर्युपासनं\ (भ \textendash\ शम , नि )}नर्म नर्मद्युतिश्चैव तथा \renewcommand{\thefootnote}{5}\footnote{ढ \textendash\ प्रसवणं, न \textendash\ प्रगमनं च \textendash\ प्रशमनं}प्रगयणं पुनः~॥~५९

निरोधश्चैव विज्ञेयः पर्युपासुनमेव च~।\\
\renewcommand{\thefootnote}{6}\footnote{ड \textendash\ वज्रं पुष्यं}पुष्पं वज्रमुपन्यासो वर्णसंहार एव च\renewcommand{\thefootnote}{7}\footnote{अतःपरं, नमयोः \textendash\ संपदार्थानि बीजस्य संप्रसिद्धिकराणि च{\qt इत्यर्धमप्युपलभ्यते}~॥~६०

\renewcommand{\thefootnote}{8}\footnote{भ \textendash\ एतान्येव}एतानि वै प्रतिसुखे \renewcommand{\thefootnote}{9}\footnote{भ \textendash\ गर्भाङ्गानि, य \textendash\ गर्भे चैव}गर्भेऽङ्गानि निबोधत~।\\
अभूताहरणं मार्गो रूपोदाहरणे क्रमः\renewcommand{\thefootnote}{10}\footnote{भ \textendash\ रूपमाहरणं क्रमः ढ \textendash\ हरणक्रमाः}~॥~६१

संग्रहश्चानुमानं च \renewcommand{\thefootnote}{11}\footnote{च \textendash\ प्रार्थनाक्षिप्तिरेव भ \textendash\ तोटकाधिबले तथा~। उद्वेगसंभ्रमाक्षेपा गर्भाङ्गानीति योजयेत्}प्रार्थनाक्षिप्तमेव च~।\\
तोटिकाधिबले चैव \renewcommand{\thefootnote}{12}\footnote{ट \textendash\ चोद्भेदो विभ्रमः}ह्युद्वेगो विद्रवस्तथा~॥~६२}}
\end{quote}

\hrule

\vspace{2mm}
अथाङ्गानामुद्देशमाह \underline{उपक्षेपः परिकर} इत्यादिना \underline{एतान्यङ्गानि सन्धिषु} (६७) इत्यन्तेन~। मुखे द्वादश , प्रतिमुखे गर्भे च त्रयोदश, अवमर्शे द्वादश , निर्वहणे चतुर्दशेति मिलित्वा चतुःषष्टिः~।\\

केचिन्मन्यन्ते \textendash\ इह उपक्रम उपसंहारो मध्यमिति प्रत्यवस्थं स्थानभेदत्रयं , तत्र प्रत्येकं सूक्ष्मेणारम्भावस्थापञ्चकेन भाव्यमिति पञ्चदश्यो

\newpage
% एकोनविंशोऽध्यायः ३५

\begin{quote}
{\na \renewcommand{\thefootnote}{1}\footnote{भ \textendash\ अवमर्शे तु वक्ष्यानि यथाङ्गानि भवन्ति हि ड \textendash\ अङ्गान्येतानि वै गर्भे विमर्शे च,}एतान्यङ्गानि वै गर्भे \renewcommand{\thefootnote}{2}\footnote{ट \textendash\ विमर्शे च}ह्यवमर्शे निबोधत~।\\
\renewcommand{\thefootnote}{3}\footnote{भ \textendash\ अवपातोऽथ}अपवादश्च\renewcommand{\thefootnote}{4}\footnote{म \textendash\ अथ} संफेटो\renewcommand{\thefootnote}{5}\footnote{प \textendash\ संस्फोटो} विद्रवः \renewcommand{\thefootnote}{6}\footnote{भ \textendash\ द्रव एव च~। युक्तिर्द्युतिः प्रसङ्गश्च व्यवसायो विरोधनम्~। प्ररोचनाविचलनमादानं च्छेदनं तथा~। अवमर्शे समाख्यातान्येतान्यङ्गानि वै द्विजाः~॥}शक्तिरेव च~॥~६३

\renewcommand{\thefootnote}{7}\footnote{न \textendash\ प्रसङ्गो व्यवसायश्च विरोधश्च प्रकीर्तितः~। प्ररोचनातिर्बलनमादानं खलनं तथा~। ड \textendash\ एतान्यवमृशेऽङ्गानि भूयो निर्वहणे शृणु~॥}व्यवसायः प्रसङ्गश्च द्युतिः खेदो निषेधनम्~।\\
विरोधनमथादानं छादनं\renewcommand{\thefootnote}{8}\footnote{च \textendash\ छन्दनं} च प्ररोचना~॥~६४

\renewcommand{\thefootnote}{9}\footnote{य \textendash\ व्याहा\textendash\ रश्चैव}व्यवहारश्च युक्तिश्च विमर्शाङान्यमूनि च~।\\
सन्धिर्निरोधो\renewcommand{\thefootnote}{10}\footnote{ड \textendash\ विरोधो} ग्रथनं निर्णयः परिभीषणम्~॥~६५

\renewcommand{\thefootnote}{11}\footnote{भ \textendash\ स्तुतिः, न \textendash\ कृतिः}द्युतिः प्रसाद आनन्दः\renewcommand{\thefootnote}{12}\footnote{ड \textendash\ प्रसादानन्दौ च} समयो \renewcommand{\thefootnote}{13}\footnote{ड \textendash\ अप्युप, य \textendash\ श्चोप,}ह्युपगूहनम्~।\\
\renewcommand{\thefootnote}{14}\footnote{य \textendash\ आभाषणं पूर्वभावं}भाषणं पूर्ववाक्यं च \renewcommand{\thefootnote}{15}\footnote{ड \textendash\ कार्य}काव्यसंहार एव च~॥~६६

प्रशस्तिरिति \renewcommand{\thefootnote}{16}\footnote{भ \textendash\ चाङ्गानि कुर्यान्निर्वहणे पुनः, म \textendash\ संहारे सन्ध्यङ्गानि चतुर्दश}संहारे ज्ञेयान्यङ्गानि नामतः\renewcommand{\thefootnote}{17}\footnote{अतः परं यढ योः {\qt सन्धौ निर्वहणाख्ये तु कर्तव्यानि प्रयोक्तृभिः~। एतेषामर्थसंबन्धं पुनर्वक्ष्यामि लक्षणम्~।} इत्यधिकः श्लोको विद्यते~।}~।\\
चतुष्षष्ठि बुधैर्ज्ञेयान्येतान्यङ्गानि सन्धिषु~॥~६७}
\end{quote}

\hrule

\vspace{2mm}
\noindent
दशाः क्रमभाविन्यः , तत्राद्यास्तावद्दशानामङ्गत्वेन् वर्ण्यन्ते~। अङ्गिबुद्ध्युदयात्~। तत्रेति चतुर्दश निर्वहणे फलयोगबलात् सर्वा एवोपपाद्यन्ते~। अन्यत्र तु मुखादौ काश्चिल्लीनीक्रियन्ते , न द्वादशादिभेदानि तत्राङ्गानीति~। तदेत \textendash

\newpage
% ३६ नाट्यशास्त्रम्

\begin{quote}
{\na [ संपादनार्थं बीजस्य सम्यक्\renewcommand{\thefootnote}{1}\footnote{ड \textendash\ सन्धिकराणि तु}सिद्धिकराणि च~।\\
कार्याण्येतानि कविभि\renewcommand{\thefootnote}{2}\footnote{ड \textendash\ विस्पष्टार्थानि}र्विभज्यार्थानि नाटके~॥ ]~६८

\renewcommand{\thefootnote}{3}\footnote{च \textendash\ एतेषां तु पुनर्वक्ष्ये}पुनरेषां प्रवक्ष्यामि लक्षणानि यथाक्रमम्~।}
\end{quote}

\hrule

\vspace{2mm}
\noindent
दसत्~। एवं हि वक्ष्यमाणेषु तेनैव क्रमेण भाव्यम्~। न चासावस्ति प्रयोजनशङ्काषट्कं ततश्चानुपपन्नं स्यात् , अनुपक्षय इत्येकमेव हि प्रयोजनं भवेत्~। बीजकरणेऽपि च नियमनिदानानुपपत्तौ द्वयोर्द्वादश द्वयोस्त्रयोदशेति कुतस्त्यो विभाग इत्यास्तामदः~।\\

\underline{पुनरेषामिति~।} पुनश्शब्दो विशेषद्योतकः, लक्षणं एवायं क्रमो न निबन्धन इति यावत्~। तेन यदुद्भटप्रभृतयोऽङ्गानां सन्धौ क्रमे च नियममाहुस्तद्युक्त्यागमविरुद्धमेव~। तथा हि \textendash\ {\qt संप्रधारणमर्थानां युक्तिरित्यभिधीयते} इति यन्मुखसन्धौ पञ्चममङ्गं वक्ष्यति तत्सर्वेषु सन्धिषु तावन्निबन्धनयोग्यं, न च तथा निवेश्यं बध्यमानमदृष्टकृतं विदध्यात्~। न च लक्ष्येन दृश्यते~। वेणीसंहारे हि तृतीयेऽङ्के गर्भसन्धौ दुर्योधनकर्णयोर्महति संप्रधारणे द्रोणवधे वृत्ते \textendash

\begin{quote}
{\qt तेजस्वी रिपुहतबन्धुदुःखपारं\\
बाहुभ्यां तरति धृतायुधप्लवाभ्याम्~।\\
आचार्यः सुतनिधनं निशम्य संख्ये\\
किं शस्त्रग्रहसमये विशस्त्र आसीत्~॥}
\end{quote}

\noindent
इत्यादि यावत्\textendash

\begin{quote}
{\qt दत्त्वाभयं सोऽतिरथो हन्यमानं किरीटिना~।\\
सिन्धुराजमुपेक्षेत नैवं चेत् कथमन्यथा~॥} इति~।
\end{quote}

\noindent
न चात्र प्रतीतिव्युत्पत्त्योः क्षतिः काचित्~। यत्तु सन्धिनैयत्येनाभिधानं तत्र सन्धाववश्यंभावित्वख्यापनार्थं युक्तिर्मुखे भवत्येव~। सन्ध्यन्तरालानि तु नेत्थमिति पृथक् तानि वर्णयिष्यन्ते~। कानिचित्त्वङ्गानि स्वरूपबलादेव

\newpage
% एकोनविंशोऽध्यायः ३७

\noindent
नियमभाञ्जि, यथोपक्षेपो मुखसन्धावेव प्रथमे~। एवं च न ह्यनुपक्षिप्ते वस्तुनि किञ्चिदपि शक्यक्रियम्~। यत्तूच्यते {\qt (अ १९) चतुःषष्ट्यङ्गसंयुत} मिति तेन संभवमात्रमेषामुक्तं, न तु नियमः~। यथासन्धिकृतः कर्तव्यानीति वचनं प्रत्युत सन्ध्यौचित्येनैषां निबन्धनमभिदधदस्मदभिहितनीतिपथोपदेश्येव, योग्यतार्थवृत्तिना हि यथाशब्देनायमव्ययीभावः~।\\

यत्तूक्तं शरीराङ्गनियमदर्शनात् कथमेतदिति, तत्रापि दृष्टान्ताद् व्यवस्थापि तु न संभवतः स चास्तीत्युक्तं, शाखादयश्च वृक्षावयवा मध्येऽपि ब्रध्नेऽप्यूर्ध्वेऽपि भवन्ति, न च शरीरे पादपादिवदुपक्षेपादिभिरवयवविकल्पः सन्धिरारभ्यते, यच्च प्रतिसन्ध्यभिधानं तद्बाहुल्येन तथा दर्शनात्~। तथाप्युपक्षिप्तेऽर्थे विस्तारिते निश्चितगुणादभिलषिते संभावनीयमुपायादिविषयं संप्रधारणमित्युपक्षेपपरिकरपरिन्यासविलोभनहेतुत्वादन्यान्यभिधाय युक्तिरुक्ता, न तु तत्रैव सद्भावात्~। आनन्तर्यनियमश्च मुनेरनभिमतो लक्ष्यते~। अन्यथा सन्ध्यन्तरालानि सामादीनि मदान्तान्येकविंशतिः, लास्याङ्गानि गेयपदादीनि दश यानि वक्ष्यन्ते, तेषां कुत्र निवेशः स्यात्~। सन्धिपञ्चकमयं हि रूपकं क्रमनियतं, तदङ्गसंहारभावितश्च सन्धिरिति, न च क्रमेणानेनैव तानि प्रयोज्यानीति वचनमस्ति~। सदपि वा न्यायापेतमन्यथा योज्येत~। न चोद्देशक्रममनूच्यते निबन्धं,\renewcommand{\thefootnote}{*}\footnote{{\qt क्रम इत्युच्यते निबन्धः} इति स्यात्~।} लक्षणालङ्कारगुणवीथ्यङ्गसन्ध्यन्तराणि लास्याङ्गवृत्तितदङ्गान्यपि तु असाधर्म्यदृष्टान्तः~। तदेतत्प्रत्येकं लक्षणे स्फुटीभविष्यतीत्यास्तां तावत्~।\\

१ \underline{उपक्षेपः} \textendash\ तत्र प्रस्तावना न तावद्रूपकाङ्गं नटवृत्तव्याप्ततयेतिवृत्ताननुप्रवेशात्~। \underline{इति} तदनन्तरं पूर्वं, \underline{काव्यार्थ} इतिवृत्तशरीरलक्षणोऽभिधेयः प्रधानरसलक्षणं च प्रयोजनसंक्षेपेणोपक्षिप्यते~। यथा वेणीसंहारे भीमः \textendash

\begin{quote}
{\qt लाक्षागृहानलविषान्नगृहप्रवेशैः\\
प्राणेषु वित्तनिचयेषु च नः प्रहत्य~।\\
आकृष्य पाण्डववधूपरिधानकेशान्\\
स्वस्था भवन्तु कुरुराजसुताः सभृत्याः~॥} इति
\end{quote}

\newpage
% ३८ नाट्यशास्त्रम्

\begin{quote}
{\na \renewcommand{\thefootnote}{1}\footnote{भ \textendash\ काव्यस्यार्थ, ड \textendash\ कार्यस्यार्थ}काव्यार्थस्य समुत्पत्तिरुपक्षेप इति स्मृतः~॥~६९}
\end{quote}

\hrule

\vspace{2mm}
२ \underline{परिकरः} \textendash\ तत ईषद् विस्तार्यते (परिकरः)~। यथा भीमः\textendash

\begin{quote}
{\qt प्रवृद्धं यद्वैरं मम खलु शिशोरेव कुरुभि \textendash\ \\
र्न तत्रार्यो हेतुर्न भवति किरीटी न च युवाम्~।\\
जरासन्धस्योरस्तलमिव विरूढं पुनरपि\\
क्रुधा भीमः सन्धिं विघटयति यूयं घटयत~॥} (अ १ \textendash\ १०)
\end{quote}

३ \underline{परिन्यासः} \textendash\ ततोऽपि निश्चयापत्तिरूपतया परितो हृदये सोऽर्थोन्यस्यते (परिन्यासः)~।

\begin{quote}
यथा \textendash\ {\qt चञ्चद्भुजाभ्रमितचण्डगदाभिघात\textendash \\
संचूर्णितोरुयुगलस्य सुयोधनस्य~।\\
स्त्यानावनद्धघनशोणितशोणपाणि\textendash \\
रुत्तंसयिष्यति कचान्स्तव देवि भीमः~॥} (१ \textendash\ २१) इत्यादि~।
\end{quote}

४ विलोभनम् \textendash\ ततस्तदेव गुणवदिति श्लाघ्यते, श्लाघैव विलोभनहेतुत्वा\underline{द्विलोभनम्}~। यथा \textendash\ द्रौपदी \textendash\ अणुगह्णन्तु मए एदं वअणं देवदाओ (अनुगृह्णन्तु मे एतद्वचनं देवताः) इत्यादि~। यथा वा विक्रमोर्वश्यां \textendash

\begin{quote}
{\qt अस्याः सर्गविधौ प्रजापतिरभूच्चन्द्रो नु कान्तिप्रदः\\
श्रृङ्गारैकरसः स्वयं नु मदनो मासो नु पुष्पाकरः~।\\
वेदाभ्यासजडः कथं नु विषयव्यावृत्तकौतूहलो\\
निर्मातुं प्रभवेन्मनोहरमिदं रूपं पुराणो मुनिः~॥} इत्यादि~।
\end{quote}

\noindent
तदेतदुपक्षेपाद्यङ्गचतुष्कं प्रायशो मुखसन्धौ भवति~। उक्तेनैव न पौर्वापर्येण भवति~। आनन्तर्यनियमस्तु नास्ति, न सन्ध्यन्तराणां सामादीनां मध्येऽनुप्रवेशात्~। तदेतदाह मुनिः \underline{काव्यार्थस्य समुत्पत्ति}रित्यादिना \underline{विलोभनमिति स्मृतमि}त्यन्तेन तत्र वृत्तान्तेनोपक्षयः सर्वेषां प्रयोजनमित्युक्तम्~। परिकरस्य प्रयोजनमिष्टार्थस्य रचनापि~।

\newpage
% एकोनविंशोऽध्यायः ३९

\begin{quote}
{\na \renewcommand{\thefootnote}{1}\footnote{भ \textendash\ समुत्पन्न, ढ \textendash\ यदल्पं नार्थ}यदुत्पन्नार्थबाहुल्यं\renewcommand{\thefootnote}{2}\footnote{भ \textendash\ बाहुल्ये} ज्ञेयः परिकरस्तु सः~।\\
\renewcommand{\thefootnote}{3}\footnote{ड \textendash\ तन्निष्पत्त्या तु कथनं परिन्यासः प्रकीर्तितः , च \textendash\ तन्निर्वृत्तिः ट \textendash\ तन्निष्पत्तिस्तु कथनं परिन्यासः प्रचक्ष्यते}तन्निष्पत्तिः परिन्यासो विज्ञेयः कविभिः सदा~॥

गुणनिर्वर्णनं चैव\renewcommand{\thefootnote}{4}\footnote{ढ \textendash\ निर्वहणं तज्ज्ञैः ड \textendash\ यत्तु ट \textendash\ निर्वर्णना यत्तत्} विलोभनमिति स्मृतम्~।\\
संप्रधारणमर्थानां युक्तिरित्यभिधीयते\renewcommand{\thefootnote}{5}\footnote{भ \textendash\ संज्ञितम्}~॥~७१

\renewcommand{\thefootnote}{6}\footnote{भ \textendash\ मुख्यार्थस्योप, प \textendash\ सुखार्थस्योप,}सुखार्थस्याभिगमनं प्राप्तिरित्यभिसंज्ञिता\renewcommand{\thefootnote}{7}\footnote{न \textendash\ धीयते}~।\\
\renewcommand{\thefootnote}{8}\footnote{भ \textendash\ बीजस्यागमनं यत्तु तत्समाधानमिष्यते ढ \textendash\ बीजार्थस्याभिः , फ \textendash\ बीजार्थागमनं यत्तु तत् \ldots. उच्यते}बीजार्थस्योपगमनं समाधानमिति स्मृतम्~॥~७२}
\end{quote}

\hrule

\vspace{2mm}
५ युक्तिर्यथा \textendash\ {\qt सहदेवः \textendash\ आर्य, किंचन महाराजसन्देशोऽयं आर्येणव्युत्पन्न एवं गृहीतः} इत्यतः प्रभृति यावद्भीमवचनम् \textendash

\begin{quote}
{\qt युष्मान् ह्वेपयते क्रोधाल्लोके शत्रुकुलक्षयः~।\\
न लज्जयति दाराणां सभायां केशकर्षणम्~॥} (१ \textendash\ १७) इति~।
\end{quote}

\noindent
अस्याः प्रयोजनं प्रकाश्यप्काशनमपि~।\\

\underline{६ प्राप्तिः} \textendash\ \underline{सुखार्थस्याभिगमनं प्राप्ति}रिति~। सुखयतीति सुखं तादृशस्य वस्तुनः~। यथा (वेण्याम्) \textendash\ {\qt एष खलु भगवान् वासुदेवः पाण्डवपक्षपातामर्षितेन सुयोधनेन संयमितुमारव्धः} इत्यादि कुमारमविलम्बितं द्रष्टुमिच्छामीति~। {\qt अयं ह्यर्थो भीमस्य चेतः सुखय} तीति सन्धेर्विघटनात् (प्राप्तिः)~।\\

७ \underline{समाधानम्} \textendash\ \underline{बीजार्थस्योपगपनामिति}~। यस्मिन् बीजं तदिदानीं

\newpage
% ४० नाट्यशास्त्रम्

\begin{quote}
{\na \renewcommand{\thefootnote}{1}\footnote{भ \textendash\ सुखतो दुःखतो योऽर्थस्तद्विधानमिहोच्यते}सुखदुःखकृतो योऽर्थस्तद्विधानमिति स्मृतम्~।\\
\renewcommand{\thefootnote}{2}\footnote{ड \textendash\ कौतूहलेत्तरो वेधो भवेत्तु}कुतूहलोत्तरावेगो\renewcommand{\thefootnote}{3}\footnote{भ \textendash\ वेध्यो, न \textendash\ वेधो, ट \textendash\ वेदो, च \textendash\ वेध्या} विज्ञेया \renewcommand{\thefootnote}{4}\footnote{प \textendash\ प्रोक्ता तु न \textendash\ भवेत्तु}परिभावना~॥~७३}
\end{quote}

\hrule

\vspace{2mm}
\noindent
प्रधाननायकानुगतत्वेन सम्यगाहितं भवतीति (समाधानम् )~। {\qt यौधिष्ठिर} \textendash\ मित्यनेन\renewcommand{\thefootnote}{*}\footnote{\begin{quote}
{\qt यत्सत्यव्रतभङ्गभीरुमनसा यत्नेन मन्दीकृतं\\
यद्विस्मर्तुमपीहितं शमवता शान्तिं कुलस्येचच्छता~।\\
तद् द्यूतारणिसंभृतं नृपशुना केशाम्बराकर्षणैः\\
क्रोधज्योतिरिदं महत् कुरुवने यौधिष्ठिरं जृम्भते~॥} (१ \textendash\ २५)
\end{quote}} समाधानं दर्शितम्~।\\

८ \underline{विधानम्} \textendash\ \underline{सुखदुःखकृतो योऽर्थस्तद्विधानमिति~।} व्यामिश्रतया सुखदुःखे अभिधीयेते यत्रेति (विधानम्)~। यथा \textendash \\

भीमः \textendash\ तत्पाञ्चालि गच्छामो वयमिदानीं कुरुकुलक्षयाय~।\\

द्रौपदी \textendash\ णाह जं असुरसमराहिमुहस्स हरिणो मङ्गलं तं तुंहाण होदु(नाथ, यदसुरसमराभिमुखस्य हरेर्मङ्गलं तत्तव भवतु) इत्यादि (अ १) तथा{\qt मा अनवेक्खिदसरीरा संचरह, अप्पमत्त संचरिणिज्जाइं रिपुबलाइं(मा अनपेक्षितशरीराः संचरथ, अप्रमत्तसंचरणीयानि रिपुबलानि)} इति~। अत्र द्रौपद्याः प्रहर्षो भयं च मिश्रतया विहितमिति विचित्रत्वाद् रसवत्ता भवति~। तेनेष्टस्यार्यस्य रचना, तथा निगूह्यस्य नायिकाचित्तनिस्त्रिंशभावस्य निगूहनं प्रयोजनम्~। एवमन्यत्रापि प्रयोजनमुत्प्रेक्ष्यम्~। युक्तिवच्चेदमन्यत्रापि संभवत्येवेत्येवमन्यत्राप्यूह्यम्~।\\

९ परिभावना \textendash\ कुतूहलेति कौतुकेन जिज्ञासातिशयेन व्यामिश्रो य आवेगः सा परिभावना किमेतदिति~। यथा \textendash\ संग्रामं संघटनया संशयमाना द्रौपदी तूर्यशब्दं श्रुत्वाह \textendash\ {\qt णाह किं दाणि एसो पळअंतजळहरत्थणिदमंसळो}

\newpage
% एकोनविंशोऽध्यायः ४१

\begin{quote}
{\na \renewcommand{\thefootnote}{1}\footnote{भ \textendash\ उद्भेदस्तद्विनिष्पत्तिर्विज्ञेयो द्विजसत्तमाः (ब \textendash\ उद्भेदस्य विनि)}बीजार्थस्य प्ररोहो यः स उद्भेद इति स्मृतः\renewcommand{\thefootnote}{2}\footnote{प \textendash\ उद्भेदः स तु कीर्तितः}~।\\
\renewcommand{\thefootnote}{3}\footnote{भ \textendash\ अर्थानुस्मरणं चैव प \textendash\ प्रत्यक्षार्थ}प्रकृतार्थसमारम्भः करणं नाम तद्भेवेत्\renewcommand{\thefootnote}{4}\footnote{ड \textendash\ समारम्भं करणं परिचक्षते}~॥~७४

\renewcommand{\thefootnote}{5}\footnote{च \textendash\ संपात भ \textendash\ उत्साहजननं भेदो विज्ञेयस्तु प्रयोक्तृभिः प \textendash\ संघातरूपभेदो यः ट \textendash\ संभूत}संघातभेदनार्थो यः स भेद इति कीर्तितः\renewcommand{\thefootnote}{6}\footnote{ड \textendash\ संज्ञितः}~।}
\end{quote}

\hrule

\vspace{2mm}
\noindent
खणे खणे समरदुन्दुभी ताडीअदि \textendash\ (नाथ किमिदानीमेष प्रलयान्तजलधरस्तनितमांसलो क्षणे क्षणे समरदुन्दुभिस्ताड्यते) इति\\

१० \underline{उद्भेदः} \textendash\ \underline{बीजार्थस्य प्ररोह} इति~। यथा \textendash\ द्रौपदी \textendash\ हा णाह पुणो वि तुए अहं समस्ससइदव्वा (हा नाथ, पुनरपि त्वयाहं समाश्वासयितव्या)\\

\begin{quote}
भीमः \textendash\ {\qt भूयः परिभवक्लान्तिलज्जाबन्धुरिताननम्~।\\
अनिश्शेषितकौरव्यं न पश्यसि वृकोदरम्~॥} इति~।
\end{quote}

\noindent
न चेदमुद्घाटनं येन प्रतिमुखं भवेत्, अपि तु शत्रुक्षयारम्भं बीजस्याङ्कुरः कुरुकुलोद्घाटनेन विनापि प्ररोहमात्रमनुस्थानानुगुण्यात् भूमिसंश्लोष इव बीजस्य~।\\

११ \underline{करणम्} \textendash\ \underline{प्रकृतार्थसमारम्भः करणमिति}\renewcommand{\thefootnote}{*}\footnote{अन्ये तु विपदां शमनं करणमाहुः}~। यथा \textendash \\

सहदेवः \textendash\ गच्छामो वयमिदानीं कुरुराजानुज्ञाताः विक्रमानुरूपमाचरितुम्, इत्यादि (वेणी \textendash\ १)~।\\

१२ भेदः \textendash\ \underline{संघातभेदनार्थो यः स भेद} इति~। पात्रसंघातस्य यन्निजप्रयोजनोपक्षेपेण निष्क्रमणसिद्धये भेदनं प्रकरणमिव स भेदः~। सर्वत्राङ्केऽन्तर्भावी वस्तूपायात्मा भेदः, स सन्ध्यन्तरैकविंशतौ वक्ष्यते~। अस्यो \textendash

\lfoot{6}

\newpage
\lfoot{}
% ४२ नाट्यशास्त्रम्

\begin{quote}
{\na [ एतानि तु मुखाङ्गानि वक्ष्ये प्रतिमुखे पुनः ]~॥

1समीहा रतिभोगार्था विलासा इति संज्ञितः\renewcommand{\thefootnote}{2}\footnote{ड \textendash\ कीर्तितः}~।}
\end{quote}

\hrule

\vspace{2mm}
\noindent
दाहरणं \textendash\ (वेण्यां \textendash\ अ. १ \textendash\ भीमवाक्यम्) {\qt आन्योन्यास्फालभिन्न} इत्यादि यावत् {\qt पाण्डुपुत्राः}\renewcommand{\thefootnote}{*}\footnote{\begin{quote}
{\qt अन्योन्यास्फालभिन्नद्विपरुधिरवसासान्द्रमस्तिष्कपङ्के\\
मग्नानां स्यन्दनानामुपरिकृतपदन्यासिक्रान्तपत्तौ~।\\
स्फीतासृक्पानगोष्ठीरसदशिवशिवातूर्यनृत्यत्कबन्धे\\
संग्रामैकार्णवान्तःपथसि विचरितुं पण्डिताः पाण्डुपुत्राः~॥}
\end{quote}} इति~।\\

अथ प्रतिमुखोद्दिष्टानामङ्गानामुद्देशक्रमेण लक्षणमाह\\

१३ \underline{विलासः} \textendash\ \underline{समीहा रतिभोगार्था विलास} इति~। रतिलक्षणस्य भावस्य हेतुभूतो यो भोगो विषयः प्रमदा पुरुषो वा तदर्था या समीहा स विलासः~। कामफलेषु रूपकेषु प्रतिमुख एव ह्यास्थाफलत्वेन रतिरूपेण भाव्यम्~। यथाभिज्ञानशाकुन्तले \textendash \\

तापसः \textendash\ कस्येदमुशीरानुलेपनमित्यादि~। तथा राजा \textendash \\

\begin{quote}
{\qt कामं प्रिया न सुलभा मनस्तु तद्भावदर्शनाश्वासि~।\\
अकृतार्थेऽपि मनसिजे रतिमुभयप्रार्थना कुरुते~॥} इत्यादि~।
\end{quote}

\noindent
यस्तु वेणीसंहारे भानुमत्या सह दुर्योधनस्य दर्शितो विलासः, स नायकस्य तादृशेऽवसरेऽत्यनुचित इति चिरन्तनैरेवोक्तम्~। यथा सहृदयालोककारः\textendash

\begin{quote}
{\qt सन्धिसन्ध्यङ्गघटनं रसबन्धव्यपेक्षया~।\\
न तु केवलशास्त्रार्थस्थितिसंपादनेच्छया~॥} (ध्वन्या ३)
\end{quote}

\noindent
एतच्च विवरण एवास्माभिर्वितत्य दर्शितम्~।\\

इह च रतिग्रहणं पुमर्थोपयोगि रसगतस्थाथिभावोपलक्षणं तेन वीरप्रधानेषु रूपकेषु प्रतिमुख एव ह्यास्था रतिरूपेण उत्साहः~। सम्यग्विषया

\newpage
% एकोनविंशोऽध्यायः ४३

\begin{quote}
{\na दृष्टनष्टानुसरणं परिसर्प इति स्मृतः\renewcommand{\thefootnote}{1}\footnote{स वर्ण्यते, प \textendash\ श्चकथ्यते}~॥~७६\\

\renewcommand{\thefootnote}{2}\footnote{भ \textendash\ विधूतमरतिं प्राहुस्तथा च द्विजसत्तमाः ट \textendash\ क्रुद्धस्य न \textendash\ कृतस्य विनयस्य}कृतस्यानुनयस्यादौ विधूतं विधूतं ह्यपरिग्रहः \\
\renewcommand{\thefootnote}{3}\footnote{भ \textendash\ विलापवचनं म \textendash\ तस्यापनयनं यत्र शमनं}अपायदर्शनं यत्तु तापनं\renewcommand{\thefootnote}{4}\footnote{ट शमनं} नाम तद्भवेत्~॥~७७}
\end{quote}

\hrule

\vspace{2mm}
\noindent
समीहा चेष्टा विलास इति मन्तव्यम्~। युक्तचैतन्य एव हि रसो मुख उपक्षिप्तः~। तस्यैव स्वयं प्रतिमुखस्वोचितारम्भसंभावितः कर्तव्यः~। लस श्लेषणेऽपि हि पठ्यते~।\\

१४ \underline{परिसर्पः} \textendash\ \underline{दृष्टनष्टानुसरणं परिसर्प} इति~। यथा (वेण्यां) कञ्चुकी\textendash\ {\qt आशस्त्रग्रहणादकुण्ठपरशोः}\renewcommand{\thefootnote}{*}\footnote{वेणीसंहारे (अ २ \textendash\ २)} इत्यादि दृष्टनष्टप्रायो हि कार्यान्तरव्यासङ्गात्~। कुरुकुलक्षयो भीष्मवधेन स्थानपरितोषसूचितेन च दुर्योधनस्यायुक्तचेष्टितत्वेनानुसृत इति प्रकृतस्यार्थस्य परिसर्पणात् प्रसरणात् परिसर्पः~। यथा चाभिज्ञानशाकुन्तले भवितव्यमात्रतया~। तथा हि \textendash

\begin{quote}
{\qt अभ्युन्नता पुरस्तादवगाढा जघनगौरवात्पश्चात्~।\\
द्वारेऽस्य पाण्डुसिकते पदपङ्क्तिर्दृश्यते हि नवा~॥} (३ \textendash\ ५) इति~।
\end{quote}

१५ \underline{विधूतम्} \textendash\ \underline{\renewcommand{\thefootnote}{$\oint$}\footnote{विधूतमरतिं प्राहुः केचित्~। तत्रारतिरित्यभीष्टानवाप्तितो दुःखम्}कृतस्यानुनयस्येति}~। आदौ प्रथमतः, कृतस्यानुनयस्य सामवचसो नाङ्गीकरणं विधूतं, पश्चात् पुनरङ्गीकरणमिति~। आदिशब्दात् (उपरोधः), यथा \textendash\ तत्रैव \textendash\ शकुन्तला \textendash\ अइ किं अंतेउरविरहपय्युसिएण राएसिणा अवरुद्धेण \textendash\ इत्यादि~।\\

१६ \underline{तापनम्} \textendash\ \underline{\renewcommand{\thefootnote}{$\dagger$}\footnote{केचित्तु तापनस्थाने शमनं पठन्ति, अरतेः शमनमथवानुनयग्रहणादरतेर्निग्रहः शमनम्~।}अपायदर्शनं यत्तु तापन}मिति~। यथा रत्नावल्याम् \textendash

\newpage
% ४४ नाट्यशास्त्रम्

\begin{quote}
{\na \renewcommand{\thefootnote}{1}\footnote{थ \textendash\ क्रीडाविलोभनार्थं तु न \textendash\ क्रीडाविनोदनार्थं तु भ \textendash\ हास्यप्रायं तु यद्वाक्यं तन्नर्म परिकीर्तितम्}क्रीडार्थं विहितं यत्तु हास्यं नर्मेति \renewcommand{\thefootnote}{2}\footnote{च \textendash\ संज्ञितम् न \textendash\ कीर्तितम्}तत्स्मृतम्~।\\
\renewcommand{\thefootnote}{3}\footnote{भ \textendash\ रतिर्नर्मकृता चैव द्युतिरित्यभिसंज्ञिता}दोषप्रच्छादनार्थं तु हास्यं नर्मद्युतिः स्मृता\renewcommand{\thefootnote}{4}\footnote{न \textendash\ नर्मद्युति स्मृतम्}~॥~७८}
\end{quote}

\hrule

\begin{quote}
{\qt दुल्लहजणोणुराओ लज्जागुरुई परवसो अप्पा~।\\
पिअसहि विसमं पेम्मं मरणं सरणं णु वरमेक्कम्~॥}\renewcommand{\thefootnote}{*}\footnote{\begin{quote}
{\qt दुर्लभजनानुरागः लज्जा गुर्वी परवश आत्मा~।\\
प्रियसखि विषमं प्रेम मरणं शरणं नु वरमेकम्~॥}
\end{quote}} (२ \textendash\ ७)
\end{quote}

\noindent
\ldots १७ नर्म \textendash\ क्रीडार्थं विहितं यत्तु हास्यं नर्मेति~। यथा (रत्नावल्यां द्वितीयेऽङ्के) विदूषकः \textendash \\

भो मा पांडिच्चगव्वं उव्वह, अहं एदाआ मुहादो सुणिअ वक्खाणइस्सं \textendash\ (भो मा पाण्डित्यगर्वमुद्वह~। अहं एतस्या मुखात् श्रुत्वा व्याख्यास्यामि) इत्यादि~।\\

१८ \underline{नर्मद्युतिः} \textendash\ \underline{दोषप्रच्छादनार्थं तु हास्यं नर्मद्युति}रिति~। दोषो येनोक्तेन प्रच्छादयितुमिष्यते तस्यापि हास्यजननत्वेन नर्म च सुतरां द्योतितं भवतीति नर्मद्युतिः~। यथा च (रत्नावल्यां द्वितीयेऽङ्के विदूषकः) \textendash\ चउव्वेई विअ बम्हणो रिअइम पट्ठिदुं पवुत्ता~। (चतुर्वेदी ब्राह्मण इव ऋचः पठितुं प्रवृत्ता~। इत्याभिहिते

\begin{center}
राजा \textendash\ नावधारितं मया\\
ततो विदूषकः \textendash\ दुल्लहजणाणुराओ\renewcommand{\thefootnote}{$\dagger$}\footnote{पूर्णा गाथा छाया चास्मिन् पृष्ठे दत्ता.}
\end{center}

\noindent
इति पठति~। अत्र हि मौर्ख्यदोषं छादयितुं यद्विदूषकेणोच्यते तद्राज्ञो हास्यजननमिति नर्मैव द्योतितं भवति~। तथा हि राजा \textendash\ महावब्राह्मण कोऽन्य एवमृचामभिज्ञः \textendash\ इति~।

\newpage
% एकोनविंशोऽध्यायः ४५

\begin{quote}
{\na \renewcommand{\thefootnote}{1}\footnote{च \textendash\ अधरोत्तर भ \textendash\ विज्ञेयं तु प्रशमनं विषादशमनोद्भवम्}उत्तरोत्तरवाक्यं तु भवेत्प्रगयणं\renewcommand{\thefootnote}{2}\footnote{प \textendash\ प्रगमणं ड \textendash\ प्रगमनं बुधाः} पुनः~।\\
या तु व्यसनसंप्राप्तिः स निरोधः प्रकीर्तितः\renewcommand{\thefootnote}{3}\footnote{भ \textendash\ विरोध इति संस्मृतः ढ \textendash\ विरोधः स तु संज्ञितः प \textendash\ मुखानां संनिवेशो यः स निरोध इति स्मृतः}~॥~७९

क्रुद्धस्यानुनयो \renewcommand{\thefootnote}{4}\footnote{भ \textendash\ यश्च तद्भवेत्}यस्तु भवेत्तत्पर्युपासनम्~।}
\end{quote}

\hrule

\vspace{2mm}
१९ \underline{प्रगयणमु} \textendash\ \underline{उत्तरोत्तरवाक्यन्तु भवेत्प्रगयण}मिति~। (यथा \textendash\ रत्नावल्यां द्वितीयेऽङ्के) विदूषकः \textendash\ किं णु खु दाणिं गाहेयम् (किं नु खलु इदानीं गाथेयम्)\\

राजा \textendash\ कयापि श्लाघ्यनवयौवनया प्रियतममनासादयन्त्या जीवितनिरपेक्षयेदमुक्तम्~। विदूषकः \textendash\ भो किं एदे हि णं\ldots (भोः किमेतैः न) इत्यादि~। प्रगयणमिति रूढिशब्दः~।\\

अन्ये तु प्रजाशब्दात् विचि क्लिष्ययत्नशब्देन शता क्विना व्युत्पप्तिं कल्पयन्ति~। प्रागयणमित्यन्ये पठन्ति \textendash\ प्रागिति पूर्ववचनं ततोऽयनं प्राप्तिः यस्योत्तरवचनस्येति~।\renewcommand{\thefootnote}{*}\footnote{अन्ये तु प्रगमनमिति प्रशमनमिति च पठन्ति}\\

२० \underline{निरोधः} \textendash\ \underline{या तु व्यसनसंप्राप्तिः स निरोध}\renewcommand{\thefootnote}{$\dagger$}\footnote{केचिद्विरोध इति अन्ये रोध इति च पठन्ति} इति~। (यथारत्नावल्यां द्वितीयेऽङ्के) राजा {\qt उच्चैर्हसता त्वयेयं त्रासिता} (इति), व्यसनमत्र खेदमात्रमभीष्टोपरोधान्निरोधः~।\\

२१ \underline{पर्युपासनम्} \textendash\ \underline{क्रुद्धस्यानुनयो यस्त्विति}~। यथा \textendash\ (तत्रैव) विदूषकः\textendash\ भो मा कुप्प एसा खु कदलीघरन्ते\ldots एहि\renewcommand{\thefootnote}{$\ddagger$}\footnote{भो मा कुप्य, एषा खलु कदलीगृहान्तरे (वर्तते), एहि} इत्यादि~। राजा अनुनीतः सन्नाह \textendash

\newpage
% ४६ नाट्यशास्त्रम्

\begin{quote}
{\na विशेषवचनं यत्तु तत्पुष्पमिति संज्ञितम्~॥~८०

\renewcommand{\thefootnote}{1}\footnote{च \textendash\ प्रत्यक्षरूपं भ \textendash\ रूक्षप्रायं तु}प्रत्यक्षरूक्षं यद्वाक्यं वज्रं तदभिधीयते~।\\
\renewcommand{\thefootnote}{2}\footnote{भ \textendash\ सोपायवचनं यत्तु स उपन्यास उच्यते}उपपत्तिकृतो योऽर्थ उपन्यासश्च\renewcommand{\thefootnote}{3}\footnote{न \textendash\ श्च संस्मृतः} स स्मृतः~॥~८१}
\end{quote}

\hrule

\begin{quote}
{\qt दुर्वारां कुसुमशरव्यथां वहन्त्या\\
कामिन्या यदभिहितं पुरः सखीनाम्~।\\
तद्भूयः शुकशिशुसारिकाभिरुक्तं\\
धन्यानां श्रवणपथातिथित्वमेति~॥} इत्यादि~।
\end{quote}

२२ \underline{पुष्पम्} \textendash\ \underline{विशेषवचनं यत्तु पुष्पमि}ति~। यथा (तत्रैव विदूषकः)\textendash\ एसो को वि चित्तफळहओ (एष कोऽपि चित्रफलकः) \textendash\ इत्यादि विदूषकोक्तेः प्रभृति यावत् {\qt परिच्युतस्तं कुचकुम्भमध्यात्} इत्यादि\renewcommand{\thefootnote}{*}\footnote{\begin{quote}
{\qt परिच्युतस्तत्कुचकुम्भमध्यात् किं शोषमायासि मृणालहार~।\\
न सूक्ष्मतन्तोरपि तावकस्य तत्रावकाशो भवतः कि मु स्यात्~॥}
\end{quote}}~। यथा हि प्रेमविकासि पुष्पं भवत्येवमत्रापि राज्ञ उत्तरोत्तरानुरागविशेषसूचकं वचो विकासमस्यानुरागस्य दर्शयति~। तथा हि सुसङ्गता \textendash\ सहि गरुआणुगागविक्खित्तहिअओ असंबद्धं भट्टा मन्तेदुं पवुत्तो (सखि गुर्वनुरागविक्षिप्तहृदयोऽसंबद्धं भर्ता मन्त्रितुं प्रवृत्तः) इत्यादि~।\\

२३ \underline{वज्रम्} \textendash\ \underline{प्रत्यक्षरूक्षं यद्वाक्यं वज्र}मिति~। यथा (तत्रैव) \textendash\ {\qt कथमिहस्थोऽहं भवत्या ज्ञात} इति राजन्युक्तवति सुसङ्गता \textendash\ ण केवलं तुमं, चित्त फलहेण~। ता जाव गदुअ देवीए णिवेदेमि~। (न केवलं त्वं, चित्रफलकेण~। तद्यावद्गत्वा देव्यै निवेदयामि)~।\\

२४ \underline{उपन्यासः} \textendash\ \underline{उपपत्तिकृतो योऽर्थ उपन्यास} इति~। यथा (तत्रैव) विदूषकः (ससाध्वसं) \textendash\ अदिमुहरा खु एसा गब्भदासी (अतिमुखरा खल्वेषा गर्भदासी)~। अत्र मौखर्यात्मिकोपपत्तिरुपन्यस्ता~।\renewcommand{\thefootnote}{$\dagger$}\footnote{केचिदुपन्यासः प्रसादनमित्याहुः~। भोजेन तूपन्यासाङ्गं परिहृतम्~।}

\newpage
% एकोनविंशोऽध्यायः ४७

\begin{quote}
{\na \renewcommand{\thefootnote}{1}\footnote{भ \textendash\ चतुर्वर्णामिगमनं प \textendash\ वर्णितार्थतिरस्कारो प \textendash\ उच्यते}चातुर्वर्ण्योपगमनं वर्णसंहार इष्यते\renewcommand{\thefootnote}{2}\footnote{भ \textendash\ इतः परं \textendash\ {\qt एतानि तु प्रतिमुखे गर्भे चापि निबोधत} इत्यर्धमधिकं वर्तते}~।\\
\renewcommand{\thefootnote}{3}\footnote{य \textendash\ अभूताहरणं तत्स्याद् वाक्यं यत्कपटाश्रयम् भ \textendash\ कपटाय तु यद्वाक्यमभूताहरणं तु तत्}कपटापाश्रयं वाक्यमभूताहरणं विदुः~॥~८२

\renewcommand{\thefootnote}{4}\footnote{च \textendash\ सत्त्वार्थ}तत्त्वार्थवचनं चैव मार्ग इत्यभिधीयते~।}
\end{quote}

\hrule

\vspace{2mm}
२५ \underline{वर्णसंहारः} \textendash\ \underline{\renewcommand{\thefootnote}{*}\footnote{वर्णितार्थतिरस्कारो वर्णसंहार इति पाठे उक्तार्थस्य विषयान्तरप्रसक्त्या प्रच्छादनम्~।}चातुर्वर्ण्योपगमनं वर्णसंहार} इति~। चातुर्वर्ण्यशब्देन पात्राण्युपलक्ष्यन्ते~। तेन यत्र पात्राणि पृथक् स्थितान्यपि ढौक्यन्ते स वर्णसंहारः~। उपाध्यायास्त्वाहुः \textendash\ इह वीरप्रधाने तावन्नायकप्रतिनायकौ तत्सचिवौ च प्रधानत्वेन वर्ण्यन्त इति वर्णाः , कामप्रधानेऽपि नायको नायिका तत्सचिवौ चेति~। तथा हि रत्नावल्यां (द्वितीयेऽङ्के) सुसङ्गताया वचनात् {\qt अदी मे अअं गरुओ पसाओ (अतो ममायं गुरुः प्रसादः)} इत्यारभ्य, राजा \textendash\ क्वासौ~। सुसङ्गता \textendash\ हत्थे गेह्वअ सरिं पसाएहि णं ( हस्ते गृहीत्वा सखीं प्रसादयैनाम् ) \textendash\ इत्यादि~। अत्र चतुर्णामेकीभावः प्रयोगस्य, इष्टस्य रचना, प्रकाश्ये प्रकाशनमित्यपि प्रयोजनानि~। यत्तु ब्राह्मणादिवर्णचतुष्टयमेलनमिति तदफलत्वादनादृत्यमेव~।\\

अथ गर्भाङ्गान्युद्देशक्रमेण लक्षयति~।\\

२६ \underline{अभूताहरणम्} \textendash\ \underline{कपटापाश्रयं वाक्यमभूताहरण}मिति~। यथा वासवदत्तया चित्रफलके दृष्टे विदूषकवचनं \textendash\ अप्पा किल दुक्खेण आलिहिदुत्ति मम वअणं सुणिअ पिअवयस्सेण विण्णाणं दंसिअं (आत्मा किलदुःखेनालिखितुमिति मम वचनं श्रुत्वा प्रियवयस्येन विज्ञानं दर्शितम्) \textendash\ इत्यादि~।\\

२७ \underline{मार्गः} \textendash\ \underline{तत्त्वार्थवचनं मार्ग} इति~। (तत्रैव) {\qt भट्टिणि कदा वि घुणक्खरं वि संभावीयदि (भर्त्रि कदापि घुणाक्षरमपि संभाव्यते)} इति

\newpage
% ४८ नाट्यशास्त्रम्

\begin{quote}
{\na \renewcommand{\thefootnote}{1}\footnote{भ \textendash\ चित्रार्थे वाक्यसंयोगे रूपकं तु विनिर्दिशेत् न \textendash\ चिन्तार्थ, म \textendash\ चिन्त्यार्थ, ढ \textendash\ चित्रार्थसमवायो यस्तद्रूपमिति कीर्तितम्}चित्रार्थसमवाये तु वितर्को रूपमिष्यते~॥~८३

\renewcommand{\thefootnote}{2}\footnote{य \textendash\ यत्र सातिशयं वाक्यमुदाहरणमिष्यते भ \textendash\ यत्तु सातिशयं वाक्यं तदाहरणभिष्यते}यत्सातिशयवद्वाक्यं तदुदाहरणं स्मृतम्~।}
\end{quote}

\hrule

\vspace{2mm}
\noindent
काञ्चनमालयोक्ते वासवदत्ता समयानुसारि परमार्थोचितं वचनमाह \textendash\ अइ उज्जुए वसन्दओ खु एसो (अयि ऋजुके वसन्तकः खल्वसौ) \textendash\ इत्यादिमार्गवच्च प्रसिद्धत्वात् परमार्थे मार्ग इति व्यपदेशः~।\\

२८ \underline{रूपम्} \textendash\ \underline{\renewcommand{\thefootnote}{*}\footnote{चित्रार्थो वाक्यसंयोगो रूपकमिति पाठे रूपकं संशयस्य तर्केण च्छेदनमिति केचित्~। अन्ये तु चित्रार्थमेव वचो रूपकमिति मन्यन्ते~।}चित्रार्थसमवाये तु वितर्को रूप}मिति~। यथा (रत्नावल्यां द्वितीयेऽङ्के राजा \textendash\ प्रसीदेति ब्रूयामिदमसति कोपे न घटते\renewcommand{\thefootnote}{$\dagger$}\footnote{\begin{quote}
{\qt करिष्याम्येवं नो पुनरिति भवेदभ्युपगमः~।\\
न मे दोषोऽस्तीति त्वमिदमपि च ज्ञास्यसि मृषा\\
किमेतस्मिन् वक्तुं सममिति न वेद्मि प्रियतमे~॥}
\end{quote}}~। इत्यादि विचित्रार्थानां समवाये संभावने सर्वविषय एव विरुद्धस्तर्कः , इदं नोचितमिदं नोचितमिति प्रतियुक्तिपर्यन्तः~। युक्तिस्तु नियतप्रतिपत्तिपर्यन्तेति विशेषः , रूपमिति चानियता आकृतिरुच्यते~। तत्र विशेषप्रतिपत्तिरिहापि तथोपचाराद् व्यपदेशः~।\\

२९ \underline{उदाहरणम्} \textendash\ \underline{यत्सातिशयवद्वाक्यं तदुदाहरण}मिति~। लोकप्रसिद्धवस्त्वपेक्षया यत् सातिशयमुच्यते उत्कर्षमाहरतीत्युदाहरणम्~। यथा (तत्रैव तृतीयेऽङ्के) \textendash

\begin{quote}
{\qt मनः प्रकृत्यैव चलं दुर्लक्ष्यं च तथापि मे~।\\
कामेनैतत्कथं विद्धं समं सर्वैः शिलीमुखैः~॥} इति~।
\end{quote}

\newpage
% एकोनविंशोऽध्यायः ४९

\begin{quote}
{\na \renewcommand{\thefootnote}{1}\footnote{भ \textendash\ तत्त्वोपलब्धिर्वाक्यस्य ढ \textendash\ तत्त्वोपपत्तिर्भावस्व}भावतत्त्वोपलब्धिस्तु क्रम इत्यभिधीयते~॥~८४

\renewcommand{\thefootnote}{2}\footnote{भ \textendash\ युक्तस्तु सामदानाभ्यां विज्ञेयः संग्रहो बुधैः च \textendash\ सामदानार्थ}सामदानादिसंपन्नः \renewcommand{\thefootnote}{3}\footnote{ड \textendash\ संयोगः संग्रहः स तु कीर्तितः प \textendash\ संयुक्तः}संग्रहः परिकीर्तितः~।\\
\renewcommand{\thefootnote}{4}\footnote{भ \textendash\ रूपं तु गमनं लिङ्गादनुमान इति स्मृतः}रूपानुरूपगमनमनुमानमिति स्मृतम्~॥}
\end{quote}

\hrule

\vspace{2mm}
\noindent
तथा च \textendash

\begin{quote}
{\qt वाणाः पञ्च मनोभवस्य नियतास्तेषामसंख्यो जनः\\
प्रायोऽस्मद्विध एव लक्ष्य इति यल्लोके प्रसिद्धिं गतम्~।\\
दृष्टं तत्त्वयि विप्रतीपमधुना यस्मादसंख्यैरयं\\
विद्धः कामिजनः शरैरशरणो नीतस्त्वया पञ्चताम्~॥}
\end{quote}

\noindent
इत्यादि~।\\

३० \underline{क्रमः} \textendash\ \underline{भावतत्त्वोपलब्धिस्तु क्रम} इति~। भावस्य भाव्यमानस्य वस्तुनो भावनातिशये सत्यूहं प्रति भावनादिबलात् स्यात् या परमार्थोपलब्धिः सा क्रमः~। बुद्धिर्हि तत्र क्रमते न प्रतिहन्यते~। यथा (तत्रैव) \textendash

\begin{quote}
{\qt ह्रिया सर्वस्यासौ हरति विदितास्मीति वदनं\\
द्वयोर्दृष्ट्वालापं कलयति कथामात्मविषयाम्~।\\
सखीषु स्मेरासु प्रकटयति वैलक्ष्यमधिकं\\
प्रिया प्रायेणास्ते हृदयनिहितातङ्कविधुरम्~॥} इत्यादि~।
\end{quote}

३१ \underline{संग्रहः} \textendash\ \underline{सामदानादिसंपन्नः सङ्ग्रह} इति~। साम्ना सङ्केतादिवार्ताः श्रुत्वा (राज्ञा विदूषकाय) कटकस्य दानम्~। एवमन्यदपि~।\\

३२ \underline{अनुमानम्} \textendash\ \underline{रूपानुरूपगपन}मिति~। रूप्यमानेन प्रत्यक्षाद्युपलभ्य \textendash\ मानेन रूपस्य व्यापकस्याविनाभाविनो गमनं ज्ञानमनुमानं निश्चयात्मकत्वादूहः, उपायायुक्तेरन्यत्वात्~। यथा (तत्रैव) \textendash

\lfoot{7}

\newpage
\lfoot{}
% ५० नाट्यशास्त्रम्

\begin{quote}
{\na \renewcommand{\thefootnote}{1}\footnote{म \textendash\ कार्यानुनयपूर्वस्तु नियोगः प \textendash\ अभ्यर्थनापरं वाक्यं प्रार्थनेत्यमिधीयते भ \textendash\ मातृकायां प्रार्थनालक्षणं तोपलभ्यते ट \textendash\ अतिहर्षोत्सवार्थानां}रतिहर्षोत्सवानां तु प्रार्थना प्रार्थना भवेत्~।\\
गर्भस्योद्भेदनं यत्साक्षिप्तिरित्यभिधीयते~॥\renewcommand{\thefootnote}{2}\footnote{भ \textendash\ यत्तु तमाक्षेपं विदुर्बुधाः ढ \textendash\ उच्छेदनं यत्तु तदाक्षिप्तमिति स्मृतम्, म \textendash\ उद्भेदनं यत्तु तदुपक्षिप्तमिष्यते}~८६}
\end{quote}

\hrule

\begin{quote}
{\qt पालीयं चम्पकानां नियतमयमसौ सुन्दरः सिन्दुवारः\\
सान्द्रा वीयी तथेयं वकुलविटपिनां पाटलापङ्क्तिरेषा~।\\
आघ्रायाघ्राय गन्धं विविधमधिगतैः पादपैरेवमस्मिन्\\
व्यक्तिं पन्थाः प्रयाति द्विगुणतरतमोनिह्नुतोऽप्येष चिह्नैः~॥}
\end{quote}

\noindent
इत्यादि~। अत्र ह्याघ्रायाघ्राय गन्धमिति गन्धानि(त्) कुसुमानि तेभ्यः पादपाः, तेभ्योऽपि मार्गमनुमापितमिति राज्ञा विदूषकस्योक्तेः~।\\

३३ \underline{प्रार्थनां} \textendash\ \underline{रतिहर्षोत्सवानां तु प्रार्थना प्रार्थनेति}~। एतत् साध्यफलोचितभावलक्षणं, तत्र साध्यफले यः प्राधान्येन समुचितो भावस्तद्विषया या प्रकर्षेणाभ्यर्थना सा प्रार्थनाख्यमङ्गम्~। यथा (तत्रैव) \textendash\ संकेतस्थः प्रतिपालयन् राजा \textendash

\begin{quote}
{\qt तीव्रः स्मरसन्तापो न तदादौ वाधते यथासन्ने~।\\
तपति प्रावृषि नि तरामभ्यर्णजलागमो दिवसः~॥} इति~।
\end{quote}

३४ \underline{आक्षिप्तिः} \textendash\ \underline{गर्भस्योद्भेदनाक्षिप्ति}रिति~। हृदयान्तःस्थितं (तस्य) पुनः प्रतिष्ठापितस्यापि यतः कुतश्चिन्निमित्तादुद्भेदनमनपह्नवनीया या स्फुटतापत्तिः सा आक्षिप्तिः, अभिप्रायस्य हि तत्राक्षेपो बहिः कर्षणं~। वासवदत्तायामेव सागरिकेति राज्ञा विदूषकेण च परिगृहीतायां तदुक्तिषु {\qt सागरिके, \renewcommand{\thefootnote}{*}\footnote{\begin{quote}
{\qt शीतांशुर्मुखमुत्पले तव दशौ पद्मानुकारौ करौ\\
रम्भागर्भनिभं तवोरुयुगलं बाहू मृणालोपमौ~।\\
इत्याह्रादकराखिलाङ्गि रभसान्निः शङ्कमालिङ्ग मा\textendash \\
मङ्गानि त्वमनङ्गतापविधुराण्येह्येहि निर्वापय~॥}
\end{quote}}शीतांशुर्मुखमुत्पले तव दृशौ} इत्यादिषु~।

\newpage
% एकोनविंशोऽध्यायः ५१

\begin{quote}
{\na संरम्भवचनं \renewcommand{\thefootnote}{1}\footnote{ढ \textendash\ (वचन) प्रायं तोटकं त्विह, भ \textendash\ यच्च तोटकं नाम तद्भवेत्}चैव तोटकं त्विति संज्ञितम्~।\\
\renewcommand{\thefootnote}{2}\footnote{भ \textendash\ कपटप्रवृत्तो योऽर्थो विज्ञेयोऽधिबलो हि सः ड \textendash\ कपटेनाभिसन्धानं ज्ञेयं त्वधिबलं बुधैः (न \textendash\ त्वति. ढ \textendash\ चातिबलं) प \textendash\ कपटस्यान्यथाभावं य \textendash\ अनुमानार्थसंयुक्तं विद्यादतिबलं तथा}कपटेनातिसन्धानं ब्रुवतेऽधिबलं बुधाः~॥~८७

भयं \renewcommand{\thefootnote}{3}\footnote{भ \textendash\ नृपारिसंयुक्तमुद्वेग इति कीर्त्यते ट \textendash\ नृपादि, प \textendash\ नृपारिजनितं}नृपारिदस्यूत्थमुद्वेगः परिकीर्तितः~।}
\end{quote}

\hrule

\vspace{2mm}
३५ \underline{तोटकम्} \textendash\ \underline{संरम्भवचनं चैव तोटक}मिति~। आवेगगर्भं यद्वचनं तत्तोटकम्~। स चावेगो हर्षात्, क्रोधात्, अन्यतोऽपि वा~। भिनत्ति यतो हृदयं ततस्तोटकम्~। यथा (तत्रैव) विदूषकः \textendash\ अज्ज वि दाव से देवीए णिच्चरुठ्ठाए वासवदत्ताए वअणेहि कडुइदे कण्णे सुहावीअदु (अद्यापि तावत्तस्या देव्या नित्यरुष्टाया वासवदत्ताया वचनैः कटूकृते कर्णे सुखय) इत्यादि~।\\

३६ \underline{अधिबलम्} \textendash\ \underline{कपटेनातिसन्धानमधिबल}मिति\renewcommand{\thefootnote}{1}\footnote{अतिबलमिति}~। परस्परवचनप्रवृत्तयोर्यस्यैवाधिकं (कर्म) सहायबुद्ध्यादीनबलम्बयति स एव तमतिसन्धातुं वञ्चयितुं समर्थ इति तदिदं कर्माधिबलम्2~। यथा \textendash\ सागरिकावेषं धारयन्ती वासवदत्ता विदूषकबुद्धिदौर्बल्याद्राजानमतिसंधत्ते {\qt किं पद्मस्य रुचिं न हन्ति}\renewcommand{\thefootnote}{$\dagger$}\footnote{गर्भसन्धिलक्षणेऽयं श्लोकः उदाहृतः~।} इत्यादि श्लोकान्तमधिबलम्\renewcommand{\thefootnote}{2}\footnote{अतिबलम्}~।\\

३७ \underline{उद्वेगः} \textendash\ \underline{भयं नृपारिदस्यूत्थमुद्वेग} इति~। \renewcommand{\thefootnote}{3}\footnote{आदि}अरिशब्दान्नाचिकादि~। यथा (तत्रैव) राजा \textendash\ कथं देवी वासवदत्ता, वयस्य किमेतत्~। विदूषकः \textendash\ णं अंहाणं जीविअसंशओ (ननु अस्माकं जीवितसंशयः) \textendash\ इत्यादि~।

\newpage
% ५२ नाट्यशास्त्रम्

\begin{quote}
{\na \renewcommand{\thefootnote}{1}\footnote{भ \textendash\ नृपारिभयसंयुक्तः संभ्रमस्त्वभिसंज्ञित ड \textendash\ नृपाग्निभयसंयुक्तः संभ्रमो विद्रवः स्मृतः}शङ्का भयत्रासकृतो विद्रवः सुमुदाहृतः~॥~८८

\renewcommand{\thefootnote}{2}\footnote{इतः पूर्वं भढपादिषु \textendash\ {\qt एतान्यङ्गानि गर्भे तु वक्ष्येऽवमर्शने पनः (ढ \textendash\ क्ष्याम्यवमृशे)} इत्यर्धमुपलभ्यते, न \textendash\ {\qt गर्भाङ्गलक्षणं प्रोक्तं विमर्शे च निबोधत } इति संदृश्यते भ \textendash\ विशेषवचनं यत्तु ट \textendash\ दोषप्रस्थापनं यत्स्यात्}दोषप्रख्यापनं यत्तु सोऽपवाद इति स्मृतः\renewcommand{\thefootnote}{3}\footnote{न \textendash\ यत् स्यादपवादस्तु स स्मृतः}~।}
\end{quote}

\hrule

\vspace{2mm}
३८ \underline{विद्रवः} \textendash\ \underline{शङ्का भुगत्रासकृतो विद्रव} इति~। भयत्रासकारिणो वस्तुनो या शङ्का यदाशङ्कनं स विद्रवः, विद्रवति विलीयते हृदयं येनेति~। यथा (तत्रैव) \textendash

\begin{quote}
{\qt \renewcommand{\thefootnote}{*}\footnote{\begin{quote}
पूर्वार्धं \textendash\ {\qt समारुढा प्रीतिः प्रणयबहुमानादनुदिनं\\
व्यलोकं वीक्ष्येदं कृतमकृतपूर्वं खलु मया~।}
\end{quote}}प्रिया मुञ्चत्यद्य ध्रुवमसहना जीवितमसौ\\
प्रकृष्टस्य प्रेम्णः स्खलितमविषह्यं हि भवति~॥} इति~।
\end{quote}

\noindent
अन्ये तु शङ्काभयत्रासैः कृतो यः स विद्रव इति~। तत्र च विशेष्यपदमन्वेष्यम्, समुदाय एव विशेष्य इति श्रीशङ्कुकः उदाहरति च कृत्यारावणे षष्ठेऽङ्के गर्भसन्धौ, (नेपथ्ये) (मण्डोदरी) \textendash\ हा अय्यउत्त परित्ताआहि परित्ताआहि (हा आर्यपुत्र परित्रायस्व परित्रायस्व)~। प्रतीहारी (श्रुत्वा आत्मगतं)\textendash\ अंहो भट्टिणी विअ आक्खंददि~। (अंहो भर्त्रीवाक्रन्दति) (प्रकाशं) भट्टा भवदो अन्तेउरे महन्दो कलकळो सुणीअदि~। (भर्तः भवतोऽन्तःपुरे महान् कलकलः श्रूयते)

\begin{quote}
राजा \textendash\ ज्ञायतां किमेतदिति~।\\
अत्र रावणस्याशङ्का प्रतिहार्यास्त्रासभये~।\\
अथावमर्शसन्धावङ्गानां लक्षणमाह \textendash
\end{quote}

३९ \underline{अपवादः} \textendash\ \underline{दोषप्रख्यापनं यत्तु सोऽपवाद} इति~। यथा (तत्रैव) सागरिकोक्तेरनन्तरं\renewcommand{\thefootnote}{$\dagger$}\footnote{सागरिका \textendash\ अय्यउत्त किं अळीअदक्खिणदाए जीविदादो वि वळळह \textendash\ दाराए देवीए अत्ताणअं अवराहिणं करोसि (आर्यपुत्र, किमलीकदक्षिणतया जीविताद्वल्लभाया देव्या आत्मानमपराधिनं करोषि)~।} राजा \textendash\ अयि मिथ्यावादिनी खल्वसि \textendash

\newpage
% एकोनविंशोऽध्यायः ५३

\begin{quote}
{\na \renewcommand{\thefootnote}{1}\footnote{ट \textendash\ दोष}रोषग्रथितवाक्यं तु संफेटः\renewcommand{\thefootnote}{2}\footnote{प \textendash\ संस्फोट इति} परिकीर्तितः~॥~८९

\renewcommand{\thefootnote}{3}\footnote{भ \textendash\ ताडनं वधबन्धो वा विद्रवः; समुदाहृतः भ \textendash\ द्रवस्तत्रावबोद्धव्यो गुरूणां व्यतिक्रमः}गुरुव्यतिक्रमो यस्तु\renewcommand{\thefootnote}{4}\footnote{च \textendash\ विज्ञेयोऽ \textendash\ भिदवस्तु स (ड \textendash\ विद्रवः)} स द्रवः परिकीर्तितः~।\\
\renewcommand{\thefootnote}{5}\footnote{म \textendash\ विरोध ड \textendash\ विरोधोपगमो यस्तु न \textendash\ निरोधशमयं युक्तिस्तर्जनाधर्षणं द्युतिः}विरोधिप्रशमो यश्च सा शक्तिः परिकीर्तिता~॥~९०}
\end{quote}

\hrule

\begin{quote}
{\qt श्वासोत्कम्पिनि कम्पितं स्तनपुगे मौने प्रियं भाषितं\\
वक्त्रेऽस्याः कुटिलीकृतभ्रुणि रुषा यातं मया पादयोः~।\\
इत्थं नः सहजाभिजात्यजानिता सेवैव देव्याः परं\\
प्रेमावद्धविवर्धिताधिकरसा प्रीतिस्तु या सा त्वयि~॥} इति
\end{quote}

\noindent
अत्र देवीगुणानां सातिशयकोपनत्वेनापवदनं कृतम्~।\\

४० \underline{संफेटः} \textendash\ \underline{रोषग्रथितवाक्यन्तु संफेट} इति~। केचित्तु स्फोट अनादर इति धातुं मनस्कृत्य संस्फोट इति पठन्ति~। यथा (तत्रैव) \textendash\ वासवदत्ता (सरोषं सहसोपसृत्य) अय्यउत्त, जुत्तं\ldots\ldots सरिसं (आर्यपुत्र, युक्तं, सदृशम् ) \ldots इत्यादि~।\\

४१ \underline{द्रव} \textendash\ \underline{गुरुव्यतिक्रमो यस्तु स द्रव} इति~। यथा (तत्रैव) \textendash\ भर्तृसंनिधानेऽपि विदूषकस्य सागरिकायाश्च वासवदत्तया बन्धनम्~। यथा वा \textendash\ तापसवत्सराजे षष्ठेऽङ्के वासवदत्ताया यौगन्धरायणवचनातिक्रमेण मरणाध्यवसायः~। द्रवणं चलनं मार्गादिति द्रवः~।\\

४२ \underline{शक्तिः} \textendash\ \underline{विरोधिप्रशमः शक्ति}रिति~। विरोधिनः कुपितस्य प्रशमः प्रसादनं शक्तिः बुद्धिविभवादिशक्तिकार्यत्वात्~। यथा (तत्रैव) \textendash

\begin{quote}
{\qt सव्याजैः शपथैः प्रियेण वचसा चित्तानुवृत्त्या भृशं\\
वैलक्ष्येण परेण पादपतनैर्वाक्यैः सखीनां मुहुः~।\\
प्रत्यापत्तिमुपागता मम तथा देवी रुदत्या तथा\\
प्रक्षाल्यैव तथैव बाष्पसलिलैः कोपोऽपनीतः स्वयम्~॥} इत्यादि~।
\end{quote}

\newpage
% ५४ नाट्यशास्त्रम्

\begin{quote}
{\na व्यवसायश्च\renewcommand{\thefootnote}{1}\footnote{ड \textendash\ तु} विज्ञेयः प्रतिज्ञाहेतुसंभवः\renewcommand{\thefootnote}{2}\footnote{ड \textendash\ दोषसंभवः, य \textendash\ संश्रयः}~।\\
प्रसङ्गश्चैव विज्ञेयो गुरूणां परिकीर्तनम्\renewcommand{\thefootnote}{3}\footnote{ट \textendash\ नित्यं परिभवात्मकः न \textendash\ वाक्योमर्षप्रयोजितः भ \textendash\ गुणागुणविवृद्धिस्तु प्रसङ्ग इति कीर्तितः प \textendash\ अप्रस्तुतार्थवचनं प्रसङ्गः परिकीर्तितः}~॥~९१

वाक्यमाधर्षसंयुक्तं\renewcommand{\thefootnote}{4}\footnote{ड \textendash\ घर्षणयुतं} द्युतिस्तज्ज्ञैरुदाहृता~।\\
मनश्चेष्टाविनिष्पन्नः\renewcommand{\thefootnote}{5}\footnote{च \textendash\ समुत्पन्नः} श्रमः खेद उदाहृतः~॥~९२}
\end{quote}

\hrule

\vspace{2mm}
४३ \underline{व्यवसायः} \textendash\ \underline{व्यवसायश्च विज्ञेयः प्रतिज्ञाहेतुसंभव} इति~। प्रतिज्ञातस्याङ्गीकृतस्यार्थस्य हेतुनो ये तेषां संभवः प्राप्तिर्व्यवसायः~। यथा (तत्रैव) ऐन्द्रजालिकप्रवेशादितो यावत् {\qt एको उण खेडओ अवस्सं पेक्खितव्वो} इति तावत् यौगन्धरायणेन यत्कर्तुमङ्गीकृतं तस्यैव हेतुः (तस्य) प्राप्तिः~।\\

४४ \underline{प्रसङ्गः} \textendash\ \underline{प्रसङ्गश्चापि (शैव ?) विज्ञेयो गुरुणां परिकीर्त}नमिति~। यथा (तत्रैव) वासवदत्ताः उज्जयणीदो आअदोत्ति अत्थि मे तस्सि इन्दआळिए पक्खवादो (उज्जयिन्या आगत इति अस्ति भे तस्मिन्निन्द्रजालिके पक्षपातेः) \textendash\ इत्यादि~। अत्र हि बन्धुकुलादागमोऽस्य बहुमानकारणम्~।\\

४५ \underline{द्युतिः} \textendash\ \underline{वाक्यमाधर्षसंयुक्तं द्युति}रिति~। आधर्षो न्यक्कारः तेन संयुक्तम्~। यथा विदूषकः \textendash\ हा दासीए उत्त इन्दआलिअ (आः दास्याः पृत्र इन्द्रजालिक) \textendash\ इत्यादि~।\\

४६ \underline{खेदः} \textendash\ \underline{मुदश्चेष्टाविनिष्पन्नः श्रमः खेद} इति मानसः कायीयश्चेत्युभयोऽपि यावत्~। आद्यो यथा \textendash\ सिंहलेश्वरस्य कुशलप्रश्ने यथा वसुभूतिर्निश्चस्य {\qt देव न जाते किं कथयामि} इत्यत आरभ्य रत्नावल्याः समुद्रपतनाकर्णनोदितवासवदत्ताविलापपर्यन्तम्~। शारीरस्तु खेदः (विक्रमोर्वश्याम्) पुरूरवसा {\qt अहो श्रान्तोऽस्मि यावतस्या गिरिनद्यास्तीर} इत्यादि~।\renewcommand{\thefootnote}{*}\footnote{एकं पुनः खेलनमवश्यं प्रेक्षितव्यम्}

\newpage
% एकोनविंशोऽध्यायः ५५

\begin{quote}
{\na ईप्सितार्थप्रतीघातः प्रतिषेधः प्रकीर्तितः\renewcommand{\thefootnote}{1}\footnote{ड \textendash\ निषेधः स तु कीर्तितः}~।\\
\renewcommand{\thefootnote}{2}\footnote{भ \textendash\ उत्तरोत्तरवाक्यं तु विरोध इति संज्ञितः ड \textendash\ विरोधनं तु संरम्भादुत्तरोत्तरभाषणम् , नय \textendash\ उत्तरोत्तरवाक्यं च}कार्यात्ययोपगमनं विरोधनमिति स्मृतम्~॥~९३

बीजकार्योपगमन\renewcommand{\thefootnote}{3}\footnote{भ \textendash\ नयनं य \textendash\ शमनं}मातानमिति संज्ञितम्~।\\
\renewcommand{\thefootnote}{4}\footnote{भ \textendash\ अवमानार्थजनिता छलना परिकीर्तिता, ड \textendash\ अवमानात् य \textendash\ अवमानादिजनितः समूहः छन्दनं भवेत् (स मोहः छलनं) ट \textendash\ अवमानादिभणनं}अपमानकृतं वाक्यं कार्यार्थं च च्छादनं भवेत्~॥~९४}
\end{quote}

\hrule

\vspace{2mm}
यद्यपि श्रमोद्वेगवितर्कलज्जाप्रभृतयो व्यभिचारिवर्गे पूर्वमृक्तास्तथाप्येते सत्यवसरेऽवश्यप्रयोज्याः प्रागुक्तप्रयोजनार्थसिद्धये, ते पृथक्प्रयोजनत्वात् सन्ध्यङ्गत्वेनोक्ता मन्तव्याः~।\\

४७ \underline{प्रतिषेधः} \textendash\ \underline{ईप्सितार्थप्रतीयातः प्रतिषेध} इति~। यथा रत्नावलीवृत्तान्तवर्णने ईप्सितार्थप्रतीघाते बाभ्रव्येण प्रस्तुते तस्य प्रतिघातोऽन्तःपुरदाहेन~।\\

४८ \underline{निरोधनम्} \textendash\ \underline{कार्यात्ययोपगमनं निरोधनमि}ति~। यथा राजा {\qt कथमन्तःपुरेऽग्निः~। हा हा धिक्कष्टं दग्धा देवी वासवदत्ता} इत्यादि यावत् सागरिकोत्सादनपर्यन्तम्~। अत्र हि कार्ये वासवदत्ता सागरिकाप्रेमविस्रम्भस्यात्ययो विनाशमुपगतः प्राप्तः~।\\

४९ \underline{आदानम्} \textendash\ \underline{ बीजकार्योपगमनमादान}मिति बीजफलस्य समीपताभवनमित्यर्थः~। यथा सागरिका राजन दृष्ट्वा (स्वगतं) {\qt अय्यउत्त} इत्यादि, अत्र हि बन्धुकुलादागमो यावद्राज्ञ उक्तिः \textendash

\begin{quote}
{\qt व्यक्तं लग्नोऽपि भवतीं न धक्ष्यति दुताशनः~।\\
यतः सन्तापमेवायं स्पर्शस्ते हरति प्रिये~॥}
\end{quote}

\noindent
इत्यन्तम्~।\\

५० \underline{छादनम्} \textendash\ \underline{अपमानकृतं वाक्यं छादन}मिति~। वाक्यमिति तदर्थो लक्ष्यते~। करोतिः बहुमाने वर्तने, तेन दुष्टोऽप्यर्थोऽपमानेन बहुमतीकृतः~।

\newpage
% ५६ नाट्यशास्त्रम्

\begin{quote}
{\na प्ररोचना च विज्ञेया संहारार्थप्रदशिनी\renewcommand{\thefootnote}{1}\footnote{भ \textendash\ या कार्यार्थप्रदर्शिनी, भ \textendash\ सत्कारस्य विदर्शिका ड \textendash\ प्रकाशिनी}~।\\
( प्रत्यक्षवचनं यत्तु स व्याहार इति स्मृतः~॥~९५

सविच्छेदं वचो यत्र सा युक्तिरिति संज्ञिता~।\\
ज्ञेया विचलना तज्ज्ञैरवमानार्थसंयुता )~॥~९६

\renewcommand{\thefootnote}{2}\footnote{भ \textendash\ एतान्यङ्गान्यवमृशे ट \textendash\ विमर्श एतान्यङ्गानि ढ \textendash\ विमर्शऽङ्गानि चोक्तानि}( एतान्यवमृशेऽङ्गानि संहारे तु निबोधत)~।}
\end{quote}

\hrule

\vspace{2mm}
\noindent
तदपमानकलङ्कापवारणाच्छादनमिति~। यथा सागरिका \textendash\ दिठ्ठिआ पज्जळिदो भअवं हुदासणो, अज्ज करइस्सदि मे सअळदुक्खावसाणम्~। (दिष्ट्या प्रज्वलितो भगवान् हुताशनः , अद्य करिष्यति मे सकलदुःखाव\textendash\ 
सानम्~। ) इति~।\\

५१ \underline{प्ररोचना} \textendash\ \underline{प्ररोचना च विज्ञेया संहारार्थप्रदर्शिनी} इति~। संह्रिय \textendash\ माणस्य निर्वाह्यमाणस्यार्थस्य दर्शिका प्रकर्षेण रोचत इति प्ररोचना~। यथा \textendash

\begin{quote}
{\qt क्वासौ ज्वलन् हुतवहस्तदवस्थमेत\\
दन्तः पुरं कथमवन्तिनृपात्मजेयम्~।\\
वाभ्रव्य एष वसुभूतिरयं वयस्यः\\
स्वमो मतिभ्रममिति\renewcommand{\thefootnote}{*}\footnote{स्वप्नो मतिभ्रम इदं नु किमिन्द्रजालम् \textendash\ इति चतुर्थः पादः~। स्वप्ने मतिर्भ्रमति किं त्विदमिन्द्रजालम् \textendash\ इति पाठान्तरम्}}
\end{quote}

\noindent
युक्तिरित्यन्ये इदमङ्गं व्यवहरन्ति~। अत्रोद्देशक्रमत्यागे यत्केषांचिदङ्गानां लक्षणं तत्क्रमानियमसूचनार्थः~। अनेन पाठविपर्यासेन यत्कैश्चिदुद्देशस्यान्यथापठनं तद्ग्रन्थकाराशयापरिज्ञानकृतम्~। केचिदत्रान्यतममङ्गं नाधीयते, द्वाद \textendash\ शाङ्गमेवैतत्सन्धिमाहुः~। अन्ये तु त्रयोदशाङ्गत्वेऽप्यस्य निर्वहणसन्धावापि प्रसक्तेरितिवृत्तान्तर्भूतत्वेन गणनमन्याय्यमिति त्रयोदशाङ्गत्वात् चतुःषष्टिसंख्यां समर्थयन्ते~।

\newpage
% एकोनविंशोऽध्यायः ५७

\begin{quote}
{\na मुखबीजोपगमनं\renewcommand{\thefootnote}{1}\footnote{भ \textendash\ नयनं} सन्धिरित्यभिधीयते\renewcommand{\thefootnote}{2}\footnote{भ \textendash\ संज्ञितः}~॥~९७

\renewcommand{\thefootnote}{3}\footnote{भ \textendash\ अन्वेषणं तु कार्याणां निरोध \textendash\ समुदाहृतः}कार्यस्यान्वेषणं युक्त्या\renewcommand{\thefootnote}{4}\footnote{न \textendash\ यत्र ट \textendash\ वत्तु निरोध इति संज्ञितः} निरोध\renewcommand{\thefootnote}{5}\footnote{ड \textendash\ विरोध} इति कीर्तितः~।\\
\renewcommand{\thefootnote}{6}\footnote{ट \textendash\ अप}उपक्षेपस्तु कार्याणां \renewcommand{\thefootnote}{7}\footnote{भ \textendash\ प्रसवं नाम तद्भवेत्}ग्रथनं परिकीर्तितम्~॥~९८

\renewcommand{\thefootnote}{8}\footnote{ड \textendash\ अनुभूतस्य, भ \textendash\ अनुभाव्यस्तथा वोऽर्था निर्णयः सोऽभिधीयते}अनुभूतार्थकथनं निर्णयः समुदाहृतः~।}
\end{quote}

\hrule

\vspace{2mm}
अथ निर्वहणसन्धावुद्देशक्रमेणाङ्गानि लक्षायितुं प्रक्रमते\\

५२ \underline{सन्धिः} \textendash\ \underline{मुखबीजोपगमनं सन्धि}ति~। यथा वसुभूतिः \textendash\ वाभ्रव्य, सदृशीयं राजपुत्र्याः \textendash\ इत्यादि मुखे यदुक्तं तदिह निकटीभूतं सन्धानं सन्धिः~।\\

५३ \underline{निरोधः} \textendash\ \underline{कार्यस्यान्वेषणं युक्त्या निरोध} इति यथा वसुभूतिः\textendash\ कुत इयं कन्यकेत्यादि~।\\

५४ \underline{ग्रथनम्} \textendash\ \underline{उपक्षेपस्तु कार्याणां ग्रथनमि}ति यथा (यौगधरायणः\textendash\ देव क्षम्यतां यन्मयाऽनिवेद्य कृतम् \textendash\ इत्यादि~। अत्र रत्नावलीलाभरूपकार्यस्योपेक्षपाद् ग्रथनम्~।)\\

५५ \underline{निर्णयः} \textendash\ \underline{अनुभूतार्थकथनं निर्णय} इति~। प्रमाणसिद्धस्य वस्तुनः कथनमित्यर्थः~। यथा रत्नावल्यां चतुर्थेऽङ्के वसुभूतिः \textendash\ अपि रत्नावली, ननु त्वमीदृशीमवस्थां प्राप्तासि~।\\

सागरीका \textendash\ ( सप्रत्यभिज्ञं ) तुमं पि किं अमच्चवसुभूदी

वसु \textendash\ स एवाहं मन्दभाग्यः इति\textendash 

प्रभृति यावद् विदूषकवाक्यं {\qt सविहवो होदु } इति~।

\lfoot{8}

\newpage
\lfoot{}
% ५८ नाट्यशास्त्रम्

\begin{quote}
{\na \renewcommand{\thefootnote}{1}\footnote{ट \textendash\ परिवाहात्मकं}परिवादकृतं \renewcommand{\thefootnote}{2}\footnote{भ \textendash\ यत्तु तद्भवेत्}यत्स्यात्तदाहुः परिभाषणम्~॥~९९

\renewcommand{\thefootnote}{3}\footnote{प \textendash\ ईर्ष्याकोपोपशमनं}लब्धस्यार्थस्य \renewcommand{\thefootnote}{4}\footnote{भ \textendash\ गमनं}शमनं द्यतिमाचक्षते पुनः\renewcommand{\thefootnote}{5}\footnote{ट \textendash\ बुधाः भ \textendash\ कृतिरित्यभिधीयते ड \textendash\ द्युतिरित्यभिधीयते}~।\\
\renewcommand{\thefootnote}{6}\footnote{ड \textendash\ समागमस्तु योऽर्यानामानन्दः र तु कीर्तितः}समागमस्तथार्थानामानन्दः परिकीर्तितः\renewcommand{\thefootnote}{7}\footnote{भ \textendash\ कीर्त्यते}~॥~१००}
\end{quote}

\hrule

\vspace{2mm}
५६ \underline{परिभाषणम्} \textendash\ \underline{परिवादकृतं यत् तत् परिभाषणमि}ति~।\\

यथा सागरिका \textendash\ किदापराहा खु अहं देवीए ता ण सक्खुणोमि मुहं दंसेदुं कृतापराधा खल्वहं देव्या, तत् न शक्नोमि मुखं दर्शयितुम् )~। वासवदत्ता अपवार्य ) \textendash\ अय्यउत्त ळज्जामि खु अहं इमिणा णिसंसत्तणेण ता अवणेहि से बन्धणं~। आर्यपुत्न, लज्जे खल्वहमनेन नृशंसत्वेन, तदपनयास्या बन्धनम्~।\\

एतदुभयोरप्यन्योन्यापराधोद्घट्टनं वचनम् यौगन्धरायणोऽपि प्राविश्यैवश्यैवमेवापराधमुद्घट्टयति तथा \textendash

\begin{quote}
{\qt देव्या मद्वचनाद्यदाभ्युपगतः पत्युविंयोगस्तदा\\
सा चाप्यन्यकलत्रसंघटनया दुःखं मया प्रापिता\renewcommand{\thefootnote}{*}\footnote{\begin{quote}
{\qt तस्योः प्रीतिमयं करिष्यति, जगत्स्वामित्वलाभः प्रभोः\\
सत्यं दर्शयितुं तथापि बदन शक्नोमि नो लज्जया~॥}
\end{quote}}~।} इत्यादि~।
\end{quote}

५७ \underline{द्युतिः} \textendash\ \underline{लब्धस्यार्थस्याशमनं द्युतिरिति}~। सामर्थ्यात्प्रशमनीयस्य क्रोधादेरर्थस्य प्राप्तस्यापि यत्प्रशमनं सा द्युतिः~। (यथा तत्रैव) देव श्रूयतामिदम्~। सिंहलेश्वरदुहिता सिद्धैरादिष्टा इत्यादि यावद्देव्या उक्तिः, अय्य अमच्च फुढं एव्व किं ण भणासि पडिवादेहि रअणावळिंत्ति~। (आर्य अमात्य, स्फुटमेव किं न भणासि प्रतिपादय तस्य रत्नावलीमिति~। )\\

५८ \underline{आनन्दः} \textendash\ \underline{समागमस्तथार्यानामानन्द} इति~। अर्थितस्य तथेति प्रकारशतै प्रार्थितस्य सम्यगपुनर्वियोगवद्यदागमनं तदानन्तहेतुत्वादानन्दः~। यथा (तत्रैव) राजा \textendash\ को देव्याः प्रसादं न बहुमन्यते \textendash\ इत्यादि~।

\newpage
% एकोनविंशोऽध्यायः ५९

\begin{quote}
{\na \renewcommand{\thefootnote}{1}\footnote{भ \textendash\ दुःखापनयनं चैव समयः परिकीर्तितः प \textendash\ दुःखस्योपशमो न \textendash\ दुःखोपशमनं यत्तु}दुःखस्यापगमो यस्तु समयः स निगद्यते~।\\
शुश्रूषाद्युपसंपन्नः प्रसादः प्रीतिरुच्यते\renewcommand{\thefootnote}{2}\footnote{भ \textendash\ शुश्रूषावचनोपेतः प्रसाद इति संज्ञितः च \textendash\ उपसंपन्नः प्रलाद इति भण्यते ट \textendash\ सुप्रसन्नता}~॥~१०१

\renewcommand{\thefootnote}{3}\footnote{य \textendash\ अत्यद्भुतस्य संप्राप्तिर्भवेत्तदुपगूहनम्}अद्भुतस्य \renewcommand{\thefootnote}{4}\footnote{न \textendash\ च}तु संप्राप्तिरूपगूहनमिष्यते\\
\renewcommand{\thefootnote}{5}\footnote{म \textendash\ दानमानविनिष्यन्नमाभाषणमुदाहृतम् , ( भ \textendash\ भाषणं नाम तद्भवेत्}सामदानादि\renewcommand{\thefootnote}{6}\footnote{ड \textendash\ संयुक्तं भाषणं तूच्यते बुधैः}सपन्नं भाषणं समुदाहृतम्\renewcommand{\thefootnote}{7}\footnote{प \textendash\ ब्रुवते बुधाः}~॥~१०२}
\end{quote}

\hrule

\vspace{2mm}
५९ \underline{समयः} \textendash\ \underline{दुःखस्यापगमो यस्तु समय} इति~। अपगमनमपगमः~। यथा \textendash\ अय्यजउत्त दूरे खु एदाए णादिउळं ता तह अणुचिठ्ठ जहा बन्धुजणं ण सुमरेति (वासवदत्ता \textendash\ आर्यपुत्र , दूरे खलु अस्या ज्ञातिकुलं , तत्तथानुततिष्ठ यथा बन्धुजनं न स्मरति )\\

६० \underline{प्रसादः} \textendash\ \underline{शुश्रूषायुपसंपन्नः प्रसाद} इति~। यथा वासवदत्ता\textendash\ {\qt एतिअं दाव मम बहिणिआ अणुरूपं हृोदु} इति स्वैराभरणैरलङ्करोतीति ??~। (एतावता तावन्मेभगिन्युनुरूपं भवतु)~।\\

केचिदद द्युतेरनन्तरमिदमङ्गं पठन्ति~।\\

६१ \underline{उपगूहनम्} \textendash\ \underline{अद्भुतस्य तु संप्राप्तिरुपगूहन}मिति~। यथा विदूषकः \textendash\ ही ही भो कहं कहं संपुण्णमणोरहा संउत्तह्म ( इत्युत्थाय नृत्यति ) ( ही ही भोः कथं कथं संपूर्णमनोरथाः संवृत्ताः स्मः )\\

६२ \underline{भाषणम्} \textendash\ \underline{सामदानादिसंपन्नं भाषणमि}ति~। यद्यपि तदार्थेऽपि संग्रहाख्यमिदमङ्गमुक्तं तथाप्यत्र स्थानेऽवश्यं प्रयोक्तव्यता ख्यापयितुं पुनरुपादानं शब्दान्तरेण च~। यथा वसुभूतिः \textendash\ देवि स्थाने देवीशब्दमुद्वहसि \textendash

\newpage
% ६० नाट्यशास्त्रम्

\begin{quote}
{\na \renewcommand{\thefootnote}{1}\footnote{भ \textendash\ पूर्वभावश्च विज्ञेयः कार्योपक्षेप दर्शकः , प \textendash\ पुनर्वाक्यं\ldots\ldots.यथोक्ताक्षेपदर्शनम्, न \textendash\ पूर्वभावस्तु विज्ञेयः सद्भिः कार्योपदर्शकः ( ट \textendash\ दर्शकः )}पूर्ववाक्यं तु विज्ञेयं यथोक्तार्थप्रदर्शनम्\renewcommand{\thefootnote}{2}\footnote{ड \textendash\ प्रदर्शकम्}~।\\
\renewcommand{\thefootnote}{3}\footnote{च \textendash\ कर}वरप्रदानसंप्राप्तिः काव्यसंहार इष्यते~॥~१०३}
\end{quote}

\hrule

\vspace{2mm}
\noindent
इति~। सामदानं तु यथा भगवती जीमूतवाहनस्य वरं ददाति \textendash\ {\qt त्वां विद्याधरचक्रवर्तिनमहं प्रीत्या करोमि क्षणात्\renewcommand{\thefootnote}{*}\footnote{\begin{quote}
{\qt हंसांसाहतहैमपङ्कजरजः संपर्कपङ्कोक्षितै\textendash \\
रुत्पन्नैर्मम मानसादुपनतैस्तोयैर्महापावनैः~।\\
स्वेच्छानिर्मितरत्नकुम्भनिहितैरेषाभिषिच्य स्वयं\\
त्वां\ldots.~॥ ( नागानन्दे \textendash\ अ ५ )}
\end{quote}}} इत्यादि~।\\

अन्ये मन्यन्ते \textendash\ आदिशब्देन भेददण्डादेरुपायान्तरस्य संग्राह्यत्वं, तस्य वेग ( चेह ? ) स्थाने स्पष्टेन पथानौचित्यात्, गर्भसन्धयुक्तसामाद्यपायानु \textendash\ वदनमात्रमत्र यत्क्रियते इत्थमिदं प्राप्तमित्येवंप्रायं तदिदं भाषणाख्यमङ्गमिति~।\\

६३ \underline{पूर्ववाक्यं} \textendash\ \underline{पूर्ववाक्यं यथोक्तार्थप्रकाशन}मिति~। यथा बाभ्रव्यः\textendash\ इदानीं सफलपरिश्रमोऽस्मि संपन्न \textendash\ इति~।\\

६४ \underline{काव्यसंहारः} \textendash\ \underline{वरप्रदानसंप्राप्तिः काव्यसंहार} इति~। यथा यौगन्धरायणः \textendash\ देव तदुच्यतां किं ते भूयः प्रियमुपहरामीत्यादि यावत्, {\qt यातो विक्रमबाहुरात्मसमतां\renewcommand{\thefootnote}{$\dagger$}\footnote{\begin{quote}
{\qt यातो विक्रमबाहुरात्मसमतां प्राप्तेयमुर्वीतले\\
सारं सागरिका ससागरमहीप्राप्त्येकहेतुः प्रिया~।\\
देवी प्रीतिमुपागता च भगिनीलाभाज्जिताः कोशलाः ;\\
किं नास्ति त्वयि सत्यमात्यवृषभे यस्मिन् करोमि स्पृहाम्~॥}
\end{quote}}} इत्यादि~।

\newpage
% एकोनविंशोऽध्यायः ६१

\begin{quote}
{\na \renewcommand{\thefootnote}{1}\footnote{भ \textendash\ नृपराष्ट्रप्रशान्तिश्च प्रशस्तिरिति संज्ञिता च \textendash\ नृपदेव, प \textendash\ नृप देवादिशान्तिः ड \textendash\ देवद्विजनृपादीनां प्रशस्तिः स्यात्प्रशंसनम्, ढ \textendash\ नृपदोष}नृपदेशप्रशान्तिश्च प्रशस्तिरभिधीयते~।\\
\renewcommand{\thefootnote}{2}\footnote{ड \textendash\ इत्येतानि यथासन्धि कार्याण्यङ्गानि रूपके}यथासन्धि तु कर्तव्यान्येतान्यङ्गानि नाटके~॥~१०४

कविभिः \renewcommand{\thefootnote}{3}\footnote{ड \textendash\ कार्य, भ \textendash\ नाट्यतत्त्वज्ञैः}काव्यकुशलै \renewcommand{\thefootnote}{4}\footnote{भड रसभावान्}रसभावमपेक्ष्य तु~।}
\end{quote}

\hrule

\vspace{2mm}
६४ \underline{प्रशस्तिः} \textendash\ \underline{नृपदेशप्रशान्तिश्च प्रशस्तिरिति}~। ( यथा रत्नावल्यां ) \textendash

\begin{quote}
{\qt उर्वीमुद्दामसस्यां जनयतु विसृजन् वासवो वृष्टिमिष्टां\\
इष्टैस्त्रैविष्टपानां विदधतु विधिवत्प्रीणनं विप्रमुख्याः~।\\
आकल्पान्तं क्रियायाः \renewcommand{\thefootnote}{*}\footnote{कृषीष्ट}क्रमसमुपचितं संगमं सज्जनानां\\
निर्विश्लेषावकाशं पिशुनजनवचोवर्जनाद्वज्रलेपः~॥}
\end{quote}

\underline{यथासन्धि त्विति} यो यस्मिन् सन्धौ योग्य इत्यर्थः~। योग्यतां च कविरेव जानाति, न च मुक्तककविः किन्तु प्रबन्धयोजनासमर्थ \textendash\ ~। तदाह \underline{कविभि}रित्यादि~।\\

ननु कवेः कीदृशं तत्प्रबन्धनिर्माणकौशलमित्याह \underline{रसभावमपेक्ष्येति}, तदपेक्षा च कौशलमित्यर्थः~। रस एव हि प्रीत्या व्युत्पत्तिप्रदं नाट्यात्मकं शास्त्रमित्युक्तम्~। ततश्च यद्यथा यद्यस्यानुपयोगि तदरोचकिनो रुचितदधिशर्करापयः \textendash\ प्रभृतिरसान्तरमध्ययोजितं \textendash\ तद्द्वारेणान्तः प्रविष्टं सत् पुष्टिं व्याधिनिवृत्तिं च विधत्ते , तथैव पुमर्थोपायो हृदयमनुप्रवेष्टुमसमर्थ सुन्दरतदुचितरससङ्क्रमणया प्राप्तान्तःप्रवेशो विनेयजनस्य संपाद्ये वस्तुनि कल्पपादपकल्पनायै कल्पते~। रससंक्रान्तिश्च विभावादिरूपतयैव नान्यथेत्युक्तं षष्ठे~। \underline{एतानीति~।} तान्य \textendash

\newpage
% ६२ नाट्यशास्त्रम्

\begin{quote}
{\na \renewcommand{\thefootnote}{1}\footnote{ड \textendash\ सर्वाङ्गानि}संमिश्राणि कदाचित्तु\renewcommand{\thefootnote}{2}\footnote{भ \textendash\ स्युः सर्वाण्येतानि वा पनु~।} द्वित्रियोगेन वा पुनः~॥~१०५}
\end{quote}

\hrule

\vspace{2mm}
\noindent
ङ्गानि लिखितानि विवक्षितरसभावादिसंपूर्णभावभाञ्जि भवन्ति यानि त्वेकरसावहितमनसो यत्रान्तरनिरपेक्षयैवाहमहमिकया समुचितभावेन बन्धशय्यामनुवर्तन्ते~। इति वृत्ताविच्छेदोऽपि हि रसस्यैव पोषकः, अन्यथा विच्छेदे स्थाय्यादेस्त्रुटितत्वात् क्व रसवार्ता~। तेन रसस्यैवायं विभावादिपरिकरो यदङ्गचक्रमिति~। तथा हि {\qt \renewcommand{\thefootnote}{*}\footnote{वेणीसंहारे}लाक्षागृहानले} त्युपक्षेपो वीररौद्रयोर्विभावांशपूरकः {\qt \footnotemark{*}प्रवृद्धं यद्वरै}मिति क्रोधस्य वीरे व्यभिचारिणो रौद्रे स्थायिनः स्वरूपं प्रत्युज्जीवकः परिकरः, {\qt \footnotemark{*}चञ्चद्भुज} इति च परिन्यासोऽनुभावांशं पुष्णाति, {\qt \footnotemark{*}अणुगह्णन्तु एदं ववसिदं देवदाओ} ( द्रौपदीअनुगृह्णन्त्वेतद्व्यवसितं देवताः ) ( वेण्या \textendash\ अङ्क? ) इत्यादि विलोभनम्~। अतो निवृत्त्योत्सुक्यहर्षमतिस्मृतिप्रभृति व्यभिचारि , स चायं सन्धानधुर्यः एवमन्यदपि योज्यम्~।\\

ननु सन्धिपरतन्त्रैरङ्गैर्भवितव्यम्, तद्रसपारतन्त्र्यमेषां कुतस्त्यम्, उच्यते \textendash\ सन्धयो ह्यवस्थापरतन्त्राः, प्रारम्भाभिधानदशाविशेषोपयोगिकथाखण्डलकं मुखसन्धिरित्युक्तम्, एवमन्यत्र~। अवस्था अप्यन्यकृतिविशेषमनूच्यन्ते~। नन्वतः किम्, इदमतो भवतीत्याह \textendash\ रसभावापेक्षया तु कार्यं स्थितं तस्यापेक्षया अवस्थानं ज्ञात्वेति , कार्यमपि रसप्रवाहजननपर्यन्तत्वेन कृतार्थता संपद्यते इनि यावत्~। संमिश्राणीति सन्ध्यन्तरोक्तं सन्ध्यन्तरेऽपीत्यर्यः~। यथा युक्तिर्मुखेऽप्युक्ता गर्भेऽप्युपनिबद्धा वितर्कव्यभिचार्यंशपोषकभावेन वेणीसंहारे , यथोदाहृतं

प्राक् {\qt तेजस्त्री रिपुहतबन्धुदुःखभार} मित्यादि~। द्वित्रीति द्वित्वत्रित्वयोगेनेत्यर्थः~। तेनैकमपि सन्ध्यङ्गं तत्रैव सन्धौ द्विस्त्रिर्वा कर्तव्यम्~। यथारत्नावल्यां प्रतिमुखे विलासः सागरिकायां राज्ञि वाऽसकृदुपनिबद्धः प्रधानं शृङ्गारं समुद्दीपयति~। वेणीसंहारे संफेडविद्रवौ पुनः प्रदर्शितो वीररौद्रोद्दीपगौ

\newpage
% एकोनविंशोऽध्यायः ६३

\begin{quote}
{\na \renewcommand{\thefootnote}{1}\footnote{न \textendash\ कार्यं कालमवरथां च ज्ञात्वा कार्याणि सन्धिषु ( भ \textendash\ काव्यं )}ज्ञात्वा कार्यमवस्थां च कार्याण्यङ्गानि सन्धिषु~।\\
\renewcommand{\thefootnote}{2}\footnote{इदमर्धं चमय \textendash\ मातृकास्येवोपलभ्यते}एतेषामेव चाङ्गानां संबद्धान्यर्थयुक्तितः~॥~१०६

सन्ध्यन्तराणि\renewcommand{\thefootnote}{3}\footnote{एकविंशति सन्ध्यन्तरणि केषुचिदा \textendash\ दर्शेषु चतुःषष्ट्यङ्गोद्देशग्रन्थपूर्वमेव पठितानि} \renewcommand{\thefootnote}{4}\footnote{नमय \textendash\ वक्ष्यामि त्वर्थोपक्षेपकाणि च}सन्धीनां विशेषास्त्वेकविंशतिः~।\\
साम भेदस्तथा दण्डः\renewcommand{\thefootnote}{5}\footnote{न \textendash\ धीः} प्रदानं वध एव च~॥~१०७

प्रत्युत्पन्नमतित्वं च गोत्रस्खलितमेव च~।\\
साहसं च भयं चैव ह्रीर्माया क्रोध एव च~॥~१०८}
\end{quote}

\hrule

\vspace{2mm}
\noindent
भवतः~। अतिशयेन तु पौनःपुन्ये वैरस्यं स्यादिति द्वित्रिग्रहणम्~। तथा द्वयोर्योगो द्वाभ्यामङ्गाभ्यां संपाद्यं तदेकेनैव चेद्घटते तत्किमपरेण~। एवं त्रियोगः~। द्वियोगो यथा प्रतिमानिरुद्धे भीमसूनोर्वसुनागस्य कृते \textendash\ उपक्षेपानन्तरमेवं न परिकरः , आद्येनैव कृते परिन्यासदर्शनम्~। एवं त्रियोगः, यथा भेज्जलविरचिते राधाविप्रलम्भे रासकाङ्के उपक्षेपेणैवहि {\qt लिअलीस्सा} \renewcommand{\thefootnote}{*}\footnote{राधाप्रलम्बे\ldots..{\qt विहवस्सि}}इत्यादि परिकरपरिन्यासकार्यगुरुभूते पालिते एकोद्देशेन (?) विलोभननिरूपणे~। एवं चतुरङ्गो यातु सन्धिर्भवतीति~।\\

अथ सन्ध्यन्तराणि दर्शयितुमाह {\qt एतेषामेव चाङ्गानामित्यादि}~। तत्र केचिदाहुः \textendash\ अन्तरं छिद्रं सन्धिरिति~। तदङ्गमात्रं \textendash\ तात्स्थ्याच्च तत्स्थान्यं तेन सन्ध्यङ्गच्छिद्रवर्तित्वात् सन्ध्यन्तराणि , अत एव चाङ्गानां संबद्धानि~। ननु किं शेषमात्रेण, नेत्याह, किं त्वर्थस्य प्रयोजनस्य योगेन, अत एव सन्ध्यङ्गानां विशेषकाः, तदर्थविशेषसंबद्धं हि तदङ्गं भवति (इति)~।\\

अन्ये मन्यन्ते \textendash\ य एवोपक्षेपाद्या सामान्या उक्ताः तेषामेवैतद्विशेषा अवान्तरभेदाः~। उपक्षेपो हि सामादिविशेषभिन्नः, तथा हि {\qt लाक्षागृहा \textendash }

\newpage
% ६४ नाट्यशास्त्रम्

\begin{quote}
{\na ओजः संवरणं भ्रान्तिस्तथा हेत्ववधारणम्\renewcommand{\thefootnote}{1}\footnote{न \textendash\ अवधारणम्}~।\\
दूतो लेखस्तथा स्वप्नश्चित्रं मद इति स्मृतम्\renewcommand{\thefootnote}{2}\footnote{म \textendash\ द्विजा}~॥

\renewcommand{\thefootnote}{$\ddagger$}\footnote{अर्थोपक्षेपपञ्चकोद्देशलक्षणविधायिन एते सप्त श्लोकाः प्रक्षिप्ता एव, यतः पूर्वाध्यायेऽङ्कलक्षणावसरे यत्रार्थस्य समाप्तिरित्यत्र (१६ \textendash\ २६) तल्लक्षणानि सूचितानि मुनिना व्याख्यात्रा च कोहलमतानुसारेन वा संग्रहकारमतेन वा श्लोकाश्चैते कैश्चद्विनिवेशिताः~। एतेषां सप्तमः कोहलस्यैव, द्वितीयस्तु कोहलश्लोकाच्चतुर्थपादे मिद्यते, तृतीयषञ्चमौ भरतस्यैवाष्टादशाद्गृहीतौ~।}[ विष्कम्भश्चूलिका चैव तथा चैव प्रवेशकः~।\\
अङ्कोवतारोऽङ्कमुखमर्थोपक्षेपपञ्चकम्~॥~११०}
\end{quote}

\hrule

\vspace{2mm}
\noindent
{\qt नल} ( वेणी \textendash\ १ ) इति क्रोधात्मोपक्षेपः, रामाभ्युदये भयात्मोपक्षेपः, प्रतिमानिरुद्धे स्वप्नरूपः, उदात्तराघवे हेत्ववधारणात्मा~। एवमन्यदनुसरणीयम् ( इति~। एते च विभावानुभावव्यभिचारिरूपा एव~। न तु तदतिरिक्तं जगति किंचिदस्ति प्रयोगे~। प्रयोगोज्ज्वलत्वोपयोगाय तूपलक्षणत्वेनैकविंशतिरित्युक्तं कवेर्मार्गं प्रदर्शयितुम्~।\\

तत्र सामादयो वीरे उज्ज्वलत्वहेतवः, वधो रौद्रे, प्रत्युत्पन्नमतित्वं मतिलक्षणं व्यभिचारिरूपं, सर्वत्र गोत्रस्खलनमीर्ष्याविप्रलम्भे, साहसं ( शृङ्गारवीरादौ ), चापलं हास्यादौ~। एवमन्यत्र~। \underline{ओज} इति तेजः, सामान्याभिनये (अ \textendash\ २२) लक्षयिष्यते \textendash\ {\qt अधिक्षेपावमानादेः} ( इत्यत्र ), संवरणमवहित्थं, चित्रं विस्मयः शिल्पविशेषश्च~। एते सर्वेषु नाटकादिरूपकेषु सुलभाः स्वयं च सुज्ञाना इति तदुदाहारणपरिवर्तनेन ग्रन्थो न विस्तारितः\renewcommand{\thefootnote}{*}\footnote{सामादीनामुदाहरणानि ग्रन्थान्तेऽनुबन्धरूपेण दर्शयिष्यन्ते~।}~।

\newpage
% एकोनविंशोऽध्यायः ६५

\begin{quote}
{\na मध्यमपुरुषनियोज्यो नाटकमुखसन्धिमात्रसंचारः~।\\
विष्कम्भकस्तु कार्यः पुरोहितामात्यकञ्चुकिभिः~॥~१११

शुद्धः संकीर्णो वा द्विविधो विष्कम्भकस्तु विज्ञेयः~।\\
मध्यमपात्रैः शुद्धः संकीर्णो नीचमध्यकृतः~॥~११२

अन्तर्यवनिकासंस्थैः सूतादिभिरनेकधा~।\\
अर्थोपक्षेपणं यत्तु क्रियते सा हि चूलिका~॥~११३

अङ्कान्तरानुसारी संक्षेपार्थमधिकृत्य बिन्दूनाम्~।\\
प्रकरणनाटकविषये प्रवेशको नाम विज्ञेयः~॥

अङ्कान्त एव चाङ्को निपतति यस्मिन् प्रयोगमासाद्य~।\\
बीजार्थयुक्तियुक्तो ज्ञेयो ह्यङ्कावतारोऽसौ~॥~११५

विश्लिष्टमुखमङ्कस्य स्त्रिया वा पुरुषेण वा~।\\
यदुपक्षिप्यते पूर्वं तदङ्कमुखमुच्यते ]~॥ ११६}
\end{quote}

\hrule

\vspace{2mm}
एवमितिवृत्तनिरूपणनान्तरीयकत्वेन सन्धयः सन्ध्यङ्गानि सन्ध्यन्तराणि चात्मभूतरसोपयोगीन्यपि प्राधान्येनेतिवृत्तात्मकं शरीरांशमभिनिविशमानानि, तत एव वृत्तिचतुष्कसाधारणे निदर्शितानि~। अधुना तु यस्याः प्रसादेन शास्त्रेतिहासादिभ्योऽभ्युद्धरकन्धरीभूतं सर्वजनाहरणीयतास्पदत्वं तु नाट्यं, यामुद्दिश्य प्रथमेऽध्याये {\qt कैशिकीमपि योजय यच्च तस्याः क्षमं द्रव्यं} इत्यादि बहुतरमुक्तं, तदाविर्भावकानि, अत एवात्मभूतरसभावभागाभिनिवेशशालीन्येव लास्याङ्गान्यपि कविप्रयोक्तृभिरभिनेतव्यकाव्यविषये

\lfoot{9}

\newpage
\lfoot{}
% ६६ नाट्यशास्त्रम्

\begin{quote}
{\na \renewcommand{\thefootnote}{*}\footnote{लास्याङ्गलक्षणं भ \textendash\ मातृकायां न दृश्यते , चपम \textendash\ मातृकासु विना सर्वास्वन्यास्वष्टादशाध्याय एव पठितम्, लक्षणपाठोऽपि भिन्नमातृकासु बहुभेदतया विद्यते~। भोजशारदातनयादिभिरपि तल्लक्षणे मात्रया भिन्नं मतमुपन्यस्तम्~।}अन्यान्यपि\renewcommand{\thefootnote}{1}\footnote{उ \textendash\ अन्यानि च} लास्यविधावङ्गानि तु नाटकोपयोगीनि\renewcommand{\thefootnote}{2}\footnote{ड \textendash\ उपयुक्तानि य \textendash\ योगीति}~।\\
\renewcommand{\thefootnote}{3}\footnote{य \textendash\ तस्मात्}अस्माद्विनिःसृतानि\renewcommand{\thefootnote}{4}\footnote{फ \textendash\ विनिर्गतानि} तु भाण इवैकप्रयोज्यानि~॥~११८

[ भाणाकृतिवल्लास्यं विज्ञेयं त्वेकपात्रहार्यं वा\renewcommand{\thefootnote}{5}\footnote{ढ \textendash\ च}~।\\
प्रकरणवदूह्य कार्यासंस्तवयुक्तं विविधभावम्~॥ ]

गेयपदं स्थितपाठ्यमासीनं पुष्पगण्डिका\renewcommand{\thefootnote}{6}\footnote{ड \textendash\ पुष्पगन्धिका}~।\\
प्रच्छेदकं त्रिमूढं च सैन्धवाख्यं द्विमूढकम्~॥~११९}
\end{quote}

\hrule

\vspace{2mm}
\begin{sloppypar}
\noindent
सर्वथैव योज्यानीति दर्शयितुमाह \underline{अन्यान्यपि लास्यविधावङ्गानी}त्यादि~। \underline{नाटक}मित्यभिनेयमात्रम्~। इतःपरमध्यायान्तमुक्तेभ्योऽङ्गेभ्यो लास्यविधौ यान्यङ्गानि वक्ष्यन्ते तानि नाटकोपयोगीन्यपि भवन्ति~।\\
\end{sloppypar}

नन्वेवमङ्गानामभेदादङ्गिनोऽपि लास्यस्य नाटके को भेद इत्याशङ्कां शमयति ( \underline{अस्मादिति} )~। अस्मान्नाटकादनुकाराभिनेयलक्षणात् विनिस्मृतानि बहिर्भूतानि, एकपात्रहार्याणि~। \underline{भाण} इति \underline{इवश}ब्देन नाटकमाह, भाणे नाट्यरूपता समस्ति, न तु लास्ये कथंचिदपि तस्य नाट्यरूपवैलक्षण्यात्~। तच्चोपपादितं वितत्य तुर्येऽध्याये~।\\

ननु कानि लास्याङ्गानि नाट्ये वक्ष्यन्त इत्याह गेयपदमित्यादि दशविधं ह्येतदङ्गनिर्देशलक्षणमित्यन्तम्~। एतस्य अङ्गनिर्देशस्याङ्गोद्देशस्य दशविधं, यद्विशेषलक्षणं तालाध्याये ( अ \textendash\ ३१ ) लास्यनिरूपणावसरे वक्ष्यते, तथा चोपसंहरिष्यति {\qt एतेषां लास्यविधौ विज्ञेयं लक्षण} मिति ( अ \textendash\ १९ )~। तत्रैव हि संपूर्णमङ्गानां रूपं, इह तर्हि कथमुपयोग इति नाट्योपयोगितां

\newpage
% एकोनविंशोऽध्यायः ६७

\begin{quote}
{\na \renewcommand{\thefootnote}{1}\footnote{ड \textendash\ उत्तमोत्तमकं चैव विचित्रपदमेव च~। उक्तप्रत्युक्तमावं च लास्याङ्गानि विदुर्बुधाः}उत्तमोत्तमकं चैवमुक्तप्रत्युक्तमेव च~।\\
लास्ये दशविधं ह्येतदङ्गनिर्देशलक्षणम्~॥~१२०

\renewcommand{\thefootnote}{2}\footnote{ड \textendash\ आसने चोपविष्टायां}आसनेषूपविष्टैर्यत्तन्त्रीभाण्डोपबृंहितम्~।}
\end{quote}

\hrule

\vspace{2mm}
\noindent
गमयितुं \underline{आसनेषूपविष्टैर्य} इत्यादिग्रन्थः~। तेनेदं तात्पर्यम् \textendash\ यानि लास्याङ्गानि वक्ष्यन्ते तेभ्यः कश्चिद्वैचित्र्यांशो लोकापरिदृष्टोऽपि रञ्जनावैचित्र्याय कविप्रयोक्तृभिर्नाट्ये निबन्धनीयः\\

अन्ये तु व्याचक्षते \textendash\ तथाविधलास्याङ्गयोजनैवात्र क्रियते, तथा हि गेयपदे निदर्शनं दर्शयति ततः प्रविशति वीणां वादयन्ती मलयवती चेटी च~। मलयवती {\qt उत्फुल्लकमलकेसर} इत्यादि गायति ( नागा \textendash\ १ ) इति~। तच्चेदमसत्~। अत्र ह्यन्यव्यापारवद् देवतापरितोषः, किंचिद्गेयं जप्यसहस्रतुल्यं तन्मिश्रं जप्यं कोटिफलसाधनमित्यादिपुराणवाक्यबलात्, कर्तव्यत्वेनाभिसंहितो मलयवत्याः~। {\qt सा च प्रयोज्ये}ति न लास्यार्थोऽत्र किञ्चित्, न लास्याङ्गतापि~। यत्रापि, ततः प्रविशतो गायन्त्यौ चेट्यौ, कुसुमाउहपिअदूअउपे ( रत्ना \textendash\ २ ) इत्यादि तत्रापि परिभ्रमणादिवदेव लौकिकवृत्तं वसन्तोत्सवप्रमोदाभ्युदयावसरकृतं~। प्रयोजने चेद्यावत् क्रियते तत्र यद्यप्यनुकार्यस्य तथापि नाट्याङ्गत्वे पृथगनुपदेश्यतापत्तिः, यथा ( यदा ) ह्यश्वमेधयागाद्यनुकारः कर्तव्यस्तदोपयुज्यते यज्ञाङ्गज्ञानमिति यज्ञाङ्गान्युपदेश्यानि भवेयुः~। न हि तादृग्वस्तुमात्रमप्यस्ति यन्नाट्ये नोपयुज्यते च~। तस्माल्लास्ये यान्यङ्गानि तत उपजीव्ये लौकिक एवांशो रञ्जनोपयोगी लास्याङ्गत्वेन मुनेरिह विवक्षितः~। अन्यथा {\qt आसनेषूपविष्टैर्य} दित्यादि किमिहोक्त्या, {\qt एतेषां लक्षणं व्याख्यास्ये} इत्येतदिहैव तु नैवोक्तमित्येतद् विशृङ्खलं स्यात्, {\qt ततश्च परिधानक} मित्यादि ( ३१ \textendash\ ३५० ) ( यत्पूर्वरङ्गविधौ ( अ \textendash\ ३१ ) लास्याङ्गलक्षणमुक्तं तदप्यत्राभिनेयभागे प्रयोक्तव्यं स्यात्~। )\\

ततो यावानंशो नाट्योपयोगी तं दर्शयितुमाह \underline{आसनेषूपविष्टैर्यदि}ति~। यच्छब्दो निपातो यस्मिन्नित्यत्रार्थे, तेन यत्र काव्ये प्रयोगे वा \underline{शुष्कमि}त्यनु \textendash

\newpage
% ६८ नाट्यशास्त्रम्

\begin{quote}
{\na गायनैर्गीयते शुष्कं तद्गेयपदमुच्यते\renewcommand{\thefootnote}{1}\footnote{फ \textendash\ गेयं पदभिष्यते च \textendash\ उत्तमम्}~॥~१२१

[ या नृत्यत्यासना नारी गेयं प्रियगुणान्वितम्~।\\
साङ्गोपाङ्गविधानेन तद्गेयपदमुच्यते~॥ ]~१२२

प्राकृतं \renewcommand{\thefootnote}{2}\footnote{ड \textendash\ या विमुक्ता तु}यद्वियुक्ता तु पठेदात्तरसं स्थिता\renewcommand{\thefootnote}{3}\footnote{ड \textendash\ आसनसंस्थिता य \textendash\ आन्तरसंस्थिता}~।\\
मदनानलतप्ताङ्गी स्थितपाठ्यं तदुच्यते~॥~१२३}
\end{quote}

\hrule

\vspace{2mm}
\noindent
करणीयतया शून्यं \underline{गायनैरिति} न तु पात्रैः, \underline{आसनोपवि}ष्टैरिति स्वस्थैः, न तु नेपथ्ये गीयत इत्यादिवत्, कविप्रयोगायातमावेशविशेषं जुम्भद्भिर्गीयते यत्~। \underline{तन्त्रीभाण्डान्वि}तमिति सर्वातोद्ययुतं, न तु भाव्यासीनपाठ्यवत्तद्विहीनं तद्गेयस्य पदं स्थानमिति कृत्वा गेयपदम्, तेन ध्रुवागानपञ्चकमन्तरालापस्वररहितं यत्र प्रयोगयोग्यं भवति स काव्यप्रयोगो गेयपदमित्युक्तं भवति, यत्र हि प्रयोगे तत्तत्राभिनिविष्टं सामाजिकरञ्जकं भवतीति यावानंशोऽसौ लास्याङ्गादिहोपजीवितः~।\\

यत्तु गायनैः पात्रैः शुष्कमित्यर्थात् छेकाश्रितं गेयं निर्गीतमपि वा, एतत् त्र्यश्रं चतुरश्रं वेत्येवंभूतं गीयमानं गेयानि पदानि यत्र गेयपदमिति व्याचक्षते, तत्पूर्वमेवापास्तम्~।\\

\begin{sloppypar}
अथ स्थितप्रापि यल्लास्याङ्गं भविष्यति तदुपजीवितुमाह \underline{प्राकृतं यद्वियुक्ता त्वि}ति~। लास्येऽपि तावद् देवतानरपतिरञ्चनप्रधानं पाठ्यमस्ति~। तच्चित्तग्रहणं हि तत्र तेनैव मध्ये वैचित्र्याय पाठ्येनापि क्रियते, तत्र स्थिते च पठत्यासीनेवेति पाठ्यगतं तदलौकिकं रञ्जनाङ्गं चित्रत्वं तस्मादङ्गादुपजीव्यते~। तथा हि \textendash\ यद्वियुक्ता आतप्तापि सती प्राकृतभाषालक्षणयुक्तं तथात्तरसमिति रसोपयोगि स्थायिरसग्रहणपूर्वकं पठेत्~। एतल्लौकिकं यल्लास्याङ्गादुपजीव्यमानं स्थितपाठ्यम्~। एतच्चावेशोपलक्षणं तेन क्रोधाविष्टोऽपि संस्कृतेन पठतीत्याद्यपि मन्तव्यम्~।
\end{sloppypar}

\newpage
% एकोनविंशोऽध्यायः ६९

\begin{quote}
{\na [ \renewcommand{\thefootnote}{1}\footnote{न \textendash\ भूमि}बहुचारीसमायुक्तं पञ्चपाणिकलानुगम्~।\\
चच्चत्पुटेन वा युक्तं स्थितपाठ्यं विधीयते~॥ ]~१२४

\renewcommand{\thefootnote}{2}\footnote{न \textendash\ आसीनमासनस्थस्य}असीनमास्यते यत्र सर्वातोद्यविवर्जितम्~।\\
\renewcommand{\thefootnote}{3}\footnote{ढ \textendash\ अपसारित, फ \textendash\ अवसारित, न \textendash\ अप्रसाधितगात्रं तु प \textendash\ संप्रसारित (ड सु)}अप्रसारितगात्रं च \renewcommand{\thefootnote}{4}\footnote{ड \textendash\ जिह्मदृष्टिनिरीक्षितम्, य \textendash\ चिन्ताशोकान्वितं च यत्}चिन्ताशोकसमन्वितम्~॥~१२५

\renewcommand{\thefootnote}{5}\footnote{प \textendash\ नृत्तानि,}वृत्तानि विविधानि स्युर्गेयं \renewcommand{\thefootnote}{6}\footnote{प \textendash\ चातोद्य प \textendash\ गानैश्च संस्सृतम्}गाने च संश्रितम्~।\\
\renewcommand{\thefootnote}{7}\footnote{प \textendash\ चेष्टा च विविधा}चेष्टाभिश्चाश्रयः पुंसां\renewcommand{\thefootnote}{8}\footnote{च \textendash\ पुंसा} यत्र सा पुष्यगण्डिका\renewcommand{\thefootnote}{9}\footnote{पुष्पगन्धिकेति च बहुष्ट्वादर्शेषु दृश्यते}~॥~१२६}
\end{quote}

\hrule

\vspace{2mm}
\underline{अन्ये तु} बहुचारीयुतेन चच्चत्पुटेनोत्तरेण यत् स्थितपाठ्यमिति लक्षणं कुर्वन्ति, उदाहरन्ति, रत्नावल्यां द्वितीयेऽङ्के राजा \textendash\ {\qt उद्दामोत्कलिका} मित्यादीति, तत्पूर्वमेव निरस्तम्~। न च पाठ्ये चावसरोऽत्र तालस्त्र्यश्रश्चतुरश्रो वा, यथा तु लास्याङ्गत्वे~। तत्सर्वं तालाध्याय एव वक्ष्यामः~।\\

अथासीनुपाठ्यादुपजीवनीयमंशमाह \underline{आसीनमास्यते यत्रेति}~। अत्यन्तशोकावेशेऽभिनयादिशून्यत्वेन यत्र आस्ते सोंऽश उपरञ्जकगुणश्चतुर्विधातोद्यवर्जितोऽतिसुकुमारकाकलीप्रायप्रमदागीतमात्रावशेषो यश्चित्तग्राही ( स लौकिकः ) लौकिकादपि च तत्र हि साम्यमात्रार्थस्य तस्याश्रयमाणा च स्थितिः, तदासीनन्नामाङ्गं आसीनपाठ्यादुपजीवितेनासनांशेन योगतश्च सर्वत्र करुणादौ रञ्जनोपयोगि~। तदाह \underline{चिन्ताशोकसमन्वित}मिति~। अधःशय ( नध्यानाधोमुखाद्यनुभावयुतम्~। \underline{सर्वेति} ध्यानविलापादिचिन्ताशोकानुभावेषु ततादेरप्रयोज्यतया तद्राहित्यमुक्तम्, ) निःशब्दमिति भावः~। तदनु अप्रसारितगात्रमित्यभिनयशून्यमित्यर्थः~। सुप्रसारितगात्रमित्यन्ये पठन्ति, तत्रापि ( सोष्ठवाङ्गप्रदर्शनपृथग्यत्नराहित्येन )स्रस्तगात्रतयाभिनयशून्यतैव~।\\

पुष्पगडिकाख्यलास्याङ्गादुपजीव्यांशमाह \underline{वृत्तानीति}~। गान इति ) ततेन

\newpage
% ७० नाट्यशास्त्रम्

\begin{quote}
{\na \renewcommand{\thefootnote}{1}\footnote{डजशादिषु \textendash\ अयं पाठः}[ यत्र स्त्री नरवेषेण ललितं संस्कृतं पठेत्~।\\
सखीनां तु विनोदाय सा ज्ञेया पुष्पृगण्डिका~॥~१२७

\renewcommand{\thefootnote}{2}\footnote{नयमादिष्ट्वयं पाठः}नृत्तं तु विविधं यत्र गीतं चातोद्यसंयुतम्~।\\
स्त्रियः पुंवच्च चेष्टन्ते सा ज्ञेया पुष्पगण्डिका~॥ ]~१२८

प्रच्छेदकः स विज्ञेयो यत्र चन्द्रातपाहताः\renewcommand{\thefootnote}{3}\footnote{ठ \textendash\ चन्द्राहतातपाः}~।\\
स्त्रियः प्रियेषु \renewcommand{\thefootnote}{4}\footnote{न \textendash\ रज्यन्त अपि}सज्जन्ते ह्यपि विप्रियकारिषु~॥~१२९}
\end{quote}

\hrule

\vspace{2mm}
\noindent
मध्ये सुषिरेण मध्येऽवनद्धेन मिश्रणाकृतो विचित्रभावः पात्राणां सुकुमारप्रयोगोऽभिनेयेऽपि रञ्जक एव यद्यप्यल्लौकिकं यद्वैचित्र्यम्~। मालासादृश्यात्पुष्पगण्डिका गाननृत्तगीतगतवैचित्र्ययोगात् ( स्त्रीलिङ्गविवक्षया च स्त्रीपात्राणां पुष्पगण्डिकोक्ता~। \underline{पुंसामिति} ) सा च चेष्टाश्रयशब्दाभ्यां पर्यायेण योज्या~।\\

अथ प्रच्छेदकाङ्गकृतं वैचित्र्यं योजयितुमाह ( \underline{प्रच्छेदक} इति लास्यविधाने\textendash\ ( अ ३१ ) वक्ष्यते {\qt ज्योत्स्नायां मदिरायां वा दर्पणे सलिलेऽथवा~। ) छायासादृश्यकान्तस्य प्रहर्षार्थविभूषित}मिति ( ३१ \textendash\ अ ) त्रिधाप्रच्छेदकस्य लक्षण ( मुक्तम्~। तत्र जलक्रीडायां जले प्रसाधने दर्पणे पानगोष्ठ्यां पान ) ईषत्प्रतिफलिततत्तदाकृतिदर्शने सति कान्तायाः प्रहर्ष इति त्रिधा प्रच्छेदं प्रति विमल ( फलन ? )मिति पर्यायात् ( काव्येषु कविभिः प्रतिबिम्बदर्शनजातहर्षस्य स्त्रीणां प्रणयकोपप्रसादनसामर्थ्यं वर्णितम्~। ) यथा\textendash

\begin{quote}
{\qt पणमह पणयप्पकुविअगोरीचलणग्गलग्गपडिबिंबं~।\\
तंसु णहदप्पणेसु अ एआअअ तनुअरं रुद्दं~॥

( पणमत पणअप्पकुपिअकोरीचळणक्कलक्कपटिपिंपम्~।\\
तंसु नखतप्पनेसुं एकातस तनुतरं रुत्तम्~॥\renewcommand{\thefootnote}{*}\footnote{इयं गुणाढ्यकृतबृहत्कथायां प्रथमा गाथा \textendash\ प्रणमत प्रणयप्रकुपितगौरीचरणाग्रलग्नप्रतिबिम्बम्~। तेषु नखदर्पणेषु एकादशतनुतरं रुद्रम्~॥ इति छाया})}
\end{quote}

\newpage
% एकोनविंशोऽध्यायः ७१

\begin{quote}
{\na अनिष्ठुर\renewcommand{\thefootnote}{1}\footnote{ड \textendash\ स्वल्प}श्लक्ष्णपदं समवृत्तैरलङ्कृतम्~।\\
नाट्यं पुरुषभावाढ्यं त्रिमूढकमिति स्मृतम्\renewcommand{\thefootnote}{2}\footnote{प \textendash\ उदाहृतम्}~॥~१३०}
\end{quote}

\hrule

\vspace{2mm}
\noindent
इत्यादि~। तत्र च त्रिविधेऽपि ज्योत्स्नैवोपयोगिनी,\ldots\ldots\ldots.धानलीलानि स्वाधीनभर्तृकोचितसंभोगविशेषोपलम्भेन~। एतदुक्तं भवति \textendash\ लोकवृत्ते तावन्न सर्वदा संपूर्णचन्द्रोदयो भवति प्रयोगे तु संभावनागर्भतया रसोपयोगी तथाविधः कालविशेषो गृहीतव्यः~। तथा च रत्नावल्यां भूयसा चन्द्रोदयो वर्णितः, {\qt संप्रत्येष सरोरुहद्युतिमुष} ( १ \textendash\ २३ ) इति, {\qt उदयतटान्तरित}मिति ( १ \textendash\ २४ ),{\qt वक्त्रेन्दौ तव सत्ययं यदपरः शीतांशुरभ्युद्गत} ( ३ \textendash\ १३ ) इति~। एवं रसोप \textendash\ योग्यलौकिककालविशेषग्रहणं प्रच्छेदकादुपजीवितम्~। यदाहुरूपाध्यायपादाः \textendash

\begin{quote}
{\qt यद्यत्रास्ति न तत्रास्य कविर्वर्णनमर्हति~।\\
यन्नासंभवि तत्रास्य तद्वर्ण्यं सौमनस्यदम्~॥

देशोऽद्रिदन्तुरो द्यौर्वा तटित्कुण्डलमण्डिता~।\\
ईदृक्स्यादथवा न स्यात् किं कदाचन कुत्रचित्~॥} इत्यादि~।
\end{quote}

\noindent
अथ त्रिमूढलक्षणादुपयोगिनं भागं निरूपयितुमाह \underline{अनिष्ठुरश्लक्ष्णप}दमित्यादि~। त्रिमूढके तावल्लास्याङ्गे नायकस्य व्यलीकवशादेकस्या द्वेष्यतोऽभिनवायाः प्रथमप्रणयबद्धलज्जादिनेति मोहस्त्रयाणाम्~। तत्र नायकस्यावश्यमनिष्ठुराण्येवोदयनविषयाणीव वचांसि भवन्ति~। तत इह वचसि रसोपयोगी गुणालङ्कारांशः स्वीकर्तव्यः~। न हि लोके यदृते माधुर्यौजः प्रभृतिगुणगणोऽस्ति सर्वो ह्येवमाविष्टो यत् किंचिद्वदति तत्काव्यमेव स्यात्~। छायामात्रेण त्वंशतः स एवापूर्णत्वादनुपयोग्येव~। तस्मादलौकिकमेवैतद्वैचित्र्यं रसोपयोगे सर्वं तद्गुणग्रामकृतम्~। \underline{समवृत्तै}रित्यनेन वृत्तकृतवैचित्र्यं, \underline{पुरुषभावाढ्यय}मित्यनेन पात्रकृतमपि हेलाभावादि, एवं विशेषकृतं वैचित्र्यं यत्र नाट्ये सौन्दर्यमस्तीति दर्शयन् गुणानामेवान्वयव्यतिरेकाभ्यां तत्र प्रभावं दर्शयति~।

\newpage
% ७२ नाट्यशास्त्रम्

\begin{quote}
{\na \renewcommand{\thefootnote}{1}\footnote{ड \textendash\ पात्रं विस्मृत, न \textendash\ रूपवाद्यादिसंयुकं}पात्रं विभ्रष्टसङ्केतं सुव्यक्तकरणान्वितम्\renewcommand{\thefootnote}{2}\footnote{न \textendash\ आश्रयम्}~।\\
\renewcommand{\thefootnote}{3}\footnote{य \textendash\ पाठ्यहीनं स्वभावोक्त्या}प्राकृतैर्वचनैर्युक्तं विदुः सैन्धवकं बुधाः~॥~१३१

\renewcommand{\thefootnote}{4}\footnote{पा दि मातृकासु पाठः}[ रूपवाद्यादिसंयुक्तं पाठ्येन च विवर्जितम्~।\\
नाट्यं हि तत्तु विज्ञेयं सैन्धवं नाट्यकोविदैः~॥ ]~१३२}
\end{quote}

\hrule

\vspace{2mm}
अथ सेन्धवकादुपजीव्यमंशं स्वीकर्तुमाह पात्रं विभ्रष्टसंकेतमिति 

\begin{quote}
{\qt सैन्धवीमाश्रिता भाषा ज्ञेयं तत्सैन्धवं बुधैः~।\\
रूपवाद्यादिसयुक्तं}
\end{quote}

\noindent
इति च लास्याङ्गविधाने ( अ \textendash\ ३१ ) वक्ष्यते~। तत्र यदा नान्यप्राकृतादिभाषोपकरणत्वेन सैन्धवीप्राया आश्रीयते तद्रसोपयोगि रञ्जनाधिक्यात्, अलौकि \textendash\ कोऽयमर्थो रञ्जनोपयोगी लास्याङ्गात् स्वीकृतो भूवति~। तथा हि शृङ्गार \textendash\ रसे सातिशयोपयोगिनी प्राकृतभाषेति सट्टकः कर्पूरमञ्जर्याख्यो रजशेखरेण तन्मय एव निबद्धः, भेज्जलेन राधाविप्रलम्भाख्यो रासकाङ्कः सैन्धवभाषा \textendash\ बाहुल्येन, चन्द्रकेन स्वानि रूपकाणि वीररौद्राधिकोपयोगीनि संस्कृतभाषयैव\renewcommand{\thefootnote}{*}\footnote{भासकृतमिति मुद्रापितं भारतकथावस्तुकं पञ्चरात्रादिनाटकषट्कं चन्द्रकस्यैवेति मन्यामहे~।}~। अत एव च तत्तद्रसोपयोगतारतम्यादेवैकतमस्यातोऽन्यस्यात्र प्राधान्यं कल्प्यते~।\\

\begin{sloppypar}
तदेतदाह \underline{पात्र}मित्यादिना~। जातावेकवचनं, तेन यत्र पात्राणि \underline{प्राकृतैर्वचनैर्यु}क्तानि, अत एव विभ्रष्टः संकेतविशेषः रसोचितः काकध्याये कथितपूर्वः, स एष भ्रष्टत्मा भ्रंशं नीतः संकेतो यत्र\renewcommand{\thefootnote}{$\dagger$}\footnote{दृशानितसौबकचक्र इति तालपत्रादर्शयोः~।}, सुष्ठु व्यक्तिर्यतो रसस्य तेनैव \underline{करणेन} वीणावाद्यादिक्रिययान्वितं पात्रं \underline{सैन्धवकमिति} विषयतद्वतोरभेदोपचारात्~। तेन दशरूपकस्य यद् भाषाकृतं वैचित्र्यं कोहलादिभिरुक्तं तदिह मुनिना सैन्धवाङ्गनिरूपणे स्वीकृतमेव~।
\end{sloppypar}

\newpage
% एकोनविंशोऽध्यायः ७३

\begin{quote}
{\na \renewcommand{\thefootnote}{1}\footnote{ड \textendash\ शुभार्थगीताभिनयं}मुखप्रतिमुखोपेतं चतुरश्रपदक्रमम्~।\\
\renewcommand{\thefootnote}{2}\footnote{ड \textendash\ स्पष्ट, य \textendash\ अष्ट}श्लिष्टभावरसोपेतं \renewcommand{\thefootnote}{3}\footnote{न \textendash\ नानार्थं तु विमूढकम् च \textendash\ विचित्रार्थं प \textendash\ चित्रार्थं तद्विमूढकम् ड \textendash\ व्याजचेष्टं द्विमूढकम्}वैचित्र्यार्थं द्विमूढके~॥~१३३}
\end{quote}

\hrule

\vspace{2mm}
करुणान्वितमिति त्वपपाठः~। एवं {\qt पाठ्यविहीनं सैन्धवक} मिति, यथा रत्नावल्यां विदूषको नृत्यतीति प्रियाप्रतिनिधिप्रभ्रंश\renewcommand{\thefootnote}{1}\footnote{प्रियातिथिप्रभृतिः \textendash\ इति लेखकदोषः स्यात्}, इति मुनिमतोपेक्षयैव लक्षण उदाहृतं च कृतं न चोक्तं युक्त्या तेन किंचिदित्यसदेव~।\\

द्विमूढकाद् वैचित्र्यांशं खीकरोति, \underline{मुखप्रतिमुखमित्या}दिना~। यत्र काव्ये लास्यमित्याश्रये द्वयोर्नायकस्य नायिकायाश्च नायिकयोर्वा प्रतिमूढक इव मोहो व्यावर्ण्यते तत्र, तालनिरूपणायामेकस्तालश्चतुर्भिः पादैर्युक्तः सन्नावर्तत इति वक्ष्यते ( अ \textendash\ ३१ ) \textendash

\begin{quote}
{\qt मुखप्रतिमुखोपेतं तथा चाचपुटाश्रयम्~।\\
यथाक्षरैः सन्निपातैस्तथा द्वादशभिर्युतम्~॥} इति~।
\end{quote}

\noindent
तस्मादङ्गाङ्गपरिक्रमे यदेव तालानुसरणं चतुर्षु च मुखेषु गतिपरिसमापनं लोके गत्यादावपरिदृष्टमपि नटसामाजिकवर्गसविधाध्यासनेन\renewcommand{\thefootnote}{2}\footnote{सविधाव्यापादनेन.} साम्याङ्गस्यानन्दस्वभावतया रसांशेतराङ्गत्वादतिशयेन रसोपयोगीति तत् स्वीकृतम्~।\\

मुखं यदग्रे सामाजिकाः प्रतिमुख्यस्ततोऽन्या दिशो लास्याङ्गं नेतुम्, मुखप्रतिमुखे गीतकाङ्गत्वेन हि~। ( \underline{चतुरश्रेति} ) चतुरश्रं कृत्वा क्रान्तदिक्चतुष्टयैः पदैः क्रमणं यत्रेति~। किमेवं सति भवतीत्याह वैचित्र्यार्थं वैचित्र्यमत्र प्रयोजनमिति~। तेनापि किमिति चेत्तत आह ( \underline{श्लिष्टभावेति} ) श्लिष्टभावेन सर्वसाम्यलक्षणेन चित्तवृत्तिसंघट्टेन रसानामुपगतं स्फुठत्वं येनेति~।

\lfoot{10}

\newpage
% ७४ नाट्यशास्त्रम्
\lfoot{}

\begin{quote}
{\na उत्तमोत्तमकं विद्यादनेकरससंश्रयम्\renewcommand{\thefootnote}{1}\footnote{न \textendash\ करणान्वितम्}~।\\
\renewcommand{\thefootnote}{2}\footnote{प \textendash\ विचित्र}विचित्रैः श्लोकबन्धैश्च \renewcommand{\thefootnote}{3}\footnote{न \textendash\ लीलाभावविभूषितम्}हेलाहावविचित्रितम्\renewcommand{\thefootnote}{4}\footnote{प \textendash\ विभूषितम्.}~॥~१३४}
\end{quote}

\hrule

\vspace{2mm}
गीतकाङ्गाभ्यां मुखप्रतिमुखाभ्यामङ्गसौष्ठवेन रसभावैर्नायिकाद्वयरचनया च युतमिति श्रीशङ्गुकाद्याः, तच्च प्रागेव परीक्षितम्~।\\

मुखप्रतिमुखौ सन्धी इत्येतदपि न समीचीनं, अनुपयोगादस्यार्थस्येत्युपाध्यायमतमेवानवद्यमावेदितपूर्वं ग्राह्यम्~।\\

\begin{sloppypar}
अथोत्तमोत्तमकमङ्गं नाट्य \underline{उपयोजयितुमाह उत्तमोत्तमकं} \underline{विद्यादनेकरससंश्रयमित्यादि}~। {\qt उत्तमोत्तमके त्वादौ नर्कुटं संप्रयोजयेत्} इत्यादिना {\qt हेलाहावविभूषित} मित्यन्तेन लास्याङ्गं लक्षयिष्यते ( ३१ \textendash\ अ )~। तत्र च चित्तवृत्तिपरिपोषो हेलाहावादिचेष्टालङ्कारमुखेन यः स्थितः सोऽत्र लौकिकः सन्नुपजीव्यते~। न हि भगवदवतारस्य रामस्य कुलकलत्रमनुपेक्ष्यं; पराभवखलीकारश्च न सोढव्यः, क्षत्रियेण च लोककण्टकाभ्युद्धरणं कर्तव्यमिति यच्छास्त्रार्थपरिपालनं मुक्त्वा, रावणहृतसीतासमानयनाद्युचितं निमित्तं, परमार्थतो न तु यथा प्रतिरूपं {\qt महुर्व्याधूतास्तादृश} मित्यादि वर्ण्यते तथास्य तृतीयत्रेतावतीर्णस्य ( रायस्य ) संभाव्यते~। रामायणेऽपि मुनिनो तथा वर्णितमिति चेत् किमतो वेदेऽपि हि तथा वर्ण्यतां, न वयमतो बिभीमः~। स हि भागः काव्यं यश्च यश्च रसाभिनिष्यन्दी\renewcommand{\thefootnote}{1}\footnote{तच्च तच्च तास्वनिष्यं} वर्ण्य इत्युक्तमसकृत्~। ततश्चान्यतररञ्जनोपक्रमविनेयहृदयसंवादिवैचित्र्यांशोऽसौ नाट्ये लास्याङ्गप्रसादोपनत एव~।\\
\end{sloppypar}

उत्तमानि तावल्लास्याङ्गानि, तेभ्योऽपीदमुत्तमं, सर्वं हि रसपर्यायीति दर्शितं प्राक्~। ततः संज्ञायां कन्~।\\

अनेकस्यासाधारणस्य रसस्य संश्रयोऽस्मिन्निति, स्थायिनां बहुत्व नूतनत्वाद् घटनाविचित्रत्वमुक्तम्~। श्लोके च बध्यत इति श्लोकबन्धः, ( बन्धानां ) विधास्तैः विविधैः ( चित्रै ) परमार्थताभूतैः क्वचित्कदाचित्संभवमात्रं

\newpage
% एकोनविंशोऽध्यायः ७५

\begin{quote}
{\na कोपप्रसादजनितं\renewcommand{\thefootnote}{1}\footnote{न \textendash\ बहुलं} \renewcommand{\thefootnote}{2}\footnote{च \textendash\ सविक्षेप}साधिक्षेपपदाश्रयम्\renewcommand{\thefootnote}{3}\footnote{प \textendash\ अधिक्षेपसमाथयम्, य \textendash\ पराश्रयम्}~।\\
उक्तप्रत्युक्तमेवं\renewcommand{\thefootnote}{4}\footnote{ढ \textendash\ एव} स्या\renewcommand{\thefootnote}{5}\footnote{ढ \textendash\ किंचिद्, न \textendash\ नृत्त}च्चित्रगीतार्थयोजितम्~॥~१३५}
\end{quote}

\hrule

\vspace{2mm}
\noindent
तथा वर्णितैर्युक्तम्, यथा ( विक्रमोर्वश्यां ) \textendash\ {\qt आ दुरात्मन् रक्षः, क्व नु खलु मे प्रियतमामादाय गच्छसि~। ( विभाव्य ) नवजलधरः सन्नद्धोऽयं} \renewcommand{\thefootnote}{*}\footnote{\begin{quote}
{\qt नवजलधरः सन्नद्धोऽयं न दृप्तनिशाचरः\\
सुरधनुरिदं दुराकृष्टं न नाम शरासनम्~।\\
अयमपि पटुर्धारासारो न बाणपरम्परा\\
कनकनिकषस्निग्धा विद्युत् प्रिया न ममोर्वशी~॥} ( ४ \textendash\ १ )
\end{quote}}इत्यादि पुरूरवस उक्तिश्चानुरागेणैव व्याख्यातेति किं पुनरुक्त्या~।\\

हेलाहावैर्विशेषेण दीप्ततागमनरूपेण भूषितमिति सात्त्विकाद्यनुभाव वर्गस्य सर्वस्यादीप्ततोपलक्षिता \textendash

\begin{quote}
{\qt तत्राक्षिभ्रूविकाराद्यः शृङ्गाराकारसंयुतः~।\\
सग्रीवारेचको ज्ञेयो भावस्थितसमुत्थितः~।\\
स एव हावः सा हेला ललिताभिनयात्मिका~। ( अ २२ )}
\end{quote}

\noindent
इति हेलाहावयोर्लक्षणम्~।\\

अथान्त्यमङ्गं नाट्योपयोगि कर्तुमाह \textendash\ कीपप्रसादजनितमित्यादि प्रथमार्धेन भाविलक्षणमेकदेशद्वारेणानूद्यते, द्वितीयेन तत आकृष्य वैचित्र्यभागो नाट्योपयोगी कथ्यते~। तदयमर्थः \textendash\ कोपप्रसादजनितं साधिक्षेपपदाश्रयं चेत्यादिरूपं यदुक्तं प्रत्युक्तं तथास्यगाने ( अ \textendash\ ३१ ) वक्ष्यते, नाट्ये एवमिति समनन्तरमेवं वक्ष्यमाणेन पथिना स्यात्, तमाह चित्रेति~। चित्रं यद्गीतं ध्रुवागानकाव्यं तस्य योऽर्थस्तेन संयोजनं यस्मिन्नाट्यांशे उक्तप्रत्युक्ते हि

\newpage
% ७६ नाट्यशास्त्रम्

\begin{quote}
{\na \renewcommand{\thefootnote}{1}\footnote{एतौ श्लौकौ प्रक्षिप्तावपि न य प ट भ \textendash\ मातृका विना सर्वास्वन्यासु दृश्येते ड \textendash\ यदि प्रतिकृतिं}यत्र प्रियाकृतिं दृष्ट्वा विनोदयति मानसम्~।\\
मदनानल\renewcommand{\thefootnote}{2}\footnote{ड \textendash\ तप्तं तु विचित्रपदं}तप्ताङ्गी तच्चित्रपदमुच्यते~॥~१३६}
\end{quote}

\hrule

\vspace{2mm}
\noindent
( वैचित्र्यं ) लास्याङ्गात् स्वीकृतम्~। यथा {\qt बाले नाथ विमुञ्च मानिनि रुषम्} \renewcommand{\thefootnote}{*}\footnote{अयं श्लोकः षोडशेऽध्याये आख्यानलक्षणस्य लक्ष्यतयोदाहृतः ( सं २ \textendash\ पृ ३१० )}इत्यादौ भिन्नानां वाच्यखण्डलकानां समन्वयो रसावेशमहिम्नैव, न तु श्रुतिलिङ्गादिप्रमाणषट्कं तदन्यत्रावकाशं लभते {\qt क्वाकार्यं शशलक्ष्मणः क्व च कुलं\renewcommand{\thefootnote}{$\ddagger$}\footnote{अयमपि षोडशेऽध्याय उपपत्तिलक्षण उदाहृतः ( सं २ \textendash\ पृ \textendash\ ३१८ )}} इत्यादावियमेव वार्ता~। एवं {\qt णळिनीदळणीसहमुत्तदेहिआ~। अइदुल्लहपडिबंधाणुराइआ~। ( नलिनीदलनिस्सहमुक्तदेहा~। अतिदुर्लभप्रतिबन्धानुरागा )~॥}\\

इत्यादेर्द्विपदिकागीतार्थस्याभिनये काव्यार्थस्य च {\qt हिअअ समास्सस} ( शाकु ), इत्यादेरसावेशकृत एवाभिनयार्थसमावेशो लोकापरिदृष्ट उपरञ्जनाय सातिशयमुपयोगी रसावेशवैवश्यप्रसादादित्युक्तप्रत्यक्ताल्लास्याङ्गादुपजीवितः~।\\

यद्यपि {\qt स्थाने ध्रुवास्वभिनयो यः क्रियते} ( ३२ \textendash\ ४७ ) इति नाट्यनिरूपणायां वक्ष्यते तथापि कविना तादृक्कार्यं प्रयोक्त्रा च तादृग्गीतं कर्तव्यमिति शिक्षणाय कृतश्चेदमङ्गमुपजीवितमिति शङ्काशमनाय रूपकाङ्गत्वेनेहास्याभिधानं, सामान्याभिनये त्वभिनयत्वेनास्य निरूपणं भविष्यति {\qt षडात्मकस्तु शारीर} ( २२ \textendash\ ४० ) इत्यत्र~। कोपप्रसादजनितत्वमित्यनेन चित्तवृत्त्यावेशस्थानमस्याङ्गस्य निवेशविषय इत्युक्तम्~। यत उक्तप्रत्युक्तमहिम्ना चाकाशभाषितात्मकमप्यलौकिकरूपं स्वीकुर्वता तत्सहचरं स्वगतजनान्तिकापवारितकाद्युपलक्षितम्~। न च सर्वथा तन्नाशो लोके क्वचित् संभवन्नाट्येऽपीति गीयमानः स्यात्, अनेन श्रोतुः काकतालीयवशात् स्वचित्तवृत्त्यार्थ \textendash

\newpage
% एकोनविंशोऽध्यायः ७७

\begin{quote}
{\na दृष्ट्वा स्वप्ने प्रियं यत्र\renewcommand{\thefootnote}{1}\footnote{न \textendash\ यत्तु} मदनानलतापिता~।\\
करोति विविधान् भावां\renewcommand{\thefootnote}{2}\footnote{च \textendash\ स तु.}स्तद्वै भाविकमुच्यते~॥~१३७}
\end{quote}

\hrule

\vspace{2mm}
\noindent
संवादो भवति~। यथा हर्षचरिते भगवत्याः सरस्वत्याः \renewcommand{\thefootnote}{*}\footnote{तरलयसि दृशं किमुत्सुकामकलुषमानसवासललिते~। अवतर कलहंसि वापिकां पुनरपि यास्यसि पङ्कजालयम्~॥ ( उ \textendash\ १ )}{\qt तरलयसि दृशं समुत्सुका} मित्यपरवक्त्रं शृण्वन्त्याः~। तत्सर्वं हि लोके यथा न स्यात् तर्हि प्रत्यक्षकल्पं कथमिव तत्र प्रतीतिरिति नाट्यरूपतैव भ्रंशेत~।\\

अन्ये तु चित्रपदं भाविकं चेत्यङ्गद्वयमाहुः पठन्ति च\textendash

\begin{quote}
{\qt यत्र प्रियाकृतिं दृष्ट्वा विनोदयति मानसम्~।\\
मदनानलतकप्ताङ्गी तच्चित्रपदमुच्यते~॥

दृष्ट्वा स्वप्ने प्रियं यत्र मदनानलतापिता~।\\
करोति विविधान् भावांस्तद्वै भाविकमुच्यते~॥} इति~।
\end{quote}

\noindent
तच्चेदमसत्, {\qt लास्ये दशविधं} ( १९ \textendash\ १२० ) इत्यत्रत्येन ग्रन्थेन, {\qt दशाङ्गं लास्य}मिति च तालाध्याये ( ३१ ) पठिष्यमाणेन विरोधात्, न चास्योपयोगः कश्चित्~। तथा ह्यलौकिककैशिक्युपयोगि रसांशे सर्वथोपकारि यद्वैचित्र्यं तल्लास्याङ्गद्वारेणाह, स्वकार्यं तच्च सर्वं दशभिरेव संगृहीतम्~। तथा हि \textendash\ प्रधाने चित्तवृत्त्यंशे\renewcommand{\thefootnote}{$\dagger$}\footnote{{\qt चित्तवृत्त्यनुवृत्यंशे} इति स्यात्} वा वैचित्र्यं विभावाद्यंशे वा उपरञ्जकभावे वा~। तत्र ( चित्तवृत्तयंशे प्रच्छेदकाद्\renewcommand{\thefootnote}{$\ddagger$}\footnote{तत्रानुत्तमोत्तमकात्} ) विभावांशगतं तु वैचित्र्यं सैन्धवकात् काक्वाद्यंशे, स्थितपाठ्याल्लक्षणगुणाद्यंशे, त्रिमूढकादनुवृत्तांशे, पुष्पगण्डिकात् आहार्ये उपरञ्जकगीतातोद्ययोजने च, सात्त्विके आसीनपाठ्यात्, उपरञ्जकभावेऽपि ( निः )शब्दाद् ध्रुवागानभागे सर्वातोद्ययोगे च गेयपदात्, गीतार्थस्य पात्रे करुणार्थाभिनये उक्तप्रत्युक्ताद्वाचिकस्य, स्वरतालनुसरणाद् द्विमूढकात् तत एवाङ्गिकवैचित्र्यमपि, व्यभिचार्यंशे तु वैचित्र्यमुत्त \textendash

\newpage
% ७८ नाट्यशास्त्रम्

\begin{quote}
{\na \renewcommand{\thefootnote}{1}\footnote{ड \textendash\ एतद्वै}एतेषां लास्यविधौ \renewcommand{\thefootnote}{2}\footnote{ड \textendash\ लक्षणमुक्तं मयात्र विस्तरतः ( ढ \textendash\ सु )}विज्ञेयं लक्षणं प्रयोगज्ञैः~।}
\end{quote}

\hrule

\vspace{2mm}
\noindent
मोत्तमकात्~। एवं तदतिरिक्तो नास्त्येवांश इति च दशैवाङ्गानि~। वाग्भागस्तनुत्वेनोपरञ्जकभागः स्वरसात्मनि हि तद्वैचित्र्ये भूयसा मुनिना निरूपणं कृतम्~। अतएव चालौकिको वैचित्र्यांशः प्रत्युतरसे सातिशयोपयोग इतिशब्दतो मुनेरनुकारो, रसात्मा नाट्यमिति ( इति ) नाभिप्रेतमिति लक्ष्यते~। अलौकिकवैचित्र्यसारो हि रसः, तथा चोक्तं भट्टतोतेन\textendash

\begin{quote}
{\qt लक्षणालङ्कृतिगुणा दोषाः शब्दप्रवृत्तयः~।\\
वृत्तिसन्ध्यङ्गसंरम्भः संभारो यः कवेः किल~॥

अन्योन्यस्यानुकूल्येन संभूयैव समुत्थितैः~।\\
झटित्येव रसा यत्र व्यज्यन्ते ह्लादिभिगुणाः ( णैः ? )~॥

वृत्तैः सरल्लबन्धैर्यत् स्निग्धैश्चूर्णपदैरिपि~।\\
अश्लिष्टहृद्यघटनं भाषया सुप्रसिद्धया~॥

यच्चेदृक्काव्यमात्रं सद्रसभावानुभावकम्~।\\
सामान्याभिनये प्रोक्तं वाच्याभिनयसंज्ञया ( अ २२ )~॥

एवंप्रकारं यत्किंचिद्वस्तुजातं ( कथार्पितम् )~।\\
अनूनाधिकसामग्रीपरिणामोन्मिषद्रसम्~॥

\renewcommand{\thefootnote}{*}\footnote{\begin{quote}
{\qt रसपोषाय तज्जातं लोकान्नाट्यजगत्स्वयम्~।\\
प्रतिभायाः प्रगल्भायाः सर्वस्वं कविवेधसः~॥}
\end{quote}
इति भ्रष्टनष्टाक्षरप्रायस्यास्य श्लोकस्यायमभिप्रायः स्यात्}रसपोषाय तज्जाते नाट्यस्वयं स्वरो \ldots~।\\
प्रतिभायाः प्रगल्भायाः सर्वमोडुप एष सः~॥}
\end{quote}

\noindent
इति संभावनाप्राणनाया हि यल्लोके संभाव्यते परम् अथ ( परमार्थं तत् ?~। ) वस्तुतो लोकोत्तरत्वेनैव संभारेण युक्ता कविवाणी हठादेव रसमयी भवति साधारणताप्राणत्वादिति तत्र तात्पर्यम्~।\\

ननु लास्याङ्गेभ्यो यो भागानुपजीवति स तावदिहोक्तः, तेषां तु स्वरूपं वक्तव्यमित्याशंक्याह \underline{एतेषां लास्यविधाविति}~। एतद् दशरूपकलक्षणं

\newpage
% एकोनविंशोऽध्यायः ७९

\begin{quote}
{\na तदिहैव तु यन्नोक्तं प्रसङ्गविनिवृत्तहेतोस्तु\renewcommand{\thefootnote}{1}\footnote{न \textendash\ हेत्वर्थम्}~॥~१३८

\renewcommand{\thefootnote}{2}\footnote{जादिबा न्तेषु सामाद्येकर्विशतिसन्ध्यन्तरानन्तरमर्थोपक्षेपकाः तेषामनन्तरं \textendash\ {\qt वृत्तिवृत्त्यङ्गसंपन्नं पताकार्थप्रतिक्रियम् ( य \textendash\ पदार्थप्रकृति न पञ्चार्थप्रकृति )~। पञ्चावस्था समुत्पन्नं ( न \textendash\ विनिष्पन्नं ) पञ्चभिः सन्धिभिर्युतम्~। षट्त्रिंशल्लक्षणोपेतं \ldots. } इति वर्तते~।}पञ्चसन्धि चतुर्वृत्ति चतुःषष्ट्यङ्ग संयुतम्~।\\
षट्त्रिंशल्लक्षणोपेतं गुणालङ्कारभूषितम्~॥~१३९

महारसं महा\renewcommand{\thefootnote}{3}\footnote{भ \textendash\ योगमुदात्तवचनोद्भवम्}भोगमुदात्तवचनान्वितम्~।\\
महापुरुषसंचारं साध्वाचार\renewcommand{\thefootnote}{4}\footnote{ड \textendash\ चारं}जनप्रियम्~॥~१४०

सुश्लिष्टसन्धिसंयोगं\renewcommand{\thefootnote}{5}\footnote{ड \textendash\ योगं च} सुप्रयोगं सुखाश्रयम्~।\\
मुदुशब्दाभिधानं च कविः कुर्यात्तु नाटकम्~॥~१४१

अवस्था या तु\renewcommand{\thefootnote}{6}\footnote{द \textendash\ हि} लोकस्य सुखदुःखसमुद्भवा~।\\
नानापुरुषसंचारा नाटकेऽसौ विधीयते\renewcommand{\thefootnote}{7}\footnote{ड \textendash\ संभवेदिह, भ \textendash\ नाटकेषु क्रिया भवेत्}~॥~१४२}
\end{quote}

\hrule

\vspace{2mm}
\noindent
कवीनां सुखग्रहणाय राशीकर्तुमाह \underline{पञ्चसन्धी}त्यादि~। नाटकमित्यभिनेयकाव्यमात्रम्~। तत्र पञ्चसन्धीत्यादि संभवमात्रे मन्तव्यम्~। \underline{महा}न्तो \underline{रसाः} पुरुषार्थोपयोगिनः यत्र~। \underline{महान् भोगो} रञ्जनाप्रधानो रसो यत्र आभोग इत्यन्ये~। \underline{उदात्तवचनान्वितमिति} गुणान् श्लेषप्रसादादीन् स्वीकृरुते~। \underline{सुश्लिष्टसन्धिसंयो}गमिति सन्ध्यन्तराणि ( च ), \underline{सुप्रयोगमिति} लास्याङ्गानि, \underline{सुखाश्रय}मिति छन्दोवृत्तवैचित्र्यं, मृदुशब्दैरभिधानं वर्णना विवक्षितस्यार्थस्य यत्रेति माधुर्यप्रसादार्थव्यक्तिगुणानां प्रकर्षणं सूचयति~। एवं नाटकं कुर्यात्~। यश्च नाटकं कुर्यात् स एव कविः~। किं तेनेत्याह \underline{अवस्था} यात्विति

\newpage
% ८० नाट्यशास्त्रम्

\begin{quote}
{\na न तज्ज्ञानं न तच्छिल्पं न सा विद्या न सा कला~।\\
न तत् कर्म न \renewcommand{\thefootnote}{1}\footnote{ड \textendash\ योगोऽसौ}वा योगो नाट्येऽस्मिन्यन्न दृश्यते~॥~१४३

योऽयं स्वभावो लोकस्य नानावस्थान्तरात्मकः~।\\
\renewcommand{\thefootnote}{2}\footnote{ड \textendash\ सोऽङ्गाभिनयसंयुक्तो}सोऽङ्गाद्यभिनयैर्युक्त \renewcommand{\thefootnote}{3}\footnote{न \textendash\ नाटकं त्वभिधी \textendash\ यते य \textendash\ नाटके संविधीयते}नाट्यमित्यभिधीयते~॥~१४४

देवतानामृषीणां च राज्ञां चोत्कृष्टमेधसाम्\renewcommand{\thefootnote}{4}\footnote{ड \textendash\ लोकस्य चैव हि, द \textendash\ अथ कुटुम्बिनाम्}~।\\
\renewcommand{\thefootnote}{5}\footnote{द \textendash\ वृत्तानुकरणं नाट्यमेष (?) लोकस्म चैव यत् भ \textendash\ कृतानुकरणं लोके नाट्य \textendash\ भित्यभिधीयते}पूर्ववृत्तानुचरितं नाटकं नाम तद्भवेत्~॥~१४५

यस्मात् स्वभावं संत्यज्य\renewcommand{\thefootnote}{6}\footnote{ड \textendash\ संदृत्य} साङ्गोपाङ्गतिक्रमैः~।\\
\renewcommand{\thefootnote}{7}\footnote{ड \textendash\ अभिनीयते गम्यते भ \textendash\ अभिनीयते गम्यते}प्रयुज्यते ज्ञायते च तस्माद्वै नाटकं स्मृतम्~॥~१४६

सर्वभावैः सर्वरसैः सर्वकर्मप्रवृत्तिभिः\renewcommand{\thefootnote}{8}\footnote{द \textendash\ क्रियास्तु वै भ \textendash\ क्रियासु च}~।\\
नानावस्थान्तरोपेतं\renewcommand{\thefootnote}{9}\footnote{ड \textendash\ उपेतैः.} नाटकं संविधीयते~॥~१४७}
\end{quote}

\hrule

\vspace{2mm}
\noindent
नानापुरुषेषु सञ्चार एको भावानुप्रवेशो रसात्मना यस्याः सा तादृशी यतो नाटके विधीयते संपाद्यते रसरूपतां नीयत इत्यर्थः~। उक्तं चैतद्वितत्य~।\\

ननु सर्वे चात्र किमवस्थाना इत्याशंक्य प्रथमाध्यायोक्तमेव श्लोकद्वयं पठति \underline{न तद्ज्ञान}मिति, \underline{योऽयं स्वभावो लोकस्येति}~। एतच्च तत्रैव व्याख्यात \textendash\ मिति वकिं पुनरुक्तेन~। पूर्ववृत्तानुचरितं कथं नाटकशब्दस्यार्थ इत्याह \underline{यस्मात्स्वभावं} \underline{संतज्येति} नट नताविति नमनं स्वभावत्यागेन प्रह्वीभावलक्षणं, ये त्वन्ये नट वृत्ताविति पठन्ति तन्मतेऽपीह नमनं, नटशब्दो जनिदाच्युसूत्रेण

\newpage
% एकोनविंशोऽध्यायः ८१ 

\begin{quote}
{\na [\renewcommand{\thefootnote}{1}\footnote{ड \textendash\ यान्येव य \textendash\ यान्येक}अनेकशिल्पजातानि \renewcommand{\thefootnote}{2}\footnote{ड \textendash\ ह्येककर्मकृतानि च (य \textendash\ रूप)}नैककर्मक्रियाणि च~।\\
\renewcommand{\thefootnote}{3}\footnote{3 ड \textendash\ तानि शेषाणि}तान्यशेषाणि रूपाणि कर्तव्यानि प्रयोक्तृभिः~॥]~१४८~॥

लोकस्वभावं संप्रेक्ष्य नराणां च बलाबलम्~।\\
संभोगं चैव युक्तिं च ततः कार्यं तु\renewcommand{\thefootnote}{4}\footnote{य \textendash\ च} नाटकम्~॥~१४९~॥

\renewcommand{\thefootnote}{5}\footnote{य \textendash\ लोके प्रणश्यति, द \textendash\ लोको हसिष्यति}भविष्यति युगे प्रायो भविष्यन्त्यबुधा नराः~।\\
ये चापि हि भविष्यन्ति \renewcommand{\thefootnote}{6}\footnote{भ \textendash\ ते यत्नार्जित ड \textendash\ तेऽत्यल्प द \textendash\ तेऽप्यल्प}ते यत्नश्रुतबुद्धयः~।~१५०~॥

\renewcommand{\thefootnote}{7}\footnote{ड \textendash\ बुद्धयः कर्मशिल्पानि वैचक्षण्यं कलासु च य \textendash\ कर्मशिल्पानि वैलक्ष्यं लक्षणं च बलानि च }कर्मशिल्पानि शास्त्राणि विचक्षणबलानि च~।\\
\renewcommand{\thefootnote}{8}\footnote{ड \textendash\ सर्वाणि पुंसां}सर्वाण्येतानि नश्यन्ति \renewcommand{\thefootnote}{9}\footnote{न \textendash\ सदा, भ \textendash\ यथा}यदा लोकः प्रणश्यति~॥~१५९~॥

\renewcommand{\thefootnote}{10}\footnote{भ \textendash\ एवं लोकस्य वै भावमभावं प्रसमीक्ष्य च}तदेवं लोकभाषाणां\renewcommand{\thefootnote}{11}\footnote{ड \textendash\ भावानां} प्रसमीक्ष्य बलाबलम्~।\renewcommand{\thefootnote}{12}\footnote{य \textendash\ यथाक्रमम्} \\
मृदुशब्दं सुखार्थं च \renewcommand{\thefootnote}{13}\footnote{य \textendash\ तज्ज्ञैः कार्यं तु}कविः कुर्यात्तु नाटकम्~॥~१५२~॥

चैक्रीडिताद्यैः शब्दैस्तु काव्यबन्धा भवन्ति ये~।\\
वेश्या इव न \renewcommand{\thefootnote}{14}\footnote{ड \textendash\ शोभन्ते}ते भान्ति कमण्डलुधरैर्द्विजैः~॥~१५३~॥}
\end{quote}

\hrule

\vspace{2mm}
\noindent
(उणादि ४ \textendash\ ११५)व्युत्पादितो गृहीतव्य इति दर्शयति साङ्गोपाङ्गा ये पदक्रमा गतिर्वैचित्र्याणि~। एतच्च समस्तं नाट्याङ्गोपलक्षणं प्रयुज्यत इति नटैज्ञायते चेति सामाजिकैस्तेनोभयोरपि नमनमुक्तमिति संभावनाकृतमौचित्यम्~। \underline{मृदुशब्देति}

\lfoot{11}

\newpage
% ८२ नाट्यशास्त्रम् 
\lfoot{}

\begin{quote}
{\na \renewcommand{\thefootnote}{1}\footnote{ड \textendash\ इतिवृत्तं ससन्ध्यङ्गं, भ \textendash\ अङ्गलक्षणमेतत्तु}दशरूपविधानं च मया प्रोक्तं द्विजोत्तमाः~।\\
अतः परं\renewcommand{\thefootnote}{2}\footnote{भ \textendash\ ऊर्ध्वं} प्रवक्ष्यामि वृत्तीनामिह\renewcommand{\thefootnote}{3}\footnote{भ \textendash\ अपि} लक्षणम्~॥

इति भारतीये नाट्यशास्त्रे \renewcommand{\thefootnote}{4}\footnote{भ \textendash\ वागभिनयेऽङ्गविकल्पो नामाध्यायोऽष्टादशः}सन्धिनिरूपणं \\
नामाध्याय एकोनविंशः\renewcommand{\thefootnote}{5}\footnote{जादि बान्तेषु दटौ विना एकविंशः दट \textendash\ विंशः}~॥}
\end{quote}

\hrule

\vspace{2mm}
\noindent
यदुक्तं तस्य प्रयोजनमाह \underline{भविष्यति} त्रेतायुगापेक्षया द्वापरे कलौ वेत्यर्थः~। \underline{सुखार्थ}मित्यर्थव्यक्तिः स्वीकृता~। एतदुपसंहरन्नन्यदासूत्रयन्नध्यायमध्यायान्त रेण सङ्गमयति \underline{दशरूपविधानं} चेति शिवम्~।

\begin{quote}
{\qt द्विजवरतोतनिरूपितसन्ध्यध्यायार्थतत्त्वघटनेयम्~।\\
अभिनवगुप्तेन कृता शिवचरणाम्भोजमधुपेन~॥}
\end{quote}

\begin{center}
\textbf{इति श्रीमहामाहेश्वराभिनवगुप्ताचार्यविरचितायां}

\textbf{नाट्यवेदविवृतावभिनवभारत्यामेकोनविंशः}

\textbf{सन्ध्यध्यायः समाप्तः~॥\renewcommand{\thefootnote}{*}\footnote{क \textendash\ संज्ञक आदर्शे {\qt विवृतिर्भरतस्येयं लिखिता भारती नवा~। नारायणेन विदुषः शितिकण्ठद्विजन्मनः~॥} इति लेखकनामोल्लिखितम्~।}}

\vspace{1.5cm}
\rule{0.2\linewidth}{0.5pt}
\end{center}

\newpage
\thispagestyle{empty}
\begin{center}
\textbf{\large श्रीः}

\textbf{\LARGE नाट्यशास्त्रम्}

विंशोऽध्यायः\renewcommand{\thefootnote}{1}\footnote{जादिबान्तेषु द्वाविंशोऽध्यायः, चयट \textendash\ एकविंशोऽध्यायः, भ \textendash\ एकोन \textendash\ विंशोऽध्यायः}\\

\rule{0.2\linewidth}{0.5pt}
\end{center}

\begin{quote}
{\na समुत्थानं तु वृत्तीनां व्याख्यास्याम्यनुपूर्वशः~।\\
यथा वस्तूद्भवं चैव काव्यानां च\renewcommand{\thefootnote}{2}\footnote{ढ \textendash\ तु} विकल्पनम्~॥~१}
\end{quote}

\begin{center}
अभिनवभारती \textendash\ विंशोऽध्यायः

वृत्त्यध्यायः
\end{center}

\begin{quote}
{\qt निश्शेषशब्दव्यवहारवृत्ति\renewcommand{\thefootnote}{1}\footnote{वृत्त}वैचित्र्यमभ्येति यतः प्रतिष्ठाम्~।\\
श्रोत्रात्मकं तत्परमेश्वरस्य वन्देतमां रूपमरूपधाम्नः~॥}
\end{quote}

वृत्तिभेदात्काव्यभेदा भवन्तीत्युक्तं दशरूपकारम्भे (१८ \textendash\ ५)~। तत्र वृत्तयो न ज्ञाता इति तदवधारणार्थमाह \underline{समुत्थानं त्विति}~।\\

यद्यपि कायवाङ्मनसां चेष्टा एव सह वैचित्र्येण वृत्तयः ताश्च समस्त \textendash\ जीवलोकव्यापिन्योऽनिदंप्रथमताप्रवृत्ताः प्रवाहेन वहन्ति, तथापि विशिष्टेन हृदयावेशेन युक्ता वृत्तयो नाट्योपकारिण्यः~। आवेशश्च तारतम्यलक्षणो द्विधा लौकिकोऽन्यश्च~। तत्र लैकिक आवेशः सुखदुःखतारतम्यकृतो न रसाग \textendash\ मास्वाद्यो\renewcommand{\thefootnote}{2}\footnote{रसागमनास्वाद्यो} ह्यसावित्युक्तं रसाध्याये (अ ६)~। अलौकिकस्त्वनावेशोऽप्यावेशमयः कवेरिव सामाजिकस्येव~। क्वाप्यवसरे हृदयसंवादसरसस्यैव यो भासते स

\newpage
% ८४ नाट्यशास्त्रम् 

\begin{quote}
{\na एकार्णवं जगत् कृत्वा भगवानच्युतो यदा\renewcommand{\thefootnote}{1}\footnote{ढ \textendash\ यथा}~।\\
शेते स्म \renewcommand{\thefootnote}{2}\footnote{च \textendash\ अस्य (?)}नागपर्यङ्के लोकान्\renewcommand{\thefootnote}{3}\footnote{ड \textendash\ लोकं} संक्षिप्यं\renewcommand{\thefootnote}{4}\footnote{च \textendash\ संक्षेप्य} मायया~॥~२

अथ वीर्यं\renewcommand{\thefootnote}{5}\footnote{ड \textendash\ मदोन्मत्तौ} बलोन्मत्तावसुरौ मधुकैटभौ~।\\
\renewcommand{\thefootnote}{6}\footnote{ढ \textendash\ तर्कयामासतुः}तर्जयामासतुर्देवं तरसा युद्धकाङ्क्षया\renewcommand{\thefootnote}{7}\footnote{न \textendash\ युद्धकांक्षिणौ}~॥~३

\renewcommand{\thefootnote}{8}\footnote{ड \textendash\ बाहू विमर्दमानौ तौ न \textendash\ निजबाहू विमृश्नन्तौ भ \textendash\ बाहूनि तर्जयन्तौ तावङ्गहारैस्तथाहरिम्~। मुष्टि \textendash\ भिर्जानुभिश्चैव ताडयामासतुः प्रभुम्~। च \textendash\ निजबाहू विवृद्धन्तौ (?)}निजबाहू विमृद्नन्तौ भूतभावनमक्षयम्~।\\
जानुभिर्मुष्टिभिश्चैव \renewcommand{\thefootnote}{9}\footnote{ढ \textendash\ योज \textendash\ यामासतुः}योधयामासतुः प्रभुम्~॥~४

\renewcommand{\thefootnote}{10}\footnote{ड \textendash\ अभिद्रवन्तावन्योन्यं वाक्यैश्च परुषैस्तथा}बहुभिः परुषैर्वाक्यैरन्योन्यसमभिद्रवम्\renewcommand{\thefootnote}{11}\footnote{च \textendash\ स्पर्धया \textendash\ न्वितौ}~।\\
\renewcommand{\thefootnote}{12}\footnote{च \textendash\ नामविक्षेप भ \textendash\ नानाविक्षेप भ \textendash\ नानाविच्छेद}नानाधिक्षेपवचनैः कम्पयन्ताविवोदधिम्\renewcommand{\thefootnote}{13}\footnote{भ \textendash\ अम्बुधिम्}~॥~५}
\end{quote}

\hrule

\vspace{2mm}
\noindent
एव साधारणे चमत्कारगोचर\renewcommand{\thefootnote}{1}\footnote{क \textendash\ चमत्कारकारी ख \textendash\ चमत्कार गोचरचारी}व्यापारविशेषो रसस्योपकरणीभवति~। तादृशश्च प्रथमतः कृतयुगारम्भे भगवतो वासुदेवस्यैव~। तस्य हि स्वफलसिद्धये न किंचित् कर्तव्यमस्ति लोकानुग्रहं मुक्त्वा\renewcommand{\thefootnote}{2}\footnote{मत्वा}~। यथा हि {\qt न मे पार्थास्ति\renewcommand{\thefootnote}{3}\footnote{यथा ह्यनृ \textendash\ तमेवर्थमस्ति} कर्तव्यं (गीता ३ \textendash\ २२)} इति तेन साधारणस्य भावेन प्रविष्टानावेशेऽप्यावेश \textendash\ मयो, भगवानेव प्रथमतो नान्यः, पूर्वसर्गगामिनो व्यापारस्य प्रलयमहारजनी \textendash\ प्रस्तावेन भ्रष्टसंसारत्वात्~। अत एवाह \textendash

\begin{quote}
{\qt एकार्णवं जगत्कृत्वा भगवानच्युतो यदा~।\\
शेते स्म नागपर्यङ्के लोकान् संक्षिप्य मायया~॥} इति~।
\end{quote}

\newpage
\fancyhead[CO]{विंशोऽध्यायः}
% विंशोऽध्यायः ८५

\begin{quote}
{\na \renewcommand{\thefootnote}{1}\footnote{भ \textendash\ तयोर्नैक न \textendash\ तयोरनेकरूपाणि}तयोर्नानाप्रकाराणि वचांसि वदतोस्तदा\renewcommand{\thefootnote}{2}\footnote{च \textendash\ श्रुत्वा वाक्यानि तर्जतोः, भ \textendash\ गदतोस्तदा}~।\\
\renewcommand{\thefootnote}{3}\footnote{च \textendash\ किञ्चिदाकम्पित भ \textendash\ श्रुत्वा त्वाकम्पित ज \textendash\ किञ्चि \textendash\ दाकुञ्चित}श्रुत्वा त्वभिहतमना द्रुहिणो वाक्यमब्रवीत्~॥~६

\renewcommand{\thefootnote}{4}\footnote{न \textendash\ किमियं}किमिदं भारतीवृत्तिर्वाग्भिरेव प्रवर्तते\renewcommand{\thefootnote}{5}\footnote{न \textendash\ प्रवर्तिता न \textendash\ प्रतिष्ठिता भ \textendash\ एवं प्रवर्तते}~।\\
उत्तरोत्तरसंबद्धा\renewcommand{\thefootnote}{6}\footnote{न \textendash\ सम्बन्धा} नन्विमौ निधनं नय~॥~७

पितामहवचः श्रुत्वा प्रोवाच मधुसूदनः~।}
\end{quote}

\hrule

\vspace{2mm}
त्रेतायुगे च नान्यं प्रपत्स्यति~। यद्यपि चादिकविर्वाल्मीकिरि\renewcommand{\thefootnote}{1}\footnote{लब्ध}ति प्रवाद \textendash\ स्तथापि प्रलयानन्तरं कृतयुगे भाविनि सन्ध्यवसरप्रतिबन्धदर्पदलनजनित \textendash\ विघ्नापसरण\renewcommand{\thefootnote}{2}\footnote{सरणं}पुरस्सरसृष्टिसम्पत्तौ चिरेण वाल्मीकिमुनेः प्रादुर्भावो भवि \textendash\ ष्यतीति साधारण्याशयेनैव च भगवानेव वृत्तीनां\renewcommand{\thefootnote}{3}\footnote{क \textendash\ {\qt वृत्तीनां} इति नास्ति} स्रष्टोक्तः न तौ तु मधुकै \textendash\ टभौ, तयोरावेशेन लौकिकेन व्याप्तत्वात्\renewcommand{\thefootnote}{4}\footnote{व्यस्तत्वात्}, उद्रिक्ततमतमस्सन्तानितस्वांतत्वेना \textendash\ विद्यामयत्वात्~। विद्यावकाशवशविकसितहृदयकमलपरिमलरूपत्वाच्च, आनन्द \textendash\ सारस्य रसोपयोगिनो व्यापारानावेशस्य भगवत्येव सम्भावनं न तु दैत्ययोः~। अतएवाह \underline{किमिदं भारतीवृत्ति}र्भगवन् वाग्भिरेव प्रवर्तते इति भगवतेत्यनेन ह्युक्तं, भगवानेव हि वृत्तीनां प्रवर्तयिता नटवदनावेशदर्शनात्, न तु मधुकै \textendash\ टभौ (तयोः) लौकिके चैव स्यादिति~। \underline{द्रुहिणोऽभिहतमना वाक्यमब्र} \textendash\ वीदित्यनेन प्रेक्षकत्वं ब्रह्मणः प्रदर्शयन् सामाजिकहृदये विश्रान्तिप्राधान्यं वृत्तीनामाह~। तन्निष्ठत्वाद्रसचर्वणाया वाक्यग्रहणेन कार्यद्वारको वृत्तीनां नाट्योपयोग इति सूचयति \underline{वाग्भिरेवेति}~। वाचिकप्राधान्ये भारतीवृत्तिः~। भारतीशब्देन हि वागुच्यत इति कार्यहेतोरित्युक्तं तत्कार्यं दर्शयति

\newpage
% ८६ नाव्यशास्त्रम् 

\begin{quote}
{\na \renewcommand{\thefootnote}{1}\footnote{भ \textendash\ बाढं नाट्यक्रियाहेतोः}कार्यहेतोर्मया ब्रह्मन् भारतीयं विनिर्मिता~॥~८

\renewcommand{\thefootnote}{2}\footnote{भ \textendash\ भाषतां, ड \textendash\ भाषतोः}वदतां वाक्यभूयिष्ठा भारतीयं भविष्यति~।\\
\renewcommand{\thefootnote}{3}\footnote{ड \textendash\ अहमेतौ निहन्म्यद्य इत्युक्त्वा (न \textendash\ अयमेतौ)}तस्मादेतौ निहन्म्यद्येत्युवाच वचनं हरिः~।~९

\renewcommand{\thefootnote}{4}\footnote{शून्यैरिति पाठः स्यात्}शुद्धैरविकृतैरङ्गैः साङ्गहारैस्तथा\renewcommand{\thefootnote}{5}\footnote{ड \textendash\ तदा भ \textendash\ तथैव च} भृशम्~।\\
योधयामासतुर्दैत्यौ \renewcommand{\thefootnote}{6}\footnote{न \textendash\ बाहुयुद्ध}युद्धमार्गविशारदौ~।~१०

भूमिसंयोगसंस्थानैः\renewcommand{\thefootnote}{7}\footnote{ड \textendash\ भूतिसंस्थानसंयोगैः (न \textendash\ भूमि), भ \textendash\ भूसंस्थानैः प्रयोगैश्च}पदन्यासैर्हरेस्तदा\renewcommand{\thefootnote}{8}\footnote{ढ \textendash\ सदा हरेः च \textendash\ हरेस्ततः}~।\\
अतिभारोऽभवद् भूमेर्भारती तत्र\renewcommand{\thefootnote}{9}\footnote{न \textendash\ तेन} निर्मिता~॥~११

वल्गितैः \renewcommand{\thefootnote}{10}\footnote{भ \textendash\ शार्ङ्गपाणेस्तु}शार्ङ्गधनुष\renewcommand{\thefootnote}{11}\footnote{न \textendash\ तीव्रदीप्ततरैः (ट \textendash\ करैः, ढ \textendash\ धरैः)}स्तीव्रैर्दीप्ततरैरथ~।\\
\renewcommand{\thefootnote}{12}\footnote{भ \textendash\ सत्त्वाधिकाततभ्रान्तैः}सत्त्वाधिकैरसंभ्रान्तैः सात्त्वती तत्र निर्मिता\renewcommand{\thefootnote}{13}\footnote{ट \textendash\ च विनिर्मिता}~॥~१२}
\end{quote}

\hrule

\vspace{2mm}
\noindent
\underline{वदता}मिति कवीनामिति यावत्~। \renewcommand{\thefootnote}{1}\footnote{शून्यैरिति}शुद्धैरिति स्वोत्प्रेक्षितवैचित्र्यशून्यैः~। अङ्गानां हारान्नयतामिति भृशं युद्धमार्गे विशारदाविति संबन्धः~। भूमि \textendash\ संयोगेन संस्थान उद्धटि(ट्टि! )तसङ्कुचितादिरूपं येषां, पदन्यासानां न्यस्यमा \textendash\ नानां पदानाम्~। \underline{अतिभार} इत्यक्षरसाम्यादपि निर्वचनं दर्शयन् नटव्यापा \textendash\ रयोगेऽपि आन्तरविकल्पात्मकाभिः (वाग्भिः)संजुल्पबाहुल्ये भारत्येव वृत्तिरिति दर्शयति~। \underline{सत्त्वाधि}कैरिति मनोव्यापाराधिक्ये सात्त्वतीवृत्तिरित्याह सत् \textendash\ सत्त्वरूपं विद्यते येषां तत्त्वं तेषामयमिति~। सत्त्वं च तत्र परच्छिद्रान्वेषणोपा \textendash

\newpage
% विंशोऽध्यायः ८७ 

\begin{quote}
{\na \renewcommand{\thefootnote}{1}\footnote{ट \textendash\ विविधैः}विचित्रैरङ्गहारैस्तु देवो \renewcommand{\thefootnote}{2}\footnote{न \textendash\ हेलासमुत्थितैः}लीलासमन्वितैः\renewcommand{\thefootnote}{3}\footnote{भ \textendash\ समुद्भवैः}~।\\
बबन्ध \renewcommand{\thefootnote}{4}\footnote{प \textendash\ यः}यच्छिखापाशं कैशिकी तत्र निर्मिता~॥

संरम्भावेगबहुलैर्नानाचारी\renewcommand{\thefootnote}{5}\footnote{भ \textendash\ नानाधार}समुत्थितैः~।\\
नियुद्धकरणैश्चित्रै\renewcommand{\thefootnote}{6}\footnote{न \textendash\ अङ्गै भ \textendash\ स्पष्टैर्निर्मिता}रुत्पन्नारभटी ततः~॥~१४

यां यां देवः समाचष्टे क्रियां वृत्तिषु संस्थिताम्\renewcommand{\thefootnote}{7}\footnote{न \textendash\ वृत्तिसमुद्भवाम्, भ \textendash\ बुद्धिसमुत्थिताम्}~।}
\end{quote}

\hrule

\vspace{2mm}
\noindent
यमप्रतितानवैचित्र्योत्प्रेक्षणप्रकाशलाघवात्मकं, अत एवाह \underline{विचित्रैरिति}~। स्वोत्प्रेक्षितेन हि वैचित्र्येण विना कथं युद्धविशारदौ जय्यौ स्याताम्, द्वयो \textendash\ र्ह्येकागमानुसन्धानेन युध्यमानयोः सम्बुद्धिकयोः\renewcommand{\thefootnote}{1}\footnote{संबन्धिकयोः} कस्य जयः कस्य पराजयो वा भवेदिति, शिखापाशं पट्टबन्धम्, तुर्हेतौ श्रृङ्गारस्थानलक्ष्मीसम्भोगादि \textendash\ स्मरणात्, लीलया विलासेन अन्वयः शिखापाशमयेन आबध्योदासीनो वर्तत इति~। चारीभिः प्रायेण निश्शङ्कं प्रवर्तमानौ दैत्यौ सुखं निपात्येते मय्येव परिश्रमापनयनं सम्पद्यत इति~। \underline{देव} इति लोकवदनाविष्टः~। \underline{यां यामिति} वीप्सया सर्वैव क्रिया वृत्तिचतुष्टयव्याप्तेत्याह~। न हि\renewcommand{\thefootnote}{2}\footnote{तर्हि} वाङ्मनश्चेष्टातोऽतिरि \textendash\ क्तकर्मास्ति\renewcommand{\thefootnote}{3}\footnote{चेष्टातोमुक्तकर्षास्ति}~। न च क्रियाशून्यः कश्चिदप्यंशोऽस्ति~। मूर्छामोहमरणादावपि सूक्ष्मप्राणपरिस्पन्दाद्यनुमेयसञ्चितपरिस्पन्दसम्भावना~। मृतस्तु ताम्र \textendash\ पाषाणप्रख्यो न तस्य वृत्तिकथनेन किंचित्स्वपरम्, अन्यस्य शोकादिविभा \textendash\ वनां प्रतिपाद्यमानः काव्याङ्गतामेति स च तदानीं करुणादिरसाक्रान्तः काव्ये व्यावर्णनीयो भवति~। रसभावपर्यवसितो हि सर्वः कार्य\renewcommand{\thefootnote}{4}\footnote{काव्य}संदर्भः~। रसभा \textendash\ वाश्च चेतनेष्वेव, तेषु च न\renewcommand{\thefootnote}{5}\footnote{तेषुवचन} व्यापारत्रयशून्यः\renewcommand{\thefootnote}{6}\footnote{कस्मिन् श्चित्,}कश्चिदपि \renewcommand{\thefootnote}{7}\footnote{कामांशो, कालांशो}काव्यांशोऽस्ति~।

\newpage
% ८८ नाट्यशास्त्रम् 

\begin{quote}
{\na तां तदर्थानुगै\renewcommand{\thefootnote}{1}\footnote{ड \textendash\ वाक्यैः}र्जप्यैर्द्रुहिणः प्रत्यपूजयत्~॥~१५

यदा हतौ तावसुरौ हरिणा मधुकैटभौ~।\\
\renewcommand{\thefootnote}{2}\footnote{ड \textendash\ उक्तवांस्तु तदा ब्रह्या}ततोऽब्रवीत् पद्मयोनिर्नारायणमरिन्दमम्~॥~१६

अहो\renewcommand{\thefootnote}{3}\footnote{ड \textendash\ चित्रैः सललितैः स्फुटैः सुविशदैः न \textendash\ विचित्रैर्ललितैः भ \textendash\ अङ्गैः विचित्रलिलितैः} विचित्रैर्विषमैः स्फुटैः सललितैरपि~।\\
अङ्गहारैः कृतं देव \renewcommand{\thefootnote}{4}\footnote{भ \textendash\ दान \textendash\ वानां विनाशनम्}त्वया दानवनाशनम्~॥~१७

तस्मादयं\renewcommand{\thefootnote}{5}\footnote{ड \textendash\ सर्वलोके} हि लोकस्य नियुद्धसमयक्रमः\renewcommand{\thefootnote}{6}\footnote{भ \textendash\ समयः शुभः}~।\\
सर्वशस्त्रविमोक्षेषु\renewcommand{\thefootnote}{7}\footnote{भ \textendash\ विमोक्षश्च.} न्यायसंज्ञो भविष्यति~॥~१८}
\end{quote}

\hrule

\vspace{2mm}
\noindent
तेन पञ्च वृत्तयो द्वे वृत्ती इत्यादयोऽसंविदितभरताभिप्रायपण्डितसहृदय \textendash\ म्मन्यपरिकल्पितसद्भावाः प्रवादा निरस्ता भवन्ति~। एतच्च पूर्वमेवास्माभिः प्रदर्शितम्~।\\

\underline{जप्यै}रिति भक्त्यावेशमात्रवृत्तिं च दर्शयन् कवेर्मुख्यं रसाधिष्ठत्वमेव रूपं विस्तारणात्मकवर्णनाप्राधान्यं चेति दर्शयति~। (\underline{विषमैः})\renewcommand{\thefootnote}{1}\footnote{संविदसौस्रागम} शस्त्रागममार्गात्मना \textendash\ गम्यप्रकारैः (अद्भु)तैरपि, तथा \underline{स्फुटै}रपि सर्वलक्षणप्रसिद्धैरपि चित्रैः स्वार्थोत्प्रेक्षितवैचित्र्यवद्भिः, \underline{सललितैरिति} लोचनगोचरवर्तित्वेनातिभ्रम्यै \textendash\ रित्यर्थः~। \underline{अय}\renewcommand{\thefootnote}{2}\footnote{अहं.}मिति भवदुपज्ञातवैचित्र्यबृंहित इत्यर्थः~। \underline{नियुद्धसमयक्रम} इति \underline{सम॒य}ग्रहणेन वैचित्र्ययोगेऽपि शस्त्रागमापरित्यागमाह~। क्रमः परिक्रमणं क्रमो वा शत्रुविक्रमणयोगे गतागतस्य निभृतं शस्त्रादिशून्यं युद्धं नियुद्धं मल्ल \textendash\ युद्धं, तद्गतिमपि यदेतद्रूपं तदुत्प्रेक्ष्यवैचित्र्ययोगात्सर्वेषु शस्त्रास्त्रयुद्धेष्वप्युपयोगि भविष्यति~। \underline{विमोक्षग्रहणात्} क्षेप्तव्यकुन्तचक्रादिविषयत्वमेव केचिदाहुः~।

\newpage
% विंशोऽध्यायः ८९ 

\begin{quote}
{\na \renewcommand{\thefootnote}{1}\footnote{भ \textendash\ न्यायकृतैः, न \textendash\ न्यायात् समुत्थितैश्चित्रैरङगहारैर्विभूषितम्}न्यायाश्रितैरङ्गहारैर्न्यायाच्चैव समुत्थितैः~।\\
यस्माद्युद्धानि वर्तन्ते\renewcommand{\thefootnote}{2}\footnote{च \textendash\ अवर्तन्त, ट \textendash\ युद्धकृतं त्वेभिः, ढ \textendash\ युद्धं कृतं ताभ्यां, न \textendash\ युद्धं कृतं ह्येतत्} तस्मान्न्यायाः प्रकीर्तिताः\renewcommand{\thefootnote}{3}\footnote{ड \textendash\ न्यायः प्रकीर्तितः}~॥~१९

\renewcommand{\thefootnote}{4}\footnote{भ \textendash\ संज्ञितादर्श एव वर्ततेऽयं श्लोकः}[चारीषु च समुत्पन्नो नानाचारीसमाश्रयः~।\\
न्यायसंज्ञः कृतो ह्येष द्रुहिणेन महात्मना~॥]~२०

\renewcommand{\thefootnote}{5}\footnote{चभमयेषु श्लोक \textendash\ द्वयं न विद्यते न \textendash\ ततो देवेषु}[ततो वेदेषु निक्षिप्ता द्रुहिणेन महात्मना~।\\
पुनरिष्वस्त्रजाते च नानाचारीसमाकुले\renewcommand{\thefootnote}{6}\footnote{ज \textendash\ समाश्रयं}~॥~२१

पुनर्नाट्यप्रयोगेषु\renewcommand{\thefootnote}{7}\footnote{प \textendash\ प्रयोगे च} नानाभावसमन्विताः\renewcommand{\thefootnote}{8}\footnote{8 प \textendash\ समाश्रया, ढ \textendash\ रसाश्रया}~।\\
वृत्तिसंज्ञाः कृता ह्येताः काव्यबन्धसमाश्रयाः\renewcommand{\thefootnote}{9}\footnote{ड \textendash\ संज्ञा\ldots कृता ह्येषा\ldots \ldots समाश्रया.}~॥]~२२}
\end{quote}

\hrule

\vspace{2mm}
\noindent
तच्चासत् ,क्षिपेः पातमात्रपर्यवसायिनः खङ्गादियुद्धेष्वप्यप्रतिहतार्थयुक्त \textendash\ त्वात्~। न्यायशब्दनिर्वचनमाङ्गिकाभिनयनिरूपणायां कृतं स्मारयति \underline{न्यायाश्रितैरिति} एतच्च तत्रैव व्याख्यातम्~।\renewcommand{\thefootnote}{*}\footnote{अध्या \textendash\ १० \textendash\ ७५ तत्तत्क्रियोचितानामङ्गानां युगपदेव प्रचारो नृत्तमातृका भवति~। द्वे तिस्रश्चतस्रो वा मातृका अत्रुटितत्वेनैकक्रियावत्कृताः करणं भवेयुः, करणानामत्रुटितप्रयोगोऽङ्गहारः~। तथैव युद्धादिष्वपि नृत्तमातृकास्थाने स्थानकादि \textendash\ क्रियाविशेषो न्यायसंज्ञः स्यात्~। न्यायशब्देनाङ्गोचितक्रियया परभञ्जनात्मरक्षणा \textendash\ ख्यविधिना च अभिद्रवतोर्योधयोः मिथःक्रियाविशेषो न्याय इति संज्ञितः~।}\\

\begin{sloppypar}
एवं वृत्तीनामुत्पत्तिर्व्याख्याता~। आसामेव नाट्यं प्रत्यवतारणं कर्तुमाह \underline{चरितैर्य}स्यैति भगवद्विश्वचरितैर्गर्भाधेयभूतैः उपलक्षितं यत् पूर्वं ब्रह्मणा ष्णु
\end{sloppypar}

\lfoot{12}

\newpage
\lfoot{}
% ९० नाट्यशास्त्रम् 

\begin{quote}
{\na \renewcommand{\thefootnote}{1}\footnote{न \textendash\ वल्गितैस्तस्य}चरितैर्यस्य देवस्य जप्यं\renewcommand{\thefootnote}{2}\footnote{ड \textendash\ द्रव्यं, भ \textendash\ दिव्यं} यद्यादृशं कृतम्~।\\
ऋषिभिस्तादृशी वृत्तिः कृता पाठ्यादिसंयुता\renewcommand{\thefootnote}{3}\footnote{च \textendash\ पाठ्याभिसंयुता प \textendash\ पाठ्याङ्गसम्भवा}~॥~२३

नाट्यवेदसमुत्पन्ना वागङ्गाभिनयात्मिका~।\\
मया काव्यक्रियाहेतोः प्रक्षिप्ता द्रुहिणाज्ञया~॥~२४

\renewcommand{\thefootnote}{4}\footnote{प \textendash\ ऋग्वेदे\ldots यजुर्वेदे च\ldots. सामवेदे च\ldots. अथर्वण्यार \textendash\ भट्यपि~।}ऋग्वेदाद्भारती क्षिप्ता\renewcommand{\thefootnote}{5}\footnote{च \textendash\ वृत्तिः} यजुर्वेदाच्च\renewcommand{\thefootnote}{6}\footnote{च \textendash\ तु} सात्वती~।\\
\renewcommand{\thefootnote}{7}\footnote{ढ \textendash\ अथर्वणस्त्वारभटी सामवेदाच्च कैशिकी}कैशिकी सामवेदाच्च शेषा चाथर्वणादपि\renewcommand{\thefootnote}{8}\footnote{म \textendash\ अथर्वादारभट्यपि च \textendash\ तथा}~॥~२५}
\end{quote}

\hrule

\vspace{2mm}
\noindent
\underline{जप्यं} स्तोत्रकाव्यरूपं कृतं, तादृशं भावादिचेष्टाप्रधानं, तादृश्येव वृत्तिः~। ऋषिभिः ब्रह्मपुत्रैः पाठ्यादिभिः सम्यग्युक्ता\renewcommand{\thefootnote}{1}\footnote{संयुक्ता} \underline{कृता} अनुसृता~। तद्यथा पाठ्यप्रधाना भारती, अभिनयप्रधाना सात्त्वती, अनुभावाद्यावेश[स]मय \textendash\ रसप्रधानारभटी, गीतवाद्योपरञ्जकप्रधाना कैशिकीति~। अत एव वक्ष्यति \underline{ऋग्वेदाद्भारतीत्यादि} नाट्यवेदोत्पत्तौ प्रथमाध्याये व्याख्याता या सा~। द्रुहिणाज्ञा कैशिकीमपि योजय (१ \textendash\ ४२) इत्यादिका या तया~। काव्यस्य क्रिया काव्यरूपतापादनं, तदेव हेतुः, ततः~। प्रकर्षेण क्षिप्ता येनाभिनेयानभिनेय \textendash\ काव्यवैलक्षण्यमित्यर्थः~।\\

ननु भगवता वासुदेवेनाप्यंशावताररूपत्वात्संभाव्य\renewcommand{\thefootnote}{*}\footnote{{\qt रूपत्वसंभाविताः काव्यक्रियोदये मुनिना} इति स्यात्} \ldots\ldots\ldots\ldots\ldots व्ययोदयमूर्तिना कुत इमा वृत्तय आनीता इत्याशङ्क्याह \underline{ऋग्वेदा}दित्यादि~। छन्दोमयश्च परमेश्वरो नान्यत\renewcommand{\thefootnote}{2}\footnote{नाट्यतः} इति भावः~।

\end{document}