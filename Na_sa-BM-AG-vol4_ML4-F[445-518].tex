\documentclass[11pt, openany]{book}
\usepackage[text={4.65in,7.45in}, centering, includefoot]{geometry}
\usepackage[table, x11names]{xcolor}
\usepackage{fontspec,realscripts}
\usepackage{polyglossia}
\setdefaultlanguage{sanskrit}
\setotherlanguage{english}
\setmainfont[Scale=0.9]{Shobhika}
\newfontfamily\s[Script=Devanagari, Scale=0.9]{Shobhika}
\newfontfamily\regular{Linux Libertine O}
\newfontfamily\en[Language=English, Script=Latin]{Linux Libertine O}
\newfontfamily\na[Script=Devanagari, Scale=0.9, Color=purple]{Shobhika-Bold}
\newfontfamily\qt[Script=Devanagari, Scale=0.9, Color=violet]{Shobhika-Regular}
\newfontfamily\qtt[Script=Devanagari, Scale=0.9, Color=violet]{Shobhika-Bold}
\newcommand{\devanagarinumeral}[1]{%
	\devanagaridigits{\number \csname c@#1\endcsname}} % for devanagari page numbers
%\usepackage[Devanagari, Latin]{ucharclasses}
%\setTransitionTo{Devanagari}{\s}
%\setTransitionFrom{Devanagari}{\regular}
\XeTeXgenerateactualtext=1 % for searchable pdf
\usepackage{enumerate}
\pagestyle{plain}
\usepackage{fancyhdr}
\pagestyle{fancy}
\renewcommand{\headrulewidth}{0pt}
\usepackage{afterpage}
\usepackage{multirow}
\usepackage{multicol}
\usepackage{wrapfig}
\usepackage{vwcol}
\usepackage{microtype}
\usepackage{amsmath,amsthm, amsfonts,amssymb}
\usepackage{mathtools}% < \textendash\ new package for rcases
\usepackage{graphicx}
\usepackage{longtable}
\usepackage{setspace}
\usepackage{footnote}
\usepackage{perpage}
\MakePerPage{footnote}
%\usepackage[para]{footmisc}
%\usepackage{dblfnote}
\usepackage{xspace}
\usepackage{array}
\usepackage{emptypage}
\usepackage[para]{footmisc}
\usepackage{hyperref}% Package for hyperlinks
\hypersetup{colorlinks,
citecolor=black,
filecolor=black,
linkcolor=blue,
urlcolor=black}

\newcommand\blfootnote[1]{%
 \begingroup
 \renewcommand\thefootnote{}\footnote{#1}%
 \addtocounter{footnote}{-1}%
 \endgroup
}

\begin{document}
\fancyhead[RE]{नाट्यशास्त्रे}
\fancyhead[CE]{\rule{0.7\linewidth}{0.5pt}}
\fancyhead[CO]{\rule{0.6\linewidth}{0.5pt}}
\fancyhead[LO]{त्रयस्त्रिंशोाऽध्यायः}
\fancyhead[LE,RO]{\thepage}
\cfoot{}
\renewcommand{\thepage}{\devanagarinumeral{page}}
\setcounter{page}{402}
% ४०२ नाट्यशास्त्रे

\noindent
एवासम्भवे रसभावोचितवृत्तजातिनिबद्धम्~। यत्तु काव्ये तन्नोक्तमित्युक्ते रसपूरणोपयोगात् प्राधान्यमुपस्तुवन् तत एव चित्तवृत्त्युचितजात्यङ्गकविशेषग्रामरागाङ्गभाषाङ्गगीतिमुपरञ्जकतया समाश्रयात् क्वचित् सुरोचितम्~। भिण्डिका द्विपदीगानादौ स्वरालङ्कारशोभा इत्येव यत् क्वचिद् वीररौद्रोचितपरिक्रमणादिसरणादुत्तालीकृततालविभ्रमं सम्पादयत् क्वापि च विप्रलम्भशृङ्गारोचितबन्धत्वविचित्रादिगान इव नाट्यायितावकाशमादधानं स्वं वाच्यं प्रोन्मार्जयदवभातीति स्वरतालस्थानेषु यथोचितकारिषु नयस्वामिभावमालम्बमानभागमास्त इति~। एवं तावत् स्वरपदतालात्मकस्वरूपवैचित्र्यवैलक्षण्यम्~। अनेन च फलवैलक्षण्यमपि व्याख्यातम्~। गान्धर्वस्य प्रयोक्तरि प्राधान्येन दृष्टफलत्वाद्~। गानस्य तु पात्रवर्गे साम्यसम्पत्तिः सामाजिकजने चोपरञ्जनपूर्वकं प्राग् {\qtt रसाध्यायो} (भ.ना.६) दितसाधारणसन्धिधृत्यात्मकरस भाववृत्तानुस्यूतिहेतुसूचितसादृश्ययोगश्चेति दृष्टं मुख्यं फलं प्राङ्मुखभोजनवत्त्वदृष्टमप्यस्तु~।\\

ननु त (य) दुद्दिश्य प्रवृत्तिः सोऽयं कार्यभेदः कारणभेदोऽपि गान्धर्ववेदवत् स्वयमनादिर्वा सामवेदप्रभवत्वेऽपि वा तदर्थादितया तस्येदम्प्रथमतयोत्पत्त्यनुपलब्धेरनाद्ये च गान्धर्वप्रभवस्तु नादोत्पत्तौ तदुपयोगि कल्पितमिति~। गान्धर्वं वा नाट्योपकरणभूतमेव सत्फलसम्पत्तये~। गानं तु नाट्यसामग्रीमध्यनिमज्जितनिजस्वरं सफलायेति स्वरूपभेदोऽपि~। एवं परोक्तो हेतुरसिद्धीकृत इति यदेव केवलं यदसत्पक्षे वैलक्षण्येऽपि साधकप्रमाणज्ञानमाख्यातमेव~। यद्यत् स्वरूपेण कार्येण धर्मेण च भिद्यते तत्तु तेन विलक्षणं भावो भावान्तरादिव~। भिद्यते च स्वरूपादिना गानम्~। गान्धर्वाद् वैलक्षण्यस्य हि तन्मात्रनिमित्तत्वं व्यापारान्तरमिति व्यापकानुपलब्ध्यापि पक्षादपि वैलक्षण्याद् व्यावृत्तो हेतुः पक्षेण वैलक्षण्येन व्यापृत इति~। तत्र च गान्धर्ववेदोदितविधिप्रबन्धप्रधानत्वादिदं गान्धर्वम्~। गीतिसारत्वाच्चेदं गानमिति~। अत एव कठकाठकवद् गान्धर्ववित्त्वाद् गन्धर्वव्यपदेशमेके मन्यन्ते~। गन्धर्वाणामिदमिति तु सामान्यशब्दोऽप्ययं भवेदेव~। उक्तं हि प्राक्~।

\begin{quote}
{\qt नारदाद्याश्च गन्धर्वा गानयोगे नियोजिताः~। (भ.ना.१.५१) }
\end{quote}

\noindent
इति~। तथा गानशब्देऽपि गीतिः~। तत्राश्रयेणास्ति सामान्यवचनता~। अभाण्डमेको (कं) गानस्येत्यादौ (भ.ना ३२.४१५) पूर्वरङ्गे गान्धर्वमेव प्रधानम्~। न त्वस्य तर्हि तदङ्गत्वे कथमेकान्ततः स्वप्राधान्ये सम्भवमात्रेण तु गानस्यापि तत्~। न हि नाट्याद्वहिर्लयभङ्गा (ङ्ग्या) पि ध्रुवागानं गीयमानमुखपादमुत्पादयति~। श्रोत्रपुटे चेयं वपुर्निजं वा ज (जा) गदतीति सोऽयमत्र लालित इव गुरुसङ्बसेवावैकल्यादनुसन्धिवर्जस्तथापि न स्मर्यते सुकु (मा) रमतिरे (मे) चैव हि प्रति प्राय इदं प्रव (प्रा) वृतत्~। न पूर्वरङ्गो नामान्यः कश्चित्~। अपि तु गीतकान्येवेति 

\newpage
% त्रयस्त्रिंशोाऽध्यायः ४०३

\noindent
दर्शितं {\qtt पञ्चमे~। पादभागाः कलाश्चैवे}त्यादिना (भ.ना.५.६)~। तदेवं सामान्यशब्दत्वेऽपि द्वयोस्तत्रान्तः \textendash\ पातिपरीक्षकहृदयानुसारेण विशेषशब्दत्वं यत्तन्नायुक्तम्~। यत्तु भट्टतोतेन तद्धेदसिद्धये क्रियाभागे तु कालस्ताल इति निरूपितं तत् क्रियातोऽनन्यो वा कालो वा भावप्रबन्धः~। एवं सा ताल इत्यादि पदान्तरमानयत् तत्तथा प्रकृतोपयोगीति {\qtt काव्यकौतुकादेव} ज्ञेयमिति च नास्माभिस्तत्परिवर्तनप्रयासः कृतः~। इयतैव सिद्धो गान्धर्वगानयोर्भेदः~।\\

नन्वेवं स्वरतालादिरूपत्वे द्वयोरपि समाने किमिति विवेको ध्रुवागानोपयोगित्वेन दर्शितः~। तत्र {\qtt केचिदाहुः}~। सत्यं मधुरकण्ठत्वजितहस्ततालगुणोपादानमपि स्वरत्वादिदोषपरिवर्जनं च समानमुभयत्र~। किं  तु प्रत्यग्रवयस्त्वादिवद् गात्र (तृ) प्रभृतेर्गुणत्वेनोच्यते~। तद्गानोपयोगी नाट्यस्य सुन्दरवपुःप्राणत्वात्~। न तु गान्धर्वे~। तत् पुनरेतदुच्यमानं गेयहेतोरवमर्शस्तस्माल्लब्धेति वचनं स्मारयति~। गातरि प्रत्यग्रवयस्त्वं गायिकास्वरूपादियोग इत्येतावदेव ह्यतोऽधिकात् स्यात्~। न च तावताऽपि भेदः कश्चित् प्रत्यग्रवयस्त्वेन स्वरसम्पादनोचितबलयोगोऽप्युपलक्ष्यते~। यत्राकृतिस्तत्र गुणाः~। प्रसिद्धा च रूपादियोगेन प्रकृतकलावलक्षण्यस्यापि पादसिद्धता लक्ष्यते~। {\qtt नावं (वाचं) विसृजेदित्यत्र नक्षत्रदर्श} (ने) नैव कालविशेषः~। गानगान्धर्वोभयविशेषभावप्रतिपादनायैव च पृथगध्यायारम्भः~। अन्यथैकता शेषतैव शक्यत इति~।\\

कथं तर्हि {\qtt गुणात् प्रवर्तते गान}मिति~। प्रकृतोपयोगादेवमभिधानम्~। सामान्यशब्दत्वाद्वा~। उक्तं चैतत्~। {\qtt कलापातनतत्त्वज्ञ} (भ.ना.३३.२) इति~। {\qtt चित्रादिवाद्यकुशल} (लौ) (भ.ना.) ३३.८) इति च~। गान्धर्व एवाञ्जस्येन सङ्गच्छत इति सिद्धः पृथगध्यायारम्भोऽपि~।\\

{\qtt गुणात् प्रवर्तत} इति~। प्रकृष्टतयोत्कृष्टतया वर्तन इत्यर्थः~। दोषोपहतयोश्चित्तनिश्चयेनास्य विक्षिप्तिः~। नत्वेकाग्रं करोति~। न रञ्जयतीति यावत्~। अनेनैवेदमाह~। यद्यपि गुणाभावो दोष एव तथापि भाविकपिम (प) लापस्थितादिदोषाणामभावे च गाता तावत् क्षिपति न तूत्कृष्टदोषयोगे तु तद्गानमेव न किञ्चित्~। गर्दभाभाषितं हि तत्~। गुणायोगे तूत्कृष्टस्तद्भवति विक्षोदीति नाट्याचार्यस्य गुणवत आदान (तुं) दोषवतश्च त्यक्तुमिति~॥~१~॥

\newpage
% ४०४ नाट्यशास्त्रे 

\begin{quote}
{\na गाता प्रत्यग्रवयाः \renewcommand{\thefootnote}{1}\footnote{स्निग्धस्वरमधुरमांसलोपचित~।}स्निग्धो मधुरस्वरोपचितकण्ठः~।\\
 लयतालकलापा \renewcommand{\thefootnote}{2}\footnote{य. काल~। भ. मान~। त. गान}तप्रमाणयोगेषु तत्त्वज्ञः~॥~२~॥

 रूपगुण\renewcommand{\thefootnote}{3}\footnote{भ \textendash\ वर्णसंस्थानघैर्यमाधुर्यसम्पन्न~।} कान्तियुक्ता माधुर्योपितसत्वसम्पन्नाः~।\\
 पेशलमधुरस्निग्धानुनादिसमरक्त\renewcommand{\thefootnote}{4}\footnote{य. शुभ} गुरु (शुभ) कण्ठाः~॥~३~॥

 \renewcommand{\thefootnote}{5}\footnote{भ. अवहितशरीरमनसा सुनिविष्टा मधुररसिकसंचारः}सुविहितगमकविधायिन्योऽक्षोभ्यो (भ्यास्) ताललयकुशलाः~।\\
 आतोद्यार्पितकरणा विज्ञेया गायिकाः श्यामाः~॥~४~॥

 प्रायेण तु स्वभावात् स्त्रीणां गानं नृणां च पाठ्यविधिः~।\\
 स्त्रीणां स्वभावमधुरः कण्ठो नृणां बलित्वं च~॥~५~॥}
\end{quote}

\hrule

\vspace{2mm}
तत्र शारीरस्वरपूर्वि (र्व) कत्वाद्वैगुणानां वंश्यानां च क्रमेण गातृणां विपञ्चीवादको (क) वंशवादकानां च गुणांस्तावदाह~। {\qtt गाता प्रत्यग्रवया} इत्यादिना~। {\qtt स्निग्ध} इत्यपरुषस्वरः~। {\qtt उपचारान् मधुरशब्देन} माधुर्यम्~।  (तेन युक्तः स्वर) स्त्रे (स्ते) नोपचितः पूर्णः कण्ठ  (ण्ठो) ऽस्येति नादोऽत्र कण्ठः~। तेनानुद्वेजको विश्रान्तिदश्च यस्य स्वरः~। लयो द्रुतादि (ः)~। (ता) लश्चश्चत्पुटादि (ः)~। कला आवापादिः~। पातालः (तः) शम्यादिः~। प्रमाणं चित्रादिमार्गः~। योगो योजना समपाण्यादिः~। एकत्वं विवक्षितमित्येके~॥~२~॥\\

एवं तन्मध्य (ध्येऽ) सौ केवलं यत्र (त्रा) धारप्राया माधुर्योपितेइति~। अत्र चेष्टानुल्बणता माधुर्यं तेनोपेतं सत्वमिति बलम्~। {\qtt पेशलो विकार}शून्य (ः)~। मधुरोल्वणवर्णविभाषि (षी)~। {\qtt स्निग्ध} (:) नेत्रतल्लकम्~। {\qtt अनुनादो} (दी) विच्छेदाभावयुक्तः~। समः परस्परं कीर्ण (र्णो) मिलितः~। {\qtt रक्तः} कोमलोऽभिरामः~। शुभो  गीयमानपदश्रावककर्ण (ण्ठः)~। (रू) पाणि यासामिति तावत्~॥~३~॥\\

शक्तिः सुविहितः शक्ता च व्यतिरिक्ता चेति न्यायेन ज्ञानापरपर्यायमन्तर्मार्गतया विदधत इति व्युत्पन्नत्वं गीतवाद्ययोर्लयः शेषाणाम्~। तच्चातोद्यविषये करणस्य श्रोतृमनोऽर्पणाद्भवतीत्यभ्यास उक्तः~। {\qtt श्मामा} इति तरुण्यः~। तद्वर्णा एव च केशसहत्वात्~॥~४~॥\\

{\qtt गायिकानां} बहुत्वे हेतुं दर्शयति~। प्रायेण त (तु) स्य (स्व) भावात् स्त्रीणां गानं नृणां च संविधिरिति~। {\qtt उपमागर्भमेतत्}~। प्रायेणेति लयं व्याचष्टे~॥~५~॥

\newpage
% त्रयस्त्रिंशोऽध्यायः ४०५

\begin{quote}
{\na  यत्र स्त्रीणां पाठ्यातू (ठ्यं) गुणैर्नराणां च गानमधुरत्वम्~।\\
 ज्ञेयोऽलङ्कारोऽसौ न हि स्वभावो ह्ययं तेषाम्~॥~६~॥

 \renewcommand{\thefootnote}{1}\footnote{भ. सुनिवष्टपाणिलटाविशारदोपमितलघुहस्ताः~। सुनिविष्ट.....दोमधुरगीतलघुहस्तः}सुनिविष्टपाणिलययतियोगज्ञौ सुमधुरलघुहस्तौ~।\\
 गातृगुणैश्चोपेताववहितमनसौ सुसङ्गीतौ~॥~७~॥

 स्फुटरचितचित्रकरणौ गीतश्रवणाचलौ प्रवीणौ \renewcommand{\thefootnote}{2}\footnote{भ. जितश्रमौ रक्तकण्ठवीणौ~।}च~।\\
 चित्रादिवाद्यकुशलौ \renewcommand{\thefootnote}{3}\footnote{भ. वीणायां}वीणाभ्यां वादकौ भवतः~॥~८~॥

 बलवानवहितबुद्धिर्गीतलयज्ञस्तथा सुसङ्गीतः~।\\
 श्रावकमधुरस्त्रग्धो \renewcommand{\thefootnote}{4}\footnote{भ. दृढानिलो}दूढपाणिर्वंशवादको ज्ञेयः\renewcommand{\thefootnote}{5}\footnote{भ. वादी स्यात्}~॥~९~॥

 अविचलितमविच्छिन्नं वर्णालङ्कारबोधकं मधुरम्~।\\
 स्निग्धं दोषविहीनं वेणोरेवं स्मृतं वाद्यम्~॥~१०~॥}
\end{quote}

\hrule

\vspace{2mm}
{\qtt यत् (त्र) स्त्रीणां पाठ्यमिति~। यत्र बा} (ब) लित्वे सति स्त्रीणां गुणपाठ्यात् गुणान् जनरञ्जनमधुरकण्ठत्वे च सति नृणां गानविषये मधुरत्वं सर्वेषामिति स्त्रीणां पुंसा चालङ्कारः कादाचित्कत्वान्नादरणीयः~। एवमत्र प्रलघुरलक्ष्यमाणाचारः~॥~६~॥\\

{\qtt गात्र(तृ) गुणैरिति}~। प्रत्यग्रवया इत्यादिभिः~। स्वराङ्गगीतिगीत्तंयाभ्यामिति~॥~७~॥\\

{\qtt  मेलति} (नि) काकुशलेनाहेतुरवसानः~। स्फुटमानमगृहीतमूहापोहाभ्याम्~। स्वयं न विकृतम्~। तत एव च चित्रम्~। करणपादिकया ययोरित्यागमानुसारेऽपि वैचित्र्ये (र) से यतनीयमित्युक्तं भवति~। यदि वा वैणिकवैपञ्चिकयोस्तुल्यकालवादने परस्परमनुसारिणां वैलक्षण्यं चेत्यनेन दर्शितम्~। गीतश्रवणाः फलाइत्यभ्यासः प्रकृष्टवीणाविपश्चीययोश्चित्रादौ मार्गद्वये~। यद्वा.. मुक्तं तत्र कुशलेन~। आदिग्रहणाद् वैपश्चिको वैणिकश्च~॥~८~॥~\\

{\qtt वंशवाद} (क)~। इत्युद्देशत्वात् तस्य गुणानाह~। {\qtt बलवानिति}~। जितप्राणम् (णः)~। {\qtt दृढपाणिरिति}~। स्वरस्थानादचलाङ्गुलिः~॥~९~॥

\newpage
%४०६ नाट्यशास्त्रे
 
{\qt ज्ञानविज्ञानकरणवचनप्रयोगसिद्धिनिष्पादनानि षडाचार्यगुणा इति~। तत्र ज्ञानं शास्त्रावबोधः~। य (त) था च क्रियासम्पादनं विज्ञानम्~। कण्ठहस्तगौण्यं करणम्~। जितग्रन्थता वचनम्~। देशादिसम्पदाराधनं प्रयोगसिद्धिः~। शिष्यस्वभावमविशेष्योपात्तय उपदेशाच्छिष्यनिष्पादनमिति~॥~११~॥}

\begin{quote}
{\na श्रावको (णो) ऽथ घनः स्निग्धो मधुरो ह्यवधानवान्~।\\
 त्रिस्थानशोभीत्येवं तु षट् कण्ठस्य गुणा मताः~॥~१२~॥

 \renewcommand{\thefootnote}{1}\footnote{भ दूरात्तु}उदात्तं श्रूयते यस्मात्तस्माच्छावक (ण) उच्यते~।\\
 श्रावकः (णः) सुस्वरो यस्मादच्छिन्नः स घनो मतः\renewcommand{\thefootnote}{2}\footnote{भ. यस्तु न विहिप्तो घनः स्मृतः}~॥~१३~॥

 अरूक्षध्वनिसंयुक्तः स्निग्धस्तज्ज्ञैः प्रकीर्तितः~।\\
 मनःप्रल्हादनकरः स वै मधुर उच्यते~॥~१४~॥

 स्वरेऽधिके च हीने च ह्यविरक्तो वि (ऽव) धानवान्~।\\
 शिरः कण्ठेष्वभिहितं (हतस्) त्रिस्थानमधुरस्वरः~।\\
 त्रिस्थानशोभीत्येवं तु स हि तज्ज्ञैरुदाहृतः~॥~१५~॥}
\end{quote}

\hrule

\vspace{2mm}
अत्र फलमाह~। {\qtt अविचलितमिति}~॥~१०~॥\\

स्वस्थानाद् दोषाणां गात्रकुशलं {\qtt विज्ञानम्}~। तदुपयोगिनो वागादेः करणवर्गस्य गुणवन्तः (त्ता) करणयुक्ताः (क्ता) करणतेत्यर्थः~। {\qtt जितग्रन्थतेति} धारकत्वम्~। वचनं धारयन् हि परस्मै ब्रूयात् स (न) विस्मर्ता~। देशादिसम्पदाराधनं प्रयोगसिद्धिः~। शिष्यबुद्धिमनुसृत्यादेष्ट्टत्वं यद्वैचित्रयं तच्छिष्यनिष्पादनम्~। अथ गीते स्वराणां प्राधान्यमिति दर्शयितुमेतावन्तो गुणा अवश्याह (द) रणीया इति~॥~११~॥\\

तार्त्पर्येण {\qtt कण्ठस्य} नादस्य गुणानाह~। {\qtt श्रावण} इत्यादि~॥~१२~।\\

एतत् क्रमेण लक्षयति~। त्रिस्थानत्वं केचिदाक्षिप्य प्रतिसमादधति~।  वायुर्नाभेरूर्ध्वमभिहत्योरसि  वृतः शिरस्यनितो मुखेन वृत्त: स्छलमागच्छजयमुरस्थं नाडिकाक्षेपं तथापि तत्रैवायं शब्द इति प्रतिभास्यो यत्नवशादिति तथात्वमिति समाधिः~। अस्माभिस्तु यदत्र वक्तव्यं तदुक्तमेव काक्कध्यायादा 

\newpage
% त्रयस्त्रिंशोऽध्यायः ४०७ 

\begin{quote}
{\na कपिलो ह्य (ऽव्य) वस्थितश्चैव तथा सन्दष्ट एव च~।\\
 काकी च तुम्बकी च (चैव) पञ्च दोषा भवन्ति हि~॥~१६~॥

 वैस्वर्यं च भवेद्यत्र तथा स्य (स्या) द् घर्घरायितम्~।\\
 कपिलः स तु विज्ञेयः श्लेष्मकण्ठस्तथैव च~॥~१७~॥

 ऊनताऽधिकता चापि स्वराणां यत्र दृश्यते~।\\
 रूक्षदोषहतश्चैव ज्ञेयः स त्वव्यवस्थितः~॥~१८~॥

 दण्ड (न्त) प्रयोगात् सन्दष्टस्त्वाचार्यैः परिकीर्तितः~।\\
 यो न विस्तरति स्थाने स्वरमुच्चारणागतम्~।\\
 तथा रूक्षस्वरश्चैव स काकीत्यभिसंज्ञितः~॥~१९~॥

 नासाग्रग्रस्तशब्दस्तु तुम्बु (म्ब) की सोऽभिधीयते~॥~२०~॥}

अन्ये तु~। 

 {\na समप्रहरणे चैव जविनौ विशदौ तथा~।\\
 जितश्रमौ विकृष्टौ च मधुरौ स्वेदवर्जितौ~।\\
 तथा बृहन्नखौ चैव ज्ञेयौ हस्तस्य वै गुणाः~॥~२१~॥} इति~।
\end{quote}

\hrule

\vspace{2mm}
(भ.ना. १७) वित्यलम्~॥~१३ \textendash\ १५~॥\\

अथ दोषं तद्वेशलक्षणाभ्यामाह~। {\qtt कपिल} इत्यादिना~॥~१६~॥\\

कम्पनं कपिस्तं लाल्य (लयतीति) ...... ज्यतालं {\qtt ग (घ) र्घरायितः} स्थानाच् च्युतिः श्लेष्मणाऽपि स्थानस्यालाभस्ततो वैस्वर्यम्~॥~१७~॥\\

अव्यवस्थानादव्यवस्थितो नादः~॥~१८~॥\\

दन्तयन्त्रसन्दंशात् {\qtt सन्दष्टः}~। काकोदुम्बरिकाम (मु) क्तं किन्तुस्व (र) तन्तुवीणाशब्देन च सादृइश्यं यस्यास्ति शब्दस्य स काकी तुम्बकी च~॥~१९ \textendash\ २०~॥

\newpage
% ४०८  नाट्यशास्त्रे

\begin{quote}
{\na एते गुणाश्च दोषाश्च तत्त्वतः \renewcommand{\thefootnote}{1}\footnote{भ. कण्ठजा गदिता~।}कथितो (ता) मया~।\\
 अत ऊर्ध्वं प्रवक्ष्यामि ह्यवनद्धविधिं पुनः~॥~२२~॥

 गान्धर्वमेतत् कथितं मया तत् (वः)\\
 पूर्वं यदुक्तं प्रपितामहेन\renewcommand{\thefootnote}{2}\footnote{भ. इह नारदेन~।}~।\\
 कुर्याद् य एवं तु नरः प्रयोगे\\
 सम्मानमग्र्यं लभते स लोके\renewcommand{\thefootnote}{3}\footnote{भ. कुशलेषु गच्छेत्~।}~॥~२३~॥\renewcommand{\thefootnote}{*}\footnote{PP. 328  \textendash\ 329 (G.O.S.) and PP. 393  \textendash\  402 (G.O.S.) is a very very difficult track, where (N) seems to go completely astray. The comparision is given in Appendix  \textendash\ ll}}

इति~।

 {\na इति गुणदोषविचारो नाम त्रयस्त्रिंशः~॥~३३~॥}
\end{quote}

\hrule

\vspace{2mm}
समप्रहरणादिसव्यसाचिवद् द्वावपि प्रयोगसमौ स्वदेहिनस्तन्त्री विसारणयति~॥~२१~॥\\

अध्यायार्थस्योपसंहारमन्यस्यासूत्रणं करोति~। एते गुणाश्चेति~। एतदर्थं गानसिध्द्यर्थं गान्धर्वयुक्तमिति समासं केचिदाहुः~। अन्ये तु गान्धर्वमिति सामान्यशब्दो लोक इह चामुत्र चेति शिवम्~॥~२२ \textendash\ २३~॥\\

\begin{center}
\textbf{इति माहेश्वराभिनवगुप्ताचार्यविरचितायां नाट्यवेदवृत्तावभिनवभारत्यां  गुणविवेकाध्यायस्त्रयस्त्रिंशः~॥~३३~॥}
\end{center}

\newpage
\fancyhead[LO]{चतुस्त्रिंशोऽध्यायः}
% चतुस्त्रिंशोऽध्यायः ४०९

\begin{center}
\textbf{\LARGE ॥~श्रीः~॥}\\

\vspace{2mm}
\textbf{\LARGE  अथ चतुस्त्रिंशोऽध्यायः~।}
\end{center}

\begin{quote}
{\na  \renewcommand{\thefootnote}{1}\footnote{ज. ततातोद्यविधिस्त्वेष मया प्रोक्तः समासतः~। अवनद्धविभागेन लक्षणं कर्म चैव हि~। आत्यानां प्रवक्ष्यामि विधिं वादनमेव च}ततवाद्यविधानं तु यन् मयाऽभिहितं पुरा~।\\
 अवनद्धगतस्यापि तस्य वक्ष्यामि लक्षणम्\renewcommand{\thefootnote}{2}\footnote{र. अवनद्धस्य वक्ष्यामि लक्षणं कर्म चैव हि}~॥~१~॥\renewcommand{\thefootnote}{2a}\footnote{N; अवनद्धस्य
वक्ष्यामि लक्षणं कर्म चैव हि~।  (१ \textendash\ २) यथोक्तं मुनिभिः पूर्वं स्वातिपुष्करनारदैः~। 1cd (G.0.S.) not read in (N) , N (cd) is V.2 ab (G.0.S.)}

 यथोक्तं मुनिभिः पूर्वं स्वातिनारदपुष्करैः\renewcommand{\thefootnote}{3}\footnote{र. पुष्करनारदैः}~।\\
 सर्वलक्षणसंयुक्तं सर्वा (तथा) तोद्यविभूषितम्\renewcommand{\thefootnote}{4}\footnote{N. र. पुरस्कृतम् 5. ज. मृदङ्गपणवानां च दर्दरस्य तथैव च~। गान्धर्वं चैव वाद्यं च स्वातिना नारदेन च~। विस्तारगुणसम्पन्नमक्तं लक्षणकर्मतः~। अनावृत्या तया स्वाते  (भ.तयोरेवं) रातोद्यानां समासतः~। पौष्कराणां प्रवक्ष्यामि निर्वृत्तिं सम्भवं तथा}~॥~२~॥

 5मृदङ्गानां समासेन लक्षणं पणवस्य च~।\\
 दर्दरस्य च संक्षेपाद् विधानं वाद्यमेव च~॥~३~॥}
\end{quote}

\hrule
 
\begin{center}
अथ चतुस्त्रिंशोऽध्यायः~। 
\end{center}

\begin{quote}
{\qt मार्जनानुगतमार्गसुन्दरं\\
पुष्करत्रितयमाश्रितं सदा~।\\
चित्रदानपदक्लृप्तविग्रहं\\
श्रीसदाशिवतनुं शिवं नुमः~॥}
\end{quote}

वृत्तेऽध्यायेऽवनद्धविधिं ब्रूम इत्युक्तम्~। तत्र पुनरर्धं (र्थं)  स्फुटयितुमाह~। {\qtt ततवाद्यविधानं तु यन्  मयेइति}~। यत्तद्विधानं पूर्वमुक्तं तस्यैव लक्षणं वक्ष्यामीइति सङ्गतिः पुनरर्धः (र्थः)~। न च पुनरुक्ततया तावदन्तर्धानं गतमिति प्राप्तमपि तस्य विधेयस्वरूपं भवति तच्च नोक्तमिति~। अवहननं चर्मणा बन्धः~। स यद्यपि वीणादावस्ति तथापि स्वोंऽशोऽत्रारादुपकारी न सन्निपत्य तन्त्रीपणवादौ ततांशवदिति~। तत्र व्यपदेशोऽपि कृतः~। तत एव बद्धपाणिकरटादीनि~। यद्वादकेषु पक्षातोद्यकव्यवहारो लोके तैश्च संयुक्तं कृत्वापि~। तथेति तस्माद्धेतोरुक्तम्~। तस्मान् मृदङ्गाणां (नां) त्रयाणां पुष्कराणां पणवस्य दर्दरस्य च लक्षणात्मकं विधानं चर्मादिकारणत्वं वाद्यं

\newpage
% ४१० नाटयशस्त्रे

\begin{quote}
{\na  अनध्याये कदाचित्तु स्वातिर्महति दुर्दिने\renewcommand{\thefootnote}{1}\footnote{ज. वै दुर्दिने दिने}~।\\
 जलाशयं जगामाथ \renewcommand{\thefootnote}{2}\footnote{र. उदक}सलिलानयनं प्रति~॥~४~॥

 \renewcommand{\thefootnote}{3}\footnote{ज. अमिनृपौ सरस्थे च प्रवृत्तः}तस्मिञ् \renewcommand{\thefootnote}{4}\footnote{ड. पुरा निषण्णे च प्रविष्टः~। भ. हृष्ट प. वृष्टः}जलाशये यावत् प्रविष्टः (वृत्तः) पाकशासनः~।\\
 धाराभिर्महतीभिस्तु पूरयन्निव मेदिनीम्\renewcommand{\thefootnote}{5}\footnote{ज.कर्तुमेकार्णवं जगत्}~॥~५~॥

 पतन्तीभिश्च धाराभिर्वायुवेगाज्जलाशये~।\\
 \renewcommand{\thefootnote}{6}\footnote{र. N महत्पटपटाशब्दः कृतः पुष्करपातजः~॥}पुष्करिण्यां \renewcommand{\thefootnote}{7}\footnote{ड. कलरवः~। ज. पुटरवः}पदुः शब्दः पत्राणामभवत्तदा~॥~६~॥

 तेषां \renewcommand{\thefootnote}{8}\footnote{भ. धीरकलं शब्द}धारोद्भवं नादं निशम्य \renewcommand{\thefootnote}{9}\footnote{ज.सहसा}स महामुनिः~।\\
 आश्चर्यमिति \renewcommand{\thefootnote}{10}\footnote{र. N तं प्राप्तमुपधारितवान् तदा~। ज. संप्राप्तं~।}मन्वानश्चावधारितवान् स्वनम्\renewcommand{\thefootnote}{11}\footnote{भ. ध्वनिम्~। ब. मुनिः}~॥~७~॥

 ज्येष्टमध्यकनिष्ठानां पत्राणामवधार्य च\renewcommand{\thefootnote}{12}\footnote{र. अवधारयन्~। भ. अवधारितम्~। ज उपधार्य तु}~।\\
 गम्भीरमधुरं \renewcommand{\thefootnote}{13}\footnote{ज. शब्दं~।}हृद्यमाजगामाश्रमं ततः~॥~८~॥}
\end{quote}

\hrule

\vspace{2mm}
\noindent
च फलं {\qtt संक्षेपाद् वक्ष्यामीति} पूर्वेण सम्बन्धः~। एकश्चो यस्मादर्थे द्वितीयः समुच्चये~। अन्यथा त्रय उदीयेरन्~। धर्मे चार्थे च कामे च मोक्षे चेति (महाभारतम् १.६२.२६) यथा~। {\qtt नारदग्रहणं} गातुः पुष्करवाद्यज्ञानमवश्यमुपयोगीति शिक्षयति~। ब्रह्मग्रहणं कवेरपीति~। अत्र पूर्वनिपातात् स्वातिर्भाण्डे नियुक्तः  (१.५०) इति प्रथमाध्याये दर्शितत्वात् स्वातिरत्र प्रधानमिति पुराकल्पेन दर्शयति~॥~१ \textendash\ ३~॥\\

{\qtt अनध्याये} कदाचित्विति~। {\qtt तुरागमद्योतकः}~। नारदादिभ्यो विशेषो वा~। अनध्याये इति प्रकृतानुपरोध उक्तः~। सलिलानयनं प्रतीति तृषाहत (इति) प्रतिषेधः~॥~४~॥\\

{\qtt प्रवृत्तो} वर्षितुम्~। प्रवृत्तवर्षो (मेदिन्यां) ततो जलाशये वा~॥~५~॥\\

{\qtt  पुष्करिणी} कमलखण्डः~। तत्र पत्राणां शब्दोऽभवदिति सम्बन्धः~। स्थलकमलषण्डे हि न चित्राणि पत्राणि वितताविततरूपाणि भवन्तीति जलाशयग्रहणम्~। अत एव वक्ष्यति~। {\qtt ज्येष्टमध्यकनिष्टानामिति}~।  (भ.ना.३४.८)~॥~६~॥\\

{\qtt अवधारितवानिति}~। ईदृशः शब्द एव चित्तवृत्तिं तोषयतीति स्वसंवेद (ने) न निश्चिता (का) येत्यर्थः~॥~७ \textendash\ ८~॥

\newpage
% चतुस्त्रिंशोऽध्यायः ४११

\begin{quote}
{\na  ध्यात्वा सृष्टिं मृदङ्गानां पुष्करानसृजत् ततः\renewcommand{\thefootnote}{1}\footnote{ज. ततश्चक्रे त्रिपुष्करम्~।}~।\\
 पणवं दर्दरं चैव सहितो विश्वकर्मणा~॥~९~॥

 देवानां दुन्दुभिं दृ (भीर्दृ) ष्ट्वा चकार \renewcommand{\thefootnote}{2}\footnote{प. मुरजं~।}मुरजांस्ततः~।\\
 \renewcommand{\thefootnote}{3}\footnote{र. आलिङ्गादाङ्किकं चैव दर्दरं पणवं तथा~।}आलिङ्गम् (ङ्ग्यश्चो) र्ध्वकं चै (कश्चै) चैव तथैवाङ्किकमे (ए) चैव च~॥~१०~॥

 चर्मणा चावनद्धांस्तु मृदङ्गान् दर्दरं तथा~।\\
 तन्त्रीभिः पणवं चैवमूहापोहविशारदः\renewcommand{\thefootnote}{4}\footnote{भ. समन्वितम्~।}~॥~११~॥\renewcommand{\thefootnote}{4a}\footnote{N: तन्त्रीभिः पणवं चैव......निर्णयं ( V. १०ab \textendash\  (N))}

 \renewcommand{\thefootnote}{5}\footnote{ज. भूयश्चान्यान्यपि तथा काष्ठायसकृतान्यथ~।}त्रयं चान्यान्यपि तथा काष्ठायसकृतान्यथ~।\\
 \renewcommand{\thefootnote}{6}\footnote{भ.जर्जरी~।}झल्लरीपटहादीनि चर्मनद्धानि तानि च\renewcommand{\thefootnote}{7}\footnote{ज. निर्ममे~। प्रोक्तानि चाक्षराण्येषां सप्रकारं समार्जनम्~। अतोद्यमिति विज्ञेयं केवलं तु चतुर्विधं तु विज्ञेयमातोद्यं लक्षणान्वितम्~। अवनद्धं ततं चैव घनं सुषिरमेव चे~। पूर्वमुक्तं तस्यापि लक्षणं कर्म चैव हि~। अवनद्धस्य वक्ष्यामि विधिं लक्षणमेव च~।}~॥~१२~॥\renewcommand{\thefootnote}{7a}\footnote{N: After V १२ cd (G.O.S.)  which is १२ ab (N) ,V.१२ (N) cd reads प्रोक्तानि चाक्षराण्येषां सप्रकारं समार्जनम्~॥}}
\end{quote}

\hrule

\vspace{2mm}
ध्यात्वेति~। {\qtt अभिसन्धाय}~। मृत् अङ्गं येषाम्~। {\qtt मृच्चर्मेत्येके}~। मर्दनादिति~। {\qtt पुष्करानिति}~। पारम्पर्येण पुष्करप्रभवत्वादिति भावः~। सहितो विश्वकर्मणेति~। आकृतिकरणे त्वष्टुरिव कुम्भकारादिमौरजिकोपदेशापेक्षस्य व्यापार इति सूचयति~॥~९~॥\\

ननु च वाद्यं तेन स्वातिनाऽऽरूढपूर्वं दृष्टं प्रतिभाति~। दृष्टपूर्ववैचित्र्यमुल्लिखेदित्याशङ्क्याह~। {\qtt देवानां} दुन्दुभीर्दृष्द्धेति~। मङ्गलार्थानि चर्मनद्भानि नाभूवन्~। केवलं स्वाति (:) क्रमाक्षरादियोजनया तद्वाद्यं रसभावद्योतनोपयोगि सो (स्वो) पज्ञवैचित्र्यान्तरमुत्प्रेक्षांचक्र इत्यर्थः~। मुर (रा) सुपधिषसाश्च (सूपधिष्ठिता च) श्लक्ष्म (क्ष्ण) सुकुमारायां मृदि प्ररोहतीति मृदपि मुरा~। ततो जाता मुरजा मृदङ्गा इत्यर्थः~। ङ्यापोरिति (अष्टा. ६.३.६३) संज्ञाया ह्रस्वः~। तानाह~।\\

आलिङ्ग (ङ्ग्य) श्चोर्ध्वकश्चैव तथैवाङ्कित (क) एव च~। इति~। एकश्चकारस्त्रिपुष्करत्वं नियमयति~। अन्यश्चकारः पक्षातोद्यमनुगृह्णाति~॥~१०~॥\\

दर्दरं तथेति~। {\qtt चर्मनद्धम्}~। तन्त्रीभिरिति~। चकाराच् चर्मणा चेति~॥~११~॥\\

(काष्ठायसेति~।) काष्ठं सुषिरीकृत्य कृतानि~। अत एवायसं तेन कृतानि~। आयसं ताम्राद्युप \textendash\ लक्ष्यते~। झल्लरी कांस्यतालम्~। तन्त्रीव क्रान्तः (कान्ता) पूर्णा पटहिका~। पटहः शुद्धः~। आदिग्रहणात् करटाहुदुक्वमर्दलादेर्ग्रहणम्~॥~१२~॥

\newpage
%४१२ नाट्यशास्त्रे 

\begin{quote}
{\na  आतोद्यसमवाये तु यानि योज्यानि वादकैः~।\\
 \renewcommand{\thefootnote}{1}\footnote{भ. अक्गोपाङ्गानि चैतेषां~।}अङ्गप्रत्यङ्गयोगेन गदतो मे निबोधत~॥~१३~॥\renewcommand{\thefootnote}{1a}\footnote{N: अङ्गोपाङ्गानि एतस्य प्रत्यङ्गानि तथैव च~॥}

 विपञ्ची चैव \renewcommand{\thefootnote}{2}\footnote{भ.वीणा~। ज. वीणा च चित्रा  चैवा्गसंस्थिता}चित्रा च दारवीष्वङ्गसंज्ञिते~।\\
 कच्छपीघोषकादीनि प्रत्यङ्गानि तथैव च\renewcommand{\thefootnote}{3}\footnote{ज. विनिर्दिशेत्~।}~॥~१४~॥

 \renewcommand{\thefootnote}{4}\footnote{ज.मृदज्ञं दर्दरं चैव पणवं चाङ्गसंस्थितम्~।}मृदङ्गा दर्दराश्चैव पणवाश्राङ्गसंज्ञिताः~।\\
 झल्लरीपटहादीनि प्रत्यङ्गानि तथैव च~॥~१५~॥

 अङ्गलक्षणसंयुक्तो विज्ञेयो वंश एव हि~।\\
 शाख (शङ्ख) स्तुण्डि (ण्ड) किनी\renewcommand{\thefootnote}{5}\footnote{भ् दुन्दुभिका~। न. दुन्दुभिनी~।} चैव प्रत्यङ्गे परिकीर्तिते\renewcommand{\thefootnote}{6}\footnote{ज. अभिसंज्ञिते}~॥~१६~॥

 नास्ति किश्चिदनायोज्यमातोद्यं दशरूपके\renewcommand{\thefootnote}{7}\footnote{भ.नाटकाश्रये~। ज. चैव नाटके~।}~।\\
 \renewcommand{\thefootnote}{8}\footnote{भ. रसं कायं~।}रसभाव (वे) प्रयोगं तु ज्ञात्वा योज्यं विधानतः\renewcommand{\thefootnote}{9}\footnote{(N) निबोधत}~॥~१७~॥}
\end{quote}

\hrule
 
\vspace{2mm}
{\qtt आतोद्यसमवाय} इति~। आतोद्यपूरणनिमित्तं (त्तानि) यानि योजनाङ्गानि तानि च चकारेति पूर्वेण सम्बन्धः~। ननु त्रिपुष्करमेव वाद्यम्~। तच्चालिङ्गकोर्ध्वकाङ्गि (ङ्कि) केषु पूर्णं दर्दरपणवस्य झल्लरीपटहादेर्वा न योग इत्याशङ्क्याह~। {\qtt अङ्गप्रत्यङ्गयोगेनेति}~। स्वरूपपरिपूर्व (र्ण) त्वेनावश्यंभाव्युपकरणमङ्गम्~। अन्यत् प्रत्यङ्गम्~॥~१३~॥\\

तत्त्ववनद्भस्यैव किमयं न ततादेरपीत्याह~। {\qtt विपञ्ची चैव चित्रा} चेति~। व्याख्याते प्राक्~। {\qtt दारवीष्विति}~। कीणासु ते {\qtt अङ्गम्}~। कच्छपी कूर्मी सैरन्ध्रीति प्रसिद्धा~। घोषकः पिष्टनका~। ती प्रत्यङ्गौ~॥~१५~॥\\

{\qtt मृदङ्गा} इति~। पुष्करत्रयव्यतिरिक्ता एवात्र महन् (हा) मर्दलाकारास्ते~। पटह (हा) मृदङ्गा (न्) दर्दरं तथेत्यर्धोक्ताः (भ.ना. ३५.११)~। ते मृदङ्गदर्दरपणव (वा) मृदक्गादिषु त्रिपुष्करे प्रत्यङ्गानि~॥~१५~।\\

सुषिरेऽन्तरितान्यज्गप्रत्यङ्गानीत्याह~। अङ्गलक्षणसंयुक्त इति~। पूर्णस्वरे वंशेऽङ्गस्वरत्वादङ्गत्वेन लक्ष्यमाणो वंश एवासावित्येवकारः~। शङ्खवत् तुण्डि (ण्ड) किनी तर्तुकेति प्रसिद्धा तत् प्रत्यङ्गम्~॥~१६~॥\\

नन्वेवं न किञ्चिदिहातोद्यं त्यक्तं भवति नाद्ये इत्याह~। {\qtt नास्ति  किश्चिदनायोज्यमिति}~। क्वचित्तु~। किञ्चिदित्याह~। रसभावे तु~॥~१७~॥

\newpage
% चतुस्त्रिंशोऽध्यायः ४१३ 

\begin{quote}
{\na  उत्सवे चैव याने च नृपाणां मङ्गलेषु च\renewcommand{\thefootnote}{1}\footnote{भ.चव संगमे}~।\\
 शुभकल्याणयोगे च \renewcommand{\thefootnote}{2}\footnote{भ.विभावकरसे}विवाहकरणे तथा~॥~१८~॥

 उत्पाते \renewcommand{\thefootnote}{3}\footnote{भ.विद्रुते~। ज. विद्रवे}सम्भ्रमे चैव सङ्ग्रामे पुत्रजन्मनि\renewcommand{\thefootnote}{4}\footnote{ज.~॥  सम्भ्रमे तथा~। च. उत्पत्तौ च तनूजस्य संग्रामे युद्धसंकुले~। र. युद्धसंश्रये}~।\\
 ईदृशेषु हि कार्येषु सर्वातोद्यानि वादयेत्~॥~१९~॥

 \renewcommand{\thefootnote}{5}\footnote{र. स्वभावे}स्वभावगृहवार्तायाम\renewcommand{\thefootnote}{6}\footnote{भ. स्वल्पं}ल्पभाण्डं प्रयोजयेत्~।\\
 उत्थानका\renewcommand{\thefootnote}{7}\footnote{ज. काव्य~।}र्य (व्य) बन्धेषु सर्वातोद्यानि वादयेत्\renewcommand{\thefootnote}{8}\footnote{र. नादयेत्}~॥~२०~॥

 \renewcommand{\thefootnote}{9}\footnote{भ.हर्षार्थं तु प्रयोक्तृणां~। ज.हर्षार्थ मङ्गलार्थं च}अङ्गानां तु समत्वाच्च\renewcommand{\thefootnote}{10}\footnote{य. समग्राच्च~।} छिद्रप्रच्छादने तथा~।\renewcommand{\thefootnote}{11}\footnote{भ. नाय च}\\
 विश्रामहेतोः शोभार्थं भाण्डवाद्यं विनिर्मितम्\renewcommand{\thefootnote}{12}\footnote{र. N. प्रयोजयेत्}~॥~२१~॥

 \renewcommand{\thefootnote}{13}\footnote{भ. अथावनद्धं}तत्रावनद्धे वक्ष्यामि \renewcommand{\thefootnote}{14}\footnote{भ. ताल}विधिं स्वरसमुत्थितम्~।\\
 नानाकरणसंयुक्तं मार्गजातिविभूषितम्~॥~२२~॥}
\end{quote}

\hrule

\vspace{2mm}
क्वचित्तु सर्वमपि योज्यमित्याह~। उत्सवे दैवकार्यादि (दौ)~। एवएवंप्रकारे समुच्चयः~। द्वितीय एवकारः प्रकारान्तरव्यावृत्त्यै~। यानं दण्डयात्रादि~। एतच्च नाट्ये लोकेऽपिच~॥~१८ \textendash\ १९~॥\\

{\qtt स्वभावेन ग्र (गृ) हवार्ता}~। उद्यानगमनादिरनाविष्टभावात्मा~। तत्राल्पं भाण्डवाद्यम्~। भणतीति भाण्डंत्रिपुष्करम्~। उत्थानमित्युत्साहः~। तत्प्रधानेषु काव्यबन्धेषु वीररौद्रादिनियमेषु डिमादिषु सर्व (र्वा) तोद्यानि~॥~२०~॥\\

नन्वेवं शृङ्गारपरिक्रमादौ दीप्तत्वायोगिनि किं भाण्डवाद्येनेति शङ्कित्वाह~। {\qtt अङ्गानामिति}~। शाखोपाङ्ग (च) रणादीनां समत्वाय परिक्रमणे~। {\qtt छिद्रप्रच्छादन} इत्युत्तरध्रुवागाने~। विश्रामहेतोरित्याक्षेपगानादौ~। शोभार्थमिति प्रसादगाने~॥~२१~॥\\

एवमुत्पत्तिस्वरूपप्रयोजनान्यभिधायावनद्धस्य वाद्यं लक्षयितुमुपक्रमते~। {\qtt तत्रावनद्ध} इति~। स्वर (स) मुत्थितम्~। स्वरं शब्दमभिसन्धाय यो विधिः प्रवृत्तः इत्थमयं शब्दो भवतीति वक्ष्यामि~। स्वरषट्काद्यनुहाररूपवर्णनाऽनुहाररूपं च शब्दमात्रमिह करणरूपादिषड्विधमार्गोचितत्वं ताभिश्चतुर्धाजातिः शुध्दाधिकाष्टादशधा च लक्ष्यते~।

\newpage
% ४१४ नाट्यशास्त्रे 

\begin{quote}
{\na  यावन्ति चर्मनद्धानि ह्यातोद्यानि द्विजोत्तमाः~।\renewcommand{\thefootnote}{1}\footnote{भ.विभागतः}\\
 तानि त्रिपुष्कराद्यानि ह्यवनद्धमिति स्मृतम्~॥~२३~॥

 एतेषां तु \renewcommand{\thefootnote}{2}\footnote{N. र. तथा~। ज. यथा~। भ. यथा तेषां}पुनर्भेदाः शतसंख्याः प्रकीर्तिताः\renewcommand{\thefootnote}{3}\footnote{N. र. संख्यानसंज्ञिताः}~।\\
 किन्तु त्रिपुष्करस्यास्य\renewcommand{\thefootnote}{4}\footnote{र.अत्र} लक्षणं प्रोच्यते मया~॥~२४~॥

 शेषाणां कर्मबाहुल्यं\renewcommand{\thefootnote}{5}\footnote{भ. ल्यात्} यस्मादस्मिन्न दृश्यते~।\\
 न स्वरा न प्रहाराश्च नाक्षराणि न मार्जनाः~॥~२५~॥

 \renewcommand{\thefootnote}{6}\footnote{इतः पूर्वं चयभिन्नासु मातृकासु केवलं तत्र गाम्भीर्यमातोद्येषूपपाद्यते~। केवलं (N) तत्र गाम्भीर्य आतोद्येष्वपि लक्षयेत्~। इति पाठो दृश्यते}भेरीपटहज\renewcommand{\thefootnote}{7}\footnote{N. र. कङ्काभिः~। भ. दम्भासु~। ज. भाण्डास्तु}म्भा (ञ्झा) भिस्तथा दुन्दुभिडिण्डिमैः\renewcommand{\thefootnote}{8}\footnote{भ. डिण्डिमे}~।\\
 शैथिल्यादायतत्त्वाच \renewcommand{\thefootnote}{9}\footnote{ज. स्वर (भ. N. स्वरं) गाम्भीर्यमानयेत्}स्वरेऽगाम्भीर्यमिष्यते~॥~२६~॥}
\end{quote}

\hrule

\vspace{2mm}
लेपमार्जनादेरत्रैवोपयोग इत्याशयेन तदिह नोक्तम्~। अन्ये तु स्वरशब्देन स्वरानुहारं च~। अक्षररूपं तु वर्णः~। अनुहारकरणशून्याः संगृह्णते~॥~२२~॥\\

ननु झल्लरीपटहादौ पूर्वं वर्णाद्यनुहार इत्याह~। यावन्ति {\qtt चर्मनद्धानीति}~। आदौ त्रिपुष्करे वाद्यं प्रधानं येषां तदङ्गानि सन्ति तानि प्रयुज्यन्ते~॥~२३~॥\\

अन्यथा तु एतेषां तु पुनर्भेदा इत्यसंख्यत्वादलक्षण इ (मि) त्याशयः~। नन्ववनद्ध एवानारम्भणीयः~। {\qtt नेत्याह}~। किन्तु त्रिपुष्करस्येति~। वाद्ये {\qtt अन्यपदार्थबहु्व्रीहिः}~। अवयवेन विग्रह :~। समुदायो वृत्यर्थ इति  गुर्वाचार्यादिः (दयः)~॥~२४~॥\\

नन्विदानीमङ्गप्रत्यङ्गकृतो ज्ञेय इत्याह शेषाणामिति~। {\qtt कर्मबाहुल्यमिति}~। वाद्यं तत्तद्नहुलतेषु~। {\qtt अत्रेहेतुः}~।

\begin{quote}
{\qt न स्वरा न प्रहाराश्च नाक्षराणि न मार्जनाः~।}
\end{quote}

\noindent
इति~। षड्जादिस्वराभा (वा) त् त (त्समपत्तये)~। {\qtt मार्जना} मायूर्याद्य (द्या) स्तेऽत्रायस्तास्तत्सम्पत्तिफलाः~। अक्षराणां षोडशानां मानं तदभावात् तत्सम्पत्तये~। निरगृहीतादयस्त्रयः प्रहाराः~॥~२५~॥\\

ननुस्वराभावे कथं शब्दवैचित्र्यमित्यत आह~। {\qtt भेरीपटहजञ्ज्ञाभिरिति}~। तुरीयेऽध्याये स्वरूपमुक्तमेषाम्~। एतैर्यः क्रियते स्वरः शब्दस्तत्र केवलं शैथिल्याद् वध्रचर्मकीलिना...............त् धामभिर्यन् मृदुत्वमायतत्त्वातु न गाम्भीर्यमुच्चत्वम्~।

\newpage
% चतुस्त्रिंशोऽध्यायः  ४१५

\begin{quote}
{\na \renewcommand{\thefootnote}{1}\footnote{ज. न च तानि प्रयोज्यानि काले~।}प्रायशस्तानि \renewcommand{\thefootnote}{1a}\footnote{(N) . न च तानि न कार्याणि कालकार्ये (कालं कार्यं) समीक्ष्य तु~। (V. 30 ab .N)}कार्याणि \renewcommand{\thefootnote}{2}\footnote{र. कालकार्ये~।}कालं कार्यं समीक्ष्य तु~।\\
 \renewcommand{\thefootnote}{3}\footnote{ज. तस्मात्~।}किन्तु त्रिपुष्करस्यास्य श्रूयतां यो विधिः स्मृतः~॥~२७~॥

 वाय्वात्मको भवेच्छद्वः स चापि \renewcommand{\thefootnote}{3a}\footnote{(N) . द्विविधः स्मृतः,}द्विविधो मतः~।\\
 \renewcommand{\thefootnote}{4}\footnote{भ. स्वन \textendash\ ~।}स्वरवांश्चैव विज्ञेयस्तथा वैवाभिधानवान्~॥~२८~॥

 \renewcommand{\thefootnote}{5a}\footnote{(N) नानाभावसमाश्रितः}तत्राभिधानवान् नाम\renewcommand{\thefootnote}{5}\footnote{ज. अन्तराश्रितः~। र. भावसमाश्रिता (धानता)} नानाभाषासमाश्रयः~।\\
 \renewcommand{\thefootnote}{6}\footnote{ज. स्वन \textendash\ ~।}स्वरवानपि विज्ञेयो नानातोद्यसमाश्रयः~॥~२९~॥}
\end{quote}

\hrule

\vspace{2mm}
\noindent
आवृत्याकारः श्लिष्टः~। एतदुक्तं भवति~। उच्चनीचतामात्रमेवात्र परं तु न तावता स्वरोदयः~। स्वरान्तरश्रुत्याद (दावा) यत्तो हि संस्फुटो भवतीत्युक्तम्~। मत्तकोकिलादयस्तुन तथा~। भावरक्ति... योगादेतच्च स्वरगते न्यक्षेण चर्चितमित्यास्ताम्~॥~२६~॥\\

पुनर्न चैषां सर्वदा नोपयोग इति दर्शयति~।

\begin{quote}
{\qt प्रायशस्तानि कार्याणि कालं कार्यं समीक्ष्य तु~।}
\end{quote}

\noindent
{\qtt किन्त्विति}~। प्रकृतं सूत्रयति~। {\qtt त्रिपुष्करस्यास्य श्रूयतां} यो विधिरिति~॥~२७~॥\\

तत्र शब्दस्य तावत् स्वरूपमाह~। {\qtt वाय्वात्मको भवेच्छब्दः स चापि द्विविध} इति~। वायुरात्मस्वरूपं जन्मन्युपादानं व्यक्तौ वा निमित्तमात्रं यस्य शब्दस्य~। एतच्च

\begin{quote}
{\qt वायोरणूनां ज्ञानस्य शब्दत्वापत्तिरिष्यते~। }
\end{quote}

\noindent
इत्यत्र ब्रह्मकाण्डोद्देशे विवेचितम्~। इह त्वप्रकृतमिति न विस्तारितम्~। स्वरात्मकं स्वरूपं यस्य नित्यमस्ति स {\qtt स्वरवान्} नादमात्रात्मकः~। ननु पाण्डित्ययोगनियमपर्यवसायित्वं वदता वर्णात्मके नादरूपता नास्तीत्युक्तम्~। अभिधाने समयादिनाऽर्थप्रतिपादनं विद्यते यस्य सोऽभिधानवान् वर्णात्मकः~॥~२८~॥\\

तत्रेदमिह वक्तव्यं मत्वा पूर्वाभिधानोपयोगिनमाद्यमाह~। {\qtt स्वरवानिति}~। आतोद्ये वीणादौ स एव भवतीति~॥~२९~॥

\newpage
% ४१६ नाट्यशास्त्रे

\begin{quote}
{\na  शारीर्यामेव\renewcommand{\thefootnote}{1}\footnote{ल. अथ~।} वीणायां स्वराः सप्त प्रकीर्तिताः~।\\
 \renewcommand{\thefootnote}{2}\footnote{ज. तस्या}तेभ्यो \renewcommand{\thefootnote}{2a}\footnote{N. तेभ्योऽनुश्रितान्याह चातोद्यानि द्विजोत्तमाः~। \textendash\  ( V. ३३ cd. (N)}विनिःसृताश्रै\renewcommand{\thefootnote}{3}\footnote{र. न्याहुरातोद्यानि~।}वमातोद्येषु द्विजोत्तमाः~॥~३०~॥

 \renewcommand{\thefootnote}{4}\footnote{र सर्वे}पूर्वं शरीरादुभ्दूता\renewcommand{\thefootnote}{5}\footnote{ल. उत्पन्नाः~।}स्ततो गच्छन्ति दारवीम्~।\renewcommand{\thefootnote}{6}\footnote{N. र. पूर्यन्ते दारवीं ततः~। ज. संयान्ति दारवीम्~।}\\
 \renewcommand{\thefootnote}{7}\footnote{भ. स्वराः पुष्करजाः पश्चादनुयान्ति ध्वनिं युताः}ततः पुष्करजं \renewcommand{\thefootnote}{7a}\footnote{N.ततः पुष्करजं यस्मादनुयान्ति घनं युताः~। ( V. ३४ cd. (N)}चैवमनुयान्ति घनं (ध्वनिं) पुनः (युताः)~॥~३१~॥

 \renewcommand{\thefootnote}{8}\footnote{ल. एषां~। N. र. एतेषां करणैः भ. कार्याः प्रहाराः करणाश्रयाः~। ल. स्वमाश्रयाः}तेषां वाक्करणैर्ज्ञेयाः प्रहारा वचनाश्रयाः~।\\
 \renewcommand{\thefootnote}{9}\footnote{ज. झण्टुं जगति यद्युक्ता~। भ. संउद्या र. युञ्जो वा}झण्मण्या \renewcommand{\thefootnote}{9a}\footnote{N:युजो वाइति संजाता वीणावाद्यप्रयोक्वृभिः (V.35 cd. (N)}चेति (झण्टुं झाझेति) संयुक्ता\renewcommand{\thefootnote}{10}\footnote{र. संजाता} वीणावाद्यप्रयोगिनः\renewcommand{\thefootnote}{11}\footnote{र. क्तृभिः}~॥~३२~॥}
\end{quote}

\hrule

\vspace{2mm}
{\qtt तदसम्यक्}~। (ना) नाश्रय आतोद्य एवेति कश्चिदमंस्तेत्याह~। शारीर्यामेवेति~। {\qtt तत्रैवोक्तः}~। श्रुतिसम्बन्धः यतस्तेभ्य इति~। शार्ररेभ्यो विनिःसृता इति~। {\qtt तत्सामान्यादिति} भावः~। आतोद्येष्विति ततादिषु~॥~३०~॥\\

तदपि {\qtt गुणप्रधानतामाह}~। पूर्वं शरीरादुद्भूता उत्पन्ना अभिव्यक्ता वा ततो गच्छन्ति दारवीम्~। वीणायां पूर्णत्वात् तत्प्रा (पा) णिकृतायारक्तेः~। अनेन वैणवा {\qtt अप्युपलक्षिताः}~। ततः पुष्करजं चैवमनुयान्तीति~। वीणास्वरेषु यादृशी रक्तिः पूर्णतादृशी पुष्करेषु केवलं साम्यानुहारमात्रम्~। तदाह~। {\qtt ध्वनिमनुयान्तीति}~। \\

नन्वेवमनुहरणं साम्यस्य (दारव्या) दावप्यस्ति~। तत्कथमुक्तं स्वरा इति~। नैतत्~। तत्रैकमेवानुहारमात्रइ\textendash\  (मि) इति न स्वरानुहारत्वं तत्र स्फुटम्~। इह तु मार्जनायोगात् तत्पठनेन च सम्पादितत्रयान्तरसम्पत्तावप्येकस्त्रिष्विवान्यो भवन् नीचभावसंवेदाद् व्यक्तस्वरानुहारो मिश्रत्वात्~। तदाह~। युता इति~। सु......नत्वादित्यर्थः~॥~३१~॥\\

एवं त्रिपुष्करेमार्जनास्वरयोगं सूचयतः प्रहाराक्षरयोगमपि सूचयिष्यन् तत्प्रकृतौ. ....द्वीणायां दर्शयितुं सामान्योपक्रमं सूचयति~। {\qtt तेषां वाक्करणैर्ज्ञेयाः प्रहार (रा) वचना (श्रया)} इति~। तेषामित्यातोद्यानां परामर्शः~। आतोद्येष्विइति प्रक्रान्तत्वात्~। वाक्करणैस्तोभाक्षरेरूच्चद्धिर्ज्ञायते~। आतोद्यानां वचनाश्रयो  (य) वर्णानुहारसम्पादका हस्ताङ्गुलिप्रहारा भवन्तीति~। वाच एव क्रिया न त्वर्थं प्रतीत्यादि~। किश्चिदित्यत्र

\newpage
% चतुस्त्रिंशोऽध्यायः ४१७

\begin{quote}
{\na  शारीर्यामथ वीणायां झण्टु जगदि यादि च\renewcommand{\thefootnote}{1}\footnote{N र. पुण्यादायो भवेत्स्वनः}~॥\\
 \renewcommand{\thefootnote}{2}\footnote{र. N करणं तस्य विज्ञेयं~।}भवेद्वाक्करणं तत्र\renewcommand{\thefootnote}{3}\footnote{ज. विप्रा~। ल. तद्यत्~।} \renewcommand{\thefootnote}{3a}\footnote{N: नानाक्षरसमन्वितम्}नानाकरणसंयुतम्~॥~३३~॥

 \renewcommand{\thefootnote}{4}\footnote{र.N या हि गानाक्षरं (भ. गानस्वरं) ............थे~।}यं यं गाता स्वरं गच्छेत् तमातोद्यैः\renewcommand{\thefootnote}{4a}\footnote{N: तमातोये प्रयोजयेत्} प्रयोजयेत्~।\\
 \renewcommand{\thefootnote}{5}\footnote{भ.यदिपाद~। ज. यतिपाद}यतिपाणिसमायुक्तं गुरुलघ्वक्षरान्वितम्~॥~३४~॥}
\end{quote}

\hrule

\vspace{2mm}
\noindent
तद्वाक् वचनं करणं वर्णा इह तु तदनुहारः~। ननु कानि स्तोभाक्षराणि कुत्रेत्याह~। {\qtt झण्टुं झांझेति}~। न प्रकारे वि (हि) वीणावाद्यप्रयोगे विद्यते येषां हेतुनेति {\qtt वीणावाद्यप्रयोगिनः}~॥~३२~॥\\

ननु वीणायां वागभावाद्यावदक्षरमिति का नामेत्याशङ्क्याह~। {\qtt शारीर्यामथ}~। {\qtt झण्टुं जगति यादि} चेति~। आदिना दिगिनिगिझण्टुप्रभृतीनि~। ननु तुल्येऽङ्गुलिप्रहारे कथमकारादिस्तन्त्र्यां विशेष इत्यतः शिक्षां ददाति~।

\begin{quote}
{\qt भवेद्वाक्करणं तत्र नानाकरणसंयुतम्~।}
\end{quote}

\noindent
इति~। नखाङ्गुल्यग्रमध्यादिनाऽतीवतीव्रस्पर्शनं तत्र तस्यां तन्त्र्यादिभावेष्विति~। नानारूपक्रिया झण्टुमादिवाक्करणेन कार्यभूतेन संयुज्यते श्रुत्या~॥~३३~॥\\

अलं तदाह~।

\begin{quote}
{\qt  यं यं गाता स्वरं गच्छेत् तमातोद्यैः प्रयोजयेत्~।}
\end{quote}

\noindent
इति~। {\qtt आतोद्यैः} वीणादिभिः~। {\qtt यतिपाणयो} व्याख्याताः~। {\qtt परमगुरवश्च} प्रकृतं व्याचक्षते~।

\begin{quote}
{\na  गुरुलघ्वक्षरे गीयमाने अन्वितमाश्रितम्~।\\
 स्वन्तं स्वरं वादमानं वादयेदेव भासयेत्~।\\
 कथं यतिर्विरामः (स्यात्) तत्प्रधानेन पाणिना~॥}
\end{quote}

\noindent
इति पाणिप्रहारेण {\qtt संयुक्तम्~। पाणिना} वंशेषु फूत्कारो लक्ष्यते~। गुरौ गीयमाने गुरुणा विच्छेदौ लघोः शीघ्रतयेति~॥~३४~॥

\newpage
%४१८ नाटयशस्त्रे
\begin{quote}
{\na \renewcommand{\thefootnote}{1}\footnote{N.भ.साक्षरस्य}पौष्करस्य \renewcommand{\thefootnote}{1a}\footnote{N: च.}तु \renewcommand{\thefootnote}{2}\footnote{N र.नाट्यस्य}वाद्यस्य मृदङ्गपणवाश्रयम्~।\\
 विधानं सम्प्रवक्ष्यामि दर्दरस्य तथैव च\renewcommand{\thefootnote}{3}\footnote{भ. दर्दराश्रयमेव च}~॥~३५~॥

 षोडशाक्षरसम्पन्नं\renewcommand{\thefootnote}{4}\footnote{ज. निष्पन्नं र. N संयुक्तं~।} चतुर्मार्गं तथैव च~।\\
 \renewcommand{\thefootnote}{5}\footnote{N: द्विकलोऽयं}वि (द्वि) लेपनं षट्करणं त्रियति त्रिलयं तथा~॥~३६~॥}
\end{quote}

\hrule

\vspace{2mm}
ननु भवत्वेवं वीणादौ~। प्रकृते पुष्करे कथमेतदित्याह~। {\qtt पौष्करस्य~। तथैवेति~।} नैव स्वराक्षरं वादनप्रकारेण हेतुना पौष्करस्य लक्षणं वक्ष्यामि तदनुषङ्गी लाभश्च~। {\qtt दर्दरस्य मृदङ्गपणवाश्रयं} कृत्वा~। तेन मृदङ्गपणवस्य चानुषङ्गिणो {\qtt विधानं} स्वराक्षरसम्पादकं प्रकारं वक्ष्यामीतिसम्बन्धः~॥~३५~॥

\begin{quote}
{\qt षोडशाक्षरसम्पन्नं चतुर्मार्गं तथैव च~।}
\end{quote}

\noindent
इति~। अक्षरप्रकारेण मार्गच्छेद इति~। 

\begin{quote}
{\qt द्विलेपनं षट्करणं त्रियति त्रिलयं तथा~।\\
 त्रिगतं त्रिप्रकारं च त्रिसंयोगं त्रिपाणिकम्~।\\
 दशार्धपाणिप्रहतं त्रिप्रहारं त्रिमार्जनम्~।\\
 विंशत्यलङ्कारयुतं तथाष्टादशजातिकम्~॥}
\end{quote}

\noindent
इति~। तत्र योजनमाह~। एवं प्रकारैर्यस्मात् पुष्करजं वाद्यं परिपूर्णं तस्मान्नाट्ये रसभावाभिब्यञ्जनसमर्थमित्यतो भणनयोग्यमिति सम्भाव्यते~। तत्र पौष्करं वाद्यंप्रकारत्रयेण नाट्योपयोगिनि (गीति)~। वर्णानुसारेण स्वरानुसारेण साम्यात्मना ताले च~। तथा हि~। वर्णाः कस्मैचिदपि वाद्यविशेषतया प्रभवन्तोऽप्यज्ञातार्थथरुतिमात्रेणैव पुरुषं नागरग्राम्योपनागरिकात्मवृत्तित्रयाणि ताल (लय) भेदेषु दीप्तमध्यमसृणरूपसंविद्वृत्तिव्यञ्जका इति तावत् प्रसिद्धमेव~। ततश्च प्रागलङ्काराध्याये  (भ.ना.१६) निर्णीतमस्माभिः सहृदया लोकलोचने  (ध्वन्यालोकलोचनम् ३.२) च~। त एवत्वनुहाररूपतायां वर्णरूपत्वन्यक्करेण नादरूपतया संछादितात्मानो भवन्तो नादस्य लालित्यपारुष्यादितारतम्यभेदे नितरां द्विरावृत्तिमेवोद्वीपयन्ति~। नादो हि चित्तवृत्तेरन्तरङ्गः समयानपेक्षित एव~। लोकेऽपि चित्तवृत्ति (त्ते) र्व्यनक्ति~। संवित्प्राणितं तु वाक्तत्त्वस्य प्रथमं नादात्मना  विवृत्तस्य~। वर्णात्मा चरमो विवर्तो बहिरङ्ग~। यतः सा चेयं नादरूपतयैवं वक्ष्यमाणा पातहन्यमा (ना) न्यपुष्करमुखोदितान्येव चोच्चर्यमाणेषु वर्णेषु प्रयत्नशतैरपि न समपद्यते~। तत्र हि सर्वधा (था)

\newpage
% चतुस्त्रिंशोऽध्यायः ४१९

\begin{quote}
{\na  त्रिगतं त्रिप्रका (चा) रं\renewcommand{\thefootnote}{1}\footnote{ल. चारं} च त्रिसंयोगं \renewcommand{\thefootnote}{1a}\footnote{N. त्रिप्रयोगं}त्रिपाणिकम्~।\\
 \renewcommand{\thefootnote}{1b}\footnote{N. दशार्धं प्रहतं चैव}दशार्धपाणिप्रहतं त्रिप्रहारं त्रिमार्जनम्~॥~३७~॥

 \renewcommand{\thefootnote}{2}\footnote{ज. विंशत्प्रकारालंकारम्}विंशत्यलङ्कारयुतं तथाष्टादशजातिकम्~।\\
 \renewcommand{\thefootnote}{3}\footnote{ज. एवमेतैस्तु}एभिः \renewcommand{\thefootnote}{3a}\footnote{N.एभिर्गुणैस्तु}प्रकारैः सम्पन्नं वाद्यं पुष्करजं भवेत्~॥~३८~॥

 तत्र षोडशाक्षरमिति यदुक्तं तदनुव्याख्यास्यामः~।\\
 कखगघटठडढतथदधमरलह इति षोडशाक्षराणीह~।\renewcommand{\thefootnote}{4a}\footnote{N: 'इह' not read in (N)}\\
 \renewcommand{\thefootnote}{4}\footnote{ज. तज्ज्ञैः}नियत \renewcommand{\thefootnote}{4b}\footnote{N: पुष्करवाद्यसंविधेयानि}पुष्करवाद्ये वाक्करणैः संविधेयानि~॥~३९~॥}
\end{quote}

\hrule

\vspace{2mm}
\noindent
वर्णप्राधान्यमेवोन्मीलति~। नादप्राधान्या............उच्चार्यमाणं (र्णा)  वर्णा वाद्यमानं तु वादिभूतवर्णमयं सन्ध्यायां  शाबलेयादय इव गो........भूताः~। अत एवोच्चार्यमाणं वाद्यमपि प्रसिद्धमपि तूद्धट्टयति~। उद्धट्टनं सूत्रमुच्यते~। सोऽयं वर्णान्तरानुहारभागो रसभावप्रकृत्यादिभिः सूत्रतया विमलमत (तम) दृष्टिना मुनिना ज्ञात्वा तथैवोपदिष्टः~। तत्र वर्णस्वरूपं पुष्करहस्तयोगजन्म~। तत्र च पुष्करगतः संस्कारो लेपद्वये आलिङ्गके ऊर्ध्वके च~। निगृहीतं यत्ताडनसमनन्तरंपीडितम्~। अनुरणनक्षणे यत्तन्नि (दर्धनि) गृहीतम्~। विश्रम्य (विश्रम्य) पीडितमेतदनुरणनसिद्धये~। मुक्तं च पीडितसम्पूर्णानुरणनसम्पत्यै~। इत्येतत् प्रहारत्रयम्~।  (पञ्चपाणिप्रहतं) समार्धतदर्धपार्श्वप्रदेशिन्यात्मनि हस्तसन्निवेशात्मनि करणरूपे घातपञ्चके इतिकर्त्तव्यतां संपूरयद्धस्तस्य संस्कार इति लेपघातप्रकारप्रहारवर्णानुहारः~। प्रत्युपयोगिनां वर्णानामेव त्रिय (ग) तम्~। संयोगः मार्गे आलिप्तादिश्चतुर्विधः~। {\qtt गुरुलघुमिश्रतावर्णस्वरूपः}~। निगृहीतादित्रयमिति संयोगत्रयवर्णस्वरूपतापत्यैव~। प्रचारत्रयस्य मार्गचतुष्टय एवोपयुज्यते~। हस्तसाम्येन वैषम्येणोभयथा चेति~। यदा मार्गभेदस्तावत् सिद्धस्तज्जातयोऽष्टादश शुद्धा दुष्करकरणेत्यादयः~। यद्यपि चासां समस्तमेवैतदुद्देशगतं (लखुप्लुत) लययुताद्यधिगमनरूपमत एव हि सर्वं भीलनाज्जायत इति निर्वचनं चासामेव प्रयोगपर्यन्तम्~। तथापि मार्गारब्धमासां शरीरमिति मार्गजास्ता उक्ताः~। तेष्वेव वर्णेषु पणवदर्दरमृदङ्गं वाद्ययोजनया वैचित्र्योत्पादककरणः (म्)~। यद्रूपकृतिप्रकृतमित्यादि करणषट्कम्~। अत एव प्रचारत्रयसहितमार्गाभिधानादनन्तरं पणवदर्दरलक्षणमकालकूष्माण्डपतनदेशीयमिति न शक्यते~।\\

{\qt षोडशाक्षरसम्पन्नं चतुर्मार्गं द्विलेपं षट्करणं त्रिप्रचारं त्रिसंयोगं दशार्धपाणिप्रहतं}

\newpage
% ४२० नाट्यशास्त्रे 

\begin{quote}
{\na  चतुर्मार्गं नाम~। आलिप्ताङ्कितगोमुखवितस्ताश्चत्वारो मार्गाः~। \\
 \renewcommand{\thefootnote}{1a}\footnote{N: द्विकलेयं नाम~। वामार्धकाभ्यां कर्तव्यम्~। (V.N.2)}द्विलेपं नाम~। चैव (वा) मोर्ध्वकप्रलेपात्~।\renewcommand{\thefootnote}{1}\footnote{र. कर्तव्यम्}\\
 षट्करणं नाम~। रूपं कृतप्रतिकृतं प्रतिभेदो रूपशेषमोघः प्रतिशुल्का चेति~। \\
 त्रियतिर्नाम~। समा स्रोतोगता गोपुच्छा चेत्यन्वयात्~।\renewcommand{\thefootnote}{1b}\footnote{N: अन्वयात् not read in 'N'}\\
 त्रिलयं नाम~। द्रुतमध्यविलम्बितयोगात्~।\renewcommand{\thefootnote}{1c}\footnote{N: द्रुतमध्यविलम्बितास्तु ये लयाः}\\
 त्रिग \renewcommand{\thefootnote}{2}\footnote{भ. गतिः}तं नाम~। तत्त्वमनुगतमोघश्चेति~।\\
 त्रिप्रचारं नाम~। समप्रचारो विषमप्रचारः \renewcommand{\thefootnote}{3}\footnote{र. उभय~।}समविषमप्रचारश्चेति~।\\
 त्रिसंयोग नाम~। गुरुसंयोगो लघुसंयोगो गुरुलघुसंयोग\renewcommand{\thefootnote}{4}\footnote{ज.संचयः}श्चेति~।\\
 त्रिपाणिकं नाम~। समपाणिरवपाणिरुपरिपाणिश्चेति~।\\
 पञ्चपाणिप्रहतं नाम~। समपाण्यर्धसमपाणिरर्धार्धसमपाणिः\\
 पार्श्वपाणिः प्रदेशिनी चेति~।\\
 त्रिप्रहारं नाम~। निर्गृहीतोऽर्धनिगृहीतो मुक्तश्चेति~।\\
 त्रिमार्जनं नाम~। मायूर्यर्धमायूरी कार्मारवी चेति~।\\
 \renewcommand{\thefootnote}{5}\footnote{N विंशति प्रकाराः अष्टादशजातयश्च (N)}विंशत्यलङ्काराः (रान्) अष्टादशजावि (ती) श्च पदभेदे दर्शयिष्यामः~॥~४०~॥}
\end{quote}

\hrule

\vspace{2mm}
\noindent
{\qtt त्रिप्रकारमष्टादशजातिकमिति}~। यस्तूद्देशभागे वर्णानुसंहारविषयो यस्तुस्वरानुहारांशस्तत्र द्विलेपमिति~। स्वरूपसिध्यै त्रिमार्जनमिति~। स्वरसंयोजनांशे नादीभूतवर्णसम्पत्यनुभावि सदनुरणनरूपं तद्गानांशस्थितस्थायित्वं स्वरूपमनुवर्तमानं तत एव तत्कृतचित्तवृत्त्युद्रीपन एव सुतरामात्मानमाश्रयतीति यत्तत उपदिश्यमानस्वरानुहारांश इति प्रकारत्रयम्~। घातपञ्चकादेरत्र यत्रीपयोग एव~। गतयो लयास्तत्त्वादीनि त्रिगतम्~। समपाण्यादित्रयमिति साम्यांशात्ताल उपयुज्यते~। साम्यं यत्र यस्यान्योन्यमृदङ्गाद्यङ्गैर्झार्झार्यादिप्रत्यङ्गैस्ततादिभिरातोद्यान्तरैर्ध्रुवागानेन परिक्रमणादिपात्रचेष्टितेनेति तद्विचित्रीकरणे विंशतिरलङ्कारा इति~। एवं पुष्करवाद्यमनुरञ्जकत्वेऽपि विवाह ................ शोभामावहतीति~। तथा च नन्दिमते~।

\begin{quote}
{\qt  न पुष्करविहीनं हि वाद्यवृत्तं विराजते~।\\
 तत्रैव हि श्रुते लोक उन्मुखत्वं प्रपद्यते~॥}
\end{quote}

\noindent
इति~। विभक्तविषयत्वेऽपि मिश्रितत्वेन प्रयोगे तत्सम्पादनं न हि विभज्येति प्रतिपादयितुं मुनिः शङ्करे (संक्षेपे) णोददिक्षत्~॥~३६ \textendash\ ४०~॥

\newpage
% चतुस्तिंशोऽध्यायः ४२१ 

\begin{quote}
{\na  एतावत् सूत्रम्~। अतः पदभेदः\renewcommand{\thefootnote}{1}\footnote{र. N भेद वर्णयिष्यामः}~। तत्र षोडशाक्षरं नाम~।\\
 \renewcommand{\thefootnote}{2}\footnote{ज. पूर्वमभि}अभिव्यञ्जितानि पूर्वं यान्येतानि षोडशाद्यानि~॥

 \renewcommand{\thefootnote}{3}\footnote{N र. व्यक्तानि षोडशानि  (शाद्यानि (N)~।}तान्यक्षराणि जानीत पुष्करेषु यथाक्रमम्\renewcommand{\thefootnote}{4}\footnote{भ. यथाविधि~।}~।\\
 ऋपणवे दर्दरे चैव \renewcommand{\thefootnote}{5}\footnote{N र. कीर्तितानि मनीषिभिः}मृदङ्गेषु तथैव च~॥~४१~॥

 कखतथभा (टढतथरा) स्तु\renewcommand{\thefootnote}{6}\footnote{ज. रटास्तु~। ल. कटरथतरास्तु~। र. तटरथना अङ्कमुखे सव्ये} दक्षिणमुखेऽत्र \renewcommand{\thefootnote}{7}\footnote{र. डधमाश्च~। ज. डधमह इति ह वामे स्युः}घघम (घम) हाश्च  वामके नियताः~। \\
 गदकारौ चैवोर्ध्वे\renewcommand{\thefootnote}{8}\footnote{भ. ऊर्ध्वगतौ} (चोर्ध्वाख्ये) \renewcommand{\thefootnote}{9}\footnote{ज. गढरलभा~। भ. ठढणमाश्चेति चालिङ्गे~।}दठडोण (खठडध) लाः स्युरालिङ्गे~॥~४२~॥\renewcommand{\thefootnote}{9a}\footnote{N: (V.४३ N:) तटरथना अङ्कमुखे सव्येऽद्यधसाश्वा वामके नियतं~। गदकारौ चैवोर्चे उखणलाश्च स्युरालिङ्गे~॥}}
\end{quote}

{\na एतेषामक्षराणां \renewcommand{\thefootnote}{9b}\footnote{N: स्वरसंघातं व्यञ्जनसंघातं}स्वरसंयोगं व्यञ्जनसंयोगं च व्याख्यास्यामः~।\\
 तत्र अ आ इ ई उ (इ उ) ए ओ अं \renewcommand{\thefootnote}{10}\footnote{इत्येते}इति स्वरा व्यञ्जनैः सह संयोगं गच्छन्ति~।}

\vspace{2mm}
\hrule

\vspace{2mm}
तदत्र वस्तुतत्त्वं शिष्टं ग्रन्थयोजनामात्रम्~। तदाह मुनिः~। {\qtt एतावत् सूत्रमिति}~। स (सा) मात्यौ......नेत्याह~। अतः पदभेद इति~। सूत्रेऽस्मिन् यानि पदानीत्यवान्तरवाक्यानि तानि व्याख्यायन्तेऽस्मिन्निति स ग्रन्थस्तथोक्तः~। तत्र पणवदर्दरमृदङ्गेष्वक्षरविभागं पुष्करेषु तावदाह~॥~४१~॥

\begin{quote}
{\qt  कटढतथरास्तु दक्षिणमुखेऽत्र घमहाश्च वामके नियताः~।\\
 गदकारीौ चोर्ध्वाख्ये कढणद (खठडध) लाश्च स्युरालिङ्गे~॥}
\end{quote}

\noindent
दक्षिणस्य पुष्करस्य सव्यतो वामतश्चेति मुखद्वयम्~। अत एव सयवाङ्को लक्षणमस्तीत्यङ्कतिकः~। तस्य दक्षिणस्य {\qtt दक्षिणमुखे} षड् वर्णाः~। वामे त्रयः~। मध्यमे तु प्रधानत्वादङ्गिकालिङ्गाभ्यां सार्धं त्रयम्~। तालाभ्यामाधिक्येन वाद्यमालिङ्ग्यकरणत्वबहुलत्वमेव तद्वाद्यम्~। न तु तच्छून्यं कदाचित्~। तत्रालिङ्गके पञ्च~। एवं षोडश~॥~४२~॥\\

तेषां स्वरसंयोगमाह~। तत्रेत्यादिना हकारमा (म) कारौ शुद्भावित्यन्तेन~। ऋलृऐऔअः इत्येतद्वर्जम्~। 

\newpage
%४२२ नाट्यशास्त्रे

\noindent
{\qt \renewcommand{\thefootnote}{1}\footnote{(N) pp. ४१६ \textendash\ ४२१ (G.O.S.) VS ४३ \textendash\ ५९ = VS ४४ \textendash\ ५४ (N) But the text is very different , so the full text is given in Appendix  \textendash\  III}अकारेकारोकारैकारौकाराङ्कारा इति ककारे~। यथा~। ककिकुकेकोकमिति~।\\
इकारैकारौकारा इति खकारे~। खिखेखो इति~।}

\begin{quote}
{\na  उकारैकारौ गकारे~। यथा गुगे इति~। \\
 अकारेकारोकारौकारा इति घकारे~। यथा घधिघुघो इति~। \\
 अकाराकारेकारौकाराङ्कारा (इति) टकारे~। यथा टटाटिटोटमिति~। \\
 अकाराकारेकारैकारौकाराङ्कारा इति ठकारे~। यथा ठठाठिठेठोठमिति~। \\
 एकारौकारौ ढकारे~। यथा ढेढो इति~।\\
 अकाराकारैकारौकारा इति णकारे~। यथा णणाणेणो इति~।\\
 अकाराकारेकारैकारा इति तथयोः~। यथा ततातिते थथाधथिथे इति~।\\
 अकारोकारैकारौकारा इति दकारे~। यथा ददुदेदो इति~।\\
 अकाराकारेकारैकारौकाराङ्कारा इति धकारे~। यथा धधाधिधे (धोधमिति)~।\\
 आ (अकारा) कारेकारैकारा इति रेफे~। यथा ररारिरे इति~।\\
 अका (राका) रेकारैकारा इति लकारे~। यथा ललालिले इति~।\\
 हकारमकारौ शुद्धौ~॥~४३~॥}
\end{quote}

{\qt ककारगकारघकारतकारदकारधकाराणांरेफोऽनुबन्धः~। यथा क्र ग्र घ्र त्र द्र प्र~। ककारस्य लकारोऽनुबन्धः~। यथा क्ल क्ले कृमिति~। हकारस्य णकारोऽनुबन्धः~। यथा ह्ण इति~। तकारस्य थकारोऽनुबन्धः~। यथा त्थ इति~। दकारस्य धकारोऽनुबन्धः~। यथा द्धं द्धा द्धे इति~। एवमेतैः संयोगैर्द्विस्तसंयुक्तान्यक्षराणि भवन्ति~॥~४४~॥}

\vspace{2mm}
\hrule

\vspace{2mm}
\noindent
अत्र स्वरविनियोग इति लक्ष्ये हि हडदित्यादौ टकारो त मन्तव्यः~। हकारश्च (स्य) शुद्धस्याप्यभिधानाल्लक्ष्ये तर्हि गहतदित्यादौ गन्तव्यमित्यादि शाड्य (वाद्य) म्~। अग्रे च यद्वाद्ये कुतूहलइति कर्तव्यम्~। न तु कुतूहलमिति शिष्टं स्पष्टम्~॥~४३~॥\\

अथ व्यञ्जनसंयोगमाह~। {\qtt ककारेत्या}दिना {\qtt एवमेतैः संयोगैरित्यन्तेन}~। {\qtt संयोगैरि}इति भावसाधनम्~। एतद्योजनोपलक्षितानि यान्यक्षराणि तानि हस्तद्वयसंयोगजानि~। व्यापाराद्वृत्तिभेदाद्वहुहस्तसंयोगजत्वमपि सम्भाव्यमपीति द्विग्रहणम्~॥~४४~॥

\newpage
% चतुस्त्रिंशोऽध्यायः ४२३ 

{\qt तत्रैत एव द्विपुष्करे~। आङ्किकमृदङ्गे द्विपुष्करे समहस्तनिपातनाद्धकारः~। तत्रैवाङ्गुलिप्रचलनाद् ध्रकारः~। तत्रैवावष्टम्भात् तत्रैवार्धनिगृहीतात् स्थ (थ) कारः~। तत्रैव दक्षिणमुखे पार्ष्णिनिपीडिते ककारः~। तत्रैवाङ्गुलिकुञ्चनात् कुकारः~। ऊर्ध्वकवामकयोः समहस्तनिपातनाद्धंकारः~। प्रदेशिन्या चालिङ्गे क्रेङ्कारः~॥~४५~॥}\\

{\qt पञ्चपाणिप्रहतमिति यदुक्तं समपाण्यवपाण्यर्धार्धपाणिपार्श्वपाणिप्रदेशिन्यश्चेति~। त (य) एतेपञ्चपाणिप्रहताके (ते) निगृहीतार्धनिगृहीतयुक्ता यथायोगं कार्याः~। तत्र समपाणिप्रहतो मकारः स निरगृहीतः~। गकारधकारदकारपकारा अर्धपाणिप्रहता अर्धनिगृहीताः~। ककारखकारटकारदकाराः पार्श्वपाणिप्रहता निगृहीताश्च~। तकारथकारावर्धार्धपाणिहतावर्धनिगृहीतौ~। प्रदेशिन्या हता आलिङ्गेढकारणका (र) रेफवकारायुक्ताः~। द्विहस्तप्रहता ध्रुद्रोंक्ले इति~। मुक्तपार्श्नथिक्रा इत्यर्धपाणिहता निरगृहीता (:)~। एवमेतेष्वक्षरेषुप्रयोगवशेनकार्या प्रहाराः~॥~४६~॥}

\begin{quote}
{\na  षोडशैतानि दृष्टानि वाद्यजान्यक्षराणि तु~।\\
 अनेनैव विधानेन योज्यं वाक्करणं बुधैः~॥~४७~॥}
\end{quote}

\hrule

\vspace{2mm}
अनुबन्धशब्दत्वेन मिश्रत्वेन वैचित्र्यमाह~। {\qtt एत एव हि (द्वि) पुष्करे द्विमुखेऽङ्गिक} इति~। मुखद्वये  समं कृत्वा हस्तनिपातादिति समग्रहणं क्रमनिषेधार्थम्~। अत एवाव्यक्तपूर्णान्तरमेवेदं केवलमुद्धट्टने~। तत्रैवेत्यङ्किके दक्षिणमुखे द्वयमित्युक्तम्~॥~४५~॥\\

{\qtt षोडशाक्षरमिति निर्णीय} तदुपयोगि दशार्धपाणिप्रहतमिति निर्णेतुमाह~। {\qtt पञ्चपाणीति~।} पञ्चसन्निवेशैः पाणिभिः प्रयोगसिद्धये हतं हननं येषां ते वर्गा निगृहीतादित्रयरूपाः कार्या इति~। त्रिप्रकारस्येतिकर्तव्यतारूपतामाह~। {\qtt यथायोगमिति~।} व्याचष्टे समपाणिप्रहतमि (त इ) त्यादिना~। अर्धपाणिप्रहतार्धनिगृहीत इति तुल्यत्वेऽपि पुष्करभेदात् स (ग) कारादीनां भेदः~॥~४६~॥\\

ननु तत्र यदि वर्णानुहारस्वरभवा एतेऽन्य एव तथा सति संयोगजानां ध्वनीनां {\qtt परमार्थतोऽन्यत्वमेव}~। तथा रूपादिकरणषट्केन दर्दरपणवमृदङ्गध्वनेः पुष्करध्वनिमेलने वक्ष्यमाणे वस्तुतोऽन्य एव विचित्रो निनाद 

\newpage
% ४२४ नाट्यशास्त्रे 

\begin{quote}
{\na  चतुमार्गं यदुक्तं तमनुव्याख्यास्यामः~।\\
 अड्डितालिप्तमार्गे (र्गौ) तु वितस्ता गोमुखी तथा~।\\
 मार्गाश्चत्वार एवैते प्रहारकरणाश्रयाः~॥~४८~॥}
\end{quote}

{\qt तत्र किङ्क (त्राङ्किक) मृदङ्गप्रहारयुक्तोऽड्डितमार्गो वामोर्ध्वकप्रहारयुक्त आलिप्तमार्गः~। ऊर्ध्वकाङ्किके दक्षिणमुखे आलिप्तहस्तो वितस्तमार्गः~। आलिङ्गककरणबहुलः सर्वपुष्करहतो गोमुखीमार्गं इति~॥~४९~॥}

\vspace{2mm}
\hrule

\vspace{2mm}
इति किमर्थोऽयं षोडशाक्षरनियमप्रयास इत्याशङ्कयाह~। {\qtt षोडशैतानीति}~। {\qtt तुर्हेतौ}~। यस्मादनेन षोडशाक्षरात्मना विधानेन वाक्करणं वाचां ताल्वोरुद्धदृनं बुधैरनुहारसादृश्यविवेककुशलैर्योजयितुं शक्यते~। तद्योजनया च........ लिना तद्वाच्याभ्याससम्पादनमशक्यत्वात् प्रयोगविच्छेदमित्यवश्यं योजनार्हम्~। एतैरेव षोडशभिरुद्धट्टनिकारूपैः सर्वमेव वाक्करणं योजनार्हम्~। अधिकान्यक्षराणि नापेक्षत इति~। ततो हेतोरेतानि वाद्याज्जातानि कल्पनया तत आ (अ) पोद्धारेण लब्धस्वरूपाणि षोडशाक्षराणि दृष्टानीति सम्बन्धः~॥~४७~॥\\

अथैषां वर्णानां चतुर्धा समूहीभावं रसविशेष उपयोजयितुं दर्शयितुं प्रक्मते~। {\qtt चतुर्मार्गमिति~। यदुक्तमिति~।} उद्देशसूत्रे मार्ग आकाङ्क्षा तया सम्बन्धनं समूहीभावात्मा लक्ष्यते~। स चतुर्धा~। एको द्विक इत्यङ्गिकमुखम्~। तच्च प्रथमो (ममु) क्तम्~। कटरथकटस्थानं मृदङ्गपूर्विकमङ्गम्~। इह यत्र घटकारय (यु) क्तैरेतद्गतैः प्रहारैर्युक्तो द्विको मार्गः~॥~४८~॥\\

अड्डिता धुवा शृङ्गारविषया ध्रुवाध्याये (भ.ना ३२) दर्शिता~। तत्रोचिते कृत्वाऽङ्गि (ङ्किः) कस्य यद्वाममुखं यत्र घहमाश्च वामकमित्युक्तम्~। यच्चोर्ध्वकं कटस्थानं तत्प्रहारयुक्त आलिप्तमार्गः~। शोकातुरो हि शरीरपरिकर्मविरहान् (होन्) मुख आलिप्तः~। {\qtt तदुचितकरणोचितश्चायमिति}~। ऊर्ध्वके पदस्थानेऽङ्किकस्य दक्षिणमुखे करटशतद्वये उक्तिरसातिशयवेगताडनहस्ताभ्यां प्रहारं प्रकृष्टहननं यत्र स वितस्तमार्गश्च~। अस्य दक्षिणो स (वित) स्तस्येति नाशयति तस्य भावोचितस्तां हिंस्रतां तद्विषये दीनरौद्राभावस्य विनियोगो यतः~। वितस्तिमात्राक्षिप्तहस्तप्रहरणसूचनार्थं वितस्तमार्ग इत्यन्ये पठन्ति~। सर्वेषु पुष्करेषु यद्धननं बाहुल्येन चाङ्गि (ङ्कि) क एवनोर्ध्वकस्थानेस गोमुखीनः~। गोमुखमन्तर्वक्त्रं न तूर्ध्वम्~। तत्र बीभत्सयोग इत्पक्षरसङ्गतिरपि गोमुखीत्युक्ता~॥~४९~॥

\newpage
% चतुस्त्रिंशोऽध्यायः ४२५ 

{\qt तत्राड्डित (ता) प्रहारजातम्~। मटकटथिधधटधेधोधहमंधि धंधनधिधि इत्यड्डितामार्गः~। धडः गुटुगुटमधेदोधिंधदुधिदुधंधि (इत्यालिप्तमार्गः)~। किंकाकिटुमेटकितां किंकेकितांद... तसितां गुुगेत्येवं ज्ञेयो वितस्त्यास्तु~। शुद्धं सिद्धं मद्धिकुटधेधेमत्थिद्धिधखुखुणंधेधोटत्थिमट गोमुखीमार्गः~॥~५०~॥}

\begin{quote}
{\na  कुर्याद्वितस्तिमार्गे थरविवर्जितान् प्रहारांस्तु~।\\
 उद्धतमार्गेण विना शेषाः कार्यास्तु गोमुख्याम्~॥~५१~॥

 द्रोमांस्तु गुटुगुटुधेद्रथणैघटकेदोराडिम्~।\\
 एवं वि (तस्त) वाद्यं विज्ञेयं वादकैस्तज्ज्ञैः~।\\
 ठिणिखिठिणिहो डडडधो गोमुख्याम्~॥~५२~॥

 ये त्वालिप्तसमुत्था (:) सर्वमार्गैस्तु ते विधातव्या (:)~।\\
 ग्रहगो न लक्ष्यसत्त्वो नाट्यविधानं समासाद्य~।\\
 ग्रहमोक्षणसन्धानैस्तु मृदङ्गानां ग्रहो भवति~॥~५३~॥}
\end{quote}

\hrule

\vspace{2mm}
\noindent
तत्र मार्गचतुष्ट्ये तान्यक्षराणि दर्शयति~। {\qtt तत्राड्डिताप्रहारजातमित्यादिना} गोमुखीमार्ग इत्यन्तेन~॥~५०~॥\\

अथात्रैव मार्गचतुष्टयेनेत्युक्तम्~। {\qtt विशेषमाह~। कुर्यादिइति}~। आङ्गिकदक्षिणमुखगतो रेफो वितस्तमार्गः~। प्राप्ता ऊर्ध्वककटताश्च दकारस्तालभेदं निषिध्यते~। नकाराकारस्यालिङ्गकेऽप्यप्राप्तस्य हमयोश्चाङ्गिकल (स) ममुखगतयोरप्राप्तयोर्निषिधः~। मध्येऽस्य तद्वर्गप्रक्षेपाद्यज्ञाय तदाहरणम्~। अवर्जितान् प्रहारां स्त्वड्डितमार्गेणोपलक्ष्य (क्षि) ते वितस्ते कुर्यात्~। मत्वर्थीयोऽत्र चयोगे~। ते शेषा इह वर्जिता वर्णास्ते गोमुख्यां कार्या इति~। प्राप्तानामप्येषां पुनर्विधानं भूयस्त्वं ख्यापयति~॥~५१~॥\\

आलिप्तमार्गे रेफस्य पूर्वलक्षणाप्राप्तिरित्यन्यानुबन्धत्वेन रेफावेशेनात्र भवतीत्युदाहरणव्याजेनाह~।  {\qtt द्रोमामित्यादिना}~। गोमुखीमार्गे सर्वेषां रवादीनामविशेषे प्राप्ते ड (ङ) कारोऽन्येन तुल्य इत्युदाहरणछद्मना दर्शयति प्रणव इत्यादिना~॥~५२~॥\\

अड्डितावर्जनमार्गत्रयमुक्त्वा विशेषान्तरमपि परस्परानुग्रहतो भवतीति दर्शयितुमाह~। ये त्वालिप्तसमुत्था इति~। आलिप्तसमुत्थानमाकर्मभ्यो येषां ते~। आलिप्तादयस्त्रय इति यावत्~।तेन सर्वे त्रयोऽपि मार्गा इत्यर्थः~। {\qtt विधातव्याः}~। 

\newpage
% ४२६  नाट्यशास्त्रे

\begin{quote}
{\na  एतेषां चैव वक्ष्यामि दर्शनानि यथाक्रमम्~।\\
 चतुर्णामपि मार्गाणामक्षरग्रहणं यथा~॥~५४~॥

 दाघददथिमटां धीमटां दिथिथिक्लं थिक्लाथिंतांकलाथिकटाम्~।\\
 कटधिमटां खोखोघेटामड्डितावाद्यम्~॥~५५~॥

 तन्त्रान्तिकिता धंधंद्रघटितयेटम्~।\\
 मटथकिकेत्त (टाकुट्टिकिकिद) वितस्तायाम्~॥~५६~॥

 घ्रांमांधुटूघेधेटाघटितकथिथिधोटामाम्~।\\
 आलिप्तकसंयोगो (गः) कार्ये (र्यः) संवादने सम्यक्~॥~५७~॥

 ध्रटमिथिधेटांधेधेतमथिधोणाखमत्थिधक्लेताम्~।\\
 खोखोथाथा णाणाणाणा च गोमुख्याम्~॥~५८~॥}
\end{quote}

\hrule

\begin{quote}
{\qt ग्रहमोक्षणसन्धानैस्तु मृदङ्गानां ग्रहो भवति~।}
\end{quote}

\noindent
इति~। अयमर्थः~। एकं मृदङ्गं यदा वादयितुं प्राधान्येन स्वीक्रियते तदा मृदङ्गान्तराद् द्रुतं कुर्यात्~। तद्यथा~। ऊर्ध्वकाङ्ककालिङ्गका दक्षिणमुखाद्यतो (मुक्ति) रिति~। अथ तत्र मोक्षस्तदा (दिग्रहा) दिति सन्धानम्~। तत्र व्यामिश्रत्वेन वर्णस्य जननं विचित्रमेव लिपौ दर्शयितुमशक्यमेव~। एवं ग्रहमोक्षसन्धानानामेकभेदा (स्त्र) यो द्विभेदाः षट् त्रिभेदाश्च {\qtt षडिति}~। यथा पञ्चदशभिर्भदैः पुनर्मार्गाद्यैर्वैचित्र्यम्~। यथोक्तं {\qtt नन्दिमते}~।

\begin{quote}
{\qt षोडशस्वपि वर्णेषु भेदाः पञ्चदशोदिताः~।\\
 ताडने ग्रहसन्धानमोक्षैर्मुखचतुष्टये~॥}
\end{quote}

\noindent
इति~। अन्ये तु चत्वारो मार्गा आलिप्तसमुत्था इत्यनेन स्वीकृताः~। सूत्रे तेषां तदादित्वेनोक्तया नीत्याऽऽचक्षते~॥~५३~॥\\

अथास्यमार्गचतुष्टयस्य लक्ष्यसमवायितमुदाहरणभेदमाह~। {\qtt दर्शनानीति}~। निर्दिश्यते लक्षणं पूर्वा............तं येषु~। पूर्वं हि रूपोदाहरणमात्रं दत्तमित्यपौनरुवत्यम्~। {\qtt तदाह~। यथाक्रममिति}~। यावल्लक्षणं तावदुदाहरणं दत्तमिति यावत्~। प्रायशोऽत्राभिरेव ग्रन्थकृता लक्षणाद्यं दर्शितमिति भङो (ङ्गो) यथा भवति तथा पाठे यतितव्य (म्)~।....इत्यादिना ग्रहणमोक्षेण दर्शितौ सन्धानत्वलक्षणमित्युक्तम्~॥~५४ \textendash\ ५८~॥

\newpage
% चतुस्त्रिंशोऽध्यायः  ४२७

\begin{quote}
{\na एतेषां पुष्कराणां त्रिविधः प्रचारः~। समप्रचारो विषमप्रचारः समविषमप्रचार इति~॥~५९~॥

तत्र~।

वामोर्ध्वकयोर्वामः सव्यो वै दक्षिणोर्ध्वके चापि का (के का) र्यः~।\\
\renewcommand{\thefootnote}{1}\footnote{N: (V. ५५ cd \textendash\ N) कार्यौ हस्तौ प्रणटावालिप्त वाद्ये करणे तु~।}समप्रचारे ह्यालिप्ते लिप्तवाद्यकरणे तु~॥~६०~॥

वामोर्ध्वकसव्यानां प्रहतो वामः करस्तु कर्तव्यः~।\\
सव्योर्ध्वकसंयोगात् प्रहतो हस्तप्रचारे तु~॥~६१~॥

स्वच्छन्दकः\renewcommand{\thefootnote}{2}\footnote{N. स्वच्छन्दतः} (कं) कराणां प्रहतं शेषेषु मार्गकरणेषु~।\\
अड्डितगोमुखयोगे समविषमो हस्तसञ्चारः~॥~६२~॥}
\end{quote}

\hrule

\vspace{2mm}
अथ मार्गशेषभूतमेव {\qtt त्रिप्रचारमिति} व्याचष्टे~। {\qtt त्रिविधः} प्रचारमि  (र इ) इति~। वामस्य हस्तस्य वामभागे दक्षिणस्य दक्षिणे प्रचरणं स समप्रचारः~। {\qtt वैपरीत्येन} स्वस्तिकवर्तनादिना विषमप्रचारः~। {\qtt मिश्रतायां} समविषमः~। तेन च वर्णानामेव तीव्रादिरूपजननं न तथा वैचित्र्यान्तरमाधीयत इति मार्गशेषता~॥~५९~॥\\

तत्र मार्गभेदेन योजनमाह~। {\qtt वामोर्ध्वकयोरिति}~। आलिप्ते मार्गे वाद्यक्रियायाः कर्तव्यतायां हस्तप्रभृतिः~। तस्याः कर्तव्यतायामयं विधिः~। {\qtt वामोर्ध्वकप्रहार} \textendash\  इति~। आलिप्त इति तावदुक्तम्~। तत्र वामोर्ध्वकयोर्मध्यमाद्वामेऽङ्किके वाममुखे सव्येन गते सति ऊर्ध्वकेनापि दक्षिण एव हस्तः कार्यं इत्येकहस्तः समप्रचार आलिप्ते~। तुरप्यर्थे~॥~६०~॥\\

अथ यथाग्रहसन्धानमोक्षरचनं तथा विधिमाह~। {\qtt वामोर्ध्वकसव्यानामिति}~। ग्रहावेशाद्वामोर्ध्वकसंयोगमवलम्ब्य हस्तप्रचारे व्याप्रियमाणे हस्ते हस्तं प्रवृत्ते सति त्रयाणामपि पुष्कराणां प्रघाताय वामो हस्त इति द्वितीय एव हस्तः समप्रचारः~॥~६१~॥\\

शेषा ग्रहादिविहीना ये मार्गमेदकरणभेदा मार्गेषु क्रियाभेदास्तेषु करणांशसम्बन्धि हननं यथारुचि तेन तत्र हस्तद्वयप्रचारोऽपि~। एवं {\qtt सम्प्रचारस्त्रिविधः}~। अड्डितगोमुखयोस्तु समविषमो हस्तप्रचारः~। तेषु तु विषम एवेत्यर्थाद् दर्शितं भवति~। करणेष्विति बहुत्वं व्यापारभेदात्~॥~६२~॥

\newpage
% ४२८ नाट्यशास्त्रे

\begin{quote}
{\na  शृङ्गारहास्ययोगे वाद्यं योज्यं तथाऽड्डिते मार्गे~।\\
 वीराद्भुतरौद्राणां वितस्तमार्गेण वाद्यं तु~॥~६३~॥

 करुणरसेऽपि हि वाद्यं योज्यं ह्यालिप्तकरणमार्गे तु~।\\
 बीभत्सभयानकयोस्तथैव नित्यं हि गोमुख्या (म्)~॥~६४~॥

 रससत्त्वभाव (भावसत्त्व) योगान् (गं) दृष्ट्वाभिनयं\renewcommand{\thefootnote}{1}\footnote{N: दृष्ट्वाऽभिनये} गतिप्रचारांश्च~।\\
 वाद्यं नित्यं कार्यं यथाक्रमं (यथं) वाद्य (वृत्त) योगज्ञैः\renewcommand{\thefootnote}{2}\footnote{N. वाद्यसंयोगात्}~॥~६५~॥

 एवं प्रहतविधानं कार्यं मार्गाश्रितं बुधैः सम्यक्~।\\
 वक्ष्याम्यतश्च भूयो दर्दरपणवाश्रितं वाद्यम्~।६६~॥

 अतिवादितमनुवाद्यं समवादितमुच्यते पणववाद्यम्~।\\
 तत्रातिवादितं स्यान् मुरजानामग्रतो यत्तु~॥~६७~॥

 यत्त्वनुगतं मृदङ्गैरनुवादितमुच्यते तु तद्वाद्यम्~।\\
 समवादितं मृदङ्गैर्ज्ञेयं साम्येन यद्वाद्यम्~॥~६८~॥\renewcommand{\thefootnote}{3}\footnote{N (G.O.S V ६८ is not read fully, but Parts are read as:) यत्वनुगतं मृदङ्गैर्ज्ञेयं साम्येन यद्वाद्यम्~॥~ ( V. ६१ cd: N) V. ६७ cd: (G.O.S.) V. ६७ ab (N)} }
\end{quote}

\hrule

\vspace{2mm}
अथैषां विनियोगमाह शूृङ्गारेति~। शृङ्गारे हास्येऽप्यडितः~। {\qtt बीराद्धुतयो} रौद्रे वितस्तः~॥~६३~॥

\begin{quote}
{\qt  करुणे चालिप्तः~। बीभत्सभयानकयोर्गोमुखी~॥~६४~॥}
\end{quote}

व्यभिचारादिविषये तु स्वयमूहः कर्तव्य इति दर्शयति~। {\qtt रसभावसत्त्वयोगमिति}~। रसेषु भावा व्यभिचारिणः सत्त्वमित्युत्तमादिः प्रकृतिः~। अभिनयशब्दोऽत्र {\qtt शाखाव्यापारः}~। तत्र गतो.......... . वाद्ययोगः~। गतिप्रचारस्तु प्रवृत्ते नाट्ये परभावादित्यनाट्य एव~। {\qtt तदाह}~। {\qtt यथायथमिति}~। वृत्तशब्देनात्र नाट्यमिति~।~६५~॥\\

{\qtt प्रचारत्रयमुपसंहरति}~। {\qtt एवमिति}~। अथ मागविचित्र्योत्पादनाय करणषट्कं निरूपयिष्यन् तदुपयोगि दर्दरपणवं मन्यमानः पणवं तावदाद्यं लक्षयितुमुपक्रमते~। {\qtt वक्ष्यामीति}~। सामान्यो (न्येनो) क्तम्~। विशेषस्तूच्यत इति भूयः पातस्यः~। तच्च त्रिधा~॥~६६~॥\\

पुष्करतः पूर्वमपि वाद्यम्~। यद्यनुबाद्यं मृदङ्गैर्हेतुभिर्यदनुगतं व्यपदेशयतां नीतमित्यर्थः अन्ये तु विपर्ययेणैतदाहुः~। अग्रत इति पश्चादर्थे~। तदन्तः सामान्यसमवायं पूर्वोक्ताक्षरेभ्य एव~। केचिदनुबन्ध............... (साम्येन) तुल्यतेति दर्शयति~॥~६७ \textendash\ ६८~॥

\newpage
% चतुस्त्रिंशोऽध्यायः ४२९

\begin{quote}
{\na कखगा ठठणा देह्वा णरलाः क्रुलिलंध्रणेति किरिकिह्णा~।\\
\renewcommand{\thefootnote}{1}\footnote{N: एते वर्णाः सततं पणवे विधातव्याः (V ६२ cd: N)}एते वर्णाः पणवातोद्यविधाने विधातव्याः~॥~६९~॥

धोधोणाधोकिहुलं प्रहुलं ह्णह्णिति रिणिति रिह्णथह्णे~।\\
कंथित्वाढेढेंढणां पणववाद्यम्~।\\
ह्णोणिकिमिकिर्लैणोण इति पणववाद्यं तु~॥~७०~॥\renewcommand{\thefootnote}{2}\footnote{N:V. ७o (G.O.S.) V. E३ (N) read as: धोधोणाघो कुल्लटण्हा किकिकिरिकिणां धधः दण्णी~। कृमिणि कित्थि खो धो धांधे णां पणव वाद्यं~॥ Third line of V.so (G.O.S.) is not read in (N) 3 N: मध्याङ्गुल्या कार्यं कोणेन च संयुतं वाद्यं~॥~ ( V. ६४ cd.  \textendash\ N) : This is close to V. ७३ab (G.O.S.) 4. N. तुन्तं कन्तं योगे क्रोहलामिति वादनाग्रजा विद्या~। विहितकरणानुबद्धं तत्राग्रकराश्रयो चैकः~॥~ (V. ६५: N)}

तस्य प्रहतं कार्यं कनिष्ठिकानामिकाग्रकोणेन~।\\
नानाकरणविभागैः पणवे शिथिलाशञ्चिते तज्ज्ञैः~॥~७१~॥

वादककनिष्टिकाभ्यां शीघ्रकृताः करेऽथ नेणोहाः~।\\
3शेषास्तु वादनकृताः स्मृताः प्रहारा विविधाद्याः~॥~७२~॥

कोणानामिकवाद्यं मध्याङ्गुल्या हि वादनं कार्यम्~।\\
कोणानामिकवाद्यं शुद्धं प्रहतं भवेदेतत्~॥~७३~॥

4भ्रान्तकयोगात् क्रिहुलमित्येतत् भ्रान्तवादनाग्रेण~।\\
रिभितकरणानुविद्धं तत्रा (त्र) भस्तत्र (द्र) कारस्तु~॥~७४~॥

शीघ्रकरणानुविद्वैर्डकारस्त्वर्धहस्तसंयोगात्~।\\
स तु वादकेन वाद्ये धुर्येण विशेषतः कार्यः~॥~७५~॥}
\end{quote}

\hrule

\noindent
कखना टढणा इत्योर्ययाहरा

आर्यान्तरेण प्रयोगोपयागि तेषा....... माह~। थोथोणा इत्यादिना~॥~७०~॥

अस्य पणवस्य {\qtt कनिष्ठाऽनामिकाऽग्रकोणश्च} कोणाग्रमिति सम्बन्धः~। {\qtt अन्ये} त्वग्रे मध्यम इत्याहुः~। {\qtt मध्यमाङ्गुल्येत्या} (भ.ना.३४.७३) दि वक्ष्यते~। पूर्वं तु मध्यमानामिकामाहुः~। शिथिलता मार्दवम्~। तच्च न गाढीकरणम्~। तच्च कक्ष्याश्च स्कन्दारोहिणो वध्राश्च बन्धनरज्जो \ldots काष्टिकावस्कन्धस्य वादकसम्बन्धिनः~। शिथिलीकरणपीडनाभ्यां विचित्रवर्णनिर्वर्तितां क्रियां सम्पादयति~। {\qtt तदाह}~। नानाकरणविभागैरिति~॥~७१~॥\\

{\qtt वादकः} कोणः~। कोणश्चानामिका~॥~७२ \textendash\ ७३~॥\\

अथ तत्कोणानामिक{\qtt भ्रान्तकयोगः}~। व्याचष्टे {\qtt भ्रान्तबादनाग्रेणेति}~॥~७४ \textendash\ ७५~॥

\newpage
%४३०  नाट्यशास्त्रे

\begin{quote}
{\na स्वञ्चितकच्छौ (क्षौ) पणवौ कृत्वा धुर्यपरिवादकस्यां (काभ्यां) हि~।\\
रणणकिकिहिकिणिङ्क इति प्रहारा विधातव्याः~॥~७६~॥

अञ्चितकक्ष्ये पणवे रेफः सोऽर्धे तु हस्योर्ध्वे ग्रहे~।\\
कार्या मयूरककरा (:) सूक्ष्मौघास्तथा प्रहारा विधातव्याः~॥~७७~॥

कृत्वा च शिथिलकक्ष्यां कार्यास्तु कनिष्टिकाग्रकोणेन~।\\
तेनैव चरे प्रेद्धो भागे इत्यञ्चितेनैव~॥~७८~॥

अञ्चितशिथिले पणवे कठणानिकिणिकिण्णिकृताः~।\\
प्रहारास्तु वैभ्रान्तककरणेन तथा स्वञ्चितकक्ष्ये च शिथिले च~॥~७९~॥

अञ्चितकक्ष्ये पणवे कखरटणकृता मताः प्रहारास्तु~।\\
धिन्ने इति प्रहाराः पणवे तु सदा शिथिलकक्ष्ये~॥~८०~॥

कखरटकृताः प्रहाराः सोच्छ्वासे हि पणवे विधातव्याः~।\\
शेषा भ्रान्तकयोगाः संयुक्ताः संविधातव्याः~॥~८१~॥

\renewcommand{\thefootnote}{1}\footnote{(N) टकारं तु स्वराद्ये कक्ष्यान्वितं तु संभवति~।.... ह् इति...................... }टङ्कारेण स्वनजं वाद्यं कक्ष्याञ्चिते तु सम्भवति~।\\
तेन नकारो युक्तो ह्ण इति च पणवे प्रहारः स्यात्~॥~८२~॥

तिर्यग्गृहीतवादन\renewcommand{\thefootnote}{2}\footnote{N वादने मुखं संभव इष्यते~। उकारस्तु क्तुघ्ने क्तुहली क्रमेणैवं ह्येते प्रहारास्तु~। (V. ६९ N)}मुखवर्तित इष्यते तकारस्तु~।\\
3तहुलं तहुलं क्रमशश्चैवं ह्येते प्रहारास्तु~॥~८३~॥}
\end{quote}

\hrule

\begin{quote}
{\qt स्वस्तिक (ञ्चित) त्वं यदा पणवस्य........................... \\
यस्माद् दक्षिणशब्दं स्वयं नृत्यति शङ्करः~।\\
तस्माद् यत्नादसौ योज्यो गीतकासारितादिषु~॥~७६ \textendash\ ८०~॥

सोच्छ्वास इति~। मुक्तप्रहारे~॥~८१~॥

स्कन्दकरकं (टङ्कारेण स्वनजं) पवनेन कम्पमान इत्यर्थः~॥~८२~॥

तिर्यक् कृत्वा गृहीतस्य वादनस्य मुखे अग्रे वर्तिताः (तः) परिवर्तनेनोत्पादिताः (तः)~॥~८३~॥}
\end{quote}

\newpage
% चतुस्त्रिंशोऽध्यायः ४३१

\begin{quote}
{\na \renewcommand{\thefootnote}{1}\footnote{N. एवं त्वेककरः कृत्वा प्रहारेण द्विराहतः~। दर्दर पणवमृदङ्गर्मिश्रितं वाद्यं प्रवक्ष्यामि~। (V.७० \textendash\ N)}एवं पणवे वाद्यं विधिवत् संक्षेपतो यथाऽभिहितम्~।\\
वक्ष्याम्यतः परमहं दर्दरवाद्याक्षराण्येव~॥~८४~॥

रेक्लृतिकुत्खनोत्वनोधन्मोगोणेहधिण्णसंयुक्ताः~।\\
इति दर्दरे प्रहाराः कार्या मुक्ता निषण्णाश्च~॥~८५~॥

कार्यास्तत्र निषण्णा रग्रध्रह्णि इति च दक्षिणकरेण~।\\
वामेन गोमदोत्था न नखस्पृष्टः प्रकारोऽग्रे~॥~८६~॥

मुक्तौ तीत्रित्रिण भवेन् निपीडने करद्वयेनापि~।\\
धीमुक्तः प्रहतं स्यादनुस्वने ताडिते चैव~॥~८७~॥

स्यात् पीडिते च धिह्णेत्येतन्मुक्तं तथा विमुक्तेऽपि~।\\
शेषा भवन्ति मुक्तास्तक्ना इति चैव निरगृहीते~॥~८८~॥

शीघ्रकरणानुबन्धस्थितिरेव हि निःस्वनः (नं) स्ख (ख) लितकं स्यात्~।\\
इति दर्दरे प्रहाराः समासतस्तत्र विज्ञेयाः~॥~८९~॥

एवं त्वसङ्करकृता प्रहारशुद्धिरिह कीर्तिता तज्ज्ञैः~।\\
दर्दरपणवमृदङ्गैर्मिश्रितवाद्यं प्रवक्ष्यामि~॥~९०~॥}
\end{quote}

\hrule

\vspace{2mm}
\noindent
दर्दरवाद्यं वक्तुमाह~। वक्ष्यामीति~। दृश् (दृ) विदारणे  (पाधातुपाठः १४९४) इत्यस्य कर्तरि अचि (क्यचि) यङ्लुकि दर्दर इति पदम्~। तथा च पुराणे~। 

\begin{quote}
{\qt तस्य ये मङ्गलान्याहुः श्रुत्वा दर्दरजं ध्वनिम्~।}
\end{quote}

\noindent
इति~। अन्ये तु (द) र्द इति शब्दं राति ददातीत्याहुः~। मुरजपणवदर्दरस्य तु मुनिः स्वयमेव नामनिर्वचनं करिष्यति (भ.ना.३४.२८६)~। ततस्तत्रैव व्याख्यास्यामः~॥~८४~॥\\

{\qtt एवमिति} संक्षेपात्~। निषण्णाः पीडिताः~॥~८५ \textendash\ ८८~॥\\

{\qt शीघ्रकरणानुबन्धस्थितिरेव हि निःस्वनं खलितकं स्यात्~।}

\noindent
इति~। शीघ्रक्रियाया अनुबन्धेन चिरकालानुवर्तनेन स्थितिर्यस्य शब्दस्य तत् खलितकं शब्दसंचयरूपत्वात्~। खल सञ्चये (पा.धा. ५४५)~। यतस्तत्खलितकमिति लोके~॥~८९~॥\\

{\qtt अष्टावसङ्कराः}~। दर्दरवाद्याक्षराणामेवं सङ्करः~। तेन कृता शुद्धिः~। {\qtt अनामिश्रवाद्यम्}~। इह तु निरूपयति दर्दरपणवमृदङ्गानि पुष्कराणि~। लक्षणग्रन्थे विच्छेदः स्फुटो न प्रतीयत इत्यभिप्रायेण कुर्या (त्) ......~॥~९०~॥

\newpage
% ४३२  नाट्यशास्त्रे

\begin{quote}
{\na तत्र व्यक्तीभावं वाद्ये गच्छन्ति मिश्रिताः केचित्~।\\
केचिद्युगपत्करणं केचित् पर्यायकरणं तु~॥~९१~॥

एकैकसम्प्रयुक्ता वर्णानुगत (ता) स्तथैव सम्पृक्ता\renewcommand{\thefootnote}{1}\footnote{N.संयुक्ता (:)} (:)~।\\
\renewcommand{\thefootnote}{2}\footnote{(N) णण खे खोः देदे दिवि मने धिं धीति मिश्रास्तु~। (V.७२ cd \textendash\ N)}घोंद्रेरेंदोखोकोत्रिहुलं तत्कृता मृदङ्गेषु~।~९२~॥

अथ दर्दरेऽपि रधिरिक्लेते पणवयोस्तु संयुक्ताः~।\\
डणके खोदेदोदे दिधिरे दिधिनि मिश्रास्तु~॥~९३~॥

\renewcommand{\thefootnote}{3}\footnote{N:एभ्यो;}आभ्यो येऽन्ये शेषास्ते मिश्रा एव नित्यशः\renewcommand{\thefootnote}{4}\footnote{(N) सर्वदा} कार्याः~।\\
पूर्वो (र्वे) ऽपि च मिश्रत्वं व्रजन्ति सर्वे यथायोगम्\renewcommand{\thefootnote}{5}\footnote{(N) :यथायोग्यं~।}~॥~९४~॥

\renewcommand{\thefootnote}{6}\footnote{N (V.N ७४) : अथ युगपत् करणाः स्युः क्रमतश्ताकं घुण घ्रं दे दे ह्णा~। दे दे खिदो हे दोहा एवं योज्या पणववाद्ये~।}अथ युगपत् करणानि तु हह्णह्णेकुकुनणंप्रलंदोह्णम्~।\\
ग्रहुलं नह्णादो एवं योज्यास्तु पणवे हि~॥~९५~॥

धुर्यः (र्य) कृतासु\renewcommand{\thefootnote}{7}\footnote{(N) कृताश्च} (स्तु) क्रमशः करणे परिवादनेन\renewcommand{\thefootnote}{8}\footnote{(N) परिवादकेन} कर्तव्या (:)~।\\
\renewcommand{\thefootnote}{9}\footnote{(N) खो खो तिं तु धो धो णो णो णेति किणिकिणिति कृताश्च~।}कोखोदेदोधोणह्णेति किणिकिणिति कृताश्च~॥~९६~॥}
\end{quote}

\hrule
 
\vspace{2mm}
तत्र मिश्रा एकरूपवर्णा गतयः प्रहाराः~। अन्ये तु विविधाः~। युगपत् क्रमेण वा~। तदाह~। {\qtt व्यक्तीभावमिति}~। तत्रै (चैव) तैकशः प्रयुक्ताः पर्यायकरणा वर्णानुगता यौगपद्येन संयुक्ता मिश्रितत्वेनैकैकं कृत्वा सम्यक्प्रयुक्ताः~। पर्यायरणना इत्यन्ये पठन्ति~॥~९१~॥\\

तत्र मिश्रान् निरूपयितुमेतावत् प्रतिनियता अमिश्रा इति सार्थयाऽऽर्यया दर्शयति~। {\qtt एकैकेत्यादि}~।......... आनद्धमृदङ्ग इत्येके~॥~९२~॥\\

अन्येऽन्तेन ददरपणवयोः खत्यादिसहितार्यार्धेन पणवेऽमिश्रा उक्ताः~॥~९३~॥\\

{\qtt मिश्रानाह}~। आभ्यो येऽन्ये इति~। षोडशाक्षरादिके मा शङ्का भूदिति निःशेषपदमन्यपदं च~। सुशिक्षितवादकवशात्त्वेतेऽपि भवन्ति मिश्रा इत्याह~। {\qtt पूर्वेऽपीति}~। यथायोगमिति~। अभ्यासबलात्~॥~९४~॥\\

अथ युगपत्करणानीति~। अक्रमा उक्ताः~॥~९५~॥\\

{\qtt पर्यायकरणानाह}~। धुर्यकृतास्तु क्रमश इति~। पूर्वधुर्यस्ततः परिवादकः करोतियः पणवयोरेव क्रमः~॥~९६~॥

\newpage
% चतुस्त्रिंशोऽध्यायः ४३३

\begin{quote}
{\na \renewcommand{\thefootnote}{1}\footnote{(N) पणवानामनुबन्धं वाद्यं यस्मिन् तच्चेन्मृदऽङ्गेन~।}पणवानामनुबन्धे कार्यं धुकदुहणकेति वाद्यं तु~।\\
भूयः प्रतिकृतिभेदो मार्दङ्गिकदर्दरिभ्यां च~॥~९७~॥

यद्यत् कुर्यान् मुरजे प्रहारजातं गतिप्रचारेषु~।\\
अनुगतमक्षरवृत्तं तदेव वाक्यं तु पणवेऽपि~॥~९८~॥

न हि चित्रं कर्तव्यं गतिप्रचारेषु वादनं तज्ज्ञैः~।\\
समविषमं तत्र हि यद् वृ (त् तल्ल) क्ष्यं पादसञ्चारे~।~९९~॥

उपरिकरणे यथेष्टं कर्तव्यस्तु पणवो मृदङ्गेषु~।\\
\renewcommand{\thefootnote}{2}\footnote{(N) तत्र प्रकारकरणं मृदङ्गवाद्ये विधातव्यम्~॥ (V. ७६. cd. N)}तत्र प्रहारकरणैर्मृदङ्गवाद्यं विधातव्यम्~॥~१००~॥}
\end{quote}

\hrule

\vspace{2mm}
{\qtt अनुबन्ध} इति~। परिवादकेन छिद्रेषु प्रयोज्यमित्यर्थः~। अत परस्परं क्रममाह~। 

\begin{quote}
{\qt भूयः प्र (इति) कृतिभेदे (दो) मार्दङ्गिकर्ददरिभ्यां वा (च)~।}
\end{quote}

\noindent
इति~। प्रतिकृतिरनुकारः पश्चादनुकरणम्~। स षोढा भिद्यते~। यदा मुख्यमृदङ्गं (ङ्गः) ततः पणवस्ततो दर्दर इत्येको भेदः~। दर्दरः पणव इति द्वितीयः~॥~९७~॥\\

एवं पणवे दर्दरे च मुख्ये प्रत्येकं द्वाविंशतिः षट्~। तत्र गतिप्रचारे परिक्रमणादौ मुरजप्राधान्यं क्रमकरणयोग इति दर्शयति~। 

\begin{quote}
{\qt यद्यत् कुर्यान् मुरजे प्रहारजातं गतिप्रचारेषु~।\\
 अनुगतमक्षरवृत्तं तदेव वाद्यं तु पणवेऽपि~॥}
\end{quote}
 
\noindent
इति~। {\qtt दर्दरेऽपि}~। अनुगतं रसादिवशायातम्~। वृत्तं प्रकारः~॥~९८~॥\\

अत्र (गतौ) द्वित्रियुगपत्करणं निषेधति~। न हि चित्रं कर्तव्यं गतिप्रचारेषु वादनमिति~। अत्र हेतुः~।

\begin{quote}
{\qt समविषमं तत्र हि यत् तल्लक्ष्यं पादसश्चारे~।}
\end{quote}

\noindent
इति~। तत्र विचित्रे यत् समं विषमं च प्रकारजातं तत् पादसश्चारे लक्षणीयं योज्यम्~॥~९९~॥\\

न च ताण्डवादि मुक्त्वा गतौ तद्योजनाङ्गमित्या (त्य) कृतशेषः~। गतिप्रचारादन्यत्वे तु न शेष इत्याहा उपरिकरण इति~। वाद्यप्राधान्य इत्यर्थ~॥~१००~॥

\newpage
%४३४ नाटयशास्त्रे

\begin{quote}
{\na प्रायेण सर्ववादेष्वादौ पणवग्रहः प्रयोक्तव्यः~॥\\
\renewcommand{\thefootnote}{1}\footnote{N: वक्ष्याम्यतः परमहं दर्दुरवाद्याक्षराण्येव (७७ cd.N)}वक्ष्याम्यहमतः परं तु लक्षणं करणजातस्य~॥~१०१~॥

रूपं कृतप्रतिकृतं प्रतिभेदो रूपशेषमोघस्य (श्व)~।\\
षष्ठी च प्रतिशुक्लेत्येवं ज्ञेयं करणजातम्~॥~१०२~॥\renewcommand{\thefootnote}{2}\footnote{Verse १०२ (G.O.S.) is not read in (N) but  (N) reads three Verses (Vs. ७९ \textendash\ ८१) as:}

तत्र रूपं\renewcommand{\thefootnote}{3}\footnote{(N) रूटं (पं ?);} नाम विभक्तकरणम्~। यथा~।\\
घं (दें) घंगेघं किटिमा किटिमा घटत्थि घटपत्थि~।\\
\renewcommand{\thefootnote}{4}\footnote{(N) थं.धं दुदु किटि किटि किण्णां दो धो दो धो श्लिमिति~॥~ रूट (प) म्~॥~ cd. ८२ (N)}घदुगुरुकिटिकट कृह्णं दोघे दोघे क्ले~॥~१०३~॥}

इति रूपम्~। 

{\na कृतप्रतिकृतं नाम यत्रैकं करणं त्रिपुष्कर इत्यु (मप्यु) द्भा (चैव) यति~।}

यथा~। 

{\na दं (धुं \textendash\ लं) खु (खुं) खुण (मणे) \renewcommand{\thefootnote}{5}\footnote{(N) क्रम धिमदरदर धिणक्रु.~॥~ गुटु हंवं कुहुले ह्णा धो ध्रो धोण णो वाच्यं~॥~८३ (N)}क्रमधिमदांणेटोटतितोटमत्थिमानक्रम्~।\\
गुरुखे किहुले दोह्णं दोस्रो दोधोण खे वाद्यम्~॥~१०४~॥}
\end{quote}

\hrule

\vspace{2mm}
अथ यदर्थं पणवदर्दररूपमुक्तं तत् करणषट्कं च वक्तुमुपक्रमते~। {\qtt वक्ष्यामीति}~॥~१०१ \textendash\ १०२~॥\\

रूपं नाम विभक्तकरणमिति~। मुरजे गुरुद्वयम्~। अन्यत्र तत्समकालं लघुद्वयम्~। गुरु चेत्यादिना विभक्ता यत्र क्रिया तद्रूपम्~। देघं इति मुरजे~। किटिमा इत्यन्यत्र~॥~१०३~॥\\

{\qtt कृतप्रतिकृतम्}~। यत्रैकं करणं त्रिपुष्करमप्युद्भावयति~। सर्ववाद्याक्षरयुक्तमिति यावत्~। एकस्मिन् कृतेऽन्यदपि प्रतिकृतं संविभक्तं पुष्करं यत्रेति~। धुंखुंखुमणे इति~। {\qtt उपाध्यायास्त्वाहुः}~। नैतावता पणवादिभिः सह किञ्चिदुक्तं स्यादिति~। तदयमर्थः~। एकमिति~। एकं पणवादिकरणमित्यर्थः~। पश्चाद्भावि सपुष्करवाद्यमुद्भावयति अनुकरोति सति प्रतिकृतमिति~। वीणावाद्यमित्यनुकृत भवति~। न हि सादृश्यमात्रमेतत्~। ततोऽर्थत्वादनुबद्धस्येति~। उदाहरणमखण्डधा विभजनीयम्~। लंखुंखुणै इति मुरजे पञ्चधा~। क्रमिमोटं इति पणवे इत्यादि~॥~१०४~॥

\newpage
% चतुस्त्रिंशोऽध्यायः ४३५ 

{\qt प्रतिभेदो नाम युगपत्कृते करणे मृदङ्गानां यदुपरिकरणेन गच्छन्ति य (त) त्~।\renewcommand{\thefootnote}{1}\footnote{(N) यथा}}

\begin{quote}
{\na \renewcommand{\thefootnote}{2}\footnote{(N) धो धो णा धो दुहिणं ण ण ण कुण्णं कथोथियं भेदं~।}दाधा (धोधो) धाणिधा मटगतमधिं घटे घटे~।\\
दोघे घट मत्थिणह्नको खो~।\\
(एभिः करणविशेषैः प्रतिभेदो नाम विज्ञेयः)~॥~१०५~॥}
\end{quote}

\begin{center}
\textbf{इति वाद्यम्~।}
\end{center}

\begin{quote}
{\na रूपशेषं तु\renewcommand{\thefootnote}{3}\footnote{(N) तु not read in ()} [रूपशेषं (षो) नाम करणानामविशेषो\renewcommand{\thefootnote}{4}\footnote{करणानामविरामे यथा~।} यथा~।\\
खुखुणं णणणणणां मढाघेव मथिटां घेटां घेदो~।\\
ये घेटमढिणष्णरवो इति वाद्यं रूपशेषं तु~॥~१०६~॥\renewcommand{\thefootnote}{5}\footnote{(N) reads two Verses here: खु खु णा णणूणणं माथे घण्णाट मथि धेटं~। 
\begin{quote}
{\qt घेटं मथि णे दा दे णा खु खु खु खुणां~। \\
 ध्रुण श्लाण स्णा खु खु खि चेण्णां~।\\
 खु द्रुम थेण्णां वाद्यमिदं रूपशेषं तु~॥~८५~॥}
\end{quote}}

\renewcommand{\thefootnote}{6}\footnote{6. (N) ............. मार्दङ्गिक पाणविक दर्दरवादकानां यथा~।}प्रतिशुल्कं (ल्को) नामानुस्वारो मार्दङ्गिकरर्दरवादकानाम्~। यथा~।\\
\renewcommand{\thefootnote}{7}\footnote{(N) : (V.86.N.) : घटमत्थि कुणकिटि गखाखा ण खो खा खा णा ख कुटकुटा कूणकिटि किटिधथि थिणण णात्थि शुक्लाख्ये सर्वभाण्डविरामो~॥ च~॥~ १०८~॥~ not read in N.}घटमथि कुणकिटि गखाखा कथिधिजाणणकोखोणाखा~।\\
कुटकटि कुणकिटि कटखे प्रतिशुल्काख्यं सदा करणम्~॥~१०७~॥

ओ नाम सर्वभाण्डविधेयो द्रुतपाणिलयो नद्योघवत्~। ओघो यथा~।\\
घंकिटिमथकिटिकिटिधे घदधुदेदेण घाण दाधानणोखो~।\\
घदुगुदुघुदुगुदुपदधेरेणिण एते तथौघे च~॥~१०८~॥}
\end{quote}

\hrule

\vspace{2mm}
मृदङ्गानां युगपदेव करणे क्रियमाणे यदुपरिकरणेन~। {\qtt तद्यथा}~। लघुद्वये मुरजेन प्रयुयुक्षिते पादस्थे गुरुनियते सदा प्रतिभेदः~। धोधो इति प्रयोज्यमानः~। {\qtt कुहलमिति}~॥~१०५~॥\\

रूपशेषः करणानामिति~। यथा (दा) मौरजिको विरामं करोति तदा तच्छिद्रे पाणविको वा वादयति~। एतदन्यदपि क्रियमाण (णं) विच्छेदः~। खुखुणं इति मुरजविरामे णंणंणं इति पणवे इत्यादि~॥~१०६~॥\\

प्रतिशुष्को (ल्को) ऽनुस्वार इति~। बहुकं कालं मार्दङ्गिकेन प्रयोगे कृते ध्रुवावर्तितत्कालं पणवस्य दर्दरस्य वा प्रयोगः~। घटमधि इत्यादि~॥~१०७~॥\\

ओधे प्रयुक्ते पणववाद्यं कुटकुटण केटखटखिट~। ओघे विलम्बितलये मुख्ये एको द्रुतलयः~। एतच्च करणषट्कं (पिण्डी) प्रतिसरादिनृत्तेषु दृश्यते~। हुलडुक्वा पणवभेद एव~॥~१०८~॥

\newpage
%४३६ नाट्यशास्त्रे

\begin{quote}
{\na करणानां समायोगः षड्विधः परिकीर्तितः~।\\
\renewcommand{\thefootnote}{1}\footnote{प. (N) अनेन तु~।}अनेनैव विधानेन योज्यं वाक्करणं बुधैः~।~१०९~॥}
\end{quote}

{\qt त्रियति नाम~। समास्रोतोगता गोपुच्छा चेति~। लययति\renewcommand{\thefootnote}{2}\footnote{ज. र्नाम~।} पाणीनां त्रिविधः संयोगः~। स च त्रिप्रकारो भवति~। तद्यथा~। राद्धं विद्धं शय्यागतं चेति~।~११०~॥}

\begin{quote}
{\na त्रिलयं नाम~। द्रुतो मध्यो विलम्बितश्चेति~।~१११~॥

त्रिपणिकं नाम समपाणिरर्धपाणिरुपरिपाणिश्चेति~।~११२~॥ 

समा यतिर्द्रुतश्चैव\renewcommand{\thefootnote}{2a}\footnote{(N) : . यतिर्ध्रुवश्चापि} लयो यत्र भवेदथ~।\\
तथैवोपरिपाणिश्च \renewcommand{\thefootnote}{2b}\footnote{(N) : राद्धश्चाद्य ततो भवेत् (V. ८९ N)}राद्धस्त्वेष विधिर्भवेत्~।~११३~॥

स्रोतोगता यतिर्यत्र लयो मध्यस्तथैव च~।\\
समपाणिस्तथा चैव विद्धं वाद्यं तु तद्भवेत्~।~११४~॥

अर्धपाणिस्तु यत्र स्यात् तथा चैव स्थितो लयः~।\\
\renewcommand{\thefootnote}{3}\footnote{भ. (N) गोपुच्छा चैव हि यतिर्वाद्यं~।}यतिश्चैव तु गोपुच्छा वाद्यं शय्यागतं तु\renewcommand{\thefootnote}{4}\footnote{भ. हि~।} तत्~॥~११५~॥}
\end{quote}

\hrule

\vspace{2mm}
उपसंहारं कुर्वन् करणस्य सामान्यलक्षणमप्याह~। करणानां समायोगः षड्विध इति~। भिन्नानां वाद्यक्रियाणां मिश्रि (श्र) ता येन क्रियते तत् करणमिइति यावत्~। रूपादीनां चान्वयस्तताध्याये  (भ.ना.२९) व्याख्यातः~।एतत्करणानुसरेणैवोध्दतिका कार्या इत्याह~। {\qtt अनेनैवेति}~॥~१०९~॥\\

एवं वर्णानुहावैचित्र्यमुक्त्वा तालप्राधान्येनावनध्दप्रवृत्तं तदुपयोगि यत्यादिकमाह~। {\qtt त्रियतीति}~। यित्रयाणी तलाध्याये (भ. ना. ३१) उक्तं रूपमितीह परमां मेलनामाह~। त्रिविधः संयोगः स च त्रिप्रकार इति~। न च तावदत्र सम्भवन्ति भेदाः~। इह तत् त्रय उपादेया इति भावः~॥~११० \textendash\ ११२~॥\\

समा यर्तिद्रुतो लय उपरिपाणिरिति राद्धः सिद्धोऽनन्यापेक्षो द्रुतादेव समाप्तेः~॥~११३~॥\\

स्रोतोगता मध्यः समपाणिरिति विद्धः~॥~११४~॥\\

उभयरूपानुवेधाच्छेषवच्छय्यागतः शय्यां विश्रान्तिं प्राप्तो यतः~॥~११५~॥

\newpage
% चतुस्त्रिंशोऽध्यायः ४३७ 

\begin{quote}
{\na स्थिताल्लयात् प्रभृत्येषां प्रमाणं सम्प्रवर्तते~।\\
\renewcommand{\thefootnote}{1}\footnote{भ.कार्या हानिः~।}कार्यहानिं कलानां च शेषेष्वन्येषु पाणिषु~॥~११६~॥

यतयः पाणयश्चैव लया वै वाद्यसंश्रयाः~।\\
यथाकामं\renewcommand{\thefootnote}{2}\footnote{च. क्रमम्~।} हि कर्तव्या \renewcommand{\thefootnote}{3}\footnote{भ. नाट्ययोग~।}नाट्यशक्तिमवेक्ष्य तु~॥~११७~॥}

त्रिमार्जनं नाम~। 

{\na मायूरी ह्यर्धमायूरी तथा कार्मारवीति च~।\\
तिस्रस्तु मार्जना ज्ञेयाः पुष्करेषु स्वराश्रयाः~॥~११८~॥

गान्धारो वामके कार्यः षड्जो दक्षिणपुष्करे~।\\
\renewcommand{\thefootnote}{4}\footnote{(N) पश्चमस्त्वोर्ध्वके कार्यः कर्मारम्याः स्वरास्त्वमी~॥~९५ (N)}ऊर्ध्वके पञ्चमश्चैव मायूर्यां तु स्वरा मताः~॥~११९~॥

वामके पुष्करे षड्ज ऋषभो दक्षिणे तथा~।\\
ऊर्ध्वके धैवतश्चैवमर्धमायूर्युदाहृताः~॥~१२०~॥

ऋषभः पुष्करे वामे षड्जो दक्षिणपुष्करे~।\\
पञ्चमश्चोर्ध्वके कार्यः कार्मारव्याः स्वरास्त्वमी~॥~१२१~॥}
\end{quote}

\hrule

\vspace{2mm}
\noindent
{\qtt प्रयोगविधिमाह}~। स्थिताल्लयात् प्रभृतीति~। एषामिति राद्धादीनाम्~। हानिरिति न्यूनप्रमाणात्~॥~११६~॥\\

{\qtt उपसंहरति}~। यतय इत्यादि~। संश्रया नियामकाः~॥~११७~॥\\

अथ स्वरानुहारभागं निर्णेतुमुपक्रमते~। {\qtt त्रिमार्जनमिति}~। स्वराश्रया इति~। स्वरानुहारभेदा इत्यर्थ~॥~११८~॥\\

गान्धारो वामे आलिङ्गके~। दक्षिण आङ्निके षड्जः~। ऊर्ध्वके पञ्चमः~। {\qtt त्रिश्रुतिर्मायूरीयम्}~। मध्यमग्रामे निषादधैवतर्षभानामेतत्पीडनादेव सम्पादितत्वात् स्थितिः~। चतुर्थे त्ववशिष्टे मुखे मध्यम इति पूर्वस्वरत्वम्~॥~११९~॥\\

वामे षड्जः~। दक्षिणे ऋषभः~। ऊर्ध्वके धैवतश्चतुःश्रुतिः~। अर्धमायूरी षड्जग्रामे~॥~१२०~॥\\

ऋषभो वामे षड्जो दक्षिणे ऊर्ध्वके तु पञ्चमः स त्रिश्रुतिश्चतुःश्रुतिर्वा~॥~१२१~॥

\newpage
% ४३८ नाट्यशास्त्रे

\begin{quote}
{\na \renewcommand{\thefootnote}{1a}\footnote{N: एतेषामपिवादी तु}एतेवामनुवादी तु जातीनां यः स्वरो मतः~।\\
आलिङ्गमार्जनां \renewcommand{\thefootnote}{1}\footnote{भ.प्राप्य~।}प्राप्तो निषादः स विधीयते~॥~१२२~॥

मायूरी मध्यमग्रामे षड्जे त्वर्धा तथैव च~।\\
कार्मारवी तु कर्तव्या \renewcommand{\thefootnote}{2}\footnote{(N) साधारित विमाश्रया (९७N)}साधारणसमाश्रया~॥~१२३~॥

[स्वराः स्थानस्थिता ये तु श्रुतिसाधारणाश्रयाः~।]\\
त एव मार्जनकृताः शेषाः सञ्चारिणो मताः~॥~१२४~॥}
\end{quote}

\hrule

\vspace{2mm}
अत इयं साधारणाश्रया लक्ष्यते~। साधारणं तुल्यग्रामद्वयाश्रयणं यस्या इति~। स्वत्रयन्यासेऽपि प्राधान्यस्वरान्तरं भवेदित्याशङ्कयाह~।

\begin{quote}
{\qt एतेषामनुवादी तु जातीनां यः स्वरो मतः~।\\
 आलिङ्गमार्ग (र्ज) नां प्राप्तो निषादः स विधीयते~॥}
\end{quote}

\noindent
इति~। मृत्तिकायाः समं मार्जनाम्~। तत्कृतः स्वरयोगोऽपि तथा~। तन्त्रीशिथिलीकरणायामनात्मकसाधारणाप्रभवो हि स्वरयोगः~। साधारणायोगेत (न) एतेषां स्वराणां मध्ये य: स्वरो जातिध्रुवां गानजात्यङ्गरागादीनां यो वादी अंशस्वरः स (अनुवादि) नेति शेषः~। आलिङ्गसम्मार्जनादुत्थितमार्जनं (न) शब्दवाच्यं गान्धारम्~। अन्यत्सा दृश्यलक्षणेन यं प्राप्नोति निषादस्वरोऽत्र स विधीयते निषादांशत्वे प्राधान्येनाश्रीयते~। एष षड्जस्य तु.....मन्तव्यः~। अनुवादीह संवादीति~। अन्ये तु भिन्नं व्याचक्षिरे~॥~१२२~॥\\

{\qtt श्रुतिनियमाद्विभागमाह}~। मायूरीत्यादि~॥~१२३~।\\

जात्यंशकगतस्थायिस्वरप्राधान्यकृतं प्रधानमभिधाय स्थानस्वरकृतमणि भवति
प्राधान्यमपि (मिति) दर्शयति~।

\begin{quote}
{\qt स्वराः स्थानस्थिता ये तु श्रुतिसाधारणाश्रयाः~।\\
 त एव मार्जनकृताः शेषाः सञ्चारिणो मताः~॥}
\end{quote}

\noindent
इति~। इह स्थितः लयः~। स्वर इत्यन्ये~। स्थानस्वरोऽन्यश्चांशस्वरस्तथा च लब्धोव्यवहारो देवताप्राप्तिककुभषड्जो गीयते इत्यादि~। तेन यत्स्थानत्वेन त्वाश्रयणं ग्रहणं येषामित्यर्थः~॥~१२४~॥

\newpage
% चतुस्त्रिंशोऽध्यायः ४३९

\begin{quote}
{\na वामके चोर्ध्वके \renewcommand{\thefootnote}{1}\footnote{च. चैव~।}कार्या आहार्या लेपतः स्वराः~।\renewcommand{\thefootnote}{1a}\footnote{N: वामोर्ध्वकाद्यामाहार्याः कार्याः लेपे नवे स्वराः~।  (V.९९ ab. \textendash\ N)}\\
शैथिल्यादायतत्त्वाच्चावध्राकोटनयाऽपि च\renewcommand{\thefootnote}{2}\footnote{च. वा~।}~॥~१२५~॥

स्वराणां सम्भवः कार्यो मार्जनासु प्रयोक्तभिः~।\\
मार्जना तु कृता\renewcommand{\thefootnote}{3}\footnote{भ. मृदा~।} कार्या वामकोर्ध्वकयोः सदा~॥~१२६~॥

लक्षणं मृत्तिकायास्तु गदतो मे निबोधत~।\\
निश्शर्करा निस्सिकता निस्तृणा निस्तुषा तथा~॥~१२७~॥

न पिच्छिला न विशदा\renewcommand{\thefootnote}{4}\footnote{म. विषदा~।} न क्षारा न कटुस्तथा~।\\
नावदाता न कृष्टा च नाम्ला नैव च तिक्तका~॥~१२८~॥

मृत्तिका लेपने शस्ता तया कार्या तु मार्जना~।\\
नदीकूलप्रदेशस्था \renewcommand{\thefootnote}{5}\footnote{(N) श्यामा या मृत्तिका भवेत् ( १०३ b \textendash\  (N) );}श्यामा च मधुरा च या~॥~१२९~॥

तोयापसरणकश्लक्ष्णा तया कार्या तु मार्जना~।\\
बधिरा ह्यवदात्ता तु कृष्णा कुर्वीत न स्थिरा~॥~१३०~॥

न तुषा न स्वरकरी श्यामा स्वरकरी भवेत्~।\\
यवगोधूमचूर्णां (र्ण) वा तत्र दद्यात् प्रलेपने~।\\
एकस्तस्य तु दोषः स्यादे\renewcommand{\thefootnote}{6}\footnote{(N) एकः स्वरकरं भवेत् ( V. १०५ ७ \textendash\  (N))} कस्वरकृतं भवेत्~॥~१३१~॥}
\end{quote}

\hrule

\vspace{2mm}
अथ मार्जनासु स्वरत्वमाह~। वामके चोर्ध्वके चेति~। लेपत आहरणीयाः स्वराः कर्तव्याः~। अन्यत्तु चर्मण एवायतत्वशैथिल्यकृताभ्यां चाह~। हननं पातनं वध्राणां वा वध्रकानाशैथिल्यगाढत्वाभ्यामपि द्वित्रि (शस्त्रि) शश्चेति~। समुच्चयमाह चग्रहणेन~॥~१२५ \textendash\ १२६~॥\\

पि (पै) च्छिल्यम्~। लेपार्थं मृत्तिकां लक्षयति~। {\qtt निश्शर्करेत्यादिना}~॥~१२७ \textendash\ १३०~॥\\

कम्रभावे विधिमाह~। {\qtt यवगोधूमचूर्णेति}~। मिश्रीकृतमित्याहुः~॥~१३१~॥

\newpage
% ४४० नाट्यशास्त्रे 

\begin{quote}
{\na त्रिसंयोगं नाम~। गुरुसञ्चयो लघुसञ्चयो गुरुलघुसञ्चयश्चेति~॥~१३२~॥

तत्र गुरुसञ्चयो नाम~। गुरुस्थितलयौघवृत्तो यथा~।\\
\renewcommand{\thefootnote}{1}\footnote{(N:) धं तां केतां त्रं त्रां धं ध्रां ध्रां ध्रां कां तां कां ते~। तत धां तां कां तां वें तां इते गुरूणि स्युः~।. (V. १०६ (N) ef, gh)}धंतां केतां धंद्रां धेतं धंधं तथैव केंतांधम्~।\\
दंदंदेंदेंद्रार्घेंतांखेंताखेंतामिति च गुरु स्यात्~॥~१३३~॥

\renewcommand{\thefootnote}{2}\footnote{V. १३४,१३५ (G.O.S.) V. १०८, १०७ (N)  \textendash\  (N) reads  \textendash\  लघुसञ्चयो नाम द्रुतलयवृत्तः यथा \textendash\ घट घट मथि घट मथि गुटु गुटि न किटि मट गुट घे इति संयोगा ज्ञेया प्रयोगवाद्यैर्मृदङ्गेषु (V.१०८ N:) V.१०७ (N)  \textendash\  गुरुलघुसञ्चयो नाम~। मध्यलयप्रवृत्तः~। यथा \textendash\ धिमिधिमिथि धिक्तान् तकटा घट नघि घटयो~। घय मथि घे घं घे यो टु णु ढेकिंट गुटु घट घेमति~।}लघुसञ्चयो नाम~। गुरुलघुमध्यप्रवृत्तो यथा~।\\

घटमटघटमथटमथटमधिद्धडगुडघडधत्कृलघु~।\\
घटमथिघदुगुदुखघदुघंकृतो लघुलयश्च~।\\
इति संयोगौ ज्ञेयौ मृदङ्गवाद्ये प्रयोगज्ञैः~॥~१३४~॥

गुरुलघुसञ्चयो नाम~। गुरुलघुमध्यप्रवृत्तो यथा~।\\
घटथिम्मथिथिं मथितं किटा~।\\
घटधिं (मथि) टत्थिमटके~।\\
इति संयोगो गुरुलघुप्लुतः स्यात्~॥~१३५~॥

\renewcommand{\thefootnote}{3}\footnote{(N) has प्रकृतिर्नाम}त्रिगतं नाम~। तत्त्वमनुगतमोघं चेति~।\\
अक्षरसदृशं वाद्यं स्फुटपदवर्णं तथैव वृत्तसमम्~।\\
\renewcommand{\thefootnote}{4}\footnote{भ. अ.~।}सुविभक्तकरणयुक्तं तत्त्वे वाद्यं विधातव्यम्~॥~१३६~॥}
\end{quote}

\hrule

\vspace{2mm}
अथ त्रिसंयोगं व्याचष्टे~। गुरुसञ्चय इत्यादि स्पष्टम्~।~१३२ \textendash\ १३५~॥\\

{\qtt त्रिगतमाह}~। तत्त्वमित्यादिना~। अनेन सह यद्गमनं मिश्रता तद्गतत्वं त्रिविधम्~। अत एव स्वातन्त्र्येण ताण्डवादौ वाद्यप्रयोगे गानाभावे नाट्येऽपि छिद्रछादनाद्योधम्~। अनुगतं गाने शय्यागतसिद्धराद्धभेदाद्गाने सह मेलनेन कृतं तत्त्वादन्यमि (दि) त्यपौनरुक्त्यम्~। तत्र तत्त्वमाह~।

\begin{quote}
{\qt अक्षरसदृशं वाद्यं स्फुटपदवर्णं तथैव वृत्तसमम्~।\\
 सुविभक्तकरणयुक्तं तत्त्वे वाद्यं विधातव्यम्~॥}
\end{quote}

\noindent
इति~। इह विचित्रे यथाक्षरसादृश्यमपि नास्ति~। यदा दुकटरधादिभिरक्षरैः सादृश्यं तदा तत्त्वम्~। यथावाद्यमताद्रूप्यस्य सम्भवात् सादृश्यमुक्तं न ताद्रूप्येऽनुगतम्~। यथा~।

\newpage
% चतुस्त्रिंशोऽध्यायः ४४१

\begin{quote}
{\na समपाण्यवपाणियुतं स्फुटप्रहारकरणानुगं चैव~।\\
गेयस्य च वाद्यस्य च भवेदवघाताय \renewcommand{\thefootnote}{1}\footnote{भ. (N) चानुगतम्~।}तदनुगतम्~॥~१३७~॥

नैककरणाश्रयगतं\renewcommand{\thefootnote}{2}\footnote{(N) ० युतं} ह्युपर्युपरिपाणिकं द्रुतलयं च~।\\
आविद्धकरणबहुलं योज्यं वाद्यं बुधैरे (रो) घैः (घे)~॥~१३८~॥}
\end{quote}

\hrule

\begin{quote}
{\qt  अभिहतदर गरहरणकरण\\
 हरललाटोत्थिततरणिभारणि (मारण) काम\\
 क्रमगीतिशि (तीश) शीर्षरणिततम (मो) हते\\
 कुरु त (तं) कुरु त (तं) मम करणगणा (णम्)~।\\
 गिरिगुरुतन्द्रहा द्ध (ध) रोद्धरण\textendash \\
 मि (म) हाति (वि) मलकि (शी) लन्ततु्ना (तन्तुधा) रधा\textendash \\
 धमधामहा हाघटितोदित\textendash \\
 कातरकीटककोटिराकर्षक\\
 कदु (कुरु) तडो (ण्डो) रतमकाण्डे~॥}
\end{quote}

\noindent
इत्युच्चार्यमाणा (णे) याऽर्थप्रतीतिः सा च तत्त्वादिः स्यात्~॥~१३६~॥\\

पदे गीयमाने यो वर्णः स्थाय्यादि (:) स्वरानुहारेण यत् तत् स्फुटं वृत्तसमयमित्यष्टौ साम्यानि लक्ष्यन्ते~। एतच्च सुविभक्तं कृत्वा प्रहारक्रिया भवतीति गानेन सह ताद्रूप्यम्~। तत्त्वसमपाण्यवपाणियुतस्फुटप्रहारस्तु करु (र) णानुगं चैव गेयस्य वाद्यस्य वा उभयभेद्वाताय त....... तत्तव गीतं सर्वथा प्रधानम्~। इह तु वाद्यमपि तेन समैः पाणिप्रहारैरवपाणिभिः स्ववर्तनादिवैचित्र्यप्रवृत्तैर्यत्तदनुगतमेवाह~। त्रिप्रहारक्रियास्फुटनकृत  ..... पाटनमनुगच्छति~॥~१३७~॥\\

ननु तत्सर्वात्मनाऽवस्कन्दतीत्युभयं सा (स्या) दनुगतत्वं यस्याप्यघातकं यतः~। 

\begin{quote}
{\qt नैककरणाश्रयगतं ह्युपर्युपरिपाणिकं द्रुतलयं च~।\\
 आविद्धकरणबहुलं योज्यं वाद्यं बुधैरोघे~॥}
\end{quote}
 
\noindent
इति~। नैकशब्दः प्रातिपदिकान्तरं विविधार्थे~। करणं क्रियाघातः~।  (उ) पर्युपरिपाणेरिति प्रहाराणां न विभक्तत्वम्~। अत एवाक्षरसादृश्यमप्यर्थं (र्थ) दुर्लक्ष्यम्~। आविद्धा निरुद्धा हि पणवदर्दरकरटादिक्रिया हि~। बहुलमिति द्रुतलयमित्येतौ पौनरुक्त्यम्~॥~१३८~॥

\newpage
% ४४२ नाटयशास्त्रे

\begin{quote}
{\na सर्वस्यापि\renewcommand{\thefootnote}{1}\footnote{भ. (N) सर्वस्यैव} हि वाद्यस्य अष्टौ साम्यानि भवन्ति~। तद्यथा~।\\
अक्षरसममङ्गसमं ताललयत्वं यतिसमकं (लययतिसमं) ग्रहसमकम् (मं च)~।\\
न्यासापन्याससमं पाणिसमं चेति विज्ञेयम्~॥~१३९~॥

\renewcommand{\thefootnote}{2}\footnote{भ. (N) तत्र वृत्तं~।}यद्वृत्तं तु भवेद् गानं\renewcommand{\thefootnote}{3}\footnote{भ. (N) वाक्यं} गुरुलध्वक्षरान्वितम्~।\\
तद्वृत्तं तु\renewcommand{\thefootnote}{4}\footnote{भ. यद्} भवेद् वाद्यं\renewcommand{\thefootnote}{5}\footnote{य. वाक्यं} तदक्षरसमं भवेत्~॥~१४०~॥

ध्रुवाणां ग्रहमोक्षेषु कलान्तरकलासु च~।\\
यदङ्गं क्रियते \renewcommand{\thefootnote}{6}\footnote{भ.वाद्ये}वाद्यं तदङ्गसममुच्यते~॥~१४१~॥

यच्छरीरं भवेद्गानं कलातालप्रमाणजम्~।\\
तत्प्रमाणं तु \renewcommand{\thefootnote}{7}\footnote{च. यद्}तद्वाद्यं तद्वै तालसमं भवेत्~॥~१४२~॥

स्थिते मध्ये द्रुते वापि लये गानं \renewcommand{\thefootnote}{8}\footnote{च. (N) यथा}तु यद्भवेत्~।\\
तथा भवेत्तु तद्वाद्यं तद्वै लयसमं भवेत्\renewcommand{\thefootnote}{9}\footnote{भ. ततः}~॥~१४३~॥

समा स्रोतोगता \renewcommand{\thefootnote}{10}\footnote{भ.वापि}चापि गोपुच्छा \renewcommand{\thefootnote}{11}\footnote{भ.वा}च यतिर्यथा~।\\
\renewcommand{\thefootnote}{12}\footnote{य. (N) तत्तथा हि भवेद्वाद्यं}तथा भवेत्तु यद्वाद्यं तद्वै यतिसमं भवेत्~॥~१४४~॥}
\end{quote}

\hrule

\vspace{2mm}
तानि साम्यानि तत्त्वे सूचितान्याह~। सर्वस्यापीति~। {\qtt पणवादेरपि}~।

\begin{quote}
{\qt अक्षरसममङ्गसमं ताललययतिसमं ग्रहसमं च~।\\
 न्यासो (सा) पन्याससमं पाणिसमं चेति (विज्ञेयम्)~॥}
\end{quote}
 
\vspace{-5mm}
\begin{flushright}
इति~॥~१३९~॥
\end{flushright}
 
(अक्षरसमं) पदगतं तावद्गाने गुरुलघुनी वाद्ये समम्~॥~१४०~॥\\

(अङ्गसमं) ताले सामाङ्गानि तु चत्वारि सामानि~। तत्राङ्गं ध्रुवाणां स्थितप्रवृत्तादौ~। तेन बन्धस्फुटादिसमुदायसाम्यमङ्गसाम्यम्~॥~१४१~॥\\

तद्भाभूताभिः प्लुतलघुगुरुकलादेः साम्यम्~॥~१४२~॥\\

लययतिसाम्ये व्याख्याते~॥~१४३ \textendash\ १४४~॥

\newpage
% चतुस्त्रिंशोऽध्यायः ४४३ 

\begin{quote}
{\na ततावनद्धवंशानामेकश्रुतिकृतोऽपि च~।\\
ग्रहो गानेन सहितं तत्तु ग्रहसमं भवेत्~॥~१४५~॥

न्यासापन्यासयोगस्तु \renewcommand{\thefootnote}{1a}\footnote{(N) : स्वराणां हि यदा भवेत् (N \textendash\ ११८ b)}स्वराणां तु भवेद्यथा~।\\
\renewcommand{\thefootnote}{1b}\footnote{N: वंशवीणोभ्दवं वाद्यं ( V.११८C (N))}तद्वद्वाद्यं यदातोद्ये न्यासापन्यासजं तु तत्~॥~१४६~॥

\renewcommand{\thefootnote}{1c}\footnote{(N) : समपाण्यावपाण्युक्तं}समपाण्यर्धपाणिस्थं तथैवोपरिपाणिकम्~।\\
\renewcommand{\thefootnote}{1d}\footnote{() : गीतवाद्यं स्यात्}गीतवाद्यानुगं वाद्यं ज्ञेयं\renewcommand{\thefootnote}{1e}\footnote{भ. () स्मृतं~।}पाणिसमं तु तत्~॥~१४७~॥

अष्टादशजातिकमिति यदुक्तं तदनुव्याख्यास्यामः~। तद्यथा~।\renewcommand{\thefootnote}{1}\footnote{अष्टादशजातय}\\
शुद्धा \renewcommand{\thefootnote}{2}\footnote{भ.दुष्कर~।}पुष्करकरणा विषमा विष्कम्भितैकरूपा च~।\renewcommand{\thefootnote}{2a}\footnote{१४८  \textendash\ १४९ read as V. 120 (N) which reads: एका शुद्धा च सोपर्या चाष्टादशानुरूपा च~। देशानुरूपगोमुखपर्यायाश्चार्धकाष्टा च~। विश्वस्ता पर्यस्ता पार्ष्णिसमा समलवाविधूता च~। पुष्करकरणाच्चिति का ज्ञेया कीर्णाविकीर्णा च~॥~ () एतासां जातीनां लक्षणं व्याख्यास्यामः~। then}\\
पार्ष्णिसमा पर्यस्ता समविषमकृता च (व) कीर्णा च~॥~१४८~॥

पर्यवसानोच्चितिका संयुक्ता संप्लुता महारम्भा~।\\
विगतक्रमा विगलिता वञ्चितिका चैकवाद्या च~॥~१४९~॥

एतासां जातीनां लक्षणनिदर्शनान्यभिव्याख्यास्यामः~।\renewcommand{\thefootnote}{3}\footnote{(N) reads V.121 (extra) as धात्थि धमर्थिटो छाक एभिर्या त्वक्षरैः समायुक्ता~। स्यादेकाख्या जातिः शृङ्गारे सा तु सस्त्रीणां~॥}\\
एकाक्षरकृतं वाद्यं यद्भवेत् सार्वमार्गिकम्~।\\
नित्यं करणयोगेन सा शुद्धा नामतो यथा~॥~१५०~॥\renewcommand{\thefootnote}{4}\footnote{V १५० (G.O.S) :V१२२ (N) it reads as  \textendash\  एकाक्षरं द्वयक्षरं वा वाद्यंयत्सर्वमार्गिकम्~। भवेत्का (क) रणयोगेन सा शुद्धा जातिरूच्यते~॥  (V \textendash\ १२२ \textendash\ N)}}
\end{quote}
 
\hrule

\vspace{2mm}
{\qtt स्वरगतं}~। नन~। तद्दर्शयितुमेतत् साम्यमुक्तम्~। तद्यर्शयितुमेतत् साम्यमुक्तम्~। यत्रादौ श्रुतिस्तत्साम्यं ग्रहणम्~। विपरीतं वा साम्यं वा~॥~१४५ \textendash\ १४६~॥\\

तत्र गीतायरत यद्वाद्यं तत् पाणिसमम्~। इत्येत .......लाङ्गसाम्यमाहुः~॥~१४७~॥\\

अथोक्तं पूर्वोक्ते समस्तं यज्जायतेNजायमानत्वाद्रसभावप्रकृत्यादिविशेषप्रतिपत्तेर्जननहेतुत्वाज्जातिवाच्यं प्रयोगं वक्तुमुपक्रमते~। {\qtt अष्टादशजातिकमिति}~।

\begin{quote}
{\qt शुद्धा दु (पु) ष्करकरणा विषमा विष्कम्भितैकरूपा च~।\\
 पार्ष्णिसमा पर्यस्ता समविषमकृताऽवकीर्णा च~।\\
पर्यवसानोच्चितिका संयुक्ता संप्लुता महारम्भा~।\\
विगतक्रमा विगलिता वञ्चितिका चैकवाद्या च~॥}
\end{quote}

\noindent
इति~॥~१४८ \textendash\ १४९~॥\\

{\qtt कचिल्लक्षणमेव}~। कचिदुदाहरणमेव~। अन्यदूह्यम्~। एकस्याक्षरस्य कृता आवृत्तिर्यत्र~। {\qtt सार्वमार्गिकमिति}~। रसभावौचित्यात्~॥~१५०~॥

\newpage
% ४४४ नाट्यशास्त्रे 

\begin{quote}
{\na धंधंद्रक्लाखोखोहाणेति विहितवाक्या~।\\
सा शुद्धा विज्ञेया मध्यस्त्रीणां सदा जातिः~॥~१५१~॥\renewcommand{\thefootnote}{1}\footnote{VS. १५१ \textendash\ १६९ (pp. ४४०) (G.O.S.) : VS १२३ \textendash\ १३९ (N) which read totally differently is given as appendix \textendash\ IV}

स्वस्तिकहस्तविचारा सर्वमृदङ्गप्रहारसंयुक्ता~।\\
सा त्रिलयवाद्ययुक्ता पुष्करकरणा भवेज्जातिः~॥~१५२~॥

घेंतांकेंतांखेंतां दीर्घकृतैरक्षरैः कृता या च~।\\
राज्ञां स्वभावगमने सा विषमा नामतो जातिः~॥~१५३~॥

गुरुयुग्मं लघुयुग्मं तोटकं वाऽपि~।\\
नित्यं यत्र तु वाद्ये विष्कम्भा नाम सा जातिः~॥~१५४~॥

वामोर्ध्वकप्रवृत्तादोघे क्षिप्तावकृष्टलययुक्ता~।\\
सा करुणा स (णांश) प्राया जातिः स्यादेकरूपा तु~॥~१५५~॥

थित्थं थिकट् थिंधिकटं मटथिकरणैः सपार्ष्णिकृतैः~।\\
थेक्लेटाधेग्रथिता पार्ष्णिसमा सा भवेज्जातिः~॥~१५६~॥}
\end{quote}

\hrule
 
\vspace{2mm}
मध्यमस्त्रीणां शुद्धेत्युक्तत्वाद्विपरीतलक्षणा युक्ता~॥~१५१~॥\\

दु (पु) ष्करकरणेति~। दु (पु) ष्करं चित्रं स्वस्तिकादि करणं यस्याः~। साधनं ....तस्यामित्याहुः~। अस्या उदाहरणमूह्यम्~। घटविप्युघटगणणधुंधुं इत्यादि एवमन्यत्र~॥~१५२~।\\

दीर्घाक्षररूपात् स्वभावरूपे गमने राज्ञाम्~। {\qtt विषमेत्युक्तत्वात्}~॥~१५३~॥\\

गुरुलघुपातौ (तोटकमिति) लयुयुगैकगुरुरूपा राज्ञां त्वरितगतौ विष्कम्भिता~। त्वया मयि ते सत्त्वेन विष्कम्भो माधुर्यमस्यामिति (विष्कम्भिता)~॥~१५४~॥\\

({\qtt वामेत्यादि})~। ओघे इत्यादि~। करुणेए (णै) करूपा~। {\qtt एकाक्षरसूचनादिति}~॥~१५५~॥\\

(दि (थि) त्थमित्यादि~। पौर्ष्या (पार्ष्ण्या) घातकृता पार्ष्णिसमा~। {\qtt करुणेतरविषयसोपग्रहणोपस्थाने}  (प्रयोगः)~॥~१५६~॥

\newpage
% चतुस्त्रिंशोऽध्यायः ४४५

\begin{quote}
{\na ताघेंतांतांदोघेदोह्णमित्यक्षरैस्तु संयुक्ता~।\\
पर्यस्ता जातिरियं मध्यमपुरुषेषु कर्तव्या~॥~१५७~॥

कृत्वोपरिपाणिकृतं वाद्यं यद् द्रुतलयं समारूढम्~।\\
पुनरेव समलयं स्यात् समविषमा सा तु विज्ञेया~।१५८~॥

अवकीर्णं यद् द्विगुणैस्त्रिगुणैर्वा करणं मृदङ्गानाम्\renewcommand{\thefootnote}{1}\footnote{च. मृदङ्गेषु~।}~॥\\
पणवेषु दर्दरेष्वव्यवकीर्णा नाम सा जातिः~॥~१५९~॥

पर्यवगच्छति पूर्वं यस्मिन् वै प्रस्तुतं करणजातम्~।\\
त्रिविधेऽपि तु लययोगे पर्यवसाना तु सा जातिः~॥~१६०~॥

घेंटांदोह्णंण्णुणहामेभिः स्यात्त्वक्षरैर्हि सम्बद्धा~।\\
साम्यार्धे उच्चितिका जातिर्वाद्ये तु बोद्धव्या~॥~१६१~॥

थंकेटांकेटकिणामेभिर्या त्वक्षरैस्तु संयुक्ता~।\\
संयुक्तकृता जातिः सा सा (का) चेटगतौ विधातव्या~॥~१६२~॥}
\end{quote}

\hrule

\vspace{2mm}
(ताघेमित्यादि~।) गुर्वक्षरत्वेऽपि यत्र स्वरस्य नानाकारोऽनेकगत्यादेर्नैकरूपत्वं सा मध्यमपुरुषेषु पर्यस्ता~॥~१५७~।\\

(कृत्वेति)~। गुरुत्वेन परितः क्षेपो यस्याः स्थायिनि ह्लादत्वात् तदेव द्रुतलयोपरिपाणितया समविषमा~॥~१५८~॥\\

(अवकीर्णमिति) अधमानां (समविषमगतिप्रचारेषु) तेषामेव सम्भ्रमावकीर्णे यत्र तदेव द्विगुणं द्विगुणीक्रियते~। त्थोत्थोमटमट धिधिमटमटधिधीत्यादि~॥~१५९~॥\\

(पर्या (र्यव) गच्छतीति~।) प्रस्तुतं करणजातं युक्तं यस्याः पर्यवसाने क्रियते सा पर्यवसाना~। तेषामेव त्रिषड्विधलयेष्विति~। नृत्तादियोगलय इत्यर्थः~॥~१६०~॥\\

( द्रोह्णा (घेण्टांदोह्ण) मित्यादि~।) उल्लसिताऽपि निश्चैतन्यमस्यामिति~। उदस्यते चितिका उच्चितिका~॥~१६१~॥\\

(थंकेटांके इत्यादि~।) काचेटानां गतौ संयुक्ता~। एकाङ्किमुख एव~॥~१६२~॥

\newpage
% ४४६ नाट्यशास्त्रे

\begin{quote}
{\na सर्वाङ्गुलिचनकृता सर्वमृदङ्गप्रहारसंयुक्ता~।\\
भीतनभोयानगतौ संप्लुतजातौ विधातव्या~॥~१६३~॥

अवपाणिकरणयुक्तं कृत्वादौ मध्यलयतुल्यम्~।\\
वाद्यं द्रुतलयमन्ते जातिः स्यात् सा महारम्भा~॥~१६४~॥

ऊर्ध्वाङ्कदक्षिणमुखे क्षिप्तप्रहता वितस्तमार्गा च~।\\
अङ्कोर्ध्वकप्रवृत्ता दक्षिणवामप्रहारजाचा च~॥~१६५~॥

प्रायेणोद्धतमार्गा\renewcommand{\thefootnote}{1}\footnote{भ. वाद्या~।} वितस्तमार्गाश्रयेण दिव्यानाम्~।\\
धंद्रां (घुंधुं) धंद्रांप्राया जातिर्विगतक्लमा नाम~॥~१६६~॥

लघ्वक्षरभूयिष्टा विचित्रकरणा च सर्वमार्गेषु~।\\
वाद्याऽविसर्पिकरणा विगलितनामा\renewcommand{\thefootnote}{2}\footnote{भ. करणा~।} तु सा जातिः~॥~१६७~॥

करणैर्बहुभिश्चितैः (त्रैः) सर्वमृदङ्गप्रहारसंयुक्ता~।\\
स्वाभाविकोत्तमगतौ वञ्चितिका सा तु विज्ञेया~।~१६८~॥ 

ध्रोध्रोध्रेध्रेधोधं एभिर्या त्वक्षरैश्च संयुक्ता~।\\
सा ह्येकवाद्यजातिर्वृ (र्नू) त्तगतिविधानवत् कार्या~।~१६९~॥}
\end{quote}

\hrule

\vspace{2mm}
({\qtt सर्वाङ्गुलीति})~॥~१६३ \textendash\ १६४~॥\\

(ऊर्ध्वाङ्केति) दिव्यानामिति~। सार्धयार्यया सम्बन्धः~। वितस्ता .... खेति~। वाद्यग्रहणं दर्शयति~। अङ्कोर्ध्वक इति~। प्रवृत्तेति~। उदाहरणदिशं दर्शयति धुंधुं इति~॥~१६५ \textendash\ १६६~॥\\

(लघ्वक्षरेति~।) चित्रलघ्वक्षरवाद्ये अविसर्पि अपुनारूपेण करणं क्रिया यस्याः अपुनरुक्तवाद्येत्यर्थ:  सा विगलिता नाम~। विटादिपरिक्रमे विशेषेण गलितमनावृत्या ग्रस्तमिवाक्षराणां यत्रेति~॥~१६७~॥\\

(करणैरिति)~। चित्रप्रहाराः सर्वमृदङ्गानाम्~। राज्ञोऽन्येषामुत्तमानां गतौ सा वश्चितिका~। वञ्चि  (ञ्चु) गताविति (पा. धातुपाठः १८९)~॥~१६८~॥\\

(ध्रो ध्रो इति) एकमेवाक्षरं संयुक्तं पुनरन्येन संयुक्तं पुनरन्येनेति वैचित्र्येण येन वाद्यते नृत्ते (सा) एकवाद्या~॥~१६९~॥

\newpage
%चतुस्त्रिंशोऽध्यायः ४४७

\begin{quote}
{\na एवमेतेन विधिना कर्तव्यं वादनं बुधैः~।\\
गतिप्रचारे गीते च\renewcommand{\thefootnote}{1}\footnote{भ. वा.} दशरूपे विशेषतः\renewcommand{\thefootnote}{2}\footnote{भ. तथैव च}~॥~१७०~॥

सप्तरूपविधानेन \renewcommand{\thefootnote}{3}\footnote{भ. (N) स्वच्छन्दा}छन्दकासारितेषु च~।\\
तत्त्वं चानुगतं चैव \renewcommand{\thefootnote}{4}\footnote{भ. चापि}तद्धौ (दौ) घो वाद्यमिष्यते\renewcommand{\thefootnote}{4a}\footnote{(N) वाद्यमेव च}~॥~१७१~॥

वाद्यं गुर्वक्षरकृतं तथाल्पाक्षरमेव च~।\\
गतिप्रचारे कर्तव्यं गाने सम्यगिहेच्छता\renewcommand{\thefootnote}{5}\footnote{भ. (N) साम्यार्धमिच्छताम्}~॥~१७२~॥

तत्त्वं चानुगतं \renewcommand{\thefootnote}{6}\footnote{भ. चैव}चापि ओघश्चापि कदाचन~।\\
राज्ञां ललितगामित्वाद्वाद्यं योज्यं स्वभावजम्~॥~१७३~॥

तत्त्वं तु प्रथमे गाने द्वितीयेऽनुगतं भवेत्~।\\
तृतीये त्वोघसंज्ञं तु वाद्यं गति \renewcommand{\thefootnote}{7}\footnote{(N) : यतिपरिक्रमे}परिक्रमे~॥~१७४~॥}
\end{quote}

\hrule

\vspace{2mm}
{\qtt एतदुपसंहरति}~। एवमेतेनेति~। अस्य जात्यष्टादशकरूपस्य प्रयोगस्येतिकर्तव्यता नाट्यनृत्तापेक्षयाऽभिधान (या) स्य प्रथमं तावदुभयरूपतामाह~। गतिप्रचारे गीते चेति~। उत्तरार्धात् त्रिधाऽस्य मा (गा) नापेक्षया समुच्चये~। तेन लययोजना~। दशरूपे च गीते ध्रुवागीतविषये~॥~१७०~॥\\

गीतकसप्तकस्य विधाने प्रयोगे च छन्दकासारितप्रयोगे च वादनं कर्तव्यम्~। विशेषतो गतिप्रचार इति~। पुनरेवार्धश्लोकमावृत्य गीतकादौ तत्त्वादित्रयरूपं स्वातन्त्र्येण वाद्यमिष्यत इति सम्बन्धनीयम्~। तत्र हि वाद्यस्यापि प्राधान्यं प्रयोज्यत्वेन भवति~। यथोक्तं तुर्याध्याये~। गत्या वाद्यानुसारिण्येइति (भ.ना.४.२७४)~। नाट्ये गत्यपेक्षया नियतं वाय (द्य) म्~॥~१७१~॥\\

तदाह~। वाद्यं गुर्वक्षरकृतमिति~। अत एवाल्पवाद्याक्षरम्~॥~१७२~॥\\

तत्त्वादीनां चान्यतमं कदाचिद्भवति~॥~१७३~।\\

तत्रापि यदा नृत्तांशे वा प्रावेशिक्यादौ वा क्रमत्वरायोगे वा त्रयाणामपि तत्त्वादीनां प्रयोगस्तदा क्रममाह~। तत्त्वं तु प्रथमे गान इति~॥~१७४~॥

\newpage
% ४४८ नाटयशास्त्रे

\begin{quote}
{\na \renewcommand{\thefootnote}{1a}\footnote{(N) शेषाणां}ध्रुवासु छन्दतश्चित्रं \renewcommand{\thefootnote}{1}\footnote{भ. (N) ध्रुवाणां}शेषाणां सम्प्रयोजयेत्~।\\
तथा स्थितावकृष्टायां वाद्यं त्वनुगतं भवेत्~॥~१७५~॥

प्रावेशिकीनां कर्तव्यं तत्त्वं चानुगतं तथा~।\\
नैष्क्रामिक्यन्तरकृतं कार्यं त्रिलयवाहितम्~॥~१७६~॥

द्रुते प्रासादिकीनां तु द्रुतं चो (चौ) घं च वादयेत्~।\\
एवं ध्रुवाणां कर्तव्यं वाद्यं प्रकरणान्वितम्~॥~१७७~॥

मात्रांशकविकल्पस्तु ध्रुवापादेषु यो भवेत्~।\\
स तु भाण्डेन कर्तव्यस्तज्ज्ञैर्गतिपरिक्रमे~॥~१७८~॥

एवं गतिप्रचारेषु कार्यं वाद्यं प्रयोक्तृभिः~।\\
ताण्डवे सुकुमारे च वाद्यं वक्ष्यामि तत्त्वतः~॥~१७९~॥

अभाण्डमेकं गानस्य परिवर्तं प्रयोजयेत्~।\\
तस्या (था) न्ते सन्निपाते च कार्यो भाण्डग्रहो बुधैः~॥~१८०~॥}
\end{quote}

\hrule

\vspace{2mm}
गतौ ध्रुवासु गाने गीतिक्रियायां तद्राजविषयं राज्ञामित्यनुवृत्तौ (भ.ना. ३४.१७३)~। शेषाणामधः \textendash\   (थ) प्रकृतीनाम्~। छन्दत इति~। इच्छातः~। विचित्रं तत्त्वादि~। (स्थितावकृष्टाया) मनुगतात्मा~॥~९७५~॥\\

प्रावेशिक्यां तत्त्वौघम्~। अन्ये तु तथा शब्दादोघ इत्याहु~। नैष्क्रामिक्यामध्रुवायां च किल~॥~१७६~॥\\

द्रुतासु प्रासादिकीपु द्रुतलयेनो (नौ) धेन~।  (प्रकरणेति) प्रकरीमुद्धतमसृणादिसंयुक्तं विचार्येत्यर्थ~॥~१७७~॥\\

प्राधान्येन ध्रुवासु तत्त्वम्~। तत्राऽपि तत्त्वाष्टकमध्ये लयसममक्षरसमं च प्रधानमिति दर्शयति~। मात्रांशकविकल्पस्त्विति~॥~१७८~॥\\ 

एतत्त्वन्यविषयप्राधान्येनोक्तत्वान् नृत्तप्राधान्यमाह~। ताण्डवे सुकुमारे चेति~॥~१७९~॥\\

अभाण्डमेकमित्यादि~।......................... तदन्यत्र प्रयोगक्रमसम्पत्तये तूदितम्~। तथा चेति~। यदि चेत्यर्थ~॥~१८०~॥

\newpage
% चतुस्त्रिंशोऽध्यायः ४४९

\begin{quote}
{\na अथवा नृत्तशोभार्थमङ्गानां परिवर्तनम्~।\\
सङ्गीतस्य प्रकर्तव्यं लयस्य च निवर्तनम्~॥~१८१~॥ 

\renewcommand{\thefootnote}{1}\footnote{भ (N) यत्राङ्गहारनृत्यं}यत्राभिनेयमङ्गे तु तत्र वाद्यं प्रयोजयेत्~।\\
यत्र पाणिवशादङ्गं भूयो भूयो निवर्तते~।\\
स तत्रार्थोऽभिनेयस्तु शेषं नृत्तेन योजयेत्~॥~१८२~॥

यद्धृत्तं तु पदं गाने तादृशं वाद्यमिष्यते~।\\
गीतवाद्यप्रमाणेन कुर्याच्चाङ्गविचेष्टितम्~॥~१८३~॥

यथा गुर्वक्षरं चैव तथाल्पाक्षरमेव च~।\\
\renewcommand{\thefootnote}{2}\footnote{भ. मुखोपवहने कार्य प्रकृष्टं}मुखे सोपोहने कार्यं प्रकृतेर्वर्णतस्तथा~॥~१८४~॥}
\end{quote}

\hrule

\vspace{2mm}
\noindent
अङ्गादीनामपि सम्पिष्टकादीनाम्~। अनेनैतदुक्तम्~। क्वचिदुपवर्तनादौ शास्त्रोक्तमेव परिवर्तनम्~। कचिदनुक्तमपि नृत्तशोभार्थंक्रियत इति~। अत एव सकलस्यापि गीतस्य वस्तुप्रायस्य त्वविलम्बितादेर्गीतशोभार्थं पुनरावृत्तिर्भवतीति~। सङ्गीतस्येति~। साकल्येऽव्ययीभावः~॥~१८१~॥\\

एवं स्थिते प्रकृते किमित्याह~।\\

यत्राधीने त्वभिनयं (त्राभिनेयमङ्गे तु) तत्र वाद्यं प्रयोजयेत्~। इति~॥~१८२~॥

ननु नास्ति त (य) त्र वाद्यमिति सम्भवमात्रं तत्र वाद्यम्~। न तु प्रधानमित्युक्तम्~। तस्य विधिमाह~। यद्धृत्तं तु पदं गाने तादृशं वाद्यमिति~। अतश्चेदं सम्पाद्यत इत्याशङ्क्याह~।  गीतवाद्यप्रमाणेन कुर्यात् स्वाङ्गविचेष्टितम्~। इति~॥~१८३~॥\\

गुर्वक्षरैरेव वाद्यम्~। इदं हि गीतकयोगेनावृत्तं भवति~। {\qtt मुखोपवहनम्}~। यन्मुखं तत्~। यदा प्रकृतेइति यदा भागमपेक्ष्य वर्ण्यत इति~॥~१८४~॥

\newpage
page content missing

\newpage
page content missing

\newpage
page content missing

\newpage
page content missing

\newpage
% ४५४ नाटयाशास्त्रे

\begin{quote}
{\na   चित्रं हि करणं यत्र वाद्येनैकेन वाद्यते~।\\
 वृत्ताङ्गहारानुकृतो ज्ञेयः साचीकृतस्तु सः~॥~२०८~॥

 कृत्वोपरिगतं वाद्यं पणवो दर्दरोऽपि वा~।\\
 1प्रयाति मुरजं यत्र समलेखः स कीर्तितः~॥~२०९~॥

 चित्रं बहुविधं वाद्यं मृदङ्गपणवादिभिः~।\\
 क्रियते यत्र संरब्धैश्चित्रलेखः स उच्यते~॥~२१०~॥

 सर्वमार्गगतो यस्तु सर्वपाणिलयाश्रयः~।\\
 1aविचित्रश्च विभक्तश्च तत्समं समवायितम्~॥~२११~॥

 यस्तु मध्यलयोपेतः समः सुविहिताक्षरः~।\\
 गतिप्रचारे विहितः प्रकारो दृढ एव सः²~॥~२१२~॥

 एवमेते प्रकारास्तु कर्तव्या वाद्यसंश्रयाः~।\\
 गतिप्रचारे गीते 3वा रसभावानवेक्ष्य च4~॥~२१३~॥

 5aप्रकारा जातयश्चैव सर्वमार्गेषुः संस्थिताः~।\\
 ये वै गतिप्रचारेषु शुद्धास्ते केवला मताः~॥~२१४~॥}
\end{quote}

\hrule

% content and footnote missing

\newpage
% चतुस्त्रिंशोऽध्यायः  ४५५

{\qt  1प्रयोगमिदानीं वक्ष्यामः~। तत्रोपविष्टे प्राङ्मुखे रङ्गे 2कुतप एव विन्यासः कर्तव्यः~। तत्र पूर्वोक्तयोर्नेपथ्यगृहद्वारयोर्मध्ये कुतपविन्यासः कार्यः~। तत्र रङ्गाभिमुखो मौरजिकस्तस्य पाणविकदर्दरिकौवामतः~। एष प्रथममवनद्धकेन तस्य ततः कुतपविन्यास उक्तः~। तत्रोत्तराभिमुखो गायकः~। गायकस्य तु वामपार्श्वे वैणिकः~। वैणिकस्य दक्षिणेन वंशवादकौ~। गातुरभिमुखं गायिका~। इति कुतपविन्यासः~॥~२१५~॥}\\

\noindent
{\qt तत्र~।3}

\begin{quote}
{\na प्रयोगेषु प्रणेतव्यो देशस्थानवशानुगः~।\\
 यथास्थानस्थितेष्वेषु वामकातोद्यकेषु च~॥~२१६~॥}
\end{quote}

{\qt एवमचलाकम्पितास्खलितासन्नो (नो) पविष्टेषु  मार्दङ्गिकदार्दरिकपाणविकेषु4 शिथिलाञ्चितवध्र5 स्तनितेषु यथाग्रामरागमार्जनालिप्तेषु मृदङ्गेपूरुद्वयनिपीडने तेषु निगृहीतार्थनिगृहीतमुक्तप्रका (हा) रकृतेषु र्दरवादनविन्यस्तैर्देवतानामावाहनविसर्जनार्थं प्रथममेव तावत् त्रिसाम कर्तव्यम्~॥~२१७~॥}\\

\hrule

% content and footnote content missing

\newpage
% ४५६ नाट्यशास्त्रे

तत्र~। 

\begin{quote}
{\na  वामे चान्द्रमसं साम तत् प्रीणयति पन्नगान्~।\\
 दक्षिणे जालजं साम तदृषीन् प्रीणयत्यथ~॥~२१८~॥

 उदीच्यामपि चाग्नेयं तद्वृहद्देवतानि च~।\\
 1त्रिसाम कीर्तितं सम्यक् त्रिसामत्वाद्धुधैरथ~॥~२१९~॥

 त्रिप्रकारं त्रिगुणितं तथा चैवाड्डिताश्रयम्~।\\
 त्रिकलं षट्कलं चैव त्रिसाम परिकीर्तितम्~॥~२२०~॥

 त्रिसामाक्षरपिण्डस्तु गुरुलघ्वक्षरान्वितः~।\\
 यकारश्च मकारश्च त्रिकैस्त्रिगुणितं भवेत्~।२२१~॥}
\end{quote}

% content and footnote content missing

\newpage
% चतुस्त्रिंशोऽध्यायः ४५७ 

\noindent
{\qt समेनाक्षरसमेन वा वाद्येन बहिर्गीतविधानमनुवर्तितव्यम्~। आसारितप्रयोगे च तत्त्वानुगतप्रायं वाद्यं विधातव्यम्1~। यत्र त्रिसाम2 स प्रत्याहाराय (रो यां) चासाववतीर्णकोटी त (टिस्त) त्र वाद्यं प्रवर्तते~। तत्रादौ तावद्वामोर्ध्वकप्रहारयुतं पश्चादालिङ्गनविमर्दनकरं गोपुच्छां च त्रिसामा विचित्रकरणयुतं मार्दङ्गिकसर्वभाण्डिकमपि वाद्यं पश्चात् प्रवर्तनीयम्~। पूर्वं हि भाण्डवाद्येन सिद्धिरूत्पादनीया~। स्त्रीबालमूर्खावकीर्णे च रङ्गे कुतूहलजननसमर्थवाद्यमुपपन्नं भवति~॥~२२२~॥}

\noindent
{\qt अपि च~।}

\begin{quote}
{\na  आचार्याः सममिच्छन्ति पदच्छेदं तु पण्डिताः~।\\
 3स्त्रियो मधुरमिच्छन्ति विकृ (क्रु) ष्टमितरे जनाः~॥~२२३~॥}
\end{quote}

{\qt तस्य (स्यैव) चान्ते ताण्डवप्रयोगमधिकृत्य नर्तक्या अवतरणकाले लघुवर्णसञ्चयमङ्किकेऽङ्गुलिप्रचलनप्रायं वाद्यं योज्यम्~। गीतकसन्निपाते च लास्या  (स्य) मनुवाद्यम्~। तेन नृत्ताङ्गहारानुगतं करणवाद्यं प्रयोज्यम्~। आलिप्तमार्गेण जातिकरणान्वितमुत्थापने वाद्यं प्रयोक्तव्यम्~। ततः परं परिवर्तनं चातुर्मार्गिकं शुद्धजात्याश्रयं चतुर्थकारप्रवेशेषु वाद्यमङ्गुलिप्रचलनप्रायमिति~। नान्दीशुष्कावकृष्टासु करणाश्रितं वाद्यं}


% content and footnote content missing

\newpage
% ४५८  नाटयशास्त्रे

\noindent
{\qt योज्यम्~। रङ्गद्वारे प्ररोचनाया जर्जरे त्रिगते चार्यां चाड्डितमार्गाश्रितं सन्निपातग्रहं प्रयोक्तव्यम्~। महाचार्यां वितस्तिमार्गाश्रितं प्रदेशिनीग्रहम्~। एवं तावत् पूर्वरङ्गे भाण्डवाद्यमुक्तम्~। नायकानां तु स्वस्थगतिप्रचारं वक्ष्यामः~। तत्र घंद्रांद्रांद्रां इति देवतानाम्~। घेंतांकेतां\ldots\ldots. कृणकिटिप्रायं विप्राणाम्~। व्यञ्जकप्रायं मध्यमानाम्~॥~२२४~॥}

\begin{quote}
{\na एवं स्वस्थावस्थागतिप्रचारे विधीयते वाद्यम्~।\\
 द्विकलाश्च तथैककलाश्चतुष्कलाश्रैव पादविन्यासाः~॥~२२५~॥

 तत्र तु साम्यं कार्यं भाण्डे च समं च गानेन~।\\
 पुनश्चावस्थाकृतं वाद्यं वक्ष्यामि त्वरितगमने~॥~२२६~॥

 धंधंघेघेटं इत्येवं वाद्यं प्रयोक्तव्यम्~।\\
 रथविषमादिगतमङ्गुलिप्रचलनकृतं विहितम्~॥~२२७~॥

 (तत्रै) व पूर्वं सम्यग् लिखितं गतिप्रचारे तु~।\\
 पुनरन्यावस्थायां वाद्यविधानं प्रवक्ष्यामि~।२२८~॥

 त्वरितगमनेषु भावा ये पूर्वोक्ता गतिप्रचारेषु~।\\
 ध्रंघेंघेंघेंप्रायं तत्र तु वाद्यं प्रयोक्तव्यम्~॥~२२९~॥

 नौरथविमाननेपथ्यानिलाकाशजेऽङ्गुलीचलनात्~।\\
 कार्यं हि तच्चतुष्केऽप्यन्योन्यसमाश्रितं वाद्यमिति~॥~२३०~॥

 दुःखार्दितव्याघितेष्टजनवियोगविभवनाशवधबन्धः~।\\
 व्रतनियमोपवासादियुक्तेष्वालिप्तमार्गबन्धो विधातव्यः~॥~२३१~॥

 दैत्यदानवयक्षराक्षसपन्नगादीनां धंद्रधंद्रांधकुताम्~।\\
 घेधेटाप्रायं खञ्जविकलपङ्गुवामनादीनाम्~॥~२३२~॥

 टघटाघेप्रायं चेटकुसत्त्वादीनां थोकटखुखणेत्यादिकम्~।\\
 यतिपाशुपतशाक्यादीनां धहुधहुधेधेप्रायम्~।२३३~॥}
\end{quote}

\hrule
 
\vspace{2mm}
\noindent
यस्य ग्रहणं तत् प्रदेशिनीग्रहणम्~। एवं पूर्वरङ्ग इत्युपसंहृत्य नायकविषयमाह~। नायकानां त्विति~। यद्यपि  जातिविनिश्चययोगो गतिप्रचारे कृतस्तथापि ग्रहमोक्षे सन्धानयोगान्येतान्यक्षराणि तत्र मिश्रणीयानीत्येवं योजनामाहुः~। अन्ये तु विकल्पमिच्छन्ति~॥~२२४~॥\\

गतिप्रचार इति गत्यध्याये (अ.१२)~। चतुष्कल (ला) इति~। आङ्गिकमुखद्वये ऊर्ध्वकालिङ्गके~॥~२२५ \textendash\ २३३~॥

\newpage
% चतुस्त्रिंशोऽध्यायः १५९

\begin{quote}
{\na  औपस्थापकनिर्मुण्डवर्षवरादीना\\
 धेटांघोंटांभांटांण्णाणाप्रायम्~।\\
 वृद्धश्रोत्रियकञ्चुकिस्थूलादीनां\\
 खोंध्रोधोखोखो इति प्रायम्~॥

 गजवाजिखरोष्ट्ररथविमानयानेषु वंकिटिप्रायम्~।२३४~॥}
\end{quote}

{\qt सर्वत्रोत्तममध्यमाधमेषु पुरुषेष्वित्थं रसभावानभिसमीक्ष्य वाद्यं प्रयोक्तव्यम्~। एवं तावत्  पुरुषवाद्यम्~॥~२३५~॥}\\

{\qt स्त्रीणां पुनरभिव्याख्यास्यामः~। तत्रोत्तमस्त्रीणां मध्यानां च टोस्वधधिदोघेंटमधिके इति प्रायम्~। अथ राजस्त्रीणां धंकिह्नमथिथिदोह्नकखुखुप्रायम्~। धंकितिकिथिघटमटमथिथे इति ब्राह्मणीनाम्~। अथ मध्यमस्त्रीणां वेश्याशिल्पकारीणां किटिण्णाणं थंकथिंघटमथिहुणकिटिकिटिप्रायम्~। टमथिकुण (केडखुखिखिप्रायं गीतं स्त्रीणाम्)~। एवं तावत्  सामान्यतोऽभिहितं स्त्रीणाम्~। अवस्थान्तरितानां तु त एव पौरुषा वाद्यविशेषा भवन्ति~।  सामान्यतोऽभिनये भयशोकक्रोधादयो हि भावा आसां समुत्पद्यन्ते~। आसामपि  रसभावाभिनयापेक्षं मार्गाश्रितं वाद्यं भवति~॥~२३६~॥}\\

{\qt अपि च~। }

\begin{quote}
{\na  जातिमार्गप्रचारैस्तु करणैरक्षरैस्तथा~। \\
 वादयेद्यस्त्वसंकीर्णं स श्रेष्ठो वादको मतः~॥~२३७~॥}
\end{quote}

{\qt  अथान्तरवाद्यानि~। अनुबन्धो विप्रहारितं सिद्धिर्ग्र (ग्र) हणं परिच्छिन्नमिति~॥~२३८~॥}

\hrule 

\vspace{2mm}
औपस्थायि (प) को. .....प्रतिनटः~। निर्मुण्डो जातिखलतिः~॥~२३४ \textendash\ २३५~।\\

सामान्यतोऽभिनये सामान्याभिनये क्रोधादयः~। आसां स्त्रीणाम्~। अपतिशो वादयो दर्शिता इति~। तदनुसारेण स्त्रीणामपि वाद्यं न केवलं पुंसाम्~।~॥~२३६~॥\\

जातयः शुद्धा या इत्यादि व्याख्यातम्~॥~२३७~॥\\

गानपरिक्रमं च विना यद्वाद्यं तदुत्तत्त (दन्त) रवाद्यम्~। तच्चतुर्धा~।~२३८~॥

\newpage
% ४६० नाट्यशास्त्रे 

{\qt तत्रानुबन्धसंज्ञमन्तरवाद्यं यथा धोणाधोक्लघे इति~। पाठ्यविरामे विप्रहारितं यथा धंद्रांखोखोणा इति~॥~२३९~॥}\\

{\qt सिद्धिरपि~।}

\begin{quote}
{\na  ऐश्वर्ये विस्मृते शान्ते वस्त्राभरणसंयमे~।\\
 सिद्धिर्वादयितव्या तु नानाकरणसंश्रया~॥~२४०~॥}
\end{quote}

{\qt  सा च सिद्धिर्यथा समार्गकरणाश्रया कार्या विचित्रकरणा पञ्चकला~। अवस्थान्तरे पुनरेवमेव कृता भवति~॥~२४१~॥}\\

\begin{quote}
{\qt अपि (च) }

{\na  साध्यविरामे वाद्यं परिषत्साधनकृते विरामे च~। \\
 वस्त्राभरणनिपाते प्रकुर्यादभ्यञ्जने काले~।२४२~॥}
\end{quote}

{\qt अथ परिच्छिन्नं सार्वभाण्डिकम्~। ध्रुवायो (या) ग्रहोपशमने गम्भीरधीरप्रहारमुद्धतं यथारसं च षट्कलम्~। तत्रोत्तमानां वितस्तमार्गाश्रितं द्रंद्रंक्लेघेके इति~॥~२४३~॥}\\

{\qt अथोत्तमस्त्रीणामड्डितावाद्याश्रयणम्~। खो खो खा इति~। अधमानां खञ्जनर्कुटकस्थम्~। यथा टैकखाकिटिकिटिना इति~॥~२४४~॥}\\

\hrule

\vspace{2mm}
अनुबन्धो नाम यदङ्गानुसारेणाभिनयः~। मध्यम एव वर्तना~। प्रयोगकालवाद्यम्~। अध्यायसमापा (प्ता) ववान्तरवाक्यार्थछेदे विप्रहारितम्~। यथा यशश्शेषतामित्यत्र विशेषेण प्रहारः~॥~२३९~॥\\

अशङ्कितविभवलोभेऽभिनयपाठ्यादौ विधिः स्मृतः सिद्धिः~। पूरणार्थं सिद्धिग्रहणम्~॥~२४० \textendash\ २४२~।\\

ध्रुवागाने भविष्यति समाप्ते च यद्वाद्यं ध्रुवाषट्कपर्यन्तं नियतषट्कलत्वावधिकृतात् परिच्छिन्नम्~। तदन्तरवाद्यमेवोत्तरवाद्यविषये अक्षरैर्विभजति~। तत्रोत्तमानामित्यादिना~। ध्रुवाणां ग्रहणे उपशमने च यत् परिच्छिन्नं नाम चतुर्थमन्तरवाद्यमुक्तम्~॥~२४३ \textendash\ २४४~॥

\newpage
% चतुस्त्रिंशोऽध्याय ४६१

{\qt अथ प्रासादिकीप्रावेशिक्याक्षेपिक्यवकृष्टासु~। तत्र प्रासादिकीप्रावेशिक्योः समपाणौ विभक्तकरणं वाद्यम्~। द्रुतायामुपरिपाणिर्विचित्रकरणं स्थितावकृष्टायामर्थसन्निपातकृतमिति~। यथा धोण्णेधोयो इति~।२४५~॥}

\begin{quote}
{\na  एवमेतत् प्रयोक्तव्यं वाद्यं गतिपरिक्रमे~।\\
 प्रासादिक्यां ध्रुवायां च त्वन्तरायां तथैव च~॥~२४६~॥

 अभाण्डमेकं गीतं तु परिवर्तं प्रयोजयेत्~।\\
 सन्निपातावसानेषु भाण्डवाद्यग्रहो भवेत्~॥~२४७~॥}
\end{quote}

{\qt  अथ द्रुतविलम्बितायां यथा~। तलखो तलखो तलखो इति~। अथाड्डितायां यथा~। धोणखोखौणके...कुखुणाणाणा खो्द क्लखोंके इति~॥~२४८~॥}

\begin{quote}
{\qt अथोद्धात्यम्~।}

{\na सम्भ्रमावेगहर्षार्थे विस्मयोत्साहशोकजे~।\\
 ग्रहे गानस्य यद्वाद्यमुद्धात्यं सम्प्रकीर्तितम्~॥~२४९~॥}
\end{quote}

{\qt मोक्षमिदानीं वक्ष्यामः~। तद्यथा~। खेन्नांतिकिठिदत्तकित्तकिकिदोत्तगोगोगघेदोघेटघेघेरादो इति~। य (अ) थाड्डितायाम्~। घटमथिघंघंखोखोणखो \textendash\ खोणखोखो चेणं डक्कणिष्कषणुडुप्प इति~। अथ नर्कुटखञ्जयोः~। यौधंद्रांधंद्रातकितांगुदुगुटुघेइति~। अवस्थितायाम्~। धंधंघेटंमटघेघेत्थिमटांकटत्थिमटत्थिकटकिटीथिटी इति~। अथावकृष्टायाम्~। घटमटत्थिरथिगुटुघेट  (धोणवंधणणोणणं) इति~॥~२५०~॥}

\begin{quote}
{\na  एवमेते ग्रहा मोक्षा ध्रुवाणां गदिता मया~।\\
 निष्कामे च प्रवेशे च त्वाक्षेपिक्यन्तरासु च~॥~२५१~॥~1}
\end{quote}

\hrule

\vspace{2mm}
तत्र ग्रहे यद्वाद्यं तदेव षड्ध्रुवाभेदेन स्फुटयितुमाह~। अथ प्रासाद (दि) कीत्यादि~॥~२४५ \textendash\ २४८~॥\\

द्रुतायां ध्रुवायां विप्रहारिततस्य ध्रुवादिविषये नामान्तरेण प्रयोगमाह~। उद्धात्यम्~। उत्कृष्टं हननं यत्रेति~॥~२४९~॥\\

एवमुद्वीपनापरपर्याये ग्रहेऽभिधाय प्रशमनप्रलये मोक्षे दर्शयति~। मोक्षमिदानीमित्यादिना~॥~२५० \textendash\ २५१~॥

\newpage
page  content missing

\newpage
page content missing

\newpage
%४६४ नाटयशास्त्रे

\begin{quote}
{\na  शीतोदके निशामेकां स्थापयित्वा समुद्धरेत्~।\renewcommand{\thefootnote}{1}\footnote{म. लोमशैः रक्तचर्मकैः}\\
\renewcommand{\thefootnote}{1a}\footnote{(N) : बध्रैः सुविहितैः पात्रैः लोमशैः रक्तचर्मकैः}अ वध्रैः सुललितैर्दात्रैर्गोमयैरक्तमार्जि (मर्दि) तैः~॥~२६६~॥

 चन्द्रकैस्तनुभिः पश्चात् मृदङ्गान् योजयेद् बुधः~।\\
 पुष्पावर्तस्त (र्तं तं) तः कुर्यात् त्रिवर्तिं चन्दरकाश्रया (य) म्~॥~२६७~॥

 कक्ष्याख्यं वै परिकरं ग्रीवाख्यं स्वस्तिकं तथा~।\\
 \renewcommand{\thefootnote}{2}\footnote{भ (N) श्रितानि च}त्रिशतं च प्रकुर्वीत पुष्करे ह्याक्षिकानि तु~॥~२६८~॥

 दश तत्र हि वध्रास्तु प्रक्षेप्या वर्तिसंश्रयाः~।\\
 द्व (द्वे) हित्वा त्वाक्षिके वध्र (ध्रे) तृतीये संप्रवेशयेत्~।२६९~॥

 सर्वेष्वेवं प्रयोगोऽयमाक्षिकेषु विधीयते~।\\
 तन्त्रीभिः पणवं नह्येत् सुदृढाभिः समन्ततः~॥~२७०~॥

 वातपुष्करिकां चैव योजयेत्तनुचर्मणा~।\\
 एवं मृदङ्गपणवाः कर्तव्या दर्दरास्तथा~॥~२७१~॥}
\end{quote}

\hrule

\vspace{2mm}
पूर्वं गोमयेनाक्तास्ततः सुमर्दिताः~॥~२६६~॥\\

{\qtt चन्द्रकैरिति}~। वध्रा निरवद्यास्तथा कार्या यथा चन्द्रकाकाराः कचित् क्वचिद्वृद्धकुसुमवर्तनाकाराः~। त्रिवर्तिमिति~। त्रयाणां वध्राणामन्योन्ययोजनावैचित्र्येण~॥~२६७~॥\\

कक्ष्याख्यं वै परिकरमिति~। येन स्कन्धे निवेश्यत इत्येतत् पणववत् प्रवेशयेत्~।२६८~॥\\

अक्षके साध्ये सति द्वे वध्रे बन्धयित्वा स्पष्टात्तवध्रेषु मध्ये तृतीये वध्रे (प्र) वेशयेदिति पणवत्वम्~।२६९~।\\

ततस्त्रिभिर्बध्नीयात्~॥~२७०~। 

\begin{quote}
{\qt वातपुष्करिकां चैव योजयेत्तनुचर्मणा~।}
\end{quote}

\noindent
इति~। घातदाढर्यं विनापि लघ्वक्षरपवनवशादेव यथा मधुरशब्दादयो विचित्रा भवन्ति य (त) था कुर्यादिति~॥~२७१~॥

\newpage
% चतुस्त्रिंशोऽध्यायः ४६५

\begin{quote}
{\na  नवे मृदङ्गे दातव्यं रोहणं सततं बुधैः~।\\
 गव्यं घृतं च तैलं च तिलपिष्टं तथैव च~॥~२७२~॥

 \renewcommand{\thefootnote}{1}\footnote{म. बध्वा~।}तथा ह्येतेन विधिना त्वाङ्गिकालिङ्गकोर्ध्वकान्~।\\
 देवताभ्यर्चनं कृत्वा ततः स्थाप्या महीतले~॥~२७३~॥

 चित्रायामथवा हस्ते शुक्लपक्षे शुभेऽहनि~।\\
 उपाध्यायः शुचिर्विद्वान् कुलीनो रोगवर्जितः~॥~२७४~॥

 मतिमान् गीतितत्त्वज्ञो मधुरोऽविकलेन्द्रियः~।\\
 सोपवासोऽल्पकेशश्च शुक्लवासा दृढव्रतः~॥~२७५~॥

 मण्डलत्रयमालिप्य गोमयेन सुगन्धिना~।\\
 ब्रह्माणं शङ्करं विष्णुं त्रिषु तेषु प्रकल्पयेत्~॥~२७६~॥

 \renewcommand{\thefootnote}{2}\footnote{N लिङ्गमास्थापयेत्पूर्वं ब्रह्मणो मण्डले कृते~। (V. २१९ ab \textendash\ N)}आलिङ्गं स्थापयेत् पूर्वं कृते ब्राह्मेऽथ मण्डले~।\\
 ऊर्ध्वकं तु द्वितीयेऽस्मिन् रुद्रनाम्नि निधापयेत्~॥~२७७~॥

 तिर्यगुत्सङ्गिकं सम्यग् वैष्णवे मण्डले क्षिपेत्~।\\
 बलिपुष्पोपहारैस्तु पूजयेत् पुष्कत्रयम्~॥~२७८~॥

 पायसं घृतमध्वक्तं चन्दनं कुसुमानि च~।\\
 शुक़्लानि चैव वासांसि दत्त्वा लिङ्गे स्वयंभुवः~॥~२७९~॥

 त्र्यम्बकाय प्रदातव्यः सगणायोर्ध्वके बलिः~।\\
 स्वस्तिकैर्लाजिकापुष्परूपपिण्डाष्टकैः सह~॥~२८०~॥\renewcommand{\thefootnote}{3}\footnote{N:त्र्यम्बकस्य च दातव्यं  सगणस्यार्धके बलिः~। स्वस्तिकोल्लापिकी पूप (रूप ?) भाषपिष्टतिलैः सह~॥~ (V. २२२ \textendash\ N).}}
\end{quote}
 
\hrule
 
\vspace{2mm}
दातव्यं रोहणमिति~। येन चर्मणा (णो) वाद्यं प्ररोहति तद् द्रव्यम्~। घृत (तं) तिलतैलं तिलकल्कं चेति~। चर्मणि लेपः~॥~२७२ \textendash\ २७३~।\\

अथ मुरजपूजां यत्नेन कर्तव्य (व्या) माह~। चित्रायामित्यादि~। उपाध्याया (य) इति~। भौरजिकाः \textendash\  (कः) ~॥~२७४ \textendash\ २७६~॥\\

आलिङ्ग (ङ्ग्य) म्~॥~२७७ \textendash\ २७८~॥\\

स्वयम्भुवः ब्रह्मणा (णः)~॥~२७९ \textendash\ २८०~॥

\newpage
% ४६६  नाटयशास्त्रे

\begin{quote}
{\na  उन्मत्तकरवीरार्कपुष्पैरन्यैश्च भूषितः~।\\
 बलिः कार्यः प्रयत्नेन रक्तकौदुम्बरैः सह~॥~२८१~॥

 वैष्णवे मण्डले स्थाप्यः सर्वबीजगतोऽङ्किकः~।\\
 स्रग्वस्त्रालेपनैः \renewcommand{\thefootnote}{1}\footnote{N: प्रीतै not read in N:}प्रीतैश्चरुभिश्च सपायसैः\renewcommand{\thefootnote}{2}\footnote{N:सहासवै}~॥~२८२~॥

 वाचयित्वा द्विजैः स्वस्तिं दत्त्वा पूर्वं च दक्षिणाम्~।\\
 पूजयित्वादिन्धर्वान् पश्चाद्वाद्यं समाचरेत्~।२८३~॥\renewcommand{\thefootnote}{3}\footnote{after V. 283 which is V.225, N reads तिथिभिः पणवानदृधाः अवकृष्टा समासतः~। वातपुष्करिकाश्चैव योजयेत्तनुं चर्मणा~। एवं मृदङ्गपणवाः कर्तव्याः दर्दूरास्तथा~।  (२२७ \textendash\  abN) ; V२८४ ab (G.O.S.) V२२७ cd N}

 दैवतानि च वक्ष्यामि येषां ते च भवन्ति हि~।\\
 वज्रेक्षणः शङ्कुकर्णो ग्रहश्चापि\renewcommand{\thefootnote}{4}\footnote{N: ग्रामणीश्च,} तथा महान्~।२८४~॥

 एतास्तु देवता विप्राः पुष्करेषु प्रकीर्तिताः~।\\
 मृण्मयत्वान् मृदङ्गा (ङ्ग) स्तु भाण्डं भ्रमयतीति च~॥~२८५~॥

 मुरजास्तूर्ध्वकरणादातोद्यं तोदनादपि\renewcommand{\thefootnote}{5}\footnote{ताडनात्तथा  (V. २२९ cd. N)}~।\\
 भाण्डस्यादौ प्रणीतोऽत्र पणवश्च विधीयते~।\\
 \renewcommand{\thefootnote}{6}\footnote{N: दारयन्ति च भाण्डं हि यतः स दर्दरः स्मृतः~। (V. २३० cd. N)}दारं शब्दं दारयति तस्माद् भवति दर्दरः~॥~२८६~॥}
\end{quote}

\hrule

\vspace{2mm}
उन्मत्तं धत्तूरम्~।९८१ \textendash\ २८३~॥\\

वज्रेक्षणादयोऽग्रतः पूज्याः~॥~२८४~॥\\

{\qtt अय निर्वचनमाह}~। मृण्मयत्वान् मृदङ्ग इति~। मुखे चेति सुखं मङ्गलं च करोति~। {\qtt भाण्डम्}~। भस्य डा आदेशः~। तदाह भ्रमयतीति~। रसयतीत्येन्ये पठन्ति~। एके भणत इति च भाण्डमाहुः~।२८५~॥\\

मुरजाः सर्वकरणादिषु मुरा ( द् वेष्टनाज्) जाता मुरजाः~। ऊर्ध्वकरणमुन्नतं सौषिर्यं च~। अन्ये तु मुजि (पा.धा. २५१) ज्जिनति विक्रमः ष (मख) रूपमूर्ध्वं समनुकुर्वन्तीति प्रातिपदिकाद् धात्वर्थै  (पा.धा.१८१६) णिचि टिलोपे पचाद्यच् (अ.३०१.१३४) मुरजा इत्याहुः~। आतोद्यं तोदनात्~। भाण्डस्यादौ प्रणीतः पणव इति~। तैनैव पूर्वव्यवहाराद् भ (प) ण व्यवहारे (पा.धा.४३९) द्रव्यनिचयः~। दर्दरः~। तदाश्रयत्वाद् वाद्यमपि~। नृणामन्ते (दृ विदारणे) क्यचि रूपमित्युक्तमेव (३४ \textendash\ ८४)~॥~२८६~॥

\newpage
% चतुस्त्रिंशोऽध्यायः ४६७

\begin{quote}
{\na सृष्ट्वा मृदङ्गान् पणवं दर्दरं च महामुनिः~।\\
मेघैश्च स्वरसंयोगं\renewcommand{\thefootnote}{1}\footnote{म. N: वादं~।} मृदङ्गानामथासृजत्~॥~२८७~॥

विद्युजिह्वो भवेद्वामे मेघः स तु महास्वनः~।\\
ऐरावतो महामेघस्तथा चैवोर्ध्वके भवेत्~।२८८~॥

\renewcommand{\thefootnote}{1a}\footnote{N: आलिङ्गे चैव हृस्वांश्च नाम्ना मेघो बलाहकः}आलिङ्गके तडिद्मांश्च नाम्ना चैव बलाहकः~।\\
दक्षिणे पुष्करो मेघः पुष्करो वामयोजितः~।\renewcommand{\thefootnote}{2}\footnote{म. कोकिला नाम विश्रुतः} (कोकिलो नाम विश्रुतः)~॥~२८९~॥

मृदङ्गश्रैव नाम्ना तु ऊर्ध्वके नन्द्य श्थोच्यते~।\renewcommand{\thefootnote}{3}\footnote{N: नान्यथोच्यते}\\
\renewcommand{\thefootnote}{4}\footnote{N; आङ्किस्त्वसितः}अङ्किकः सिद्धिरित्येवमालिङ्गश्रैव पिङ्गलः~॥~२९०~॥

\renewcommand{\thefootnote}{5}\footnote{N: भूतावेद्यो बलिस्तस्य}भूतप्रियो बलिस्तेभ्यो दातव्यः सिद्धिमिच्छता~।\\
अपूजयित्वा ह्येतान् वै नैव प्रेक्षां प्रयोजयेत्~।२९१~॥

करीषस्य तु सङ्काते मृदङ्गं स्थापयेद् बुधः~।\\
\renewcommand{\thefootnote}{6}\footnote{(N) ; सुसंघातानकयोश्च तथा विचलितां पुनः}सुसम्या (ङ्वा) तानचलिता (तां) स्तथा च (चा) पतिता (न्) पुनः~॥~२९२~॥

आतोद्यपणवैश्चैव नर्तकोपनतास्तथा~।\\
शान्तिकर्म प्रयुञ्जीत विधिदृष्टेन कर्मणा~॥~२९३~॥}
\end{quote}

\hrule
 
\vspace{2mm}
अथ मेघानां तत्राधिष्ठातृत्वं पुराकल्पमाह~। मेघैश्च स्वरसंयोग इ (मि) ति~। मेघगर्जितेन हेतुना ततःस्वरयोजनामाश्रित्य~। वर्णाः षोडशेति भावः~॥~२८७~॥\\

{\qtt तेनोर्ध्वके} चैरावतः~। आलिङ्गेन वामेन निनादः~॥~२८८~॥\\

दक्षिणे त्वङ्किके कोकिलः~॥~२८९~॥\\

ऊर्ध्वके नन्दी~। {\qtt आङ्किके} सिद्धिः~। आलिङ्गे पिङ्गलः~॥~२९०~॥\\

इति परमेश्वरगणविशेषाधिष्ठानद्वारेण सूचयन् तेषामप्यत्र पूज्यत्वमाह~। भूतान्यन्ताद् वेदनीयानि {\qtt तर्पणमनुभावनीयानि}~। येन तामृब्दली (तेषां प्रियो बलिः)~॥~२९१~॥\\

करीषस्य शुष्कगोमयद्वा (रा) शिशेषमध्ये~। यत्रैषां चर्मबद्धविकृतिर्न भवति~। 

\newpage
%४६८ नाट्यशास्त्रे

\begin{quote}
{\na  चत्वारः पणवाः कार्याः दशरूपविधौ पुनः~।\\
 आतोद्यान्यपि तान्येव नानावस्थासु वादयेत्~।२९४~॥

 \renewcommand{\thefootnote}{1a}\footnote{म N वीथ्यां भाणे डिमे~।}नाटके सप्रकरणे \renewcommand{\thefootnote}{1}\footnote{N. नाटकेऽङ्कप्रकरणे}भाणके (णे) प्रहसने तथा~।\\
 \renewcommand{\thefootnote}{2}\footnote{N. मृदङ्गपणवाश्चापि दर्दरं चापि वादयेत्~। (V २३९ cd \textendash\  N)}मृदङ्गं पणवं चैव दर्दरं चैव वादयेत्~।\\
 एवमेतद् बुधैर्ज्ञेयं मृदङ्गानां तु लक्षणम्~॥~२९५~॥\renewcommand{\thefootnote}{3}\footnote{N. after एवमेतत् .\ldots लक्षणम्~। (N) reads :अत ऊर्ध्वं प्रवक्ष्यामि वादकानां तु लक्षणम्~।}

 गतिवाद्यतालपाठ्यग्रहमोक्षविशारदोऽथ लघुहस्तः~।\\
 \renewcommand{\thefootnote}{4}\footnote{N.छिद्राचरणविधिज्ञः सिद्धिस्थाने (V २४१ cd N)}छिद्राच्छेदविधिज्ञः सिद्धस्थाने ध्रुवाकुशलः~॥~२९६~॥

 कलरिभितमधुरहस्तः सुनिविष्टो रक्तमार्जनो बलवान्~।\\
 \renewcommand{\thefootnote}{5}\footnote{N अवहितमतिर्विधिज्ञः संसिद्धो वादकः श्रेष्ठः~॥~ (२४२  \textendash\ N \textendash\ cd)}अवहितशरीरबुद्धिर्मृदङ्गवादी गुणैरेतैः~॥~२९७~॥

 आलेपनप्रमाणज्ञश्चतुर्मार्गकृतश्रमः~।\\
 प्रतिग्राही च (ग्रहीता) सिद्धीनामङ्गदोषविवर्जितः~॥~२९८~॥

 स्वभ्यस्तकरणः सामे गीतब्ञो गुणितग्रहः~।\\
 सुसङ्गीतप्रयोगज्ञो मृदङ्गी तु भवेद् गुणैः~॥~२९९~॥

 भ्रान्तोर्ध्वहस्तः कालज्ञश्छिद्रावरणपण्डितः~।\\
 अभ्यस्तकरणश्चैव भवेत् पाणविको गुणैः~॥~३००~॥

 निश्चलो निपुणः शीघ्रो लघुहस्ती विधानवित्~।\\
 वादनान्तरवादी च दर्दरी तु प्रशस्यते~॥~३०१~॥}
\end{quote}

\hrule
 
\vspace{2mm}
सङ्कातानन्योन्यविश्लिष्टान्~।२९२ \textendash\ २९६~॥\\

कलश्चतुरो रिभितो दीप्तो मधुरः श्रुतिसुखदो हस्तो यस्येति~। सुश्लिष्टो दृ (ह) ष्टो दृष्टा (ढो) सहः~॥~२९७~॥\\

प्रतिगृ (ग्र) हीता सिद्धीनामिति~॥~२९८~॥\\

{\qtt न्तरवआद्यज्ञः}~। गुणिता अभ्यस्ता ग्रहा ग्रहमोक्षसन्धयो (ये) न~। सुष्ठु सङ्गीतप्रयोगमेलनिकां जानातीति तथोक्तम्~॥~२९९~॥\\

भ्रान्तेत्यत्र भ्रमणक्रियाचतुर ऊर्ध्वाधः स्थिरश्च हस्तो यस्य~॥~३००~॥

\newpage
% चतुस्त्रिशेऽध्याय ४६९

\begin{quote}
{\na  एतत् सर्वं नाट्ययोगं समीक्ष्य\\
 प्रोक्तं वाद्यं सर्वलोकानुभावात्~।\\
 नोक्तं यच्चेदागमाद्वस्तुबुद्धया\\
 सद्भिः कार्यं मार्गजातीः समीक्ष्य~॥~३०२~॥

 स्फुटप्रहारं विशदं विभक्तं\\
 रक्तं विकृष्टं करलेपनं च~।\\
 त्रिमार्जनापूरितरागगम्यं\\
 मृदङ्गवाद्यं गुणतो वदन्ति~॥~३०३~॥

 वाद्ये\renewcommand{\thefootnote}{1}\footnote{N : वाद्येषु} तु यत्नः प्रथमं तु कार्यः~।\renewcommand{\thefootnote}{2}\footnote{N: कार्यो}\\
 \renewcommand{\thefootnote}{3}\footnote{N: गतिं तु}शय्या हि नाट्यस्य वदन्ति वाद्यम्~।\\
 वाद्ये च गीते च हि सु (सं) प्रयुक्ते (क्तो)\\
 नाट्यप्रयोगे (गो) न विपत्तिमेति~॥~३०४~॥}
\end{quote}

\begin{center}
\textbf{ \renewcommand{\thefootnote}{4}\footnote{N: इति भारतीये द्वात्रिंशः}इति भारतीये नाट्यशास्त्रे पुष्करवाद्यो नामाध्यायश्चतुस्त्रिंशः~॥}
\end{center}

\hrule

\vspace{2mm}
वादनस्य मुरजवाद्य (द्या) न्तरं मध्यान्तरं चेति विशेषश्च (३०१ नोक्तं यच्चेदितीति~। हुडुक्वा विसिसंक्षानशङ्कामात्रमेतत्~। वस्तुतस्तु न सर्वमुक्तमिति दर्शयति~॥~३०२~।\\

{\qtt विकृष्टमिति}~। आराच्छ्वणयोग्यम्~॥~३०३~॥\\

नाट्यस्य बदन्ति वाद्यमिति~। पञ्चधेति साम्यात् प्रतिष्ठितं भवति~। तदाह~। गीते~। वाचाऽपि तु सम्प्रयुक्तो नाट्यप्रयोगो न विपत्तिमेति~। गीतातोद्याभ्यां रामरावणादिविशेषास्पर्शनेनैव रसभावचर्वणौचित्यमाधीयते साम्यं च यतो भवन्ति सर्वाः सिद्धय इति शिवम्~॥~३०४~॥

\begin{quote}
{\qt  नादान्तसादाशिवभक्तिधाम\\
 सन्तानकृत्यः कखिलाभिधानः~।\\
 रात्रिन्दिवं तस्य सुतेन दृष्टा\\
 सुपुष्कराध्यायसमाप्तिवृत्तिः~॥}
\end{quote}

\begin{center}
इति श्रीमहामाहेश्वराभि (न) वगुप्ताचार्यविरचितायां नाट्यवेदवृत्तावभिनवभारत्यां पुष्कराध्यायश्चतुस्त्रिंशः~। 
\end{center}

\newpage
%४७०  नाटयशास्त्रे

\begin{center}
\textbf{\Large ॥~श्रीः~॥}\\

\vspace{2mm}
\textbf{\Large अथ चतुस्त्रिंशोऽध्यायः~।}\\

\vspace{2mm}
\textbf{\Large भिन्नपाठक्रमः~।}
\end{center}

\begin{quote}
{\na  ततातोद्यविधिस्त्वेष मया प्रोक्तः समासतः~।\\
 अवनद्धविभागेन लक्षणं कर्म चैव हि~॥~१~॥

 आतोद्यानां प्रवक्ष्यामि विर्धि वादनमेव च~।\\
 मृदङ्गपणवानां च दर्दरस्य तथैव च~॥~२~॥

 गान्धर्वं चैव वाद्यं च स्वातिना नारदेन च~।\\
 विस्तारगुणसम्पन्नमुक्तं लक्षणकर्मतः~॥~३~॥

 अनुवृत्त्या तयोः स्वातेरातोद्यानां समासतः~।\\
 पौष्कराणां प्रवक्ष्यामि निवृत्तिं सम्भवं तथा~॥~४~॥

 अनध्याये कदाचित्तु स्वातिर्वै दुर्दिने दिने~।\\
 जलाशयं जगामाथ सलिलानयनं प्रति~॥~५~॥

 तस्मिन् सरोनिषण्णे\renewcommand{\thefootnote}{1}\footnote{मुनौ सरस्थे} च प्रविष्टः\renewcommand{\thefootnote}{2}\footnote{प्रहृष्टः} पाकशासनः~।\\
 धाराभिर्महतीभिश्च कर्तुमेकार्णवं जगत्~॥~६~॥

 पतन्तीभिश्च धाराभिर्वायुवेगाज्जलाशये~।\\
 पुष्करी (रा) णां \renewcommand{\thefootnote}{3}\footnote{पटुः शब्दः}कलरवः पत्राणामभवत्तदा~॥~५~॥

 तेषां \renewcommand{\thefootnote}{4}\footnote{धीरं~।}धीरकलं शब्दं निशम्य स\renewcommand{\thefootnote}{5}\footnote{सहसा~।} महामुनिः~।\\
 आश्चर्यमिति सम्प्राप्तमवधारितवान् स्वयम्~॥~८~॥}
\end{quote}

{\qt केचिदेकवक्त्रजं केचित् त्रिवक्त्त्रजं केचिन्निर्वकत्रजमिति~। यथा रेफः सर्ववक्त्रेषु~। दका (र) धकारावालिङ्ग्यवामके~। आलिङ्ग्य दक्षिणे मकारः~। वामोध्ववो (र्ध्वोवा) गकारः~। क्वचिदालिङ्गेऽपि लाघवार्थं धकारः कर्तव्यः~। एवं तत्र प्रतिषेधो न कार्य इति नु (नो]तुं}

\newpage
\fancyhead[LO]{चतुस्त्रिशोऽवध्याय}
% चतुस्त्रिशोऽवध्याय ४७१

\noindent
{\qt स्वरव्यञ्जनसंयोगः\renewcommand{\thefootnote}{1}\footnote{संयोगजः} पञ्चपाणिः \renewcommand{\thefootnote}{2}\footnote{प्रहतानां}प्रगतानां समपाणिरर्धपाणि\renewcommand{\thefootnote}{3}\footnote{अर्धार्ध}र्धार्धपाणिः प्रदेशिनी चेति~। एते नीच (नि) गृहीतार्धगृहीतमुक्ता भवन्ति~। तत्र समपाणिहतो मकारः स निगृहीतः~। दकार\renewcommand{\thefootnote}{4}\footnote{धकारा}पकारा अक्षता \renewcommand{\thefootnote}{5}\footnote{अक्षरार्ध}र्धगृहीताः~। अर्धपाणिप्रहताश्च \renewcommand{\thefootnote}{6}\footnote{ककार}कार (क) कारखकारगकारडकाराः~। पार्श्वपाणिप्रहता निगृहीताश्च तकारथकारहकाराः~। अर्धार्धपाणिप्रहताश्च प्रदेशिन्या हताश्चालिङ्गे \renewcommand{\thefootnote}{7}\footnote{डकार}मकारडकारणकाराः~। मुक्ताश्च विहस्तप्रहस्तप्रहता अपि \renewcommand{\thefootnote}{8}\footnote{श्रङ्क्लेड्ति}द्रङ्ब्रङ्क्लेङिति~। मुक्ताश्च \renewcommand{\thefootnote}{9}\footnote{क्रुत्सर्ध}क्रुद्ध (द्धा) इत्यर्धपाणि\renewcommand{\thefootnote}{10}\footnote{प्रहता निगॄ \textendash\ } प्र (ह) तोऽ (ता अ) र्धनिगृहीताश्च~। एवमेतेष्वक्षरेषुप्रयोगवशेन कार्य प्रहतम्~॥~९~॥~कुतः~।}

\begin{quote}
{\na  षोडशैव तु दृष्टानि वाद्यजान्यक्षराणि तु~।\\
 अनेनैव तु योगेन योज्यं वा (क्) करणं बुधैः~॥~१०~॥

 चतुर्मार्गमिति यदुक्तं तदनुव्याख्यास्यामः~।\\
 अड्डितालिप्तमार्गौ तु वितस्तो गोमुखस्तथा~।\\
 मार्गाश्चत्वार एवैते प्रहारकरणाश्रयाः~॥~११~॥

 तत्रालिङ्गमृदङ्गप्रहारयुक्तस्त्वड्डितो मार्गः~।\\
 वामोर्ध्वगप्रहारयुक्तस्त्वालिप्तमार्गः~॥~१२~॥

 ऊर्ध्वाङ्गदक्षिणोत्क्षिप्तप्रहस्तो\renewcommand{\thefootnote}{11}\footnote{प्रहतो} वितस्तमार्गः~।\\
 आलिङ्गकरणबहुलः सर्वपुष्करप्रहतो गोमुखमार्गः~॥~१३~॥}
\end{quote}

\noindent
{\qt इति~। तत्राडित\renewcommand{\thefootnote}{12}\footnote{प्रकारजो~।} प्रहारजो यथा~। मट्टंकत्थितघटधघेण्डाघट्टंगत्थि धुंगमगत्थिमगघेण्डां सन्धित्थ इत्यड्डितो मार्गः~।}\\

{\qt इदानीमालिप्तमार्गः\renewcommand{\thefootnote}{13}\footnote{मार्गः दघ्रो}~। आलिङ्गकरणबहुलः सर्वादध्रो मामादध्रो मांगुदु्ेगुदुरनुदुर्घ्रेन्द्र \renewcommand{\thefootnote}{14}\footnote{घेन्द्रोमां}धेन्द्रामाङ्~।}

\newpage
%४७२  नाटयशास्त्रे

{\qt आलिप्तमार्गवाद्यं विज्ञेयं वादकैरेवम्~।}\\

{\qt इदार्नी वितस्तमार्गः~। तकितां तकिता \renewcommand{\thefootnote}{1}\footnote{झेन्तां}क्लेन्नां किन्तान्ध्रि  संकेता उदुहुदुकेन्तां वितस्ते  स्यात्~। अङ्गुलिप्रचालनावर्तितो\renewcommand{\thefootnote}{2}\footnote{ना्वर्तितो} नकारो मुक्तश्च स एव स्वस्तिकोनो ध्वां \renewcommand{\thefootnote}{3}\footnote{ध्वां क}कुकयोरर्धनिगृहीतः~। तयोरेव महस्तदर्शनात् प्रचालनात् \renewcommand{\thefootnote}{4}\footnote{हाकारः}हकारो मुक्त इति~। \renewcommand{\thefootnote}{5}\footnote{धित्थः}धत्थतित्थडित्थडितकिता खदेड् खदेड् गुरुखेन्दु खेण्डझत्थित्थि\renewcommand{\thefootnote}{6}\footnote{ति} त्थन्धित्थां\renewcommand{\thefootnote}{7}\footnote{क्तां} विधातव्यः~। एवं वितस्तमा\renewcommand{\thefootnote}{8}\footnote{मार्गे} र्गगणे \renewcommand{\thefootnote}{9}\footnote{लमर}लमकरवर्जिताः प्रहाराः स्युः~। ऊर्ध्वमार्गेण विशेषो\renewcommand{\thefootnote}{10}\footnote{विशेषा} कार्या (र्य) श्च गोमुख्याः~। \renewcommand{\thefootnote}{11}\footnote{खट}खगमत्थिमटटधेण्डाखुखुणं \renewcommand{\thefootnote}{12}\footnote{घर}खरखेण्डां खेदुङ् खुणुमां दुणकितिकिन्तिकिट्टिमां \renewcommand{\thefootnote}{13}\footnote{टिमां}खुखुणुद्वेधे\renewcommand{\thefootnote}{14}\footnote{द्वेधेधोधोवि} धोधा विधातव्या~।}\\

{\qt इदानीं गोमुखीवाद्यम्~। डेणघेण्डणखुखुन्धु \renewcommand{\thefootnote}{15}\footnote{न्दु}लघुदुलेड्\renewcommand{\thefootnote}{16}\footnote{खेड्} घटमटटाङ् दुणुघुणांत्थि घटंघिटिमां कुकुटाङुङुणुघेङ्किदिमाङ् धे धे खो खो च गोमुख्याः~॥~१४~॥}

\begin{quote}
{\na  ये त्वालिप्तसमुत्थाः सर्वैर्मार्गैस्तु ते विधातव्याः~।\\
 ग्रहणाक्षरसन्धानेऽ \renewcommand{\thefootnote}{17}\footnote{ह्या~।}प्याद (दि) मृदङ्गाद् ग्रहो भवति~॥~१५~॥

 एतेषां तु प्रवक्ष्यामि दर्शनानि यथाक्रमम्~।\\
 चतुर्णामपि मार्गाणामक्षरग्रहणं यथा~॥~१६~॥}
\end{quote}

{\qt धृङ् \renewcommand{\thefootnote}{18}\footnote{धृङ्}वृङ् घट्टघङ् \renewcommand{\thefootnote}{19}\footnote{मत्थि}मत्थिटत्थिमटामटत्थिदिङ्मनधेङ् कृङ्क्लेड्\renewcommand{\thefootnote}{20}\footnote{पिकटाङ्} क्विकाङ् त्वडिते वाद्यम्~।}\\

{\qt मद्रा तकिता धृङ् \renewcommand{\thefootnote}{21}\footnote{घृङ्}वृङ् \renewcommand{\thefootnote}{22}\footnote{ध्रो}म्रो किटि घेडाङ् गंत्थि\renewcommand{\thefootnote}{23}\footnote{गधि} किटिकेत्था \renewcommand{\thefootnote}{24}\footnote{घकुता}यकुताकितां किरटामिति संयोगो वितस्तेस्तु \renewcommand{\thefootnote}{25}\footnote{घ्रो~।}स्रोमाङ् गुदुघेङ् घेघेघेण्डां घेघेटद्वमां~।}

\newpage
% चतुस्त्रिशोऽध्याय ४७३ 

{\qt आलिप्ते संयोगे कार्योऽयं वादने तज्ज्ञैः~॥}\\

{\qt धैघेटत्थिकटांगुदुधेङ् घेटां घेण्डां \renewcommand{\thefootnote}{1}\footnote{धिमधित्थि}धिमितित्थिधंकेसरेधेगधेणोणोणमिति वैष गोमुख्याः~। उक्ता मार्गाः~।}\\

{\qt एतेषां पुष्कराणां त्रिविधः प्रचारः~। समो विषमः समविषम इति~। वामोर्ध्वगयोर्वामसव्ययोरड्डितामार्गेऽपि कार्यः~। समप्रचारस्त्वालिप्ते वाद्यकरणे तु वाध्वथ्\renewcommand{\thefootnote}{2}\footnote{ग} (ग) सव्यानां ग्रहणे\renewcommand{\thefootnote}{3}\footnote{तो} वामस्तु कर्तव्यः~। सव्यो\renewcommand{\thefootnote}{4}\footnote{र्ध्व} ध्वगमनानां\renewcommand{\thefootnote}{5}\footnote{नात्} सव्यो विषमप्रचारे तु~॥~१७~॥}

\begin{quote}
{\na  एवं स्वस्तिकयोगाद् द्वाभ्यामपि यद् भवेत्तु हस्ताभ्याम्~।\\
 माने वितस्तिसंज्ञे प्रहते विषमः प्रचारः सः~॥~१८~॥

 स्वच्छन्दकस्तु करयोः शेषः प्रहतेषु मार्गकरणेषु~।\\
 अड्डितगोमुखयोगात् समविषमो हस्तसञ्चारः~॥~१९~॥

 शृङ्गारहास्ययोर्वाद्य\renewcommand{\thefootnote}{6}\footnote{योगे वाद्यं} योज्यं तथाड्डिते मार्गे~।\\
 वीराद्भुतरौद्राणां वितस्तमार्गेण वाद्यं स्यात्~॥~२०~॥

 करुणरसेऽपि च कार्यं वाद्यं चालिप्तकरणमार्गेण~।\\
 बीभत्सभयानकयोस्तथैव नित्यं हि गोमुख्याः~॥~२१~॥

 \renewcommand{\thefootnote}{7}\footnote{ससत्व}रसत्वभावयोगं दृष्ट्वाऽभिनयं गतिप्रचारं च~।\\
 वाद्यं नित्यं कार्य यथाक्रमं स्थानसंयोगात्~॥~२२~॥

 एतत् प्रहतविधानं कार्यं मार्गाश्रिते\renewcommand{\thefootnote}{8}\footnote{श्रितै} रसैः सम्यक्~।\\
 वक्ष्याम्यतश्च भूयो दर्दुरपणवाश्रितं वाद्यम्~।२३~॥

 तत्रातिवादितं स्यात् मुखानामग्रतो यत्र~।\\
 यत्त्वनुगतं मृदङ्गत (ङ्गं) \renewcommand{\thefootnote}{9}\footnote{मृदङ्गानु}मृदङ्गादनुवादितमुच्यते तु तद्वाद्यम्~॥~२४~॥}
\end{quote}

\newpage
% ४७४ नाट्यशास्त्रे

\begin{quote}
{\na \renewcommand{\thefootnote}{1}\footnote{समवादितं}समवादितमृदङैर्ज्ञेयं साम्येन यद् वाद्यम्~।\\
 \renewcommand{\thefootnote}{2}\footnote{कखगा}कखगवं पणवमुखां दोह्लां ब्रुहुलाम्\renewcommand{\thefootnote}{3}\footnote{लां कुहुलां ध्राहुलाम्}~।\\
 एते वर्णा नित्यं पणववाद्ये प्रयोक्तव्याः~॥~२५~॥}
\end{quote}

\noindent
{\qt किरिखिण्डां \renewcommand{\thefootnote}{4}\footnote{थोथोणो}थोणोथोत्रहुला \renewcommand{\thefootnote}{5}\footnote{लां~।}किरिकिण्डां णोणोणाणाश्च किरिकण्डां \renewcommand{\thefootnote}{6}\footnote{म}माटामटत्थि टे टे दोण्णां पणवाद्यं तस्य प्रहतं कार्यं कनिष्टा नामिकाग्रेणैव\renewcommand{\thefootnote}{7}\footnote{ग्रकेणैव~।}~॥~२६~॥}

\begin{quote}
{\na  प्रशिथिलवश्चितकक्ष्ये पणवे कङ्कार एकप्रकारस्तु\renewcommand{\thefootnote}{8}\footnote{पक्ष्ये}~।\\
 तथै (था) न्ततः प्रहारः पणवे कार्यः प्रशिथिल (लं) वक्ष्ये~॥~२७~॥

 कखररकृतः प्रहारः \renewcommand{\thefootnote}{9}\footnote{स्वो}सोच्छ्वासे तु पणवे विधातव्याः (व्यः)~।\\
 शेषा भ्रान्तकरणसंयोगात् संयोगकृता विधातव्याः~॥~२८~॥

 धोकारस्तु खारजे पणवे कक्ष्याश्चिते तु संभवति~।\\
 णैणकरो\renewcommand{\thefootnote}{10}\footnote{कारो} युक्तो क्व इति पणवप्रकारः स्यात्~॥~२९~॥

 तिर्यग्गृहीतवादननखवर्तितमिष्यते प्रहारोऽयम्~।\\
 कुहुलां कुहुलां \renewcommand{\thefootnote}{11}\footnote{कमक्रिमिकु}क्रेमससिन्धिः कुहुलामितीत्थं प्रकारः स्यात्~॥~३०~॥

 एवं च सङ्करकृताः प्रहाराः शुद्धाः प्रकीर्तितास्तज्ज्ञैः~।\\
 दर्दुरपणवमृद्रैर्मिश्रितवाद्यं प्रवक्ष्यामि~॥~३१~॥

 तत्र वर्गिकाभावं केचिदिच्छन्ति मिश्रितं केचित्~।\\
 केचिद् युगपत्करणं केचित् पर्यायकरणं तु~॥~३२~॥

 \renewcommand{\thefootnote}{12}\footnote{एकेकं}एकैकसम्प्रयुक्ता वर्गानुगतास्तथैव सम्प्रयोक्तव्याः~।\\
 घेता कथं त्रैवो\renewcommand{\thefootnote}{13}\footnote{रवो} कुहुलां तकिता मृदङ्गेषु~॥~३३~॥

 अथ दररे \renewcommand{\thefootnote}{14}\footnote{दङ्शेड्द्रे}दङ् द्रेरे्ङ् कुहुलां मटंथि पणवे च~।\\
 थमंटं \renewcommand{\thefootnote}{15}\footnote{टात्थेझ्ं~।}हिझं कुहुलां मटत्थि देड् वेङ् विमिश्रितस्तु~॥~३४~॥}
\end{quote}

\newpage
% चतुस्त्रिंशिऽध्याय ४७५

\begin{quote}
{\na एभ्यो येऽन्ये शेषा न मिश्रा एवं नित्यशः कार्याः~।\\
पूर्वेऽपि च मिश्रत्वं व्रजन्ति सर्वे यथायोगम्~॥~३५~॥

अथ युगपत् करणं कुहुलाणणां\renewcommand{\thefootnote}{1}\footnote{ण्ण ण्णां} खुखुणोखेद्रोमादोण्णा\\
\renewcommand{\thefootnote}{2}\footnote{थेधोधोत्थि}थथोधौत्थि द्वे योज्यं पणववाद्यम् \renewcommand{\thefootnote}{3}\footnote{धु}ध (धु) र्यकृतं च\\
क्रमशः करणं परिवादने के (दके) \renewcommand{\thefootnote}{4}\footnote{दकेन}न कर्तव्यम्~॥~३६~॥

\renewcommand{\thefootnote}{5}\footnote{टरटे धोणणकि}भटभेभेधोधोणाणकिरिणि किण्णाकृतं चेति~।\\
पणवस्याचारबन्धे कार्यं धाखुखुणेति वाद्यत्रयम्~॥~३७~॥

भूयः कृतं च प्रतिकृत (तं) मार्दङ्गिकदार्दुरिकाभ्यां तु~।\\
यद्यत् कुर्यात् समुखं प्रहारजातं गतिप्रचारेषु~॥~३८~॥

अनुगच्छतमक्षरवृत्तं तदेव वाद्यं तु पणवेन~।\\
न हि चित्रं कर्तव्यं गतिप्रचारेषु \renewcommand{\thefootnote}{6}\footnote{वादहमतं तज्ञैः}वाद्य (दन) मनन्तज्ञैः~॥~३९~॥

समसहिं तत्र हि तन्न लक्ष्यते पादसश्चारे~।\\
उपरिकरणं \renewcommand{\thefootnote}{7}\footnote{करं~।}यथेष्टं कर्तव्यं पणवे मृदङ्गे च~॥~४०~॥

तद्वत् \renewcommand{\thefootnote}{8}\footnote{प्रकार}प्रहारकरणं मृदङ्गवाद्ये विधातव्यम्~।\\
प्रायेण सर्ववाद्येष्वादौ पणवग्रहाः प्रयोक्तव्याः~॥~४१~॥

वक्ष्याम्यतः परमहं दसृक्\renewcommand{\thefootnote}{9}\footnote{जे} घेत्ततिता \renewcommand{\thefootnote}{10}\footnote{एछदेड्झेदे}केसदेङ्~।\\
केसदेवेदेरेभिण्णत\renewcommand{\thefootnote}{11}\footnote{तति}मिततिमत्थि~॥~४२~॥

इति दर्दुरे प्रहरः कार्यो मुक्ते निषण्णे च~।\\
कार्यस्तत्र णणण्णारेद्रेग्रेधिकित्थि\renewcommand{\thefootnote}{12}\footnote{ण्थीति दक्षिणकरेण}दक्किक्किकरणे~॥~४३~॥

वामेन तु \renewcommand{\thefootnote}{13}\footnote{क्तिय क्लंतत्त्व पृष्टौक}झित्थिकसं (लं) तत्त्ववृ (ष्ट) ष्टै (:) कराग्रेण~।\\
मुक्ते तमिति चैव निगृह्याशीघ्रकरणानुबन्धास्थितिरेव हितकं स्यात्~॥~४४~॥}
\end{quote}


\end{document}