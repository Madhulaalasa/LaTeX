\documentclass[11pt, openany]{book}
\usepackage[text={4.65in,7.45in}, centering, includefoot]{geometry}
\usepackage[table, x11names]{xcolor}
\usepackage{fontspec,realscripts}
\usepackage{polyglossia}
\setdefaultlanguage{sanskrit}
\setotherlanguage{english}
\setmainfont[Scale=0.9]{Shobhika}
\newfontfamily\s[Script=Devanagari, Scale=0.8]{Shobhika}
\newfontfamily\regular{Linux Libertine O}
\newfontfamily\en[Language=English, Script=Latin]{Linux Libertine O}
\newfontfamily\ha[Script=Devanagari, Scale=0.9, Color=purple]{Shobhika-Bold}
\newfontfamily\haq[Script=Devanagari, Scale=0.9, Color=violet]{Shobhika-Bold}
\newfontfamily\qt[Script=Devanagari, Scale=0.8, Color=violet]{Shobhika-Regular}
\newfontfamily\qtt[Script=Devanagari, Scale=0.9]{Shobhika-Bold}
\newcommand{\devanagarinumeral}[1]{
\devanagaridigits{\number \csname c@#1\endcsname}} % for devanagari page numbers
\XeTeXgenerateactualtext=1 % for searchable pdf
\usepackage{enumerate}
\pagestyle{plain}
\usepackage{fancyhdr}
\pagestyle{fancy}
\renewcommand{\headrulewidth}{0pt}
\usepackage{afterpage}
\usepackage{multirow}
\usepackage{multicol}
\usepackage{wrapfig}
\usepackage{vwcol}
\usepackage{microtype}
\usepackage{amsmath,amsthm, amsfonts,amssymb}
\usepackage{mathtools}% <-- new package for rcases
\usepackage{graphicx}
\usepackage{longtable}
\usepackage{setspace}
\usepackage{footnote}
\usepackage{perpage}
\MakePerPage{footnote}
\usepackage{xspace}
\usepackage{array}
\usepackage{emptypage}
\usepackage{hyperref}% Package for hyperlinks
\hypersetup{colorlinks, citecolor=black, filecolor=black, linkcolor=blue, urlcolor=black}

\begin{document}
\cfoot{}

\begin{center}
\onehalfspacing

THE\\

\vspace{3mm}
\textbf{\huge HARSHACHARITA}\\

\vspace{3mm}
OF\\

\vspace{3mm}
\textbf{\LARGE BÂṆABHATTA}\\

WITH\\

\textbf{\large The Commentary ( Saṅketa )}

OF

\textbf{\large ŚANKARA.}\\

\rule{0.2\linewidth}{0.5pt}\\

\vspace{5mm}
EDITIED BY\\

\textbf{\large KÂŚINÂTH PÂṆḌURANG PARAB.}\\

\rule{0.2\linewidth}{0.5pt}\\

\textbf{Third Edition}\\

\rule{0.2\linewidth}{0.5pt}\\

REVISED BY\\

\textbf{\large SRÎNIVÂS VENKATRÂM TOPPÚR}\\

\rule{0.2\linewidth}{0.5pt}\\

\textbf{PUBLISHED}\\

\vspace{2mm}
BY\\

\vspace{2mm}
{\large TUKÂRÂM JÂVAJÎ}\\

\vspace{2mm}
PROPRIETOR OF JÂVAJÎ DÂDÂJÎS {\qt NIRṆAYA \textendash\ SÂGAR} PRESS,\\

\vspace{2mm}
BOMBAY:

\rule{0.1\linewidth}{0.5pt}\\

1912.\\

\rule{0.1\linewidth}{0.5pt}\\

\emph{\en Price 2 Rupees}
\end{center}

\newpage
\vspace*{\fill}
\onehalfspacing

\begin{center}
\emph{\en ( Registered according to Act XXV of 1867. )}\\

( All rights reserved by the publisher.)\\

\rule{0.4\linewidth}{0.5pt}

\begin{table}[h!]
 \centering
 \begin{tabular}{ccc}
 PUBLISHER: \textendash\ Tukaram Javaji, & \multirow{2}{*}{$\Bigg\}$}& the {\qt Nirnaya \textendash\ sagar} Press,\\
 Printer: \textendash\ B. R. Ghznekar, & & 23, Kolbhat Lane, Bombay.
 \end{tabular}
\end{table}
\end{center}

\vspace*{\fill}
\onehalfspacing

\newpage
\begin{center}
\doublespacing

\textbf{॥~श्रीः~॥}\\

\vspace{3mm}
\textbf{\Large महाकविश्रीबाणभट्टकृतं}\\

\vspace{5mm}
\textbf{\huge हर्षचरितम्~।}\\

\vspace{3mm}
\textbf{महाकविचूडामणिशंकरकविरचितया संकेताख्यया व्याख्यया समेतम्~।}\\

\rule{0.4\linewidth}{0.5pt}\\

\vspace{3mm}
\textbf{\Large काशीनाथ पाण्डुरङ्ग परब}\\

{\large इत्यनेन संशोधितम्~।}\\

\vspace{3mm}
\textbf{\Large श्रीनिवास वेंकट्राम तोप्पूर}\\

{\large इत्यनेन संस्कृतञ्च~।}\\

\vspace{3mm}
\rule{0.2\linewidth}{0.5pt}\\

\textbf{तृतीयं संस्करणम्~।}\\

\rule{0.2\linewidth}{0.5pt}\\

\vspace{3mm}
\textbf{तच्च}\\

शाके १८३४ वत्सरे\\

\vspace{3mm}
\textbf{\large मुम्बय्यां}\\

\vspace{3mm}
निर्णयसागरयन्त्रालयाधिपतिना मुद्रयित्वा प्राकाश्यं नीतम्~।\\

\rule{0.3\linewidth}{0.5pt}\\

\vspace{3mm}
\textbf{\large मूल्यं रूप्यकद्वयम्~।}
\end{center}

\newpage
\begin{center}
\textbf{\large ॥~श्रीः~॥}\\

\textbf{\LARGE हर्षचरितम्~।}\\

{\large शंकरकृतया संकेताख्यया व्याख्यया समेतम्~।}\\

\rule{0.2\linewidth}{0.5pt}\\

{\s प्रथम उच्छ्वासः~।}
\end{center}

\begin{quote}
{\ha नमस्तुङ्गशिरश्चुम्बिचन्द्रचामरचारवे~।\\
त्रैलोक्यनगरारम्भमूलस्तम्भाय शंभवे~॥~१~॥

हरकण्ठग्रहानन्दमीलिताक्षीं नमाम्युमाम्~।\\
कालकूटविषस्पर्शजातमूर्च्छागमामिव~॥~२~॥}
\end{quote}

\hrule

\begin{center}
\textbf{संकेतः~।}
\end{center}

\begin{quote}
{\qt श्चोतन्मदाम्बुभरनिर्भरचण्डगण्डशुण्डाग्रशौण्डपरिमण्डितभूरिभृङ्गान्~।\\ 
विघ्नानिवानवरतं चलगण्डतालैरुत्सारयञ्जयति जातघृणो गणेशः~॥

शंकरनामा कश्चिच्छ्रीमत्पुण्याकरात्मजो व्यलिखत्~।\\
शिष्टोपरोधवशतः संकेतं हर्षचरितस्य~॥}
\end{quote}

\setlength{\parskip}{1em}

{\s {\qt सर्वकर्माणि कुर्वीत प्रणिपत्येष्टदेवताम्} इति शिष्टाचारमनुपालयन् {\qt अपारे काव्यसंसारे कविरेव प्रजापतिः~। यथास्मै रोचते विश्वं तथेदं परिवर्तते~॥} इति काव्यलक्षणामपूर्वा सृष्टिं स्थिरां प्रवर्तयन्नेष कविः शिवं बहुशक्तियुतमपि नियतशक्त्यात्मकमेव स्तौति \textendash\ {\qtt नमस्तुङ्गेत्यादिना}~। न क्वचित्प्रणतो यो मूर्धा तत्स्पर्शी चन्द्र एव सितवालतुल्यप्रभाप्रसरतया स्वेदादिविनाशाद्विशिष्टस्थानस्थितश्च चामरम्~। त्रैलोक्यमेव नानाभङ्गिशोभित्वान्नगरं तदारम्भे मूलस्तम्भः~। नगरारम्भे हि मूलस्तम्भो भवति~। तत्र च पट्टबन्धादिवदुत्प्रेक्षणानन्तरमुन्नते पृष्ठदेशे चन्द्रतुल्यं श्वेतं चामरं क्रियत इति स्थितिः~। केचित्पुनः \textendash\ त्रैलोक्यनगरस्यारम्भे मूलं मूलकारणं परमाणवस्तेषामुपाश्रयेण मूलकारणत्वात्स्तम्भ इव~। ते हि तद्वशात्कार्यमारभन्ते~। तस्य निमित्तकारणत्वादित्याहुः~। {\qt स्वयंभूः शंभुरादित्यः} इति नामसहस्रे दृष्टवाद्धरेः, {\qt शंभूब्रह्मत्रिलोचनौ} इत्यभिधाकोशदर्शनाच्च ब्रह्मणोऽपि नमस्कारोऽयमित्यन्ये वदन्ति~। व्याकुर्वते च हरिपक्षे \textendash\ त्रैलोक्याक्रमणकाले~। यद्वा {\qt यस्याग्निरास्यं द्यौर्मूर्धा स्वं नाभिवरणौ मही} इत्यभिप्रायेण तुङ्गमुच्छ्रितं द्युलक्षणं यच्छिरस्तच्चुम्बि चन्द्र एव चामरं तेन चारवे~। ब्रह्मपक्षे \textendash\ चन्द्रः स्वर्णं तन्मयं चामरमिव चामरे केशकलापः~। हिरण्यकेशो हि ब्रह्मा त्रैलोक्यादीनि सर्वत्र तुल्यमिति~॥~१~॥

{\qtt हरेत्यादिना} प्रियं प्रति गाढस्नेहादि सौकुमार्यं चोमयोच्यते~। कालकूटविषेति प्रशंसार्थः सामान्यवदप्रयोगो मेरुमहीधरचूतवृक्षादिवत्~। आगमः प्रारम्भः~॥~२~॥}

\newpage
\fancyhead[CE]{हर्षचरिते}
\fancyhead[CO]{प्रथम उच्छ्वासः~।}
\fancyhead[LE,RO]{\thepage}
\renewcommand{\thepage}{\devanagarinumeral{page}}
\setcounter{page}{2}
% २ हर्षचरिते

\begin{quote}
{\ha नमः सर्वविदे तस्मै व्यासाय कविवेधसे~।\\
चक्रे पुण्यं सरस्वत्या यो वर्षमिव भारतम्~॥~३~॥

प्रायः कुकवयो लोके रागाधिष्ठितदृष्टयः~।\\
कोकिला इव जायन्ते वाचालाः कामकारिणः~॥~४~॥

सन्ति श्वान इवासंख्या जातिभाजो गृहे गृहे~।}
\end{quote}

\hrule

{\s संप्रत्युत्कृष्टकवित्वाभिमानेन तादृशमेव कविवरं स्तौति \textendash\ {\qtt नमः सर्वेत्यादिना}~। सर्वा वेदादिका विद्या गीतादिकलाश्च वेत्ति यस्तस्मै~। तदुक्तम् \textendash\ {\qt नासौ शब्दो न तद्वाच्यं न सा विद्या न सा कला~। जायते यन्न काव्याङ्गमहो भारो महाकवेः~॥} इति कविरेव वेधाः~। उक्तं च \textendash\ {\qt अपारे काव्यसंसारे कविरेव प्रजापतिः}~। कवीनां वेधाः~। कविशब्दोऽत्रोपचारात्कविबुद्धिषु वर्तते~। तेन कविबुद्धीनां श्रेष्ठ इत्यर्थः~। तथा चाह मुनिः \textendash\ {\qt इतिहासोत्तमादस्माज्जायन्ते कविबुद्धयः} इति~। यद्वा व्युत्पत्त्युत्पादनद्वारेण कवय एवंभूताः सन्तः क्रियन्ते~। मुख्य एव कविशब्दस्यार्थः~। यदुक्तम् \textendash\ {\qt इदं कविवरैः सर्वैराख्यानमुपजीव्यते} इति~। पुण्यं पावनम्~। यदुक्तम् \textendash\ {\qt भारताध्ययनात्पुण्यादपि पादमधीयतः~। श्रद्दधानस्य पूण्यन्ते सर्वपापानि देहिनः~॥} इति~। सरस्वती वाणी तस्या लताया इव पुष्पादिहेतुत्वाद्वर्षं वृष्टिमिव वर्षं वा स्थानविशेषः~। यतोऽसौ तत्रास्ते~। यदुक्तम् \textendash\ {\qt यदिहास्ति तदन्यत्र यन्नेहास्ति न तत्क्वचित्}~। भरतानधिकृत्य कृतो ग्रन्थो भारतस्तम्~। यद्वा भारतं वर्षमिव~। भरतः कश्चिद्राजा तस्य निवासं भारतं वर्षं भूमागैकदेशस्तदिव~। उक्तं च \textendash\ {\qt स्यद्वृष्ट्यां लोकधात्र्यंशे वत्सरे वर्षमस्त्रियाम्} इति~। यद्वा भारतवर्षान्तरस्था भावा मनुष्येषु सुलभास्तद्वन्महाभारतस्था सरस्वती~। एतदपि सरस्वत्याख्यया नद्या पुण्यम्~॥~३~॥

एवं सर्वज्ञतागुणकथनेन कविप्रशंसा कृत्वा काव्यप्रशंसामाह \textendash\ {\qtt प्राय इत्यादिना}~। काव्यमेवं नाम स्वभावसुभगम्~। येनेदृशा अपि कवयः प्रायः प्राचुर्येण कोकिला इव जायन्ते वल्गुवाचः संपद्यन्ते~। किं पुनः संविशिष्टा न जायरेन्~। केचित्पुनर्भूयसा कुत्सिता कवयो जायन्त इति कुकविन्दैवेयमिति व्याख्यातवन्तः~। रागो द्वेषपूर्वकोऽनर्थाभिनिवेशस्तेनाधिष्ठिता दृष्टिर्बुद्धिर्येषाम्~। वाचाला असंबद्धप्रलापिनः~। कामेन स्वेच्छया, न त्वलंकारकृद्दर्शितनीत्या, कुर्वन्ति ये ते~। कोकिलापक्षे \textendash\ कुकन्ति, गृह्णन्ति चेतांसीति कुकाः~। ते च ते वयो मयूरप्रवराः पक्षिणः~। रागो लौहित्यम्~। दृष्टिश्चक्षुः~। वाचा भारत्या आला आ समन्ताल्लान्त्यावर्जयन्ति यतस्तादृशाः सन्तः~। कामं व्यसनं कुर्वन्ति तच्छीलाः~। कामिद्दीपनविभावतां यान्तीत्यर्थः यद्वा अवाचालाः~। अकारप्रश्लेषोऽत्र~॥~४~॥

{\qtt सन्तीत्यादि}~। असंख्या अगणनार्हाः~। जातिं स्वरूपवर्णनामात्ररूपां वक्रोक्तिशून्यां भजन्ते~। {\qt गतोऽस्तमर्को भातीन्दुर्यान्ति वासाय पक्षिणः} इत्यादिवत्~।}

\newpage
% प्रथम उच्छ्वासः~। ३

\begin{quote}
{\ha उत्पादका न बहवः कवयः शरभा इव~॥~५~॥

अन्यवर्णपरावृत्त्या बन्धचिह्ननिगूहनैः~।\\
अनाख्यातः सतां मध्ये कविश्चौरो विभाव्यते~॥~६~॥

श्लेषप्रायमुदीच्येषु प्रतीच्येष्वर्थमात्रकम्~।\\
उत्प्रेक्षा दाक्षिणात्येषु गौडेष्वक्षरडम्बरः~॥~७~॥

नवोऽर्थो जातिरग्राम्या श्लेषोऽक्लिष्टः स्फुटो रसः~।\\
विकटाक्षरबन्धश्च कृत्स्नमेकत्र दुष्करम्~॥~८~॥

किं कवेस्तस्य काव्येन सर्ववृत्तान्तगामिनी~।}
\end{quote}

\hrule

\noindent
{\s वानोऽप्यसंख्या नास्ति संख्यं सङ्ग्रामो येषां ते~। जातिशब्देनात्र श्वजातिसमवेता अमेघ्यभक्षणादयो गृहीताः~। यद्वा श्वत्वं नाम जातिस्तत्प्रतिपादनं प्रयोजनान्तरशून्यतामावेदयति~। उत्पादका नवनिर्माणकारिणः, ऊर्ध्वपादाश्च~। शरभा हि प्राणिभेदाः~। अष्टपादा एते~। श्वजातीया इति केचित्~॥~५~॥

{\qtt अन्येति}~। कविश्चौरः सहृदयानां मध्येऽनाख्यातः कथितोऽपि न ज्ञायते~। न आ समन्तात्ख्यातः, अपि तु किंचित्प्रथितो वा~। अन्ये पूर्वकविनिबद्धविलक्षणा ये वर्णा अक्षराणि तेषां रचनेन बन्धचिन्हं श्रीलक्ष्मीप्रभृतिरचनालिङ्गम्~। अन्ये तु भाषालंकारप्रभृतिबन्धचिह्नमाहुः~। अथ च सतां साधूनां मध्ये चौरो लक्ष्यते~। क्रीदृक्~। न ना अना कापुरुषः, अख्यातोऽप्रसिद्धः~। केन~। अन्यः प्राक्तनच्छायाव्यतिरिक्तस्त्रासकृतः पाण्डिमादिर्वर्णो मुखरागविशेषस्तत्परिवर्तनेन~। यद्वा शूद्रत्वे सति द्विजादिवर्णाश्रयेण~। स्वजात्युचितस्य स्वभावस्य त्यक्तुमशक्यत्वाद्भावप्रकटनमवश्यमेव भवति~। यतो बन्धः शृङ्खलादिकृतो ग्रन्थिस्तच्चिह्नं त्वग्दूषणादि~॥~६~॥

{\qtt श्लेषेत्यादि}~। मात्रकपदेन श्लेषयमकाद्यलंकारशून्यत्वं दर्शयति~। अक्षरेत्यादिनार्थविशेषाभावं प्रसादादिगुणगुम्फनाभावं चाख्याति~। एतदुक्तं भवति \textendash\ क्वचित्कश्चिद्गुणोऽपि भवति~। स च भवन्नपि न सहृदयजनावर्जक इति~। अमुनैवाभिप्रायेण नव इत्यादीनि प्रत्येकं विशेषणपदानि वक्ष्यति~॥~७~॥

{\qtt नव इत्यादि}~। नव आद्यैः कविभिरनिबद्धः, चमत्कारी च~। जातिः स्वभावोक्तिः~। अग्राम्येति, न तु {\qt गतोऽस्तमर्कः} इत्यादिरूपा~। सधर्मेषु तत्रप्रयोगः~। श्लेषः~। अक्लिष्टः सम्यगनेकार्थप्रतिपादनक्षमः~। स्फुटो दुर्बोधभङ्ग्यादिभिरदूषितो रसः शृङ्गारादिः~। विकट उदारतालक्षणबन्धयुक्तः~। यत्र सति नृत्यन्तीव पदानि प्रतिभासन्ते~॥~८~॥

{\qtt किमित्यादि}~। वृत्तानि वर्णमात्रागणसमार्धसमविषमरूपाणि तदन्तगमनं तद्विरचनक्षमत्वम्~। भारती वाणी~। न व्याप्नोति~। अदृष्टमपि दृष्टमिव जगत्रयं प्रतिभानवशाद्व्युत्पत्तेश्च तथात्वेन प्रकाशयति~। यद्वा जगत्रयप्रथिता भवतीति स्फुट एवार्थः~। भरतानधिकृत्य गरथिता भारती कथेव~। सापि सर्वे ये वृत्तान्ताः सत्पुरुषचरितान्यु\textendash}

\newpage
% ४ हर्षचरिते 

\begin{quote}
{\ha कथेव भारती यस्य न व्याप्नोति जगत्त्रयम्~॥~९~॥

उच्छ्वासान्तेऽप्यखिन्नासे येषां वक्रे सरस्वती~।\\
कथनाख्यायिकाकारा न ते वन्द्याः कवीश्वराः~॥~१०~॥

कवीनामगलद्दर्पो नूनं वासवदत्तया~।\\
शक्त्येव पाण्डुपुत्राणां गतया कर्णगोचरम्~॥~११~॥

पदबन्धौज्ज्वलो हारी कृतवर्णक्रमस्थितिः~।\\
भट्टारहरिचन्द्रस्य गद्यबन्धो नृपायते~॥~१२~॥

अविनाशिनमग्राम्यमकरोत्सातवाहनः~।}
\end{quote}

\hrule

\noindent
{\s पाख्यानानि च तान्गमयति बोधयति~। तथा सर्वत्र ज्ञेया भवति~। तथा च \textendash\ नारदोऽश्रावयद्देवानसितो देवलः पितॄन्~। गन्धर्वयक्षरक्षांसि श्रावयामास वै शुकः~॥ इत्युक्तम्~॥~९~॥

अधुना स्वगुरुतः स्वप्रभृतिभिः कृतानाख्यायिकादीन्काव्यभेदान्स्तुवन्ननदयार्थं सर्वत्र नमस्कारमाह \textendash\ {\qtt उच्छ्रासान्त इति}~। उच्छ्वास इवोच्छ्वासो विश्रान्तिस्थानं सर्गादिवत्कथासंधिस्तस्यान्तेऽप्यखिन्ना उच्छ्वासान्तरकरणक्षमाः~। अविच्छिन्नप्रतिभाना इति यावत्~। गुरुत्वाद्बहुवचनम्~। {\qt नाद्यान्तो ह्यम्बुधेवऋम्} इति वक्रलक्षणम्~। वक्रे सरस्वती~। वृत्तविशेषयोगिनीत्यर्थः~। एतस्मिन्नाख्यायिकाकृद्भिर्भाविवस्तुसंसूचनाय वाग्विरच्यते~। तथा चाह भामहः \textendash\ {\qt वक्रं चापरवक्रं च काव्बे काव्यार्थशंसिनि} इति~। आख्यायिकाः कुर्वन्तीत्याख्यायिकाकाराः~। यद्वा आख्यायिकेवाकारो येषाम्~। अथ {\qt कविं पुराणम्} इति न्यायेन कवयश्च त ईश्वरा हरिहरब्रह्माणः~। उच्छ्वसन्ति भूतान्यस्मिन्नित्युच्छ्वासः कल्पस्तदन्ते संहारेऽपि तेऽखिन्नाः कल्पान्तरजननोद्योगिनस्तेषां मुखे वागीशी~। उक्तं च \textendash\ {\qt सरस्वतीयाग्बलमुत्तमोऽनिलः} इत्यादि~। आख्यायिकाभिराख्यानैराकारी येषाम्~। सर्वस्य हि शास्त्रागमसमधिगम्याः, न पुनः प्रत्यक्षलक्ष्याः~। ते च वन्द्याः सर्वस्य~॥~१०~॥

{\qtt कवीनामिति}~। वासवदत्ता कथा, वासवेन शकेण दत्ता च~। कर्णः श्रवणम्, राधेयश्च~। कवीनां काव्यकर्तॄणाम् द्रोणादीनां च~॥~११~॥

{\qtt पदेत्यादि}~। पदानां सुप्तिङन्तानां बन्धः, प्रकृष्टा रचना~। रीतिरित्यर्थः~। स्वमण्डलावष्टम्भव~। हारी हृद्यः दारयुक्त~। अहारीति वा~। न कस्यचिदपि यो हरति~। कृता वर्णानामक्षराणां क्रमेण भामहादिप्रदर्शितनीया स्थितिरवस्थानं यत्र, कृतयुगवद्वर्णानां द्विजादीनां क्रमेण मन्वादिस्मृतिकारप्रकाशितमार्गेण स्थितिः पालनं यस्मिन्सतीति च~। भट्टारेति पूजावचनम्~॥~१२~॥

{\qtt अविनाशिनमित्यादि}~। अविनाशिनं प्रसिद्धम्, अनश्वरं च~। अग्राम्यं वैदरध्ययुक्तम्, अग्रामभवं च~। जातिः स्वभावोक्तिरूपोऽलंकारः~। कोशः समुच्चयः,}

\newpage
% प्रथम उच्छ्वासः~। ५ 

\begin{quote}
{\ha विशुद्धजातिभिः कोशं रत्नैरिव सुभाषितैः~॥~१३~॥

कीर्तिः प्रवरसेनस्य प्रयाता कुमुदोज्ज्वला~।\\
सागरस्य परं पारं कपिसेनेव सेतुना~॥~१४~॥

सूत्रधारकृतारम्मैर्नाटकैर्बहुभूमिकैः~।\\
सपताकैर्यशो लेभे भासो देवकुलैरिव~॥~१५~॥

निगर्तासु न वा कस्य कालिदासस्य सूक्तिषु~।\\
प्रीतिर्मधुरसान्द्रासु मञ्जरीष्विव जायते~॥~१६~॥

समुद्दीपितकंदर्पा कृतगौरीप्रसाधना~।}
\end{quote}

\hrule

\noindent
{\s गञ्जश्च~। सुभाषितैः सूक्तिभिः, शोभनं च भाषितं प्रभावर्णनं येषां तैः~॥~१२~॥

{\qtt कीर्तिरित्यादि}~। प्रवरसेनः कश्चित्कविः प्रवे ते रसो येषां ते प्रवरसा वानरास्तेषामिनः स्वामी प्रवरा च सेना यस्य स सुग्रीवश्च~। कुमुदवत्कैरववत्~। यद्वा कुर्भूमिस्तस्या मुत्प्रहर्षस्वयेति, कुमुदेन वानरसेनापतिना च~। सेतुः प्राकृतकाव्यग्रन्थः, सेतुश्च~॥~१४~॥

{\qtt सूत्रेत्यादि}~। सूत्रधारः पूर्वरङ्गस्य प्रवक्ता चार्चिक्यः, स्थपतिश्च~। भूमिकाः पात्राणि रामायनुकार्यावस्थाभूमयः, उपभोगनिमित्तान्युत्पत्तिस्थानानि~। पताका अर्थप्रकृतिः~। उक्त च \textendash\ {\qt बीजं बिन्दुः पताका व प्रकरीकार्यमेव च~। अर्थप्रकृतयो ह्येताः पश्च सर्वप्रयोगगाः~॥} इति~। यद्वृत्तं तु परार्थस्योपकारकम्~। प्रधानवच कल्पेत सा पताकेति कीर्त्यते~॥ वैजयन्ती च पताका~॥~१५~॥

{\qtt निर्गतास्विति}~। निर्गता उच्चारितमात्राः~। आस्तां तावदर्भावगतिः, आपात एव गीतघ्वनिवत्किमपि श्रोत्रहारियः~। यदुक्तम् \textendash\ {\qt अपर्यालोचितेऽप्यर्थे बन्धसौ न्दर्यसंपदा~। गीतवद्धृदयाहादं तद्विदां विघाति यत्~॥ तत्काव्यम्} इत्यादि~। तथा निर्गताः सर्वदेशप्रतीताः अन्यत्र निर्गता अभिनवोद्भिन्नाः~। न वा कस्येत्यनेनैतदुक्तम्~। आसतां तावत्काव्यतत्त्वविदः सहृदया विवेतारः, येऽपि शास्त्राप्रहितबुद्धयो दुर्दुरुढा मत्सरप्रायास्तेषामपि या हृदयमाह्लादयन्ति~। तथा चोक्तम् \textendash\ {\qt असुणिअ परमन्थाण वि हरेइ वाआमणं कइन्माण~। आणाणजकुचलअवणमलद्धगन्धाण वि सुहाइ~॥} इति~। मधुराक्ष ताः सान्द्राः सरसाः~। अन्यत्र मधुना मकरन्देन किंजल्केन रसेन सान्द्राः सुगन्धयः~॥~१६~॥

{\qtt समुदित्यादि}~। बृहत्कथा कस्य न विस्मयाय~। अपि तु सर्वस्यैव गर्वविनाशाय भवतीत्यर्थः~। अद्भुतकथावर्णनाद्वाश्चर्याय~। समुद्दीपितो वृद्धिं नीतः कंदर्भों यस्याम्~। कामजननानां बहूनां वृत्तान्तानां वर्णनादुद्बोधितः स्मरो ययेति वा~। काव्यसेवया हि शृङ्गाररसः समुद्भवति~। तथा चोकम् \textendash\ {\qt ऋतुमाल्यालंकारप्रिय जनगान्धर्वकाव्यसेवाभिः~। उपवनगमनविहारैः शृङ्गाररसः समुद्भवति~॥} यद्वा समुद्दीपितः प्रकाशितः ख्याति नीतः कंदर्पो नरवाहनदत्तो यस्यामिति~। स हि}

\newpage
% ६ हर्षचरिते

\begin{quote}
{\ha हरलीलेव नो कस्य विस्मयाय बृहत्कथा~॥~१७~॥

आढ्यराजकृतोत्साहैर्हृदयस्थैः स्मृतैरपि~।\\
जिह्वान्तः कृष्यमाणेव न कवित्वे प्रवर्तते~॥~१८~॥

तथापि नृपतेर्भक्त्याभीतो निर्वहणाकुलः~।\\
करोम्याख्यायिकाम्भोधौ जिह्वाप्लवनचापलम्~॥~१९~॥

सुखप्रबोधललिता सुवर्णघटनोज्ज्वलैः~।\\
शब्दैराख्यायिका भाति शय्येव प्रतिपादकैः~॥~२०~॥

जयति ज्वलत्प्रतापज्वलनप्राकारकृतजगद्रक्षः~।}
\end{quote}

\hrule

\noindent
{\s कामांश इत्यागमः~। कृतं गौर्या विद्याभेदस्याराधनं यस्याम्~। सा हि नरवाहनदत्तेनेशारूपाराधितेति तत्रोक्तम्~। यद्वा गौरीं प्राति पूरयति गौरीप्रः~। साधनं परिकरबन्धो यथाप्रस्तावो यस्याम्~। गौरीप्रेरितेन हि हरेण तथा तस्यां परिकरबन्धः कृतो यथा सातीव पिप्रिये~। हरलीलापि समुत्सहर्षा दग्धकामा च~। कृतं गौर्याः प्रसाधनं मण्डनं यस्याम्~। क्व कामे प्रति तादृग्द्वेषः, क्व च कान्तां प्रति प्रसाधनमिति कृत्वा विस्मयमाश्चर्यम्~॥~१७~॥

{\qtt आढ्येति}~। आढ्यराजः कश्चित्कविः~। उत्साहो वृत्ते तालविशेषः~। उदीर्यमाणगीत्याधारभूतपदोपचारात्काव्यमप्युत्साह इति केचित्~। यत्र पूर्व श्लोकेनार्थ उपक्षिप्यते, पश्चात्स एव गद्येन वितन्यते, मध्ये वृत्तनिबन्धश्च भवति, स परिसमाप्तार्थ उत्साह उच्यत इलन्ये अपिः समुचये~। यद्वा आढ्यराजहृदयस्था अप्यन्तर्जिह्वां नाकर्षयन्ति, तत्कथं त एवं स्मृता इत्यपिशब्दार्थः~॥~१८~॥

एवमनौद्धत्यमुक्त्वाह \textendash\ {\qtt तथेत्यादि}~। तथापीत्थं जानन्नपि जिह्वालवनलक्षणं चापलं करोमि~। यतो नृपतेर्भक्त्याइनभि इतः समन्तायुक्तः~। निर्वहणे समाप्तावाकुल:~। जिह्वा चाव्धावकालवातस्तत्र वहन्त्यां कश्चिद्यथालवनरूपं चापलं करोति~। अत्र पक्षे अभीतोऽत्रस्तः~। निर्वहणं पारप्राप्तिः~। {\qt कृत्ये च} इति णत्वम्~॥~१९~॥

{\qtt सुखेत्यादि}~। सुखेन जायासंमितत्वेन हृदयाह्लादनपूर्वम्, न तु वेदेतिहासादिवत्, यः प्रबोधः प्रकृष्टं बोधनं धर्मादिसाधनव्युत्पत्तिः~। उक्तं च \textendash\ {\qt कटुकौषधिवत्काव्यमविद्याव्याविभेषजम्~। आह्वाद्यमृतवत्काव्यमविवेकगदापहम्~॥} इति~। सुवर्णघटना शोभनाक्षररचना~। प्रतिपादकैर्विवक्षिताभिधायकैः~। शय्यापक्षे \textendash\ सुखं यः प्रबोधः खापादुत्थानम्~। सुवर्णेघटना हेमयोजना~। प्रतिपादकैः खट्टाया उन्नामकैः~। तदापादानां प्रतिच्छन्दाः प्रतिपादकाः पुरुषयत्नोत्थापिताः पादमुद्रास्तैः~। अत्र च शोभनो वर्णोडलक्तादिकृतः~॥~२०~॥

इदानीं यमुद्दिश्येयमाख्यायिका क्रियते तस्य {\qt तथापि नृपतेर्भक्त्या} इत्यनेन नृपतिशब्देन सामान्येन निर्देशं कृत्वा विशेषेणाह \textendash\ {\qtt जयतीत्यादि}~। ज्वलन्दीप्रतया\textendash }

\newpage
% प्रथम उच्छ्वासः~। ७ 

\begin{quote}
{\ha सकलप्रणयिमनोरथसिद्धिश्रीपर्वतो हर्षः~॥~२१~॥}
\end{quote}

\vspace{-3mm}
एवमनुश्रूयते \textendash\ पुरा किल भगवान्स्वलोकमधितिष्ठन्परमेष्ठी विकासिनि पद्मविष्टरे समुपविष्टः सुनासीरप्रमुखैर्गीर्वाणैः परिवृतो ब्रह्मोद्याः कथाः कुर्वन्नन्याश्च निरवद्या विद्यागोष्ठीर्भावयन्कदाचिदासांचक्रे~। तथासीनं च तं त्रिभुवनप्रतीक्ष्यं मनुदक्षचाक्षुषप्रभृतयः प्रजापतयः सर्वे च सप्तर्षिपुरःसरा महर्षयः सिषेविरे~। केचिदृचः स्तुतिचतुराः समुदचारयन्~। केचिदपचितिभाञ्जि यजूंष्यपठन् केचित्मशंसासामानि जगुः~। अपरे विवृत\textendash

\vspace{2mm}
\hrule

\noindent
{\s प्रसरन्प्रताप एव ज्वलनस्तं प्राति पूरयति य आकारस्तेन कृता जगति रक्षा येन सः~। सकलानां प्रणयिनां ये मनोरथास्तत्सिद्धौ श्रियां पर्वतो गिरिः~। श्रियस्तत्र कूटीभूता इव स्थिता इति यावत्~। यद्वा यथा पर्वतस्थः कश्चिद्दुरभिभवः, तद्वद्धर्षस्था श्रीरिति~। अथ च श्रीपर्वताख्यो गिरिरीदृगेव~। तथा च ज्वलत्प्रकृष्टतापो यो ज्वलनो जठराग्निः स एव निषेधकत्वात्प्राक्रारः सालस्तेन कृता मुक्तेविघ्नहेतुतया जगतो भूलोकस्य रक्षा येन सः~। अन्यत्रोत्सादनं तद्यावत्~। अन्ये तु \textendash\ त्रिपुरदाहे यो विघ्नमकरोद्गणेशस्तदा हरेण ज ज्वलत्प्रकृष्टतापो ज्वलनप्राकारो निर्मितः~। तेन च तत्र रक्षा विधीयत इत्याहुः~। ज्वलत्प्रतापो ज्वलनप्राकारश्च द्वौ मुद्राख्पौ मन्त्रविशेषौ स्तः, ताभ्यां कृतजगद्रक्ष इति केचित्~। प्रणयिनः सिद्धिकामाः~। हर्षः कथानायकः~। इतरत्र हर्षकारितया हर्षः~। सर्वत्र च परमार्थतो हर्ष एव जयति~। तस्यैवाभिलषणीयत्वात्स एव काव्येन क्रियत इति ध्वनति~॥~२१~॥

{\qtt एवमिति}~। अनुश्रूयते पारम्पर्येणाकर्ण्यते~। किलेल्यत एवागमसूचनाय~। भगवानिति केवलनिर्देश उल्लुण्ठनपरिहारार्थम्~। ब्रह्मलोकमित्युक्ते सत्युत्कर्षदायिन्यात्मीयताप्रतिपत्तिर्न स्यादिति स्वग्रहणं साभिप्रायम्~। अधितिष्ठन्बहुमानेन तद्योगक्षेमादिकसुद्वहन्~। परमे पदे तिष्ठतीति परमेष्ठी~। विकासिनीति नित्ययोग इनिः~। विष्टरमासनम्~। सुनासीर इन्द्रः~। गिरः स्तुतिरूपा वर्णन्ति भजन्तीति गीर्वाणा देवाः~। गीरेव वाणः शरो येषामिति~। परिवृतश्चतुर्दिक वृतः परिवलितः~। तस्य चतुर्मुखत्वात्~। ब्रह्म वदन्तीति ब्रह्मोद्याः~। {\qt वदः स्वपि क्यप्च}~। ब्रह्मणा वेदेन, ब्रह्मणि परमात्मनि वा वेदितव्या ब्रह्मोवाः~। उक्तं च \textendash\ ब्रह्मोद्या सा कथा यस्यामुच्यते ब्रह्म शाश्वतम् इति~। सामान्यविशेषभावेन {\qt उष्ट्रासिकामासवे} इतिवत्~। ब्रह्मवदनरूपा वा कथास्तासां वक्ष्यमाणगोष्ठ्यभिप्रायेण प्राधान्यात्स्वयंकरणम्~। निरवद्या दोषरहिताः~। तथा च वात्स्यायनः \textendash\ {\qt या गोष्टी लोकविद्रिष्टा या च स्वैरविसर्पिणी~। परहिंसात्मिका या च न तामवतरेद्बुधः~॥ लोकचित्तानुवर्तिन्या क्रीडामात्रैककार्यया~। गोष्ठ्या सह चरन्विोकसिद्धिं नियच्छति~॥} समानविद्यावित्तशीलबुद्धिवयसामनुरूपैरालापैरेकनासनबन्धी गोष्ठी~। प्रतीक्ष्यः पूज्यः~। सम्य\textendash }

\newpage
% ८ हर्षचरिते

\noindent
क्रतुक्रियातन्त्रान्मन्त्रान्व्याचचक्षिरे~। विद्याविसंवादकृताश्च तत्र ते घामन्योन्यस्य विद्याविवादाः प्रादुरभवन्~।

अथातिरोषण: प्रकृत्या महातपा मुनिरन्नेस्तनयस्तारापतेर्भ्राता नाम्ना दुर्वासा द्वितीयेन मन्दपालनाम्ना मुनिना सह कलहायमानः साम गायन्क्रोधान्धो विस्वरमकरोत्~। सर्वेषु च शापभयप्रतिपन्नमौ नेषु मुनिष्वन्यालापलीलयावधीरयति कमलसंभवे भगवती कुमारी किंचिदुन्मुक्तबालभावे भूषितनवयौवने नवे वयसि वर्तमाना, गृही तचामरप्रचलद्भुजलता पितामहमुपवीजयन्ती, निर्भर्त्सनताडनजा वरागाभ्यामिव स्वभावारुणाभ्यां पापल्लवाभ्यां समुद्भासमाना, शिष्यद्वयेनेव पदक्रममुखरेण नूपुरयुगलेन वाचालितचरणा, मदन नगरतोरणस्तम्भविभ्रमं बिभ्राणा जङ्घाद्वितयम्, सलीलमुत्ककल हंसकुलकलालापग्रलापिनि मेखलादानि विन्यस्तवामहस्तकिसलयाविद्वन्मानसनिवासलग्नेन गुणकलापेनेवांसावलम्बिना ब्रह्मसूत्रेण प वित्रीकृतकाया, भास्वन्मध्यनाय कमनेकमुक्तानुयातम पवर्गमार्गमिव

\vspace{2mm}
\hrule

\noindent
{\s गुदात्तादित्रैस्वर्यादिप्राधान्यादुदचारयञ्जगुः~। अपचितिः पूजा~। सामानि जगुरिति साम्नां गानमेवोचितम्~। विद्याविसंवादकृता इति, न तु मात्सर्यादिना प्रादुरभवनित्यनौचित्यशङ्कया तत्कर्तृत्वपरिहारः~।

{\qtt प्रकृत्येति}~। अन्यथा ब्रह्मसंविधानेन कथमीदृगाक्षेपः~। कथमीदृशोऽवकाश इत्याह \textendash\ {\qtt महातपा इति}~। मुनिरित्यनेनास्य ज्ञानप्राधान्यात्तुल्यतोद्भासनमतीवापकारः~। अत्रेस्तनय इति न केवलं महातपस्त्वेन, यावदन्त्रितनयत्वेन ब्रह्मलोकप्राप्तिरस्य~। तत खारापतेरित्यादिना तथाभूतपरमप्रजापतिसंबन्धयोग्यत्वमस्याख्यायते~। द्वितीयेनेति तत्समत्वमुच्यते~। कथं सामगानेऽप्यनवहित इत्याह \textendash\ {\qtt क्रोधान्ध इति}~। सर्वेध्वित्यादौ देवी सरस्वती श्रुत्वा जहासेति क्रियाप्रतिपत्तिरस्य मा भूदित्युत्तमप्रकृतित्वादन्येत्यायुतम्~। अन्येन सहायापलीलाकथाकीडया~। {\qtt कुमारीति}~। .कुमारीत्वेनास्या हास्यादिकं नानुचितमिति दर्शयति~। भूषितेत्यनेन दर्शनीय त्वमाह~। {\qtt पितामहमिति}~। सर्वप्राधान्यमनेनोक्तम्~। निर्भर्त्सनताडनं तेन तदर्थ वा यत्ताडनं रोषाडूमिहननं तद्शाच जातरागाभ्यामिव पादपलवाभ्यामित्यनेनारुणत्वं सौकुमार्यं चाह~। अतएव गाढताडनेन रचत्वनुत्प्रेक्षितम्~। ताडितो वायं ताडितस्ततुल्यो रागो जातो ययोरिति व्याख्येयम्~। पदकमं पादन्यासपरिपाटी~। अन्यत्र च पदानि च कमथ तत्पदक्रमम्~। चरणौ पादौ चरणाथ विशिष्टशाखापाठकता वाचालिताः क्षोभिता ययेति~। उत्का उत्सुकाः~। मेखलादाम्नि रशनागुणे~। मानसं चित्तम्, सरोविशेपश्च~। गुणा अपि मास्वन्दीयो मध्यनायकः पदकं यत्र तत्~। अथ च भास्वती मध्यं}

\newpage
% प्रथम उच्छ्वासः~। ९ 

\noindent
हारमुद्वहन्ती, वदनप्रविष्टसर्वविद्यालक्तकरसेनेव पाटलेन स्फुरता दशनच्छदेन विराजमाना, संक्रान्तकमलासनकृष्णाजिनप्रतिमां मधुरगीताकर्णनावतीर्णशशिहरिणामिव कपोलस्थलीं दधाना, तिर्यक्सावज्ञमुन्नमितैकभ्रूलता, श्रोत्रमेकं विस्वरश्रवणकलुषितं प्रक्षालयन्तीवापाङ्गनिर्गतेन लोचनाश्रुजलप्रवाहेणेतरश्रवणेन च विकसितसितसिन्धुवारमञ्जरीजुषा हसतेव प्रकटितविद्यासदा, श्रुतिप्रणयिभिः प्रणवैरिव कर्णावतंसकुसुम मधुकरकुलैरुपास्यमाना, सूक्ष्मविमलेन प्रज्ञाप्रतानेनेवांशुकेनाच्छादितशरीरा, वाङ्मयमिव निर्मलं दिक्षु दशनज्योत्स्नालोकं विकिरन्ती देवी सरस्वती श्रुत्वा जहास~।

दृष्ट्वा च तां तथाहसन्तीं स मुनिः {\haq आः पापकारिणि, दुई हीतविद्यालवावलेपदुर्विदग्धे, मामुपहससि} इत्युक्त्वा शिरःकम्पशीर्यमाणबन्धविशरारोरुन्मिपत्पिङ्गलिम्नो जटाकलापस्य रोचिषा सिञ्चन्निव रोषद्हनद्रवेण दश दिशः, कृतकालसंनिधानानिवान्धकारितललाटपट्टाष्टापदामन्तकान्तःपुरमण्डनपत्रभङ्गमकरिकां भ्रु\textendash

\vspace{2mm}
\hrule

\noindent
{\s तेन नयति सः~। यदुक्तम् \textendash\ {\qt परिव्राड्योगयुक्तश्च शूरश्चाभिमुखं हृतः~। द्वाविमौ पुरुषो लोके सूर्यमण्डलभेदिनौ~॥} इति~। मुक्ता मौक्तिकानि, मोक्षगामिनथ~। हारं मुक्ता कलापश्च, अपवर्गमपि~। हारं हरसंबन्धिनं तत्प्रसादप्राप्यत्वात्~। {\qt अलक्तकरसेनेव पाटलेन} इति वा पाठः स्फुरतेति रोषात्~। भगवतीकपोले शशिहरिणस्यैवावतारः संभाव्यत इति शशिपदम्~। अत्र हि कपोले ब्रह्मकृष्णाजिनसंक्रान्तिः, तत्र कामसंभावना सामान्यहरिणस्यावतरणे~। {\qtt कलुषितं प्रक्षालयन्तीवेति}~। सलिलस्य क्षालनमेव युक्तमिति समुचितेयमुक्तिः~। {\qtt श्रुतिप्रणयिभिरिति}~। श्रूयते इति श्रुतिर्ध्वनिस्तया प्रणयः प्रशंसातिशयो येषां तैः~। यद्वा श्रुती श्रोत्रे तत्कर्तृकः प्रणयः प्रार्थना मथुरध्वनित्वाद्येषां तैः~। कर्णसंबन्धैरिति तु व्याख्याने कर्णावतंसेत्यादिना पौनरुक्त्यमपरिहार्यम्~। श्रुतिर्वेदोऽपि~। सूक्ष्मार्थदर्शितत्वात्सूक्ष्मस्तीक्ष्णः, विमलस्तत्त्वग्राही~। अन्यत्र सूक्ष्मं तनु, विमलं शुक्रम्~। प्रतानः प्रसरः~।

दृष्ट्वेत्यादौ स मुनिस्तां तथाहसन्ती हा शापजलं जमाहेति संबन्धः~। तथेति पादताडनभ्रूक्षेपादिपूर्वम्~। स मुनिरिति प्राग्वर्णितस्वरूपः~। आः इत्यक्षमायाम्~। मामिति योऽहं त्रैलोक्यप्रख्यातरोषणस्त मेवेति~। समीप एव विशीर्यते तच्छीलो विशरारुरितश्चामुतश्च~। अत एवोन्मिपत्पिङ्गलिमारोचिषा दीत्या~। रोषदहनो द्रवो रस इव द्रवलं च यद्यपि विशिष्टस्यैवं तेजसः सुवर्णादि संभवति, तथाप्यत्रोपचारात्सादृश्यम्~। कालः कृष्णो गुणो यमच~। अन्धकारितं संकुचितत्वाददर्शनीयमेव चकितं ललाटपट्टमेवाष्टापदम्~। यथा प्रतिपङ्क्ति, अष्टौ पदान्यस्येत्यष्टापदं चतुरङ्गफलकम्~।}

\lfoot{ह० २}

\newpage
\lfoot{}
% १० हर्षचरिते 

\noindent
कुटिमाबध्नन, अतिलोहितेन चक्षुषामदेवतायै स्वरुधिरोपहारमिव , प्रयच्छन्, निर्दयदष्टदशनच्छदभयपलायमानामिव वाचं रुन्धन्दन्तांशुच्छलेन, अंसावस्रंसिनः शापशासनपट्टस्येव ग्रन्थन्ग्रन्थिमन्यथा कृष्णाजिनस्य, स्वेदकणप्रतिबिम्बितैः शापशङ्काशरणागतैरिव सुरासुरमुनिभिः प्रतिपन्नसर्वावयवः, कोपकम्पतर लिताङ्गुलिना करेण प्रसादनलग्नामश्क्षरमालामिवाक्षमालामाक्षिप्य कामण्डलवेन वारिणा समुपस्पृश्य शापजलं जग्राह~।

अत्रान्तरे स्वयंभुवोऽभ्याशे समुपविष्टा देवी मूर्तिमती पीयूषफे नपटलपाण्डरं कल्पद्रुमदुकूलवल्कलं वसाना, विसतन्तुमयेनांशुकेनोन्नतस्तनमध्यबद्धगात्रिकाग्रन्थिः, तपोबलनिर्जितत्रिभुवनजयपताकाभिरिव तिसृभिर्भस्मपुण्डकराजिभिर्विराजितललाटाजिरा, स्कन्धावलम्बिना सुधाफेनधवलेन तपःप्रभावकुण्डलीकृतेन गङ्गास्रोतसेव योगपट्टकेण विरचितवैकक्ष्यका, सव्येन ब्रह्मोत्पत्तिपुण्डरीकमुकुलमिव स्फटिककमण्डलुं करेण कलयन्ती, दक्षिणमक्षमालाकृतपरिक्षेपं कम्बुनिर्मितोर्मिकादन्तुरितं तर्जनतरङ्गिततर्जनीकमुत्क्षिपन्ती

\vspace{2mm}
\hrule

\noindent
{\s अत एवानेन भ्रूसमुन्नमनमव्यक्तीकृतरेखचत्तया विस्पष्टव्यली कमेतत्~। {\qt ललाटमुपगीयते~। भ्रुवोर्मूलसमुत्क्षेपाकुटिं परिचक्षते}~। सुब्शब्दः सुतरां नैरपेक्ष्यसूचनाय वा चोभयसंबन्धः~। {\qtt अंसावस्रंसिन इति}~। संरम्भाच्छासनपट्टः शुक्रवालिपि कार्ष्याच सितासितवर्णसंवलितमध्यः पर्यन्तशुक्लश्च भवति~। अत एव ते विन्दु चित्रत्वादुपान्तशुक्लत्वाच कृष्णाजिनमुत्प्रेक्षते~। यथा शासनपट्टे सति क्वचिद्रामादा चधिकारो भवति, तद्न्च जनसमूहः प्रार्थनां करोति~। सहस्तपादादिके सर्वेस्मिन्नने गलति~। कोपेत्यादौ कम्पग्रहणं रोषः~। शरीरं बाधत इति यावत्~। संनिवेश साधर्म्यादुक्तम् \textendash\ {\qtt अक्षरमालामिवेति}~। सरस्वतीसंबन्धतया चोक्तम् \textendash\ {\qtt प्रसादनलग्नामिति}~। विक्षिप्यन्ते~। यश्च विरुद्धपक्षः प्रसादयति स विक्षिप्यते तिर स्क्रियते~। कामण्डलवेन मुनिकरकभवेन~। सनुपस्पृश्याचम्य~।

अत्रान्तर इत्यादौ मूर्तेचतुर्भिर्वेदैः सह सावित्री समुत्तस्थाविति संबन्धः~। अभ्याशे समीपे~। गात्रिकाप्रन्थिर्मन्थिविशेषः स्वस्तिकाकारः स्त्रीणामुत्तरीयस्त्र स्तनोद्देशे भवति~। तिलकं पुण्ड्रकं स्कन्धायंसौ वायुस्थानानि च स्कन्धाः~। फेनैस्तद्वच धवलेन~। {\qt तिर्यग्वक्षसि विक्षिप्तं वैकक्ष्यकमुदाहृतम्}~। सव्येन वामेन~। पुण्डरीकमुकुलं मुकुलितं पद्म~। कलयन्ती क्षिपन्ती, धारयन्ती वा~। परिक्षेपः परिवलनम् कम्बुः शङ्खः~। ऊर्मिका वालिका~। दन्तुर इव दन्तुरो व्याप्तस्तम्~। तर्जनं निर्भर्त्सनम्~।}

\newpage
% प्रथम उच्छ्वासः~। ११ 

\noindent
करम्, आः पाप, क्रोधोपहत, दुरात्मन्, अज्ञ, अनात्मज्ञ, ब्रह्मबन्धो, मुनिखेट, अपसद, निराकृत, कथमात्मस्खलितविलक्षः सुरासुरमुनिमनुजवृन्दवन्दनीयां त्रिभुवनमातरं भगवतीं सरस्वतीं शप्तुमभिलषसि इत्यमिदधाना, रोषविमुक्तवेत्रासनैरोंकारमुखरित मुखैरुत्क्षेपदोलायमानजटाभारभरितदिग्भिः परिकरबन्धभ्रमितकृष्णाजिनाटोपच्छायाश्यामायमानदिवसैरमर्षनिःश्वासदोलाप्रेसोलितब्रह्मलोकैः सोमरसमिव स्वेदविसरव्याजेन स्रवद्भिरग्निहोत्रपवित्रभस्मस्मेरललाटैः कुशतन्तुचारुचामरचीरचीवरिभिराषाढिभिः ग्रहरणीकृतकमण्डलुमण्डलैर्मूर्तेश्चतुभिर्वेदैः सह वृषीमपहाय सावित्री समुत्तस्थौ~।

ततो {\haq मर्षय भगवन् , अभूमिरेषा शापस्य} इत्यनुनाध्यमानो ऽपि विबुधैः, {\haq उपाध्याय, स्खलितमेकं क्षमस्व} इति बद्धाञ्जलि पुटैः प्रसाद्यमानोऽपि स्वशिष्यैः, {\haq पुत्र, मा कृथास्तपसः प्रत्यूहम्} इति निवार्यमाणोऽप्यत्रिणा, रोषावेशविवशो दुर्वासाः {\haq दुर्विनीते, व्यपनयामि ते विद्याजनितामुन्नतिमिमाम्~। अधस्ताद्गच्छ मर्त्यलोकम्} इत्युक्त्वा तच्छापोदकं विससर्ज~। प्रतिशापदानोद्यतां सावित्रीम् {\haq सखि, संहर रोषम्~। असंस्कृतमतयोऽपि जायैव द्विजन्मानो माननीयाः} इत्यभिद्धाना सरस्वत्येव न्यवारयत्~।

\vspace{2mm}
\hrule

\noindent
{\s तरङ्गिता तर्जिता चलिता~। तर्जनी प्रदेशिन्यष्ठनिकालिः~। क्रोधोपहतेयात्मविनाशायैव ते क्रोध इत्युक्तं भवति~। ब्रह्मबन्धो निकटब्राह्मण अपसदो नीचः~। निराकृतोऽस्वाध्यायः~। विलक्षो लज्जितः~। सुरासुरमनुजाथ परस्परविरुद्धानुष्ठानाः~। अत्र पुनरीदृशामपि न विप्रतिपत्तिरिति भावः~। {\qtt अभिलषसीति}~। इच्छामात्रकमपीदं महत्साहसमित्यर्थः~। ॐकार एव मुखरितं मुखं येषां तैः~। परिकरबन्धःपर्यध्वन्धः~। स चोत्थितस्यापि संरम्भभाजो भवति~। आटोपो वक्षः प्रदेशे श्यामा यमानो रात्रिरिवाचरद्दिवसा यैर्हेतुभिरित्यर्थः~। अमर्षनिःश्वासैलावत्प्रेोलि तबलितो ब्रह्मलोको यैः~। कुशतन्तूनां चामरमिव चामरं गुच्छः~। कुशतन्तुचामरं दर्भपिञ्जरम्, चीरचीवरं वृक्षत्वग्वस्त्रं ते विद्येते येषां तैः~। {\qt आषाढसंशो दण्डस्तु पालाशो व्रतचारिणाम्}~।

तत इत्यादौ शापोदकं जमाहेति विससजैति संबन्धः~। मर्षय क्षमस्व~। अनुनाथ्यमानः प्रार्थ्यमानः~। प्रत्यूहं विघ्नम्~। {\qtt उन्नतिमिति}~। उच्चदेशस्थथाधस्तात्रीयत इति समुचितेयमुक्तिः~। असंस्कृतमतयः संस्काररहिताः~।}

\newpage
% १२ हर्षचरिते 

अथ तां तथाशप्तां सरस्वतीं दृष्ट्वा पितामहो भगवान्कमलोत्सत्तिलग्नमृणालसूत्रामिव धवलयज्ञोपवीतिनीं तनुमुद्वहन, उद्गच्छदच्छाङ्गुलीयमरकतमयूखलताकलापेन त्रिभुवनोपलवप्रशमकुशापीडधारिणेव दक्षिणेन करेण निवार्य शापकलकलमतिविमलदीर्घैर्भाविकृतयुगारम्भसूत्रपातमिव दिक्षु पातयन् दशनकिरणैः सरस्वतीप्रस्थानमङ्गलपटहेनेव पूरयन्नाशाः, स्वरेण सुधीरमुवाच \textendash\ ब्रह्मन्, न खलु साधुसेवितोऽयं पन्थाः, येनासि प्रवृत्तः~। निहन्त्येष परस्तात्~। उद्दामप्रसृतेन्द्रियाश्वसमुत्थापितं हि रजः कलुषयति दृष्टिमनक्षजिताम्~। कियहूरं वा चक्षुरीक्षते~। विशुद्धया हि धिया पश्यन्ति कृतबुद्धयः सर्वानर्थानसतः सतो वा~। निसर्गविरोधिनी चेयं पयःपावकयोरिव धर्मक्रोधयोरेकत्र वृत्तिः~। आलोकमपहाय कथं तमसि निमज्जसि~। क्षमा हि मूलं सर्वतपसाम्~। परदोषदर्शनदक्षा दृष्टिरिव कुपिता बुद्धिर्न त आत्मरागदोषं पश्यति~। व महातपोभारवैवधिकता~। क्व पुरोभागित्वम्~। अतिरोषणश्चक्षुष्मानन्ध एव जनः~। नहि कोपकलुषिता विसृशति मतिः कर्तव्यमकर्तव्यं वा~। कुपितस्य प्रथममन्धकारीभवति विद्या, ततो भ्रुकुटि:~। आदाविन्द्रियाणि रागः समास्कन्दति, चरमं चक्षुः~।

\vspace{2mm}
\hrule

{\s अथेत्यादौ भगवान्पितामहः सुधीरमुवाचेति संबन्धः~। {\qtt तथेति}~। तेन प्रकारेण~। निरपराधां सरस्वतीमित्यर्थः~। {\qtt धवलयज्ञोपवीतिनीमिति}~। प्रशंसायां नित्ययोगे वा मत्वर्थीयः~। {\qt बिसकिसलयच्छेदपाथेयवन्तः} इतिवत्~। अन्यथा कर्मधारये कृते मत्वर्थीय एकबुद्ध्यनुमितौ बहुव्रीही प्रतिपत्तिर्भवतीति~। इतरत्र तु बुद्धिद्वयमिति लघुत्वात्प्रत्युतम्~। सच्छनच्छाङ्गुलीयमरकतस्य मयूखलताकलापो यस्य तेन करेण आपीडः समूहः~। पातं विन्यासम्~। पातयन्कुर्वन्~। अत्र हि धात्वर्थगतानुष्ठानमात्रवृत्तिः क्रिया~। यथा \textendash\ {\qt संवस्ते क्षालिते वस्त्रे} इति~। पन्था व्यवहारः, मार्गश्च~। निहन्ति पातयति~। प्रसृतानि गन्तुं प्रवृत्तानि, प्रसृता च जङ्घा~। रजो रागः, धूलिश्च~। कलुषयति कार्याकार्यदर्शनासमर्थो करोति~। दृष्टिं बुद्धिम्, नेत्रं च~। अक्षाणीन्द्रियाणि, रथाङ्गं चाक्षः~। तेन च रथो लक्ष्यते~। कृतबुद्धयः संस्कृतमतयः~। {\qtt असदविद्यमानम्}~। निसर्गः स्वभावः~। आलोको विवेकः, प्रकाशश्च~। तमः, अज्ञानमपि~। दोषाः, सव्यमण्डललादीनि च~। कुपिता क्रुद्धा, धातुवैषम्यदूषिता च~। आत्मरामदोषमिति~। आत्मभूतगुणदर्शनम्, लौहियलक्षणं च विकारम्, {\qt वोढा भारस्य धीमद्भिर्जनैर्वैवधिकः स्मृतः~। दोषैकग्राहिहृदयः पुरोभागी निगद्यते~॥}}

\newpage
% प्रथम उच्छ्वासः~। १३ 

\noindent
आरम्भे तपो गलति, पश्चात्स्वेदसलिलम्~। पूर्वमयशः स्फुरति, अनन्तरमधरः~। कथं लोकविनाशाय ते विषपादपस्येव जटावल्कलानि जातानि~। अनुचिता खल्वस्य मुनिवेशस्य हारयष्टिरिव वृत्तमुक्ता चित्तवृत्तिः~। शैलूष इव वृथा वहसि कृत्रिममुपशमशून्येन चेतसा तापसाकल्पम्~। अल्पमपि न ते पश्यामि कुशलजातम्~। अनेनातिलघिन्नाद्याप्युपर्येव प्रवसे ज्ञानोदन्वतः~। न खल्वनेलमूका: एडा जडा वा सर्व एते सहर्षयः~। रोषदोषनिषद्ये स्वहृदये निग्राह्ये किमर्थमसि निगृहीतवाननागसं सरस्वतीम्~। एतानि तान्यात्मप्रमादस्खलितवैलक्ष्याणि, यैर्याप्यतां यात्यविदग्धो जनः इत्युक्त्वा पुनराह \textendash\ {\haq वत्से सरस्वति विषादं मा गाः~। एषा त्वामनुयास्यति सावित्री~। विनोदयिष्यति चास्मद्विरहदुःखिताम्~। आत्मजमुखकमलावलोकनावधिश्च ते शापोऽयं भविष्यति} इति~। एतावदभिधाय विसर्जितसुरासुरमुनिमनुजमण्डलः ससंभ्रमोपगतनारदस्कन्धविन्यस्तहस्तः समुचिताहिककरणायोतिष्ठन्~। सरस्वत्यपि शप्ता किंचिदधोमुखी धवलकृष्णशारां कृष्णाजिनलेखामिव दृष्टिमुरसि पातयन्ती सुरभिनि:श्वासपरिमललमैर्मूर्तैः शापाक्षरैरिव षट्चरणचकैराकृष्यमाणा शापशोकशिथिलितहस्ताधोमुखीभूतनोपदिश्यमानमर्त्यलोकावतरणमार्गेव नखमयूखजालकेन

\vspace{2mm}
\hrule

\noindent
{\s रागोऽभूतगुणादिनन्दनम्, रक्तता च~। जटाः शिखाः, मूलानि च~। वल्कलानि मुनिवस्त्राणि, त्वचश्च~। वृत्तमुक्ताशीलेन त्यक्ता, परिवर्तुलमौक्तिका च~। {\qt जायोपजीवी हि जनः शैलूषः कथितो बुधैः}~। आकल्पो वेषः~। जातं प्रकारः~। अतिलघिमानु पादेयता तुच्छतमम्~। उपर्ये वेयन्तः प्रवेशाभावात्~। लघु जलोपरि लवते~। {\qt कथिता अनेलमूकाः श्रोतुं वक्तुं च खलु न ये शक्ताः~। एडास्तु श्रुतिहीना जडास्तु मूर्खा बुधैः प्रोकाः}~॥ रोष एवं दोषस्तस्य निषया नियमेनावस्थितिर्यत्र तस्मिन्स्वहृदये ते~। यद्वा रोषदोषस्य निषद्या आपणस्त्वं तस्यामन्त्रणम् हे रोषदोषनिषद्ये इति व्याख्येयम्~। निगृहीतवान्प्राप्तवान्~। {\qt आगः पापापराधयोः}~। वैलक्ष्यं लज्जितम्~। याप्यो गर्हाः~। {\qtt पुनराहेति}~। अविश्रान्तेऽप्युतिकमे पुनरित्युपादानं वाच्यतापरिहाराय~। वत्से इति प्रसादाविष्करणार्थम्~। {\qtt एषेति}~। या तवैव स्निग्धा~। विनोदयिष्यति सुखयिष्यति~। सरस्वतीत्यादौ सरस्वत्यपि शप्ता गृहमगादिति संबन्धः~। शारशबलां धवलकृष्णामित्येव वक्तव्ये शारग्रहणं संवलितवर्णद्वयप्रतीत्यर्थम्~। {\qtt अधोमुखीभूतेनेति}~। योऽधिकरणवशादनिष्ठसुपदिशति}

\newpage
% १४ प्रथम उच्छ्वासः

\noindent
नूपुरव्याहाराहूतैर्भवनकलहंसकुलैर्ब्रह्मलोकनि वासिहृद्यै रिवानुगम्य माना समं सावित्र्या गृहमगात्~।

अत्रान्तरे सरस्वत्यवतरणवार्तामिव कथयितुं मध्यमं लोकमवततारांशुमाली~। क्रमेण च मन्दायमाने मुकुलितत्रिसिनीविसर व्यसनविषण्णसरसि वासरे, मधुमदमुदितकामिनीकोपकुटिलकटाक्षक्षिप्यमाण इव क्षेपीयः क्षितिधरशिखरमवतरति तरुणतरकपिलपनलोहिते लोकैकचक्षुषि भगवति, प्रस्नुतमुखमाहेयीयूथक्षरत्क्षीरधाराधवलितेष्वासन्नचन्द्रोदयोद्दामक्षीरोदलहरीक्षालितेष्विव दिव्याश्रमोपशल्येष्वपराह्नप्रचारचलिते, चामरिणि चामीकरतटताडनरणितरद्ने रदति सुरस्रवन्तीरोधांसि स्वैरमैरावते, प्रसृतानेकविद्याधराभिसारिकासहस्रचरणालक्तकरसानुलिप्त इव प्रकदयति च तारापथे पाटलताम्, तारापथप्रस्थितसिद्धदत्तदिनकरास्तमयार्थ्यावर्जिते रञ्जितककुभि, कुसुम्भभासि स्रवति पिनाकिप्र गतिमुदितसंध्यास्वेदसलिल इव रक्तचन्दनद्रवे, वन्दारुमुनिवृन्दारकवृन्दबध्यमानसंध्याञ्जलिवने, ब्रह्मोत्पत्तिकमलसेवागतसकलकमलाकर इव राजति ब्रह्मलोके, समुच्चारिततृतीयसवनब्रह्मणि ब्रह्मणि, ज्वलितवैतानज्वलनज्वालाजटालाजिरेष्वारब्धधर्मसाधनशिबिरनीराजनेष्विव सप्तर्षिमन्दिरेष्वघमर्षणमुषितकिल्बिषविषगदो\textendash

\vspace{2mm}
\hrule

\noindent
{\s स लज्जादिनावश्यमधोमुखी भवति~। जालकं समूहः~। व्याहार उक्तिः~।

मध्यमं लोकं भूमिम्~। अंशुमाली रविः~। कमेणेत्यादावस्मिन्सति सावित्री सरखतीमवादीदिति संबन्धः~। विसरशब्द औणादिकः षण्डपर्यायः~। मुदिताः संजातमन्मथाः~। कामिन्यः शृङ्गारिण्यः~। संभोगान्तरायकारी कथमयमद्यापि नास्तमेतीत्यतः कोपः~। क्षिप्यमाणञ्चातित्वरितं पत्तति~। क्षेपीयस्तूर्णतरम्~। लपनं वदनम्~। लोकेत्यादिना संभोगविघ्नकारित्वमेव प्रकाश्यते~। माहेयी गौः~। उद्दामः प्रवृद्धिं गतः~। उपशल्यं समीपम्~। चामीकरं सुवर्णम्~। रदना दन्ताः रदति विलिखति~। सुरस्रवन्ती गङ्गा~। रोघस्तटम्~। स्वैरं खेच्छम्~। {\qt या दूतिका गमनकालमपाहरन्ती सोढुं स्मरज्वरभरार्तिपिपासितेव~। नियति बल्लभजनाधरपानलोभात्सा कथ्यते कविवरैरभिसारिकेति~॥} तारापथो नमः~। आवर्जिते प्रकीर्णे~। ककुभो दिशः~। कुसुम्भं पद्मकम्~। रक्तचन्दनद्रवे स्रवति सतीति योजना~। वन्दारु वन्दनशीलम्~। वृन्दारकशब्दः प्रशंसायाम्~। सवनं प्रातर्मध्याह्ने सायं च सोमयागैकदेशः स्नानमित्यन्ये~। ब्रह्म वेदः~। वैतानो यज्ञभवः~। जटालानि व्याप्तानि~। अजिरायङ्गनानि~। आरब्धे धर्मसाधने शिबिरे पुण्योपकरस्कन्धावारे नीराजनाख्यं शान्तिकर्म येषु~। धर्मोपकरणविषये मा दोषाः प्रादुरभवन्निति~। {\qt शमनं सर्वपापानां जप्यं त्रिष्वघ\textendash }}

\newpage
% प्रथम उच्छ्वासः~। १५ 

\noindent
ल्लाघलघुषु यतिषु संध्योपासनासीनतपस्विपङ्किपूतपुलिने प्लवमाननलिनयोनिया नहंं सहा सदन्तुरितोर्मिणि मन्दाकिनीजले, जलदेवतातपत्रे पत्ररथकुलकलत्रान्तःपुरसौधे, निजमधुमधुरामोदिनि कृतमधुपमुदि मुमुदिषमाणे कुमुदवने, दिवसावसानताम्यत्तामरसमधुरमधुसपीतिप्रीते सुषुप्सति मृदुमृणालकाण्डकण्डूयनकुण्डलितकंधरे धुतपक्षराजिनीजितराजीवसरसि राजहंसयूथे, तटलताकुसुमधूलिधूसरितसरिति सिद्धपुरपुरंध्रचम्मिलमल्लिकागन्धमाहिणि सायंतने तनीयसि निशानिःश्वासनिभे नभस्वति, संकोचोदश्चदुच्चकेसरकोटिसंकटकुशेशयकोशकोटरकुटीशायिनि षट्चरणचक्रे, नृत्तोद्धृतधूर्जटिजटाटवीकुटजकुड्यलनिकरनिभे नभस्थलं स्तबकयति तारागणे, संध्यानुबन्धताम्रे परिणमत्तालफलत्वक्त्विषि कालमेघमेदुरे, मेदिनीं मीलयति नववयसि तमसि तरुणतरतिमिरपटलपाटनपटीयसि समुन्मिपति यामिनीकामिनीकर्णपूरचम्पककलिकाकदम्बके प्रदीपप्रकरे, प्रतनुतुहिनकिरणकिरणलावण्यालोकपाण्डुन्याश्याननीलनीरमुक्तकालिन्दीकूलबालपुलिनायमाने शातक्रतवे, कृ\textendash

\vspace{2mm}
\hrule

\noindent
{\s {\qt मर्षणम्}~। गदो रोगः~। उल्लाघं स्वस्थीकरम्~। यतयश्चतुर्थाश्रमिणः~। सद्यो जलत्यक्तं तटं पुलिनम्~। नलिनयोनिर्ब्रह्मा~। हंसानां हासः शौक्ल्यं हंसा एव वा शुक्लतया हासः~। दन्तुरा एव दन्तुरिताः~। ये च सहासास्ते च लक्ष्यमाणदन्तद्वया दन्तुरा इव दृश्यन्ते~। आतपत्रं छत्रम्~। पत्ररथाः पक्षिणः~। कलत्रं दाराः~। मधु मकरन्दः, मयं च~। मधुपा भ्रमराः, मद्यपाश्च~। मुमुदिषमाणे विचकिसिषति~। अन्यत्र मोदितुमिच्छति~। प्रारिप्स्यमानगीतादिगोष्ठीबन्ध इति यावत्~। {\qt मञ्चाः कोशन्ति} इतिवत्~। {\qtt ताभ्यदिति}~। ताम्यन्ति, न तु तान्तानि, प्रदोषस्य न तावत्प्रवृत्तत्वात्~। मधु, मद्यमपि~। सपीतिस्तु सहपानम्~। अनेन तु प्रेमातिशय आवेद्यते~। सुषुप्सति निद्रासति~। {\qtt मृद्विति}~। कण्डूयनं विक्रियाविशेषणम्~। कुण्डलिता चक्रीकृता~। राजीवं पद्मम्~। राजहंसा इयत्रैकशेषः~। तदशब्दः प्रत्यासत्त्युपलक्षणार्थः~। पुरंध्रिरुत्तममहिला~। धम्मिल्लाः संयताः कचाः~। मल्लिका भूपदी~। एषा च सायमेवोन्मिवति~। सायंतने दिनान्तभचे~। कोशः कुडमलम्~। कोटरमभ्यन्तरम्~। कुटी गेहम्~। शयनमंत्र विश्रमणम्, न तु वापः~। पौनरुक्त्यापत्तेः अटवीति विवक्षितम्~। तत्रैवाकृत्रिमकुसुमसंबन्धात्~। कुटजं गिरिमटिका~। कुलं कलिका~। निकरः समूहः~। अनुबन्धः संस्कारः~। परिणमन्जरठीभवत्~। तालस्तुणराजः~। मेदुरं धनम्~। मीलयति स्थगयति~। नववयसि प्रत्य~। चम्पको हेमपुष्पकः~। आश्यानमीषच्छुष्कम्~। नीरें जलम्~। कालिन्दी यमुना~। नीलिमाभिप्रायेणैतत्पदम्~। यस्तटभागो बारिणा व्यक्तस्तत्पुलिनम्~। कूलं ततोऽन्यत्~। कृशयति तनूकुर्वति~।}

\newpage
% १६ हर्षचरिते

\begin{sloppypar}
\noindent
शयति तिमिरमाशामुखे खमुचि मेचकितविकचितकुवलयसरसि शशधरकरनिकरकचमहाविले विलीयमाने मानिनीमनसीव शर्वरीशबरीचिकुरचये चाषपक्षत्विपि तमम्युदिते, भगवत्युदयगिरिशिखरकटककुहरहरिखरनखरनिवहहेतिनिहतनिजहरिणगलितरुधिरनिचयनिचितमिव लोहितं वपुरुदयरागधरमधरमिव विभावरीवध्वा धारयति श्वेतभानौ, अचलद्युतचन्द्रकान्तजलधाराधौत इव ध्वस्ते ध्वान्ते, गोलोकगलित दुग्धविसरवाहिनि दन्तमयमकरमुखमहाप्रणाल इवापूरयितुं प्रवृत्ते पयोधिसिन्दुमण्डले, स्पष्टे प्रदोषसमये सावित्री शून्यहृदयामिव किमपि ध्यायन्तीं सास्रां सरस्वतीमवादीत् \textendash\ सखि, त्रिभुवनोपदेशदानदक्षायास्तव पुरो जिह्वा जिह्वेति मे जल्पन्ती~। जानास्येव यादृश्यो विसंस्थला गुणवत्यपि जने दुर्जनवन्निर्दाक्षिण्याः क्षणभङ्गिन्यो दुरतिक्रमणीया न रमणीया दैवस्य वामा वृत्तयः~। निष्कारणा च निकारकणिकापि कलुषयति मनस्विनोऽपि मानसमसदृशजनादापतन्ती~। अनवरतनयनजलसिच्यमानश्च तरुरिव विपल्लवोऽपि सहस्रधा प्ररोहति~। अतिसुकुमारं च जनं संतापपरमाणवो मालतीकुसुममिव म्लानिमानयन्ति~। महतां चोपरि निपतन्नणुरपि सृणिरिव करिणां क्लेशः कदर्थनायालम्~। सहजस्नेहपाशप्रन्थिबन्धनाश्च बा\textendash
\end{sloppypar}

\vspace{2mm}
\hrule

\noindent
{\s खमुचि त्यक्ताकाशे~। भूभागमवलम्बमान इत्यर्थः~। मेचकितं निर्विभागतां नीतम्~। शशधरकरैः स्वीकारेण करम्बितेऽत एव क्षयं गच्छति~। अन्यत्र चन्द्ररश्मीनां धारणेन सेवनेन किंकर्तव्यतामूढ एवमविगललार्द्रतां भजमाने~। केचपाशपक्षे तु विस्रंसमाने~। चाषः किकीदिविः पक्षी~। हरिः सिंहः~। नखरा नखाः~। हेतिरायुधम्~। विभावरी रात्रिः~। श्वेतभानुचन्द्रः~। अचलोऽर्थादुदयाचलः, मोलोको रश्मिसमूहो चा मकरसुखमिव मुखमममस्येति समासः~। विसंस्थुला निर्मर्यादाः~। दुर्जनवन्निदक्षिण्याः क्रूराः क्षणमङ्गिन्य इत्याश्वासनगर्भेयमुक्तिः~। वामाश्य स्त्रिय ईदृश्य एव~। निकारः परिभवः~। कणिका लेशः, शर्केरिका च~। कलुषयति दूषयति~। काळुष्यं नयति च~। मानसं चेतः, सरव~। अनवरतमत्रुणा सिच्यमानः~। अनवरतं घटसारणीप्रणालादिना नयनं प्रापणं यस्य तादृशा जलेनोदयमाणश्च~। विपल्लव आपल्लेशः, विगतपलवच प्ररोहति स्थिरीभवति~। तरुपक्षे प्ररोहा वि यन्ते यस्य स प्ररोहः, स इवाचरति प्ररोहतीति व्याख्या संतापः खेदः, ऊष्मा च~। मालतीकुसुमं सुमनःपुष्पमतिसुकुमारम्~। महान्त उत्तमाः द्राधीयांसव~। सु रिश मातरोऽपि जन्मभूमयः~। दारुणो विषमः, काष्ठस्य च~। क्रकचः कर\textendash }

\newpage
% प्रथम उच्छ्वासः~। १७ 

\noindent
न्धवभूता दुस्त्यजा जन्मभूमयः~। दारयति दारुणः क्रकचपात इव हृदयं संस्तुतजनविरहः~। सा नार्हस्येवं भवितुम्~। अभूमि: खल्वसि दुःखक्ष्वेडाङ्कुरप्रसवानाम्~। अपि च पुराकृते कर्मणि बलवति शुभेऽशुभे वा फलकृति तिष्ठत्यधिष्ठातरि प्रष्ठे पृष्ठतश्च कोऽवसरो विदुषि, शुचाम्~। इदं च ते त्रिभुवनमङ्गलैककमलममङ्गलभूताः कथमिव मुखमपवित्रयन्त्यश्रुविन्दवः~। तदलम्~। अधुना कथय कतसं भुवो भागमलंकर्तुमिच्छसि~। कस्मिन्नवतितीर्षति ते पुण्यभाजि प्रदेशे हृदयम्~। कानि वा तीर्थान्यनुग्रहीतुमभिलषसि केषु वा धन्येषु तपोवनधामसु तपस्यन्ती स्थातुमिच्छसि~। सज्जोऽयमुपचरणचतुरः सहपांशुक्रीडापरिचयपेशल: प्रेयान्सखीजन: क्षितितलावतरणाय~। अनन्यशरणा चाद्यैव प्रभृति प्रतिपद्यस्व मनसा वाचा क्रियया च सर्वविद्याविधातारं दातारं च श्वःश्रेयसय चरणरजःपवित्रितत्रिदशासुरं सुधासृतिकलिकाकल्पितकर्णावतंसं देवदेवं त्रिभुवनगुरुं त्र्यम्बकम्~। अल्पीयसैव कालेन स ते शापशोकविरतिं वितरिष्यति~। इति~।

एवमुक्ता मुक्तमुक्ताफलधवललोचनजललवा सरस्वती प्रत्यवादीत् \textendash\ प्रियसखि, त्वया सह विचरन्या न मे कांचिदपि पीडामुत्पादयिष्यति ब्रह्मलोकविरहः शापशोको वा~। केवलं कमलासनसेवासुखमार्द्रयति मे हृदयम्~। अपि च त्वमेव वेत्सि मे भुवि धर्मधामानि समाधिसाधनानि योगयोग्यानि च स्थानानि स्थातुम् इत्येवमभिधाय विरराम~। रणरणकोपनीतप्रजागरा चाचानिमीलितलोचनैव तां निशामनयत्~।

\vspace{2mm}
\hrule

\noindent
{\s पत्रम्~। हृदयं चित्तिम्, मध्यं च~। संस्तुतः परिचितः~। {\qtt सेति}~। सर्वनामपदं जानासीत्यादि पूर्वोक्तार्थगर्भीकारेण~। अभूमिरस्थानम्, अक्षेत्रं च~। क्वेडो विषम्~। {\qtt शुभेऽशुभे वेत्यादि}~। सप्रतिपक्षा लोकोक्तिरियम्~। {\qt अव्युत्पन्नमतिः कृतेन न सता नैवासताप्याकुलः}, {\qt गतासूनगतासुंश्च नानुशोचन्ति पण्डिताः} इत्यादिवत्~। अधिष्ठातरि स्वामिनि~। प्रष्ठेऽग्रगामिनि~। अपवित्रता नयन्ति, न तु शोभां त्याजयन्ति~। वामसु स्थानेषु तपस्यन्ती तपश्चरन्ती~। सज्जः प्रगुणः~। आज्ञाकार्ययमिति दृष्टस्वरूपः~। तिःसामान्यविस्रम्भभाजनतामभिव्यनक्ति सखीजनशब्दः~। वःश्रेयसस्य कल्याणश्च दातारम्~। सुधासूतिश्चन्द्रः~। कलिका तरिका~। शापविरतिर्ब्रह्मणैवोक्ता~। अतस्तत्र किमन्यापेक्षयेत्याशङ्कयाह \textendash\ {\qtt अल्पीयसैव कालेनेति}~।

आर्द्रयति स्नेहयति~। धर्मधामानि मध्यदेशादीनि~। समाधिश्चित्तैकाम्यम्~।}

\newpage
% १८ हर्षचरिते 

अपरेधुरुदिते भगवति त्रिभुवनशेखरे खणखणायमानस्खलत्खलीनक्षतनिजतुरगमुखक्षिप्तेन क्षतजेनेव पाटलितवपुष्युदयाचलचूडामणौ जरत्कृकबाकुचूडारुणारुणपुरःसरे विरोचने नातिदूरवर्ती विविच्य पितामहविमानहंसकुलपालः पर्यटन्नपरवकमुञ्चैरगायत् \textendash\ 

\vspace{-2mm}
\begin{quote}
{\ha तरलयसि दृशं किमुत्सुकाम कलुपमानसवासलालिते~।\\
अवतर कलहंसि वापिकां पुनरपि यास्वसि पङ्कजालयम्~॥~२१~॥}
\end{quote}

\vspace{-2mm}
तच्छ्रुत्वा सरस्वती पुनरचिन्तयत् \textendash\ {\haq अहमिवानेन पर्यनुयुक्ता~। भवतु~। मानयामि मुनेर्वचनम्} इत्युक्त्वोत्थाय कृतमहीतलावतरणसंकल्पा परित्यज्य वियोगविक्लवं स्वपरिजनं ज्ञातिवर्गमवगणय्यावगणा त्रिः प्रदक्षिणीकृत्य चतुर्मुखं कथमप्यनुनयनिवर्तितानुयायिव्रतित्राता ब्रह्मलोकतः सावित्रीद्वितीया निर्जगाम~।

ततः क्रमेण ध्रुवप्रवृत्तां धर्मधेनुमिबाधोधावमानधवलपयोधराम्,

\vspace{2mm}
\hrule

\noindent
{\s योगे हि तदुक्तम् \textendash\ {\qt आदौ समाविमासीत पश्चाद्योगमुपाचरेत्} इति~। रणरणको दुःखमरतिकृतम्~।

अपरेद्युरपरस्मिन्नहनि~। एते च कालाः संख्यादयो व्यवहारा इहत्या ब्रह्मलोक उपचरिताः~। शेखर इव शेखरो मुण्डमालकः~। खलीनं कविका~। क्षतजं रक्तम्~। कृकवाकुः कुक्कुटः~। चूडा सांसमयी शेखरिका~। विविच्य विचार्यं~। विमानपालः स्वप्रस्तावे हंसीं यदाह तेन सरस्वती पर्यनुयोजितेयाभूत्~। अपरवऋाख्यं वृत्तमाख्यायिकासु प्रयोज्यम्~। तथा चाह भामहः \textendash\ {\qt वक्रं चापरवक्रं च काव्ये काव्यार्थशंसिनि} इति~। {\qtt तरलयसीत्यादि}~। अकलुषं मानसं यस्य स निर्मलचेता ब्रह्मा, मानसाख्यं च सरः~। लालिता शीलिता~। वापिकापुष्करिणी उप्यन्तेऽस्यां तारण कर्मणीति वापिका मर्त्यभूमिरपि पङ्कजमालयो यस्य स ब्रह्मा, सरक्ष पर्यनुयुक्ता उपपत्त्या बोधिता~। अवगणा केवला सावित्रीव्यतिरेकेण नान्यपरिवारा~। {\qtt कथमपीति}~। न मृत्यादिवत्~। व्रतिव्रातस्तपखिसमूहः~।

तत इत्यादावीदृशं मन्दाकिनीमनुसरन्ती सरस्वती मीलोकमवततारेति संबन्धः~। ध्रुवं नित्यं वियत्~। तस्मात्प्रवृत्तम्~। ध्रुवस्तारकविशेषो ध्रुवन्नित्यस्थानाद्वा विष्णोर्वा ध्रुवाचूरू पश्चाद्भागौ सक्थिनी ध्रुवे वा तयोः प्रकर्षेण वृत्तां परिवर्तुलां वा~। अध इति पदेन धावनक्रियासहत्वाजलस्य ग्रहणं सूच्यते~। अत एव धवलाःशुकाः पयोधरा मेघा यस्यास्ताम्~। इतरत्राधोधावमानाः पयःपूर्णत्वाल्लम्वमानाः क्षीरखतेश्व धवलाः स्तना यस्याः~। अधोधावमानं वेगेन प्रसरद्धवलं पयो धास्यति या तामू, अधोधावमानो धवलो यः पयोधो वत्सस्तं राति ददाति या ताम्, धवलो वृषस्तस्मै पयो धारयति या तां वेत्यादिकाः कुव्याख्या एव~। उद्घुर उद्भटः~। अन्ध\textendash}

\newpage
% प्रथम उच्छ्वासः~। १९ 

\noindent
उद्धुरध्वनिमन्धकमथनमौलिमालतीमालिकाम्, आलीयमानवालखिल्यरुद्धरोधसमरुन्धतीधौततारवत्वचम्, त्वङ्गतुङ्गतरङ्गतरत्तरलतरतारतारकाम्, तापसवितीर्णतरलतिलोदकपुलकितपुलिनाम्, आप्लवनपूतपितामहपातितपितृपिण्डपाण्डुरितपाराम्, पर्यन्तसुप्तसप्तर्षिकुशशयनसूचितसूर्यग्रहसूतकोपवासाम्, आचमनशुचिशचीपतिमुच्यमानार्चनकुसुमनिकरशाराम, शिवपुरापतितनिर्माल्य मन्दारदामकामनादरदारितमन्दरदरीदृषदम्, अनेकनाकनायकनिकायकामिनीकुचकलशविलुलितविग्रहाम्, ग्राहग्रावग्रामस्खलनमुखरितस्रोतसं, सुषुम्णास्रुतशशिसुधाशीकरस्तबकतारकिततीराम्, धिषणाग्निकार्यधूमधूसरितसैकतां, सिद्धविरचितवालुकालिङ्गलङ्घनत्रासविद्रुतविद्याधराम्, निर्मोकमुक्तिमिव गगनोरगस्य, लीलाललाटिकामिव त्रिविष्टपविटस्य, विक्रयवीथीमिव पुण्यपण्यस्य, दत्तार्गलामिव नरकनगरद्वारस्य, अंशुकोष्णीषपट्टिकामिव सुमेरुनृपस्य, दुगूलकदलिकामिव कैलासकुञ्जरस्य, पद्धतिमिवापवर्गस्य, नेमिमिव कृतयुगस्य, सप्तसागरराजमहिषीं मन्दाकिनीमनुसरन्ती मर्त्यलोकमवततार~। अपश्यञ्चाम्बरतलस्थितैव हारमिव वरुणस्य, अमृतनिर्झरमिव चन्द्राचलस्य, शशिमणिनिष्यन्दमिव विन्ध्यस्य, कर्पूरदुमद्रवप्रवाहमिव दण्डकारण्यस्य, लावण्यरसप्रस्रवणमिव दिशाम, स्फाटिकशिलापदृशयनमिवाम्बरश्रियाः, स्वच्छशिशिरसुरसवारिपूर्ण भगवतः पितामहस्यापत्यं हिरण्यवाहनासानं महानदम्, य जना: शोण इति कथ\textendash

\vspace{2mm}
\hrule

\noindent
{\s कमथनः शिवः~। आलीयमानाः श्लिष्यन्तः~। बालखिल्या मुनिभेदाः~। रोधस्तदम्~। त्वचरत्~। आठवनं स्नानम्~। पितरो देवविशेषाः, आज्यपाः, सोमपाः, बर्हिषघश्च~। आचमनेत्यादिना पितामहवन्न स्नानादिनिष्ठात्वमस्योच्यते~। अत एव शचीपदेन संभोगासक्तत्वमिव पोषितम्~। निकायः समूहः~। सुषुम्णाख्योऽमृतमयो रविरश्मिः~। घिषणो बृहस्पतिः~। सिद्धकृतत्वेन लिङ्गेषु भगवत्संनिधानमावेद्यते~। निर्मोकः सर्पकञ्जुकः~। विसंसतया शुक्लत्वेन लहरिकावलीत्वेन च निर्मोकमुक्तिमिवेत्युप्रेक्षा~। गगनमिवोरगः कृष्णतया~। ललाटिका ललाटालंकारः~। विटो भुजङ्गः~। उष्णीषं शिरोवेष्टनं दिक्षु प्रसिद्धम्~। दुगूलशब्दो दुकूलसमानार्थः~। पद्धतिमार्गः~। अपवर्गों मोक्षः~। कृतयुगस्य रचितयुगकाष्ठस्य रथस्येत्यर्थः~। यथा नेमिवशाद्रयग्रहणं तथा तद्वशात्कृताख्यस्य युगस्य~। सप्तसागरराजः क्षीरसमुद्रः~। चन्द्राख्यः पर्वत इति केचित्~। शशिमणिश्चन्द्रकान्तः~। पितामहस्येति~। तद्भक्त्या तदाश्रयणम्~। तिकता विद्यन्ते यस्य स सिकतिलः~। मत्तशब्देन सशब्दत्वम्, वेणीपदेन च तन्त्रीसंनिवेश\textendash}

\newpage
% २० हर्षचरिते 

\noindent
यन्ति~। दृष्ट्वा च तं रामणीयकहृतहृदया तस्यैव तीरे वासमरचयत्~। उवाच च सावित्रीम् \textendash\ {\haq सखि, मधुरमयूरविरुतयः कुसुमपांशुपटलसिकतिलतरुतलाः परिमलमत्तमधुपवेणीवीणारणितरमणीया रमयन्ति मां मन्दीकृतमन्दाकिनीद्युतेरस्य महानदस्योपकण्ठभूमयः~। पक्षपाति च हृदयमत्रैव स्थातुं मे} इति~। अभिनन्दितवचना च तथेति तथा तस्य पश्चिमे तीरे समवातरत्~। एकस्मिंश्च शुचौ शिलातलसनाथे तटलतामण्डपे गृहबुद्धिं बबन्ध~। विश्रान्ता च नातिचिरादुत्थाय सावित्र्या सार्धमुचितार्चनकुसुमा सस्त्रौ~। पुलिनपृष्ठप्रतिष्ठितसैकतशिवलिङ्गा च भक्त्या परमया पञ्चब्रह्मपुरःसरां सम्यङ्युद्राबन्धविहितपरिकरां ध्रुवागीतिगर्भामवनिपवनवनगगनदहनतपनतुहिनकिरणयजमानमयीमूर्तीरष्टावपि ध्यायन्ती सुचिरमष्टपुष्पिकामदात्~। अयत्नोपनतेन फलमूलेनामृतरसमप्यतिशिशयिषमाणेन च स्वादिना शिशिरेण शोणवारिणा शरीरस्थितिमकरोत्~। अतिवाहितदिवसा च तस्मिल्लतामण्डपशिलातले कल्पितपलवशयना सुष्वाप~। अन्येद्युरप्यनेनैव क्रमेण नक्त॑दिनमत्यवाहयत्~।

एवमतिक्रामत्सु दिवसेषु गच्छति च काले याममात्रोद्गते च रवावुत्तरस्यां ककुभि प्रतिशब्दपूरितवनगहरं गम्भीरतारतरं तुरसङ्गहेषितहादमशृणोत्~। उपजातकूतूहला च निर्गत्य लतामण्डपाद्विलोकयन्ती विकचकेतकीगर्भपञ्चपाण्डुरं रजःसंघातं नातिदवी\textendash

\vspace{2mm}
\hrule

\noindent
{\s सादृश्यमाह~। वेणी पङ्क्तिः~। लिङ्ग्यतेऽनेनेति लिङ्गमाकारः~। पञ्च ब्रह्माणि सद्योजातः, वामदेवः अघोरः, तत्पुरुषः, ईशानश्चेति~। मुद्रावन्धो विशिष्टः कराङ्गुलिसंनि वेशः~। ध्रुवाख्या विशिष्टा गीतिः~। वनं तोयम्~। यजमान उमः~। अञ्चै पुष्पाप्येवाटपुष्पिका~। तत्र प्रभृति गन्धप्रधानं पार्थिवम्, अर्घस्नानादिकं रसप्रधानमाप्यम्, प्रदीपा भरणप्रभादिरूपग्रवानं तैजसम्, अनुलेपनप्रभृति प्रधानं वायवीयम्, सुषिरातोद्यगीतादिकं शब्दप्रधानमाकाशीयम्, अनुष्यानं मानसम्, अस्ति सर्वत्रैवेश्वर इति निश्चयो बौद्धम्, अहमेवेश्वर इत्याहंकारिकम्~। यद्वा आसनवर्गप्रभृतिष्वष्टसु प्रत्येकमटपुष्पिका~। अतिशेतुमभिभवितुमिच्छतातिशिशयिषमाणेन~। स्वादिम्ना मृष्टत्वेन~। {\qtt शरीरस्थितिमिति}~। न त्वातृप्तिभोजनम्~। अन्बेद्युरन्यस्मिन्नहनि~।

यामः प्रहरः~। नऊंदिनशब्देन तत्कालनिवर्तनीय कर्मैव लक्ष्यते~। गम्भीरश्चिरकालस्थितः~। तारतरो दूरदेशश्रयमाणः~। हेषितमश्वशब्दः, तद्रूपो ह्रादो ध्वनिस्तम्~। क्रमेणेत्वादावश्ववन्दं संददर्शेति संबन्धः~। शफरा मत्स्याः~। तदुदरक्त्तैश्च धूसरे~। प्रलम्बेत्यादिना सज्जत्वमुक्तम्~। कचाः केशाः~। सौकुमार्यात्मवानीव~। घ\textendash }

\newpage
% प्रथम उच्छ्वासः~। २१ 

\noindent
यसि संमुखमापतन्तमपश्यत्~। क्रमेण च सामीप्योपजायमानाभिव्यक्ति तस्मिन्महति शफरोदरधूसरे रजसि पयसीव मकरचक्रं प्लवमानं पुरः प्रधावमानेन, प्रलम्बकुटिलकचपल्लवघटितललाटजूटकेन, धवलदन्तपत्रिकाद्युतिहसितकपोलभित्तिना, पिनद्धकृष्णागुरुपङ्ककल्कच्छ्ररणकृष्णशबलकषायक चुकेन, उत्तरीय कृतशिरोवेष्टनेन, वामप्रकोष्ठनिविष्टस्पष्टहाटककटकेन, द्विगुणपट्टपट्टिकागाढप्रन्थिप्रथितासिधेनुना, अनवरतव्यायामकृशकर्कशशरीरेण, वातहरिणयूथेनेव मुहुर्मुहुः खमुड्डीयमानेन, लङ्घितसमविपमावटविटपेन, कोणधारिणा, कृपाणपाणिना, सेवागृहीतविविधवनकुसुम फलमूलपर्णेन, {\haq चल चल, याहि याहि, अपसर्पापसर्प, पुरः प्रयच्छ पन्थानम्} इत्यनवरतकृतकलकलेन युवप्रायेण, सहस्रमात्रेण पदातिबलेन सनाथमश्ववृन्दं संददर्श~।

मध्ये च तस्य सार्धचन्द्रेण मुक्ताफलजालमालिना विविधरत्नखण्डखचितेन शङ्खक्षीरफेनपाण्डुरेण क्षीरोदेनेव स्वयं लक्ष्मीं दातुमागतेन गगनवतेनातपत्रेण कृतच्छायम्, अच्छाच्छेनाभरणयुतीनां निवहेन दिशामिव दर्शनानुरागलग्नेन चक्रवालेनानुगम्यमानम्, आनितम्बविलम्बिन्या मालतीशेखरखजासकलभुवनविजयार्जितया रूपपताकयेव विराजमानम्, उत्सर्पिभिः शिखण्डखण्डिकापद्मरागमणेररुणैरंशुजालैरदृश्यमानवनदेवतावितैबलपवैरिव प्रमृज्यमानमार्गरेणुपरुषवपुषम्, बकुलकुडालमण्डलीमुण्डमालामण्डनमनोहरेण कुटिलकुन्तलस्तवकमालिना मौलिना मीलितातपं

\vspace{2mm}
\hrule

\noindent
{\s टितललाटजूटता दाक्षिणात्येषु वेशः~। दन्तपत्रिका कर्णाभरणभेदः~। पिनद्धो बद्धः~। कृष्णागुरुणः पङ्को निर्घृष्टं कृष्णागुरुः, तस्य शक्कस्य सत्तः कल्कचूर्णः, तच्छुरणात्कृष्णेन गुणेन शबलं कषायं साधिवासितं कञ्चकं चारवाणं यस्य~। उतरीयेत्यादिना संनद्धतां वर्मादिप्रसङ्गं चाह~। वामेत्यनेनाश्रमिस्वभाववर्णना शृङ्गारिता चोक्ता~। {\qt प्रकोष्ठमन्तरं विद्यादरलिमणिबन्धयोः}~। हाटकं स्वर्णम्~। यदेव द्विगुणात एव गाढग्रन्थिसहत्वम्~। ग्रथिता विसंतिनी~। असिधेनुछुरिका~। वातहारिणो यो वाताभिमुखं धावति~। अवट उन्मार्गः~। कोणो लगुडः~।

मध्य इत्यादौ तस्य च मध्येऽष्टादश वर्षदेशीयं युवानमद्राक्षीदिति संबन्धः~। क्षीरोदत्याप्यर्धचन्द्रादि सर्वे योज्यम्~। छाया कान्तिरपि~। चक्रवाळेन समूहेन~। नितम्बशब्दो मुख्यार्थः~। {\qt पश्चान्नितम्बः स्त्रीकव्याः} इत्यर्थः~। शिखण्डखण्डिकां चूडाभरणम्~। {\qtt प्रमृज्यमानेति}~। वर्तमानकालोऽत्र विवक्षितः~। बकुलेत्यादिना}

\lfoot{ह० ३}

\newpage
\lfoot{}
% २२ हर्षचरिते 

\noindent
पिबन्तमिव दिवसम्, पशुपतिजटामुकुटमृगाङ्कद्वितीयशकलघटितस्येव सहजलक्ष्मीसमालिङ्गितस्य ललाटपट्टस्य मनःशिलापकपिङ्गलेन लावण्येन लिम्पन्तमिवान्तरिक्षम्, अभिनवयौवनारम्भावष्टम्भप्रगल्भदृष्टिपाततृणीकृतत्रिभुवनस्य चक्षुषः प्रथिन्ना विकचकुमुदकुवलयकमलसरःसहस्रसंछादितदशदिशं शरदमिव प्रवर्तयन्तम्, आयतनयननदीसीमान्तसेतुबन्धेन ललाटतटशशिमणिशिलावलगलितेन गान्तिसलिलस्रोतसेव द्वाघीयसा घोणावंशेन शोभमानम्, अतिसुरभिसहकारकर्पूरकक्कोललवपारिजातकपरिम लमुचा मत्तमधुकरकुलकोलाहलमुखरेण मुखेन सनन्दनवनं वसन्तमिव वसन्तम्, आसन्नसुहृत्परिहासभावनोत्तानितमुखमुग्धहसितैर्दशनज्योत्स्नास्नपितदिङ्युखैः पुनःपुनर्नभसि संचारिणं चन्द्रालोकमिव कल्पयन्तम्, कदम्बमुकुलस्थूलमुक्ताफलयुगलमध्याध्यासितमरकतस्य त्रिकण्टककर्णाभरणस्य प्रेशतः प्रभया समुत्सर्पन्त्या कृतसकुसुमहरितकुन्दपल्लवकर्णावतंसमिवोपलक्ष्यमाणम्, आमोदितमृगमदपङ्कलिखितपत्रभङ्गभाखरम्, भुजयुगलमुद्दाममकराक्रान्तशिखरमिव मकरकेतुकेतुदण्डद्वयं दधानम्, धवलब्रह्मसूत्रसीमन्तितं सागरमथनसामर्पगङ्गास्रोतःसंदानितमिव मन्दरं देहमुद्वहन्तम्, कर्पूरक्षोदमुष्टिच्छुरणपांशुलेनेव कान्तोच्चकुचचक्रवाकयुगलविपुलपुलिनेन्नेरःस्थलेन स्थूलभुजायामपुञ्जितम्, पुरो विस्तारयन्तमिव दिक्चक्रम्, पुरस्तादीषधोनाभिनिहितैककोणकमनीयेन पृष्ठतः कक्ष्याधिकक्षिप्तपल्लवेनोभयतः संवलनप्रकटितोरुत्रिभागेन हारीतहरिता निविडनिपीडितेनाधरवाससा विभज्यमानतनुतरमध्यभागम्, अनवरतश्रमोपचितमांसकठिनविकटमक\textendash

\vspace{2mm}
\hrule

\noindent
{\s निपीयमानातपतुल्यवस्तुनिर्देशः~। कुन्तलः केशहस्तः स एव स्तबकः~। पुष्पस्तबकः पुष्पसंघातः~। सहजाकृत्रिमा, सहोत्पन्ना च~। लक्ष्मीः शोभना, श्रीथ~। लावण्यमत्र कान्तिः~। अवष्टम्भो गर्वः~। द्राधीयसा दीर्घतरेण~। सहकारः सुगन्धद्रव्यभेदः सहकारफलेनेव क्रियते~। पारिजातकोऽनेकद्रव्यसंस्कृतः मुखवासविशेषः, देववृक्षश्च~। वसन्तथैवविधेनैव मुखेन प्रारम्भेनोपलक्षितो भवति~। रत्नत्रितयेन कृतं त्रिकोणकण्टकाख्यं कर्णाभरणम्~। मृगमदः कस्तूरिका~। संदानितं बद्धम्~। वेष्टितमित्यर्थः~। कुचावत्र कान्ता संबन्धिनावेव~। चक्रवाकयुगलं तस्य कृते पुलिनसदृशम्~। कोणः पल्लवः पृष्ठतः पश्चाद्भागे कक्ष्यायाः परिवलनादधिकश्रुतिरित्युक्तः~। क्षिप्तो लम्बमानः पलवो यस्य तत्~। संवलनं संकोचनम्~। हारीतः पक्षिभेदः~। हरिता नीलेन~।}

\newpage
% प्रथमं उच्छ्वासः~। २३ 

\noindent
रमुखसंलग्नजानुभ्यां विशालवक्षःस्थलोपलवेदिकोत्तम्भनशिलास्तम्भाभ्यां चारुचन्दनस्थासकस्थूलकान्तिभ्यामूरुदण्डाभ्यामुपहसन्तमिवैरावतकरायामम्, अतिभरितोरूभारवहनखेदेनेव तनुतरजङ्घाकाण्डं कल्पपादपपल्लवद्वयस्येव पाटलस्योभयपार्श्वावलम्बिनः पादद्वयस्य दोलायमानैर्नखमयूखैरश्वमण्डनचामरमालामिक रचयन्तम्, अभिमुखमुचैरुद श्चद्भिरतिचिरमुपरिविश्रान्यद्भिरिव वलितविकटम्, पतद्भिः खुरैः खण्डितभुवि प्रतिक्षणदशनविमुक्तखणखणायितखरखलीने दीर्घव्राणलीनलालिकललाटलुलितचारुचामीकरचक्रके शिखानशातकौम्भजयनशोभिनि मनोरंहसि चामीकरचक्रके शिखानशातकौम्भजयनशोभिनि मनोरंहसि गोलाङ्गलकपोलकालकायलोन्नि नीलसिन्धुवारवर्णे वाजिनि महति समारूढम्, उभयतः पर्याणपट्टलिष्टहस्ताभ्यामासन्नपरिचारकाभ्यां दोधूयमानधवलचामरिकायुगलम्, अग्रतः पठतो बन्दिनः सुभाषितमुत्कण्टकितकपोलफलकेन लग्नकर्णोत्पलकेसरपक्ष्मशक लेनेव मुखशशिना भावयन्तम्, अनङ्गयुगावतारमिव दर्शयन्तम्, चन्द्रमयीमिव सृष्टिमुत्पादयन्तम्, विलासप्रायमिव जीवलोकं जनयन्तम्, अनुरागमयमिव मार्गान्तरमानयन्तम्, शृङ्गारमयमिव दिवसमापादयन्तम्, रागराज्यमिव प्रवर्तयन्तम्, आकर्षणाञ्जनमिव चक्षुषोः, वशीकरणमन्त्रमिव मनसः, स्वस्थावेशचूर्णमिवेन्द्रियाणाम्, असंतोषमिव कौतुकस्य, सिद्धयोगमिव सौभाग्यस्य,

\vspace{2mm}
\hrule

\noindent
{\s मकरमुखं जानुनोरुपरिभागः~। उत्तम्भनं धारणम्~। स्थासकश्चन्द्रकः~। आयामो दैर्घ्यम्~। न केवलमायामं शुक्कत्वमप्युपहसन्तम्~। धर्मयोरेकनिर्देशोऽन्यसंवित्साहचर्यात्~। {\qt अतिभरितोरूभारवहनेन} इति पाठः~। ऊरू एवं भारः~। प्रशस्ता जंघा जंघाकाण्डम्~। कल्पपादपसंबन्धितया न केवलं लौहित्यं सौकुमार्याद्युच्यते~। यावत्सकलसंपत्फलप्रदत्वादिप्रकर्षान्तरम्~। अतिचिरमित्यादिनानाकुललमुच्यते~। यदुक्तम् \textendash\ {\qt आवृता कुञ्चिता स्थूलदलपाल्यग्रसंस्थिताः~। विवर्ज्याचाकुलपदन्यासेन गमनेन च~॥} इति~। विकटं चित्रम्~। {\qtt खुरैरिति}~। तयापारवैचित्र्याहुत्वमग्रिमयोरेव~। एवंविवसंनिवेशसंभवात्~। खलीनं कविका~। लालिका कविकाशेखरम्~। जयनं हयमण्डनमाला~। गोलाङ्गूलः कृष्णमुखो वानरः~। नीलेयादौ कुमुदकुन्दनृणालगौर इलादिवन्न पौनरुक्त्यम्~। {\qtt महतीति}~। उक्तं च \textendash\ {\qt सर्वलक्षणहीनोऽपि महाकायः प्रशस्यते} इति~। आसन्नेत्यनेन विश्वसनीयत्वमुक्तम्~। {\qtt अनङ्गयुगेति}~। अनङ्गजन्मना यदुपलक्षितं युगं कालविशेषस्तस्य नूतनमदनसादृश्यात्~। यद्वा अनङ्गयोयुगं तदवतारमिव~। द्वित्वसंख्यापूर्वकत्वात्~। चन्द्रमयीमिवेति कान्तिमयत्वेन~। आकर्षणाअनं वशीकरणार्थं कज्जलम्~। {\qtt असंतोषमिवेति}~। यस्यैनं प्राप्य कौतुकं न निवर्तते}

\newpage
% २४ हर्षचरिते 

\noindent
पुनर्जन्म दिवसभिव मन्मथस्य, रसायनमिव यौवनस्य, एकराज्य मिव रामणीयकस्य, कीर्तिस्तम्भमिव रूपस्य, मूलकोपभिव लावण्यस्य, पुण्यकर्मपरिणाममिव संसारस्य, प्रथमाङ्करमिव कान्तिलतायाः, सर्गाभ्यासफलमिव प्रजापतेः; प्रतापमिव विभ्रमस्य, यशःप्रवाहमिव वैदध्यस्य, अष्टादशवर्षदेशीयं युवानमद्राक्षीत्~।

पार्श्वे च तस्य द्वितीयमपरसंश्लिष्टतुरङ्गम्, प्रांशुमुत्तप्ततपनीयस्तम्भाकारम्, परिणतवयसमपि व्यायामकठिनकायम्, नीचनखश्मश्रुकचम्, शुक्तिखलतिम्, ईषत्तुन्दिलम्, रोमशोरःस्थलम्, अनुल्बणोदारवेशतया जरामपि विनयमिव शिक्षयन्तम्, गुणानपि गरिमाणमिवानयन्तम्, महानुभावतामपि शिष्यतामिवानयन्तम्, आचारस्याचार्यकमिव कुर्वाणम्, धवलवारवाणधारिणम्, धौतदुकूलपट्टिकापरिवेष्टितमौलिं पुरुषम्~।

अथ स युवा पुरोयायिनां यथादर्शनं प्रतीत्य विस्मितमनसां कथयतां पदातीनां सकाशादुपलभ्य दिव्याकृति तत्कन्यायुगलमुपजातकुतूहलः प्रतूर्णतुरगो दिदृक्षुस्तं लतामण्डपोद्देशमाजगाम~। दूरादेव च तुरगादवततार~। निवारितपरिजनश्च तेन द्वितीयेन साधुना सह चरणाभ्यामेव सविनयमुपससर्प~। कृतोपसं\textendash

\vspace{2mm}
\hrule

\noindent
{\s तस्य संतोष एव नास्ति~। केषांचिदेव द्रव्याणां संबन्धी यो न कदाचित्कार्ये व्यभिचरति स सिद्धयोगः~। सौभाग्यं तावत्सर्वे किंचन वशीकुरुते, एवं चास्य तदेव सिद्धयोग इव तदाश्रयणतन्निःशेषलोकवशीकरणक्षमत्वम्~। जन्मदिवसमिति तद्गो चरपतितानां कामोत्पत्तेः~। {\qtt रसायनमिवेति}~। यथा रसायनवशात्कचित्परिपूर्णश्च स्थिरश्च भवति, तदेतदाश्रयेण यौवनम्~। ईषदसमाप्तोऽटादशवर्षोऽष्टाद शवर्षदेशीयस्तम्~। न परेण संस्तुरो यस्य तम्~। दधीचस्य तु पर्याणश्लिष्टा द्युक्तम्~। परिणतवयस्त्वेन सत्यवादिना सावित्रीसरस्वत्यौ प्रति च विनम्भकारित्वमुच्यते~। अन्यथोपक्रम एव संभाषणमात्रं न प्रवर्तते~।

शुक्तिखलतिं शुक्लाकारखल्वाटम्~। तुन्दिलं लम्बोदरम्~। अत एवास्य विकुक्षिरिति नाम~। अनुत्बणोऽनुद्धतः~। उदारः श्रेष्ठः~। {\qtt जरामिति}~। जरा किल सर्व विनयं शिक्षयति~। महानुभावता महाशयता अनुभावयति कार्यमकार्य वा बोधयतीत्यनुभावः~। {\qtt शिष्यतामिति}~। परशासनदक्षकर्म महानुभावतया तत एवावसीयत इत्युक्तं भवति~। आचारः शास्त्रकारप्रदर्शिता विशिष्टा नीतिः~। स च सर्वस्मिन्नाचार्यक्रमवलम्बते संस्कारातिशय मापादयतीत्यर्थः~। वलक्षः शुकः~। वारबाणः कक्षुकः~। मौलेयाः केशाः~।

{\qtt अथेति}~। नतु गतागतिकतया सर्वचेतनाभिप्रायेण सौन्दर्यमेतयोरभिव्यज्यते~। प्रतीत्य न पुनः प्रसङ्गत उपेत्य~। कन्यकात्वादेतन्नानुचितम्~। प्रतुर्णो वेगगामी~।}

\newpage
% प्रथम उच्छ्वासः~। २५ 

\noindent
ग्रहणौ तौ सावित्री समं सरस्वत्या किसलयासनदानादिना सकुसुमफलार्थ्यावसानेन वनवासोचितेनातिथ्येन यथाक्रममुपजग्राह~। आसीनयोश्च तयोरासीना नातिचिरमिव स्थित्वा तं द्वितीयं प्रवयसमुद्दिश्यावादीत् \textendash\ आर्य, सहजलजाधनस्य प्रमदाजनस्य प्रथमाभिभाषणमशालीनता, विशेषतो वनमृगीमुग्धस्य कुलकुमारीजनस्य~। केवलमियमालोकनकृतार्थाय चक्षुषे स्पृहयन्ती प्रेरयत्युदन्तश्रवणकुतूहलिनी श्रोत्रवृत्तिः~। प्रथमदर्शने चोपायनमिवोपनयति सज्जनः प्रणयम्~। अप्रगल्भमपि जनं प्रभवता प्रश्रयेणार्पितं मनो मध्विव वाचालयति~। अयत्नेनैव चातिनम्रे साधौ धनुषीव गुणः परां कोटिमारोपयति विस्रम्भः~। जनयन्ति च विस्मयमतिधीरधियामदृष्टपूर्वा दृश्यमाना जगति स्रष्टुः सृष्ट्यतिशयाः यतस्त्रिभुवनाभिभावि रूपमिदमस्य महानुभावस्य~। सौजन्यपरतत्रा चेयं देवानांप्रियस्यातिभद्रता कारयति कथां न तु युवतिजने सहोत्था तरलता~। तत्कथयागमनेनापुण्यभाकतमो विजृम्भितविरहव्यथः शून्यतां नीतो देशः~। क्व वा गन्तव्यम्~। कस्य वामपहुंकाराहंकारोऽपर इवानन्यजो युवा~। किंनाम्नः समृद्धतपसः पितुरयममृतवर्षी कौस्तुभमणिरिव हरेर्हृदयमाहादयति~। का चास्य त्रिभुवननमस्या प्रभातसंध्येव महतस्तेजसो जननी~। कानि वास्य पुण्यभाजि भजन्त्यभिख्यामक्षराणि~।

\vspace{2mm}
\hrule

\noindent
{\s साधुना विनीतेन~। {\qt उपसंग्रहणं धीराः कथयन्त्यभिवादनम्}~। आतिथ्यमेवोपजग्राहापूजयत्~। {\qt प्रवयाः स्यात्परिणतः}~। अशालीनता वृष्टता~। वनशब्देन मृगीसामान्येऽपि जनसंपर्कायभावमाह~। उपायनं ढौकनिका~। उपनयति ढौकयति~। {\qtt प्रगल्भमित्यादि}~। मनः कर्तृ अप्रगल्भमपि जनं वाचालयति~। कीदृशम्~। प्रभवता स्वामिना प्रश्रयेण प्रत्यर्पितं दत्तमेवंविधमस्मदीयं युष्मासु मन इति बहिःप्रकाशित यश्च परतश्च केनापि प्रभावशीलेन ढौकितं मध्वप्रगल्भमपि जनं कुलयोषिप्रायं वाचालयति किंचन जल्पयति~। अत्रापि प्रश्रयेणेवि साभिप्रायम्~। तथा च \textendash\ {\qt अन्यथान्यवनितागतचित्तं चित्तनाथमभिशक्तिवत्या~। पीतभूरिसुस्यापि न मोदे निर्वृतिर्हि मनसो मदहेतोः~॥} इत्युक्तम्~। नम्रे प्रह्वे, कुब्जे च~। गुणो विनयादिः, ज्या च~। कोटिः प्रकर्षः, धनुः शिखा च देवानांप्रियस्येति पूजावचनम्~। षष्ठ्या अलुक्~। अत्रागमनेत्यादिना ब्रह्मोक्तशापबुद्ध्या दवीचस्य तद्भर्तृयोग्यतया कतम इति देशोत्कर्षकुलादिकं पृच्छति~। {\qtt कस्येति}~। देवस्य~। सिद्धा देवाः~। अनन्यतः कामः~। {\qtt महतस्तेजस इति}~। महव तेजः सूर्याख्यम्~। अभिख्या नाम~। {\qtt अयमेव}}

\newpage
% २६ र्हर्षचरिते 

\noindent
आर्यपरिज्ञानेऽप्ययमेव क्रमः कौतुकानुरोधिनो हृदयस्य इत्युक्तवत्यां तस्यां प्रकटितप्रश्रयोऽसौ प्रतिव्याजहार \textendash\ आयुष्मति, सतां हि प्रियंवदता कुलविद्या~। न केवलमाननं हृदयमपि च ते चन्द्रमयमिव सुधाशीकरशीतलैरानन्दयति बचोभिः~। सौजन्यजन्मभूमयो भूयसा शुभेन सज्जननिर्माणशिल्पकला भवादृश्यो जायन्ते~। दूरे तावदन्योन्यस्यालापनमभिजातैः सह दृशोऽपि मिश्रीभूता महत भूमिमारोपयन्ति~। श्रूयताम् \textendash\ अयं खलु भूषणं भार्गववंशस्य भगवतो भूर्भुवःस्वस्त्रितयतिलकस्य, अद्भ्रप्रभावस्तम्भितजम्भारिभुजस्तम्भस्य, सुरासुरमुकुटमणिशिलाशयनदुर्ललितपादपङ्केरुहस्य, निजतेजःप्रसरमुद्रपुलोन्नश्चयवनस्य बहिवृत्तिजीवितं दधीचो नाम तनयः~। जनन्यस्य जितजगतोऽनेकपार्थिवसहस्रानुयातस्य शर्यातस्य सुता राजपुत्री त्रिभुवनकन्यारत्नं सुकन्या नाम~। तां खलु देवीमन्तर्वनीं विदित्वा वैजनने मासि प्रसवाय पिता पत्युः पार्श्वात्स्वगृहमानाययत्~। असूत च सा तत्र देवी दीर्घायुषमेनम्~। अनेहसावर्धत तत्रैवायमानन्दितज्ञातिवर्गो बालस्तारकराज इव राजीवलोचनो राजगृहे~। भर्तृभवनमागच्छन्त्यामपि दुहितरि नासेचनकदर्शनमिमममुञ्चन्मातामहो मनोविनोदनं नतारम्~। अशिक्षतायं तत्रैव सर्वा विद्याः सकला कलाः~। कालेन चोपारूढयौवनमिममालोक्याहमिवासावप्यनुभवतु मुखकमलावलोकनानन्दमस्येति मातामहः कथंकथमप्येनं पितुर\textendash

\vspace{2mm}
\hrule

\noindent
{\s {\qtt क्रम इति}~। यथास्योत्पत्त्यादिकं तद्ब्रद्भवतोऽपीत्यर्थः~। कला उपायः~। भूरिति रेफान्तो भूवाची~। भुव इति रेफान्तः पातालवाची~। भूख भुवध स्वश्च भूर्भुवःखः~। एषां त्रयमिति समासः~। अदम्रोऽनल्पः~। जम्मारिरिन्द्रः~। स ह्यश्विभ्यां यज्ञभागभुजौ कुर्बावामिति चिरं प्रार्थितः~। तथेति प्रतिपद्ये ताभ्यां भागं दददिन्द्रेणोद्यतवज्रेण रोषितः~। ततस्तेनास्य सवज्रः स्तम्भितो भुज इति~। दुर्ललितोऽलभ्यविषयः~। {\qtt प्लुष्टपुलोम्न इति}~। अनवरतं रुदल्यां दुहितरि कोपान्मात्रा गृहाणेमामिति पुलोम्नो राक्षसस्योक्तम्~। ततखां प्रतिगृह्य तत्रैव स्थापयित्वा क्वापि गते रक्षसि सा भृगुणा विवाहिता~। ततः सगर्भा सती पुलोम्नागलापह्रियमाणतया च्यवनं गर्भमत्याक्षीत्~। तेन चान्वर्थनान्ना तद्रक्षो दृष्ट्यैवादह्यत~। अन्तर्वत्नीं गर्भिणीम्~। वैजनने मासि प्रसवमासे~। दीर्घायुषमिति साभित्रायम्~। रूपकुलाद्युत्कर्षे वर्णिते सत्येतदेव वरगुणवर्णनमवशिष्यते~। अनेहसा परिपूर्णेन कालेन~। {\qt न जायते यत्र तृप्तिस्तदासेचनकं विदुः}~। नप्तारं पौत्रम्~।}

\newpage
% प्रथम उच्छ्वासः~। २७ 

\noindent
न्तिकमधुना व्यसर्जयत्~। मामपि तस्य देवस्य सुगृहीतनान्नः शर्यातस्याज्ञाकारिणं विकुक्षिनामानं भृत्यपरमाणुमवधारयतु भवती~। पितुः पादमूलमायान्तं मया साभिसारमकरोत्स्वामी~। तद्धि नः कुलक्रमागतं राजकुलम्~। उत्तमानां च चिरंतनता जनयत्यनुजीविन्यपि जने कियन्मात्रमपि मन्दाक्षम्~। अक्षीणः खलु दाक्षिण्यकोशो महताम्~। इतच गव्यूतिमात्रमिव पारेशोणं तस्य भगवतव्यवनस्य स्वनाम्ना निर्मितव्यपदेशं च्यावनं नाम चैत्ररथकल्पं काननं निवासः~। तदवधिश्वेयं नौ यात्रा~। यदि च गृहीतक्षणं दाक्षिण्यमनवहेलं वा हृदयमस्साकमुपरि भूमिर्वा प्रसादानामयं जनः श्रवणार्हो वा, ततो न विमाननीयोऽयं नः प्रथमः प्रणयः कुतूहलस्य~। वयमपि शुश्रूषवो वृत्तान्तमायुष्मत्योः~। नेयमाकृतिर्दिव्यतां व्यभिचरति~। गोत्रनामनी तु श्रोतुम मिलषति नौ हृदयम्~। तत्कथय कतमो वंशः स्पृहणीयतां जन्मना नीतः~। का चेयमत्रभवती भवत्याः समीपे समवाय इव विरोधितां पदार्थानाम्~। तथा हि~। सनिहितबालान्धकारा भास्वन्मूर्तिश्च, पुण्डरीकमुखी हरिणलोचना च, वालातपप्रभाधरा कुमुदहासिनी च, कलहंंसखना समुन्नतपयोधरा च, कमलकोमलकरा हिमगिरिशि लापृथुनितम्बा च, करभोरुर्विलम्बितगमना च, अमुक्तकुमारभावा स्निग्धतारका च इति~। सा त्ववादीत् \textendash\ आर्य, श्रोष्यसि कालेन~। भूयसो दिवसानत्र स्थातुम भिलषति नौ हृदयम्~। अल्पीयञ्चायमध्वा~। परिचय एव प्रकटीकरिष्यति~। आर्येण न विस्मरणीयोऽयमनुषङ्ग\textendash

\vspace{2mm}
\hrule

\noindent
{\s साभिसारं ससहायम्~। मन्दाक्षमुपरोधम्~। गव्यूतिः कोशद्वयम्~। यात्रा प्रस्थानम्~। गोत्रं वंशः~। समवाय एकत्रस्थितिः~। बालेषु केशेष्वन्धकारं तम इति यस्या बालं प्रत्ययम्~। भास्वती मूर्तिमती~। भास्वत आदित्यस्य च मूर्तिः~। न कदाचित्संनिहितबालान्धकारा भवतीति विरोधः~। पुण्डरीकं पद्मम्, सिंहव यस्या मुखं तत्र कथं हरिणस्य विलोचने रत इति विरोधः पयोधरौ स्तनौ, मेघाव पयोधराः~। कलहंसानां स्वनो यस्यां सा~। सरित्कथं प्रावृड् भवतीति विरोधः~। करो हस्तः, रश्मिथ~। शिला वातवज्रीभूतं हिमम्~। यत्र च हिमगिरिशिलाभिः पृथुर्मध्यभागस्तत्र कथं पद्मकोमलकान्तिः~। हिमस्पर्शे पद्मनाशात्~। {\qt मणिबन्धादाकनिष्टं करस्य करभो बहिः}~। करभवोष्टः~। विलम्बितं सविलासम्~। लम्बितच करभो यस्याः~। करभोर कथं विगतकरभगमनेति विरोधः~। कुमारभावो वाल्यम्, कुमारे च भावो भक्तिः~। स्निग्धो रम्यः, प्रतीतश्च~। तारका क्ष्णः कनीनिका, दैत्यभेदश्च तारकः स्कन्देन यो हतः~। परिचयः संस्तवः~। अनुषङ्गः}

\newpage
% २८ हर्षचरिते 

\noindent
दृष्टोजनः इत्यभिधाय तूष्णीमभूत्~। दधीचस्तु नवाम्भोभरगम्भीराम्भोधरध्वाननिभया भारत्या नर्तयन्वनलताभवनभाजो भुजगभुजः सुधीरमुवाच \textendash\ {\haq आर्य, करिष्यसि प्रसादमार्याराध्यमाना~। पश्यामस्तावत्तातम्~। उत्तिष्ठ~। व्रजामः} इति~। तथेति च तेनाभ्यनुज्ञातः शनकैरुत्थाय कृतनमस्कृतिरुञ्चचाल~। तुरगारूढं च तं प्रयान्तं सरस्वती सुचिरमुत्तम्भितपक्ष्मणा निश्चलतारकेण लिखितेनेव च क्षुषा व्यलोकयत्~। उत्तीर्य शोणमचिरेणैव कालेन दधीचः पितुराश्रमपदं जगाम~। गते च तस्मिन्सा तामेव दिशमालोकयन्ती सचिरमतिष्टत्~। कृच्छ्रादिव च संजहार दृशम्~।

अथ मुहूर्तमिव स्थित्वा स्मृत्वा च तां तस्य रूपसंपदं पुनः पुनर्व्यस्मयतास्या हृदयम्~। भूयोऽपि चक्षुराचकाङ्क्ष तद्दर्शनम्~। अवशेष केनाप्यनीयत तामेव दिशं दृष्टिः~। अप्रहितमपि मनस्तेनैव सार्धमगात्~। अजायत च नवपल्लव इव बालवनलतायाः कुतोऽप्यस्या अनुरागश्चेतसि~। सालस्येव शून्येव सनिद्रेव दिवसमनयत्~। अस्तमुपयाति च प्रत्यक्पर्यस्त्रमण्डले लाङ्गलिकास्तबकताम्रत्विषि कमलिनीकामुके कठोरसारसशिरःशोणशोचिषि सावित्रे त्रयीमये तेजसि, तरुणतरतमालश्यामले च मलिनयति व्योम व्योमव्यापिनि तिमिरसंचये, संचरत्सिद्धसुन्दरी\textendash

\vspace{2mm}
\hrule

\noindent
{\s प्रसङ्गः~। विकुक्षिप्रार्थितयापि सावित्र्या कौतुकनिर्वृत्तिर्मा भूदियात्मस्वरूपं नोक्तम्~। अत एवोत्तरत्र तदनुबन्ध एवोक्तः \textendash\ {\qtt भूयसो दिवसानित्यादिना}~। खरूपोक्तौ च ज्ञातसरस्वतीकत्वेनापत्यजननकार्यभङ्गो भवेत्~। भारती वाक्~। भुजगभुजो मयूरान्, भुजग इव भुजावस्येति च~। उच्चचाल गन्तुं प्रवृत्तः~। उत्तम्भितान्युत्क्षिप्तानि~।

कुतोस्मि कस्मादपि न ज्ञायत इत्यर्थः~। मनुष्यतस्तथाविधस्तादृश्याः कथम नुराग इति~। कथमेतदस्या उपपद्यत इति न वाच्यम्~। यदाह मुनिः \textendash\ {\qt शापभ्रंशातु दिव्यानां तथा चापत्यलिप्सया~। कार्यो मानुषसंयोगः शृङ्गाररससंश्रयः~॥} इति~। अन्यत्र कुतः जितेर्नवपल्लवोऽनुरागहतो लतार्थो जायत इत्येवमभिलाषरूपं प्रथमं दशान्तरमालभ्येत्यादिना द्वितीय चिन्तनरूपमाह~। अनयत्कष्टेनात्यवाहयत्~। अस्तमित्यादौ पल्लवशयने तस्थाविति संबन्धः~। प्रतीच्या पश्चिमायाम्~। लाङ्गलिका फलिनी~। मयूरशिखौषधिरित्यपरे, रक्ति केल्यन्ये~। कमलिनीकानुक इति सरस्वतीदयिताभिप्रायेणोक्तम्~। कठोरो जरठः~। सारसो लक्ष्मणः~। शोणो लोहितः~। शोचितिः~। {\qt ऋग्यजुःसामनामानि त्रयो वेदाखयी स्मृता~। वेदे च पठ्यते सैषा}~।}

\newpage
% प्रथम उच्छ्वासः~। २९ 

\noindent
नूपुररवानुसारिणि च मन्दं मन्द मन्दाकिनीहंस इव समुत्सर्पति शशिनि गगनतलम्, कृतसंध्याप्रणामा निशामुख एव निपत्य विमुक्ताङ्गी पल्लवशयने तस्थौ~। सावित्र्यपि कृत्वा यथाक्रियमाणं सायंतनं क्रियाकलापमुचिते शयनकाले किसलयशयनमभजत~। जातनिद्रा च सुष्वाप~।

इतरा तु मुहुर्मुहुरङ्गवलनैर्विलुलितकिसलयशयनतला निमीलितलोचनापि नाभजत निद्राम~। अचिन्तयच्च \textendash\ मर्त्यलोकः खलु सर्वलोकानामुपरि, यस्मिन्नेवंविधानि संभवन्ति त्रिभुवनभूषणानि सकलगुणग्रामगुरूणि रत्नानि~। तथा हि~। तस्य मुखलावण्यप्रवाहस्य निष्यन्दविन्दुरिन्दुः~। तस्य च चक्षुषो विक्षेपा विकचकुमुदकुवलयकमलाकराः~। तस्य चाधरमणेदींधितयो विकसितबन्धूकवनराजयः~। तस्व चाङ्गस्य परभागोपकरणमनङ्गः~। पुण्यभाचि तानि चक्षूंषि चेतांसि यौवनानि वा स्त्रैणानि, येषामसौ विषयो दर्शनस्य~। क्षणं तु दर्शयता च तमन्यजन्मजनितेनेव मे फलितमधर्मेण~। का प्रतिपत्तिरिदानीम् इति चिन्तयन्त्येव कथंकथमप्युपजातनिद्रा चिरात्क्षणमशेत~। सुप्ता च तं दीर्घलोचनं ददर्श~।

\vspace{2mm}
\hrule

\noindent
{\s त्रय्येव विद्या तपतीति~। {\qt कृत \textendash\ }इत्यादिना {\qt तस्थौ} इत्यन्तेन क्रियान्तरल्यागेन वैमनसमावेद्यते~। {\qt वेपते श्वसते चैव मनोरयविचिन्तनैः~। प्रद्वेषेणान्य कार्याणामनुस्मृतिरपीष्यते~॥} {\qtt निशामुख एवेति}~। न पुनरुचिते शयनकाले विमुक्ताङ्गीयनेन निःसहाङ्गत्वमस्या दर्श्यते~। {\qtt तस्थाविति}~। न पुनर्निद्रामलभत~। यथाक्रियमाणमित्यनेन च सरस्वतीतोऽस्या व्यतिरेकं दर्शयन्सरस्वत्या एवानद्भावस्थामाह~।

विलुलितं विपर्यासितम्~। मर्त्यलोक इत्यादिना गुणकीर्तनम्~। चतुर्थमवस्थाविशेषमाह~। तदुक्तम् \textendash\ {\qt अङ्गप्रत्यङ्गलीलाभिर्वाओष्टासहितेक्षणैः~। नास्त्यन्यः सदृशस्तेन तदेतद्रुणकीर्तनम्~॥} इति~। गुणा वैदग्ध्यादयः, सूत्राणि च~। तद्वशेन गुरुणि बहुमानभाजि~। इतरन्न तु तिष्ठतु तावदेकः~। गुणग्रामस्थापि गुणिरूपिते नापि दुवैहानीति यावत्~। {\qtt तस्येति}~। पूर्वानुभूतस्य बिन्दुरिति न केवलं लावण्यप्रवाहाभिप्रायेण यावत्संनिवेशसादृश्यात्~। विक्षेपाः परतः प्रेरणानि~। कुमुदेत्याद्युतम्~। शुक्लृकृष्णरक्तरुचित्वाचक्षुषो दीवितय इति मणिशब्दाभिप्रायेण~। विकसितशब्देन लौहित्यातिशयमाह~। अङ्गानि विद्यन्ते यस्य तदनं शरीरम्~। परभागो वर्णस्य वर्णान्तरेण शोभातिशयः~। स्त्रैणानि स्त्रीसंबन्धीनि का प्रतिपत्तिः किमनुश्यम्~। मदन \textendash\ इत्यादिनोद्गरूपं पञ्चमवस्थाभेदमाद~। यदुतम् \textendash\ {\qt आसने शयने वापि न हृष्यति न तुष्यति~। नित्यमेवोत्सुका च स्यादुद्वेगस्थानमाश्रिताः~॥ चिन्तानिःश्वासखे देन हृद्दाहाभिनयेन च~। कुर्यात्तदेवमत्वन्तमुद्योगाभिनयेन च~॥} इति~। दश किला}

\newpage
% ३० हर्षचरिते 

\noindent
स्वप्नासादितद्वितीयदर्शना चाकर्णाकृष्टकार्मुकेण मनसि निर्दयमताड्यत मकरकेतुना~। प्रतिबुद्धाया मदनशरताडितायाश्च तस्या वार्तामिवोपलव्धुमरतिराजगाम~। तथा हि~। ततः प्रभृति कुसुमधूलिधवलाभिर्वनलताभिरताडितापि वेदनामवत्त~। मन्दमन्दमारुतविधुतैः कुसुमरजोभिरदूषितलोचनाप्यनुजलं मुमोच~। हंसपक्षतालवृन्तत्रातवातविततैः शोणशीकरैरसिक्ताप्याईतामगात्~। प्रेङ्खत्कादम्बमिथुनाभिरनूढाप्यचूर्णत वनकमलिनीकल्लोलदोलाभिः~। विघटमानचक्रवाकयुगलविसृष्टैरस्पृष्टापि श्यामतामाससाद विरहनिःश्वासधूमैः~। पुष्पधूलिधूसरैरदष्टापि व्यचेष्टत मधुकरकुलैः~।

अथ गणरात्रापगमे निवर्तमानस्तेनैव वर्त्मना तं देशमागत्य तथैव निवारितपरिजनश्छत्रधारद्वितीयो विकुक्षिर्डुढौके~। सरस्वती तु तं दूरादेव संमुखमागच्छन्तं प्रीत्या सम्यक्समुत्थाय वनमृगीवोद्ग्रीवा विलोकयन्ती मार्गोपरि श्रान्तमस्नपयदिव धवलितदशदिशा दृशा~। कृतासनपरिग्रहं तु तं प्रीत्या सावित्री, पप्रच्छ \textendash\ {\haq आर्य, कञ्चित्कुशली कुमारः} इति~। सोऽब्रवीत् \textendash\ आयुष्मति, कुशली~। स्मरति च भवत्योः~। केवलममीषु दिवसेषु तनीयसीमिव तनुं विभर्ति~। अविज्ञायमानां चानिमित्तां शून्यतामिवाधत्ते~। अपि

\vspace{2mm}
\hrule

\noindent
{\s कामावस्थाः~। तदुक्तम् \textendash\ {\qt प्रथमे लभिलाषः स्याद्वितीये चिन्तनं भवेत्~। अनुस्मृतिस्तृतीये तु चतुर्थे गुणकीर्तनम~॥ उद्वेगः पञ्चमे प्रोक्तः प्रलापः षष्ठ उच्यते~। उन्मादः सप्तमश्चैव भवेद्याधिस्तथाष्टमे~॥ नवमे जडता प्रोक्ता दशमेमरणं भवेत्~॥} इति~। अरतिर्दुःखासिका हि~। कामवधूप्रतिपक्षभूतेति तदागमनाभिधानम्~। हंसपक्षा इव तालवृन्तं व्यञ्जनम्~। आर्द्रता सस्नेहताम्, क्लिन्नतां च~। प्रेङ्खद्दोलायमानम्~। कादम्बा: कृष्णहंसाः~। श्यामता शृङ्गाररसाविष्कारिवैवर्ण्यम्~। यदुक्तम्\textendash\ {\qt शृङ्गारदेवो भगवान्मुरारिः संगीयते श्यामवपुर्मुरारिः~। श्यामो मनाक्स्निग्धतरच तेन शृङ्गारशंसी मुखराग उक्तः~॥} अथ श्यामतासधूमता~। सधूमता इति विरोधाभासः~।

गणरात्रं निशाबह्वयः~। {\qtt तेनैव वर्त्मनेति}~। अनेन तस्य यदृच्छया तदाश्रयमासमनमिति दर्शयति~। प्रधानप्रकृतेः स्थवीयसस्तथाविधव्यापारविनियोगादिनौचियात्~। अत एव वक्ष्यति \textendash\ {\qt यथाभिलषितं देशमयासीत्}~। डुढौके इत्यनेन निमित्तपरतन्त्रया संनिकृष्टमेचैनमालुलोकेति प्रदर्शितम्~। यदुक्तम् \textendash\ {\qt पटुता धार्ष्ट्यतामिङ्गिताकारज्ञानं प्रतारणे देशकालज्ञता कार्येषु विषह्यबुद्धिवं लध्वी प्रतिपत्तिः सापाया च इति दूतीगुणाः}~। भरतमुनिरपि \textendash\ {\qt विज्ञानगुणसंपन्ना कथिनी लिङ्गिनी तथा~। रङ्गोपजीविनी चापि प्रतिपत्तिविचक्षणा~॥ प्रोत्साहनैककुशलेल्या दिदूतीगुणैर्युता~॥}}

\newpage
% प्रथम उच्छ्वासः~। ३१ 

\noindent
च~। अन्वक्षमांगमिष्यत्येव मालतीति नाम्ना वाणिनी वार्तां वो विज्ञातुम्~। उच्छ्वसितं सा कुमारस्य इति~। तच्छ्रुत्वा पुनरपि सावित्री समभाषत \textendash\ अतिमहानुभावः खलु कुमारो यदेवमविज्ञायमाने क्षणदृष्टेऽपि जने परिचितिमनुबध्नाति~। तस्य हि गच्छतो यदृच्छया कथमप्यंशुकमिच मार्गलतासु मानसमस्मासु मुहूर्तमासक्तमासीत्~। अशून्यं हि सौजन्यमाभिजात्येन वः स्वामिसुनोः~। अलसः खलु लोको यदेवं सुलभसौहार्दानि येनकेनचिन्न क्रीणाति महतां मनांसि~। सोऽयमौदार्यातिशयः कोऽपि महात्मनामितरजनदुर्लभो येनोपकरणीकुर्वन्ति त्रिभुवनम् इति~। विकुक्षिरुच्चावचैरालापैः सुचिरमिव स्थित्वा यथाभिलषितं देशमयासीत्~।

अपरेधुरुद्यति भगवति युमणावामधुताभिद्रुततारके तिरस्कृततमसि तामरसव्यासव्यसनिनि सहस्ररश्मौ शोणमुत्तीर्यायान्ती, तरलदेहप्रभावितानच्छलेनात्यच्छं सकलं शोणसलिलमिवानयन्ती, स्फुटितातिमुक्तककुसुमस्तवकसमत्विषि सटाले महति मृगपताविव गौरी तुरंगमे स्थिता, सलीनमुरोवध्रारोपितस्य तिर्यगुत्कर्णतुरगाकर्ण्यमाननूपुरपटुरणितस्थातिबहलेन पिण्डालक्तकेन पल्लवितस्य कुङ्कुमपिञ्जरितपृष्ठस्य चरणयुगलस्य प्रसरद्भिरतिलोहितैः प्रभाप्रवाहरुभयतस्ताङनदोहद्लोभागतानि किसलयितानि रक्ताशोकवनानीवाकर्षयन्ती, सकलजीवलोकयहठहरणाघोषणयेव रशनया शिञ्जानजघनस्थला, धौतधवलनेत्रनिर्मितेन निर्मोकलघुतरेणाप्रपदीनेन कञ्चुकेन तिरोहित्ततनुलता,

\vspace{2mm}
\hrule

\noindent
{\s इति~। अत एवागृह्णाचाकारतः प्रभृतीत्यादि वक्ष्यते~। अन्वक्षं प्रत्यक्षम्~। वाणिनी दूती~। उच्छ्वसितमित्यनेनेति विनम्भवत्ता ख्याता~। उच्छत्तितं प्राण इति वा~। यदृच्छया यथाकथंचित्~। यश्च तथागच्छति तस्य निरवधानतया वचिदंशुकादि गलति~। आभिजन्यत्वेन महाकुलीनत्वेनोपकरणीकुर्वन्त्यायततां नयन्ति~। उच्चावचैः प्रकृतवस्त्वसंसर्शिभिः, विचित्रैरिति वा~।

अपरेधुरित्यादावीदृशी मालती समदृश्यतेति संबन्धः~। दिवि मणिरिव मणिः~। वियद्भूषणं सूर्यः~। अभिद्रुता न्यकृता~। तामरतं पद्मम्~। व्यासो विकासः~। अति मुक्तकं पुष्पभेदः~। क्रेचिन्मालतीलताकुसुममाहुः~। सटास्ति यस्येति~। {\qt प्राणिस्थादातो लजन्यतरस्याम्}~। गौरी गौराङ्गी, पार्वती च~। सजलतुरङ्गाङ्गस्पर्शपरिजिहीर्षयोरोवध्रेत्यायुक्तम्~। प्रियमधुरशब्दत्वादश्वानामाकर्ण्यमानेत्युक्तम्~। पिण्डालक्तकः क्वथितोऽलक्तकरसः~। दोहदोऽभिलाषः~। वाद्यविशेषानुगताझ्यघोषणा~। रचना मेखला~। शिक्षानं शब्दायमानम्~। निर्मोकः सर्पलक्~। आम्रपदं प्राप्नोत्वा\textendash}

\newpage
% ३२ हर्षचरिते

\noindent
छातक चुकान्तरदृश्यमानैराश्यानचन्दनधवलैरवयवैः स्वच्छसलिलाभ्यन्तरविभाव्यमानमृणालकाण्डेव सरसी, कुसुम्भरागपाटलं पुलकबन्धचित्रं चण्डातकमन्तःस्फुटं स्फटिकभूमिरिव रत्ननिधानमादधाना, हारेणामलकीफलनिस्तुलमुक्ताफलेन स्फुरितस्थूलग्रहगणशारा, शारदीव वेतविरलजलधरपटलावृता द्यौः, कुचपूर्णकलशयोरुपरि रत्नप्रालम्बमालिकामरुणहरितकिरणकिसलयिनी कस्यापि पुण्यवतो हृदयप्रवेशवनमालिकामिव बद्धां धारयन्ती, प्रकोष्ठनिविष्टस्यैकैकस्य हाटककटकस्य मरकतमकरवेविकास सरकतमकरवेदिकासनाथस्य हरितीकृतदिगन्ताभिर्मयूखसंततिभिः स्थलकमलिनीभिरिव लक्ष्मीशङ्कयानुगम्यमाना, बहलताम्बूलकृष्णिकान्धकारितेनाधरसंपुटेन मुखशशिपीतं ससंध्यारागं तिमिरमिव वसन्ती, विकचनयनकुवलयकुतूहलालीनयालिकुलसंह्त्या नीलांशुकजालिकचेव निरुद्धार्धवदना, नीलीरागनिहितनीलिम्ना शितिगलशितिना वामश्रवणाश्रयणा दन्तपत्रेण कालमेघ पवेनेव विद्युदिव द्योतमाना, बकुलफलानुकारिणीभिस्तिसृभिर्मुक्ताभिः कल्पितेन बालिकायुगलेना धोमुखेनालोकजलवार्षिणा सिञ्चन्तीवातिकोमले भुजलते, दक्षिणकर्णावतंसितया केतकीगर्भपलाशलेखयारजनिकरजिह्वालतयेव लावण्यलोमेन लिह्यमानकपोलतला, तमालश्यामलेन मृगमदामोदनिष्यन्दिना तिलकबिन्दुना मुद्रितमिव मनोभवसर्वस्वं वदनमुद्रहन्ती, ललाटलासकस्य सीमन्तचुम्बिनश्चटुलेनांकमणेरुद्भवता चटुलेनांशुजालेन रक्तांशुकेनेव कृतशिरोवगुण्ठना, पृष्ठप्रेसदनादरसं यमनशिथिलजूटिकाबन्धा, नीलचामरावचूलिनीव चू\textendash

\vspace{2mm}
\hrule

\noindent
{\s प्रपदीनः पादं यावत्~। छातस्तनुः~। कुसुम्भं पद्मकम्~। नानावर्णबिन्दुन्यासः पुलकबन्धः, मणिविशेषाश्च पुलकाः~। चण्डातकमर्धोरुकम्~। कुचाचेव कस्यापि पुण्यवत इवेति वक्ष्यमाणाभिप्रायेण पूर्णकलशौ~। कस्यापीत्यलौकिकस्य~। वनमाला पत्रपुष्पयोजिता स्रुक्~। सापि पूर्णकलशयोरुपरि बद्ध्यते~। प्रकोष्ठः प्रकुञ्चनकः वेदिका रत्नप्रतिष्ठापीठिका~। बहलं पौनःपुन्येन कृतम्~। कृष्णिका कृष्णलेखा~। मुखमेव तमःपारप्रतिपिपादयिषया शशी~। ताम्बूलकारणत्वेन हिलमेव संभवतीति ससंध्यारागमित्युक्तम्~। नील्योषधिभेदः~। शितिर्नीलः~। पल्लवः पिण्डः~। बालिका कर्णोपवेधेऽलंकारः~। अधोमुखेन घटादिना जलवर्षिणा लता सिच्यते~। मृगमद: कस्तूरिका~। तिलकबिन्दुः परिवर्तुलस्तिलकः~। लासको नर्तकः~। सुवर्ण\textendash}

\newpage
% प्रथम उच्छ्वासः~। ३३ 

\noindent
डामणिमकरिकासनाथा, मकरकेतुकेतुपताका, कुलदेवतेव, चन्द्रमसः, पुनःसंजीवनौषधिरिव पुष्पधनुषः, वेलेव रागसागरस्य, ज्योत्स्नेव यौवनचन्द्रोदयस्य, महानदीव रतिरसामृतस्य, कुसुमोद्गतिरिव सुरततरोः, बालविद्येव वैदग्ध्यस्य, कौमुदीव कान्तेः, धृतिरिव धैर्यस्य, गुरुशालेव गौरवस्य, बीजभूनिरिव विनयस्य, गोष्टीव गुणानाम्, मनस्वितेव महानुभावतायाः, तृप्तिरिव तारुण्यस्य, कुवलयदलदामदीर्घलोचनया पाटलाधरया कुन्दकुङ्गलस्फुटदशनया शिरीषमालासुकुमारभुजयुगलया कमलकोमलकरया बकुलसुरभिनिःश्वसितया चम्पकावदातया कुसुममय्येव ताम्बूलकरङ्कवाहिन्या महाप्रमाणाश्वतरारूढयानुगम्यमाना, कतिपयपरिचारकपरिकरा मालती समदृश्यत~। दूरादेव व दधीचप्रेम्णा सरस्वत्या लुण्ठितेव मनोरथैः, आकृष्टेव कुतूहलेन, प्रत्युद्गतेवोत्कलिकाभिः, आलिङ्गितवोत्कण्ठया, अन्तःप्रवेशितेव

\vspace{2mm}
\hrule

\noindent
{\s शृङ्खलाबद्धो नानारत्नौघमण्डितः~। ललाटलम्ब्यलंकारदुलातिलको मतः~॥ अवचूलं चिह्नम्~। मकरिका मकराकारं रूपम्~। वेला यथा सागरं क्षोभयति तद्देव रागम्~। क्षोभेन यथा सागरो दुरुत्तर एवमेतयापि रागः~। यथा~। ज्योतया विना चन्द्रोदयो भवन्नपि न क्कापि विलसन्विभाव्यते तथैतया विना यौवनं सविलासमन्यत्र न दृश्यते~। रतिप्रधानो रसः शृङ्गार एव~। माधुर्यातिशययोगित्वात्प्रकृष्टत्वाच्च~। ह्लादतममृतम्~। यदुक्तम् \textendash\ {\qt शृङ्गार एव परमः परः प्रह्लादनो रसः} इति~। संप्रयोगो रतं रहःशयनं मोहन मिति पर्यायाः~। बालविद्या न कंचन मुञ्चति, तद्वदेष वैदग्ध्यम्~। {\qtt कौमुदीति}~। तथाविधकान्त्यतिशयसंभवात्~। प्रियते येन धृतिः~। अस्यां सत्यां धैर्यमपि~। यद्वा धृतिः प्रवेशरक्षणम्~। यथा प्रविशन्क विद्वाजनिकटं प्रियते केनचित्तथा वैये तावत्प्रसरति ! यावदेषा न दृष्टा एतस्यां दृष्टायां सर्वे धैर्यशून्या इति~। समानविद्यावित्तशीलबुद्धिवयसामनुरूपैरालापैरेकनासनबन्धो गोष्ठी मनस्विता इत्यनेनैतस्या महानुभावताया व्यभिचारित्वमुच्यते~। यस्माद्यत्र मनखिता तत्र महाशयत्वमेवावश्यं संभावयतीति स्थितमेव~। {\qtt तृप्तिरिवेति}~। यथा कश्चित्संजाततृप्तिर्नान्यत्किंचित्पुनरपेक्षते तद्वदासादितमालतीकं तारुण्यम्~। एतदाश्रयणेन परिपूर्णवैषयिकोपभोगप्राप्तिस्तारुण्यस्येत्यर्थः~। {\qtt कुसुममय्येवेति}~। कुवलयादिभिर्नयनादीनां विधानम्~। तरुणोऽश्वोऽश्वतरः~। {\qt वत्सोक्षाश्वभेभ्यश्च तनुत्वम्} इति तनुत्वे तरप्~। अत्र च व्याख्यातम् \textendash\ {\qt तनुत्वं द्वितीयचय प्राप्तिः} इति~। अश्वतरो वा गर्दनाश्वायां जातः~। {\qtt मालतीति}~। एवं दधीचपरिवारभूतया मा लल्या गणवर्णनदारेण सरस्वत्या एवं निःसामान्यगुणातिशयो व्यन्यते~। {\qtt लुण्ठितेवेति}~। चक्ष्यमाणं प्रार्थनादि~। तथा मनोरथैरुत्प्रेक्ष्य स्वीकृतमित्यतस्तैईण्ठितेवे\textendash}

\lfoot{ह० \textendash\ ४}

\newpage
\lfoot{}
% ३४ हर्षचरिते 

\noindent
हृदयेन, स्नपितेवानन्दाश्रुभिः, विलुप्तेव स्मितेन, वीजितेवोच्छ्वसितैः, आच्छादितेव चक्षुषा, अभ्यर्चितेव बदनपुण्डरीकेण, सस्वीकृतेवाशया सविधमुपययौ~। अवतीर्य च तुरगाहूरादेवावनतेन मूर्ध्ना प्रणाममकरोत्~। आलिङ्गिता च ताभ्यां सविनयमुपाविशत्~। सप्रश्रयं ताभ्यां संभाषिता च पुण्यभाजमात्मानममन्यत~। अकथयच्च दधीचसंदिष्टं शिरसि विनिहितेनाञ्जलिना नमस्कारम्~। अगृह्णाचाकारतः प्रभृत्यग्राम्यतया तैस्तैरपि पेशलैरालापैः सावित्रीसरस्वत्योर्मनसी~।

क्रमेण चातीते मध्यंदिनसमये शोणमवतीर्णायां सावित्र्यां स्नातुमुत्सारितपरिजना साकूता मालती कुसुमप्रस्तरशायिनीं समुपसृत्य सरस्वतीमाबभाषे \textendash\ {\haq देवि, विज्ञप्यं नः किंचिदस्ति रहसि~। अतो मुहूर्तमवधानदानेन प्रसाद क्रियमाणमिच्छामि} इति~। सरस्वती तु दधीचसंदेशाशङ्किनी किं वक्ष्यतीति स्तनविनिहितवामकरनखकिरणदन्तुरितमुद्भिद्यमानकुतूहलाङ्करनिकरमिव हृदयमुत्तरीयदुकूलवल्कलैकदेशेन संछादयन्ती, गलतावतंसपल्लवेन श्रोतुं श्रवणेनेव धावसानेना नवरतश्वाससंदोहदोलायितां जीविताशामिव समासन्नलतामवलम्बमानी, समुत्फुल्लस्य मुखशशिनो लावण्यप्रवाहेण शृङ्गाररसेनेव प्लावयन्ती जीवलोकम्, शयनकुसुमपरिमललग्नैर्मधुकरकदम्बकैर्मदनानलदाहश्यामलैर्मनोरथैरिव निर्गय मूर्तैरुत्क्षिप्यमाणा, कुसुमशयनीयात्स्मरशरसंज्वरिणी, मन्दं मन्दमुदगात्~। {\haq उपांशु कथय} इति कपोलतलप्रतिविम्वितां लज्जयेव कर्णमूलं मालती प्रवेशयन्ती मधुरया गिरा सुधीरमुवाच \textendash\ सखि मालति किमर्थमेवमभिदधासि~। कामवधानदानस्य शरीरस्य प्राणानां वा~। सर्वस्याप्रार्थितोऽपि प्रभवत्येवातिबेलं चक्षुष्यो जनः~। सा न

\vspace{2mm}
\hrule

\noindent
{\s त्युक्तम्~। लुण्ठनं च पाथेयाभिवितरणमेवमन्यत्~। उत्कलिका सहरुहिका~। सविधं समीपम्~। अपि च यः स्निग्धो दूरात्सविधमायाति, तस्य लुण्ठनादिसर्वमर्चनावसानं क्रियत इति ध्वनिः~। पेशलैर्हृयैः~।

आकूतमभिप्रायः~। रहस्येकान्ते~। सरस्वतीत्यादौ सरखती कुसुमशयनीयादुदमादुदतिष्ठदिति संबन्धः~। अवतंसपलचेन गलतेतीत्थंभूतलक्षणे तृतीया~। संदोहः समूह:~। संज्वरः संतापः~। उपधनुक्तम्~। अतिवेलमतिमात्रम्~। {\qt अतिपेशलः} इति पाठे पेशलः सुन्दरः~। चक्षुष्योऽनुकूलः~। लमिव व्यक्तम्~। चक्षुष्य इति भक्तया दधी\textendash}

\newpage
% प्रथम उच्छ्वासः~। ३५

\noindent
काचिद्या न भवसि मे वसा सखी प्रणयिनी प्राणसमा च~। नियुज्यतां यावतः कार्यस्य क्षमं क्षोदीयसो गरीयसो वा शरीरकमिदम्~। अनवस्करमाश्रवं मे त्वयि हृदयम्~। प्रीत्या प्रतिसर विधेयास्मि ते~। व्यावृणु वरवर्णिनि, विवक्षितम्, इति~। सा त्ववादीत् \textendash\ देवि, जानास्येव माधुर्ये विषयाणाम्, लोलुपतां चेन्द्रियग्रामस्य, उन्मादितां च नवयौवनस्य, पारिप्लवतां च मनसः~। प्रख्यातैव मन्मथस्य दुर्निवारता~। अतो न मामुपालम्भेनोपस्थातुमर्हसि~। न च बालिशता चपलता चारणता वा वाचालतायाः कारणम्~। न किंचिन्न कारयत्यसाधारणा स्वामिभक्तिः~। सा त्वं देवि, यदैव दृष्टासि देवेन तत एवारभ्यास्य कामो गुरुः, चन्द्रमा जीवितेशः, मलयमरुदुच्छ्वास हेतुः, आधयोऽन्तरङ्गस्थानेषु, संतापः परमसुहृत्, प्रजागर आप्तः, मनोरथाः सर्वगताः, निःश्वासा विग्रहात्रेसराः, मृत्युः पार्श्ववर्ती, रणरणक: संचारकः, संकल्पा बुद्ध्युपदेशवृद्धाः~। किं वा विज्ञा\textendash

\vspace{2mm}
\hrule

\noindent
{\s च इति ध्वनति~। स्वसा भगिनी~। प्रणयिनी विश्वता~। अतिशयेनक्षुद्रमल्पं क्षोदीयः~। {\qt ज्ञेयं गुह्यमवस्करम्}~। आश्रयं वचसि स्थितम्~। प्रतिसरानुकूला~। विधेया वश्या~। व्यावृणु प्रकटय~। वरवर्णिनि वरारोहे~। लोलुपतां साभिलाषलम्~। {\qt चलार्थकौ निगद्येते पारिप्लवपरिप्लवौ}~। बालिशोऽज्ञः~। चारणता धूर्तता~। असाधारणानन्यसदृशी~। देवी देवेनेति च परस्परसमगुणयोगित्वमभिव्यनक्ति~। गुरुर्गरीयान, उपदेष्टा वा~। तद्वशवर्तित्वात्~। यश्च देवस्तस्य गुरुराचार्यः कश्चि दवश्यं संभवति~। जीवितस्येश्वर: स्वामी जीवितेशः~। शिशिरतया मदनदा हप्रशमनहेतुत्वात्~। अमृतमयत्वेन च जीवितसंधारणसक्तत्वात्~। अथ च जीवित्तेशो मृत्युः~। चन्द्रादयो त्यापातत एव तापं शमयन्ति, अनवरत सेव्यमानाः पुनः कामोद्दीपकत्वेन मृत्युं दिशन्ति~। राजपक्षे जीवितेशः कश्चित्पुरोहितप्रायः~। उच्छुसनमुच्छ्वासस्तत्र हेतुः~। अथ च श्वासोत्कान्तौ कारणम्, इतरत्र सचिवप्राया विश्वसनीयाः~। आधयश्चित्तपीडाः~। अत एवान्तरङ्गमन्तः शरीरं यानि स्थानानि तेषु, इतरत्रान्तरङ्गान्तवैशिकस्तत्स्थानेषु विश्वसनीयजनाधिकारेषु~। परं प्रकृष्टम्~। असुहरोऽमित्रो वा, अन्यत्र परमसुहृन्मिलं च आप्तः प्राप्तो बान्धवप्रायः कचित्~। सर्वगताधारा अपि संस्थाख्याः~। विग्रहो विरोधः, देव~। {\qtt मृत्युरिति}~। त्वदनङ्गीकारेण निश्चितं म्रियते~। राज्ञोऽपि पार्श्वे मृत्युस्तिष्ठत्येव~। रणरणको दुःखमरतिकृतम्~। अत एव संचारक एकत्र नरे संभववितरत्र संचारयति, चरितं वस्तु यः प्रापयते सः~। द्विविधा हि चारा \textendash\ संस्थाः, संचारकाश्च~।}

\newpage
% ३६ हर्षचरिते 

\noindent
पयामि~। अनुरूपो देव्या इयात्मसंभावना, शीलवानिति प्रक्रमविरुद्धम्, धीर इत्यवस्थाविपरीतम्, सुभग इति त्वदायत्तम्, स्थिरप्रीतिरिति निपुणोपक्षेपः, जानाति सेवितुमित्यस्वामिभावोचितम्, इच्छति दासभावमामरणात्कर्तुमिति धूर्तालापः, भवनस्वामिनी भवसीत्युपप्रलोभनम, पुण्यभागिनी भजति भर्तारं तादृशमिति स्वामिपक्षपातः, त्वं तस्य मृत्युरित्यप्रियम्, अगुणज्ञासीत्यधिक्षेपः, स्वशेऽस्य बहुशः कृतप्रसादासीत्यसाक्षिकम्, प्राणरक्षार्थमर्थयत इति कातरता, तत्रागम्यतामित्याज्ञा, वारितोऽपि बलादागच्छतीति परिभवः~। तदेवमगोचरे गिरामसीति श्रुत्वा देवी प्रमाणम् इत्यभिधाय तूष्णीमभूत्~।

अथ सरस्वती प्रीतिविस्फारितेन चक्षुषा प्रत्यवादीत् \textendash\ {\haq अयि, न शक्नोमि बहु भाषितुम्~। एषास्मि ते स्मितवादिनि वचसि स्थिता~। गृह्यन्ताममी प्राणाः} इति~। मालती तु {\haq यदाज्ञापयस्यतिप्रसादः} इति व्याहृत्य प्रहर्षपरवशा प्रणम्य प्रजविना तुरगेण ततार झोणम्~। अगाच्च दधीचमानेतुं व्यवनाश्रमपदम्~। इतरातु सखीस्नेहेन सावित्रीमपि विदितवृत्तान्तामकरोत्~। उत्कण्ठाभारभृता च ताम्यता चेतसा कल्पायितं कथंकथमपि दिवसशेषमनैषीत्~। अस्तमुपगतवति भगवति गभस्तिमति, स्तिमिततरमवतरति तमसि, प्रहसितामिव सितां दिशं पौरंदरी दरीमिक केसरिणि मुञ्चति चन्द्रमसि, सरस्वती शुचिनि चीनांशुकसुकुमारे तरङ्गिणी दुगूलकोमले शयन इव शोणसैकते समुपविष्टस्वप्नकृतप्रार्थनापाद्पतनलग्नां दधीचचरणनखचन्द्रिकामिव ललाटिकां दधाना, गण्डस्थलादर्शप्रतिबिम्बितेन {\haq चारुहासिनि, अयमसावाहृतो हृदययितो जनः} इति श्रवणसमीपवर्तिना निवेद्यमान\textendash

\vspace{2mm}
\hrule

\noindent
{\s वृद्धा महान्तः, स्थविराध~। अनुरूप इत्यादिनेदमिदं तत्रास्तीति वक्रोक्त्या सातिशयं मालती वैदग्ध्येनाह~। प्रक्रम आरम्भः~। निपुणोपक्षेपो बुद्धिमत्प्रक्रमः~। धूर्तालापः प्रतारणावचनम्~। {\qtt वारित इति}~। भवत्येवेत्यर्थात्~।

प्रजविनेति साभिप्रायम्~। अस्तमित्यादौ सरस्वती प्रतिपालयामासेति संबन्धः~। गभस्तिमान्रविः~। पौरंदयन्द्री~। दरी खदा~। चीनेत्यादि सैकतविशेषणम्~। उपमानस्य तु दुगूलकोमल इत्युक्तम्~। तरङ्गिणी प्रतिदिनं क्षीयमाणेन वारिणा कृतलेखे भनियुक्ते च चन्द्रिका कान्तिरत्र~। ललाटालंकारो ललाटिका~।}

\newpage
% प्रथम उच्छ्वासः~। ३७

\noindent
मदनसंदेशेवेन्दुना, विकीर्यमाणनखकिरणचक्रवालेन बालव्यजनीकृतचन्द्रकलाकलापेनेव करेण वीजयन्ती स्वेदिनं कपोलपट्टम, {\haq अत्र दुधीचाहते न केनचित्प्रवेष्टव्यम्} इति तिरश्चीनं चित्तभुवा पातितां विलासवेबलतामिव बालमृणालिकामधिस्तनं स्तनयन्ती कथमपि हृदयेन वहन्ती प्रतिपालयामास~। आसीवास्या मनसि \textendash\ अह्मपि नाम सरस्वती चत्रामुना मनोजन्मना जघन्येव परवशीकृता~। तत्र~। का गणनेतरासु तपखिनीष्वतितरलासु तरुणीषु इति~।

आजगाम च मधुमास इव सुरभिगन्धवहः, हंस इव कृतमृणालधृतिः, शिखण्डीव घनप्रीत्युन्मुखः, मलयानिल इवाहितसरसचन्दनधवलतनुलतोत्कम्पः, कृष्यमाण इव कृतकरकचत्रहेण ग्रहपतिना, प्रेर्यमाण इव कंदर्पोद्दीपनदक्षेण दक्षिणानिलेन, उह्यमान इवोत्कलिकाबहलेन रतिरसेन, परिमलसंपातिना मधुपपटलेन पटेनेव नीलेनाच्छादिताङ्गयष्टिः, अन्तःस्फुरता मत्तमदनकरिकर्णशङ्खायमानेन प्रतिमेन्दुना प्रथमसमागमविलासविलक्षस्मितेनेव धवलीक्रियमाणैककपोलोदरो मालतीद्वितीयो दधीचः~। आगत्य च हृदयगतदयितानपुररवमिश्रयेव हंसगद्गदया गिरा कृतसंभाषणो यथा मन्मथः समाज्ञापयति, यथा यौवनमुपदिशति, यथानुराग: शिक्षयति, यथा विदग्धताध्यापयति, तथा तामभिरामां

\vspace{2mm}
\hrule

\noindent
{\s चक्रवालं समूहः वालव्यजनं चामरम्~। स्तनमध्ये प्रवेशाभावात्तिरवीनमि त्युक्तम्~। यश्च वेत्री प्रवेश निषेधननिमित्तं वेत्रलतां पातयति स तिरवीनं स्तनयोरधिस्तनम्~। विभक्त्यर्थेऽव्ययीभावः~। कुचपृष्ठ इत्यर्थः~। स्तनयन्ती कलयन्ती~। स्तनिः शब्दार्थचौरादिकः~। {\qt स्तनन्ति} इति वा पाठः~। तपस्विनीषु वराकीषु~।

आजगामेत्यादावाजगामेति संबन्धः~। सुरभिगन्धवहो वातः सुरभिगन्धं च यो वहति~। धृतिर्धारणम्, प्राणयात्रा च~। घनः~। सरसं सान्द्रं यच्चन्दनं तेन घवल्या तनुलतयाहितत्रप उत्कम्पः कामधर्मो यस्य~। अन्यत्र चन्दनाश्च धनाश्च लान्ति श्रयन्ति यास्तन्व्यो लताखासामाहित उत्कम्पः कम्पनं येनेति~। कृष्यमाण इत्युद्दीपनकारणत्वात्~। करा रश्मयः, हस्तश्च करः~। हस्तस्य कर्षणं समुचितम्~। ग्रहपतिश्चन्द्रः~। {\qtt प्रेर्यमाण इति}~। अनिलस्योचितमेतत्कर्म~। {\qtt उह्यमान इति}~। जलस्योचितमेतत्~। उत्कलिका रुहरुहिका, ऊर्मयश्च~। रसोऽभिलाषः, जलं च~। परिमल आमोदः~। पटलं समूहः~। प्रतिमा प्रातिच्छन्दकम्~। {\qtt यथा सन्मथ इति}~। मन्मथस्य प्रभवनशीलत्वेनाज्ञादानसुचितम्~। एवं सर्वत्रो दिशतीतीत्यमित्थं वर्तस्वेत्युपदेशः~। देवताविषयं संभोगशृङ्गारवर्णनमनुचितमिति न तत्र विस्तरः प्रवर्तते~। कुमारीत्वे च गान्धर्वविवाहो विस्तरेण}

\newpage
% ३८ हर्षचरिते

\noindent
रामामरमयत्~। उपजातवित्रम्भा चात्मानमकथयस्य सरस्वती~। तया तु सार्धमेकं दिवसमिवानयत्संवत्सरमधिकम्~।

अथ दैवयोगात्सरस्वती बभार गर्भम्~। असूत चानेहसा सर्वलक्षणाभिरामं तनयम्~। तस्मै च जातमात्रायैव {\haq सम्यक्सरहस्याः सर्वे वेदाः सर्वाणि च शास्त्राणि सकला कलाः मत्प्रसादात्स्वयमाविर्भविष्यन्ति} इति वरमदात्~। सर्तृलाघया दर्शयितुमित्र हृदयेनादाय दधीचं पितामहादेशात्समं सावित्र्या ब्रह्मलोकमारुरोह~। गतायां च तस्यां दधीचोऽपि हृदये ह्रादिन्येवाभिहतो भार्गववंशसंभूतस्य भ्रातुर्ब्राह्मणस्य जायामक्षमालामि धानां मुनिकन्यकामात्मसूनोः संवर्धनाय नियुज्य विरहातुरस्तपसे वनमगात्~। यस्मिन्नेवावसरे सरस्वत्यसूत तनयं तस्मिन्नेवाक्षमालापि सुतं प्रसूतवती~। तौ तु सा निर्विशेष सामान्यस्तन्या शनैः शनैः शिरा समवर्धयन एकस्तयोः सारस्वताख्य एवाभवत्, द्वितीयोऽपि वत्सनामाभवत्~। आसीच तयोः सोदयोरिव स्पृहणीया प्रीतिः~।

अथ सारस्वतो मातुर्महिम्ना यौवनारम्भ एवाविर्भूताशेषविद्यासंभारस्तस्मिन्सवयसि भ्रातरि प्रेयसि प्राणसमे सुहृदि वत्से वाङ्मयं समस्तमेव संचारयामास~। चकार च कृतदारपरिग्रहस्यास्य तस्मिन्नेव प्रदेशे प्रीत्या प्रीतिकूटनामानं निवासम्~। आत्मनाप्या षाढी, कृष्णाजिनी, वल्कली, अक्षवलयी, मेखली, जटी च भूत्वा तपस्यतो जनयितुरेव जगामान्तिकम्~।

अथ तस्मात्प्रवर्धमानादिपुरुषज नितात्मचरणोन्नतिनिर्गतप्रघोषः, परमेश्वरशिरोवृतः, सकलकलागमगम्भीरः, महामुनिमान्यो

\vspace{2mm}
\hrule

\noindent
{\s न तथा वार्णितः शापनिर्वाहणमात्रपरत्वादिति~। वृत्तस्यान्यथानिजभर्तृत्यागो दोषावहः किमर्थं कृत इत्यादिकाः कुविकल्या उत्पथेरन्निति~।

अनेहसा कालेन~। रहस्यं ज्ञानभागः~। ह्रादिनी वज्रम्~।

वाक्प्रस्तुता यत्र तद्वाङ्मयम्~। {\qt आषाढसंज्ञो दण्डः स्यात्पालाशो व्रतचारिणाम्~। वृक्षवनिर्मितं वस्त्रं वल्कलं समुदाहृतम्~॥} मेखला मुजतृणादिर चितं कटिसूत्रम्~। जटा रुक्षसंहतकेशाः~।

अथेत्यादौ वत्सात्प्रावर्तत विपुलो वंश इति संबन्धः~। प्रवर्धमानाः संतानादिना}

\newpage
% प्रथम उच्छ्वासः~। ३९ 

\noindent
विपक्षक्षोभक्षमः, क्षितितललव्धायतिः, अस्खलितप्रवृत्तो भागीरथीप्रवाह इव पावन: प्रावर्तत विपुलो वंशः~। यस्मादजायन्त वात्स्यायना नाम गृहमुनयः, आश्रितश्रौता अप्यनालम्बितालीकवककाकवः, कृतकुक्कुटव्रता अप्यवैडालवृत्तयः, विवर्जितजनपङ्कयः, परिहृतकपटकीरकुचीकूर्चाकूताः, अगृहीतगह्वराः, न्यक्कृतनिकृतयः, प्रसन्नप्रकृतयः, विगतविकृतयः, परपरिवादपराचीनचेतसः, वर्णत्रयव्यावृत्तिविशुद्धान्धसः, धीरधिषणावधूताध्येषणाः, असङ्कसुकस्वभावाः, प्रणतप्रणयिनः, शमितसमस्तशाखान्तरसंशीतयः, उद्घाटितसमग्रग्रन्थार्थमन्थयः, कवयः, वाग्मिनः, विमत्सराः, सरसभाषितव्यसनिनः, विदग्धपरिहासवेदिन:, परिचयपेशलाः, नुत्यगीतवादित्रेष्ववाह्याः, ऐतिह्यस्यावितृष्णाः, सानुकोशाः, सत्यशुचयः, साधुसंमताः, सर्वसत्त्वसौ\textendash

\vspace{2mm}
\hrule

\noindent
{\s वृद्धिं यच्छन्तो य आदिपुरुषाः पूर्ववान्धवाः शुक्राधास्तैः कृताः खेषां चरणानां कठादिशाखाध्यायिनामुनतिरुत्कर्षो यस्य सः~। अन्यत्र प्रवर्धमानस्तु वामनरूपो य आदिपुरुषो हरिस्तेन जनिता स्वपदोन्नतिर्माहात्म्यं यस्य स इति~। किल त्रैलोक्याक्रान्तिकाले ब्रह्मलोकप्राप्ताद्विष्णुपदाब्रह्मणा कमण्डलुजलक्षालिताद्गङ्गा समभवदिति वार्ता~। प्रदोषो यशः, शब्दव~। परमेश्वरो राजा, हरश्च~। सकलानां कलानां वृत्ताद्यानामागमस्तेन सह कलकलेन च सकलकले यदागमनं तु न च~। महामुनिर्जडुरपि~। विपक्षाः शत्रवः, शैलाश्च~। वीनां पक्षिणां वा पक्षच्छेदेषु सहिष्णुः~। आयतिः~। प्रतापः, विस्तारथ~। स्खलितं स्वाचारच्युतिः~। प्रवृत्तः प्रकृष्टहृत्तः~। अस्खलितं असंरुद्धं कृत्वा गतव~। श्रौतं वेदभवम्, चिरवृत्तं च~। {\qt भिन्नो भयाद्वा शोकाद्वा ध्वनिः काऊरुदाहृता}~। अत्र च छद्म लक्ष्यते~। बकस्य काकुः बकच्छद्म यैश्च चिरवृत्तमाथितं ते छद्मचारित्वादाश्रितबककाकवो भवन्त्येव~। अभी तुन तथेति विरोधः~। कुकुठवतं नियमविशेषः~। यत्र कुक्कुटाण्डप्रमाणप्रासभोजनम्~। न बैडाली हिंसावृत्तिर्येषां तैः~। विरोधे तु कुक्कुटानां व्रतं भक्षणं येन कृतं स कथं विडालवृत्तिर्न स्यात्~। पकिर्लोकप्रसिद्धो व्यवहारः, पाको वा~। कपटो व्याजवृत्तिः~। कूची: स्फुटाः~। आत्ममहिना व्यवहारः, समूह इत्यन्ये~। एतेष्वाकृतं परिहृतं यैः~। गह्वरं पापम्~। निकृतिः शाट्यम्~। प्रकृतिः स्वभावः~। पराचीनं पराङ्मुखम्~। अन्धोऽन्नम् धीरा स्थिरा~। घिषणा बुद्धिः~। अध्येषणा याञ्चा~। असुकः स्थिरः, मृदुर्वा~। शाखाः कठाद्याः~। संशीतिः संशयः~। ग्रन्थिर्दुर्बोधः प्रदेशः~। परिहासं विदन्ति, न तु स्वयं कुर्वन्ति~। परिचयः संस्तवः~। सुकुमारा: अइन्द्रकूटा इत्यर्थः~। अवायाः, न तु वदेक निष्ठाः~। ऐतित्यमागमः~। अनुक्रोशो दया~। संमता इष्टाः~। सौहार्दं प्रीतिः~। सर्वे गुणा धै\textendash}

\newpage
% ४० हर्षचरिते 

\noindent
हार्दद्रवार्द्रहृदयाः, तथा सर्वगुणोपेता राजसेनानभिभूताः, क्षमभाज आश्रितनन्दनाः, अनिस्त्रिंशा विद्याधरा अनडाः कलावन्तः, अदोषास्तारकाः, अपरोपतापिनो भाखन्तः, अनुष्माणो हुतभुजः, अकुसृतयो भोगिनः, अस्तम्भाः पुण्यालयाः, अलुप्तक्रतुक्रिया दक्षाः, अव्यालाः कामजितः, असाधारणा द्विजातयः~।

तेषु चैवमुत्पद्यमानेषु, संसरति संसारे, यात्सु युगेषु, अवतीर्णे कलौ, वहत्सु वत्सरेषु, त्रजत्सु वासरेषु, अतिक्रामति च काले, प्रसवपरम्पराभिरनवरतमापतति विकारिनि वात्स्यायनकुले, क्रमेण कुबेरनामा वैनतेय इव गुरुपक्षपाती द्विजो जन्म लेभे~। तस्याभवव्रच्युत ईशानो हरः पाशुपतश्चेति चत्वाते युगारम्भा इव ब्राह्मतेजोजन्यमानप्रजाविस्तारा नारायणबाहुदण्डा इव सच्चक्रनन्दकास्तनयाः~। तत्र पाशुपतस्यैक एवाभवद्भूभार इवा\textendash

\vspace{2mm}
\hrule

\noindent
{\s र्याद्याः~। राज्ञां सेनया चानभिभूता ये च सर्वैर्गुणैः सत्त्वरजस्तमोभिर्युक्तास्ते कथं राजसेन गुणेनानभिभूता भवन्तीति विरोधः~। एवमुत्तरत्र विरोध उद्भावनीयः~। क्षमा क्षान्ति भूश्च आश्रितानां नन्दना नन्दयितारः, देवोद्यानं नन्दनं च~। न नित्रिंशा अकूराः~। विद्यां धारयन्तीति विद्याधराः पण्डिताः निश्रिशाश्च खङ्गा एव~। ये च विद्याधरा देवभूतास्ते सखङ्गा एव~। न त्वनिस्त्रिंशा इति मालाखङ्गगुलिकाञ्जनादिना भेदेन भिन्नानामपि विद्याधराणां खङ्गहस्तत्वं न व्यभिचरति~। अजडा अनन्दधियः अशीताश्च~। कलावन्तो गीताभिज्ञाः, कलावाश्चन्द्रः स चाजडोऽशीत इति विरोधः~। दोषा द्वेषाद्याः, रात्रिश्च~। तारयन्तीति तारका आचार्याः, नक्षत्राणि च~। उपतापः पीडा, उष्णत्वं च~। भास्वन्तस्तेजस्विनः, आदित्याश्च~। परांस्पयन्ति ऊष्मा स्मयः दाहिकाशक्तिशा~। हुताशशब्देन हुतमिष्टमुच्यते हुतं भुञ्जते हुतभुजः, आहितामयो वह्नयश्च~। कुसृतिः शाठ्यम्, कौ भूमौ सृतिः सरणम्~। भोगिनः सुखिनः , सर्पाश्च~। स्तम्भः स्तब्धता, सात्विको भावभेदश्च अप्रणतिर्वा, गृहधारणकाष्ठं च~। पुण्यालयाः सुकृतिनः, मठादिस्थानानि च~। दक्षाश्चतुराः प्रजापतिभेदश्च दक्षः~। स च लुप्तक्रतुकियो हररोषजेन वीरभद्रेण~। व्यालाः शठाः, सर्पाश्च~। कामजीतः संतुष्टाः, हरश्च कामजित्~। असाधारणाः सर्वोत्कृष्टाः~। द्विजातयो विप्राः~। येषां च द्वे जाती तेषां कथं नासादृश्यम्~।}

\newpage
% प्रथम उच्छ्वासः~। ४१ 

\noindent
चलकुलस्थितिश्चतुरुदधिगम्भीरोऽर्थपतिरिति नाम्ना समग्रामजन्मचक्रचूडामणिर्महात्मा सूनुः~। सोऽजनयद्भृगुं हंसं शुचिं कविं महीदत्तं धर्मे जातवेदसं चित्रभानुं त्र्यक्षमहिदत्तं विश्वरूपं चेत्येकादश रुद्रानिव सोमामृतरसशीकरच्छुरितमुखान्पवित्रान्पुत्रान्~। अलभत च चित्रभानुस्तेषां मध्ये राजदेव्यमिधानायां ब्राह्मण्यां बाणमात्मजम्~। स बाल एव विधेर्बलवतो वशादुपसंपन्नया व्ययुज्यत जनन्या~। जातस्नेहस्तु नितरां पितैवास्य मातृतामकरोत्~। अवर्धत च तेनाधिकतरमेधीयमानधृतिर्धाम्नि निजे~।

कृतोपनयनादिक्रियाकलापस्य समावृत्तस्य चतुर्दशवर्षदेशीयस्य पितापि श्रुतिस्मृतिविहितं कृत्वा द्विजजनोचितं निखिलं पुण्यजातं कालेनादशमीस्थ एवास्तमगात्~। संस्थिते च पितरि महता शोकेनाभीलमनुप्राप्तो दिवानिशं दद्यमानहृदयः कथंकथमपि कतिपयान्दिवसानात्मगृह एवानैषीत्~। गते च विरलतां शोके शनैः शनैरविनयनिदानतया स्वातंत्र्यस्य, कुतूहलबद्दलतया च बालभावस्य, धैर्यप्रतिपक्षतया च यौवनारम्भस्य, शैशवोचितान्यनेकानि चापलान्याचरन्नित्वरो बभूव~। अभवंश्चास्य वयसा समानाः सुहृदः सहायाञ्च~। तथा च~। भ्रातरौ पारशवौ चन्द्र सेनमातृषेणौ, भाषाकविरीशानः परं मित्रम्, प्रणयिनौ रुद्रनारायणौ, विद्वांसौ वारबाणवासवाणी, वर्णकविर्वेणीभारतः, प्राकृतकृत्कुलपुत्रो वायुविकारः, बन्दिनावनङ्गबाणसूचीवाणी, कात्यायनिका

\vspace{2mm}
\hrule

\noindent
{\s सुदर्शनं च~। नन्दकः खङ्गच बाहवोऽपि चत्वारः~। अचलकुलस्थितिरभित्रवर्षमर्यादः~। अचलानां गिरीणां कुलैर्वृन्दैः स्थितिर्यस्य~। चतुरुदधिवत्तैश्च गम्भीरः~। अग्रजन्यानो द्विजाः~। सोभस्तृणभेदः, इन्दुश्व~। उपसंपन्ना मृताः~। निजे धाम्नि स्वे गृहे~।

उपनयनं मेखलादानम्~। समावृत्तो निष्पादितवृत्तः~। स्नातक इत्यर्थः~। वेदवेदाङ्गपाठक इत्यन्ये~। ईषदसमाप्तश्चतुर्दशवर्षश्चतुर्दशवर्षदेशीयः~। {\qt श्रुतिस्तु वेदो विज्ञेयो धर्मशास्त्रं तु वै स्मृतिः}~। दशामुपेतो दशमीस्थ उदाहृतः, न दशमीस्थः~। अपूर्णायुरित्यर्थः~। संस्थितो मृतः~। आभीलं कष्टम्~। इल्वरो गमनशीलः~। {\qt अभयंच} इत्यादिनात्मनस्तथा भूतकला वित्संपर्कमैश्वर्यातिशय दर्शयति~। पारशवो द्विजः शुद्रायां जातः~। {\qt परस्त्रीपरश्वम्} इति विद्याद्यञ्~। परश्वादेराश्च~। भाषागेयवस्तुवाचस्तेषु कविः~। गाथादिषु गीतिद इत्यर्थः~। अपभ्रष्टगीतविद्यः~। पञ्चाश\textendash}

\newpage
% ४२ हर्षचरिते 

\noindent
चक्रवाकिका, जाङ्गुलिको मयूरकः, ताम्बूलदायकञ्चण्डकः, भिषक्पुत्रो मन्दारकः, पुस्तकवाचकः सुदृष्टिः, कलादश्चामीकरः, हैरिक: सिन्धुपेण:, लेखको गोविन्दकः, चित्रकद्वीरवर्मा, पुस्तकृत्कुमारदत्त:, मार्दङ्गिको जीमूतः, गायनौ सोमिलग्रहादित्यौ, सैरन्ध्री कुरङ्गिका, वांशिकौ मधुकरपारावतो, गान्धर्वोपाध्यायो दर्दुरकः, संवाहिका केरलिका, लासकयुवा ताण्डविकः, आक्षिक आखण्डलः, कितवो भीमकः, शैलालियुवा शिखण्डकः, नर्तकी हरिणिका, पारशरी सुमतिः, क्षणको वीरदेवः, कथको जयसेनः, शैवो वक्रघोणः मन्त्रसाधकः करालः, असुरविवरव्यसनी लोहिताक्षः, धातुवादविद्विहंगमः, दार्दुरिको दामोदरः, ऐन्द्रजालिकश्चकोराक्षः, मस्करी ताम्रचूडः~। स एतैश्चान्यैश्चानुगम्यमानो बालतया निम्नतामुपगतो देशान्तरालोकनकौतुकाक्षिप्त हृदयः सत्स्वपि पितृपितामहोपात्तेषु ब्राह्मणजनोचितेषु विभवेषु सति चाविच्छिन्ने विद्याप्रसङ्गे गृहान्निरगात्~। अगाव निरवग्रहो ग्रहवानिव नवयौवनेन खैरिणा मनसा महत्तामुपहास्यताम्~।

अथ शनैः शनैरत्युदारव्यवहृतिमनोहन्ति बृहन्ति राजकुलानि वीक्षमाणः, निरवद्यविद्याविद्योतितानि च गुरुकुलानि सेवमानः, महालापगम्भीरगुणवद्गोष्ठीचोपविष्ठमानः, स्वभावगम्भीरधीर्घ\textendash

\vspace{2mm}
\hrule

\noindent
{\s द्वर्षदेशीयां वीरां संस्थितभर्तृकाम्~। वदन्ति कात्यायनिकां धृतकाषायवाससम्~॥ जाङ्गुलिको गारुडिकः~। भिषग्वैद्यः~। {\qt स्वर्णकारः कलादः स्यात्तद्ध्यक्षस्तु हैरिकः}~। पुस्तकृल्लेप्यकारः~। {\qt प्रसाधनोपचारज्ञा सैरन्ध्री स्ववशा स्मृता}~। संवाहिका या पादादिमर्दनं विधत्ते~। लासको नर्तयति यः~। युवेत्यादिना वयसः समानत्वमुच्यते~। अक्षैव्यतीत्याक्षिको द्यूतकारः~। कितवो धूर्तः~। शैलाली स्वयं यो नृ त्यति नटः~। पारशरी भिक्षुः~। असुरविवरव्यसनी पातालाभिलाषी~। धातुवाद विद्रसवादज्ञः~। मस्करी परिव्राट्~। निम्नतामस्वातन्त्र्यम्~। {\qtt कौतुकेति}~। न पुनरर्थभिलिप्सया~। एतदेव सत्स्वपीत्यादिना प्रकाशयति~। निरवग्रहः स्वतन्त्रः~। ग्रहवान्भूतगृहीतः~। स्वैरिणा स्वतन्त्रेण~।

अत्युदारेत्यादिप्रकृतोपयोगी यस्मात्कविना तथाविधवस्तुवेदिनावश्यमेव भवितव्यम्~। वीक्षमाण इत्यनेनात्मनः किमपि प्रकृटमुत्कर्षातिशययोगित्वमाह~। अथ च वीक्षमाणो न तु गुरुकुलवत्सेवमानः~। गाहमान इत्यनेन तेजखित्वमाहात्मनः~। वैपश्चितीं विद्वजनोचिताम्~। संस्तव आदरः~। ज्ञातीनां कर्म ज्ञातेयं बन्धुलम्~। कवि\textendash}

\newpage
% प्रथम उच्छ्वासः~। ४३ 

\noindent
नानि विदृग्धमण्डलानि च गाहमानः, पुनरपि तामेव वैपश्चितीमात्मवंशोचितां प्रकृतिमभजत्~। महतश्च कालात्तामेव भूयो वात्स्यायनवंशाश्रयामात्मनो जन्मभुवं ब्राह्मणाधिवासमगमत्~। तत्र च चिरदर्शनादभिनवीभूतस्नेहसद्भावैः ससंस्तवप्रकटितज्ञायैराप्तैरुत्सवदिवस इवानन्दिताभिगमनो बालमित्रमण्डलस्य मध्यगतो मोक्षसुखमिवान्वभवत्~।

\begin{center}
{\s इति श्रीबाणभट्टकृतौ हर्षचरिते वात्स्यायनवंशवर्णनं नाम प्रथम उच्छ्वासः~।}
\end{center}

\hrule

\noindent
{\s ज्ञायोष्ठक्~। {\qtt आप्तैरिति}~। बन्धुभिर्योगिभिश्र~। योगिपक्षे बाल इव बालो मित्रो रविर्निस्तेजस्त्वात्~। उक्तं च \textendash\ {\qt तपस्यन्तं रविं दृष्टा निस्तेजा जायते रविः~। मोक्षमार्गप्रयत्ने नु तेजो नैवास्य विद्यते~॥} इति~। मित्रं सखा, सूर्य मित्रः~। मण्डलं समूहः, बिम्बम्~। मोक्षसुखमपि सूर्यबिम्बगतैरनुभूयत इति~। आख्यायिकासु कविभिर्निजवंशवर्णनं कानने तथा वंशः ख्यापितः स्यादिति~। आत्मनश्च विटवर्णनम्~। सकलकलाकौशलं ममास्तीति~। हर्षस्य चरिते च वर्णयितव्ये नाम प्रस्तुतं चैतदिति शिवम्~॥}

\begin{center}
{\s इति श्रीशंकरकविरचिते हर्षचरितसंकेते प्रथम उच्छ्वासः~।}
\end{center}

\vspace*{\fill}

\begin{center}
\rule{0.2\linewidth}{0.5pt}
\end{center}

\vspace*{\fill}

\newpage
% ४४ हर्षचरिते

\begin{center}
{\s द्वितीय उच्छ्वासः~।}
\end{center}

\vspace{-5mm}
\begin{quote}
{\ha अतिगम्भीरे भूपे कृप इव जनस्य निरवतारस्य~।\\
दधति समीहितसिद्धिं गुणवन्तः पार्थिवा घटकाः~॥~१~॥

रागिणि नलिने लक्ष्मी दिवसो निद्धाति दिनकरप्रभवाम्~।\\
अनपेक्षितगुणदोषः परोपकारः सतां व्यसनम्~॥~२~॥}
\end{quote}

\vspace{-5mm}
अथ तत्रानवरताध्ययनध्वनिमुखराणि, भस्मपुण्डूकपाण्डुरललाटैः कपिलशिखानालजटिलै: कृशानुभिरिव क्रतुलोभागतैर्बदुभिरध्यास्यमानानि, सेकसुकुमारसोमकेदारिकाहरितायमानप्रघनानि, कृष्णाजिनविकीर्णशुष्यत्पुरोडाशीयश्यामाकतण्डुलानि, बालिकाविकीर्यमाणनीवारबलीनि, शुचिशिष्यशतानीयमानहरितकुशपूलीपलाशसमिन्धि, इन्धनगोमयपिण्डकूटसंकटानि, आमिक्षीयक्षीरक्षारिणीनामग्रिहोत्रधेनूनां खुरवलयैर्विलिखिताजिरवितर्दिकानि, कामण्डलव्यमृत्पिण्डमर्दनव्यग्रयतिजनानि, वैतानवेदीशङ्कव्यानामौदुम्बरीणां शाखानां राशिभिः पवित्रितपर्यन्तानि, वैश्वदे\textendash

\vspace{2mm}
\hrule

{\s {\qtt अतीत्यादि}~। यस्य कोधादिभावगण इङ्गितादिना परेण न चेयते स गम्भीरः~। उक्तं च \textendash\ {\qt यस्य प्रसादादाकारात्को वर्ष भयादयः~। भावस्था नोपलभ्यन्ते तद्गाम्भीर्यमुदाहृतम्~॥} इति~। अगाधश्च~। अवतरणमवतारः, प्रवेशनम्~। अवतरन्ति येनेत्यवतारः, सोपानादिश्च~। समीहितसिद्धिं राजगृह आत्मनः प्रवेशलक्षणम्, जलग्रहणलक्षणं च~। गुणा औदार्यादयः, आकर्षणरज्जवश्च~। पार्थिवा राजानः, पृथ्वीविकाराध~। घटयन्ति वाञ्छितेन प्रयोजयन्तीति घटकाः, कुम्भाश्च~। अनेन तादृशे राशि वाणस्य कृष्ण एव समीहितसिद्धीराध्यास्यत इति सूचितम्~॥~१~॥

रागिणि रक्ते, विषयाभिषङ्गिणि च~। लक्ष्मी शोभाम्, समृद्धिं च~। अत्र नलिनादिकमप्रस्तुतम्, बाणाधारत्वप्रस्तुताः~। अनेन कृष्ण ईदृशे बाणे राजप्रभवां श्रियं निधास्यतीत्युक्तम्~॥~२~॥

{\qtt अथेत्यादि}~। बाणो बान्धवानां भवनानि भ्रमन्सुखम तिष्ठदिति संबन्धः~। शिखा चूडा, ज्वाला च~। सोमो यज्ञियं द्रव्यम्~। केदारिकं खल्पं क्षेत्रम्~। प्रघटनेषु तथोचितवात्~। अहरिता हरिता: संपद्यमाना हरितायमानाः~। लोहितादित्वात्स्यप्~। प्रघनान्यङ्गनानि~। {\qt उशन्ति प्रधनाभिख्यामेकदेशे तु वेश्मनः}~। पुरोडाशीयेत्यादि सहितेऽर्थईयः~। बालिकाः कुमार्यः~। नीवारा अकृष्टपच्या ब्रीहयः~। कूटो राशिः~। आमिक्षीयमिति तप्ते पयसि दध्यानयति सा वैश्वदेव्यामिक्षा~। {\qt आमिक्षा सा शृतोष्णे या क्षीरे स्याद्दधियोगतः} इति~। तस्यै हितमामिक्षीयम्~। आमिक्षाप्रकृतित्वमस्य च योग्यत्वात्~। अग्निहोत्रेषु तस्या अनान्नातलात्, यद्वा यदन्नस्य जुहुयादिति~। तस्या}

\newpage
\fancyhead[CO]{द्वितीय उच्छ्वासः~।}

\noindent
बपिण्डपङ्किपाण्डुरितप्रदेशानि, हविर्धूमधूसरिताङ्गनविटपिकिसलयानि, वत्सीयबालकलालितललत्तरलतर्णकानि, क्रीडत्कृष्णशारच्छागशावकप्रकटितपशुबन्धप्रबन्धानि, शुकशारिकारख्याध्ययनदीयमानोपाध्यायविश्रान्तिसुखानि, साक्षात्रयीतपोवनानीव चिरदृष्टानां बान्धवानां प्रीयमाणो भ्रमन्भवनानि, सुखमतिष्ठत्~।

तंत्रस्थस्य चास्य कदाचित्कुसुमसमययुगमुपसंहरन्नजृम्भत ग्रीष्माभिधानः संफुल्लमल्लिकाधवलागृहासो महाकालः~। प्रत्यग्रनिर्जितस्यास्तमुपगतवतो वसन्तसामन्तस्य वालापत्येष्विव पयःपायिषु नवोद्यानेषु दर्शितस्त्रेहो मृदुरभूत्~। अभिनवोदितश्च सर्वस्यां पृथिव्यां सकलकुसुमबन्धनमोक्षमकरोत्प्रतपन्नुष्णसमयः~। स्वयमृतुराजस्याभिषेकाद्रीश्चामरकलापा इवागृह्यन्त कामिनीनां चिकुर\textendash

\vspace{2mm}
\hrule

\noindent
{\s अपि हवनं संभवत्येव~। वलयैः समूहैः~। वितर्दिका वेदिका~। कमण्डलुर्मुनिकरकस्तस्मै हिताः कामण्डलव्याः~। यत्~। {\qt उगवादिस्यो यत्}~। यतीनां निष्किचनत्वादादरत्वाच स्वयंकरणम्~। वितानो यज्ञः~। तत्र भवा वैतानी यज्ञाग्निकार्यभूः~। शङ्कुः कीलकः~। तस्मै हितः शङ्कव्यः~। {\qtt औदुम्बरीणामिति}~। तासां यज्ञियत्वात्~। वत्सेभ्यो हिता वत्सीयाः~। वत्सपरिचर्याश्चतुराः~। तर्णकाः सद्योजाता वत्साः~। कृष्णशारेति छागविशेषणम्~। तदुक्तम् \textendash\ {\qt लोहितसारङ्गः कृष्णासारङ्गो वा} इति सारङ्गशब्दः शबले वर्तते~। कृष्णशारा मृगा इति केचित्~। तत्तु न तेषां तदानुपयुक्तत्वात्~। पशुबन्धा यज्ञाः~।

कुसुमसमयो वसन्तः स एव युगं कल्पस्तलक्षणं वा युगं मासद्व्यम्~। समुत्फुल्लमल्लिकाभिर्धवला अट्टा विक्रयस्थानानि तेषां विकासो यत्र~। अन्यत्र तद्वदट्टहास उद्धतं हसितं यस्य~। शब्दशक्तिमूलानुरणनव्यङ्गयरूपो ध्वनिश्च~। प्रकृतवर्णने त्यन्यदप्यत्र प्रतीयते~। न वाच्यतया~। तथा च \textendash\ महाकालः साहासः कल्पमुपसंहरञ्जभते मुखं च विदारयति~। महान्कालो ग्रीष्माख्यः, भैरवध~। पयो जलम्, क्षीरं च~। बालापत्यपक्षे \textendash\ नवमुद्यानमुद्गमनं येषां तेषु~। इदं प्रथमतयागमनप्रवृत्तेष्वित्यर्थः~। दर्शितनेह इत्यनेनास्य विजिगीषुव्यवहार आरोपितः~। निर्जितस्य च पुनः प्रतिष्ठापनमेव युक्तम्~। स्नेहः आर्द्रता, प्रीतिश्च~। मृदुरकठोरः, सदयश्च~। अभिनवोदित इति~। साधारणं विशेषणम्~। वासन्तिकपुष्पाभिप्रायेण सकलपदबन्धनं वृन्तकारी व प्रतपन्त्रकर्षेण तपन, अन्यत्र शत्रुहृदयेषु प्रतापं जनयन्~। अभिनवोदित इति~। साधारणं विशेषणम्~। वासन्तिकपुष्पाभिप्रायेण सकलपदबन्धनं वृन्तकारी च प्रतपन्त्रकर्षेण तपन्, अन्यत्र शत्रुहृदयेषु प्रतापं जनयन्~। अभिनवोदितव राजा बन्धनमोक्षं करोति~। उक्तं हि \textendash\ {\qt युवराजाभिषेके वा परचक्रावरोपणे~। पुत्रजन्मनि वा मोक्षो बन्धनस्य विधीयते~॥} इति~। आदरप्रतिपादनाय स्वयंदशब्दः~। अभिषेकः स्नानम्~। अन्यत्र मङ्गलजलपातन तत्संपर्कवशाचाईलम्~।}

\lfoot{ह० ५}

\newpage
\lfoot{}
\noindent
चयाः कुसुमायुधेन~। हिमदग्धसकलकमलिनीकोपेनेव हिमालयाभिमुख यात्रामदादंशुमाली~।

अथ ललाटंतपे तपति तपने लिखितललाटिकापुण्डकैरलकचीरचीवरसंवीतैः स्वेदोदबिन्दुमुक्ताक्षवलयवाहिभिर्दिनकराराधननियमा इवागृह्यन्त ललनाललाटेन्दुभिः~। चन्दनधूसराभिरसूर्यंपश्यामिः कुमुदिनीभिरिव दिवसमसुध्यत सुन्दरीभिः~। निद्रालसा रत्नालोकमपि नासहन्त दृशः, किमुत जरठमातपम्~। अशिशिरसमयेन चक्रवाकमिथुनाभिनन्दिताः सरित इव तनिमानमानीयन्त सोडुपाः शर्वर्यः~। अभिनवपटुपाटलामो सुरभिपरिमलं न केवलं जलम्, जनस्य पवनमपि पातुमभूमिलाषो दिवसकरसंतापात्~।

क्रमेण च खरखगमयूखे, खण्डितशैशवे, शुष्यत्सरसि, सीदत्स्रोतसि, मन्दनिर्झरे, झिल्लिकाझांकारिणि, कातरकपोतकूजि\textendash

\vspace{2mm}
\hrule

\noindent
{\s चिकुराः केशाः~। ते हि तदा स्नानार्द्रतया संयमनात्सुन्दरतया विशेषतः शृङ्गारमुद्दीपयन्ति~। तथा च महाकनेः कालिदासस्य \textendash\ {\qt स्नानाद्रमुक्तेष्वनुधूपवासं विन्यस्तसायंतनमलिकेषु}~। कामो वसन्तात्ययमन्दवीर्यः केशेषु लेभे रतिमझनानाम्~॥ यथा वा~। राजशेखरस्य \textendash\ {\qt तदात्वे स्नातानां दरदलितमल्लीमुकुरिणाम्} इत्यादि~। हि माभिप्राये च हिमालयग्रहणम्~। अंशून्मलति धारयतीत्यनेन हिमं प्रति भवनशीलत्वमस्योच्यते~।

ललाटं तपतीति ललाटंतपः इति खश खरतर इत्यर्थः~। ललाटेऽलंकारो ललाटिका~। {\qt कर्णललाटात्कनलंकारे}~। ललाटिकैव पुण्ट्रकं तिलकमिति सर्वत्र रूपकम्~। संवीतैः प्रावृतैः~। चन्दनेन च तद्वदूसराः~। {\qtt असूर्यपश्याभिरिति}~। आतपासहिष्णुतया~। अन्यत्र स्वभावात्~। दिवसं सुप्यत इति द्रव्यकर्मणि लादिवि धानात्कर्मणि द्वितीयैव~। भावे लः~। यदा तु कर्माप्याख्याततया विवक्ष्यते, तदा दिवसः सुप्यत इति~। भाव्यमिति निर्णीतम्~। खापो निद्रा, मुकुलता च~। जरठं कठोरम्~। यतो ग्रीष्मेण तनूकृता अत आह \textendash\ {\qtt चक्रवाकेत्यादि}~। रात्रौ किल न्वक्रवाकानां वियोगो भवतीत्यल्पतया तैस्ता अभिनन्द्यन्ते~। सरितश्च वृत्तिकारिकास्तेषामिति तदभिनन्दनम्~। उड्डुपः शशी, लवथ~।

क्रमेण चेत्यादावेवंविधै निदाघकाले कठोरीभवति सत्युन्मत्ता मातरिश्वानः प्रावर्तन्तेति संबन्धः~। खगो रविः~। शुष्यदिति साभिप्रायम्~। स्रोतसच प्रसरणधर्मवादाह \textendash\ {\qtt सीददिति}~। समन्तादावेगगामिनः~। झिल्लिका चीरीनामकः प्राणी यो वर्षासु तरुषु सीत्कारमुचैः करोति~। {\qtt कातरेति}~। कपोता हि मेदोम\textendash}

\newpage
% द्वितीय उच्छ्वासः~। ४७

\begin{sloppypar}
\noindent
तानुबन्धबधिरितविश्वे, विश्वसत्पतत्रिणि, करीषंकषमरुति, विरलवीरुधि, रुघिरकुतूहलिकेसरिकिशोरकलिह्यमानकठोरघातकीस्तबके, ताम्यत्स्तम्बेरमयूथवमथुतिम्यन्महामहीधरनितम्बे, दूयमानद्विरदद्दीनदानाश्यानश्यामिकालीनमूकमधुलिहि, लोहितायमानमन्दारसिन्दूरितसीनि, सलिलस्यन्दसंदोहसंदेहमुह्यन्महामहिषविषाणकोटिविलिख्यमानस्फुटत्स्फाटिकटषदि, धर्मममेरितगर्मुति, तप्तपांशुकुकूलकातरविकिरे, विवरशरणश्वाविधि, तटार्जुनकुररकूटज्वरनिवर्तमानोत्तानशफरशारपङ्कशेष पल्वलाम्भसि, दावजनितजगन्नीराजने, रजनीराजयक्ष्मणि, कठोरीभवति निदाघकाले, प्रतिदिशमाटीकमाना इवोषरेषु प्रपावाटकुटीपटलप्रकटलुण्ठकाः,
\end{sloppypar}

\vspace{2mm}
\hrule

\noindent
{\s यत्वान्नितान्तं धर्मासहाः~। अत एव पतत्रित्वेऽपि पृथगुपादानम्~। पतत्रित्वाभिप्रायेण श्वासमित्येतावदेव समुचितम्~। एषां तथाभूतरुजाभावात्~। करीषो गोमयम्~। वीरुत्सपर्णशाखाजटिलं कुप्यकादि~। {\qtt किशोरकेति}~। बालत्वेन तृष्णायसहिष्णुता, मुग्धतातिशयश्च द्योल्यते~। धातकी लताभेदः~। स्तबकः पुष्पगुच्छः~। स्तम्बेरमो इखि~। चमथुः करिकरशीकरः~। तिम्यन्त आद्रीभवन्तः~। नितम्बाः सानवः~। द्विरदाः करिणः~। दीनं क्षीणम्~। आइयाना अप्रसरणधर्मकत्वादीषच्छुष्कश्यामिका मदलेखासंबन्धिनी~। लीना अतितर्षाच्छुिष्टाः~। मूका गुजितहीनाः~। अलोहिता लोहिता भवन्तो लोहितायमानाः~। मन्दाराः पारिभद्रद्रुमाः~। सिन्दूरिता आहिततिन्दूरा इव~। लोहितत्वात्~। ग्रामस्य प्रामान्तरेण मर्यादा सीमा~। स्वन्दः श्रुतिः~। विलिख्यमाना विपाठ्यमानाः~। मर्मरिताः शुष्कत्वेन शब्दायमानाः~। गर्मुतो लताः~। कुकूलं तुषामिः~। विकिराः कुक्कुटायाः~। श्वाविधः शललाः सेहिकाख्या हिंसाः प्राणिनः~। तटशब्देन नैकट्यमाह~। अर्जुनाः ककुभवृक्षाः~। कुरराः कौञ्चपक्षिणः~। कूटः शब्दः एव संतापकारिवाज्ज्वरस्तेन स्फुरन्तः शफरा मत्स्यास्तैः~। शारं सितोदरत्वात्~। पल्वले नडूले~। कुररास्तटस्था यदा कूजन्ति तदा मत्स्याः पीडिताः सन्त उत्लवन्तीति वस्तुधर्मोऽयम्~। {\qtt नीराजनमिति}~। नीराजनं शान्तिकर्म~। राजयक्ष्मा क्षयव्याधिः~। शनैः शनैरपचयकारित्वात्~। मातरिश्वानः कीदृशाः प्रावर्तन्तेत्याह \textendash\ {\qtt प्रतिदिशमित्यादि}~। आदीकमाना उचैत्रमन्तः~। साभिप्रायमेतत्~। रजोवशादेतेषां तथाविधसंनिवेशात्~। ग्रीष्मे क्षेत्र विधा मारुताः प्रावर्तन्तेति कालधर्मः~। उन्मत्तपक्षे \textendash\ आटीकमाना इत्यादि सर्व वक्ष्यमाणयोग्यतया योजनीयम्~। उद्धृतभ्रमणाद्या ह्युन्मादस्यानुभावाः~। तदुक्तम् \textendash\ {\qt अनिमित्त सितरुदितोत्कृष्टाबद्धप्रलापशयनोत्थितप्रवावितवृत्तगीतपठितस्मितपखवधूनन निर्माल्यचीरघटवक्रशरावाभरणस्पर्शनोपभोगैरन्यैथाव्यवस्थित्तचेष्टानुकारणादिभिरनुभावैरभिनयेत्} इति~। ऊपरं}

\newpage
% ४८ हर्षचरिते 

\noindent
प्रपक्वकपिकच्छूगुच्छच्छटाच्छोटनचापलैरकाण्डकण्डूला इव कर्षन्तः शर्करिलाः कर्करस्थलीः, स्थूलहपञ्चूर्णमुचः, मुचुकुन्दकन्द लदलनद्न्तुराः समन्ततः पतन्मुखरचीरीगणमुखशीकरशीक्यमानतनवः, तरुणतरतरणितापतरले तरन्त इव तरङ्गिणि मृगतृष्णिकातरङ्गिणीनामलीकवारिणि, शुष्यच्छमीमर्मरमारवमार्गलङ्घनलाववजवजङ्घालाः, रैणवावर्तमण्डलीरेचकरासरसरभसारब्धनर्तनारम्भारभटीनटाः, दावदग्धस्थलीमषीमलनमलिनाः, शिक्षितक्षपणकवृत्तय इव वनमयूरपिच्छचयानुञ्चिन्वन्तः, सप्रयाणगुजा इव

\vspace{2mm}
\hrule

\noindent
{\s सिकताबहुलो रूक्षो देशः~। प्रपा सत्रम्~। वाटः कुनालम्~। पटलं छदिः~। कपिकच्छुः कण्डूदायको द्रव्यभेदः~। अत एवाह \textendash\ {\qtt कर्षन्त इति}~। शर्कराः पाषाणकणिका विद्यन्ते यासु ताः शर्करिलाः~। पिच्छादित्वादिलच्~। कर्करस्थली ऊपरभूः, पाषाणभूः~। अत एवाह \textendash\ {\qtt स्थूलेत्यादिना}~। मुचुकुन्दं पुष्पभेदः~। कन्दलं नवनालम्~। {\qtt दन्तुरा इति}~। कपिकच्छूस्पर्शचालनेन च ये कण्डूलास्तादृशावर्णमुचो प्रकटदन्ताः परुषं कषन्ति~। शीक्यमानाः सिच्यमानाः~। तरुणतरः प्रौदः~। तरणिरादित्यः~। {\qtt तरन्त इवेति}~। वालुकावशात्तथा लक्ष्यमाणत्वात्~। मृगतृष्णिका मरीचिका~। तृषितमृगाणां रविरश्मिखचितासु सिकतासु नीलत्वदर्शना जलबुद्धिः~। {\qtt वारिणीति}~। सतरने वारिणि ये सभीकास्ते सतापं देशं तरन्ति~। उन्मत्तपक्षेऽपि विचित्तत्वेनैवंकारित्वम्~। शम्योऽग्निगर्भा वल्लीभेदाः~। लाघवं नैपुणम्~। सव्या यामाश्च विषमं मार्गे लाघवेन तरन्ति~। जडाला वेगवन्तः~। रैणवावर्ताः पांसुसंबन्धिन आवर्तनरूपाः संनिवेशारतेषां मण्डली समूहः~। रेचयति पृथकरोतीति रेचकम्~। रैणवावर्तमण्डल्या रेचकं तथा रासे रसिते यो रसस्तेन यो रभसस्तद्वशेनारव्वं यन्नर्तनमिव नर्तनं तदारम्भे विषय आरभटीनटा इव~। आरभटीनटाः~। इयरतेति अराः~। अराव ते भटा अरभटाः~। तेषामियमारभटी नटजातिविशेषो वीररसप्र धानः~। उक्तं च \textendash\ {\qt प्लुष्टावपातनुतगर्जितानि च्छेद्यानि मायाकृतमिन्द्रजालम्~। चित्राणि यूथानि च यत्र नित्यं तां तादृशीमारभटीं वदन्ति~॥} इति नृत्तपक्षे \textendash\ आवर्ता आवृत्तयः~। यदाह मुनिः \textendash\ {\qt यदा नृत्तवशादङ्गं भूयोभूयो निवर्तते~। तत्राद्यमभिनेयं स्याच्छेषं नृत्ते नियोजयेत्~॥} इति~। मण्डलीनृत्तं हलीमकम्~। यदाह \textendash\ {\qt मण्डलेन तु यन्नृत्तं हलीम कमिति स्मृतम्~। एकस्तत्र तु नेता स्यानोपस्त्रीणां यथा हरिः~॥} इति~। रेचकास्त्रयः \textendash\ कटीरेचकः, हस्तरेचकः, ग्रीवारेचकश्चेति~। रासलक्षणम् \textendash\ {\qt अष्टौ षोडशद्वात्रिंशयत्र नृत्यन्ति नायकाः~। पिण्डीबन्धानुसारेण तन्नृत्तं रासकं स्मृतम्~॥} इति~। अस्यैव तु हलीमकाया विशेषाः~। {\qtt क्षपणकवृत्तय इवेति}~। क्षपणकाश्च मषीमलिना बर्हिपिच्छानि शास्त्रचोदनया वहन्ति~। उन्मत्तपक्षे \textendash\ निर्विवेकतया मयूरपिच्छचय इत्युक्तं प्राक्~। गुञ्जन्तीति गुञ्जा उंकाभेदाः~। उन्मत्तानां नृत्ताव\textendash}

\newpage
% द्वितीय उच्छ्वासः~। ४९

\noindent
शिञ्जानजरत्कर जम जरीचीजजालकैः, समरोहा इवातपातुरवनमहिषनासानिकुञ्जस्थूलनिःश्वासैः, सापत्या इत्रोड्डीयमानजवनवातहरिणपरिपाटीपेटकै, सभ्रुकुटय इव दह्यमानखलधानबुसकूटकुटिलधूमकोटिभिः, सावीचिवीचय इव महोष्ममुक्तिभिः, लोमशा इव शीर्यमाणशाल्मलिफलतूलतन्तुभिः, दद्रुणा इव शुष्कपत्तप्रकराकृष्टिभिः, सिराला इव तृणवेणीविकिरणैः, उच्छ्मश्रव इव धूयमाननवयवशुकशकलशङ्कुभिः, दंष्ट्राला इव चलितशललसूचीशतैः, जिह्वाला इव वैश्वानरशिखाभिः, उत्सर्पत्सर्पक चुकचूडाला ब्रह्मस्तम्भरसाभ्यवहरणाय कबलग्रहमिवोष्णै: कमलमधुभिरभ्यस्यन्तः सकलसलिलोच्छोषधर्मघोषणापटहैरिव शुष्कवेणुवनास्फोटनपटुरवैस्त्रिभुवनविभीषिकामुद्भावयन्तः, च्युतचलचापपक्षश्रेणीशारितसृतयः, त्विषिमन्मयूखलतालातप्लोषकल्माषवपुष इव स्फुटि\textendash

\vspace{2mm}
\hrule

\noindent
{\s सरे सर्व एव करतलादि वादयन्ति~। शिक्षानाः शब्दायमानाः करजो वृक्षमेदः~। प्ररोहोऽङ्कुरः~। उन्मत्ता अपि खेदान्निःश्वसन्ति~। {\qtt सापत्या इवेति~।} उन्मत्ता अपि श्वभ्रादिपतनभयादपत्यानि न लजन्ति~। पेटकैर्यथैः {\qtt सभ्रुकुटय इवेति}~। दह्यमानाभिप्रायेणोक्तम्~। उन्मत्ता अपि क्रोधप्राया एव क्रोधस्य भ्रुकुव्यादयोऽनुभावाः~। खलधानं क्षोदादिदेशः~। क्षुद्यमानं धान्यमियन्ये~। सस्यस्य ज्वालाभावाडूमवर्णनं समुचितम्~। कुटिलपदेन व अकुटीसादृश्यमाह~। अवीचिर्नरकमेदस्तस्य वीचय इव वीचयो ज्वालाः~। {\qtt महोष्मेति}~। उन्मत्ता अपि खेदादिवशदूष्मायन्ते~। {\qtt लोमशा इवेति}~। उन्मत्ता अपि क्षुरकर्मविना लोमशाः~। तूलं कपीसः~। दनः कुष्ठविकारः~। सास्यास्तीति दद्वणः~। {\qt दद्र्वा ह्रस्वत्वं च} इति नः~। उन्मत्ता अप्युद्वर्तनं विना दद्र्युक्ता भवन्ति~। सिरालाः प्रकटनायवः~। उन्मत्ता अपि कृशत्वात्सिराला भवन्ति~। वेणी पङ्किः~। सिरासादृश्यप्रतिपादनाय बेणीपदम्~। इमश्रुः कूर्चः~। शुकाः किंशारवः~। उन्मत्ता अपि केशवपनाभावादीर्घश्मश्रवः दंष्ट्रा चहिर्निर्गता दन्ताः~। शललः श्वावित्~। सूची दीर्घकण्टकरूपाणि रोमाणि~। अन्ये तु \textendash\ दंष्ट्रालाः शललाः, श्वाविधः पक्षाथ शलला उच्यन्ते~। तथा च \textendash\ {\qt श्वाविधः शललैरिव} इति महाभारते दृश्यत इत्याहुः~। उन्मत्ता अप्येवमादिविकारेण सर्वे भीषयन्ते~। एवं जिह्वाला अपि~। एवमेव स्नानादिना विनोन्मुक्तचूडत्वादुत्सर्पदित्यादि~। कञ्जकस्त्वक्~। ब्रह्मस्तम्भो ब्रह्माण्डः~। रसाभ्यवहणं शोषणम्~। रसानां च मधुरादीनां भोजनम्~। {\qt असंचार्यो मुखे पूर्णे गण्डूषः कवलोऽन्यथा}~। अभ्यस्यन्तीति~। एवमिदं शोषयिष्याम इति~। धर्मो ग्रीष्मः~। घोषणा श्रावणा~। {\qtt विभीषिकामिति}~। ये सगर्वा जगद्रसनशीलास्ते त्रिभुवनेऽपि भयमुत्पादयन्ति~। चाषः किकीदिविः पक्षिभेदः~। उन्मत्त पक्षे \textendash\ विस्मरणशीललायुतेत्यादि योज्यम्~। सृतिर्मार्गः~। विषिमान्रविः~। अलातमु\textendash}

\newpage
% ५० हर्षचरिते~। 

\noindent
तगुञ्जाफलस्फुलिङ्गाङ्गाराङ्किताङ्गाः, गिरिगुहागम्भीरझांकारभीषणभ्रान्तयः, भुवनभस्मीकरणाभिचारचरुपचनचतुरा रुधिराहुतिभिरिव पारिभद्रद्रुमस्तबकवृष्टिभिस्तर्पयन्तस्तारवान्वनविभावसून, अशिशिरसिकतातारकितरंहसः, तप्तशैलविलीयमानशिलाजतुरसलवलिप्तदिशः, दावदहनपच्यमानचटकाण्डखण्डखचिततरुकोटरकीटपटलपुटपाकगन्धकटवः, प्रावर्तन्तोन्मत्ता मातरिश्वानः~।

सर्वतश्च भूरिभखासहस्रसंधुक्षणक्षुभिता इव जरठाजगरंगम्भीरगलगुहावाहिवायवः, क्वचित्खच्छन्दतृणचारिणो हरिणाः, क्वचित्तरुतलविवरविवर्तिनो बभ्रवः, कचिजटावलम्बित्तः कपिलाः कचिच्छकुनकुलकुलायपातिनः श्येनाः, कचिद्विलीनलाक्षारसलो\textendash

\vspace{2mm}
\hrule

\noindent
{\s ल्मुकम्~। कल्माषं रक्तकृष्णम्~। गुञ्जा रक्तिकोत्पलानि लोहितकृष्णानि भवन्ति~। स्फुलिङ्गा अभिकणाः~। अङ्कितानि दग्धान्यज्ञानि~। ये च साङ्गारास्ते च मलिनशरीरा भवन्ति~। उन्मत्ता अप्यभिशस्त्रशुभ्रादिषु बलादतिपतन्ति~। झांकारभीषणा भ्रमन्ति च~। अभिचार उच्चाटनम्~। अभिचारिणञ्चोच्चाटन मारणाद्यर्थ चरुपचनं कुर्वन्ति~। रक्तेन चाग्नीन्प्रीणयन्ति~। पारिभद्रा निम्बाः~। मदना इत्यन्ये~। उन्मत्ता अपि निर्विवेकतया रक्तादि यत्किचिदशुचिप्रायमग्निषु निक्षिपन्ति, उत एवं विश्वस्य दोषाय पर्यवस्यन्ति~। तारकितमिव रहो वेगो येषां ते~। शिलाजतुरश्मसः~। दावदहनेन पच्यमानानि यानि चटकाण्डानि तेषां विदारणवशात्स्फुटिता ये खण्डाः कपालानि तैः~। दोलावदुपरिपतितैः खचितानि कचायमानानि यानि तरुकोटरेषु कीटपटलानि किमिसमूहास्तेषामतिपेशलत्वेन यत एव ततैः खण्डैरुपर्याच्छादकतया स्थितैः पुटपाकैः प्रसूतधूमोऽभ्यन्तरपातस्तद्गन्धेन कटव उद्वेजकाः~। अत्राग्निपाकेन खण्डलं खण्डेभ्यो रसनिःसरणात्खचितत्वं कीटानाम्~। {\qtt उन्मत्ता इति}~। ये चोन्मत्तास्ते सिकताव्याप्ताः कर्दमविलिप्तदिशो गन्धकटवः साटीकराद्याः पूर्वोक्ताः क्रिया: प्रायेण कुवैत इति~। सर्वत्रात्र महावाक्ये ध्वनिच्छायान्वेष्या~। मातरिश्वानो वायवः~। 

सर्वतश्चेत्यादौ दावामयः प्रत्यदृश्यन्तेति संबन्धः~। मना हतिः~। संधुक्षणमुद्दीपनम्~। जरठाजगरा वृद्धसर्पाः~। गला एव गुहा गलगुहाः~। खच्छन्दमपविनम्, यथारुचि~। चरणं भक्षणम्, गमनम्~। हरिणाः शुक्लाः, मृगाव~। वभ्रुवः कषिलाः, नकुलाब~। इतरत्र जटामूलानि च कपिलाः पिङ्गलाः~। कपिलाख्यमुनित्रतग्रहणान्मनुष्या एवाभेदोपचारेण कपिलाः~। एते च जटावल्कलधारिणः~। कुलाया नीडाः~। श्येनाः शुक्लाः, पाजिकाच~। अधरा धर्तुमशक्याः, अधोभवा वा~। लाक्षाया विलीनतयापीतत्वात्~। ओष्ठाश्चाधराः~। आ समन्तात्सादिता आहताः, स्वी\textendash}

\newpage
% द्वितीय उच्छ्वासः~। ५१

\noindent
हितच्छवयोऽधराः, क्वचिदासादितशकुनिपक्षकृतपटुगतयो विशिखाः, कचिद्दग्धनिःशेषजन्महेतवो निर्वाणाः, क्वचित्कुसुभवासिताम्बरसुरभयो रागिणः, क्वचित्सधूमोद्द्वारा मन्दरुचयः, क्वचित्सकलजगद्वासघस्मराः सभस्मकाः, क्वचिद्वेणुशिखरलग्नमूर्तयोऽत्यन्तवृद्धाः, क्वचिदचलोपयुक्तशिलाजतवः क्षयिणः, क्वचित्सर्वरसभुज: पीवानः, क्वचिदग्धगुग्गुलवो रौद्राः, क्वचिज्वलितनेत्रगहनदग्धसकुसुमशरमदनाः कृतस्थाणुस्थितयः, चटुलशिखानतेनारम्भारभटीनटाः शुष्ककासारसृतिभिः, स्फुटन्नीरसनीवारबीजलाजवर्षिभिर्व्वालाञ्जलिभिरचर्यन्त इव धर्मघृणिम्, अघृणा इक हठहूयमानकठोरस्थलकमठवसावित्रगन्धगृनवः स्वमपि धूममम्भोदसमुद्भूतिभियेव भक्षयन्तः, सतिलाहृतय इव स्फुटद्रहलबालकीटपटलाः कक्ष्येषु, श्वित्रिण इव प्लोषविचटद्वल्कलधवलशम्बूक\textendash

\vspace{2mm}
\hrule

\noindent
{\s कृताश्च~। स्निग्धतया नीरसतया च~। शकुनीनां पक्षेषु कृतपटुगतयः~। निःसारतया कालस्थापितत्वात्~। विगता शिखा ज्वाला येषां ते, विविधशिखाः शराश्च~। निःशेषाः समस्ताः, प्राक्तनजन्मान्तरसंचिता अपि~। जन्महेतवस्तृणायाः, कर्माणि च~। निर्वाणाः शान्ताः, मोक्षगामिनश्च~। कुसुमं धूमः, पुष्पं च, अम्बरं नमः, वस्त्र च~। रागिणो लोहिताः, शृङ्गारिणश्च~। अजीर्णकृतोऽपि धूमोद्गारः~। रुचिर्दीप्तिः, भोजनाभिलाषथ~। जगदेव ग्रासः कवलं तद्भक्षणशीलाः~। भस्मभूरिकचायशनव्याधिः वृद्धा वृद्धिं गताः, स्थविराच~। ते वेणुशिखरमवलम्बन्ते यष्टिं गृहन्ति~। अचलाः पर्वताः~। अन्यत्र क्षयस्य दीर्घकालपर्यवसायिवादचलमविच्छिन्नं भक्षितशिलाह्वयाः~। उक्तं च \textendash\ {\qt शिलाधातुप्रयोगाद्वा प्रसादाद्वाथ शांकरात्~। अजामूत्रप्रयोगाद्वा क्षयः क्षीयेत च नान्यथा~॥} इति~। क्षयो विनाशः, व्याधिभेदव यक्ष्माख्यः~। रसः सलिलादिः~। अत एव पीवानः~। अन्यथा कथं सलिलादिभक्षणशक्तिलमनीषां प्रसज्येत~। ये च मधुरादिसर्व रसानुपभुञ्जते ते स्थूला भवन्ति~। रौद्रा भीषणाः, रुद्रभक्ताथ~। नेत्राणां मूलानां दहनेन दग्धाः सकुसुमाः काण्डानि मदना वृक्षभेदाच यैः~। स्थाणुछिनशाखो वृक्षः, शिवश्च~। स्थितिः स्थानम्, व्यवहारश्च~। स्थाणुनापि नयनाग्निना सकुसुमशरः कामो दग्धः~। चटुलत्वेन नर्तनाम्भः, रवश्च~। शुष्कलाचटुलादेशरभटीग्रहणम्~। कासाराणि नडलास्तेषु याः सृतयः~। कचित् {\qt स्मृतयः} इति पाठः~। इतस्त्र तु \textendash\ शुष्ककं शुष्कगीतं झण्डमादि~। {\qtt आसार्यन्त इत्यासाराः}~। आसरितानि यद्यपि गीयन्त एव, तथापि {\qt वर्धमानमथापीह ताण्डवं यत्र योज्यते} इति~। ताण्डवं ह्यारभटीप्रधानम्~। अर्चयन्त इवेति~। तेषां तदभिमुखखात्~। धर्मवृणिः सूर्यः~। अवृणा अजुगुप्साः~। कमठः कूर्मः~। {\qt विस्रं स्यादामगन्धि यत्}~। गृध्नचो लम्पटाः~। समुद्भूतिः संभारः~। धूमात्किल मेघोत्पत्तिर्मेघाः रामयन्ति~। कीटाः कृमयः~। प्लोषो}

\newpage
% ५२ हर्षचरिते 

\noindent
शुक्तयः शुष्केषु सर:सु, स्वेदिन इव विलीयमानमधुपटलगोलगलितमधूच्छिष्टवृष्टयः काननेषु, खलतय इव परिशीर्यमाणशिखासंहतयो महोपरेषु गृहीतशिलाकवला इव ज्वलितसूर्यमणिशकलेषु शिलोचयेषु, प्रत्यदृश्यन्त दारुणा दावाग्नयः~।

तथाभूते च तस्मिन्नत्युप्रे ग्रीष्मसमये कदाचिदस्य स्वगृहावस्थितस्य भुक्तवतोऽपराहसमये भ्राता पारशवश्चन्द्रसेननामा प्रविश्या कथयत् \textendash\ एष खलु देवस्य चतुःसमुद्राधिपतेः सकलराजचक्रचूडामणिश्रेणीशाणकोणकषण निर्मलीकृतचरणनखमणेः सवेचक्रवतिनां धौरेयस्य महाराजाधिराजपरमेश्वरश्रीहर्षदेवस्य भ्राता कृष्णनाम्ना भवतामन्तिकं प्रज्ञाततमो दीर्घाध्वगः प्रहितो द्वारमध्यास्ते इति~। सोऽब्रवीत् \textendash\ {\haq आयुष्मन्, अविलम्बितं प्रवेशयैनम्} इति~।

अथ तेनानीयमानम्, अतिदूरगमनगुरुजडजङ्घम्, कार्दमिकचेलचीरिकानियमितोच्चण्डचण्डातकम्, पृष्ठशेपटञ्चरकपेटघटितगलितप्रन्थिम्, अतिनिविडसूत्रवन्धनिम्नितान्तरालकृतव्यवच्छेद्या लेखमालिकया परिकलितमूर्धानम्, प्रविशन्तं लेखहारकमद्राक्षीत्~। अप्राक्षीच दूरादेव \textendash\ {\haq भद्र, भद्रमशेषभुवननिष्कारणबन्धोस्तत्रभवतः कृष्णस्य} इति~। सः {\haq भद्रम्} इत्युक्त्वा प्रणम्य नातिदूरे समुपाविशत्~। विश्रान्तश्चाब्रवीत् \textendash\ {\haq एष खलु स्वामिनो माननीयस्य लेखः प्रहितः} इति विमुच्य चार्पयत्~। अथ वाणः सादरं गृही\textendash

\vspace{2mm}
\hrule

\noindent
{\s दाहः~। वल्कलशब्दस्त्वगुपलक्षणार्थः~। शम्बूकाः शुक्तिमन्तः प्राणिभेदाः~। मधुपटल गोलो माक्षिककरण्डः~। मधूच्छिष्टं सिक्थकम्~। खलतयः खल्वाटाः~। शिखा ज्वाला, चूडा च~। ऊषरं सिकताबहलो रूक्षो देशः~। शिलोच्चयो गिरिः~। {\qt दावो वह्निगतो वह्निविच वनमुच्यते}~।

तथाभूतदेश इत्यादिनात्मानं प्रति तेषामादरातिशयं दर्शयति \textendash\ {\qtt आकुर्वत इति}~। न प्रस्तावे~। एतेन स्वस्य किमपि माहात्म्यमाह~। स्वयमवसरमन्तरेण वा तस्य तदा प्रवेशाभावात्~। एतदेव देवस्येत्यादिविशेषणसंदर्भमुखेन द्वारमध्यास्त इत्यनेन पोषयिष्यते~। पारशवः शूद्रापुत्रः~। शाणो मणिकषणम्~। कोणोऽश्रिः~। चक्रवर्तिनः सार्वभौमाः~। धौरेयो मुख्यः~। प्रज्ञाततमोऽतिप्रतीतः~। एतेन च वार्ण प्रति बहुमान एव गम्यते~।

जडा गमनाशक्ताः~। कर्दमेन रत्तीकार्दमिकम्~। चेलं वस्त्रम्~। चीरिका खण्डिका~। उच्चण्डमुच्चम्~। गाढमित्यन्ये~। चण्डातकमर्धोरुकं वासः~। पटच्चरं जीर्णवस्त्रम्~। निम्नितं नमितम्~। {\qtt लेखमालिकेति}~। अन्यैरपि तद्धस्खे लेखः ग्रहित इति प रागतः संबन्धः~। {\qt परिकरित} इति पाठे वेष्टित इत्यर्थः~। तत्रभवतः पूज्यस्य~। {\qtt नाति}\textendash}

\newpage
% द्वितीय उच्छ्वासः~। ५३

\noindent
त्वा स्वयमेवावाचयत् \textendash\ मेखलकात्संदिष्टमवधार्य फलप्रतिबन्धी धीमद्भिरपहरणीयः कालातिपात इत्येतावदत्रार्थजातम्~। इतरद्वार्तासंवादनमात्रकम्~। अववृतलेखार्थश्च समुत्सारितपरिजनः संदेशं पृष्टवान्~। मेखलकस्त्ववादीत् \textendash\ एवमाह मेधाविनं स्वामी \textendash\ जानात्येव मान्यः यथैकगोत्रता वा, समानजातिता वा, समं संवर्धनं वा, एकदेशनिवासो वा, दर्शनाभ्यासो वा, परस्परानुरागश्रवणं वा, परोक्षोपकारकरणं वा, समानशीलता वा, स्नेहस्य हेतवः~। त्वयि तु विना कारणेनादृष्टेऽपि प्रत्यासन्ने बन्धाविव बद्धपक्षपातं किमपि लिह्यति मे हृदयं दूरस्थेऽपीन्दोरिव कुमुदाकरे~। भवन्तमन्तरेणान्यथा चान्यथा चायं चक्रवर्ती दुर्जनैग्रहित आसीत्~। न च तत्तथा~। न सन्त्येव ते येषां सतामपि सतां न विद्यन्ते मित्रोदासीनशलवः~। शिशुचापलापरा चीनचेतोवृत्तितया च भवतः केनचिदसहिष्णुना यत्किंचिद्सदृशमुदीरितम्~। इतरो लोकस्तथैव तद्गृह्णाति वक्ति च~। सलिलानीव गतागतिकानि लोलानि खलु भवन्त्यविवेकिनां मनांसि~। बहुमुखश्रवणनिश्चलीकृतनिश्चयः किं करोतु पृथिवीपतिः~। तत्त्वान्वेषिभिश्चास्माभिर्दूरस्थितोऽपि प्रत्यक्षीकृतोऽसि~। विज्ञप्तश्चक्रवर्ती त्वदर्थम् \textendash\ यथा प्रायेण प्रथमे वयसि सर्वस्यैव चा पलैः शैशवमपराधीति~। तथेति च प्रतिपन्नं स्वामिना~। अतो भवता राजकुलमकृतकालक्षेपमागन्तव्यम्~। अवकेशीवादृष्टपरमेश्वरो बन्धुमध्यमधिवसन्नासि मे बहुमत्तः~। न च सेवावैषम्यविषादिना~। च मे वा परमेश्वरोपसर्पणभीरुणा भवता भवितव्यम्~। यतो यद्यपि\textendash

\vspace{2mm}
\hrule

\noindent
{\s {\qtt दूर इति}~। अपि तु दूर एवेति सर्वत्रैव खस्य प्रभावातिशचं प्रतिपादयति~। फलं प्रतिबध्नाति रुणद्धीति फलप्रतिबन्धी~। कालातिपातः कालाययः~। अर्थजातमभिधेयप्रकारः~। अवघृतो ज्ञातः~। एकेत्यादि कारणमुत्तरोत्तरमप्रधानम्~। अन्यथा चान्यथा चेति~। एतेन किंचिदेव संभवतीति दर्शयति~। अत एवाह \textendash\ {\qtt न च तत्तथेति}~। तथात्वे तु वाणस्य दुर्वृत्तता प्रसज्येत~। कृष्णस्यापि तादृशः पक्षपातः स्वामिप्रतारणादि च दोषायैव भवेत्~। अत एव वक्ष्यति \textendash\ {\qtt तत्त्वान्वेषिभिरित्यादि}~। ग्राहित इत्येतावति वक्तव्य आसीदित्यनेन दुर्जनाः संपत्तिनिरवकाशा इति प्रतिपादितम्~। अत एव वक्ष्यति \textendash\ {\qtt तथेति च प्रतिपत्नं स्वाभिनेति}~। सतां साधूनामपि~। सतां भवताम्~। उदासीनो मध्यस्थः~। अपराचीनापी चेतोवृत्तिर्यस्याः~। अवकेशी}

\newpage
% ५४ हर्षचरिते 

\begin{quote}
{\ha स्वेच्छोपजातविषयोऽपि न याति वक्तुं\\
देहीति मार्गणशतैश्च ददाति दुःखम्~। \\
मोहात्समाक्षिपति जीवनमध्यकाण्डे\\
कथं मनोभव इवेश्वरदुर्विदग्धः~॥~३~॥}
\end{quote}

\vspace{-3mm}
तथाप्यन्ये ते भूपतयः, अन्य एवायम्~। न्यकृतनृगनलनिषधनहुषाम्बरीषदशरथदिलीपनाभागभरतभगीरथययातिरमृतमयः स्वामी~। नास्याहंकारकालकूटविपदिग्धदुष्टा दृष्टयः, न गर्वगुरुगरगलग्रहगद्गद्गदा गिरः, नातिस्मयोष्मापस्मारविस्मृतस्थैर्याणि स्थानकानि, नोदाम दर्पदाहज्वरवेगविलवा विकाराः, नाभिमानमहासंनिपातनिर्मिताङ्गभङ्गानि गतानि, न मदार्दितवकीकृतौष्ठनिष्ठ्यूतनिष्ठुराक्षराणि जल्पितानि~। तथा च~। अस्य विमलेषु साधुषु रत्न बुद्धिः, न शिलाकलेषु~। मुक्ताघवलेषु गुणेषु प्रसाधनधीः, नाभरणभारेषु~। दानवत्सु कर्मसु साधनश्रद्धा, न करिकीटेषु~। सर्वाग्रेसरे यशसि महाप्रीतिः, न जीवितजरत्तृणे~। गृहीतकरास्खाशासु

\vspace{2mm}
\hrule

\noindent
{\s निष्फलंतरुः~। स चादृष्टरविस्तरुमध्यगो न कस्यचित्प्रियः~। स्वेच्छोपजाता विषया मण्डलानि यस्मात्ताहगपि देहि प्रयच्छेति वक्तुं न पार्यते~। इतरत्र स्वेच्छया स्वसंकल्पेनोपजात उत्पन्नो विषयो गोचरो यस्य~। तथा चोच्यते \textendash\ {\qt काम जानामि ते मूलं संकल्पात्किल जायसे} इति~। अथ च खेच्छाया उपजाता विषया यस्यायं देही च शरीरवानिति वक्तुं न याति~। न शक्यत इति विरोधः~। कामथानङ्गत्वाद्देही शरीरवानिति वक्तुं न युज्यत इत्यन्यार्थः~। मार्गणा याचकाः, शराच मार्गणाः~। जीव्यतेऽनेनेति जीवनम्, ग्रामादि जीवितं च~। ईश्वरो राजा, हरथ~। दुर्विदग्धो दुरूढः, दुष्टत्वा द्विशेषेण दग्धश्च~। अमृतेत्यादि साभिप्रायम्~। यस्मादहंकारादि कालकूटादिना रूपयति, अतवाहंकारादीनामयन्ताभावप्रकाशनेच्छ्यामृतमयनस्य दर्शयति~। अमृतमयस्य च कालकूटादिभिर्न योगः~। गरं विषम्~। स्मयो गर्वः~। स्थानकानि स्थितयः~। अर्दितं वातव्याधिभेदः~। तस्मिन्सति मुखं वकं भवति~। तथा चोक्तम् \textendash\ {\qt वायुः प्रवृद्धत्तैस्तैव नातलैरुर्ध्वमाश्रितः~। वक्रीकरोति वक्तारमुक्तं हसितमीक्षितम्~॥} इति निष्ठधूतानि निर्गतानि~। विमलेष्वपापेषु, अन्यत्र सुच्छायेषु पद्मरागादिष्विति वक्तव्ये शिलेत्यादिपदमादरार्थम्~। एवमुत्तरत्रापि वाच्यम्~। मुक्तवत्ताभिश्व धवलास्तेषु गुणेष्वौदार्यादिषु च, सूत्रेषु च~। प्रसाधनं प्रकृष्टं साधनम्, अर्जनम्, भूषणं च~। दानं धनत्यागः, मदश्व~। साधनं संपादनम्, सैन्यं न्च~। सान्यतेऽनेनेति कृत्वा~। करो दण्डः, पाणिव~। आशा दिशः, चेतः, वाञ्छा}

\newpage
% द्वितीय उच्छ्वासः~। ५५ 

\noindent
प्रसाधनताभियोगः, न निजकलत्रचर्मपुत्रिकासु~। गुणवति धनुषि सहायबुद्धिः, न पिण्डोपजीविनी सेवकजने~। अपि च~। अस्य मित्रोपकरणमात्मा, भृत्योपकरणं प्रभुत्वम्, पण्डितोपकरणं वैदग्ध्वम्, बान्धवोपकरणं लक्ष्मीः, कृपणोपकरणमैश्वर्यम्, द्विजोपकरणं सर्वस्वम्, सुकृतसंस्मरणोपकरणं हृदयम्, धर्मोपकरणमायुः, साहसोपकरणं शरीरम्, असिलतोपकरणं पृथिवी, विनोदोपकरणं राजकम्, प्रतापोपकरणं प्रतिपक्षः~। नास्याल्पपुण्यैरवाप्येत सर्वातिशायिसुखरसप्रसूतिः पादपलवच्छाया इति~। श्रुत्वा च तमेव चन्द्रसेनं समादिशत् \textendash\ {\haq कृतकशिपुं विश्रान्तसुखिनमेनं कारय} इति~।

अथ गते च तस्मिन्, पर्यस्ते च वासरे, संघट्टमानरक्तपङ्कजसंपुटपीयमान इव क्षयिणि क्षामतां व्रजति चालवायसास्यारुणेऽपराहातपे, शिथिलितनिजवाजिजवे जपापीडपाटलेऽस्ताचलशिखरस्खलिते खञ्जतीव कमलिनीकण्टकक्षतपादपल्लवे पतङ्गे, पुरः

\vspace{2mm}
\hrule

\noindent
{\s च~। प्रसाधनं संपादनम्, दण्डश्च~। गुणो ज्या, शौर्यायाश्च गुणाः~। उपक्रियमतेऽनेनेत्युपकरणमुपयोगः~। {\qtt आत्मेति}~। नहि मित्राणि मित्रव्यतिरेकेण चान्धवादिव लक्ष्यादि किंचिदमेक्ष्यन्ते~। {\qtt प्रभुत्वमिति}~। तस्य प्रभुत्वं सेवकादीनां दानसंपादनादि~। यथाह \textendash\ {\qt यथाकालं प्रवर्तन्ते पण्डिताः} इत्यादिवैदग्ध्यमात्रापेक्षया पण्डितानां क्षपणादिवदर्थादनपेक्षितया हि तेषामौचियं न प्रतीयते~। न अनेन पण्डितसामान्यात्तदभिप्रायेण स्वस्य समुचितमेव हेवाकमभिव्यनक्ति वैदग्ध्यापेक्षिवं दर्शयतीति यावत्~। बान्धवाः कुल्याः~। लक्ष्मीश्छलचा मरादिप्रतिपत्तिरूपा छत्रादिवत्तल्या एव~। लभन्तेऽन्येषामनर्हत्वात्~। {\qtt कृपणेत्यादि}~। कृपणानां पोषणमेव समुचितम्~। तत्र चैश्वर्यमेव हेतुः~। ऐश्वर्यमर्थवत्ता~। न तु द्विजातिवदेते सर्वस्वमर्हन्ति~। सर्वशब्देन दारा अप्युच्यन्ते एवमादि तु द्विजा एव लभन्ते~। तद्व्यतिरेकेणान्येषामनर्हत्वात्~। एवं हृदयादि तत्तदभिप्रायेण विचारणीयम्~। सुखमेवास्वाद्यतया रस एव~। रसः सुखरसः~। छाया कान्तिः~। यद्वा छायावत्त्वमेषां सर्वस्य कस्यचिदाश्रयणीयत्वादुपचर्यते~। अल्पेत्याद्यभिप्रायेण पादयोः कल्पवृक्षतुल्यत्वमभिव्यज्यते~। पुण्यवशासदवाप्तेः~। एतत्पक्षे छायातपप्रतिपक्षजातिः~। {\qt भोजनाच्छादने सद्भिरुभे कशिपुरुच्यते}~।

वायसः काकः~। जमा रविप्रियं पुष्पम्~। आपीड: स्तबकः~। कोऽत्रास्तेत्यादिखरूपकथनं क्षतपादपलवलादुत्प्रेक्षणम्~। {\qtt खञ्जतीवेति}~। यथ खञ्जति स शिखरप्राये विषमे पथि~। ये पुनरस्ताचले शिखरस्खलनकारणकं खञ्जनमित्युत्प्रेक्षयन्ते तान्प्रति कमलिनीत्यादि निरर्थकम्~। खञ्जतीय स्खलतीव~। पुरः पूर्वस्यां}

\newpage
% ५६ हर्षचरिते 

\noindent
परापतति प्रेङ्खदन्धकारलेशलम्बालके शशिविरहशोकश्याम इव श्यामामुखे, कृतसंध्योपासनः शयनीयमगात्~। अचिन्तयसैका की \textendash\ किं करोमि~। अन्यथा संभावितोऽस्मि राज्ञा~। निर्निमित्तबन्धुना च संदिष्टमेवं कृष्णेन~। कष्टा च सेवा~। चिपमं च भृत्यत्वम्~। अतिगम्भीरं महद्राजकुलम्~। न च तत्र मे पूर्वजप्रवर्तिता प्रीतिः, न कुलक्रमागता गतिः, नोपकारस्मरणानुरोधः, न बालसेवास्नेहः, न गोत्रगौरवम्, न पूर्वदर्शनदाक्षिण्यम्, न प्रज्ञासंविभागोपप्रलोभनम्, न विद्यातिशयकुतूहलम्, नाकारसौन्दर्यादरः, न सेवाकाकुकौशलम्, विद्वगोष्ठीवन्धवैदग्भ्यम्, न वित्तव्ययवशीकरणम्, न राजवल्लभपरिचयः~। अवश्यं गन्तत्र्यम्~। सर्वथा भगवान्पुरारा तिर्भुवनगुरुतस्य मे सर्व सांप्रतमाचरिष्यति इत्यवधार्य गमनाय मतिमकरोत्~।

अथान्यस्मिन्नहन्युत्थाय, प्रातरेव स्नात्वा, वृतधवलदुकूलवासाः, गृहीताक्षमालः, प्रास्थानिकानि सूक्तानि मन्त्रपदानि च बहुशः समावर्त्य, देवदेवस्य विरूपाक्षस्य क्षीरस्नपनपुरःसरां सुरभिकुसुमधूपगन्धध्वजवलिविलेपनप्रदीपकबहुलां विधाय पूजाम्, परमया भक्त्या प्रथमहुततर लतिलत्वग्विचटनचटुलमुखर शिखाशेखरं प्राज्याज्याहुतिप्रवर्धितदक्षिणार्चिषं भगवन्तमाशुशुक्षणिं हुत्वा, दत्वा द्युन्नं यथाविद्यमानं द्विजेभ्यः, प्रदक्षिणीकृत्य प्राङ्मुखीं नैचिकीम्, शुक्लाङ्गरागः, शुक्लमाल्य:, शुक्लवासाः, रोचनाचित्रदुर्वाग्रपल्लवप्रग्र थितगिरिकार्णकाकुसुमकृतकर्णपूरः, शिखासक्तसिद्धार्थकः, पितुः कनीयस्या स्वस्त्रा मात्रेव स्नेहार्द्रहृदयया श्वेतवाससा साक्षादिव

\vspace{2mm}
\hrule

\noindent
{\s दिशि~। श्यामा रात्रिः, योषिच~। मुखमारम्भः, वदनं च~। निर्निमित्तत्यायभिप्रायेण वक्ष्यति~। अवश्यं गन्तव्यं चेत्यादि~। {\qt काकुः स्त्रियां विकारो यः शोकभीत्यादिभिर्ध्वनेः}~। इह च लक्षणया वक्रोक्तिः~। सांप्रतं युक्तम्~।

अथेत्यादावन्यस्मिन्नहनि प्रीतिकूटान्निरगादिति संबन्धः~। प्रस्थानं प्रयोजनं येषां तानि प्रास्थानिकानि सूतानि, वेदोक्ता मन्त्रविशेषाः~। विरूपाक्षत्र्यक्षः~। प्राज्यं भूरि~। आज्यं घृतम्~। घुम्नं धनम्~। यथाविद्यमानमित्यनेन निर्लोभतोक्ता~। नैचिकों वराङ्गीम्, होमधेनुं वा शुक्लां वा~। गिरिकार्णिकाश्वखरी मङ्गल्यौषधिः~। सिद्धार्थकानि सषेपाः~। स्वस्त्रा भगिन्या~। महाश्वेता देवताविशेषः~। रविस्थदेवते}

\newpage
% द्वितीय उच्छ्वासः~। ५७ 

\noindent
भगवत्या महाश्वेतया मालत्याख्यया कृतसकलगमनमङ्गलः, दत्ताशीर्वादः, वान्धववृद्धाभिरभिनन्दितः, परिजनजरन्तीभिर्वन्दितचरणैरभ्यनुज्ञातः, गुरुभिरभिवादितैराघ्रातः शिरसि, कुलवृद्धैर्वर्धितगमनोत्साहः, शकुनैमौहूर्तिकमतेन कृतनक्षत्रदोहदः, शोभने मुहूर्ते हरितगोमयोपलिप्ताजिरस्थण्डिलस्थापितमसितेतरकुसुममालापरिक्षिप्तकण्ठं पिष्टपञ्चाङ्गुलपाण्डुरं मुखनिहितनवचूतपल्लवं पूर्णकलशमुदीक्षमाणः, प्रणम्य कुलदेवताभ्यः कुसुमफलपाणिभिरप्रतिरथं जपद्धिजिद्विजैरनुगम्यमानः, प्रथमचलितदक्षिणचरणः, प्रीतिकूटान्निरगात्~।

प्रथमेऽहनि घर्मकालकष्टं निरुदकं निष्पत्रपादपविषमं पथिकजननमस्क्रियमाणप्रवेशपादपोत्कीर्णकात्यायनीप्रतियातनं शुष्कमपि पल्लवितमिव तृषितश्वापदकुललम्बितलोलजिह्वालतासहस्रैः पुलकितमिवाच्छ भङगोलाङ्गूललिामा नमधुगोलचलितसरघासंघातै रोमाञ्चितमिव दग्धस्थलीरूढस्थूलाभीरुकन्दलशतैः शनैश्चण्डिकाकाननमतिक्रम्य मल्लकूटनामानं ग्राममगात्~। तत्र च हृदयनिर्विशेषेण भ्रात्रा सुहृदा च जगत्पतिनाम्ना संपादितसपर्यः सुखमवसत्~। अथापरेचुरुत्तीर्य भगवतीं भागीरथीं यष्टिग्रहकनाम्नि वनग्रामके निशामनयत्~। अन्यस्मिन्दिवसे स्कन्धावार मुपमणितारमन्वजिरवति कृतसंनिवेशमाससाद~। अतिष्ठच नातिदूरे राजभवनस्य~।

\vspace{2mm}
\hrule

\noindent
{\s त्यन्ये~। {\qtt दत्तेत्यादि}~। बान्धववृद्धाभिप्रायेण समुचित एवम् {\qtt अभिनन्दित इति}~। प्रतिपदं द्वयमूह्यम्~। जरत्यो वृद्धाः~। आप्रातः शिरति चुम्बितः~। मौहूर्तिका गणकाः~। नक्षत्रदोहदं प्रति नक्षत्रप्राशनम्, नक्षत्रविषयोऽभिलाषो वा~। अजिरमङ्गनम्~। स्थण्डिलं भूः~। परिक्षितो चेष्टितः~। पिटपञ्चाङ्गुलमाजकोक्ताभिः~। पञ्चभिरङ्गुलीभिर्मकल्याय दीयते~। अप्रतिरथं प्रास्थानिकं मन्त्रम्~। निजेयादिना स्वस्य दातृत्वमुक्तम्~।

उत्कीर्णा निखाता~। कात्यायनी दुर्गा प्रतियातना प्रतिमा काननत्वात्पल्लवितमिवेत्युत्प्रेक्षा~। जिह्वैव लता~। दीर्घत्वात्~। गोलाङ्गूलः कृष्णमुखो वानरः~। मधुगोलं माक्षिककरण्डः~। सरघा मधुमक्षिकाः~। अभीरुः शतावरी~। कन्दलानि नवनालानि~। भ्रात्रेति चन्द्रसेनेन~। हृदयेत्याद्यभिप्रायेण सुखमित्युक्तम्~। मणितारं यत्तनभेदम्~। अन्वजिरवति नदीभेदनिकटे~। संनिवेशो गृहादिरचना~।}

\lfoot{ह० ६}

\newpage
\lfoot{}
% ५८ हर्षचरिते 

निर्वर्तितस्त्रानाशनव्यतिकरो विश्रान्तव मेखलकेन सह याममात्रावशेषे दिवसे भुक्तवति भूभुजि प्रख्यातानां क्षितिभुजां बहूशिबिरसंनिवेशान्वीक्षमाणः शनैः शनैः पट्टबन्धार्थमुपस्थापितैश्च डिण्डिमाधिरोहणायाहतैश्वामिनवबद्धैश्च विशेषोपार्जितैश्च कौशलिकागतैश्च नागवीथीपालप्रेषितैश्च प्रथमदर्शनकुतूहलोपनीतैश्च दूतसंप्रेषणप्रेषितैश्च पल्लीपरिवृढढौ कितैश्च स्वेच्छायुद्धक्रीडाकौतुकाकारितैश्च दीयमानैश्वाच्छिद्यमानैश्च मुच्यमानैश्च यामस्थापितैश्च सर्वद्वीपजिगीषया गिरिभिरिव सागरसेतुबन्धनार्थमेकीकृतैर्ध्वजपटपटुपटह्शङ्खचामराङ्गरागरमणीयैः पुष्वामिषेकदिवसैरिव कल्पितैर्वारणेन्द्रैः श्यामायमानम्, अनवरतचलितखुरपुटप्रहतमृदङ्गैर्नर्तयद्भिरिव राजलक्ष्मीमुपहसद्भिरिव सृकपुटप्रसृतफेनाट्टहासेन जवजडजवां हरिणजातिमाकारयद्भिरिव संघट्टहेतोर्हषद्वेषितेनोञ्चैःश्र वसमुत्पतद्भिरिव दिवसकररथतुरगरुषा पक्षायमाणमण्डनचामर मालैर्गगनतलं तुरङ्गैस्तरङ्गायमानम्, अन्यत्र प्रेषितैश्च प्रेष्यमाणैश्च प्रेषितप्रतीपनिवृत्तैश्च बहुयोजनगमनगणनसंख्याक्षरावलीभिरिव वराटिकावलीभिर्घटितमुखमण्डनकैस्तारकितैरिव संध्यातपच्छेदैररुणचामरिकारचितकर्णपूरैः सरकोत्पलैरिव रक्तशालिशालेयैरनवरतझणझणायमानचारुचामीकरघुरुघुरुकमालिकैर्जरत्करञ्जवनैरिव रणितशुष्कबीजकोशीशतैः श्रवणोपान्तप्रेङ्खत्पञ्चरागवर्णार्णाचित्र सूत्रजूटजटाजालैः कपिकपोलकपिलैः क्रमेलककुलैः कपिलायमानम्

\vspace{2mm}
\hrule

\noindent
{\s निर्वर्तितेत्यादौ राजद्वारमीदृशमगमदिति संबन्धः~। निर्वर्तितेत्यादि राजदर्शने कातरत्वमात्मनः प्रतिपादयति~। वारणेन्द्रैः श्यामायमानमिति राजद्वारविशेषणम्~। डिण्डिमःपटहः~। विक्षेपः करः~। नागवीथीह स्तिभूः~। पल्ली शबरवसतिः~। परिः स्वामी~। आकारितैराह्वानैः~। आच्छिद्यमानैरपिहियमाणैः~। यत्र दिने पुष्यनक्षत्रे राजा स्नाति तद्दिनं पुष्याभिषेकाख्यम्~। श्यामायमानं कालत्वमापद्यमानम्~। अथ च दिवसः श्यामायति रात्रिवदाचरतीति वक्रोक्तिः~। अभिषेकदिनानि च ध्वजादिरम्याणि~। अनवरतेयादौ तुरङ्गैस्वरायमानमिति संबन्धः~। मृदोऽङ्गं मृदङ्गव सुरजः~। सृक्किण्योष्ठपर्यन्तौ~। अन्यत्रेयादौ क्रमेलककुलैः कपिलायमान मिल्यन्वयः~। वराटिकाः श्वेतिकाः~। शालीनां भवनं क्षेत्रं शालेयम्~। {\qt शालिव्रीह्योर्डक्}~। बीजकोशी शिम्बिका~। क्रमेलका उष्ट्राः अन्यत्रेत्यादिनातपत्रखण्डैः श्वेतायमानमित्यन्वयः~। सय इत्यायभिप्रायेण शर\textendash}

\newpage
% द्वितीय उच्छ्वासः~। ५९ 

\noindent
अन्यन्त्र शरज्जलधरैरिव सद्यःस्रुतपयः पटलघवलतनुभिः कल्पपादपैरिव मुक्ताफलजालकजायमानालोकलुप्तच्छायामण्डलैर्नारायणनाभिपुण्डरीकैरिवाश्लिष्टगरुडपक्षैः क्षीरोदोद्देशैरिव द्योतमानविकटविद्रुमदण्डैः शेषफणाफलकैरिवोपरिस्फुरत्स्फीतमाणिक्यखण्डैः श्वेतगङ्गापुलिनैरिव राजहसोपसेवितैरभिभवद्भिरिव निदाघसमयमुपहसद्भिरिव विवस्वतः प्रतापमापिबद्भिरिवातपं चन्द्रलोकमयमिव जीवलोकं जनयद्भिः कुमुदमयमिव कालं कुर्वद्भिज्योतनामय मित्र वासरं विरचयद्भिः फेनमयीनिव दिवं दर्शयद्भिरकालकौमुदीसहस्राणीव सृजद्विरुपहसद्भिरिव शातकतवीं श्रियं श्वेतायमानैरातपत्रखण्डैः श्वेतायमानम्, क्षणहष्टनष्टाष्टदिङ्मुखं च मुष्णद्भिरिव भुवनमाक्षेपोत्क्षेपदोलायितं दिनं गतागतानीव कारयविरुत्सारयद्भि रिव कुनृपतिकलङ्ककालीं कालेयीं स्थिति विकचविशदकाशवनपाण्डुरदिशं शरत्समयमिवोपपादयद्भिर्बिसतन्तुमयमिवान्तरिक्षमाविर्भावयद्भिः शशिकरचीनां चलतां चामराणां सहस्रैर्दोलायमानम्, अपि च हंसयूथायमानं करिकर्णशः, कल्पलतावनाय\textendash

\vspace{2mm}
\hrule

\noindent
{\s द्ग्रहणम्~। स्रुतं निर्गतम्~। पयः क्षीरम्, जलं च~। पटलवत्तेन च चवला तनुराकारो येषाम्~। अन्यत्र धवलाश्च ते तनवः, क्षीणाच ते पुण्डरीकग्रहणे~। नाकारसदृशत्वमप्युच्यते~। गरुडपक्षा रत्नभेदाः, गरुडस्य चामरुहाः~। {\qtt क्षीरोदेति}~। शुक्कृतया~। राजहंसाः प्रख्यनृपाः, रक्तचञ्चुचरणा राजहंसाः~। निदाघस्य तिरस्करणादभिभवद्भिरिवेत्युक्तम् \textendash\ {\qtt उपहसद्भिरिवेति}~। प्रतापस्योपहास एव समुचितो वैयर्थ्यात्~। अथ च प्रतापपदेन भक्क्या विवस्वत आरोपितविजिगीषुव्यवहारत्वाच्छत्रुमनःसंतापकार यश उक्तम्~। आतपं प्रकाशम्~। {\qtt आपिबद्भिरिति}~। तस्य सर्वत एवातिदर्शनात्~। {\qtt जीवलोकमिति}~। यश्च जीवानां लोकस्तत्र कथं चन्द्रलोक इति विरोधः~। {\qtt कुमुदमयमिवेति}~। कुमुदमयत्वाच्छुक्कं भवति~। न तु कालम्~। कुमुदमयं च समयं कार्तिकादि~। {\qtt ज्योत्स्नेति}~। वासरे ज्योत्त्ना न संभवतीति विरोधः~। एवं च दिवः फेनमयीत्वम्~। जलदे हि फेनानामभावः~। कौमुदी कुमुदिनी, कार्तिकी च ज्योना~। पूर्व सामान्येनोक्ता इति~। विशेषेण श्वेता इवाचरन्तः श्वेतायमानाः~। तैस्तत्र तेषां स्वत एक श्वेतत्वाच्छ्रुतपदेन कथमुपमानतेत्युच्यते~। श्वेतगुणा इवाचरन्तः श्वेतायमानाः~। तेन यथा श्वेतगुणयोगादन्यत्किंचिच्छेतते तद्वदेतद्योगात्~। राजद्वारमिति श्वेताः स्फटिका इत्यन्ये~। केचित्तु {\qt श्वेतमानैः} इति पठन्ति~। क्षणेत्यादौ चामराणां सहस्रैर्दौलायमानमित्यन्वयः~। कलेरियं कालेयी~। सर्वत्राभिकलिम्यां ढकू~। पद्मरागा इव बाला}

\newpage
% ६० हर्षचरिते 

\noindent
मानं कलिकाभिः, माणिक्यवृक्षकवनायमानं मायूरातपत्रैः, मन्दाकिनीप्रवाहायमानमंशुकैः, क्षीरोदायमानं क्षौमैः, कदलीवनायमानं मरकतमयूखैः, जन्यमानान्यदिवसमिव पद्मरागवालातपैः, उत्पद्यमानापराम्बरमिन्द्रनीलप्रभाटलैः आरभ्यमाणापूर्वनिशमिव महानीलमयूखान्धकारैः, स्यन्दमानानेककालिन्दीसहस्रमित्र गरुडमणिप्रभाप्रतानैः, अङ्गारकित्तमिव पुष्परागरश्मिभिः, कैश्चित्प्रवेशमलभमानैरधोमुखैञ्चरणनखपतितवदनप्रतिबिम्बनिभेन लज्जया स्वाभानीव विशद्भिः कैश्चिदङ्गुलीलिखितायाः क्षितेर्विकीर्यमाणकरनखकिरणकदम्बकव्याजेन सेवाचामराणीवार्पयद्भिः कैश्चिदुरः \textendash\ स्थलदोलायमानेन्द्रनीलतरलप्रभापदैः स्वामिप्रकोपप्रशमनाय कण्ठबद्धकृपाणपट्टैरिव कैश्चिदुच्छाससौरमभ्राम्यद्धमरपटलान्धकारितमुखैरपहृतलक्ष्मीशोकघृतलम्वश्मश्रुमिरिवान्यैः शेखरोड्डीयमानमधुपमण्डलैः प्रणामविडम्बनाभयपलायमानमौलिभिरिव निर्जितैरपि संमानितैरिवानन्यशरणैरन्तरान्तरा निष्पततां प्रविशतां चान्तरप्रतीहाराणामनुमार्गप्रधावितानेकार्थिजनसहस्राणामनुयायिनः पुरुषानश्रान्तैः पुनः पुनः पृच्छद्भिः भद्र, अद्य भविष्यति भुक्त्वा स्थाने दास्यति दर्शनं परमेश्वरः, निष्पतिष्यति वा बाह्यां कक्ष्याम् इति दर्शनाशया दिवसं नयद्भिर्मुजनिर्जितैः शत्रुमहासामन्तैः स मन्तादासेव्यमानम्, अन्यैश्च प्रतापानुरागागतैनीनादेशजैर्महीपालैः प्रतिपालयद्भिर्नरपतिदर्शनका लमध्यास्यमानम्, एकान्तोपविष्टैश्च प्रतिपालयद्भिर्नरपतिदर्शनकालमध्यास्यमानम्, जैनैरार्हतैः पाशुपतैः पाराशरिभिर्वर्णिभिश्च सर्वदेशजन्मभिश्च जनपदैः सर्वाम्भोधिवेलाबनवलयवासिभिश्च म्लेच्छजातिभिः सर्वदेशान्तरागतैश्च दूतमण्डलैरुपास्यमानम्, सर्वप्रजानिर्माणभूमिमिव

\vspace{2mm}
\hrule

\noindent
{\s तपास्तैः~। महानीला गरुडमणयः~। पुष्परागाश्च मणिभेदाः~। कैविदित्यादौ शत्रुमहासामन्तैः समन्तादासेव्यमान नित्यन्वयः~। {\qtt सेवेत्यादि}~। त्वयेदानीं चामरग्रहणेन सेवनीय इति तेषां हि क्षितिः कलत्रमतस्तद्वारेण सेवनेच्छा~। {\qt हारस्य यो मध्यमणिस्तरलः स प्रकीर्तितः}~। चपलो वा शेखरं मुण्डमालिकम्~। मौलयः केशाः~। निर्जितैः पुरस्कृतन्यक्कृतैः, राजसेवाप्राप्तैः, संमानितैः पूजितैरिव~। {\qtt अनुयायिन इति}~। तेषां स्वयं सुलभत्वात् जैनैः शाक्यैः~। आर्हतैर्ननक्षपणकैः~। पाशुपतैः शैवभेदैः~। पराशरेण प्रोक्तमधीयन्ते पाराशरिणो यतयस्तैः~। वर्णिभिर्ब्रह्मचारिभिः~। {\qtt सर्वप्रजेति}~। अत्र हि स्थिवा यदि प्रजापतयो न सृजेयुः,}

\newpage
% द्वितीय उच्छ्वासः~। ६१ 

\noindent
प्रजापतीनां लोकलयसारोच्चयरचितं चतुर्थमिव लोकम्, महाभारतशतैरप्यकथनीयसमृद्धिसंभारम्, कृतयुगसहस्रैरिव कल्पितसंनिवेशम्, स्वर्गार्वुदैरिव विहितरामणीयकम्, राजलक्ष्मीकोटिभिरित्र कृतपरिग्रहं राजद्वारमगमत्~।

अभवच्चास्त्र जातविस्मयस्य मनसि \textendash\ कथमिवेदमियत्प्रमाणं प्राणिजातं जनयतां प्रजासृजां नासीन्महाभूतानां वा परिक्षयः, परमाणूनां वा परिच्छेदः, कालस्य वान्तः, आयुषो वा व्युपरमः, आकृतीनां वा परिसमाप्तिः इति~। मेखलकस्तु दूरादेव द्वारपाललोकेन प्रत्यभिज्ञायमानः {\qt तिष्ठतु तावत्क्षणमात्रमन्त्रैव पुण्यभागी} इति तमभिधायाप्रतिहृतः पुरः प्राविशत्~।

अथ स मुहूर्तादिव प्रांशुना, कर्णिकारगौरेण, वीध्रकञ्चुकच्छन्नवपुषा, समुन्मिषन्माणिक्यपदकबन्धबन्धुरशस्तबन्धकुशावलग्नेन, हिमशैलशिलाविशालवक्षसा, हरवृषककुदकूटविकटांसतटेन, उरसा चपलहृषीकहरिणकुलसंयमनपाशमिव हारं बिभ्रता, {\qt कथयतं यदि सोमवंशसंभवः सूर्यवंशसंभवो वा भूपतिरभूदेवंविधः} इति प्रष्टुमानीताभ्यां सोमसूर्याभ्यामिव श्रवणगताभ्यां मणिकुण्डलाभ्यां समुद्भासमानेन, बृहद्वदनलावण्यविसरवेणिकाक्षिप्यमाणैरविकारगौरवादीयमानमार्गेणेव दिनकृतः किरणैः प्रसादलब्धया विकचपुण्डरीकमुण्डमालिकयेव दीर्घया दृष्या दूरादेवानन्दयता, नैष्ठुर्याधिष्ठानेऽपि प्रतिष्ठितेन पदे प्रश्रयमिवावनम्रेण,

\vspace{2mm}
\hrule

\noindent
{\s तत्कथं सर्वे भावाः कारणभूता इव तत्र लक्षेरन्~। अर्बुदं दश कोटयः~। कोटिर्लक्षशतम्~। इव तु बहुसंख्योपलक्षणार्थावर्बुदकोटिशब्दौ~।

परिसमाप्तिरनारम्भः~। {\qtt तिष्टविति}~। विद्यायुक्ते कदाचिदनादरशङ्केत्येतदर्थमाह \textendash\ {\qtt पुण्यभागीति}~।

अथेत्यादावीदृशपुरुषेणानुगम्यमानो निर्गत्यावोचदिति संबन्धः~। अन्तराले वस्त्वन्तरादिवर्णनाभावादथेत्यादिना समनन्तरमेव निर्गमनेन पुनरादर एव प्रतीयते~। अत आह \textendash\ {\qtt मुहूर्तादिवेति}~। पुरुषानुगतत्वेन चादर एव पोष्यते~। वीघ्रं निर्मलम्~। बन्धुरं शोभनम्~। शस्तं सुवर्णपट्टिकाकटिसूत्रम्~। तस्य बन्धेन निवेशनेन कृशमवलनं मध्यं यस्य तेन~। हिमशैले हिमग्रहणं राज्ञो ववलवात्~। हरग्रहणं जराशौक्ल्यप्रतिपादनाय पूर्ववत्~। हृषीकाणीन्द्रियाणि~। {\qtt आनीताभ्यामिति}~। आनयने तस्य प्रभविष्णुता ध्वन्यते~। यश्च स्रष्टुमानीयते स सवर्ण}

\newpage
% ६२ हर्षचरिते 

\noindent
मौलिना पाण्डुरमुष्णीपमुबहता, वामेन स्थूलमुक्ताफलच्छुरणदन्तुरत्सरं करकिसलयेन कलयता कृपाणम्, इतरेणापनीततरलतां ताडितीमिव लतां शातकौम्भीं वेत्रयष्टिमुन्मृष्टां धारयता पुरुषेणानुगम्यमानो निर्गत्यावोचत् \textendash\ {\haq एष खलु महाप्रतीहाराणामनन्तरचक्षुष्यो देवस्य पारियात्रनामा दौवारिकः~। समनुगृहात्वेनमनुरूपया प्रतिपत्त्या कल्याणाभिनिवेशी} इति~। दौवारिकः समुपसृत्य कृतप्रणामो मधुरया गिरा सविनयमभाषत \textendash\ {\haq आगच्छत~। प्रविशत दर्शनाय~। कृतप्रसादो देवः} इति~। बाणस्तु {\haq धन्योऽस्मि, यदेवमनुग्राह्यं मां देवो मन्यते} इत्युक्त्वा तेनोपदिश्यमानमार्गः प्राविशदभ्यन्तरम्~।

अथ वनायुजैः, आरजैः, काम्बोजैः, भारद्वाजैः, सिन्धुदेशजैः, पारसीकैश्च, शोणैश्च, श्यामैच, श्वेतैश्च, पिश्चरैश्च, हरिद्भिश्च, तित्तिरिकल्माषैश्च, पञ्चमद्रैश्च, मल्लिकाक्षैच, कृत्तिकापिञ्जरैश्च, आयतनिर्मीसमुखै:, अनुत्कटकर्णकोरौः, सुवृत्तलक्ष्णसुघटितघण्टिकाबन्धैः, यूपानुपूर्वीवक्रायतोद्ग्रग्रीवैः, उपचयश्वयत्स्कन्ध\textendash

\vspace{2mm}
\hrule

\noindent
{\s गच्छति~। वेणिका प्रवाहाः~। {\qtt वामेनेति}~। तदा तस्य व्यापारानुपपत्तेः~। अपनीतेत्यादिनाय नियमविधायिलं पोष्यते~। उन्मृष्टामुत्तसिताम्~। अनेन भास्वरतैव पोष्यते~। अनन्तरः प्रधानम्~। चक्षुष्यः प्रियः~। आगच्छतेयादाँ बहुत्वनिर्देशनादर एवास्यापाद्यते~।

अर्थत्यादावेवंविधैरौरारचितां मन्दुरां विलोकयन्दूरादिभधिष्ण्यागारमपश्यदिति संबन्धः~। वनायुजादीनि देश विशेषेणाश्वानां नामानि~। शोणैरित्यादि चूर्णविशेषवर्णनम्~। {\qt शोण: पद्मारुणः स्मृतः}~। पिञ्जरैरीषत्कपिलैः~। हरिच्छुक निभो वर्णः~। तित्तिरिः पक्षिभेदस्तद्वचित्रैः~। {\qt सिताश्च यस्य वाजिनः शफाः समस्तकं मुखं स पञ्चभद्रनामको नृपस्य राज्यसौख्यदः}~। शुक्लपर्यन्ते असिततारके नयने येषां ते मल्लिकाक्षाः~। उक्तं च \textendash\ {\qt पृथुस्निग्धा समा चैव मलिकाकुसुमप्रभा~। राजी यस्य तु पर्यन्ते परिक्षिप्ये तु लोचने~॥ सह यो महिकाक्षस्तु दृष्टिपर्यन्ततारकः~॥} इति तारका कदम्बककल्पानेकबिन्दुकल्माषितत्वचः~। कृत्तिकापिञ्जराः~। यतः~। {\qtt आयतेत्यादि}~। तदुक्तम् \textendash\ {\qt मुखं तन्वायतनतं चतुरङ्गं समाहितम्~। नु चैवोपदिष्टं च परिपूर्ण च शस्यते~॥} इति~। कृष्णेनाप्युक्तम् \textendash\ {\qt उज्जा अतुङ्गपत्थं णिम्मं संवाहिराण अच्चअणम्} इति~। अनुत्कटो ह्रस्वः~। कोशो मध्यम्~। शिरसो ग्रीवायाथ यन्मध्यं च घण्टिकाबन्धः~। यो निगाल इत्युच्यते~। तस्य सुवृत्तादि शस्यते~। यदाह \textendash\ ग्रीवाशिरोऽन्तरश्लिष्टो दीर्घवृत्तः समाहितः}

\newpage
% द्वितीय उच्छ्वासः~। ६३ 

\begin{sloppypar}
\noindent
संधिभिः, निर्भुग्नोरःस्थलै:, अस्थूलप्रगुणप्रसृतैर्लोहपीठकठिनखुरमण्डलैः, अतिजवत्रुटनभयादनिर्मितान्त्राणीवोद्राणि वृत्तानि धारयद्भिः, उद्यद्रोणीविभज्यमानपृथुजघनैः, जगतीदोलायमानवालपलवैः कथमप्युभयतो निखातदृढभूरिपाशसंयमननियन्त्रितैः, आयतैरपि पश्चात्पाशबन्धप्रसारितैकाङ्क्षिभिरायततरैरित्रोपलक्ष्यमाणैः, बहुगुणसूत्रग्रथितग्रीवागण्डकैरामीललोचनैः, दूर्वारसश्यामलफेनलवशबलान्दशनगृहीतमुक्तान्फरफरितत्वचः कण्डूजुषः प्रतीकान्प्रचालयद्भिः, सालसवलितवालधिभिः, एकशफविश्रान्तिस्रस्तशिथिलितजघनार्धैः, निद्रया प्रध्यायद्भिश्च, स्खलितहुंकारमन्दमन्दशब्दायमानैश्च, ताडितखुरधरणीरणितमुखरशिखरखुरलिखितक्ष्मातलैर्घासममिलषद्भिश्च, प्रकीर्येमाणयवसमासरसमत्सरोद्भूतक्षोभैश्च, प्रकुपितचण्डचण्डालहुंकारकातरतरतरलतारकैश्च, कुङ्कुमप्रमृष्टिपिञ्जराङ्गतया सततसंनिहितनीराजनानलरक्ष्यमाणैरिवोपरि \textendash
\end{sloppypar}

\vspace{2mm}
\hrule

\noindent
{\s नोद्वर्तौ नार्धितो नातिदुर्नाहोऽतिविधानतः~॥ सुदिग्धोऽनुपदिग्धश्च निगालो गदितः शुभः~॥ इति~। यूपो यज्ञचिह्नम्~। तस्यैवानुपूर्वी यस्याः~। तथा वक्रा आयता उदया उद्धुरा श्रीवा येषाम्~। तदुक्तम् \textendash\ {\qt ग्रीवा भूलम्बिनी वृत्ता दीर्घा च सुसमाहिता~। गले बद्धा विदौईत्ता तथा शिरसि चोद्यता~॥ निगाले स्याच निर्मासा मृद्धी साकुञ्चिता भृशम्~। विष्ठमांसाश्रद्धा च तुरगस्य प्रशस्यते~॥} इति~। {\qtt उपचयेत्यादि}~। तदुक्तम् \textendash\ स्कन्धः सुपरिपूर्णः स्याव्यक्तमांसः पृष्टुत्रिकः~। बहुमांसासंश्लिष्टः स्थिरमांसच पूरितः~॥ इति~। निर्मुनं स्कूलवाहिर्निःसृ तम्~। उक्तं च \textendash\ {\qt स्थूलास्थिमहदच्छिदं पृथुलं यच्च निर्बलि~। उर ईदृक्शंसन्ति स्थूलकोडं महत्तरम्~॥} इति~। निर्भुनमुत्पनद्रोणिकमिति केचित्~। अस्थूलप्रगुणप्रस्थितैर्निर्मासऋजुजद्वैः~। उक्तं च \textendash\ {\qt जङ्घे वृत्ते दीर्घे निर्मोसे पूजिते निगूढसिरे} इति~। मण्डलशब्देन वृत्तत्वमुच्यते~। तदुक्तम् \textendash\ {\qt खुरास्तुरङ्गे वृत्ताश्च हखाश्च सुदृढा घनाः} इति~। तथा शिलातलनिभैः सुरैरिति~। {\qtt उदाणीति}~। तदुक्तम् \textendash\ {\qt उदरं वृत्तमगुरुमृगस्योपचितं तथा~। अच्छिद्रहखवृत्ताल्पसमकुक्षि च पूजितम्~॥} इति~। द्रोणी शोभाविशेषः~। यदाह \textendash\ {\qt पृष्ठोरःकटिपार्श्वस्य मांसोत्कर्षणनिर्मिता~। द्रोणिकेति प्रशंसन्ति शोभा वाजिनि पञ्चमी~॥} इति~। बाला एव पलवाः~। {\qtt उभयत इति}~। अत्युद्दामवेगवत्त्वादुभयन्त्र पारबन्धः~। गण्डको भूषणभेदः~। फरफरिताः पुनः पुनरीषत्कम्पिताः~। वालविः पुच्छः~। शफः समुद्रयुक्तः पादः~। खुरधरणी खराधःकाष्टपाच्छादिता भूः~। चण्डालोऽश्वपालः~। प्रसृष्टिः प्रमार्जनम्~। वितानक्रं रफकम्~। देवतात्र गोविन्दः~। आरचितां भूषिताम्~। हस्तवामशब्दो भाष्य\textendash}

\newpage
% ६४ हर्षचरिते 

\noindent
विततवितानैः, पुरः पूजिताभिमतदैवतैः, भूपालवल्लभैस्तुरङ्गैरारचितां मन्दुरां विलोकयन, कुतूहलाक्षिप्तहृदयः किंचिदन्तरमतिक्रान्तो हस्तवामेनात्युचतया निरवकाशमिवाकाशं कुर्वाणम्, महता कदलीवनेन परिवृतपर्यन्तं सर्वतो मधुकरमयीभिर्मदस्रुतिभिनदीभिरिवापतन्तीभिरापूर्यमाणम्, आशामुखविसर्पिणा बकुलवनानामिव विकसतामामोदेन लिम्पन्तं व्राणेन्द्रियं दूराव्यक्तमिभधृष्ण्यागारमपश्यत्~। अप्रूच्छच \textendash\ {\qtt अत्र देवः किं करोति} इति~। असावकथयत् \textendash\ एष खलु देवस्यौपवाह्यो वाह्यं हृदयं जात्यन्तरित आत्मा बहिश्वराः प्राणा विक्रमक्रीडासुहृदर्पशात इति यथार्थनामा वारणपतिः~। तस्यावस्थानमण्डपोऽयं महान्दृश्यते इति~। स तमवादीत् \textendash\ {\haq भद्र, श्रूयते दर्पशातः~। यद्येवमदोषो वा पश्यामि तावद्वारणेन्द्रमेव~। अतोऽर्हसि मामत्र प्रापयितुम्~। अतिपरवानस्मि कुतूहलेन} इति~। सोऽभाषत \textendash\ {\haq भवत्वेवम्~। आगच्छतु भवान्~। को दोषः~। पश्यतु तावद्वारणेन्द्रम्} इति~।

गत्वा च तं प्रदेशं दूरादेव गम्भीरगलगर्जितोर्जितैर्वियति चातककदम्बकर्भुवि च भवननीलकण्ठकुलैः कलकेकाकलकलमुखर मुखैः क्रियमाणकलकोलाहलम्, विकचकदम्बसंवादिमदसुरासौर भभरितभुवनम्, कायवन्तमिवाकालमेघकालम्, अविरलमधुबि\textendash

\vspace{2mm}
\hrule

\noindent
{\s कृता वामहस्तमार्ग इत्यर्थे~। बकुलेत्यादिना प्राशस्त्यमेव पोषयति~। तदुक्तम् \textendash\ {\qt मालतीमुक्तपुंनागबकुलोपमसौरभम्~। दानं पिष्टाम्बुसदृशं मुबच्छूतं तु शीतलम्~॥} इति~। श्लैष्मिका दानलक्षणम्~। एवं च धर्मलक्षणे तु प्रकोपसमयेऽपि तथाविधमदवर्णनया श्लेषप्रकृतिलं प्रकाशयति \textendash\ {\qt श्लेषप्रकृतिकं श्रेष्ठं भद्रजातिं तथैव च} इति च शास्त्रकृता दर्शितम्~। धिष्ण्यं मण्डपम्~। औपवाह्यः क्रीडाहस्ती~। यस्मात्केचन संनायाः केचिद्भद्रजातीया उभगस्वभावा भवन्ति करिणः~। अस्य च यद्यपि विक्रमक्रीडासुहृदित्यनेन दर्पशात इति यथार्थनामा वारणपति रित्यनेन च सांनह्यत्वमेवोक्तम् तथा ह्यौपवाह्या इति कथनेऽर्धद्वयेऽपि योग्यवाद्भद्रजातीयं चास्य निश्चीयते~। जात्यन्तरितो द्वितीयां जातिं हस्तिरूपां प्राप्तः~। {\qtt यद्येवमिति}~। यदि सत्यं दर्पशातोऽयमदोषो वेति~। वाशब्दश्छार्थे~। यदि च दोष इत्यर्थः~। यतो रसदानादिभयेन केनचिद्रष्टुं न लभ्यते~। कुतूहलेन परवान्कुतूहलायितः~।

गत्वेत्यादौ दूरादेवं दर्पशातमपश्यदिति संबन्धः~। गर्जितं बृंहितम्~। न्यातकाः स्तोककाः~। नीलकण्ठा मयूराः~। केका मयूररुतानि~। {\qtt मेघकाल\textendash}}

\newpage
% द्वितीय उच्छ्वासः~। ६५ 

\noindent
न्दुपिङ्गलपद्मजालकितां सरसीमिवाभ्यवगाढां दशां चतुर्थीमुत्सृजन्तम्, अनवरतमवतंसशङ्करामन्द्रकर्णतालदुन्दुभिध्वनिभिः पञ्चमीप्रवेशमङ्गलारम्भमिव गायन्तम्, अविरतचलनचिवविपदील लितलास्यलयैर्दोलायमानदीर्घदेहाभोगतया मेदिनीविदलनभयेन भारमिव लघयन्तम्, दिग्भित्तितटेपु कायमित्र कण्डूयमानम्, आहवायोदस्तहस्ततया दिग्वारणानिवाह्वयमानम् ब्रह्मस्तम्भनिव स्थूलनिशितदन्तेन करपलेण पाटयन्तम्, अमान्तं भुवनाभ्यन्तरे बहिरिव निर्गन्तुमीहमानम्, सर्वतः सरसकिशलयलतालासिभिर्लेशिकैश्चिरपरिचयोपचितैर्घनैरिव विक्षिप्त सशैवलबिसविसरशवलसलिलैः सरोभिरिव चाधोरणैराधीयमाननिदाघसमयसमुचितो पचारानन्दम् अपि च प्रतिगजदानपवनादानदूरोत्क्षिप्तेनानेकसमरविजयगणनालेखाभिरिव वलिवलयराजिभिस्तनीयसीभिस्तरङ्गितोदरेणातिस्थवीयसा हस्तार्गलदण्डेनार्गलयन्तमिव सकलं सकुलशैलसमुद्रद्वीपकाननं ककुभां चक्रवालम् एकं करान्तरार्पितेनोत्पलाशेन कदलीदण्डेनान्तर्गतशीकरसिच्यमानमूलम्, मुक्तपल्लवमिव परं लीलावलम्बिना मृणालजालकेन समररसोच्चरोमाञ्चकण्टकितमिव दन्तकाण्डं वहन्तम्, विसर्पन्त्या च दन्तकाण्डयुगलकस्य कान्त्या सर:क्रीडास्वादितानीव कुमुदवनानि बहुधा वमन्तम्,

\vspace{2mm}
\hrule

\noindent
{\s {\qtt मिति}~। मेघकालय चातककदम्ब नीलकण्ठकुलकदम्ब कसौरभादियुक्तः~। अविरला घना ये मधुबिन्दव इव मधुबिन्दवो नाक्षिककणास्तद्वत्पिङ्गलानि पद्मजालकानि संजातानि यस्याम्~। पद्मकं बिन्दुजालं स्याद्गात्रकं करिणामिति~। यथा \textendash\ {\qt पद्मस्वस्तिकसंस्थानो बिन्दुभिश्च कन्वैस्तथा~। खचिताङ्गस्तुषाराभः शावः शक्तिकरः करी~॥} इत्युक्तम्~। अन्ये मधुबिन्दवो मकरन्दकणास्तैः पिङ्गलानीति व्याख्येयम्~। महत्सरः सरसी~। {\qtt अभ्यवगाढामिति}~। परिणताम्~। दशांकालावस्थाम्~। {\qtt चतुर्थीमिति}~। {\qt चतुर्थ्यांमवगाढायां लेखाविन्दुभिराचितः} इत्युक्तम्~। शङ्खैः~। शङ्खशब्दैरित्यर्थः~। {\qtt कर्णेत्यादि}~। कर्णौ च दुन्दुभिध्वनितौ~। {\qt कर्णौ च करिणः कार्य कारिणौ सत्प्रशंसिनौ} इति~। पञ्चमी दशा त्रिपदी~। एकपदोत्क्षेपे पादत्रयावस्थितिः~। लयो लीलाः~। आहवः सङ्ग्रामः~। ब्रह्मस्तम्भो ब्रह्माण्डम्~। करपत्रं ककचम्~। स्थूलनिशितदन्तं भवति~। तत्र भेदयति स्तम्भम्~। अमन्तिमवर्तमानम्~। लेशिकैर्धा सिकैः आधोरणैर्गुजारोहै:~। वलयाकारा बलिवेलिवल्यम्~। अर्गलयन्तं सनाटकं कुवार्णम्~। कुमुदवनानीत्युत्प्रेक्षा~। दन्तयोर्वर्णप्राशस्त्यमाह \textendash\ पयःकुमुदकु\textendash}

\newpage
% ६६ हर्षचरिते

\begin{sloppypar}
\noindent
निजयशोराशिमिव दिशामर्पयन्तम, कुकरिकीटपाटनदुर्ललितान्सिंहानिवोपहसन्तम्, कल्पद्रुमदुकुलमुखपटमित्र चात्मनः कलयन्तम्, हस्तकाण्डदण्डोद्धरणलीलासु च लक्ष्यमाणेन रक्तांशुकसुकुमारतलेन तालुना कवलितानि रक्तपद्मवनानीव वर्षन्तम्, अभिनवकिसलयराशिमिवोद्भिरन्तम्, कमलकवलपीतं मधुरसमिव स्वभावपिङ्गलेन वमन्तम्, चक्षुषा चूतचम्पकलवलीलवङ्गकक्कोलवन्त्येलालतामिश्रितानि ससहकाराणि कर्पूरपूरपूरितानि पारिजातकवनानीवोपभुक्तानि पुरः करटाभ्यां बहलमदामोदव्याजेन विसृजन्तम्, अहर्निशं विभ्रमकृतहस्तस्थितिभिरर्धखण्डितपुण्ड्रेक्षुकाण्डकण्डूयनलिखितैरलिकुलवाचालितैदीनपट्टकै विलममानमिव सवकाननानि करिपतीनाम विरलोद विन्दुस्यन्दिना हिमशिलाशकलमयेन विभ्रमनक्षत्रमालागुणेन शिशिरीक्रियमाणम्, सकलवारणेन्द्राधिपत्यपट्टबन्धबन्धुरमिवोचैस्तरां शिरो दधानम्, मुहुर्मुहुः स्थगितापावृतदिङ्मुखाभ्यां कर्णतालतालवृन्ताभ्यां वीजयन्तमिव भतृभक्त्या दन्तपर्यङ्किकास्थितां राजलक्ष्मीमायतवंशक्रमागतेन गजा\textendash
\end{sloppypar}

\vspace{2mm}
\hrule

\noindent
{\s न्दाभौ केतकी कुमुदद्युती~। मृगाइकिरणालोकौ कीर्तिकल्याणकारकौ~॥ इत्युक्तम्~। {\qtt रक्तांशुक्रेति}~। उक्तं च \textendash\ {\qt रक्तोष्ठतारसनम्} इति~। {\qtt स्वभावपिङ्गलेनेति}~। उक्तं च \textendash\ {\qt शशिसूर्यसमाभासे कलविङ्काक्षसँनिभे~। प्रसन्नमधुमिते च स्थिरे चामीलने तथा~॥ अपरित्राविणी चैव कुशानिनिभभाखरे~। नेत्रे शस्ते समे स्त्रिग्घे दीर्घे चाविलपक्ष्मणी~॥} इति~। चूतेत्यादिना प्रशस्यसमाह~। यदाह \textendash\ {\qt उभय तिरप्येष विवर्णो हर्षवर्जितः~। यदि स्यादपगन्धश्च तदासौ न सतां मतः~॥} इति~। करटाभ्यां गण्डाभ्याम्~। अर्धेत्यादिनेक्षुकाण्डकस्य लेखनीसादृश्यमाह~। लिखितैः कृतलेखैरप्यलिकुलेषु सत्सु वान्चालितशब्दयोगो येषामियनेनालिकुलस्य लिप्यक्षररूपतां ध्वनति~। लिप्यक्षषु च सत्सु पाठ्यमानेषु वाचालता~। दानपट्टकलिखितैः किंचिद्धि लभ्यते~। अक्षरपाटिकेच तेषां हस्तस्थितिनं क्रियते~। तानि च वाच्यन्ते~। यद्वा वहस्तेनाक्षरकरणं हस्तस्थितिः~। हिमशिला वातवज्रीभूतं हिमम्~। केचित्तु {\qt हिमानि हिमशकलानि चन्द्रकान्ताः} इत्याहुः~। हिमस्य च तदा वर्णनानुचितत्वात्~। पर्वतेभ्यो हिमानयनं सुलभमेवेति पूर्वोक्तमेव श्रेष्ठम्~। यतश्चन्द्रकान्तानां दिवा स्रुतिर्न भवतीति~। नक्षत्रमाला हस्त्याभरणभेदः~। {\qtt उच्चैस्तरामिति}~। उच्चं हि शिरः करिणः शस्यते~। यदुक्तम् \textendash\ {\qt समं महच्च पूर्णं च नातिखब्बोच्चनस्तकम्~। नावाग्नं नातिपृथुलं वितानावग्रहं मृदु~॥} इति~। दन्तावेव तद्वस्थानसमुचितलात्~। पर्यङ्किका च दन्तमयःपर्यङ्कः आस्त इति श्लेषः~। आयतवंशः, वक्रवंशः,}

\newpage
% द्वितीय उच्छ्वासः~। ६७

\noindent
धिपत्यचिह्नेन चामरेणेव चलता वालधिना विराजमानम्, स्वच्छशिशिरशीकरच्छलेन दिग्विजयपीताः सरित इव पुनःपुनर्मुखेन मुञ्चन्तम्, क्षणमवधानदाननिस्पन्दीकृतसकलावयवानामन्यद्विरदडिण्डिमाकर्णनाङ्गवलनानामन्ते दीर्घशुत्कारैः परिभवदुःखमिवावेदयन्तम्, अलव्धयुद्धमिवात्मानमनुशोचन्तम, आरोद्दाधिरूदिपरिभवेन लज्जमानमिवाङ्गुलीलिखितमहीतलं, मदं सुश्चन्तम्, अवज्ञागृहीतमुक्तक्रवलकुपितारोद्दारटनानुरोधेन मदतन्द्रीनिमीलितनेत्रत्रिभागम्, कथं कथमपि मन्दमन्दमनादरादाददानम्, कवलानवजग्धतमालपलवस्रुतश्यामलरसेन प्रभूततया सद्ग्रवाहमिव मुखेनाप्युत्सृजन्तम्, चलन्तमिव दर्पेण, श्वसन्तमिव शौर्येण, मूर्च्छन्तमिव मदेन, त्रुट्यन्तमित्र तारुण्येन, द्रवन्तमिव दानेन, वल्गन्तमिव बलेन, माचन्तमिव मानेन, उद्यन्तमिवोत्साहेन, ताम्यन्तमिव तेजसा, लिम्पन्तमिव लावण्येन, सिञ्चन्तमिव सौभाग्येन, स्निग्धं नखेषु, परुषं रोमविषये, गुरुं मुखे, सच्छिष्यं विनये, मृदुं शिरसि,, दृढं परिचयेषु, ह्रस्वं स्कन्धबन्धे, दीर्घमायुषि, दरिद्रमुद्रे, सततप्रवृत्तं दाने, बलभद्रं मदलीलासु, कुलकलवमायत्ततासु, जिनं क्षमासु, वह्निवर्षे क्रोधमोक्षेषु, गरुडं नागोद्धतिषु, नारदं

\vspace{2mm}
\hrule

\noindent
{\s शरवंशः, बालवंशश्चेति चत्वारो वंशाः~। तेषु बालवंश आयत एव शास्त्रकृतामभिप्रेतः~। तथा च \textendash\ {\qt यावत्पूरितपाशच वंशश्चापलता कृर्तिः~। शुभो ज्ञेयो गजेन्द्राणामायतः कुरुते सुखम्~॥} इति तैरुतम्~। आयताद्वंशात्तत्क्रमेण गोपुच्छवदायत इति विग्रहः~। समानार्हो हि वालधिः शोकं करोति~। यदुक्तम् \textendash\ {\qt वक्रं स्थूलं च ह्रस्वं च पुच्छं कचविवर्जितम्~। समानार्हे हि नागस्य भर्तुः शोककरं स्मृतम्~॥} इति~। वंश पृष्ठ नाभि, कुलं च~। क्रम आनुपूर्वी, पारम्पर्ये च वालधिः पुच्छम्~। {\qtt लजमानमिति}~। यश्च लज्जते स भूमिं लिखति, दर्प चोज्झति~। अड्डली करिकरामावयवः, करशाखा च तन्द्री आलस्यम्, गाढनिद्रा वा~। चलन्तमित्यादि दर्षाधिकारणसमुचितक्रियाप्रतिपादनसाभिप्रायं व्याख्येयम्~। {\qtt स्निग्धमिति}~। उक्तं च \textendash\ {\qt नखाः निग्वाः सिताः शस्ताः} इति~। परुषं निष्कृपम्~। यश्च स्निग्धः प्रीतिमान्स कथं परुषः प्रीतिशून्यो भवतीति विरोधः~। एवं गुरुर्विस्तीर्णः, आचार्यश्च~। {\qtt विनय इति}~। उक्त च \textendash\ {\qt विनये सुनिभिस्तुल्याः क्रुद्धा नागाश्च राक्षसाः~। निस्त्रिंशस्याधिकत्वाच शस्त्रं नागा महीपतेः~॥} इति~। स्कन्धबन्धे प्रीवामूळे~। दरिद्रः कृशः, दुर्गतच~। दानं मदवारि, वितरणं च~। बलभद्रो हलधरः~। मदो दानम्, सुराकृतश्च~। नागाः करिणः, सर्पाच~। कलहो रणोऽपि~। अविदितशत्रुसैन्ये पातोऽवस्कन्दः~। मकरं}

\newpage
% ६८ हर्षचरिते 

\noindent
कलहकुतूहलेषु, शुष्काशनिपातमबस्कन्देषु, मकरं वाहिनीक्षोमेषु, आशीविषं दशनकर्मसु, वरुणं हस्तपाशाकृष्टिषु, यमवागुरामरातिसंवेष्टनेषु, कालं परिणतिषु, राहुं तीक्ष्णकरग्रहणेषु, लोहिताङ्गं वक्रचारेषु, अलातचक्रं मण्डलभ्रान्तिविज्ञानेषु, मनोरथसंपादक चिन्तामणिपर्वतं विक्रमस्य, दन्तमुक्ताशैलस्तम्भनिवासप्रासाद्मभिमानस्य, घण्टाचामरमण्डनमनोहरमिच्छासंचरणविमानं मनस्वितायाः, मद्धारादुर्दिनान्धकारं गन्धोदकधारागृहं क्रोधस्य, सकावनप्रतिमं महानिकेतनमहंकारस्य, सगण्डशैलप्रस्रवणं क्रीडापर्वतमवलेपस्य, सदन्ततोरणं वञमन्दिरं दर्पस्य, उच्चकुम्भकूटाट्टालकविकटं संचारि गिरिदुर्ग राज्यस्य, कृतानेकवाणविवरसहस्रं लोहप्राकारं पृथिव्याः, शिलीमुखशतझांकारितं पारिजातपादपं भूनन्दनस्य, तथा च संगीतगृहं कर्णतालताण्डवानाम्, आपानमण्डप मधुपमण्डलानाम्, अन्तःपुरं शृङ्गाराभरणानाम्, मदनोत्सवं मदलीलालास्यानाम्, अक्षुण्णप्रदोषं नक्षत्रमालामण्डलानाम्, अकालप्रावृट्काल मदमहानदीपूरलवानाम्, अलीकशरत्समयं सप्तच्छद्\textendash

\vspace{2mm}
\hrule

\noindent
{\s कूर्मम्~। वाहिनी सेना, नदी च~। दशनकर्म दन्तव्यापारः, दशनरूपा च क्रिया~। हस्त एव पाशाः, प्रशस्तहस्तो हस्तपाश इति वा~। हस्ते च पाशः~। चागुरा जालम् परिणतिषु दन्तविदारणकर्मसु~। कालं यमम्~। शुभाशुभादिकर्मविपाकेषु च कालमहरादिरूपम्~। तीक्ष्णं कृत्वा करेण हस्तेन ग्रहणम्~। रविश्च तीक्ष्णकरः~। लोहिताङ्गोऽङ्गारकः~। वक्रं कुटिलम्~। पश्चाच मण्डलाकृत्या भ्रान्तेर्भमणस्य विज्ञानानि कौशला तिशयगतिः~। गोमूत्रिकामण्डले त्रिविधा हि गतिः~। तत्रालातचक्रमुल्मुकचक्रं भ्रमणं करोति~। {\qtt मनोरथसंपादकमिति}~। शेषे षष्ठीसमासः~। {\qt कर्मण्यण} इति वाणिकृते खार्थे कः~। दन्तौ मुक्ताशैलस्य वैतपाषाणस्य स्तम्भाविव यस्य~। अन्यत्र दन्तस्य मुक्ताशैलानां च स्तम्भा यत्र~। प्रतिमा दन्तकोशः, देवताकृ तिश्च~। महानिकेतनं साधुदेवगृहम्~। गण्डावेव शैली तत्र प्रस्रवणं दाननिर्यासः~। सह तेन वर्तते निर्झरश्च {\qt महतो मुक्तपाषाणान्गण्डशैलान्प्रचक्षते}~। संचारी जङ्गमः~। यदाह कौटिल्यः \textendash\ {\qt हस्तिनो हि जङ्गमं दुर्गम्} इति~। कृतान्यनेकानि बाणैर्विवरसहस्राणि यस्य तम्~। प्राकारेषु बाणानुस्रष्टुं विवरसहस्राणि क्रियन्ते, य इन्द्रकोशा इति चाणक्यादिषु प्रसिद्धाः~। भूनन्दनो राजा~। {\qt देवोद्यानं च नन्दनम्}~। कर्णतालानां ताण्डवानीव ताण्डवानि~। अन्यत्र लाभप्रधानानि ताण्ड\textendash}

\newpage
% द्वितीय उच्छ्वासः~। ६९ 

\noindent
वनपरिमलानाम्, अपूर्वदिमागमं शीकरनीहाराणाम्, मिथ्याजलधरं गर्जिताडम्बराणां दर्पशातमपश्यत्~।

आसीचास्य चेतसि \textendash\ {\haq नूनमस्व निर्माणे गिरयो प्राहिताः परमाणुताम्~। कुतोऽन्यथा गौरवमिदम्~। आश्चर्यमेतत्~। विन्ध्यस्य दन्तावादिवराहस्य करः} इति विस्मयमानमेतं दौवारिकोऽब्रवी त् \textendash\ पश्य~।

\vspace{-5mm}
\begin{quote}
{\ha मिथ्यैवालिखितां मनोरथशतैनिःशेषनष्टां श्रियं\\
चिन्तासाधनकल्पनाकुलधियां भूयो वने विद्विषाम्~।\\
आयातः कथमध्ययं स्मृतिपथं शून्यीभवचेतसां\\
नागेन्द्रः सहते न मानसगतानाशागजेन्द्रानपि~॥~४~॥}
\end{quote}

\vspace{-5mm}
तदेहि~। पुनरप्येनं द्रक्ष्यसि~। पश्य तावदेवम् इत्यभिधीयमानश्च तेन मद्जलपङ्किलकपोलपट्टपतितां मत्तामिव मदपरिमलेन मुकुलितां कथमपि तस्मादृष्टिमाकृष्य तेनैव दौवारिकेणोपदिश्यमानवर्त्मा समतिक्रन्य भूपालसहस्रसंकुलानि त्रीणि कक्ष्यान्तराणि चतुर्थे मुक्तास्थानमण्डपस्य पुरस्तादजिरे स्थितम्, दूरादूर्ध्वस्थितेन प्रांशुना कर्णिकारगौरेण व्यायामव्यायतवपुषा शस्त्रिणा मौलेन शरीरपरिचारकलोकेन परिस्थितेन कार्तस्वरस्तम्भमण्डलेनेव परि\textendash

\vspace{2mm}
\hrule

\noindent
{\s वानि~। मधुपा भ्रमराः, विटाथ~। शृङ्गारः सिन्दूरादिदानम्, रसभेदश्र~। अक्षुण्णः परिपूर्णः, अभ्रादिनानावृतः, अपूर्वो वा~।

ग्राहिताः प्रापिताः~। {\qtt मिथ्यैवेति}~। तस्या निःशेषनष्टत्वात्पुनरभावप्रसङ्गानिःशेषेत्यायभिप्रायेणाह \textendash\ {\qtt मनोरथशतैरिति}~। तस्यां व्यापाररहितत्वाच्छून्यमनस्कत्वादेवाह \textendash\ {\qtt सहत इत्यादि}~। मानसं मनः, सरोभेदोऽपि~। आशा दिशः, अभिलाषोऽपि~। {\qt देवमिति}इत्यादौ चक्रवर्तिनं हर्षमद्राक्षीदिति संबन्धः~। मदजलेन पङ्किले कपोलपट्टे पतिताम्~। {\qtt मत्तामिवेति}~। मत्तश्च पतति, मुकुलितदृष्टिश्च भवति, गतिवैकल्यादन्येन कृष्यते~। भोजनं भोक्तव्यम्~। भुक्ते सत्यास्थानं लोकदर्शनं तदर्थं मण्डपस्तस्य~। ऊर्ध्वस्थितेत्यादि साधारणम्~। प्रांशुनोन्नतेन, अन्यत्र प्रकृष्ट अंशवो यस्य तेन~। कर्णिकारमारस्वधपुष्पम्~। व्यायामः श्रमः~। व्यायतं विभक्तावयवम्, विशेषेण दीर्घ च~। शस्त्रिणा सायुधेन, स्तम्भा अपि शस्त्रेण वध्यन्ते मौलमृतकश्रेणिमित्रा मित्राटविकभेदेन षट्प्रकाराः सद्दाया भवन्ति~। अन्यत्र मूले बुध्ने भवं मौलम्~। बुनप्रतिष्ठमित्यर्थः~। परिस्थितेनेति साधारणम्~। कार्तस्वरं}

\lfoot{ह० ७}

\newpage
\lfoot{}
% ७० हर्षचरिते 

\noindent
वृतम्, आसन्नोपविष्टविशिष्टेष्टलोकम्, हरिचन्दनरसप्रक्षालिते तुषारशीकरशीतलतले दन्तपाण्डुरपादे शशिमय इव मुक्ताशैलशिलापट्टशयने समुपविष्टम्, शयनीयपर्यन्तविन्यस्ते समर्पितसकलविमहभारम्, भुजे दिङ्युखविसर्पिणि देहप्रभाविताने विततमणिमयूखे घर्मसमयसुभगे सरसीव मृदुमृणालजालजटिलजले सराजकं रममाणम्, तेजसः परमाणुमिरित्र केबलैनिर्मितम्, अनिच्छन्तं बलादारोपयितुमिव सिंहासनम्, सर्वावयवेषु सर्वलक्षणैगृहीतम्, गृहीतत्रह्मचर्यमालिङ्गितं राजलक्ष्म्या, प्रतिपन्ना सिधाराधारणव्रतम्विसंवादिनं राजर्पिम्, विषमराजमार्गविनितिपद्मखलनमियेव सुलग्नं धर्मे, सकलभूपालपरित्यक्तेन भीतेनेव लब्धवाचा सर्वात्मना सत्येन सेव्यमानम्, आसन्नवार विलासिनीप्रतियात नाभिश्चरणनखपातिनीसिर्दिग्भिरिव दशभिः प्रणम्यमानम्, दीर्घैर्दिगन्तपातिमिर्दृष्टिपातैर्लोकपालानां कृताकृतमिव प्रत्यवेक्षमाणम्, मणिपादपी\textendash

\vspace{2mm}
\hrule

\noindent
{\s सुवर्णम्, यस्योद्धृष्यमाणस्य सतः कुङ्कुमस्येव रागो जायते~। सौगन्ध्यं च तद्धरिचन्दनम्~। शशिमय इति वक्ष्यमाणाभिप्रायेण तुषारेत्यादिना शीतत्वममुष्य दर्शयति~। दन्ते तद्वच पाण्डुरे पादे रदमयोऽपि पादाः~। मुक्तेल्यादिना शुलतयापि शशिमय इवेत्येतदेव पोषयति~। विग्रहः कायः, रणश्च~। {\qtt धर्मेत्यादि}~। मणीनां स्वभावत एव शीतत्वात्तदीया मयूखा अपि ह्रादयन्ति~। यो हि बलवानारोप्यते स सर्वाङ्गेषु गृह्यते~। {\qtt गृहीतब्रह्मचर्यमिति}~। खदारसंतुष्ट ऋतुकालगामी {\qt गृहस्थोचितव्यापारो ब्रह्मचार्येव} इति श्रुतेः~। या त्वेवमनुश्रूयते~। {\qt यावन्मया न सकला जिता भूमिरतावन्मे ब्रह्मचर्यम्} इति श्रीहर्षः प्रतिज्ञातवान्~। द्वादशभिच वषैर्जित्वा तां महिषीमब्रवीत् \textendash\ {\qt प्रतिज्ञा मे निर्व्यूढा} इति~। ततो रोषात् {\qt अहमपि द्वादशवर्षं ब्रह्यचर्यं चरामि} इति सा प्रतिज्ञामकरोत्~। इति ब्रह्मचर्येणाझाकालोऽतिवाहितः~। यश्च गृहीतब्रह्मचर्यः स कथं योषितालिङ्गयत इति विरोधः~। असिवारा खङ्गधारा, व्रतविशेषश्च~। यत्र स्त्रीपुंसावकपटौ ब्रह्मचर्येण तिष्टतः~। यश्च प्रतिपन्नेषु विश्वासितेषु खङ्गधारां पातयति स कस्मान्न विसंवदते~। नान्यथा भवति~। कथं च राजर्षिरसावुच्यत इति विरोधः~। यश्च राजर्षिरुत्तममुनिगृहीतासिधारो ब्रह्मचारी च, स क्याचिदालियेत~। विषमोऽराक्यानुष्ठानो नतोन्नतरूपः~। मार्गों व्यवहारः पन्थाथ~। विषमे पथि च स्खलति येन क्वचित्सुलप्तेन भूयते~। {\qtt लब्धवाचेति}~। सत्यस्य वागेवाश्रयणीयश्च~। सर्वैस्त्यक्तः सन्भीतः संस्त्वां यजामीति वाचं लब्ध्वान्यं सेवते~। चारविलासिनी शरीरोपचारचतुरा मुख्यललनाप्रतिबिम्बम्~। {\qtt दशभिरिति}~। नखानां दिशां च दशसंख्याकत्वात्~।}

\newpage
% द्वितीय उच्छ्वासः~। ७१

\noindent
ठपृष्ठप्रतिष्ठितकरेणोपरिगमनाभ्यनुज्ञां मृग्यमाणमिव दिवसकरेण, भूषणप्रभासमुत्सारणवद्धपर्यन्तमण्डलेन प्रदक्षिणीक्रियमाणमिव दिवसेन, अप्रणमद्भिर्गिरिभिरपि दूयमानं शौर्योष्मणा, फेनायमानमिव चन्दनथवलं लावण्यजलधिमुद्रुहन्तम्, एकराज्यौर्जित्येन निजप्रतिविम्वान्यपि नृपचऋचूडामणिधृतान्यसहमानमिव दर्पदुःखासिकया चामरानिलनिभेन बहुधेव श्वसन्ती राजलक्ष्मी दधानम्,सकलमिव चतुःसमुद्रलावण्यमादायोत्थितया श्रिया समुपश्लिष्टम्, आभरणप्रभाजालजायमानानीन्द्रधनुः सहस्राणीन्द्रप्राभृतप्रहितानि विलभमानमिव राज्ञां संभाषणेषु परित्यक्तमपि मधु वर्षन्तम्, काव्यकथास्वपीतममृतमुद्वमन्तम्, विसम्भभाषितेष्वनाकृष्टमपि हृदयं दर्शयन्तम्, प्रसादेषु निश्चलामपि श्रियं स्थाने स्थाने स्थापयन्तम्, वीरगोष्ठीषु पुलकितेन कपोलस्थलेनानुरागसंदेशमिवोपांशु रणश्रियः शृण्वन्तम्, अतिक्रान्तसुभटकलहालापेषु स्नेहवृष्टिमिव दृष्टिमिष्टे कृपाणे पातयन्तम्, परिहासस्मितेषु गुरुप्रतापभीतस्य राजकस्य स्वच्छमाशयमिव दशनांशुभिः कथयन्तम्, सकललोकहृदयस्थितमपि न्याये तिष्ठन्तम्, अगोचरे गुणानामभूमौ सौभाग्यानामविषये वरप्रदानानामशक्य आशिषाममार्गे मनोरथानामतिदूरे दैवस्यादिश्युपमानानामसाध्ये धर्मस्यादृष्टपूर्वे लक्ष्म्या महत्त्वे स्थितम्, अरुणपादपल्लवेन सुगतमन्थरोरुणा वज्रायुधनि\textendash

\vspace{2mm}
\hrule

\noindent
{\s {\qtt मणिपादेति}~। मणिसंबन्धप्रतिष्ठानमेव पोषयति~। करो हस्तोऽपि~। {\qtt फेनायमानमिति}~। जलं संतापेन सफेनं भवति~। असहमानमिवेति~। कथं सामान्येन समान इति~। {\qtt सकलमित्यादि}~। सकलपदेन, चतुःशब्देन च शौरेरस्य विशेषमाह~। यतो लवणत्वस्य तत्रायापि शिष्यमाणत्वात्~। अलं लावण्यमादाय~। एकस्माच्च समुद्रादुत्थाय लक्ष्म्याः शौरिः समुपश्लिष्टः~। लावण्यं लवणता, सौन्दर्य च~। प्रामृतं ढौकनिकम्~। मधु मद्यम्, अमृतं च~। विस्रम्भ आश्वासः~। उपांश्वप्रकटम्~। अतिक्रान्ते कलहे रणे शस्त्राणां नेहो दीयते~। रुधि रादिसलिलनिवारणाय~। स्वच्छ निर्मलम्~। सुप्रसादमाशयं भावं प्रकृष्टतापभीतस्य च स्वच्छो निर्मल आशयो जलाधारो दृश्यते~। अत्र प्रतापेत्यादिप्रकरणसाहचर्यात्स्वच्छतान्यथानुपपत्त्या च~। जलशब्द चिना जलाशय एवं प्रतीयते~। न्याये तिष्ठन्तम्~। न्यायममुश्चन्तमित्यर्थः~। यः सर्वेषां हृदयस्थितः स एकस्मिन्नेव तिष्ठतीति विरोधः~। अरुणो लोहितः, अनूरुव शोभनं गमनं ययोस्तौ मन्थरावूरू यस्य~। बुद्धश्च}

\newpage
% ७२ हर्षचरिते 

\noindent
ष्ठुरप्रकोष्ठपृष्ठेन वृषस्कन्धेन भास्वम्बिाधरेण प्रसन्नावलोकितेन चन्द्रमुखेन कृष्णकेशेन वपुषा सर्वदेवतावतारसित्रैकव दर्शयन्तम्, अपि च मांसलमयूखमालामलिनितमहीतले महति महाहें माणिक्यमालामण्डितमेखले महानीलमये पादपीठे कलिकालशिरसीव सलीलं विन्यस्तवामचरणम्, आक्रान्तकालियफणाचक्रवालं बालमिव पुण्डरीकाक्षम्, श्रीमपाण्डुरेण चरणनखदीधितिप्रतानेन प्रसरता महीं महादेवी वन्धेनेव महिमानमारोपयन्तम्, अप्रणतलोकपालकोपेनेवातिलोहितौ सकलनृपति मौलिमालास्वतिपीतं पद्मरागरत्ना तपमिव वमन्तौ सर्वतेजस्विमण्डलास्तमयसंध्यामिव धारयन्तावशेषराजकशखरकुसुममधुरसस्रोतांसीव स्रवन्तौ समस्तसामन्तसीमन्तोत्तंसस्रक्सौरभभ्रान्तै भ्रमरमण्डलैर मिलोत्तमाङ्गैरिव मुहूर्तमप्यविरहितौ संवाहनतत्परायाः श्रियो विकचरक्तपङ्कजवनबासभवनानीव कल्पयन्तौ जलजशङ्खमीनमकरसनाथतलतया कथितचतुरम्भोधिभोगचिह्नाविव चरणौ दधानम्, दिङ्नागद्न्तमुसलाभ्यामिव विकटमकरमुखप्रतिवन्धवन्धुराभ्यामुद्वेललावण्यपयोनि विप्रवाहाभ्यासिव फेनाहितशोभाभ्यां चन्दनद्रुमाभ्यामिव भोगिमण्डलशिरोरत्नरश्मिरज्यमानमूलाभ्यां हृदयारोपितभूभारधारणमाणिक्यस्तम्माभ्यामुरुदण्डाभ्यां विराजमानम्, अमृतफेनपिण्डपाण्डुना मेखलामणिमयूखखचितेन नितम्वबिम्बव्यासङ्गिना विमलपयोधौतेन नेवसूत्रनिवेशशोभिनाधरवाससा वासुकीनिमोंकेणेव

\vspace{2mm}
\hrule

\noindent
{\s सुगतः~। वज्राख्यमायुधं तद्वन्निष्ठुरं कटोरं प्रकोष्टस्य पृष्टं यस्य तेन~। इन्द्रश्चास्य वज्रमायुधम्~। {\qt प्रकोष्टमन्तरं विद्यादरत्निमणिबन्धयोः}~। वृषो दान्तः, धर्मथ~। भास्वद्भास्वरम्, रविश्व भाखान्~। बिम्बं फलभेदः, मण्डलं च~। अवलोकितं वीक्षितम्, बुद्धिभेदश्चावलोकितः~। कृष्णः कालः, हरिश्च कृष्णः~। {\qtt कलिकालेति}~। कलिकालस्य मलिनवादेवमुत्प्रेक्षा~। वामपादेन पराभवनीयत्वमेव पोष्यते~। कालियो नागभेदः~। पुण्डरीकाक्षमिति राज्ञो विशेषणम्~। तेजखिनो वीराः, आदित्याश्च~। जलजेत्यादीनि महाराजविशेषणानि लक्षणानि~। एवमादि च संभवति~। मकरमुखं जानुसंधिः, मकरमुखचिह्नितान्त कपोलक्ष~। उद्देलतया लावण्यस्य समुच्छलद्रूपत्वमाह~। फेनो रससंतानः, डिण्डीरथ~। भोगिनो नृपाः, सर्पाश्च~। फेनवत्तैव पाण्डु~। मेखला रशना, पर्वतमव्यभूमिश्च~। पयो जलम्, क्षीरं च~। नेत्रसूत्रं पहसूत्रम्, मन्थनरजुश्च~। अघनेन छातेन, अनत्रेण च~। ताराः सूत्रबि\textendash}

\newpage
% द्वितीय उच्छ्वासः~। ७३ 

\noindent
मन्दरं द्योतमानम्, अघनेन सतारागणेनोपरिकृतेन द्वितीयाम्बरेण भुवनाभोगसिव भासमानम्, इभपतिदशनमुसलसहस्रोल्लेखकठिनमसृणेनापर्याप्ताम्बरप्रथिना विविधवाहिनीसंक्षोभकलकलसंमर्दसहिष्णुना कैलासमिव महता स्फटिकतटेनोरुणोरःकवाटेन राजमानम्, श्रीसरस्वत्योरुरोवदनोपभोगविभागसूत्रेणेव पातितेन शेषेणेव च तद्भुजस्तम्भविन्यस्तसमस्तभूभारलव्धविश्रान्तिसुखप्रसुप्तेन हारदण्डेन परिवलितकंवरम्, जीवितावधिगृहीतसर्वस्वमहादानदीक्षाचीरेणेव हारमुक्ताफलानां किरणनिकरेण प्रावृतवक्षःस्थलम्, अजजिगीषया बालैर्भुजैरिवापरैः प्ररोहद्भिर्वाहूपधानशायिन्याः श्रियाः कर्णोत्पलमधुरसधारसंतानैरिव गलर्भुिजजन्मनः प्रतापस्य निर्गमनमार्गेरिवाविर्भवद्भिररुणैः केयूररत्नकिरणदण्डैरुभयतः प्रसारितमणिमयपक्षवितानमिव माणिक्यमहीवरम्, सकललोकालोकमार्गार्गलेन चतुरुदधिपरिक्षेपखातशिलाप्राकारेण सर्वराजहंसबन्धवज्रपञ्जरेण भुवनलक्ष्मीप्रवेशमङ्गलमहामणितोरणेनातिदीर्घदोर्दण्डयुगलेन दिशां दिक्पालानां च युगपदायतिमपहरन्तम्, सोदर्यलक्ष्मीचुम्बनलोभेन कौस्तुभमणेरिव मुखावयक्तां गतस्याधरस्य गलता रागेण पारिजातपल्लवरसेनेव सिवन्तम्, दिड्मुखान्यन्तरान्तरा सुहृत्परिहासस्मितैः प्रकीर्यमाणविमलदशन शिखाप्रतानैः प्रकृतिमूढाया राजश्रियाः प्रज्ञालोकमिव दर्शयन्तम्, मुखजनितेन्दुसंदेहागतानि कुमुदिनीवनानीव प्रेषयन्तम्, स्फुटधवलद्शनपङ्किकृतकुमुदवनशङ्काप्रविष्टां शरज्ज्योत्स्नामिव विसर्जयन्तम्, मदिरामृतपारिजातगन्धगर्मेण भरितसकलककुभा मुखामोदेनामृतमथनदिवसमिव सृजन्तम्, विकचमुखकमलकार्णिकां

\vspace{2mm}
\hrule

\noindent
{\s न्दवः, नक्षत्राणि च~। अम्बरं वासः, नभव~। इभपतीत्यादि साधारणम्~। अपर्याप्तमम्बरं वासो यस्य तादृक्प्रथिमा यस्य, अम्बरं च खम्~। वाहिनी सेना, नदी च~। अन्यपर्वतसावारण्येऽपि छायावत्त्वादुन्नतत्वाव कैलासमिवेत्युक्तम्~। हारेत्यादिना उरुत्वं काठिन्यमाह~। परिवलिता~। {\qt परिवेष्टित} \textendash\ इति पाठे व्यातेत्यर्थः~। अजो हरिः~। भुजेत्यादिना सेनादिकृतं नयादिकृत च प्रतापं व्यवच्छिनत्ति~। माणिक्यमुत्कृष्टो मणिः~। चतुर्णानुदधीनां संबन्धी परिक्षेप एवं खातं परिखा यस्य स तादृग्दाच्छिलाप्राकार इव तेन~। परिखां कृलान्तरे प्रकारो दीयते इति स्थितिः~। राजहंसा राजोत्तमाः, हंसमेदाश्च~। आयतिदेयम्, प्रता\textendash}

\newpage
% ७४ हर्षचरिते

\noindent
कोशेनानवरतमापीयमानश्वाससौरभ मिवाधोमुखेन नासावंशेन चक्षुषः क्षीरस्निग्धस्य धवलिना दिड्युखान्यपूर्ववदनचन्द्रोद्वेलक्षीरोदलावितानीव कुर्वाणम्, विमलकपोलफलकप्रतिविम्बितां चामरग्राहिणी विग्रहिणीमिव मुखनिवासिनीं सरस्वत धानम्, अरुणेन चूडामणिशोचिषा सरस्वतीर्ण्याकुपितलक्ष्मीप्रसादनलग्नेन चरण लचकेनेव लोहितायतललाटतटम्, आपाटलांशुतन्त्रीसंतानवलयिनी कुण्डलमणिकुटिलकोटिवालवीणा मनवरतच लितचरणानां वायत्तामुपवीणयतामिव स्वरव्याकरणविवेकविशारदम्, श्रवणावतंसमधुकरकुलानां कलक्कणितमाकर्णयन्तम्, उत्फुल्लमालतीमयेन राजलक्ष्म्याः कचग्रहलीलालग्नेन नखज्योत्स्नावलयेनेव मुखशशिपरिवेशमण्डलेन मुण्डमालागुणेन परिकलितकेशान्तम्, शिखण्डाभरणभुवा मुक्ताफलालोकेन मरकतमणिकिरणकलापेन चान्योन्यसंवलनवृजिनेन प्रयागप्रवाहवेणिकावारिणेवागय स्वयमभिषिच्यमानम्, श्रमजलविलीनबहलकृष्णागुरुपङ्कतिलककलङ्ककल्पितेन कालिन्ना प्रार्थनाचाटुचतुरचरणपतनशतश्यामिकाकिणेनेव नीला यमानललाटलेखाभिः क्षुभित्तमानसोद्द्तैरुत्कलिकाकलापैरिव हारैरुल्लसद्भिरवष्टभ्यमानाभिर्विलासवलानचटुलताकल्पैरीया श्रियमिव तर्जयन्तीमिरायामिभिः श्वसितैरविरलपरिमलैर्मलयमारुतमयैः पाशैरिवाकर्षन्तीभिर्विकट बकुलाव लीवराटकवेष्टितमुखैर्बृ\textendash

\vspace{2mm}
\hrule

\noindent
{\s पश्च~। कर्णिका कोशः, चक्रं च~। आपीयमानं श्वाससौरभं यस्य तम्~। {\qtt अधोमुखेनेति}~। अनेन सुलक्ष्यलं सौरभस तथापीयमानानुमितिं दर्शयति~। अंशुरेव तन्त्रीसंतानः~। स एव वलयाकारयाद्वलयं विद्यते यस्यास्ताम्~। कुण्डलमणिकुटिलकोटिमेव बालवीणां तक विपक्ष वादयताम्~। अनवरतेत्यादिना व्यापारसादृश्येनोक्तम्~। {\qtt वद(वादय)तामिति}~। वीणयोपगायत्तामुपवीण यतामिति गानस्य प्राधान्यं प्रतिपादयति~। स्वरव्याकरणं विशालादिकमित्यादिना गानं दर्शयति~। परिवेशः परिधिः~। वृजिनेन शकलेन, कलुषेण वा~। प्रयागो गजायमुनासंगमः~। तत्प्रवाहस्य वैणिकारूपेण वारिणेद~। श्रमजलेत्यादौ बारविलासिनीभिः सर्वतो विदुप्यमान सौभाग्यमिवेति संबन्धः~। प्रार्थनाचाट्वित्यादौ प्रार्थनादीनि सर्वाणि श्रीहर्षविषयाणि ज्ञेयानि~। मानसं सरः, चेतश्च~। उत्कलिका रुहरुहिकाः, वीचयश्च~। अविरलेत्यादिना वारणम्~। आकर्षणं वशीकरणम्, समीपप्रापणं च~। विकटेयादिनोद्दीपनभावमेव पोषयति~। वराटको रज्जुः~। {\qtt बृहद्भिरिति}~। बृहत्त्वेन हृद्यत्वमेषामाह~। वृहस्वादेव च वक्ष्यति \textendash\ {\qtt अशेष}\textendash}

\newpage
%द्वितीय उच्छ्वासः~। ७५ 

\noindent
हद्भिः स्तनकलशैः स्खदारसंतोषरसमिवाशेपमुद्धरन्तीभिः कुचोत्कम्पिकाविकारप्रेङ्खितानां हारतरलमणीनां रश्मिभिराकृष्य हृदयमिव हठात्प्रवेशयन्तीभिः प्रभामुचामाभरणमणीनां मयूखैः प्रसारितैर्बहुभिरिव बाहुभिरालिङ्गन्तीभिजृम्भानुबन्धबन्धुरवदनारविन्दावरणीकृतैरुत्तानैः करकिसलयैः सरसप्रधावितानि मानसानीव निरुन्धतीभिर्मदनान्धमधुकरकुलकीर्य माणकर्णकुसुमरजःकणकूणितकोणानि कुसुमशरशरनिकरप्रहारसूर्च्छामुकुलितानीव लोचनानि चतुरं संचारयन्तीभिरन्योन्यमत्सरादाविर्भवद्भङ्गुरभ्रुकुटिविभ्रमक्षिप्तैः कटाक्षैः कर्णेन्दीवराणीव ताडयन्तीभिरनिमेषदर्शनसुखरसराशि मन्थरितपक्ष्मणा चक्षुषा पीतमिव कोमलकपोलपालीप्रतिबिम्बितं वहन्तीभिरभिलाषलीलानिर्निमित्तस्मितैश्चन्द्रोदयानिव मदनसाहायकाय संपादयन्तीभिरङ्गभङ्गवलनान्योन्यघटितोत्तानकरवेणिकाभिः स्फुटनमुखराङ्गुलीकाण्डकुण्डलीक्रियमाणनखदीधितिनिवहनिभेनाकिंचित्करकामकार्मुकाणीव रुपा भञ्जतीभिर्वारविलासिनीभिर्विलुप्यमानसौभाग्यमिव सर्वतः, स्पर्शस्विनवेपमानकरकिसलयगलितचरणारविन्दां चरणग्राहिणीं विहस्य कोणेन लीलालसं शिरसि ताडयन्तम्, अनवरतकरकलितकोणतया चात्मनः प्रियां वीणामिव श्रियमपि शिक्षयन्तम्, निःस्नेह इति धनैरनाश्रयणीय इति दोषैर्निग्रहरुचिरितीन्द्रियैर्दुरुपसर्प इति कदिना नीरस इति व्यसनैर्भीरुरित्ययशसा दुर्ग्रचित्तवृत्तिरिति चित्तभुवा स्त्रीपर इति सरस्वत्या षण्ढ इति परकललैः काष्ठामुनिरिति यतिभिर्वृर्त इति वेश्याभिर्नेय इति सुहृद्भिः कर्मकर इति

\vspace{2mm}
\hrule

\noindent
{\s {\qtt मिति~। स्तनकलौरिति}~। स्तनैः किल रज्जुबेष्टितमुखै रसो जलमुद्रियते रसोऽभिलाषः, जलं च~। बन्धुरं हृयम्~। कूणितः संकोचितः~। मदनादिशब्दे विद्यमानेऽपि मदनान्धेत्यभिप्रायेण कुसुमशरग्रहणम्~। अत्र पक्षे कर्णपदं त्यज्यते~। अनिमेषदर्शनसुखरसराशिमिव श्रीहर्षम्~। {\qtt प्रतिविम्बितमिति}~। अथ च रसो जलादिः~। बिमले मणिभाजनादावन्तर्वर्त्यपि प्रतिबिम्बितो लक्ष्यते~। करवेशिका परस्परानुबन्धस्थितकरद्वयाह लिविन्यासः~। विलुप्यमानसौभाग्याविना ताः सुभगा इत्यर्थः~। कोणो वीणादिवाद्नभाण्डम्~। {\qtt प्रियामिति}~। वीणायाः श्रियाश्च विशेषणम्~। निःस्नेह इत्यादा वेतैरेकमप्यनेकथा गृह्यमाणमिति संबन्धः~। षण्ढः प्रजननाक्षमः~। काष्ठा पराधारा तत्प्रधानो मुनिः काष्ठामुनिरतिशयवांस्तपस्वी~।}

\newpage
% ७६ हर्षचरिते 

\noindent
विप्रैः सुसहाय इति शत्रुयोधैरेकमप्यनेकथा गृह्यमाणम्, शन्तनोर्महावाहिनीपतिम्, भीष्माज्जितकाशिनम्, द्रोणाचापलालसम्, गुरुपुवादमोवमार्गणम्, कर्णान्मित्रप्रियम्, युधिष्ठिराद्धदक्षमम्, भीमाद्नेकनागायुतबलम्, धनंजयान्महाभारतरणयोग्यम्, कारणमिव कृतयुगस्य, बीजमिव विबुधसर्गस्य, उत्पत्तिद्वीपमिव दर्पस्य, एकागारभिव करुणायाः, प्रातिवेशिकमिव पुरुषोत्तमस्य, खनिपर्वतमिव पराक्रमस्य, सर्वविद्यासंगीतगृहमिव सरस्वत्याः, द्वितीयामृतमथनदिवसमिव लक्ष्मीसमुत्थानस्य, बलदर्शनमिव वैदग्ध्यस्थ, एकस्थानमिव स्थितीनाम्, सर्वस्वकथनमिव कान्तेः, अपवर्गमिव रूपपरमाणुसर्गस्य, सकलदुश्चरितप्रायश्चित्तमिव राज्यस्य, सर्वबलसंदोहावस्कन्दमिव कन्दर्पस्य, उपायमिव पुरंदरदर्शनस्य, आवर्तनमिव धर्मस्य, कन्यान्तः पुरमिव कलानाम्, परमप्रमाणमिव सौ

\vspace{2mm}
\hrule

\noindent
{\s नेयः परवशः~। शन्तनुर्नाम राजा भीष्मस्य पिता चाहिन्या गङ्गायाः पतिः~। अयं तु तस्मादपि महतीनां वाहिनीनां सेनानां पतिः शन्तनुरिति~। {\qt पञ्चमी विभक्ते} इति पञ्चमी~। भीष्मो जितकाशी जितेन्द्रियः~। यतस्त्वयि त्वत्पुत्रे वा सत्यस्मद्दौहित्रस्य कुतो राज्यमिति यदा हि दाशाधिपतिना खसुता मत्स्योदरोगता मात्स्यावती नामास्मै पित्रर्थमर्थयते न दत्ता, तदैतेन प्रतिज्ञातम् \textendash\ {\qt नाहं राज्यं विवाहं वा करिष्यामि} इति~। अत एव ब्रह्मचार्येवाभूत्~। राजा च ततोऽपि जितकाशितमः, जितकाशी वा~। जितेन जयेन काशते शोभते यः~। तथा हि भीष्मेण रामो जितः~। सर्वराजमहितं च काशिराजं च जित्ला भ्रात्रर्थमम्बादिकन्यात्रयमनैषीत्~। राजा तु ततोऽपि जितकाशितमः~। द्रोणश्चापाचार्यः~। स चापे धनुषि लालसः~। चपलं न करोतीत्यर्थः~। यद्वा चः समुच्चये~। अपगता लालसा यस्य सोऽपलालसः~। निरभिलाष इत्यर्थः~। गुरुपुत्रोऽश्वत्थामा तस्य सफलशरता~। तथा शस्त्रोपसंहारोऽक्षमयाचितोऽपि कस्यचिदेकस्य भारणमन्तरेण न तदुप संजहार~। तत उत्तराया उदरस्थे परीक्षिति पाटिते तस्मिंस्तदुपसंहृतवान्~। अन्यत्रामोधा मार्गणा याचका यस्येति~। मित्रः सूर्यः, सुहृञ्च मित्रम्~। क्षमा शान्तिः, भुश्च~। अनेकानि बहूनि, अनन्यसदृशानि च~। एकशब्दस्य च साघारणार्थं तृच्~। बलं सामर्थ्यम्, सैन्यं च~। धनंजयोऽर्जुनः~। महाभारतानां कुरूणां यो रणः सङ्ग्रामः~। अन्यत्र महतो भारस्य कार्यवरायास्तरणं निर्वाहणम्~। प्रातिवेशिकं प्रतिविम्बम्~। खनिराकरः~। अपवर्गः समाप्तिः~। संदोह समूहः~। अवनृश्रो यज्ञान्तः~। गम्भीरं प्रसन्नं चेति परस्परापेक्षं बोद्धव्यम्~। तथा च सति गम्भीरत्वे प्रसन्नत्वं ऋजुत्वं चेन्न स्यात्ततो जियप्रकृतित्वं, प्रसज्येत~। एवं त्रासेत्यादौ}

\newpage
% द्वितीय उच्छ्वासः~। ७७ 

\noindent
भाग्यस्य, राजसर्गसमाप्त्यवभृथस्नानदिवसमिव सर्वप्रजापतीनाम्, गम्भीरं च, प्रसन्नं च, वासजननं च, रमणीयं च, कौतुकजननं च, पुण्यं च, चक्रवर्तिनं हर्षमद्राक्षीत्~।

दृष्ट्वा चानुगृहीत इव निगृहीत इव साभिलाष इव तृप्त इव रोमाञ्चमुचा मुखेन मुञ्चन्नानन्दवाष्पवारिबिन्दून्दूरादेव विस्मयस्मेरः समचिन्तयत् \textendash\ सोऽयं सुजन्मा, सुगृहीतनामा, तेजसा राशिः, चतुरुद्धिकेदारकुटुम्बी, भोक्ता ब्रह्मस्तम्भफलस्य, सकलादिराजचरितजयज्येष्ठमल्लो देवः परमेश्वरो हर्षः~। एतेन च खलु राजन्वती पृथ्वी~। नास्य हरेरिव वृषविरोधीनि वालचरितानि, न पशुपतेरिव दक्षोद्वेगकारीण्यैश्वर्य विलसितानि, न शतक्रतोरिव गोत्रविनाशपिशुना: प्रवादाः, न यमस्येवातिवल्लभानि दण्डग्रहणानि, न वरुणस्येव निखिंशग्राहसहस्ररक्षिता रत्नालयाः, न धनदस्थेव निष्फलाः सन्निधिलाभाः, न जिनस्येवार्थवादशून्यानि दर्शनानि, न चन्द्रमस इव बहुलदोषोपहताः श्रियः~। चित्रमिदमत्यमरं राज\textendash

\vspace{2mm}
\hrule

\noindent
{\s बोद्धव्यम्~। तथा च कालिदासः {\qt भीमकान्तैर्नृपगुणैः स बभूवोपजीविनाम्~। अनुष्यवाधिगम्यश्च यादोरलैरिवार्णवः~॥} इति दिलीपं प्रति वर्णितवान्~। कौतुकजननमपुण्यवादपि संभाव्यते~। अत आह \textendash\ {\qtt पुण्यमिति}~। गम्भीरं च प्रसन्नं चेत्यादौ सर्वत्र विरोध उद्भाव्यः~। गम्भीरं सतमिस्रं प्रसन्नं निर्मलं न भवतीति~।

{\qtt अनुगृहीत इवेत्यादि}~। एवंविधमहीपतिप्रसादवशात्~। {\qtt निगृहीत इवेति}~। संकोचवशात्~। {\qtt साभिलाष इवेति}~। तस्य दर्शनीयत्वात्~। {\qtt तृप्त इवेति}~। तथैव तस्य {\qtt कृतार्थत्वात्}~। विरोधो यत्र सुबोध:~। केदारं क्षेत्रम्~। ब्रह्मस्तम्भं जगत्~। फलं रत्नादि~। यच्च स्तम्भस्य फलं धान्यादि तद्भोक्ता कर्षको भवति~। राजन्वती प्रशस्त राजयुता~। वृषो धर्मः, अरिष्टासुरो दान्तरूपश्च~। {\qtt बालेति}~। वाला हि विवेकहीनत्वाद्धर्मविरुद्धमाचरन्ति~। अस्य तु तस्यामपि दशायां धर्मविरोधाभावः~। दक्षः कुशलः, प्रजापतिमेदव~। महेश्वरपक्ष ऐश्वर्यशब्दो मुख्यवृत्तिः, इतरत्र गौणः~। गोत्रं कुलम्, कुलपर्वता गोत्राः~। {\qtt अतिवल्लभानीति}~। अतिशब्देन युक्तदण्डत्वमाह~। दण्डः करः, यमायुधं च~। निशिग्राहाः खङ्गहस्ताः, तत्र जलचरभेदाच~। रत्नालया भाण्डागाराणि, समुद्राक्ष~। निष्फला ऐश्वर्यादिफलप्राप्तिशून्याः, दानादिविनाकृताश्च~। सन्निधिः सन्निधानम्~। एतस्य दर्शनं सर्वस्य फलदायि भवतीत्यर्थः~। अन्यत्र संविषयः शोभनानि निधनान्यस्य~। दर्शनानि जिनस्येवार्थवादशज्यानि~। अर्थों धनं तस्य वादः, अनेनेदं लब्धमिति, तेन शून्यानि~। सर्वे तद्दर्शिनोऽर्थेन युज्यन्ते~। जिनस्य पुनरर्थवा\textendash}

\newpage
% ७८ हर्षचरिते 

\noindent
त्वम्~। अपि चास्य त्यागस्यार्थिनः, अज्ञायाः शास्त्राणि, कवित्वस्य वाचः, सत्त्वस्य साहसस्थानानि, उत्साहस्य व्यापाराः, कीर्तेर्दिङ्मुखानि, अनुरागस्य लोकहृदयानि, गुणगणस्य संख्या, कौशलस्य कला, न पर्याप्तो विषयः~। अस्मिंश्च राजनि यतीनां योगपट्टकाः, पुस्तकर्मणां पार्थिवविग्रहाः, षट्पदानां दानग्रहणकलहाः, वृत्तानां पादच्छेदाः, अष्टापदानां चतुरङ्गकल्पना, पन्नगानां द्विजगुरुद्वेषाः, वाक्यविदामधिकरणविचाराः इति समुपसृत्य चोपवीती स्वस्ति शब्दमकरोत्~।

अथोत्तरे नातिदूरे राजधिष्ण्यस्य गजपरिचारको मधुरमपर वक्रमुच्चैरगायत् \textendash\ 

\vspace{-5mm}
\begin{quote}
{\ha करिकलभ विमुच लोलतां चर विनयत्रतमानताननः~।\\
मृगपतिनखकोटिभङ्गुरो गुरुरुपरि क्षमते न तेऽङ्कुशः~॥~५~॥}
\end{quote}

\vspace{-5mm}
राजा तु तच्छ्रुत्वा दृष्ट्वा च तं गिरिगुहागत सिंहबृंहितगम्भीरेण स्वरेण पूरयन्निव नभोभागमपृच्छत् \textendash\ {\haq एष स बाणः} इति~। {\haq यथाज्ञापयति देवः~। सोऽयम्} इति विज्ञापितो दौवारिकेण~। न ता

\vspace{2mm}
\hrule

\noindent
{\s दशून्यानि महायानयोगाचारमाध्यमिकदर्शनानि~। बहुलाः प्रभूता दोषा रागाद्याः~। बहुलदोषाच कृष्णपक्षरात्रयः~। श्रियः समृद्धयः, शोभाच~। पर्याप्तः परिपूर्णः~। योगपटका यतीनामुपकरणं पर्यङ्कबन्धनार्थम्~। ते यतीनां चतुर्थाश्रमिणामेव~। न पुनयोगेन युक्ताः पट्टका: कूटप्रधानानि लेख्यपन्नाणि केषांचित्~। एवमन्यत्रापि~। पुस्तकर्म लेप्यम्~। पार्थिववित्रहा मृण्मयशरीराणि, राजभिः सह वैराणि च~। दानग्रहं मदजलम् दानमृणव्यवहारथ~। वृत्तानां गुरुलघुनियमात्मकानां समाश्व समविषमानां पादच्छेदा भागविरामाः, चरणकर्तनानि च~। अष्टापदानां चतुरङ्गफलकानाम्~। {\qt चलार्यङ्गानि सेनाया हस्त्यश्वरथपत्तयः}~। तेषां कल्पना रचना~। चतुर्णामङ्गानां पाणिपादस्य च छेदः~। द्विजगुरुर्गरडोऽपि, वाक्यविदां गीमांसकानामधिकरणविश्रान्तिस्थानानि राज्ञां च धर्मनिर्णयस्थानानि~। अधिकवलो वारणः सङ्ग्राम इति केचिन्~। उपवीती दक्षिणावीती करः~। उक्तं च \textendash\ {\qt उद्धृते दक्षिणे पाणावुपवीत्युच्यते द्विजः} इति~।

{\qtt गजपरिचारक इति}~। अन्यगजपरिचारकस्य खजातिसमुचित वस्तु राज्ञः प्रकृतस्मारकं जातम्~। तत्र करिणां स्वभावत एव रागित्वादस्थापि रागवत्त्वाद्भुजंगतास्मृतिः संजातेति~। भङ्गुरो वक्रः~। मृगपतिनखकोटिवद्वक इति स्पधा व्याख्या~। गुरुर्भारः, शासिता च~। उपरि पृष्ठदेशे, प्रभुभावे च~। अङ्कुश इवाङ्कुश इलपि~। अत आह \textendash\ {\qtt तच्छ्रुत्वेति}~। बृंहितं गर्जित्तम्~। अंशव एवांशुकाः~।}

\newpage
% द्वितीय उच्छ्वासः~। ७९ 

\noindent
वदेनमकृतप्रसादः पश्यामि इति तिर्यङ्नीलधवलांशुकशारां तिरस्करिणीमिव भ्रमयन्नपाङ्गनीयमानतरलतारकस्यायामिनीं चक्षुषः प्रभां परिवृत्य प्रेष्ठस्य पृष्ठतो निषण्णस्य मालवराजसूनोरकथयत् \textendash\ {\haq महानयं भुजङ्गः} इति~। तूष्णींभावेन त्वगमितनरेन्द्रवचसि तस्मिन्मूके च राजलोके मुहूर्तमिव तूष्णीं स्थित्वा बाणो व्यज्ञापयत् \textendash\ देव, अविज्ञाततत्त्व इव, अश्रद्धान इव, नेय इव, अविदितलोकवृत्तान्त इव च कस्मादेवमाज्ञापयसि~। स्वैरिणो विचित्राश्च लोकस्य स्वभावाः प्रवादाश्च~। महद्भिस्तु यथार्थदर्शिभिर्भवितव्यम्~। नार्हसि मामन्यथा संभावयितुम विशिष्टमिव~। ब्राह्मणोऽस्मि जातः सोमपायिनां वंशे वात्स्यायनानाम्~। यथाकालमुपनयनादयः कृताः संस्काराः~। सम्यक्पठितः साङ्गो वेदः~। श्रुतानि यथाशक्ति शास्त्राणि~। दारपरिग्रहाद्भ्यगारिकोऽस्मि~। कामे भुजङ्गता~। लोकद्वयाविरोधिभिस्तु चापलैः शैशवमशून्यमासीत्~। अलानपलापोऽस्मि~। अनेनैव च गृहीतविप्रतीसारमिव मे हृदयम्~। इदानीं तु सुगत इव शान्तमनसि मनाविव कर्तरि वर्णाश्रमव्यवस्थानां समवर्तिनीव च साक्षादण्डभृति देवे शासति सप्ताम्बुराशिरशनाम शेषद्वीपमालिनीं महीं के इवाविशङ्कः सर्वव्यसनबन्धोरविनयस्य मनसाध्यभिनय कल्पयिष्यति~। आसतां तावन्मानुष्यकोपेताः~। त्वत्प्रभावादलयोऽपि भीता इव मधु पिबन्ति~। रथाङ्गनामानोऽपि लज्जन्त इवाभ्यनुवृत्तिव्यसनैः प्रियाणाम्~। कपयोऽपि चकिता इव चपलायन्ते~। शरारवोऽपि सानुक्रोशा इव श्वापद्गणाः पिशितानि

\vspace{2mm}
\hrule

\noindent
{\s अंशुकं च वस्त्रम्~। तिरस्करिणी जवनिका~। प्रेष्टस्यातिप्रियस्य~। नेयः परक्शः~। स्वैरिणः स्वतन्त्राः~। सोमपायिनां सोमपानाम्~। {\qt शिक्षा कल्पणे व्याकरणं ज्योतिषं निरुक्तं छन्दो विधिः} इति षडङ्गानि वेदस्य~। अभ्यागारिको गृहस्थः, सम्यग्वृत्तिस्थितो वा~। कामे भुजंगतेति कामभुजंगता शृङ्गारित्वम्~। कामे मदने भुजंगता ज्ञेया, न मादृशेषु~। नहि मे काचिद्भुजं बाहुं गता प्राप्तेत्यर्थः~। लोकद्वयेत्यादिना त्रिवर्गस्यानुपधातं दर्शयति~। शास्त्रविरोधप्रसङ्गात्~। शतायुर्वै पुरुषः कालमन्योन्यानुबद्धं परस्परस्यानुपघावेन त्रिवर्गं सेवत इत्यत एवाह \textendash\ {\qtt शैशवमिति~। अशून्यमिति}~। अनेन तदेकासकत्वं परिहरति~। अनपठाणे तिरप ह्नवः~। विप्रतीसारः पश्चात्तापः~। सुगतो बुद्धः~। समवर्ती यमः~। मनुष्यस्य भावो मानुष्यकम्~। रथाशनामानवकवाकाः चपलायन्ते चपळलमांवरन्ति~। शरारवो}

\newpage
% ८० हर्षचरिते 

\noindent
भुञ्जते~। सर्वथा कालेन मां ज्ञास्यति स्वामी स्वयमेव~। अनपाचीनचित्तवृत्तिमाहिण्यो हि भवन्ति प्रज्ञावत प्रकृतयः इत्यभिधाय तूष्णीमभूत्~।

भूपतिरपि {\haq एवमस्माभिः श्रुतम्} इत्यभिधाय तूष्णीमेवाभवत्~। संभाषणासनदानादिना तु प्रसादेन नैनमन्वग्रहीत्~। केवलममृतवृष्टिभिः स्नपयन्निव स्नेहगर्भेण दृष्टिपातमात्रेणान्तर्गतां प्रीतिमकथयत्~। अस्ताभिलापिणि च लम्बमाने सवितरि विसर्जितराजलोको ऽभ्यन्तरं प्राविशत्~। बाणोऽपि निर्गत्य धौतारकूटकोमलतपत्विषि निर्वाति वासरे, अस्ताचलकूटकिरीटे निचुलमञ्जरीभांसि तेजांसि मुञ्चेति वियन्मुचि मरीचिमालिनि, अतिरोमन्थमन्थरकुरङ्गकुटुम्बकाध्यास्यमानम्नदिष्ठगोष्ठीनष्पृष्ठावरण्यस्थलीषु, शोकाकुलकोकका मिनीकूजितकरुणासु तरङ्गिणीतटीषु, वासविटपोपविष्टवाचाटचढ कचक्रवालेवालवालावर्जित सेकजलकुटेषु निष्कुटेषु, दिवसविहृतिप्रत्यागतं प्रस्तुतस्तनं स्तनंधये धयति धेनुवर्गमुद्रतक्षीरं क्षुधिततर्णकव्राते, क्रमेण चास्तधराधरधातुधुनीपूरलावित इव लोहितायमानमहसि मज्जति संध्यासिन्धुपानपात्रे पातङ्गे मण्डले, कमण्डलुजलशुचिरायचरणेषु चैत्यप्रणतिपरेषु पाराशरिषु, यज्ञपात्रपवित्रपाणौ प्रकीर्णवर्हिष्युत्तेजसि जातवेदसि, हवींषि वषट्कुर्वति यायजूकजने निद्राविद्राणद्रोणकुलकलिलकुलायेषु कापेयविकलकपिकुलेष्वारामतरुषु, निर्जिगमिषति जरत्तरुकोटरकुटीकुटुम्बिनि कौशिककुले, मु\textendash

\vspace{2mm}
\hrule

\noindent
{\s हिंसाः~। श्वापदगणाः प्राणिसमूहाः~। पिशितं मांसम्~। अनपाचीनामृष्टा~। अवि परीतेत्यर्थः~। निर्दोषा वा~।

बाणोपीइत्यादी बाणोऽप्यस्मिन्सति निवासस्थानमगादिति संवन्धः~। {\qt रीतिः स्त्रियामारकूटम्} इत्यमरः~। निर्वाति शाम्यति~। निचुलो वेतसवृक्षः~। भुक्तोद्गीर्णाहारचर्वणं रोमन्थः~। म्रदिष्ठं मृदुतमम्~। गोष्ठीपूर्व गोष्टीनम्~। {\qt गोष्ठात्खञ्भूत पूर्वे}~। उक्तं च \textendash\ गोष्टं गोस्थानकं तत्तु गोष्ठीनं भूतपूर्वकम् इति~। कोकाश्चक्र वाकाः~। तरक्षिणी नदी~। आलवालमावापः~। कुटा घटाः~। निष्कुटाः स्वगृहारामाः~। स्तनधयस्तर्णकच वत्सः~। धुनी नदी~। सिन्धुः समुद्रः~। शयः करः~। चैत्यमायतनम्~। पाराशरिषु भिक्षुषु~। वहींषि कुशाः~। वषडिति दानक्रियासु मोच नमन्त्रः~। वषट्कुर्वति~। जुहतीत्यर्थः~। यायजूकोऽत्यर्थ यजनशीलः~। विद्राणोऽलसः द्रोणः काकः~। कलिला आकुलाः~। कुलायो नीडमस्त्रियाम्~। कापेयं चाप\textendash}

\newpage
% द्वितीय उच्छ्वासः~। ८१

\noindent
निकरसहस्रप्रकीर्णसंध्यावन्दनोदबिन्दुनिकर इव दन्तुरयति तारापथस्थलीं स्थवीयसि तारकानिकुरम्बे, अम्बराश्रयिणि शर्वरीशवरीशिखण्डे, खण्डपरशुकण्ठकाले कवलयति बाले ज्योतिःशेषं सांध्यमन्धकारावतारे, तिमिरतर्जननिर्गतासु दहनप्रविष्ठदिनकरकर शाखाखिव स्फुरन्तीषु दीपलेखासु, अररसंपुटसंक्रीडनकथितावृत्तिष्विव गोपुरेषु, शयनोपजोषजुषि जरतीकथितकथे शिशयिषमाणे शिशुजने, जरन्महिषमपीमलीमसतमसि जनितपुण्यजनप्रजागरे विजम्भमाणे भीषणतमे तमीमुखे, मुखरितविततज्यधनुषि वर्षति शरनिकरमनवरतमशेषसंसारशेमुषीमुषि मकरध्वजे, रताकल्पारम्भशोभिनि शम्भलीभाषितभाजि भजति भूषां भुजिष्याजने, सैरन्ध्रीबध्यमानरशनाजालजल्पाकजघनासु जनीषु, वशिकविशिखाविहारिणीष्वनन्यजानुप्तवासु प्रचलितास्वमिसारिकासु, विरलीभ वति वरटानां वेशन्तशायिनीनां मञ्जुनि मञ्जीरशिञ्जितजडे जल्पिते, निद्राविद्राणद्राघीयसि द्रावयतीव च विरहिहृदयानि सारसरसिते, भाविवासरबीजाङ्कुरनिकर इव च विकीर्यमाणे जगति प्रदीपप्रकरे निवासस्थानमगात्~। अकरोच्च चेतस्पतिदक्षिणः खलु देवो हर्षः \textendash\ यदेवमनेकबालचरितचापलोचितकौलीनकोपितोऽपि मनसा स्त्रिह्यत्येव मयि~। यद्यहमक्षिगतः स्याम्, न मे दर्शनेन प्रसादं कुर्यात्~। इच्छति तु मां गुणवन्तम्~। उपदिशन्ति हि विनय मनुरूपप्रतिपत्त्युपपादनेन वाचा विनापि भर्तव्यानां स्वामिनः~। अपि च धिङ्मां स्वदोषान्धमानसमनादरपीडितमेवमतिगुणवति राजन्यन्यथा चान्यथा च चिन्तयन्तम्~। सर्वथा करोमि तथा, यथा यथावस्थितं जानाति मामयं कालेन इत्येवमवधार्य चापरयुनिष्क्रम्य कटकात्सुहृदां बा\textendash

\vspace{2mm}
\hrule

\noindent
{\s लम्~। कौशिका उलूकाः~। स्थवीयसि स्थूलतरे~। शिखण्डो जुटकः~। खण्डपरशुः शिवः~। करा एव शाखास्तदाकारत्वादडलयश्च करशाखाः~। अररः कपाटः~। संक्रीडनं शब्दः~। आवृत्तिः स्थगनम्~। {\qt गोपुरं स्यात्पुरद्वारं द्वारमात्रेऽपि गोपुरम्}~। उपजोषः सुखम्, तूष्णींभावो वा~। जरती वृद्धा~। शिशयिषमाणे सुषुप्सति~। यक्षाः स्युः पुण्यजनाः~। तमी रात्रिः~। शेमुषी बुद्धिः~। आकल्पो वेशः~। शम्भली कुट्टनी~। सुजिष्या दासी~। सैरन्ध्री प्रसाधनोपचारज्ञा~। जनी~। दशिका अन्या~। विशिखा रथ्या~। अनन्यजः कामः~। अनुष्ठवः सहायः~। कान्तार्थिनी}

\lfoot{ह० ८}

\newpage
\lfoot{}
% ८२ हर्षचरिते 

\noindent
न्धवानां च भवनेषु तावदतिष्ठत्, यावदस्य स्वयमेव गृहीतस्वभावः पृथिवीपतिः प्रसाद्वानभून्~। अविशञ्च पुनरपि नरपतिभवनम्~। स्वस्पैरेव चाहोभिः परमग्रीतेन प्रसाद जन्मनो मानस्य प्रेम्णो विस्रम्भय द्रविणस्य नर्मणः प्रभावस्य च परां कोटिमानीयत नरेन्द्रेणेति~।

\begin{center}
{\s इति श्रीबाणहते हर्षचरिते राजदर्शनं नाम द्वितीय उच्छ्वासः~।}
\end{center}

\hrule

\noindent
{\s तु या याति संकेतं साभिसारिका~। {\qt हंसस्य योषिद्वरटा}~। वेशन्तः पल्वलम्~। कासारमत्यल्पसरः~। मञ्जीरं नूपुरम्~। दक्षिणोऽनुकूलः~। कौलीनं जनापवादः~। अक्षिगतो द्वेष्यः~। विन्तम्भस्याश्वासस्य~। द्रविणस्य धनस्य~। नर्मणः परिहारस्य~॥}

\begin{center}
{\s इति श्रीशंकरकविविरचिते हर्षचरितसंकेते द्वितीय उच्छ्वासः समाप्तः~॥}
\end{center}

\vspace*{\fill}

\begin{center}
\rule{0.2\linewidth}{0.5pt}
\end{center}

\vspace*{\fill}

\newpage
\fancyhead[CO]{तृतीय उच्छ्वासः~।}
% तृतीय उच्छ्वासः~। ८३

\begin{center}
{\s तृतीय उच्छ्वसः~।}
\end{center}

\vspace{-5mm}
\begin{quote}
{\ha निजवर्षाहितस्नेहा बहुभक्तजनान्विताः~।\\
सुकाला इव जायन्ते प्रजापुण्येन भूभुजः~॥~१~॥

साधूनामुपकर्तुं लक्ष्मीं द्रष्टुं विहायसा गन्तुम्~।\\
न कुतूहलि कस्य मनचरितं च महात्मनां श्रोतुम्~॥~२~॥}
\end{quote}

\vspace{-5mm}
अथ कदाचिद्विरलितबलाहके, चातकातङ्ककारिणि, कणत्कादम्बे, दुर्दुरद्विषि, मयूरमदमुषि, हंसपथिकसार्थसर्वातिथौ, धौतासिनिभनभसि, भास्वरभास्वति, शुचिशशिनि, तरुणतारागणे, गलत्सुनासीरशरासने, सीदत्सौदामनीदानि, दामोदरनिद्राहि, द्रुतवैदूर्यवर्णार्णसि, घूर्णमानमिहिकालघुमेघमोघमघवति, निमीलन्नीपे, निष्कुसुमकुटजे, निर्मुकुलकन्दले, कोमलकमले, मधुस्यन्दीन्दीवरे, कह्लाराह्लादिनि, शेफालिकाशीतलीकृतनिशे, यूथिकामोदिनि, मोदमानकुमुदावदातदशदिशि, सप्तच्छदधूलिधूसरितसमीरे, स्तबकितबन्धुरबन्धूकाबध्यमानाकाण्डसंध्ये, नीराजितवा\textendash

\vspace{2mm}
\hrule

{\s {\qtt निजेति}~। निज आत्मीयः~। वर्षो लोकः, दृष्टिच वर्ष वर्षमपि निजं समुचितकालप्राप्तम्~। स्नेहः प्रीतिः, आर्द्रता च~। भक्ताः अनुरक्ताः, ओदनाथ~। भक्तं भक्तरूपाणां भूभृतां सुकालानां च प्रजापुण्यं हेतुः~। अनेन महानुभावपुष्पभूतिवर्णना सूचिता~॥~१~॥

{\qtt साधूनामित्यादिनापि} भैरवाचार्योपकारकरणम्, स्वयं लक्ष्मीदर्शनम्, विहायसा गमनं भैरवाचार्यस्य, महात्मचरितश्रवणकुतूहलं च निजभ्रात्रादीनां सूचितम्~॥~२~॥

अथेल्यादावेवंविधे शरत्समयारम्भे बन्धून्द्रष्टं बाणो ब्राह्मणाधिवासम गादिति संबन्धः~। विरलिताः, न पुनरेकान्ततोपगताः~। बलाहका मेघाः~। चातकाः स्तोककाख्याः पक्षिणः~। कादम्बाः कृष्णहंसाः~। दर्दुरा मण्डूकाः~। हंसा एक पथिकसार्थाः, तेषां निर्मलजलदानादिना स्यात्सर्वातिथित्वम्~। शुचिर्निर्मलः~। सुनासीर इन्द्रः~। सौदामनी विद्युत्~। दामोदरो हरिः~। अस्य निद्रां द्रोग्धि यस्तस्मिन्~। तदा किल हरिर्विबुध्यत इति वार्ता~। अर्णो जलम्~। घूर्णमाना भ्रमन्ती या मिहिका नीहाररतद्वल्लघवस्तुच्छा ये मेघास्तैर्मोघो निष्फलो मघवानिन्द्रो यत्र तस्मिन्~। वर्षाभावादिन्द्रस्य मोघत्वम्~। इन्द्रादेशेन हि मेधा वर्षन्ति~। मेघवद्गर्जितमित्यन्ये~। नीपाः कुटजाः~। कन्दलाश्च वृक्षभेदाः~। कहाराणि सौगन्धिकापरनामानि श्वेतोत्पलानि~। जलकुसुमपत्रिकेत्यन्ये~। शेफालिका पुष्पभेदः~। रात्रावेव विकसति~। यूथिका हरिणिका~। मोदमानानि विकसन्ति~। सप्तच्छदाः}

\newpage
% ८४ हर्षचरिते 

\noindent
जिनि, उद्दामदन्तिनि, दक्षीयौक्षके, क्षीयमाणपङ्कचक्रवाले, वालपुलिनपल्लवितसिन्धुरोधसि, परिणामाश्यानश्यामाके, जनितप्रियकुमञ्जरीरजसि, कठोरितत्रपुसत्वचि, कुसुमस्मेरशरे, शरत्समयारम्भे राज्ञः समीपाद्वाणो बन्धून्द्रष्टुं पुनरपि तं ब्राह्मणाधिवासमगात्~।

समुपलब्धभूपालसंमानातिशयपरितुष्टास्तस्य ज्ञातयः श्लाघमाना निर्ययुः~। क्रमेण च कश्चिदभिवादयमानः, कैश्चिदद्भिवाद्यमानः, कैञ्चिच्छिरसि चुम्व्यमानः, कांश्चिमूर्ध्नि समाजिघन्, कैविदालिङ्ग्यमानः, कांविदालिङ्गन्, अन्यैराशिषानुगृह्यमाणः, परामनुगृहन्, बहुबन्धुमध्यवर्ती परं मुमुदे~। संभ्रान्तपरिजनोप नीतं चासनमासीनेषु गुरुषु भेजे~। भजमानश्चार्चा दिसत्कारं नितरां ननन्द~। प्रीयमाणेन च मनसा सर्वांस्तान्पर्यपृच्छत् \textendash\ कश्चिदेतावतादिवसान्सुखिनो यूयम्~। अप्रत्यूहा वा सम्यकरणपरितोषितद्विजचका की क्रियते क्रिया~। यथावविकलमत्रभाचि सुखते हवींषि हुतभुजः~। यथाकालमधीयते वा बटवः~। प्रतिदिनमविच्छिन्नो वा वेदाभ्यासः~। कच्चित्स एव चिरंतनो यज्ञविद्याकर्मण्यमियोगः, तान्येव व्याकरणे परस्परस्पर्वानुबन्धावन्ध्यदिवसदर्शितादराणि व्याख्यानमण्डलानि, सैत्र वा पुरातनी परित्यक्तान्यकर्तव्या प्रमाणगोष्ठी, स एव वा मन्दीकृतेतरशास्त्ररसो मीमांसायामतिरसः~। कच्चित्त एवं वामिनवसुभाषितसुधावर्षिणः काव्यालापाः इति~।

अथ ते तमूचुः \textendash\ तात, संतोषजुषां सततसंनिहितविद्याविनो

\vspace{2mm}
\hrule

\noindent
{\s सप्तपर्णाख्या वृक्षभेदाः~। बन्धुरा हृथाः बन्धूका बन्धुजीवाख्या वृक्षभेदाः~। नीराजिताः कृतशान्तिविधानाः~। क्षीबाणीवौक्षकानि दान्तसमूहा यत्र तस्मिन्~। चक्रवालं समूहः~। वालं तत्क्षणस्रुतजालम्~। सिन्धवो नद्यः~। श्यामाको नीवारः~। प्रियहिभेदः~। त्रपुसं लाडुकम्~।

संभ्रान्तः सत्वरः~। सत्कारं पूजाम्~। कचिदितीष्टप्रश्ने~। प्रत्यूहो विघ्नः~। सम्यक्करणं यथाशास्त्रं संपादनम्~। ऋतूनां यज्ञानामियं कातवी~। {\qtt अधीयत इति}~। वेदपाठो बालानामेवोचितः~। प्रमाणं तर्कविद्या मीमांसा ब्रह्मनिदर्शनम्~। अतः एवाह \textendash\ {\qtt अतिरस इति}~।

तात इति पूजावचनम्~। वैतानः क्रातवः~। कौसीद्यमालस्यम्~। निष्प्रयन्नतेत्यर्थः~।}

\newpage
% तृतीय उच्छ्वासः~। ८५ 

\noindent
दानां वैतानवह्निमात्रसहायानां कियन्मानं नः कृत्यं सुखितया सकलभुवनभुजि भुजङ्गराजदेहदीर्घे रक्षति क्षितिं क्षितिभुजो भुजे~। सर्वथा सुखिन एव वयम्, विशेषेण तु त्वयि विमुक्तकौसीधे परमेश्वरपार्श्ववर्तिनि वेत्रासनमधितिष्ठति~। सर्वे च यथाशक्ति यथा विभवं यथाकालं च संपाद्यन्ते विप्रजनोचिताः क्रियाकलापाः इत्येवमादिभिरालापैः स्कन्धावारवार्ताभिश्च शैशवातिक्रान्तक्रीडानुस्मरणैः पूर्वजकथाभिश्च विनोदितमनास्तैः सह सुचिरमतिष्ठत्~। उत्थाय च मध्यंदिने यथाक्रियमाणाः स्थितीरकरोत्~। भुक्तवन्तं च तं सर्वे ज्ञातयः पर्यवारयन्~।

अत्रान्तरे दुगूलपट्टप्रभवे शिखण्ड्यपाङ्गपाण्डुनी पौण्ड्रे वाससी वसानः, स्नानावसानसमये वन्दितया तीर्थमृदा गोरोचनया च रचिततिलकः, तैलामलकमसृणितमौलिः, अनुच्चचूडाचुम्बिना निबिडेन कुसुमापीडकेन समुद्भासमानः, असकृदुपयुक्तताम्बूलविमलाघरकान्तिः, एकशलाकाञ्जनजनितलोचनरुचिरचिरभुक्तः, विनीतमार्यं च वेषं दधानः, पुस्तकवाचकः सुदृष्टिराजगाम~। नाति दूरवर्तिन्यां चासन्यां निषसाद~। स्थित्वा च मुहूर्तमित्र तत्कालापनीतसूत्रवेष्टनमपि नखकिरणैर्मृदुमृणालसूत्रैरिव वेष्टितं पुस्तकं पुरोनिहितशरशलाकायत्रके निधाय, पृष्ठतः सनीडसंनिविष्टाभ्यां मधुकरपारावताभ्यां दत्ते स्थानके प्राभातिकप्रपाठकच्छेदचिह्नीकृत मन्तरपत्रमुत्क्षिप्य, गृहीत्वा च कतिपयपत्रलध्वी कपाटिकाम्, क्षालयन्निव मषीमलिनान्यक्षराणि दन्तकान्तिभिः, अर्चयन्निव सित\textendash

\vspace{2mm}
\hrule

{\s अत्रेत्यादौ सुदृष्टिः पुस्तकवाचक आजगानेति संबन्धः~। {\qtt दुगुलेति}~। एक स्माद्दुगूलपटाद्दीर्घाच्छिला गृहीते, शिखण्ड्यपाङ्गपाण्डुत्वेन कार्कश्यमपि दर्शितम्~। पौण्ड्रे पुण्ड्रदेशजे ! गोरोचना रक्षाद्रव्यभेदः~। मौलयः केशाः~। अनुच्चेति~। अदीर्घतया कुसुमापीडकस्य श्रोत्रियत्वं विनीतत्वं चास्य दर्शितम्~। निविडेन संहतपुष्पेण~। रुचिरं नैर्मल्यम्~। भोजनं भुक्तमचिरं भुक्तं यस्य सः~। अनेन तस्मानवलिप्तलमुक्तम्~। आसन्धां वेत्रपीठिकायाम्~। स्थित्वेत्यादौ पुराणं पपाठेति संबन्धः~। सनीडे समीपे~। प्रपाठको वाचकः, प्रपठनं वा~। तस्य तत्र वा छेदः~। इयन्मात्रं वाचितं नान्यदिति तेन चिह्नीकृतं लक्षीकृतम्~। गमयन्ति रागस्त्ररूपमिति गमकाः~।}

\newpage
% ८६ हर्षचरिते 

\noindent
कुसुममुक्तिभिर्ग्रन्थम्, मुखसंनिहितसरस्वतीनृपुररवैरिव गमकैर्मधुरैराक्षिपन्मनांसि श्रोतृणां गीत्या पवमानप्रोक्तं पुराणं पपाठ~।

तस्मिंश्च तथा श्रुतिसुगगीतिगर्ने पठति सुदृष्टौ नातिदूरवर्ती बन्दी सूचीबाणस्तारमधुरेण गीतिध्वनिमनुवर्तमानः स्वरेणेदमार्यायुगलमपठत् \textendash\ 

\vspace{-5mm}
\begin{quote}
{\ha तदपि मुनिगीतमतिश्रु तदपि जगद्व्यापि पावनं तदपि~।\\
हर्षचरिताद्भिनं प्रतिभाति हि मे पुराणमिदम्~॥~३~॥

वंशानुगमविवादि स्फुटकरणं भरतमार्गभजनगुरु~।\\
श्रीकण्ठविनिर्यात गीतमिदं हर्षराज्यमित्र~।}
\end{quote}

\vspace{-5mm}
तच्छ्रुत्वा बाणस्य चत्वारः पितामहमुखपद्मा इव वेदाभ्यासपवित्रितमूर्तयः, उपाया इव सामप्रयोगललितमुखाः, गणपतिः, अधिपतिः, तारापतिः, श्यामल इति पितृव्यपुत्रा भ्रातरः, प्रसन्नवृत्तयः, गृहीतवाक्याः, कृतगुरुपद्न्यासाः, न्यायवादिनः, सुकृतसंग्रहाभ्यास\textendash

\vspace{2mm}
\hrule

\noindent
{\s असाधारयनि खराणां निमीलनानि यानि लक्ष्येष्वान्तरमार्ग इति प्रसिद्धास्तैर्गमकैः खरयतिविशेषैः~। पवमानो वायुः~।

बन्दी स्तुतिपाठकः~। पृथुरा दिनृपोऽपि~। पवमानं वायुप्रोतमपि~। गीतपक्षे \textendash\ वंशेन वेणुनानुगमो ययोस्तौ विवादिनी खरौ विश्रुत्यन्तरी गान्धारनिषादौ स्वरौ यत्र तत्~। करणमपदः~। सताल आविद्धः स्वरसंनिवेश उच्चारणस्थानं वा~। भारतं भरतमुनिकृतो ग्रन्थः~। श्रीकण्ठः श्रीयुक्तः कण्ठः~। वैखर्यादिदोषाभावात्~। यद्वा श्रीकण्ठो हर एव~। सर्वविद्यानां तत एवोत्पत्तेः~। हर्षराज्यमपीदृशमेव~। तथा च वंशं कुलमनुगच्छलनुसरति यत्तद्वंशानुगम्~। तथाविद्यमाना विवादिनो यत्र तदविवादि सौराज्यं~। न केचित्तत्र विवदन्ते~। करणमधिकरणं यत्र विद्यापरीक्षा, धर्मनिर्णयो वा क्रियते, व्यापारो वा~। भरतो नाम पूर्व राजाभूत्~। श्रीकण्ठो देशभेदः~। गीतमपि हर्षस्य प्रमोदस्य राज्यमिव~। तस्य विजम्भमाणत्वात्~। तच्छ्रुत्वेत्यादौ बाणस्य चत्वारो भ्रातरः परस्परस्य मुखानि व्यलिकयन्निति संबन्धः~। तच्छ्रुत्वेत्यादिनास्य प्रकरणस्य प्रकृतानुगुणत्वं दर्शितम्~। तेषां च ते प्रस्ताववेदित्वम्~। मुखपद्मा अपि चत्वारः सामवेदभेदाः~। सान्त्वं च सुखमारम्भोऽपि~। प्रसन्ना शुद्धा, सुबोद्धा च~। वृत्तिर्वर्तनम्, सूत्रविवरणं च~। गृहीतमाहतम् ज्ञानार्थं च वाक्यं विवरणम्, वार्तिकं च~। यत्कारणात्कात्यायनो वार्तिककार उच्यते~। कृतो गुरूणां संबन्धिनि पदे स्थाने न्यासः स्थितिर्येषां ते~। सर्वेणोपदेष्टृपदे स्थापितास्त इत्यर्थः~। यद्वा कृतो गुरुणि पदे न्यासो यैः~। महति पदे स्थिता इत्यर्थः~। अन्यत्र कृतोऽभ्यस्तो गुरुपदे दुर्बोधशब्दे न्यासो वृत्तिर्विवरणं यैः~। न्यायो युक्तम्,}

\newpage
% तृतीय उच्छ्वासः~। ८७ 

\noindent
गुरवः लब्धसाधुशब्दाः, लोक इव व्याकरणेऽपि सकलपुराणराजर्षिचरिताभिज्ञाः, महाभारतभावितात्मानः, विदितसकलेतिहासाः, महाविद्वांसः, महाकवयः, महापुरुषवृत्तान्तकुतूलिनः, सुभाषितश्रवणरसरसायनाः, वितृष्णाः, वयसि वचसि यशसि तपसि सदसि महसि वपुषि यजुषि च प्रथमाः, पूर्वमेव कृतसंगराः, विवक्षवः, स्मितसुधाधवलितकपोलोदराः, परस्परस्य मुखानि व्यलोकयन्~।

अथ तेषां कनीयान्कमलदलदीर्घलोचनः श्यामलो नाम बाणस्य प्रेयान्प्राणानामपि वशयिता दत्तसंज्ञस्तैः सप्रणयं दशनज्योत्स्नास्नपितककुभा मुखेन्दुना बभाषे \textendash\ तात बाण, द्विजानां राजा गुरुदारग्रहणमकार्षीत्~। पुरुरवा ब्राह्मणधनतृष्णया दयितेनायुषा व्ययुज्यत~। नहुषः परकलत्राभिलाषी महाभुजङ्ग आसीत्~।

\vspace{2mm}
\hrule

\noindent
{\s उपपत्त्यनुपपत्तिविचारश्च~। सुकृतं पुण्यम्, सुष्ठु विहितं च~। संग्रहः संचयः~। व्याकरणं व्याडिकृतो ग्रन्थश्च~। गुरवो महान्तः, उपाध्यायाथ साधुशब्दः साधुवादः, साधवोडमी इत्येवरूपो वा~। साधवः संस्कृताः, शब्दाच~। पाण्डित्यप्रकटनेनानेन द्रष्टुमिष्टस्य वस्तुन उत्कृष्टतोच्यते~। सकलेत्यादिविशेषणत्रयेण द्विजराजादिवृत्तान्तेऽभिज्ञतोच्यते~। {\qtt महापुरुषेत्यादि}~। हर्षचरिते शुश्रूषाया हेतुः~। {\qtt सुभाषितेत्यादि} खकाव्यप्रशंसासूचनपरम्~। सदति सभायाम्~। संगरं संकेतः~।

{\qtt कनीयानिति}~। अनेन प्रियवचनत्वमस्य दर्शितम्~। ब्रूहीति दत्तसंज्ञः~। तात बाणेत्यादिना पूर्वराजदोषोद्भावनद्वारेण हर्षस्य गरीयस्तां ख्यापयति~। अत्र क्वचिच्छब्दद्वारेण क्वचिचार्थद्वारेण यथायोग्यं दोष उद्भाव्यः~। चन्द्रादिशब्दाभिया नेन राजत्वप्रतीतिर्न स्यादिति द्विजानां राजेत्युक्तम्~। गुरुर्ब्रहसतिः, पित्राद्याश्च गुरवः~। अत्र कथा \textendash\ पुरा पूर्णचन्द्रमुदितं वीक्ष्य कामयमानां गुरुपत्नी ताराख्यामभिगच्छत्~। तदसहमानेन च बृहस्पतिना यदेन्द्राद्याः प्रोत्साहितास्तदानयनाय, तदा चन्द्रेण शुक्रः शरणमाश्रितः~। ततः शुक्रप्रेरितैर्दैयैः सह तेषामन्योन्यं दिव्यं वर्षसहस्रं युद्धमासीत्~। तारापि नारदबोधिता सगर्भा सती पुनर्गुरुमेवाभिगतेति~। दयितेनायुषा प्रियेण जीवितेन पुत्रेणायुर्नाम्ना~। कथा चात्र \textendash\ पुरूरवाः पूर्वां दिशे जेतुं गच्छन्केनाप्याहतप्रभुतघनेन विप्रेण यज्ञे निमन्त्रितो लोभाक्षिप्तस्तद्धनं जिहीर्षुस्तच्छापानष्टः~। तस्मिन्मृते स विप्रो नृपं विना प्रजा निवर्तत इति ज्ञात्वा तदायुषा राजर्षिमायुर्नामानमजीजनदिति~। भुजङ्गो विटोऽपि~। पुरा वृत्रं हत्वा ब्रह्महत्यया शक्रः पलाय्य मृणालच्छिद्रान्तरे यदातिष्ठतदा नहुषो यज्वा शव देवैरिन्द्रत्वं नीतो दर्पाच्छचीं प्रार्थयमानो बृहस्पत्युपदेशात्तयोको यथा \textendash\ {\qt यानेनापूर्वेणाराच्छ} इति~। ततो ब्रह्मर्षीन्चाहनीकृत्य व्रजन्कामवशात्त्वरमाणः पादेनाताड्य सर्प}

\newpage
% ८८ हर्षचरिते 

\noindent
व्ययातिराहितब्राह्मणीपाणिग्रहणः पपात~। सुद्युनः स्त्रीमय एवाभवत्~। सोमकस्य प्रख्याता जन्तुवधनिर्घृणता~। मांधाता मार्गणव्यसनेन सपुत्रपौत्रो रसातलमगात्~। पुरुकुत्सः कुत्सितं कर्म तपस्यन्नपि मेकलकन्यकायामकरोत्~। कुवलयाश्वो भुजङ्गलोकपरिमहादवतरकन्यामपि न परिजहार~। पृथुः प्रथमपुरुषक: परिभूतवान्पृथिवीम्~। नृगस्य ककलासभावे वर्णसंकरः समद्दश्यत~।

\vspace{2mm}
\hrule

\noindent
{\s सर्प इति चोदयन्नमस्त्वेन {\qt सर्पो भव} इति शप्तः सर्पोऽभवत्~। पपातेति नरकगामी बभूव, खाचारभ्रष्टत्वात्पतितश्चात् वृषपर्वणोऽसुरराजस्य दुहित्रा शर्मिष्ठया कलहायमाना {\qt अस्मद्धृत्यसुता वराकी भूला स्पर्धते} इत्युक्त्वा कूपान्तः पातितां शुक्रसुतां देवयानीं ज्ञात्वा ययातिर्वनविहारी पाणि गृहीत्वोज्जहार~। गते ययातौ परिभवोद्विग्ना वन एवावसत~। अथ नारदायथावृत्तं ज्ञात्वा वृषपर्वा शुक्रस्य प्रार्थनामक रोत्~। संदिष्टा च \textendash\ {\qt कुमारी शतपरिवारवतीयं शर्मिष्ठा यदा मे दास्यं करोति तदागच्छामि} इति~। शुकशापीतेन वृषपर्वणा संपादितमनोरथा देवयानी पुन रपि दासीभूतया शर्मिष्टया सह बने क्रीडन्ती ययातिमायान्तं दृष्टा बभाषे \textendash\ {\qt क्काद्य मां व्यक्ला पाणिग्राहो महानुभावो गतोऽभूत्} इति~। ततो ययातिर्ब्राह्मणीलादनकीकुर्वेस्खत्पित्रा शोकविधुरेण शुक्रेण पापं मास्तु, क्रियतामयं विधिः इति बुद्धा तां स्वीचके~। कालेन चासौ पपातेति~। सुद्युन्नो राजा, शोभनं नं बलमस्येति च~। स्त्रीमयो महिलाकृतिः, कान्तानुरक्तश्च~। योत्र तोयमुपयोक्ष्यति स स्त्रीलभापत्स्यत इति भगवता भवान्याभ्यर्थितेन भवेन शप्तः सन्सरसः पीत्वा तोयं सुझुन्नो मृगयाविहारी स्त्रीमयोऽभूदिति~। जन्तुर्नाम सोमकस्य राज्ञः पुत्रः, जन्तवः प्राणिनश्च~। सोमकस्य राज्ञो जन्तुर्नामैकः पुत्रोऽभूत्~। स चैकपुत्रत्वा दपुत्रलं वरमिति जाननुद्विग्नः पुरोबसाभ्यथायि \textendash\ बहून्पुत्रांवेदिच्छति तदस्य सुतस्य वपया होमः क्रियताम्~। ततो यावत्यो धूममाजिघ्रन्ति ताः पुत्रैर्युज्यन्ते इति~। स चापि घृणासपहाय तथा कारितवानिति~। मार्गणं याच्या, शराश्च मार्गणाः~। मार्गणेषु व्यसनं युद्धं व्यसनम्~। रसातलमगमद्धस्ताजगाम~। विनष्ट इत्यर्थः~। रसातलं पातालं च~। मांधाता च भुवं जिला खर्गे जेतुं गतः~। शक्रेणोक्तम् \textendash\ पातालं जिल्लागतस्य तव दास्यं यास्यामि~। स च तद्वचनादविचार्यैव रसातलं मतस्तत्र हरप्रसादासादित त्रिशूलेन लवणनाम्ना दानवेन ससुतसैन्योऽन्तमनीयत इति~। मेकलकन्यका नर्मदा~। पुरुकुत्सः पुरा तपश्चरन्नर्मदायां स्नानं कुर्वन्कामप्यङ्गनामालोक्य कामाविष्टो नीतिमुत्ससर्जेति~। भुजङ्गा विटा अपि~। अश्वतर कन्यां वडवामपि~। कुवल्याश्वो राजा मृगयाक्रीडाप्रसन धर्मादुरो मज्जनरभसेन सरसीमवतीर्णो रसातलं प्राप्तोऽश्वतराभिषां नागकन्यामूढवानिवि प्रथम आद्यः, प्रधानन्य कुत्सितः पुरुषः पुरुषकः~। पृथुरादिनृपो भूधराकान्तां सर्वां गां विलोक्य}

\newpage
%तृतीय उच्छ्वासः~। ८९ 

\noindent
सौदासेन~॥ नरक्षिता पर्याकुलीकृता क्षितिः~। नलम वशाक्षहृदयं कलिरभिभूतवान्~। संवरणो मित्रदुहितरि विक्लवतामगात्~। दशरथ इष्टरामोन्मादेन मृत्युमवाप~। कार्तवीर्यो गोब्राह्मणातिपीडनेन निधनमयासीत्~। मरुत्त इष्टबहुसुवर्णकोऽपि देवद्विजबहुमतो न

\vspace{2mm}
\hrule

\noindent
{\ चापकोट्या भुवः पर्यन्तेषु चिक्षेप~। धरणकारणभूतभूभृत्परिभवाद्भवो विभवः~। अत एवास्य कापुरुषत्वम्~। विष्णुपुराणे तु \textendash\ आकृष्टकार्मुकेन पृथुना {\qt देहि मे भर्तव्यभरणोपायम्} इत्यनुबध्यमाना भूर्भुवनानि बभ्राम~। ततः शरणमलब्ध्वा सास्य सर्वाः सस्यसंपदोऽजनयदिति वर्णितम्~। एतस्मात्परिभूताभूदिति~। कृकलासः प्राणिभेदः~। तद्भावेऽपि तस्यां दशायामपि किं पुना राज्यस्येति निन्द्यत्वम्~। वर्णः शुक्लादिः, ब्राह्मणादिश्च~। नृगो राजा दानप्रस्तावे कस्यचिद्विप्रस्य संबन्धिनी गामविज्ञायैव द्विजाय ददौ~। कदाचिच्च तस्या गोः स्वामी तां गां परिज्ञाय तं ययाचे~। न च तस्माद्गां लेभे~। ततस्तौ द्वावपि राजद्वारं राजविज्ञापनाय गतौ~। ग्राम्यभोगासक्तराजदर्शनमलभमानौ च क्रोधात् {\qt कुकलासो भव} इति राज्ञः शापं दत्त्वा कस्मैचिह्नां वितीर्थ यथागतं प्रतिजग्मतुरिति~। नरान्क्षिणोतीति नरक्षिता, न पालिता च~। सौदासो नाम राजा मृगयास्त्रिन्नः पथि गच्छन्कदाचिन्मुनिं शक्रनामानं मार्गमध्ये स्थितम् {\qt अपसर्प} इत्यवदत्~। {\qt पन्था देयो ब्राह्मणाय} इति वचनान्यायमनुवर्तमानो यावन चल्तिस्तावद्वाज्ञा कशयाभिहतः~। अथ रोषावेशात् {\qt गच्छ मनुष्यभक्ष्यो राक्षसो भव} इति तं शशाप~। वशमायत्तमक्षहृदयमक्षज्ञानम्, अक्षाणीन्द्रियाणि हृदयं च~। तच्च नलो राजा द्यूतव्यसनी तत्स्वरूपानभिज्ञश्च कलिनाभिभूत इति प्रसिद्धन्~। मित्रो रविः, सुहृच मित्रम्~। तपती नाम मित्रस्य रवेर्दुहिताभूत्~। तस्यां संवरणो नाम राजा व्यसनी बभूव~। रामो दशरथसुतः, राना स्त्री च~। दशरथो मृगयासक्तो घटपूरणरवं श्रुत्वा बृंहितशङ्कया शब्दपातिना शरेण मुनिपुत्रं व्यापादयत्~। तेन च बोधितान्वयः पित्रोः समीपं तं निनाय~। तद्वचनाच्छत्यमुद्धरति नृपे शिशुर्मृतः~। अथ च {\qt सदारेण वृद्धतापसेन पुत्रादहमिव त्वमपि प्राप्स्यस्यन्तम्} इति शप्तो रामवियोगात्प्राणांस्तयाजेति~। गोनिमित्तं ब्राह्मणस्य जमदग्नेरतिपीडनम्~। निधनमया सीत्~। जामदम्येन हत इत्यर्थः~। कार्तवीर्यो गवां कोटेरप्यधिकतरां धेनुमपहरञ्जमदग्निं व्यापादितवान्~। अथ च तत्सुतेन रामेण क्रोधात्परशुच्छिन्नबाहुसहखोऽसौ सर्वक्षत्रियैः सह मृत्युं लेभे~। इष्टः कृतः, अभिमतश्च~। देवद्विजो बृह सतिः, अन्यत्र देवाच द्विजावेति द्वन्द्वः~। मरुत्तो नाम राजा बहुसुवर्णकाख्येन ऋतुनापि यक्ष्यमाणो देवपुरोवसम् {\qt मां याजय} इति याचमानस्तेन {\qt मनुष्योऽय मेव दृष्टः} इति~। स चोपहसति घिषणे नारदेनोक्तो यथा \textendash\ {\qt गच्छ~। अस्यैव भ्राता संवर्तको नाम ग्रहगृहीत छद्मना वाराणस्यां स्थितः~। तं प्रार्थयख} इत्युक्वा च नारदोऽनिं विवेश~। स च नारदोतचिदैस्तं भगवत्प्रणामं कृत्वा निर्यान्तं}

\newpage
% ९० हर्षचरिते

\noindent
बभूव~। शंतनुरतिव्यसनादेकाकी वियुक्तो वाहिन्या विपिने विललाप~। पाण्डुर्वनमध्यगतो मत्स्य इव मदनरसाविष्टः प्राणान्मुमोच~। युधिष्ठिरो गुरुभयविपण्णहृदयः समरशिरसि सत्यमुत्सृष्टवान्~। इत्थं नास्ति राजत्वपकलङ्कमृते देवदेवादमुतः सर्वद्वीपभुजो हर्षात्~। अस्य हि बहून्याश्चर्याणि श्रूयन्ते~। तथा हि \textendash\ अत्र बलजिता निश्चलीकृताञ्चलन्तः कृतपक्षाः क्षितिभृतः~। अत्र प्रजापतिना शेषभोगिमण्डलस्योपरि क्षमा कृता~। अत्र पुरुषोत्त\textendash

\vspace{2mm}
\hrule

\noindent
{\s परिज्ञाय बहुशो गालीर्ददतमप्यनुद्विजमानो याजनाथ प्रार्थयामास~। संवर्तकेन कथितं च \textendash\ {\qt नेदं तवोतम्~। यावतं वक्ष्यामि~। देवेभ्यश्च श्रुत्वा यज्ञभागो न दातव्यः} इति~। राजा यथोक्तमनुतिष्ठस्तेन याजितो देवद्विजस्य नाभिमतोऽभवदिति~। अतिव्यसनादत्यन्तसङ्गात्~। वाहिनी नदी, सेना च~। महाभिषः पुरा ब्रह्मसदसि गङ्गायात्राममा हिण्याचलितवाससोऽजदर्शनहृतहृदयः शृङ्गारपदानि वदन्ब्रह्मणा शप्तः~। पतित्वा क्षत्रियगृहे शन्तनुर्नामाभूत्~। गङ्गापि {\qt मत्कृतेऽयमिमां दशां प्राप्तः} इति मत्वा सखेदमबतरन्ती धेनुहरणकुपितवत्तिष्ठशापसंपन्नमनुष्यलोकावतरणदुःखितैर्वसुभिर्विदितवृत्तान्तैरभ्यधायि \textendash\ {\qt तत्र नृपे चेत्तव प्रीतिः, तद्वयं त्वय्येवोत्पत्स्यामहे~। जातमात्राश्च वयं त्वया स्खजले क्षेप्तव्याः} इति~। सा तु तथेत्यङ्गीकृत्य वने विहरन्तं प्रार्थयमानं शन्तनुमवोचत्~। {\qt यदहं करोमि तत्र त्वया निर्बंन्धो न विधेयः~। न चाहं त्वया जन्म प्रष्टव्या} इति~। तथेति तेनाङ्गीकृतवता बहुतरं कालमरंस्त~। अथ यः कश्चित्सूनुरुदपादि सर्वस्तया स्वजले क्षिप्तः~। एवं सप्तस्वतीतेषु गङ्गामासेव्य निःसंतानोऽयं मा भूदिति मन्वानैः सप्तभिरेव वसुभिः कृतात्मसंनिधिभष्मो जातः~। ततस्तमपि जले क्षिपन्ती शन्तनुना निषिद्धा~। {\qt सापराधो भवान्} इत्युक्त्वा सा प्रतिजगाम~। ततस्तद्वियोगविधुरधीर्बहु विललापेति व्यसननिमित्तकः सेनया वियोगेन च विलापो विजिगीषोरनुचित एव~। वनं तोयम्, विपिनं च~। मदनः कामः फलविशेषश्च मदनम्~। पाण्डुर्वने मृगरूपया ब्राह्मण्या सह सुरतकर्मसक्तं मृगरूपं कर्दमाख्यं मुनिं शरेण जघान~। तेन च त्रियमाणेन {\qt त्रीसंभोगस्थो मरिष्यति} इति शप्तो माद्या सह स्मरार्तः क्रीडन्विपन्न इति~। गुरोर्द्रोणाचार्यस्य भयेन गुरुणा, महता च त्रासेन~। युधिष्ठिरो बलानि दग्धुमुद्यतं द्रोणाचार्य रणमूर्ध्नि {\qt अश्वत्थामा हतः} इत्युक्त्वा पुत्रशोकाकुलम सत्येनासूनत्याजयदिति~। {\qtt इत्थमिति}~। इत्थं कृतयुगादारभ्य कलिप्रारम्भपर्यन्तं राज्ञां नास्त्यपकलङ्क राजत्वमिति~। बलजित्प्रजापतिमुखाः शब्दा राज्ञि यथार्था वेदितव्याः~। बलं सैन्यम्, बलाख्यश्चासुरः~। निश्चलीकृता इति सहायाभावाच्छत्रुषु यानं न विदधिर इति~। अन्यत्र स्थावरत्वं लम्बिताः~। पक्षः सहायाः, पतत्राणि च~। क्षितिमृतो राजानः, गिरयश्च~। प्रजापतिना राज्ञा,}

\newpage
% तृतीय उच्छ्वासः~। ९१ 

\noindent
मेन सिन्धुराजं प्रमथ्य लक्ष्मीरात्मीकृता~। अत्र बलिना मोचितभूभृद्वेष्टनो मुक्तो महानागः~। अत्र देवेनाभिषिक्तः कुमारः~। अत्र स्वामिनैकप्रहारपातितारातिना प्रख्यापिता शक्तिः~। अत्र नरसिंहेन स्वहस्तविशसितारातिना प्रकटीकुतो विक्रमः~। अत्र परमेश्वरेण तुषारशैलभुवो दुर्गाया गृहीतः करः~। अत्र लोकनाथेन दिशां मुखेषु परिकल्पिता लोकपालाः सकलभुवनकोशश्चाप्रयजन्मनां विभक्त इति~। एवमादयः प्रथमकृतयुगस्येच दृश्यन्ते महासमारम्भाः~। अतोऽस्य सुगृहीतनान्नः पुण्यराशेः पूर्वपुरुषवंशा नुक्रमेणादितः प्रभृति चरितमिच्छामः श्रोतुम्~। सुमहान्कालो नः शुश्रूषमाणानाम्~। अयस्कान्तमणय इव लोहानि नीरसनिष्ठुराणि क्षुल्लकानामव्याकर्षन्ति मनांसि महतां गुणाः, किमुत स्वभावसरसमृदूनीतरेषाम्~। कस्य न द्वितीयमहाभारते भवेदस्य चरिते कुतूहलम्~। आचष्टां भवान्~। भवतु भार्गवोऽयं वंशः

\vspace{2mm}
\hrule

\noindent
{\s ब्रह्मणा च~। शेषस्यावशिष्टस्य भोगिमण्डलस्य राजसमूहस्योपरि विषये क्षान्तिः कृता~। अन्यत्र शेषाख्यस्य भोगिनो नागस्य मण्डलमाभोगस्तत्पृष्ठे भूमिर्निहिता~। पुरुषोत्तमो नरोत्कृष्टो राजा, हरिश्च~। सिन्धुराजो सिन्धुदेशाधिपतिः, क्षीरोदधिश्च~। लक्ष्मीश्छत्रचामरादिरूपा, देवताकृतिश्च~। बलिना बलवता, असुरेश्वरेण च~। भूमुद्राजा श्रीकुमाराख्यः~। श्रीकुमारो नाम राजा किल दर्पशातेनापजातमदेन हस्तिना वेष्टितः~। ततः श्रीहर्षेणाकृष्य खड्गं तस्मान्मोचितोऽसौ दन्ती च रोषाद्वने परित्यक्त इति वार्ता~। भूमृच्च पर्वतो मन्दराख्यः~। महानागो दर्पशातः, वासुकिश्च~। मोचितभूनद्वेष्टनोऽमृतमन्थनार्थे~। मन्थनायें कुमारः कुमारगुप्ताख्यः कुमारो वा यो दर्पशातान्मोचितः~। कुमारो गुहः, पुत्रश्च कुमारः~। स्वामी प्रभुः, कुमारश्च~। अरातयः शत्रवः, तारकञ्चासुराधिपतिः~। शक्तिः सामर्थ्यम्, आयुधभेदश्व~। नरसिंहः उत्तमनरः, नृसिंहरूपो हरिश्च~। {\qtt स्वहस्तेनेति}~। न तु साधनबलेन~। अन्यत्र तु चक्रादिनिजायुधेन परमेश्वरेण सार्वभौमेन~। न तु मण्डलमात्रस्य भोका हरेण~। दुर्गाया दुर्गमायाः, गौर्याश्च~। करो दण्डः, पाणिश्च~। लोकनाथो राजा, हरिः, बुद्धश्च~। दिशां मुखेषु सीमासु~। लोकनाथाः (लोकपालाः) सीमापतयः, इन्द्राद्या दिक्पालाश्च~। कोशो गञ्जम्, मध्यम्, ग्रन्थभेदश्च~। अम्यजन्मानो द्विजाः, आदिनृपाः, श्रमणाश्च~। {\qtt एवमादय इति}~। न त्वेतावन्तः एव~। {\qtt प्रथमकृतयुगस्येवेति}~। पर्वतपक्षशातनादयो वृत्तान्ता अभवन्~। {\qtt मणय इत्रेति}~। मणिशब्देनोपमेयानां गुणानां रत्नत्वमुक्तम्~। लोहान्यपि नीरसनिष्ठु\textendash}

\newpage
% ९२ हर्षचरिते 

\noindent
शुचिनानेन राजर्षिचरितश्रवणेन सुतरां शुचितरः इत्येवमभिवाय तूष्णीमभूत्~।

बाणस्तु विहस्याब्रवीत् \textendash\ आर्य, न युक्त्यनुरूपमभिहितम्~। अघटमानमनोरथमिव भवतां कुतूहलमवकल्पयामि~। शक्याशक्यपरिसंख्यानशून्याः प्रायेण स्वार्थतृषः~। परगुणानुरागिणी प्रियजनकथा श्रवणरसरभसमोहिता च मन्ये महतामपि मतिरपहरति प्रविवेकम्~। पश्यत्वार्यः क्व परमाणुपरिमाणं बटुहृदयम्, क्व समस्तब्रह्मास्तम्भव्यापि देवस्य चरितम्~। क्व परिमितवर्णवृत्तयः कतिपये शब्दाः, च संख्यातिगास्तद्गुणाः~। सर्वज्ञस्याप्ययमविषयः, वाचस्पतेरप्यगोचरः, सरखत्या अप्यतिभारः, किमुतास्मद्विधस्य~। कः खलु पुरुषायुषशतेनापि शक्नुयादविकलमस्य चरितं वर्णयितुम्~। एकदेशे तु यदि कुतूहलं वः, सज्जा वयम्~। इयमधिगतकतिपयाक्षलवलघीयसी जिह्वा क्वोपयोगं गमिष्यति~। भवन्तः श्रोतारः~। वर्ण्यते हर्षचरितम्~। किमन्यत्~। अद्य तु परिणतप्रायो दिवसः~। पञ्चालम्बमानकषिलकिरणजटाभारभास्वरो भग वान्भार्गवो राम इव समन्तपञ्चकरुधिरमहाह्रदे निमज्जति संध्यारागपटले पूषा~। श्वो निवेद्वित्तास्मि इति~। सर्वे च ते {\qt तथा} इति प्रत्युपद्यन्त~। नातिचिरादुत्थाय संध्यामुपासितुं शोणमयासीत्~।

अथ मधुमपल्लवितमालवीकपोलकोमलातपे मुकुलितेऽह्नि कमलिनीमीलनादिव लोहिततमे तमोलिहि रवौ लम्बमाने, रविरथ\textendash

\vspace{2mm}
\hrule

\noindent
{\s राणि~। क्षुल्लकाः खलाः~। वाला इत्यन्ये~। आचष्टामाख्यातु~। भार्गव इति नगु गोत्रत्वम्~।

अवकल्पयामि निश्चिनोमि~। शक्तमिदमित्येवंरूपेण परिसंख्यानेन गणनया स्वार्थतृषो गृध्नवः शून्याः~। शक्याशक्यविवेकं गृध्नवो न जानन्तीत्यर्थः~। बटुर्द्विजशिशुः~। ब्रह्मस्तम्भं जगत्~। पुरुषायुषेत्यादिना योग्येऽपि मयि वर्णवितरि वर्णनीयस्य भूयस्त्वम्, अल्पीयस्त्वाच्चायुषः सामस्त्येन वर्णनं न घटत इति प्रतिपादितम्~। अत एवाह \textendash\ {\qtt एकदेश इति}~। संज्ञा (सज्जा) वर्णनाभिमुखा इति भवन्त इति~। न तु यादृशतादृशाः~। हर्षचरितमिति~। न तु यदेव किंचित्~। समन्तपश्चकं कुरुक्षेत्रम्~। तथेत्येवमस्त्विति~। प्रत्यपद्यन्ताङ्गीकृतवन्तः~।

अथेत्यादावस्मिन्नस्मिन्सति बाणत्वयैव गोष्ठ्या तस्थाविति संबन्धः~। कपोलकोमलो गण्डसदृशः~। मुकुलिते प्राप्तसंकोचे कुटीरं जरगृहम्~। पटलं छोद\textendash}

\newpage
% तृतीय उच्छ्वासः~। ९३

\noindent
तुरगमार्गानुसारेण यममहिष इव धावति नभसि तमसि, क्रमेण च गृहतापसकुटीरकपटलावलम्बिषु रक्तातपच्छेदैः सह संहृतेषु चल्कलेषु, कलिकल्मषमुषि पुष्णति गगनमग्निहोलधामधूमे, सनियमे यजमानजने मौनव्रतिनि, विहारवेलाविलोले पर्यटति पत्नीजने, विकीर्यमाणहरितश्यामाकशालि पूलिकासु दुग्धासु होमकपिलासु हूयमाने वैतानतनूनपाति, पूतविष्टरोपविष्टे कृष्णाजिनजटिले जटिनि, जपति बटुजने, ब्रह्मासनाध्यासिनि ध्यायति योगिगणे, तालध्वनिधावमानानन्तान्तवासिनि अलसवृद्धश्रोत्वियानुमतेन गलग्रन्थदण्डकोद्गारिणि संध्यां समवधारयति वठरविटबटुसमाजे, समुन्मज्जति च ज्योतिषि तारकाख्ये से प्राप्ते प्रदोषारम्भे भवनमागत्योपविष्टः स्निग्धैर्बन्धुभित्र सार्धं तथैव गोष्टया तस्थौ~। नीतप्रथमयामच गणपतेर्भवने परिकल्पितं शयनीयमसेवत~। इतरेषां तु सर्वेषां निमीलितदृशामप्यनुपजातनिद्राणां कमलवनानामिव सूर्योदयं प्रतिपालयतां कुतूहलेन कथमपि सा क्षपा क्षयमगच्छत्~।

अथ यामिन्यास्तुर्ये यामे प्रतिबुद्धः स एव बन्दी लोकद्वयम गायत् \textendash\ 

\vspace{-5mm}
\begin{quote}
{\ha पश्चादति प्रसार्य बिकनतिविततं द्राघयित्वाङ्गमुच्चै\\
रासज्याभुन्नकण्ठो मुखमुरसि सटा धूलिधूम्रा विध्य~।\\
धासत्रासाभिलाषाद्नवरत चलतप्रोतुन्दस्तुरङ्गो\\
मन्दं शब्दायमानो विलिखति शयनादुत्थितः क्ष्मां खुरेण~॥~५~॥

कुर्वन्नासुग्नष्पृष्ठो मुखनिकटकटि: कंधरामातिरचीं}
\end{quote}

\vspace{2mm}
\hrule

\noindent
नम्~। विहारो वह्निसंधुक्षणमग्निहोत्रार्थम्~। पूतिको वरण्डः, परिमाणभेदः~। तनूनपाद्वह्निः~। विष्टरमासनम्~। तालव्वनिरडुलिजः शब्दः~। अन्तेवासिनः शिष्याः~। श्रोत्रियो वेदोपाध्यायः~। तदनुमतेन संध्यां स संधारयति~। वदनव्यमल्लाद्गलतो विस्मरतो ग्रन्थदण्डका ऋग्गणा उद्भिरति यस्तस्मिन्~। वठरा मूर्खाः~। विटा भुजङ्गप्रायाः~। बटवो बालाश्च~। गृहश्रोत्रियैर्बालाः संध्यावन्दनाय प्रवर्त्यन्ते निर्विवेकत्वात्~।

तुर्यश्चतुर्थः~। त्रिकं पृष्ठकटीसंधिः~। द्राघयित्वा दीर्घतरीकृत्वा~। आभुग्नो नमितः कण्ठो यस्य तत्~। मुखमुरयासज्यं कृत्वा~। धूम्रा धूसराः~। प्रतानस्यो पारे प्रोथः प्रतानमुत्तरोष्टमध्यम्~। {\qt वक्वास्ये वदनं तुण्डमाननं लपनं मुखम्}~।

\lfoot{ह० ९}

\newpage
\lfoot{}
% ९४ हर्षचरिते 

\begin{quote}
{\ha लोलेनाहन्यमानं तुहिनकणमुचा चञ्चता केसरेण~।\\
निद्राकण्डूकपायं कपति निविडितश्रोवशुक्तिस्तुरङ्ग\textendash \\
स्त्वङ्गत्पक्ष्मामलमप्रतनुबुसकणं कोणमक्ष्णः खुरेण~॥~६~॥}
\end{quote}

\vspace{-5mm}
\noindent
बाणस्तु तच्छ्रुत्वा समुत्सृज्य निद्रामुत्थाय प्रक्षाल्य वदनमुपास्य भगवतीं संध्यामुदिते भगवति सवितरि गृहीतताम्बूलस्तत्रैवातिष्ठत्~। अत्रान्तरे सर्वेऽस्य ज्ञातयः समाजग्मुः, परिवार्य चासांचक्रुः~। असावपि पूर्वोद्धातेन विदिताभिप्रायस्तेषां पुरो हर्षचरितं कथयितुमारेभे \textendash\ 

श्रूयताम् \textendash\ अस्ति पुण्यकृतामधिवासो वासवावास इव वसुधामवतीर्णः, सततमसंकीर्णवर्णव्यवहारस्थितिः कृतयुगव्यवस्थः, स्थलकमलबहलतया पोत्रोन्मूल्यमानमृणालैरुद्गीतमेदिनीसारगुणैरिव कृतमधुकरकोलाहलैहलैरुल्लिख्यमानक्षेतः, क्षीरोद्पयःपायिपयोदसिक्ताभिरिव पुण्ड्रेक्षुवाटसंततिभिर्निरन्तरः, प्रतिदिशमपूर्वपर्वतकैरिव खलधानधामभिर्विभज्यमानैः सस्यकूटैः संकटसीमान्तः, समन्तादुद्धातघटीसिच्यमानैजरकजूटैर्जटिलितभूमिः, उर्वरावरीयोभिः शालीयैरलंकृतः, पाकविशरारुराजमाषनिकरकिर्मीरितैश्च स्फुटितमुद्गफलकोशीकपिशितैर्गोधूमधामभिः स्थलीपृष्ठैरधिष्ठितः, महिषपृष्टप्रतिष्ठितगायगोपालपालितैश्च कीटपटललम्पटचटकानुसृतैरवटुघटितघण्टाघटीरटितरमणीयैरटद्भिरटवीं हरवृषभपीतमामयाशङ्कया बहुविभक्तं क्षीरोदमिव क्षीरं क्षरद्भिर्वाष्प\textendash

\vspace{2mm}
\hrule

\noindent
{\s तुहिनमवश्यायः~। केसराणि ललाटतटस्थाः केशाः, अश्वकृकाटिकालम्बनः केशपाशो वा~। कषायमापिङ्गलम्~। त्वादुच्यम्~। कोणं प्रान्तम्~। उद्धातः कथाप्रस्तावः~।

अस्तीत्यादौ श्रीकण्ठनामा जनपदोऽस्तीति संबन्धः~। पुण्यकृतो देवा अपि~। अधिवासो वसतिः~। वासवाचासः स्वर्गः~। पोत्रं हलमुखम्~। सारा उत्कृष्टाः~। अतिमाधुर्यात्क्षीरोदेत्यात्प्रेक्षा~। निरन्तरो निर्विवरः~। तदैव कल्पितत्वादपूर्वत्वम्~। खलवानधामभिः खलपालैः~। उद्धातोऽरघट्टः~। जीरकोऽजाजी~। जूटः समूहः~। उर्वरा सर्वसस्याढ्या भूः~। वरीयोभिरुरुतरैः~। शालीयैः शालीक्षेत्रैः~। युगपत्याकसंभवाद्विशरारुत्वम्~। किर्मीरैः शबलैः~। कोशी शिम्बिका~। गोधनस्य क्षतपृष्ठ त्वात्कीटसंभवः~। अवटुर्भावा~। घण्टैव घटी~। आमयोऽजीर्णम्~। हरवृषभेन पीतं संतमजीर्ण संभावनया बहुधा विभक्तम्~। बाष्पच्छेद्येति सौकुमार्यकथनपरम्~।}

\newpage
% तृतीय उच्छ्वासः~। ९५ 

\noindent
च्छेद्यतृणतृतैर्गोधनैर्धवलितविपिन:, विविधमखहोमधूमान्धशतमन्युमुक्कैर्लोचनैरिव सहस्रसंख्यैः कृष्णशारैः शारीकृतोद्देशः, धवलधूलीमुचां केतकीवनानां रजोभिः पाण्डुरीकृतैः प्रमथोडूलनधूसरैः शिवपुरस्येव प्रवेशैः प्रदेशैरुपशोभितः, शाककन्दलश्यामलितग्रामोपकण्ठकाश्यपीपृष्ठः, पदे पदे करभपालीभि: पीलुपल्लवप्रस्फोटितैः करपुटपीडितमातुलुङ्गीदलरसोपलिप्तैः स्वेच्छाविचितकुङ्कुमकेसरकृतपुष्पप्रकरैः प्रत्यप्रफलरसपानसुखसुप्तपथिकैर्वनदेवतादीयमानामृतरसप्रपागृह रिव द्राक्षामण्डपैः स्फुरत्फलानां च बीजलग्नशुकचञ्चुरागाणामिव समारूढकपिकुलकपोलसंदिह्यमानकुसुमानां दाडिमीनां वनर्विलोभनीयोपनिर्गमः, वनपालपीयमाननारिकेलरसासवैञ्च पथिकलोकलुप्यमानपिण्डखर्जूरैर्गोलाङ्गूललिह्यमानमधुरामोदपिण्डीरसैश्चकोरचचुजर्जरितारुकैरुपवतैरभिरामः, तुङ्गार्जुनपालीपरिवृतैश्चगोकुलावतारकलुषितकूलकीलालैरध्वग शतशरण्यैररण्यधराबन्धैरवन्ध्यवनरन्ध्रः, करभीयकुमारकपाल्यमानेरौष्ट्रकैरौरभ्रकैश्च कृतसंबाधः, दिशि दिशि रविरथतुरगविलोभनायेव विलोठनमृदितकुङ्कुमस्थलीरससमालब्धानामुत्प्रोथपुटैरुन्मुखैरुदरशायिकिशोरकजवजननाय प्रभञ्जनमिव चापिवन्तीनां वातहरिणीनामिव स्वच्छन्दचारिणीनां वडवानां वृन्दैर्विचरद्भिराचितः, अनवरतक्रतुधूमान्धकारप्रवृत्तैहैसयूयैरिव बाणैर्धवलितभुवनः, संगीतगतमुरजरवमत्तैर्मयूरैरिव विभवैर्मुखरितजीवलोकः, शशिकरावदातवृत्तैर्मुक्ताफलैरिव गुणिभिः प्रसाधितः, पथिकशतविलुप्यमानस्फीतफलैर्महातरुभिरिव सर्वातिथिभिरभि\textendash

\vspace{2mm}
\hrule

\noindent
{\s विपिनं गहनम्~। मुक्तैः पतितैः~। लोचनान्यपि कृष्णशाराणि सहस्रसंख्यानि च~। कृष्णशारा मृगभेदाश्च~। प्रमथा गणाः~। प्रवेशैर्मार्गैः~। काश्यपी भूः~। करभपालीभिः~। इत्थंभूतलक्षणे तृतीया~। करभो बालोष्ट्रः~। पीलुर्वृक्षभेदः~। प्रस्फोटितैनीराजनीकृतैः~। प्रपा पानीयशालिका~। उपनिर्गमनानि निर्गमनमार्गाः~। उद्यानानीति केचित्~। अर्जुनाः ककुभवृक्षाः~। कीलालं तोयम्~। धराबन्धास्तडाकानि~। करभेभ्यो हिताः करभीयाः~। औष्ट्रकैरुष्ट्रसमूहैः~। कृतसंबाध आवृतः~। किशोरका वत्साः~। प्रभजनं वातम्~। वडवा अश्वाः~। धूमान्धकारप्रवृत्तवणैर्वधितभुवन इति विरोधच्छाया~। हंसानामप्यन्धकारप्रवृत्तत्वं तमसि प्रचारात्~। वृत्तैराविर्भूतैः~। हंसपक्षे \textendash\ पलायितैः~। वृत्तं चरितम्, परिवर्तुलं च~। गुणिभिः शौर्यादिगुणयुक्तः,}

\newpage
% ९६ हर्षचरिते 

\noindent
गमनीयः, मृगमदपरिमलवाहिमृगरोमाच्छादितैर्हिमवत्पादैरिव महत्तरैः स्थिरीकृतः, प्रोद्ण्डसहस्रपत्रोपदिष्टद्विजोत्तमैर्नारायणनाभिमण्डलैरिव तोयाशयैर्मण्डितः, मथितपयःप्रवाहप्रक्षालितक्षितिभिः क्षीरोदमथनारम्मैरिव य् पूरिताशः श्रीकण्ठो नाम लनपबूं ल इवाक्षीयन्त कुदृष्टयः~। नादृश्यन्त दुरितानि~। छिद्यमानयूपदारुपरशुपाटित इव व्यदीर्यतार्मः~। मखशिखिधूमजलधरधाराधौत इव ननाश वर्णसंकरः~। दीयमानानेकगोसहस्रशृङ्गखण्ड्यमान इवापल्ायत कलिः~। सुराल्यशिल्माघट्टनदङ्कनिकरनि् स्कैमु्दै~। यस्य द्ते संज्ञा~। तथा चहवं मृगरोमजम्~। अन्यत्र मृगाणां रोमाणि~। पादाः प्रत्यन्तपर्वताः~। महत्तरर्वृदैः विपुलेश्च~। सहस्रपत्राणि पञ्मानि~। द्विजोत्तमाः पक्षिश्रेष्टाः, ब्राह्मणाश्च द्विजोत्तभाः~। मथितं तक्रम्, विलोडितं च~। पयः क्षीरम्~। उभयत्रापि नयनन्याद्क्षीरेदस्य~। घोषो गोष्ठः, शब्द्~। आशा आशंसा, नयर ह्

content missing

\newpage
% तृतीय उच्छ्वासः~। ९७

तत्र चैवंविधे नानारामाभिरामकुसुमगन्धपरिमलसुभगो यौवनारम्भ इव भुवनस्य, कुङ्कुममलनपिञ्जरितबहुमहिषीसहस्रशोभितोऽन्तः पुरनिवेश इव धर्मस्य, मरुदुद्धूयमानचमरीबालव्यजनघवलितप्रान्त एकदेश इव सुरराज्यस्य, ज्वलन्मखशिखिसहस्रदीप्यमानदशदिगन्तः शिबिरसंनिवेश इव कृतयुगस्य, पद्मासनस्थितब्रह्मर्षिध्यानाधीयमानसकलाकुशलप्रशमः प्रथमोऽवतार इव ब्रह्मलोकस्य, कलकलमुखरमहावाहिनीशतसंकुलो विपक्ष इवोत्तरकुरूणाम, ईश्वरमार्गणसंतापानभिज्ञसकलजनो विजिगीषुरिव त्रिपुरस्य, सुधारससिक्तधवलगृहपङ्क्तिपाण्डुरः प्रतिनिधिरिव चन्द्रलोकस्य, मधुमन्तमत्तकाशिनीभूषणरवभरितभुवनो नामाभिहार इव कुबेरनगरस्य, स्थाण्वीश्वराख्यो जनपदविशेषः~।

यस्तपोवनमिति मुनिभिः, कामायतनमिति वेश्याभिः, संगीतशालेति लासकैः; यमनगरमिति शत्रुभिः, चिन्तामणिभूमिरित्यर्थिभिः, वीरक्षेत्रमिति शस्त्रोपजीविभिः, गुरुकुलमिति विद्यार्थिभिः, गन्धर्वनगरमिति गायनैः, विश्वकर्ममन्दिरमिति विज्ञानिभिः, लाभभूमिरिति वैदेहकै:, द्यूतस्थानमिति बन्दिभिः, साधुसमागम इति सद्भिः, वज्रपञ्जरमिति शरणागतैः, विटगोष्ठीति विदग्धैः, सुकृतपरिणाम इति पथिकैः, असुरविवरमिति वातिकैः, शाक्या\textendash

\vspace{2mm}
\hrule

{\s तत्र चेत्यादौ स्थाण्वीश्वराख्यो जनपदविशेष इति संबन्धः~। आरामा उपवनानि, रामाश्च भार्याः~। गन्धस्य परिमलस्याभोगोऽनुभवः, संस्कारः~। मलनं निवर्तनम्, समालम्भनं च~। महिषी मुख्या जायापि~। मरुतो वाताः, देवाश्च~। शिबिरसंनिवेशः कटकबन्धः~। कृतं प्रतिसमाहितं युगं द्वयं स्वपक्षपरपक्षरूपं येन स राजोच्यते~। कृतं युगं वाद्यो युगभेदः~। पद्मासनमासनभेदः, पद्ममेवासनं च~। ब्रह्मर्षय उत्तमद्विजाः~। ब्रह्मा चासावृषिश्च~। यद्वा पद्मासनस्थितो ब्रह्मा च ऋषयश्चेति द्वन्द्वः~। वाहिन्यो नद्यः, सेना च~। विपक्षो बलम्~। मेरुसमीपवासिनो जना उत्तरकुरवः~। ईश्वरमार्गणो राजदण्डसाधनयाच्ञा , हरशरश्चेश्वरमार्गणः~। संतापानभिज्ञेति~। ईश्वरशरेण हि सस्त्रीकं त्रिपुरं दग्धम्~। योधजनास्ते हि युद्धे देवैर्हता इत्याहुः~। जेतात्र विजिगीषुः~। {\qt मुधा मक्कोलामृतयोः}~। मत्तकाशिनी मुख्या स्त्री, यक्षिणी च~। नामाभिहारः पर्यायान्तरम्~।

लासकैर्नटैः~। वैदेहकैर्वणिग्भिः~। द्यूतस्थानमिति साधुभागो दीयते तत्र~। बन्दिभ्योऽभिवाञ्छितसंपत्तेः सुकृतपरिणामता~। वातिकैर्विवरव्यसनिभिराचार्यैः~। शाक्यो बौद्धः~। चारणैः कुशीलवैः~। वसुधारा धनप्रवाहः~।}

\newpage
% ९८ हर्षचरिते

\noindent
श्रम इति शमिभिः, अप्सरः पुरमिति कामिभिः, महोत्सवसमाज इति चारणैः, वसुधारेति विप्रैरगृह्यत~।

यत्र च मातङ्गगामिन्यः शीलवत्यश्च, गौर्यो विभवरताश्च, श्यामाः पद्मरागिण्यश्च, धवलद्विजशुचिवदना मदिरामोदिश्वसनाश्च , चन्द्रकान्तवपुषः शिरीषकोमलाङ्ग्यश्च, अभुजङ्गगम्याः कञ्चुकिन्यश्च, पृथुकलत्रश्रियो दरिद्रमध्यकलिताश्च, लावण्यवत्यो मधुरभाषिण्यश्च, अप्रमत्ताः प्रसन्नोज्ज्वलरागाश्च, अकौतुका: प्रौढाश्च प्रमदाः~।

यत्र च प्रमदानां चक्षुरेव सहजं मुण्डमालामण्डनं भारः कुवलयदलदामानि~। अलकप्रतिबिम्बान्येव कपोलतलगतान्यक्लिष्टाः श्रवणावतंसाः पुनरुक्तानि तमालकिसलयानि~। प्रियकथा एव सुभगा: कर्णालंकारा आडम्बरः कुण्डलादिः~। कपोला एव सततमालोककारका विभवो निशासु मणिप्रदीपाः~। निःश्वासाकृष्टमधुकरकुलान्येव रमणीयं मुखावरणं कुलस्त्रीजनाचारो जालिका~। वाण्येव मधुरा वीणा बाह्यविज्ञानं तन्त्रीताडनम्~। हासा एवाति\textendash

\vspace{2mm}
\hrule

{\s मातङ्गेत्यादयो विरोधाः~। मातङ्गो हस्ती, चण्डालश्च~। याः प्रमदाश्चण्डाला नपि गच्छन्ति ताः कथं शीलवत्य इति विरोधः सर्वत्र ज्ञेयः~। गौर्यो गौराङ्ग्यः~। विभव ऐश्वर्ये रक्ताः~। यत्र विगतो भवस्तत्र कथं गौरी रतेति~। विगतं भवे रतं यस्या वा~। श्यामाः श्यामलाङ्ग्यः~। पद्मरागिण्यो लोहितमणिभूषणाः~। श्यामा रात्रयः कथं पद्मरागिण्यः~। रात्रौ पद्मानां संकोचात्~। द्विजैर्दन्तैः~। शुचिवदना मदिरावन्मदिरया वा~। आमोदी श्वसनो मुखमारुतो यासां धवलद्विजवच्छुद्ध ब्राह्मणवच्छुचि वदनं ताः कथं मदिरामोदिश्वसनाः~। चन्द्रवत्कान्तं वपुर्यासाम्~। शिरीषपुष्पवत्सुकुमाराङ्ग्यश्चन्द्रकान्तस्य वपुर्यासां ताः कथं शिरीषकोमलाड्ग्यः~। भुजङ्गा विटाः~। कञ्चुकं स्त्रीणां वासः, वारबाणाख्यः~। याश्च कञचुकिन्यः सर्पिण्यस्ताः कथं भुजङ्गैर्न गम्याः~। कलत्रं जघनम्~। दरिद्रं क्षामं मध्यमुदरं यासाम्~। कलत्रस्य परिवारस्य पृथ्वीः श्रीस्ताः~। कथं दरिद्राणां निर्धनानां मध्ये कलिताः संख्याता भवन्ति~। लावण्यं सौन्दर्यम्~। मधुरं हृद्यम्~। लावण्यरसवतीनां मधुरभाषितं विभाव्यते~। अप्रमत्ताः प्रमादशून्याः~। प्रसन्नो मनोहरः~। उज्ज्वलो मनो हारी~। प्रसन्ना च सुरा तयोज्ज्वलो मुखरागो यासां ताः~। कथमप्रमत्ता अक्षीबा अकौतुका अकरकङ्कणाः~। विवाहितानां हि करकङ्कणोऽवबध्यते~। {\qt रुद्राक्षदर्पसिद्धार्थशिखिपक्षोरगत्वचः~। कङ्कणौषधयश्चेति कौतुकाख्याः प्रकीर्तिताः~॥}

मुण्डमालारूपं मण्डनं मुण्डमालामण्डनम् सहजमकृत्रिमम्~। अनेककुवलयदलदामाभ्यासोत्कर्षः, न तु कुवलयदलदामसंभवेऽपि प्रतिनिधिरूपतापादनम्~।}

\newpage
% तृतीय उच्छ्वासः~। ९९

\noindent
शयसुरभयः पटवासा निरर्थकाः कर्पूरपांसवः~। अधरकान्तिविसर एवोज्ज्वलतरोऽङ्गरागो निर्गुणो लावण्यकलङ्कः कुङ्कुमपङ्कः~। बाहव एव कोमलतमाः परिहासप्रहारवेत्रलता निष्प्रयोजनानि मृणालानि~। यौवनोष्मस्वेदविन्दव एव विदग्धाः कुचालंकृतयो हारास्तु भाराः~। श्रोण्य एव विशालस्फाटिकशिलातलचतुरस्रा रागिणां विश्रमकारणमनिमित्तं भवनमणिवेदिकाः~। कमललोभनिलीनान्यलिकुलान्येव मुखराणि पदाभरणकानि निष्फलानीन्द्रनीलनूपुराणि~। नूपुररवाहूता भवनकलहंसा एव समुचिताः संचरणसहाया ऐश्वर्यप्रपञ्चाः परिजनाः~।

तत्र च साक्षात्सहस्राक्ष इव सर्ववर्णधरं धनुर्दधानः, मेरुमय इव कल्याणप्रकृतित्वे, मन्दरमय इव लक्ष्मीसमाकर्षणे, जलनिधि मय इव मर्यादायाम्, आकाशमय इव शब्दप्रादुर्भावे, शशिमय इव कलासंग्रहे, वेदमय इवाकृत्रिमालापत्वे, धरणिमय इव लोकधृतिकरणे, पवनमय इव सर्वपार्थिवरजोविकारहरणे, गुरुर्वचसि, पृथुरुरसि, विशालो मनसि, जनकस्तपसि, सुयात्रस्तेजसि, सुमन्त्रो रहसि, बुधः सदसि, अर्जुनो यशसि, भीष्मो धनुषि, निषधो व\textendash

\vspace{2mm}
\hrule

\noindent
{\s भार इत्यनेनैष एवार्थः प्रकटितः~। एवमक्लिष्टा इत्यादौ बोद्धव्यम्~। आडम्बरः स्फुटः~। जालिका शिरोवस्त्रभेदः~। चतुरस्रा रम्याः~। विश्रमकारणमिति गुरुवात्~।

तत्र चेत्यादौ तत्र पुष्पभूतिर्नाम राजासीदिति संबन्धः~। वर्णा विप्राद्याः, शुक्लाद्याश्च~। कल्याणं श्रेयः, सुवर्णं च~। मन्दरेण श्रीराकृष्टामृतमन्थने, पुष्पभूतिना भैरवाचार्यवेतालसाधने~। मर्यादाचारः, सीमा च~। शब्दो यशोऽपि प्रादुर्भावः प्रकाशता~। कला गीताद्या, लेखाश्च~। अकृत्रिमः सत्ययुक्तः, अपौरुषेयश्च~। धृतिर्धैर्यम्, धारणं च~। पार्थिवो राजा, पृथिवीसंबन्धी च~। रजोविकारा रागाद्याः, रेणुकार्याणि~। गुर्वित्यादिना वक्रोक्त्याज्ञानां गुर्वादिमयत्वं सूचयति~। गुरुरुपदेष्टा, गुरुर्महान्~। गभीरशब्दत्वा गृहसतिश्च~। पृथुर्विपुलः, आदिराजश्च कश्चित्~। विशालो विस्तीर्णः, विशालायाच नृपा अभवन्~। अथ विशालो नाम बोधिसत्त्वः स एव शान्तः शान्तमना इत्यपि प्रतीतिरस्ति~। जनको जनयिता, जनक इव तपस्वी च~। {\qtt सुयात्र इति}~। शोभना या यस्य सोऽपि~। कर्तव्याववारण मन्त्रः स शोभनो यस्येति च~। बुधः पण्डितः, ग्रहश्च~। अर्जुनः शुक्रोऽपि~। भीष्मो भयानकः, गाङ्गेयश्च~। निषेधो धर्षणीयः, कठिनो वा, नलस्य}

\newpage
% १०० हर्षचरिते

\noindent
पुषि, शत्रुघ्नः समरे, शूरः शूरसेनाक्रमणे, दक्षः प्रजाकर्मणि, सर्वादिराजतेजःपुञ्जनिर्मित इव राजा पुष्पभूतिरिति नाम्ना बभूव~।

पृथुना गौरियं कृतेति यः स्पर्धमान इव महीं महिषीं चकार~। निसर्गस्वैरिणी स्वरुच्यनुरोधिनी च भवति हि महतां मतिः~। यतस्तस्य केनचिदनुपदिष्टा सहजैव शैशवादारभ्यान्यदेवताविमुखी भगवति, भक्तिसुलभे, भुवनभृति, भूतभावने, भवच्छिदि, भवे भूयसी भक्तिरभूत्~। अकृतवृषभध्वजपूजाविधिर्न स्वनेऽप्याहारमकरोत्~। अजम्, अजरम्, अमरगुरुम्, असुरपुररिपुम्, अपरिमितगणपतिम्, अचलदुहितृपतिम्, अखिलभुवनकृतचरणनतिम्, पशुपतिं प्रपन्नोऽन्यदेवताशून्यममन्यत त्रैलोक्यम्~। भर्तृचित्तानुवर्तिन्यश्चानुजीविनां प्रकृतयः~। तथा हि~। गृहे गृहे भगवानपूज्यत खण्डपरशुः~। ववुरस्य होमालवालविलीयमानबहलगुग्गुलुगन्धगर्भाः स्नपनक्षीरशीकरक्षोदक्षारिणो बिल्वपल्लवदामदलोवाहिनः पुण्यविषयेषु वायवः~। शिवसपर्यासमुचितैरुपायनैः प्राभृतैश्च पौराः पादोपजीविनः सचिवा भुजबलनिर्जिताश्च करदीकृता महासामन्तास्तं सिषेविरे~। तथा हि~। कैलासकूटधवलैः कनकपत्रलतालंकृतविषाणकोटिभिर्महाप्रमाणैः संध्याबलिवृषैः सौवर्णैश्च स्नपनकलशैरर्घभा जनैश्च धूपपात्रैश्च पुष्पपट्टैश्च मणियष्टीप्रदीपैश्च ब्रह्मसूत्रैश्च महार्हमाणिक्यखण्डखचितैश्च मुखकोषैः परितोषमस्य मनसि चक्रुः~। अन्तःपुराण्यपि स्वयमारब्धबालेयतण्डुलकण्डनानि देवगृहोपलेपनलोहिततरकरकिसलयानि कुसुमग्रथनव्यग्रसमस्तपरिजनानि तस्यामिलषितमन्ववर्तन्त~। तथा च~। परममाहेश्वरः स भूपालो लोकतः

\vspace{2mm}
\hrule

\noindent
{\s च पिता, गिरिभेदो वा~। शूरो विक्रान्तः, यदूनां राजा च~। दक्षश्चतुरः, प्रजापतिश्च~।

महिषीं महादेवीमपि~। निसर्गः स्वभावः~। स्वैरिणी स्ववशा~। खण्डपरशुः शिवः~। ववुरवहन्~। होमालवालमग्निकुण्डम्~। सपर्या पूजा~। उपायनं ढौकनिका स्वयमानीयते~। प्राभृतं कौशलिका सखिभिः प्रहीयते~। करदीकृता दण्डजाः कृताः~। कूटं शृङ्गम्~। यत्र वस्त्रेषु पुष्पाणि सूत्रैः क्रियते स पुष्पपट्टः~। ज्वलन्मणिशिखरा स्वर्णयष्टिर्यष्ठिप्रदीपा~। ब्रह्मसूत्रैर्यज्ञोपवीतैः~। मुखयुक्ताः कोषा मुखकोषाः~। ये लिड्गोपरि दीयन्ते~। बलये हिता बालेयाः~। {\qt छदिरुषधिबलेर्ढञ्}~। ग्रथनशब्दश्चिन्त्यः~। ग्रन्थनमिति भाव्यम्~। अभिलषितमन्ववर्तन्तेत्यनेन चित्तानुवृत्तिः}

\newpage
% तृतीय उच्छ्वासः~। १०१ 

\noindent
शुश्राव भुवि भगवन्तमपरमिव साक्षाद्दक्षमखमथनं दाक्षिणात्यं बहुविधविद्याप्रभावप्रख्यातैर्गुणैः शिष्यैरिवानेकसहस्रसंख्यैर्व्याप्तमर्त्यलोकं भैरवाचार्यनामानं महाशैवम्~। उपनयन्ति हि हृदयमदृष्टमपि जनं शीलसंवादाः~। यतः स राजा श्रवणसमकालमेव तस्मिन्भैरवाचार्ये भगवति द्वितीय इव कपर्दिनि दूरगतेऽपि गरीयसीं बबन्ध भक्तिम्~। आचकाङ्क्ष च मनोरथैरप्यस्य सर्वथा दर्शनम्~।

अथ कदाचित्पर्यस्तेऽस्ताचलचुम्बिनिं वासरेऽन्तःपुरवर्तिनं राजानमुप्सृत्य प्रतीहारी विज्ञापितवती \textendash\ {\haq देव, द्वारि पारिव्राडास्ते कथयति च भैरवाचार्यवचनाद्देवमनुप्राप्तोऽस्मि} इति~। राजा तु तच्छ्रुत्वा सादरम् \textendash\ {\haq क्वासौ~। आनयात्रैव~। प्रवेशयैनम्} इति चाब्रवीत्~। तथा चाकरोत्प्रतीहारी~। न चिराच्च प्रविशन्तं प्रांशुमाजानुभुजम्, भैक्षक्षाममपि स्थूलास्थिभिरवयवैः पीवरमिवोपलक्ष्यमाणम्, पृथूत्तमाङ्गमुत्तुङ्गबलिमङ्गस्थपुटललाटम्, निर्मांसगण्डकूपकम्, मधुबिन्दुपिङ्गलपरिमण्डलाक्षम्, ईषदावक्रघोणम् , अतिप्रलम्बैककर्णपाशम्, अलाबुबीजविकटोन्नतदन्तपङ्क्तिम् , तुरगानूकश्लथाधरलेखम्, लम्बचिबुकायततरलपनम् , अंसावलम्बिना काषायेण योगपट्टकेन विरचितवैकक्षकम्, हृदयमध्यनिबद्धग्रन्थिना च रागेणेव खण्डशः कृतेन धातुरसारुणेन कर्पटेन कृतोत्तरासङ्गम्~। पुनरुक्तबालप्रग्रहवेष्ठननिश्चलमूलेन बद्धमृत्परिशोधनवंशत्वक्तितउना कौपीनसनाथशिखरेण खर्जूरपुटसमुद्गकगर्भीकृतभिक्षाकपालकेन दारकफलकत्रयत्रिकोणत्रियष्टिनिविष्टकमण्डलुना बहिरुपपा\textendash

\vspace{2mm}
\hrule

\noindent
{\s शुद्धान्तानां वर्णिता~। {\qtt भुवीति}~। भूस्थत्वेऽप्यसुलभत्वदर्शनमस्योक्तम्~। शीलसंवादाश्चारित्रसादृश्यानि~। कपर्दिनि शिवे~।

न चिराच्चेत्यादौ मस्करिणमद्राक्षीदिति संबन्धः~। प्रांशुं दीर्घम्~। जानुरुरूपर्व~। उक्तं च \textendash\ {\qt जङ्घा तु प्रसृता जानूरुपर्वाष्टीवदस्त्रियाम्}~। पीवरं स्थूलम्~। स्थपुटं निम्नोन्नतम्~। ललाटमललिकम्, गोधिः~। गण्डकूपोऽक्ष्णोरधोदेशः~। घोणा नासिका~। अलाबुस्तुम्बी~। उक्तं च \textendash\ {\qt तुम्ब्यलाबू उभे समे}~। तुरगानामधस्ता दोष्ठोऽनूकः~। {\qt अधोऽधरस्य चिबुकम्}~। लपनं मुखम्~। उत्तरासङ्गमुपरिप्रावरणम्~। पुनरुक्तं पौनःपुन्येन कृतम्~। प्रग्रहो रज्जुः~। तितउश्चालनी~। परिपवनशब्दवाच्यः~। कौपीनं गुह्यदेश उपचारात् , तदाच्छादनं च~। खर्जूराख्यस्य वृक्षस्य च संबन्धिभिः पुटैः क्लिष्टैः, पत्रैश्च~। समुद्रकः कपालभङ्गः~। भिक्षायै कियते दारवे }

\newpage
% १०२ हर्षचरिते 

\noindent
दितपादुकावस्थानेन स्थूलदशासूत्रनियन्त्रितपुस्तिकापूलिकेन वामकरधृतेन योगभारकेणाध्यासितस्कन्धम् , इतरकरगृहीतवेत्रासनं मस्करिणमद्राक्षीत्~। क्षितिपतिरप्युपगतमुचितेन चैनमादरेणान्वग्रहीत्~। आसीनं च पप्रच्छ \textendash\ {\haq क्व भैरवाचार्यः} इति~। सादरनरपतिवचनमुदितमनास्तु परिव्राट् तमुपनगरं सरस्वतीतटवनावलम्बिनि शून्यायतने स्थितमाचचक्षे~। भूयश्चाबभाषे \textendash\ अर्चयति हि महाभागं भगवानाशीर्वचसाः इत्युक्त्या चोपनिन्ये योगभारकादाकृष्य भैरवाचार्यप्रहितानि रत्नवन्ति बहलालोकलिप्तान्तःपुराणि पञ्च राजतानि पुण्डरीकाणि~। 

नरपतिस्तु प्रियजनप्रणयभङ्गकातरो दाक्षिण्यमनुरुध्यमानो ग्रहणलाघवं च लङ्घायितुमसमर्थों दोलायमानेन मनसा स्थित्वा कथं कथमप्यतिसौजन्यनिघ्नस्तानि जग्राह~। जगाद च \textendash\ {\haq सर्वफलप्रसवहेतुः शिवभाक्तिरियं नो मनोरथदुर्लभानि फलति फलानि~। येनैवमस्मासु प्रीयते भगवान्भुवनगुरुर्भैरवाचार्यः~। श्वो द्रष्टास्मि भगवन्तम्} इत्युक्त्वा च मस्करिणं व्यसर्जयत्~। अनया च वार्तया परां मुदमवाप~। 

अपरेद्युश्च प्रातरेवोत्थाय वाजिनमधिरुह्य समुच्छ्रितश्वेतातपत्रः समुद्धूयमानधवलचामरयुगलः कतिपयैरेव राजपुत्रैः परिवृतो भैरवाचार्यं सवितारमिव शशी द्रष्टुं प्रतस्थे~। गत्वा च किंचिदन्तरं तदीयमेवाभिमुखमापतन्तमन्यतमं शिष्यसमद्राक्षीत्~। अप्राक्षीच्च \textendash\ {\haq क्व भगवानास्ते} इति~। सोऽकथयत् \textendash\ {\haq अस्य जीर्णमातृगृहस्योत्तरेण बिल्ववाटिकामध्यास्ते} इति~। गत्वा च तं प्रदेशमवततार~। प्रविवेश च बिल्ववाटिकाम्~। 

अथ महतः कापेटिकवृन्दस्य मध्ये प्रातरेव स्नातम् , दत्ताष्ट\textendash

\vspace{2mm}
\hrule

\noindent
{\s काष्ठसंबन्धिनि फलकत्रये त्रयः कोणास्तेषु यास्तिस्रो यष्टयस्तासु निविष्टः कमण्डलुर्यत्र तेन~। योगभारकेण मात्राभारिकया~। मस्करीणं परिव्रजकम्~। राजतानि रौप्यानि~।

लङ्घयितुमुत्सोढुम्~। निघ्नः स्ववशः~।

अन्यतममपरम्~। उत्तरेणोत्तरस्यां दिशि~।

अथेत्यादौ भैरवाचार्यं ददर्शेति संबन्धः~। कार्पटिका व्रतिनः~। अष्टपुष्पिका}

\newpage
% तृतीय उच्छ्वासः~। १०३

\noindent
पुष्पिकम्, अनुष्ठिताग्निकार्यम्, कृतभस्मरेखापरिहारपरिकरे हरितगोमयोपलिप्तक्षितितलवितते व्याघ्रचर्मण्युपविष्टम्, कृष्णकम्बलप्रावरणनिभेनासुरविवरप्रवेशाशङ्कया पातालान्धकारावासमिवाभ्यस्यन्तम् , उन्मिषत्ता विद्युत्कपिलेनात्मतेजसा महामांसविक्रयक्रीतेन मनःशिलापङ्केनेव शिष्यलोकं लिम्पन्तम्, जटीकृतैकदेश लम्बमानरुद्राक्षशङ्खगुटिकेनोर्ध्वबद्धेन शिखापाशेन बध्नन्तमिव विद्यावलेपदुर्विदग्धानुपरि संचरतः सिद्धान्धवलकतिपयशिरोरुहेण वयसा पञचपश्चाशतं वर्षाण्यतिक्रामन्तम, खालित्यक्षीयमाणशङ्क खलोमलेखम्, लोमशकर्णशष्कुलीप्रदेशं पृथुललाटतटम् , तिरश्च्या भस्म ललाटिकया बहुशः शिरोर्धवृतदग्धगुग्गुलुसंतापस्फुटितकपालास्थिपाण्डुरराजिशङ्कामिव जनयन्तम्, सहजललाटवलिभङ्गसंकोचितकूर्चभागां बभ्रुभासं भ्रूसंगत्या निरन्तरामायामिनीमेकामिव भ्रूलेखां बिभ्राणम्, ईषत्काचकाचरकनीनिकेन रक्तापाङ्गनिर्गतांशुप्रतानेन मध्यधवलभासेन्द्रायुधेनेवातिदीर्घेण लोचनयुगलेन परितो महामण्डलमिवानेकवर्णरागमालिखन्तम्, सितपीतलोहितपताकावलीशबलम्, शिवबलिमिव दिक्षु विक्षिपन्तम्, तार्क्ष्यतुण्डकोटिकुब्जाग्रघोणम्, दूरविदीर्णसृक्कसंक्षिप्तकपोलम्, किंचिद्दन्तुरतया सदाहृदयसंनिहितहरमौलिचन्द्रातपेनेव निर्गच्छता दन्तालोकेन धवलयन्तं दिशां जालकम् , जिह्वाग्रस्थितसर्वशैवसंहितातिभारणेव मनाक्प्रलम्बितोष्ठम् , प्रलम्बश्रवणपालीप्रेङ्खिताभ्यां स्फाटिककुण्डलाभ्यां शुक्रबृहस्पतिभ्यामिव सुरासुरविजयविद्यासिद्धिश्रद्धयानुबध्यमानम्, बद्धविविधौषधिमन्त्रसूत्रपङ्क्तिना सलोहवलयेनैकप्रकोष्ठेन शङ्खखण्डं पूष्णो दन्तमिव भगवता भवेन भग्नं भक्त्या भूष\textendash

\vspace{2mm}
\hrule

\noindent
{\s प्रागुक्ता~। परिहारोऽत्र मर्यादा शङ्खे ललाटास्थ्नि~। उक्तं च \textendash\ शङ्खो निधौ ललाटास्थ्नि~। गुटिका खण्डिका~। उपरीत्यायभिप्रायेणोक्तम् \textendash\ {\qtt ऊर्ध्ववद (बद्धे)नेति}~। प्रशस्ता शिखा शिखापाशः~। अवलेपोऽहंकृतिः~। खालित्यं खल्वाटता~। शङ्खो ललाटास्थि~। शष्कुली कुहरम्~। {\qt कूर्चमस्त्री भ्रुवोर्मध्यम्}~। काचरा पीतवर्णा~। तुण्डं मुखम्~। कोटिः प्रान्तः, चञ्चवग्रम्~। {\qt प्रान्तावोष्ठस्य सृक्किणी}~। {\qt प्रकोष्ठमन्तरं विद्यादरत्रिमणिबन्धयोः}~। पूष्णो रविभेदस्य पुरा दक्षयज्ञगतस्य हरं निन्दतः मय्यनागते किमर्थमागतोऽसि इति मुष्टिप्रहारेण हरेण दन्ता भग्नाः~।}

\newpage
% १०४ हर्षचरिते 

\noindent
णीकृतं कलयन्तम् , अखिलरसकूपोदञ्चनघटीयन्त्रमालामिव रुद्राक्षमालां दक्षिणेन पाणिना भ्रमयन्तम् , उरसि दोलायमानेनापिङ्गलाग्रेण कूर्चकलापेन संमाजेयन्तमिवान्तर्गतं निजरजोनिकरम् , अतिनिबिडनीललोममण्डलविचितं च ध्यानलब्धेन ज्योतिषा दग्धमिव हृदयदेशं दधानम्, ईषत्प्रशिथिलवलिवलयबध्यमानतुन्दम्, उपचीयमानस्फिड्भांसपिण्डकम्, पाण्डुरपवित्रक्षौमावृतकौपीनम्, सावष्टम्भपर्यङ्कबन्धमण्डलितेनामृतफेनश्वेतरुचा योगपट्टकेन वासुकिनेवाप्रतिहतानेकमन्त्रप्रभावाविर्भूतेन प्रदक्षिणीक्रियमाणम्, अरुणतामरससुकुमारतलस्य पादयुगलस्य निर्मलैर्नखमयूखजालकैर्जर्जरयन्तमिव महानिधानोद्धरणरसेन रसातलम् , तोयक्षालितशुचिना धौतपादुकायुगलेन हंसमिथुनेनेव भागीरथीतीर्थयात्रापरिचयागतेनामुच्यमानचरणान्तिकम्, शिखरनिखातकुब्जकालायसकण्टकेन वैणवेन विशाखिकादण्डेन सर्वविद्यासिद्धिविघ्नविनायकापनयनाङ्कुशेनेव सततपार्श्ववर्तिना विराजमानम्, अबहुभाषिणं मन्दहासिनं सर्वोपकारिणं कुमारब्रह्मचारिणम्, अतितपस्विनं महामनस्विनं कृशक्रोधम् , अकृशानुरोधम् , महानगरमिवादीनप्रकृतिशोभितम्, मेरुमिव कल्पतरुपल्लवराशिसुकुमारच्छायम्, कैलासमिव पशुपतिचरणरजः \textendash\ पवित्रितशिरसम् , शिवलोकमिव माहेश्वरगणानुयातम्, जलनिधिमिवानेकनदनदीसहस्रप्रक्षालितशरीरम् , जाह्नवीप्रवाहमिव बहुपुण्यतीर्थस्थानशुचिम् , धाम धर्मस्य, तीर्थं तथ्यस्य, कोशं कुशलस्य, पत्तनं पूततायाः, शालां शीलस्य, क्षेत्रं क्षमायाः, शालेयं शा\textendash

\vspace{2mm}
\hrule

\noindent
{\s तत्करस्पर्शेन पावनत्वात्तत्र भक्तिः~। अखिलस्य रसस्य कूपादुदञ्चनाय घटीयन्त्रमालापि भ्रम्यते~। दोलायमानत्वेन संमार्जनसंभावना~। कलापग्रहणं मार्जनीसादृश्यार्थम्~। रजो रागः, रेणुश्च~। विचितं व्याप्तम्~। तुन्दमुदरम्~। स्फिजावुभ इति प्रसिद्धे~। {\qt स्त्रियां स्फिजौ कटीप्रोथौ} इत्यमरः~। फेनवत्तैश्च श्वेता~। {\qtt वासुकिनेवेति}~। न सामान्येनेति प्रभावपरिशोधकम् जर्जरयन्तं खण्डशः कुर्वाणं~। {\qtt तोयेत्यादि}~। हंसमिथुनस्यापि विशेषणम्~। शिखरेत्यादिनाङ्कुशसादृश्यं विशाखिकादण्डस्योक्तम्~। निखात उत्कीर्णः~। कालायसं शस्त्रभेदः~। विशाखिका खनित्रिका~। विघ्न इव विनायको गजाननः~। प्रकृतिः स्वभावः, मायादिका च~। राशिवत्तेन च सुकुमाराः~। गणाः समूहाः प्रमथाश्च~। नदनदीत्येकशेषो युक्तः~। सहस्रेषु तैः प्रक्षालितशि(शरी)राः~। तीर्थेषु यत्स्थानं}

\newpage
% तृतीय उच्छ्वासः~। १०५ 

\noindent
लीनतायाः, स्थानं स्थितेः, आधारं धृतेः, आकरं करुणायाः, निकेतनं कौतुकस्य, आरामं रामणीयकस्य, प्रासादं प्रसादस्य, आगारं गौरवस्य, समाजं सौजन्यस्य, संभवं सद्भावस्य, कालं कलेः, भगवन्तं साक्षादिव विरूपाक्षं भैरवाचार्यं ददर्श~।

भैरवाचार्यस्तु दूरादेव राजानं दृष्ट्वा शशिनमिव जलनिधिश्चचाल~। प्रथमतरणोेत्थितशिष्यलोकश्चोत्थाय प्रत्युज्जगाम समर्पितश्रीफलोपायनश्च~। जह्नुकर्णसमुद्गीर्यमाणगङ्गाप्रवाहह्रादगम्भीरया गिरा स्वस्तिशब्दमकरोत्~।

नरपतिरपि प्रीतिविस्तार्यमाणधवलिम्ना चक्षुषा प्रत्यर्पयन्निव बहुतराणि पुण्डरीकवनानि ललाटपट्टपर्यस्तेन चोदंशुना शिखामणिना महेश्वरप्रसादमिव तृतीयनयनोद्गमेन प्रकाशयन्नावर्जितकर्णपक्कवपलायमानमधुकरः शिवसेवासमुन्मूलिताशेषपापलवमुच्यमान इव दूरावनतः प्रणाममभिनवं चकार~। आचार्योऽपि \textendash\ {\haq आगच्छ~। अत्रोपविश} इति शार्दूलचर्मात्मीयमदर्शयत्~। उपदर्शितप्रश्रयस्तु राजा मत्तहंसकलगद्गदस्वरसुभगां मधुरसमयी महानदीमिव प्रवर्तयन्वाचं व्याजहार \textendash\ भगवन् , नार्हसि मामन्यनृपस्खलितैः खलीकर्तुम्~। अशेषराजकोपेक्षिताया हतलक्ष्म्याः खल्वयं शीलापराधो द्रविणदौरात्म्यं वा यदेवमाचरति मयि गुरुः~। अभूमिरयमुपचाराणाम्~। अलमतियन्त्रणया~। दूरस्थितोऽपि मनोरथशिष्योऽयं जनो भवताम्~। माननीर्य च गुरुवन्नोल्लङ्घनमर्हति गुरोरासनम्~। आसतां च भवन्त एवात्र इति व्याहृत्य परिजनोपनीते वाससि निषसाद~। भैरर्वाचार्योऽपि प्रीत्यानतिक्रमणीयं नृपवचनमनुवर्तमानः पूर्ववत्तदेव व्याघ्राजिनमभजत~।

आसीने च सराजके परिजने शिष्यजने च समुचितमर्घ्या\textendash

\vspace{2mm}
\hrule

\noindent
{\s वसनं तेन शुचिम्~। तीर्थस्नानैः कनखलाद्यवस्थितिभिश्च शुचिः~। शालीनता विनीतत्वम्~। निकेतनं गृहम्~। तत्र हि सर्वस्य कौतुको जायते~।

शश्यपि राजा, तं च दूरादेव दृष्ट्वा जलनिधिश्चलति~। गाम्भीर्याच्च जलनिधिरेत्वेत्युक्तम्~। बिल्वं श्रीफलम्~। गङ्गेत्यादिना पवित्रत्वमाह~।

धवलिम्रेत्यनेन पुण्डरीकानां धवलत्वमाह~। प्राभृतपुण्डरीकानां राजत्वात्~। आवर्जितं नमितं स्वरवच्च तेन सुभगात्~। शार्दूलो व्याघ्रः~।}

\lfoot{ह० १०}

\newpage
\lfoot{}
% १०६ हर्षचरिते 

\noindent
दिकं चक्रे~। क्रमेण च नृपमाधुर्यहृतान्तःकरणः शशिकरनिकरविमला दशनदीधितीः स्फुरन्तीः शिवभक्तीरिव साक्षाद्दर्शयन्नुवाच \textendash\ तात, अतिनम्रतैव ते कथयति गुणानां गौरवम्~। सकलसंपत्पात्रमसि~। विभवानुरूपास्तु प्रतिपत्तयः~। जन्मनः प्रभृत्यदत्तदृष्टिरस्मि स्वापतेयेषु~। यतः सकलदोषकलापानलेन्धनैर्धनैरविक्रीतं क्वचिच्छरीरकमस्ति~। भैक्षरक्षिताः सन्ति प्राणाः~। दुर्गृहीतानि कतिचिद्विद्यन्ते विद्याक्षराणि~। भगवच्छिवभट्टारकपादसेवया समुपार्जिता कियत्यपि संनिहिता पुण्यकणिका~। स्वीक्रियतां यदत्रोपयोगार्हम्~। प्रतनुगुणग्राह्याणि कुसुमानीव हि भवन्ति सतां मनांसि~। अपि च~। विद्वत्संमताः श्रूयमाणा अपि साधवः शब्दा इव सुधीरेऽपि हि मनसि यशांसि कुर्वन्ति~। विवरं विशतः कुतूहलस्य फेनधवलैः स्रोतोभिरिवापह्रियमाणो गुणगणैरानीतोऽस्मि कल्याणिना इति~।

रजा तु तं प्रत्यवादीत् \textendash\ भगवन् , अनुरक्तेष्वपि शरीरादिषु साधूनां स्वामिन एव प्रणयिनः~। युष्मद्दर्शनादुपार्जितमेव चापरिमितं कुशलजातम्~। अनेनैवागमनेन स्पृहणीयं पदमारोपितोऽस्मि गुरुणा~। इति विविधाभिश्च कथाभिश्चिरं स्थित्वा गृहमगात्~।

अन्यस्मिन्दिवसे भैरवाचार्योऽपि राजानं द्रष्टुं ययौ~। तस्मै च राजा सान्तःपुरं सपरिजनं सकोषसमात्मानं निवेदितवान्~। स च विहस्योवाच \textendash\ तात, क्क विभवः, क्क च वयं वनवर्धिताः~।

\vspace{2mm}
\hrule

{\s अन्तःकरणं मनः~। गौरवमुत्कर्षः, भारवत्त्वं च~। {\qtt अदत्तदृष्टिरिति}~। न तु मया धनान्यलभ्यानि~। स्वापतेयेषु धनेषु~। संरक्षिता इति यदि कदाचित्क्वचिदुपयोगं यास्यन्तीति~। अनेन प्राणादिदानमेवोचितमित्युक्तम्~। सकलसंपत्पात्रस्येयतः कियती वसुसंपत्तिर्भविष्यतीत्याशङ्क्याह \textendash\ {\qtt प्रतन्वित्यादि}~। गुणा उत्कर्षाः, तन्तवश्च~। {\qtt कुसुमानीवेति}~। कुसुमसादृश्येन मनसः सौकुमार्यमप्युक्तम्~। साधवः शिष्टाः, शब्दा इव साधवः~। संस्कृता विद्वत्संमताश्च~। फेनवैत्तैश्च धवलैर्गुणगणैः स्रोतोभिश्च~।

{\qtt स्वामिन एव प्रणयिन इति}~। अनुक्तान्यपि शरीरादीनि प्रणयिनां स्वायत्तानीत्यर्थः~।}

\end{document}