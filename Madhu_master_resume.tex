\documentclass[10pt,article]{article}
\usepackage[letterpaper,margin=0.3in]{geometry}
\usepackage{graphicx}
\usepackage{booktabs}
\usepackage{url}
\usepackage{enumitem}
\usepackage{palatino}
\usepackage{tabularx}
\usepackage[table]{xcolor}
\fontfamily{SansSerif}
\selectfont
\definecolor{lightgray}{HTML}{C0C0C0}

\usepackage[T1]{fontenc}
\usepackage[utf8]{inputenc}
\usepackage{amssymb}
\usepackage{fontawesome}
\usepackage[hidelinks]{hyperref}

\usepackage{color}
\definecolor{mygrey}{gray}{0.82}
\textheight=10.25in
\raggedbottom

\setlength{\tabcolsep}{0in}
\newcommand{\isep}{-2 pt}
\newcommand{\lsep}{-0.5cm}
\newcommand{\psep}{-0.6cm}
\renewcommand{\labelitemii}{$\circ$}

\pagestyle{empty}
%-----------------------------------------------------------
%Custom commands
\newcommand{\resitem}[1]{\item #1 \vspace{-2pt}}
\newcommand{\resheading}[1]{{\noindent \large \colorbox{mygrey} { \begin{minipage}{0.99\textwidth}\centering{\textbf{#1 \vphantom{p\^{E}}}}\end{minipage}}}}
\newcommand{\ressubheading}[3]{
\begin{tabular*}{6.62in}{l @{\extracolsep{\fill}} r}
	\textsc{{\textbf{#1}}} & \rightline\textsc{\textit{[#2]}} \\
\end{tabular*}\vspace{-8pt}}
%-----------------------------------------------------------


\begin{document}
\begin{table}
    \begin{minipage}{0\linewidth}
        \centering
        \includegraphics[height =0.8in]{logo.jpg}
    \end{minipage}
    \hspace{5mm}
    \begin{minipage}{0.8\linewidth}
        %\centering
        \setlength{\tabcolsep}{50pt}
        \def\arraystretch{1.2}
        \begin{tabular}{l l l}
            {\Huge\bfseries\MakeUppercase{K Madhulaalasa}\par} &  {Mob. +91 9393576480} \\
            {\large\bfseries{M.Tech, Aerospace Engineering | IIT Kanpur}\par} & {kmadhu21@iitk.ac.in} \\
            {\large\bfseries{Structures, Structural Dynamics \& Aeroelasticity}\par} &  {janukurma00@gmail.com}
        \end{tabular}
    \end{minipage}
\end{table}

\resheading{ACADEMIC QUALIFICATIONS}
\vspace{2mm}
\setlength{\tabcolsep}{20pt}
\noindent
\begin{tabular}{p{4.5cm} p{6cm} p{2cm} p{2cm}}
\toprule
\toprule
\textbf{Degree}    & \textbf{College / University}     & \textbf{Year}     & \textbf{CGPA /\%} \\ 
\toprule
\textbf{M.Tech} & Indian Institute of Technology, Kanpur & 2021-23   & 8.75/10\\ 
\textit{(Aerospace Engineering)} & \hfill \textit{Kanpur, Uttar Pradesh} & & \\
\textbf{B.Tech} & Institute of Aeronautical Engineering & 2017-21 & 9.67/10\\ 
\textit{(Aeronautical Engineering)} & \hfill \textit{Hyderabad, Telangana} & & \\
\textbf{Intermediate} & Srigayatri Educational Institutions & 2017 & 97.4\%\\
& \hfill \textit{Hyderabad, Telangana} & & \\
\textbf{High School} & Kendriya Vidyalaya & 2015 & 9.6/10\\
& \hfill \textit{Hyderabad, Telangana} & & \\
\bottomrule
\bottomrule
\end{tabular}

\vspace{2mm}
\resheading{ACHIEVEMENTS}
\begin{itemize}
\setlength\itemsep{-0.3em}
\item Secured \textbf{AIR 64} in GATE 2021 in the discipline of Aerospace Engineering.
\item Stood \textbf{3$^{rd}$} place in the paper presentation on \textbf{SELF-HEALING MATERIALS} conducted by Institute of Aeronautical Engineering during Consortium technical fest, 2019.
\item Won \textbf{3$^{rd}$} position in Avion-E \textbf{RC plane design} competition conducted by National Institute of Technology, Warangal during Technozion, 2018.
\item Secured \textbf{1$^{st}$} position at regional level \textbf{National Youth Parliament Competition} held in session 2014-15 and received certificate from \textbf{Ministry of Parliamentary Affairs}.
\item Received a Merit certificate from the \textbf{Ministry of Enviornment, Forests and Climate Change} for securing \textbf{1$^{st}$} in Slogan competition organised on the occasion of 20$^{th}$ International day for the preservation of Ozone layer.
\end{itemize}

\resheading{PROJECTS}
\begin{itemize}
\setlength\itemsep{-0.2em}
\item \textbf{M.Tech Thesis} (ongoing) \hfill \textit{May'22-Present}\\
\textbf{\textit{Title:} Manufacturing and Testing of Helicopter Composite Blade}\\
\textit{Supervisors:} Dr PM Mohite \& Dr Abhishek
\vspace{-2mm}
\begin{itemize}
    \item To construct a composite rotor blade and in addition, chordwise, spanwise mass balancing and flap-wise inertia matching were to be carried out to minimize blade to blade dis-similarities.
    \item Carry out structural analysis for individual standalone parts of composite rotor.
    \item To manufacture composite blade parts including mould used for curing process and to assemble such that the blade withstands the specified loads.
    \item To design and perform experimental setup to precise bending and torsional loads on the rotor blade. The blade deformations need to be observed.
\end{itemize}

\item \textbf{DRDL Project} \hfill \textit{Jun'21-Aug'21}\\
\textbf{\textit{Title:} Design and Structural Analysis of Boiler Pressure Vessel}\\
\textit{Supervisor:} Darshan Trivedi \\
\textit{Defence Research and Development Laboratory, Kanchanbagh, Hyderabad}
\vspace{-2mm}
\begin{itemize}
    \item The aim of the project is to design boiler pressure vessel based on the requirement of the storage conditions and perform static structural analysis for the vessel. 
    \item To legalise the design, efforts are undertaken in this project to design the pressure vessel using ASME norms and standards (ASME code sec VIII division I).
    \item In ANSYS, a finite element study of the pressure vessel was performed. The vessel's static structural analysis was carried out by providing internal pressure, standard earth gravity, and fastening both legs.
    \item In addition, with the design and analysis, the failure modes possible for the pressure vessels are also studied.
\end{itemize}

\item \textbf{B.Tech Training} \hfill \textit{Feb'21-May'21}\\
\textbf{\textit{Title:} Effect of Leading Edge Sweep on Wrap-Around Fins}\\
\textit{Supervisors:} Dr PK Mohanta \& Swapnil Spakale \\
\textit{Bharat Dynamics Limited, Kanchanbagh, Hyderabad}
\vspace{-2mm}
\begin{itemize}
    \item The project was focused on the context of a numerical study of the supersonic flow over varying trajectories and geometries of wrap-around fins. 
    \item The investigation is being performed to determine the effect of leading-edge angle and leading-edge sweep on the aerodynamic coefficients of the projectile geometry and analyze the effect of roll moment on the fins at various operating conditions. 
    \item A group of four models is considered standard TTCP with blunt leading edge and with 450 leading-edge along with modified TTCP with the blunt leading edge along with 300 sweep and with 450 leading-edge along with 300 sweep.
    \item The flow field solution is used to compute the roll moment coefficients, which are then compared to other numerical models and experimental results. The standard wrap-around TTCP models are subjected to varying velocities ranging from Mach 1.5-2.5.
\end{itemize}

\item \textbf{Mini-Project} \hfill \textit{Dec'19-May'20}\\
\textbf{\textit{Title:} Conceptual Design of Multi-Cambered Bioinspired Morphing Wing}\\
\textit{Supervisor:} Yagna Dutta Dwivedi \\
\textit{Institute of Aeronautical Engineering}
\vspace{-2mm}
\begin{itemize}
    \item The objective of this project is to design the structure of bird wing conceptually by using variable camber along the length of the wing to reach the artificial requirement of the bird performance.
    \item The complexity of creating the bird's wing is studied. Instead of providing the changing shape in the wings to achieve its flight, the multi-camber wing will be designed and drafted in Catia V5, in which the camber and airfoil will be different along its length.
\end{itemize}
\end{itemize}

\resheading{COURSE PROJECT WORK AND TERM PAPER}
\begin{itemize}
\setlength\itemsep{-0.2em}
\item \textbf{Course: Instrumentation, Measurements and Experiments in Fluids (AE696A) | Project} \hfill \textit{Semester II}\\
\textit{Supervisor: Dr Debopam Das}\\
\textbf{\textit{Title:} Convection Pattern in Heated Hele-Shaw Cell}
\vspace{-2mm}
\begin{itemize}
    \item Experiments were conducted in a Hele-Shaw cell filled with oil at atmospheric pressure to evaluate convective heat transfer during boiling.
    \item There were two setups in consideration, one is an aluminium plate and a glass plate; another is a copper plate and an acrylic sheet were used to create a heated Hele-Shaw cell. For 2mm plate spacing, the resulting patterns were recorded.
\end{itemize}

\item \textbf{Course: Composite Materials (AE681A) | Term Paper} \hfill \textit{Semester II}\\
\textit{Supervisor: Dr Tanmoy Mukhopadhyay} \\
\textbf{\textit{Title:} Manufacturing and Service Life; Uncertainties of Laminated Composite}
\vspace{-2mm}
\begin{itemize}
    \item Aimed to discuss the most common probabilistic methodologies for modelling and propagating uncertainties in laminated composites in the term paper.
    \item The key advantages of sensitivity-uncertainty analysis for composite structures were studied.
\end{itemize}

\item \textbf{Course: Finite Element Analysis (AE675A) | Project} \hfill \textit{Semester II}\\
\textit{Supervisor: Dr Pritam Chakraborty}\\
\textbf{\textit{Title:} FEM solution for Prandtl-Stress function using MATLAB}
\vspace{-2mm}
\begin{itemize}
    \item The Prandtl stress theory for torsion has been solved using finite element method in MATLAB. 
    \item A two-dimensional boundary value problem of solid mechanics is given in terms of a single dependent unknown and includes torsion of cylindrical members and transverse deflection of membranes.
\end{itemize}

\item \textbf{Course: Aerospace Structural Analysis-I (AE670A) | Term Paper} \hfill \textit{Semester I}\\
\textit{Supervisor: Dr Tanmoy Mukhopadhyay}\\
\textbf{\textit{Title:} Analysis of Buckling Induced Instability in Slender Columns}
\vspace{-2mm}
\begin{itemize}
    \item The evaluation was performed to determine the buckling instability for columns for different cross-sections and analyze the critical load for the four boundary conditions.
    \item A group of three columns of same length and different cross-sections were considered namely T-section, modified C-section and L section. The structural behavior of each cross-section was obtained and considered as a function of eigen values and its modes of buckling. 
    \item The results of the buckling for different cross-sections were obtained through ABAQUS CAE solver and compared.
\end{itemize}
\end{itemize}

\resheading{INTERNSHIP}
\begin{itemize}
\setlength\itemsep{-0.2em}
    \item \textbf{LaTeX Formatting} \hfill \textit{Feb'21 - Feb'22}\\
    Projects were assigned in overleaf along with the content. Worked on content research and development. Developed creative content for the company's various activities.
    \item \textbf{TATA Sikorsky} \hfill \textit{Jun'19-Jul'19}\\
    In this internship, I aimed to gain an understanding of the production methods of aircraft fuselage parts. I was assigned to a project which required me to design an ideal trolley that could hold most of the produced parts and be ergonomically viable such that the transportation of manufactured parts from one shop floor to another would be efficient and hence increase the productivity of the factory. 
    \item \textbf{Drone Development Internship} \hfill \textit{Jun'18}\\
    Training covered details on Drone industry, UAV systems, Kinetics and dynamics, Autopilot Mechanisms, Calibration parameters, Ardupilot, Mission planner, Radio controllers and other allied softwares. Exposed to the Indian Drone Industry practices and went training in the form of both theoretical and practical work.
\end{itemize}

\resheading{PUBLICATIONS}
\begin{itemize}
    \item \textbf{\textit{Title:} Effect of the Leading-Edge Sweep on Wrap-Around Fins} \hfill \textit{2022}\\
	\textit{Authors:} \textbf{K Madhulaalasa}, P Shishir, P Venkata Sai Prasad, P K Mohanta, Swapnil Sapkale\\
	\textit{Publication:} INCAS BULLETIN, Volume 14, Issue 1/ 2022, pp. 69-78. \\DOI: 10.13111/2066-8201.2022.14.1.6
    \item \textbf{\textit{Title:} Conceptual Design of Multi-Cambered Bioinspired Morphing Wing} \hfill \textit{2021}\\
	\textit{Authors:} \textbf{K Madhulaalasa}, D Divya Govind, M Manoj Kumar, Y D Dwivedi\\
	\textit{Publication:} {International Journal of Mechatronics and Manufacturing Technology, Volume 6, Issue 1}
\end{itemize}

\resheading{RELEVANT COURSES}
\vspace{5mm}
\setlength{\tabcolsep}{15pt}
\renewcommand{\arraystretch}{1.2}
\noindent
\begin{tabular}{|m{8.25cm}|m{9.5cm}|}
\hline
\rowcolor{lightgray} \multicolumn{2}{|c|}{M.Tech Courses}\\
\hline
\hline
Aerospace Structural Analysis-I &  Aerospace Structural Analysis-II \\
\hline
Mathematics for Aerospace Engineering & Introduction to Finite Element Methods\\
\hline
Dynamics and Vibration & Composite Materials\\
\hline
Helicopter Theory: Dynamics and Aeroelasticity & Instrumentation, Measurements and experiments in Fluids \\
\hline
\hline
\rowcolor{lightgray} \multicolumn{2}{|c|}{B.Tech Courses}\\
\hline
\hline
Aerospace Structures & Aircraft Material \& Production\\
\hline
Low \& High Speed Aerodynamics & Aircraft \& Space Propulsion\\
\hline
Computational Aerodynamics & Heat Transfer\\
\hline
Aircraft Performance & Flight Stability \& Control \\
\hline
Air Transportation Systems & Airport Operations\\
\hline
Aircraft Navigation Systems & Flight Vechile Design\\
\hline
\end{tabular}

\resheading{CERTIFICATION COURSES}
\begin{itemize}
\setlength\itemsep{-0.2em}
    \item \textbf{Industry- Institute Interaction in Modern Aerospace Engineering: Marching towards Self-Reliance} \hfill \textit{Online}\\
    \textit{Outcomes of course:} Present developments in aerospace engineering.
    \item \textbf{MATLAB Programming} \hfill \textit{Central Institute of Tool Designing, Hyderabad}\\
    \textit{Outcomes of course:} Reinforce a structured, top-down approach to formulate and solve problems, Apply a variety of common numeric techniques to solve and visualize engineering-related computational problems.
    \item \textbf{Solidworks} \hfill \textit{Central Institute of Tool Designing, Hyderabad}\\
    \textit{Outcomes of course:} Sketcher, Part modelling, Assembly. Drafting module.
    \item \textbf{Siemens NX} \hfill \textit{Central Institute of Tool Designing, Hyderabad}\\
    \textit{Outcomes of course:} Sketcher, Part modelling, Surface modelling, Assembly. Drafting module.
    \item \textbf{SQL Introduction Course 2022: SQL Crash Course.} \hfill \textit{Udemy}\\
    \textit{Outcomes of course:} Using PostgreSQL \& applicable to Oracle SQL, Microsoft SQL Server, and MySQL for Data, Apps and Web development
    \item \textbf{Practical Git \& Github Bootcamp for Developers} \hfill \textit{Udemy}\\
    \textit{Outcomes of course:} Git commands from Git bash \& Terminal, Visual  Studio Code editor, Intellij editor
    \item \textbf{Python for Machine Learning: The Complete Beginner's Course} \hfill \textit{Udemy}\\
    \textit{Outcomes of course:} Python programming and Scikit learn applied to machine learning regression, linear and logistic regression.
    \item \textbf{Python for Machine Learning with Numpy, Pandas \& Matplotlib} \hfill \textit{Udemy}\\
    \textit{Outcomes of course:} Programming with Python, NumPy with Python, Using pandas Data Frames to solve complex tasks, Use pandas to handle Excel Files, Visualize Data using Matplotlib and Seaborn
    \item \textbf{Aerospace Masterclass: Aircraft Design} \hfill \textit{Udemy}\\
    \textit{Outcomes of course:} Aircraft Design, Aircraft Types and Missions, Apply Statistics to Engineering, Estimating Aircraft Masses, Design and Positioning of Vehicle Components, Solving Aircraft Balancing Characteristics.
\end{itemize}

\resheading{SELF PROJECTS}
\begin{itemize}
\setlength\itemsep{-0.2em}
    \item \textbf{Chat Bot using Python \& Deep Learning}\\
    Building a computer program that simulates human conversation through voice commands or text chats or both.
    \item \textbf{Smart calculator using Python}\\
    Building a calculation which doesn't need to add symbols just give numbers and ask for any operation.
    \item \textbf{Fashion Product Classifier}\\
    A deep learning classification model which classifies the category of a fashion product.
\end{itemize}

\resheading{SKILLS}
\begin{itemize}
\setlength\itemsep{-0.2em}
    \item \textbf{Programming:} Python, LaTeX, MATLAB, SQL
    \item \textbf{Designing Software:} Solidworks, Siemens NX (Unigraphics), Autocad, CatiaV5
    \item \textbf{Analysis:} Ansys (Workbench, Mechanical APDL, FLUENT), Pointwise, ABAQUS, FEM
    \item \textbf{Others:} MS Office, Github
    \item \textbf{Languages:} English, Telugu, Hindi
\end{itemize}

\resheading{WORKSHOPS \& COMPETITIONS}
\begin{itemize}
\setlength\itemsep{-0.2em}
    \item Attended Online Workshop on \textbf{High Performance Computing in Engineering} Organized by Indian Institute of Technology Kharagpur and Ansys under the aegis of the National Supercomputing Mission October 21-22, 2021.
    \item Participated in \textbf{Pro-Simulator} competition during Tech Fest Consortium 2019 conducted by Institute of Aeronautical Engineering.
    \item Participated in \textbf{Engineering Drawing} \& \textbf{Engineering Design} competitions conducted by SAE Collegiate Club under SAE Student Convection Tier-1 Events 2018-19.
    \item Participated in \textbf{Avion-E (Drone)} competition conducted by National Institute of Technology, Warangal during Technozion'18.
    \item Attended the workshop on \textbf{Astro Engineering} organized by VNR Vignana Jyothi Institute of Engineering and Technology during Convergence (reloaded) 2018.
    \item Attended the workshop on \textbf{Aeromodelling} on 28$th$-29$th$ October at \textbf{ATMOS 2017}, the National Management Festival of BITS PILANI hyderabad Campus.
\end{itemize}

\resheading{POSITIONS OF RESPONSIBILITY}
\begin{itemize}
\setlength\itemsep{-0.5em}
\item \textbf{PoR | Tutoring Bharatanatyam}\hfill {[Dec'18-Present]} 
\begin{itemize}[noitemsep,nolistsep]
    \item Running a dance institute named \textbf{Veda Sarawati Kala Mandalam} in Hyderabad
    \item Tutoring students in the Bharatanatyam art form and involving them to take Bangiya Sangeet Parishad examination every year.
\end{itemize}

\vspace{1mm}
\item \textbf{PoR | Class Representative}\hfill {[Jul'19-May’21]} 
\begin{itemize}[noitemsep,nolistsep]
    \item Represent the class for HoD \& Principal meetings
    \item Managed class database of students
\end{itemize}

\vspace{1mm}
\item \textbf{PoR | Tech fest Organizer}\hfill {[Sep'19]} 
\begin{itemize}[noitemsep,nolistsep]
    \item Maintained and managed department events as per the scheduled database.
    \item Coordinated the participants in registering for the event.
\end{itemize}
 
\vspace{1mm}
\item \textbf{PoR | Coordinator for SAE Student Convection}\hfill {[Sep'19]} 
\begin{itemize}[noitemsep,nolistsep]
    \item Coordinated the attendees to sign up for the event.
    \item Maintained the database of the winners of every event.
\end{itemize}
\end{itemize}

\resheading{\textbf{LINKS}}
\vspace{2mm}
\indent
\faGithub\ Github: \href{https://github.com/Madhulaalasa?tab=repositories}{ML Cloud}\\
\indent
\faLinkedin\ LinkedIn: \href{https://www.linkedin.com/in/madhulaalasa-k-b22a71b5/}{K Madhulaalasa}

\vspace{1mm}
\resheading{\textbf{EXTRACURRICULAR ACTIVITIES}}

\vspace{2mm}
\indent
Dancing, Singing, Painting, Volunteering for Social service campaigns
\end{document}