\documentclass[11pt, openany]{book}
\usepackage[text={4.65in,7.45in}, centering, includefoot]{geometry}
\usepackage[table, x11names]{xcolor}
\usepackage{fontspec,realscripts}
\usepackage{polyglossia}
\setdefaultlanguage{sanskrit}
\setotherlanguage{english}
\setmainfont[Scale=1]{Shobhika}
% \defaultfontfeatures[Scale=MatchUppercase]{Ligatures=TeX} 
% \newfontfamily\sanskritfont[Script=Devanagari,Mapping=devanagarinumerals]{Shobhika}
\newfontfamily\s[Script=Devanagari, Scale=0.8]{Shobhika}
\newfontfamily\regular{Linux Libertine O}
\newfontfamily\en[Language=English, Script=Latin]{Linux Libertine O}
\newfontfamily\na[Script=Devanagari, Scale=1.1, Color=purple]{Shobhika-Bold}
\newfontfamily\qt[Script=Devanagari, Scale=1, Color=violet]{Shobhika-Regular}
\newcommand{\devanagarinumeral}[1]{%
	\devanagaridigits{\number \csname c@#1\endcsname}} % for devanagari page numbers
%\usepackage[Devanagari, Latin]{ucharclasses}
%\setTransitionTo{Devanagari}{\s}
%\setTransitionFrom{Devanagari}{\regular}
\XeTeXgenerateactualtext=1 % for searchable pdf
\usepackage{enumerate}
\pagestyle{plain}
\usepackage{fancyhdr}
\pagestyle{fancy}
\renewcommand{\headrulewidth}{0pt}
\usepackage{afterpage}
\usepackage{multirow}
\usepackage{multicol}
\usepackage{mdframed,lipsum}
\usepackage{wrapfig}
\usepackage{vwcol}
\usepackage{microtype}
 \usepackage{amsmath,amsthm, amsfonts,amssymb}
\usepackage{mathtools}% <\textendash\ new package for rcases
\usepackage{graphicx}
\usepackage{longtable}
\usepackage{setspace}
\usepackage{footnote}
\usepackage{perpage}
\MakePerPage{footnote}
%\usepackage[para]{footmisc}
%\usepackage{dblfnote}
\usepackage{xspace}
\usepackage{array}
\usepackage{emptypage}
\usepackage{hyperref}% Package for hyperlinks
\hypersetup{colorlinks,
citecolor=black,
filecolor=black,
linkcolor=blue,
urlcolor=black}

\newcommand\blfootnote[1]{%
 \begingroup
 \renewcommand\thefootnote{}\footnote{#1}%
 \addtocounter{footnote}{-1}%
 \endgroup
}

\mdfdefinestyle{MyFrame}{%
 linecolor=black,
 outerlinewidth=2pt,
 %roundcorner=20pt,
 innertopmargin=4pt,
 innerbottommargin=4pt,
 innerrightmargin=4pt,
 innerleftmargin=4pt,
 leftmargin = 4pt,
 rightmargin = 4pt
 %backgroundcolor=gray!50!white}
 }
 
\begin{document}

\cfoot{}
\fancyhead[CE]{नाट्यशास्त्रम्}
\fancyhead[CO]{विंशोऽध्यायः}
\fancyhead[LE,RO]{\thepage}
\renewcommand{\thepage}{\devanagarinumeral{page}}
\setcounter{page}{91}

% विंशोऽध्यायः ९१

\begin{quote}
{\na या वाक्प्रधाना \renewcommand{\thefootnote}{1}\footnote{भ \textendash\ नृवर}पुरुषप्रयोज्या\\
स्त्रीवर्जिता संस्कृतपाठ्ययुक्ता~।\\
स्वनामधेयैर्भरतैः प्रयुक्ता\\
सा भारती नाम भवेत्तु वृत्तिः\renewcommand{\thefootnote}{2}\footnote{ट \textendash\ तां भारतीं वृत्तिमुदाहरन्ति}~॥~२६

भेदास्तस्यास्तु विज्ञेयाश्चत्वारोऽङ्गत्वमागताः~।}
\end{quote}

\hrule

\vspace{2mm}
अथासां परस्परसङ्कीर्णतया लक्ष्ये (बहुरूपतां) वहन्तीनां यत्र यत्प्राधान्येनान्यतमरूपावभासनं तत्प्रदर्शयितुमुत्तरो ग्रन्थः~। (वाङ्मःनकायचेष्टांशेषु) न ह्येकोऽपि कश्चिच्चेष्टांशोऽस्ति~। कायचेष्टा अपि हि मानसीभिः सूक्ष्माभिश्च वाचिकीभिश्चेष्टाभिर्व्याप्यन्त एव~। {\qt न सोऽस्ति प्रत्ययो लोके यः शब्दानुगमादृते ?} (वाक्यपदीयं १\textendash\ १२४) इति न्यायात् , मानस्यपि वाचिक्यपि\renewcommand{\thefootnote}{1}\footnote{वाचिक्येवापि च, वाचिक्येनापि} चेष्टा अवश्यं सूक्ष्मं काल(य?)परिस्पन्दमन्दप्राणव्यापाररूपं नाभि(ति?)वर्तते~। तदुक्तम्\textendash\ 

\begin{quote}
{\qt अर्थक्रियासु वाक् सर्वान् समीहयति देहिनः\renewcommand{\thefootnote}{2}\footnote{वेगिनम्~।}~।\\
तदुत्क्रान्तौ विसंज्ञोऽयं दृश्यते काष्ठकुड्यवत्~॥} (सङ्ग्रहः\renewcommand{\thefootnote}{*}\footnote{अयं श्लोको भगवता भर्तृहरिणा वाक्यपदीये स्ववृत्तौ (१\textendash\ १२७) उदाहृतः, व्याडिसंग्रहात् स्यात्~।})
\end{quote}

\noindent
\renewcommand{\thefootnote}{3}\footnote{नव}न च रसोपयोगिलालित्यभागशून्यः कोऽपि नाट्ये परिस्पन्द इत्यन्योन्यं संवलिता वृत्तयः केवलं क्वचित्किंचिदधिकमिति प्राधान्येन व्यपदेशः परिवर्तते~।\\

तत्र भारत्याः प्राधान्यं दर्शयति \underline{या वाक्प्रधा}नेति~। \underline{स्त्रीवर्जितेति} कैशिकीप्राधान्यावकाशं गमयति~। \underline{संस्कृतेति} वचसा प्राकृतपाठ्यलालित्यात् कैशिकीमवश्यमाक्षिपेदिति सूचयति~। \underline{भरतैरिति} नटैः स्वतो वंशकरं नामधेयं येषां (तैः), भरतसंतानत्वात्तद्धिते भरताः~। \underline{भेदा} इति \renewcommand{\thefootnote}{4}\footnote{आसां}अस्यामित्यर्थोऽत्र न तु प्रकाराः, त्रैलोक्यव्यापिन्या हि भारत्याः कश्चिदंशः प्ररोचनारूपः, ऐवमामुखस्वभाव इत्यादि~। अत एवाह \underline{अङ्गत्वमिति} अंशत्वं प्राप्ता इत्यर्थः,

\newpage
% ९२ नाट्यशास्त्रम् 

\begin{quote}
{\na प्ररोचनामुखं चैव वीथी प्रहसनं तथा~।~२७

जयाभ्युदयिनी चैव मङ्गल्या विजयावहा~।\\
सर्वपापप्रशमनी पूर्वरङ्गे प्ररोचना~॥~२८

\renewcommand{\thefootnote}{1}\footnote{अयं ढ \textendash\ मातृकायामेव वर्तते}[उपक्षेपेण काव्यस्य हेतुयुक्तिसमाश्रया~।\\
सिद्धेनामन्त्रणा या तु विज्ञेया सा प्ररोचना~॥]~२९

नटी विदूषको वापि\renewcommand{\thefootnote}{2}\footnote{ढ \textendash\ चापि} पारिपार्श्चिक एव वा\renewcommand{\thefootnote}{3}\footnote{ढ \textendash\ च}~।\\
सूत्रधारेण सहिताः संलापं \renewcommand{\thefootnote}{4}\footnote{ढ \textendash\ यत्र}यत्तु कुर्वते~।~३०

चित्रैर्वाक्यैः स्वकार्योत्थै\renewcommand{\thefootnote}{5}\footnote{न \textendash\ काव्योत्थैः ढ \textendash\ वाक्यैश्च कार्योत्थैः} र्वीर्थ्यङ्गैरन्यथापि वा\renewcommand{\thefootnote}{6}\footnote{न \textendash\ च}~।}
\end{quote}

\hrule

\vspace{2mm}
\noindent
अन्यथा यदि रूपकस्याङ्गत्वं प्राप्ता इत्युच्यते तदा वीथी प्रहसनं च रूपकभेदः, न तु रूपकस्याङ्गम्~। तथा नाट्यस्याङ्गत्वं प्राप्तं(प्राप्ता?) नाट्यमिति समुदायः~। यदि वा ऐकैकमपि काव्यं दशरूपमित्युक्तं प्राक्~। \underline{पूर्वरङ्ग} इति तद्विषये, \underline{प्ररोचना} पूर्वमुक्ता तत्र च कृते या प्ररोचना सा भारत्यंश इत्यर्थः~।\\

\underline{नटी}त्यादिना आमुखं लक्षयति~। \underline{वा} शब्देन व्यस्तानां नटीप्रभृतीनां सूत्रधारेण सङ्घातमाह~। \underline{अपि}शब्देनात्मना समस्तानां, द्वितीयो \underline{वा} शब्दः समस्तव्यस्ततां विकल्पयति~। \underline{एव}शब्दः सूत्रधारस्यावश्यंभावं दर्शयति~। \underline{चित्रै}रिति भाविरूपकार्थानुकूलविषयानुसारिभिः, स्वं कार्यं नटव्यापारं, वीथ्यङ्गैरिति श्लिष्टवक्रोक्तिप्रत्युक्तिप्रायैरित्यर्थः, यथा {\qt पीताम्बरगुरुशक्त्या\renewcommand{\thefootnote}{1}\footnote{शर्मा} हरत्युषां प्रसभमनिरुद्धः}\renewcommand{\thefootnote}{*}\footnote{प्रतिमानिरुद्धे.} इत्यादि~। \underline{अन्यथेति} स्पष्टोक्तिप्रत्यु\textendash

\newpage
% विंशोऽध्यायः ९३ 

\begin{quote}
{\na आमुखं तत्तु विज्ञेयं बुधैः\renewcommand{\thefootnote}{1}\footnote{न \textendash\ तज्ज्ञैः}प्रस्तावनापि वा~॥~३१

\renewcommand{\thefootnote}{2}\footnote{भ \textendash\ मातृकायामयं न वर्तते, श्लोकार्धमेव टनडमातृकासु न विद्यते}[लक्षणं पूर्वमुक्तं तु वीथ्याः प्रहसनस्य च~।\\
आमुखाङ्गान्यतो व वक्ष्ये यथावदनुपूर्वशः]~३२

उद्धात्यकः कथोद्धातः प्रयोगातिशयस्तथा~।\\
प्रवृत्तकावरलगिते पञ्चाङ्गान्यामुखख्य तु\renewcommand{\thefootnote}{3}\footnote{च \textendash\ आमुखाङ्गानि पञ्च वै}~।~३३}
\end{quote}

\hrule

\vspace{2mm}
\noindent
क्तिभिः, यथा नागानन्दे {\qt नाटयितव्ये किमित्यकारणमेव रुद्यते} इत्यादि~। \underline{आमुख} मिति मुखसन्धेर्निवर्तते यतः, आदङ्मर्यादायाम् , यदि वात्रामुखं प्रारम्भमीषन्मुखं वा प्रस्ताव्यतेऽनयेति(प्रस्तावना) बाहुलकेन तच्छीलसंज्ञयोरिति \ldots \ldots \ldots \ldots \ldots निकारो न भवति~। तत्र कदाचित्कार्याभिमुखं नीयते पूर्वरङ्गविधिः, तदभिमुखं वा कार्यारम्भः तन्नीयते, सा द्विधेति (५ \textendash\ १८०) पूर्वरङ्गाध्याये दर्शितमस्माभिः~। एवं च यदा स्थापकोऽपि सूत्रधारतुल्यगुणाकारो रामादिवदेव प्रयुज्यते तदेदं कविकृतमामुखं भवति~।\\

अन्ये त्वाहुः\textendash\ पूर्वरङ्गाध्यायेऽपि या प्रस्तावनोक्ता सापि भारतीभेद ऐवेति, किं द्वैविध्याभिधानप्रयासेन~। यत्तु {\qt क्रुद्धो भीम इत ऐवाभिवर्तते तन्न युक्तमस्य पुरतोऽवस्थातुं} इत्यादि, तत्तदीयसत्त्वावेशादेवास्य प्रकाशनाय सामाजिकानां साक्षात्कारकल्पाध्यवसायसम्पत्त्यर्थं सूत्रधारेणोच्यते, शिशुसन्त्रासाय कयाचिदाकृत्या कश्चिदत्रस्यन्नपि त्राससंरम्भगर्भमाह {\qt अयमागतो राक्षसः} इति, यथोक्तं भावाध्याये (७) {\qt सत्त्वशुद्धा मूर्तव्या:}\renewcommand{\thefootnote}{*}\footnote{सप्तमेऽध्याये १४७ वचनभागे {\qt सत्त्वविशुद्धाः कार्या यथास्वरूपा भवन्ति} इति भागस्य व्याख्यानावसरे~। तद्व्याख्याभागो नष्टः~।} इति, एतच्च तत्रैव निर्णीतम्~। तस्यामुखस्य पञ्चाङ्गानि भेदा इत्यर्थः~।

\newpage
% ९४ नाट्यशास्त्रम् 

\begin{quote}
{\na उद्घात्यकावलगितलक्षणं कथितं मया\renewcommand{\thefootnote}{1}\footnote{भ \textendash\ लगिते वीथ्यां संपरिभाषिते (म \textendash\ कीर्तिते) ढ \textendash\ लगिते वीथ्यां चैव प्रकीर्तिते}~।\\
शेषाणां लक्षणं विप्रा\renewcommand{\thefootnote}{2}\footnote{न अहं} व्याख्यास्याम्यनुपूर्वशः~॥

सूत्रधारस्य वाक्यं वा यत्र वाक्यार्थमेव वा~।\\
गृहीत्वा प्रविशेत्पात्रं\renewcommand{\thefootnote}{3}\footnote{भ \textendash\ यत्र} कथोद्धातः स कीर्तितः\renewcommand{\thefootnote}{4}\footnote{भ\textendash\ प्रकीर्तितः}~॥~३५}
\end{quote}

\hrule

\vspace{2mm}
ननु ऐवं सति पञ्चानां युगपत्प्रयोगे प्रस्तावना प्रयुक्ता स्यात्, न चैतन्मुनेरभिमतम्~। तथा हि वक्ष्यति\textendash

\begin{quote}
{\qt एषामन्यतमं श्लिष्टं योजयित्वार्थयुक्तिभिः~।\\
पात्रग्रन्थैरसम्बाधं प्रकुर्यादामुखं ततः~॥} (२०\textendash\ ३७)
\end{quote}

\noindent
इत्युद्देशे~। यद्यपि पश्चादवलगितमुक्तं तथा तयोस्तुल्यं, लक्षणस्य पूर्वोक्तत्वमित्याशयेनाह \underline{उद्धात्यकावलगितयोर्लक्षणं कथित}मिति वीथ्यङ्गव्याख्याने, यद्यपि च प्रस्तावनायामन्यान्यपि वीथ्यङ्गानि भवन्ति, आमुखसामान्यलक्षणेऽप्युक्तं वीथ्यङ्गैरिति, तथाप्युद्धात्यकमवलगितं च भाविकाव्यार्थप्रस्तावनेति बालरङ्गं(प्रस्तावने प्रबलमङ्गं?), तथा तयोर्लक्षणम्\textendash

\begin{quote}
{\qt पदानि त्वगतार्थानि ये नराः पुनरादरात्~।\\
योजयन्ति पदैरन्यैस्तदुद्धत्यकमुच्यते~॥

यत्रान्यस्मिन् समावेश्य कार्यमन्यत्प्रसाध्यते~।\\
तच्चावलगितं~॥} इति (१८\textendash\ ११६)~। 
\end{quote}

\begin{sloppypar}
वाक्यमिति, यथा\textendash\ (रत्नावल्यां) {\qt द्वीपादन्यस्मात्} इति~। वाक्यार्थे यथा प्रतिमानिरुद्धे\textendash\ {\qt पीताम्बरगुरुशक्त्या\renewcommand{\thefootnote}{1}\footnote{शर्मा} हरत्युषां} इति~। केवलमात्र वीथ्यङ्गनिबद्धम्~। \underline{कथा} काव्यर्थरूपा, ऊर्ध्वमेव हन्यते गम्यते तत्रेति कथोद्धातः~।
\end{sloppypar}

\newpage
% विंशोऽध्यायः ९५ 

\begin{quote}
{\na प्रयोगे तु प्रयोगं तु\renewcommand{\thefootnote}{1}\footnote{न \textendash\ यत् भ \textendash\ वा} सूत्रधारः प्रयोजयेत्\renewcommand{\thefootnote}{2}\footnote{प \textendash\ सूत्रभृद्यत्र योजयेत्}~।\\
ततश्च प्रविशेत्पात्रं प्रयोगातिशयो हि सः~॥~३६

\renewcommand{\thefootnote}{3}\footnote{न \textendash\ कालं प्रवृत्तं, भ \textendash\ प्रवृत्तकालं, ट \textendash\ प्रवृत्तं कार्ये}कालप्रवृत्तिमाश्रित्य वर्णना या प्रयुज्यते\renewcommand{\thefootnote}{4}\footnote{ढ \textendash\ यत्र चैवोपवर्णयेत्, न \textendash\ सूत्रभृद्यत्र वर्णयेत्}~।\\
तदाश्रयाच्च\renewcommand{\thefootnote}{5}\footnote{ढ \textendash\ तदाश्रयस्य, म \textendash\ तदाश्रयश्च} पात्रस्य प्रवेशस्तत्प्रवृत्तकम्~।~३७

\renewcommand{\thefootnote}{6}\footnote{भ \textendash\ कालमेषामन्यतमं}एषामन्यतमं श्लिष्टं\renewcommand{\thefootnote}{7}\footnote{न द्वेधा य \textendash\ वेधाः} योजयित्वार्थयुक्तिभिः\renewcommand{\thefootnote}{8}\footnote{न \textendash\ युक्तितः}~।\\
\renewcommand{\thefootnote}{9}\footnote{इदं श्लोकार्धं ढ \textendash\ मातृकायामेव}[तस्मादङ्गद्वयस्यापि सम्भवो न निवार्यते~॥]~३८}
\end{quote}

\hrule

\vspace{5mm}
\noindent
\underline{प्रयोग} इति प्रस्तावनात्मके, \underline{प्रयोग}मिति नाट्यात्मकं भावितम्~। एकस्तुशब्दो भेदान्तरेभ्यो व्यतिरेकमाह, द्वितीयोऽवधारणे~। सूत्रधार एव यत्र प्रयोगे प्रयोगं समुद्गककवाटयुगलवद्योजयति स प्रयोगद्वयश्लेषणात्प्रयोगातिशयः~। यथा विक्रमोर्वश्याम्\textendash\ अथ कुररीणामिवाकाशे शब्दः श्रूयते~। आः ज्ञातम्\textendash

\begin{quote}
{\qt ऊरूद्भवा नरसखस्य मुनेः सुरस्त्री\\
कैलासनाथमुपनृत्य निवर्तमाना~।\\
बन्दीकृता विबुधवैरिभिरर्धमार्गे\\
क्रन्दत्यतः करुणमप्सरसां गणोऽयम्~॥} इति~।
\end{quote}

\noindent
यदा\renewcommand{\thefootnote}{1}\footnote{यथा} \underline{कालप्रवृत्तिं} कांचिदवलम्ब्य यथा सूत्रधारेण किञ्चिद्वस्तु वर्ण्यते तदा [प्र]श्रयेण च पात्रस्य प्रवेशः तत्कालप्रवृत्त्या स्वार्थोक्तत्वात् प्रवृत्तकम्, यथा {\qt अस्यां शरदि\textendash }

\newpage
% ९६ नाव्यशास्त्रम् 

\begin{quote}
{\na \renewcommand{\thefootnote}{1}\footnote{न \textendash\ अल्पग्रन्थैः}पात्रग्रन्थैरसंबाधं प्रकुर्यादामुखं ततः\renewcommand{\thefootnote}{2}\footnote{च \textendash\ बुधः}~।\\
एवमेतद्बुेधैर्ज्ञेयमामुखं विविधाश्रयम्~॥~३९

\renewcommand{\thefootnote}{3}\footnote{इदं श्लोकार्धं केषुचिदादर्शेषु पूर्वमेव पठितम्}लक्षणं पूर्वमुक्तं तु वीथ्याः\renewcommand{\thefootnote}{4}\footnote{भ\textendash\ मातृकायां वीथ्यङ्गान्यत्रैव पठितानि न तु वीथीलक्षणावसरे} प्रहसनस्य च~॥

\renewcommand{\thefootnote}{5}\footnote{अयं श्लोको जादिमान्तेषु दृश्यते}[इत्यष्टार्धविकल्पा वृत्तिरियं भारती मयाभिहिता\renewcommand{\thefootnote}{6}\footnote{भ \textendash\ प्रोक्ता}~।\\
\renewcommand{\thefootnote}{7}\footnote{भ \textendash\ सात्त्वत्या अपि लक्षणमतः परं संप्रवक्ष्यामि}सात्त्वत्यास्तु विधानं लक्षणयुक्त्या प्रवक्ष्यामि~॥]~४०

\renewcommand{\thefootnote}{8}\footnote{च\textendash\ मातृकायमयमपि न विद्यते}या \renewcommand{\thefootnote}{9}\footnote{भ \textendash\ सात्त्विकेन}सात्त्वतेनेह गुणेन युक्ता\\
\renewcommand{\thefootnote}{10}\footnote{न \textendash\ त्यागेन शौर्येण भ \textendash\ त्यागेन वृत्तेन च सन्धिता या}न्यायेन वृत्तेन समन्विता च\renewcommand{\thefootnote}{11}\footnote{न \textendash\ या}~।\\
\renewcommand{\thefootnote}{12}\footnote{भ \textendash\ हर्षोत्तरा}हर्षोत्कटा \renewcommand{\thefootnote}{13}\footnote{ढ \textendash\ संभृत}संहृतशोकभावा\\
सा सात्त्वती नाम भवेत्तु\renewcommand{\thefootnote}{14}\footnote{प \textendash\ सत्त्ववतीह} वृत्तिः~॥~४१}
\end{quote}

\hrule

\vspace{2mm}
{\qt सत्पक्षा मधुरगिरः प्रसाधिताशा मदोद्धतारम्भाः~।} (वेणी १)\\

इत्यादि \underline{पात्रग्रन्थैरसम्बाध}मिति यत्र न भूयांसि पात्राणि अल्पपात्रेऽपि ग्रन्थबहुलत्वं तथारूपमामुखं कुर्यात्~। \underline{विविधाश्रय}मिति बहुभेदमित्यर्थः~। \underline{पूर्वमुक्त}मिति दशरूपकाध्याये~।\\

यद्यपि अभेदव्यतिरेकेणापि भारती(तो?) (न्यायो ) न दृश्यते तथापि \underline{न्यायेनेति} तत्प्रकारचतुष्कं निरूपितम् (१० \textendash\ ७२)~। सात्त्वतो गुणः मानसो व्यापारः~। \underline{सत्सत्त्वं प्रकाशः} तद्विद्यते यत्र तत्सत्त्वं मनः, तस्मिन् भवः~।

\newpage
% विंशोऽध्यायः ९७

\begin{quote}
{\na वागङ्गाभिनयवती सत्त्वोत्थान\renewcommand{\thefootnote}{1}\footnote{भ \textendash\ विविधवाक्यकरणेषु च \textendash\ वचनभ्रु}वचनप्रकरणेषु~।\\
सत्त्वाधिकारयुक्ता विज्ञेया सात्त्वती वृत्तिः\renewcommand{\thefootnote}{2}\footnote{न \textendash\ नाम}~॥~४२

वीराद्भुतरौद्ररसा\renewcommand{\thefootnote}{3}\footnote{ज\textendash\ प्रायरसा} निरस्तशृङ्गारकरुणनिर्वदा\renewcommand{\thefootnote}{4}\footnote{भ \textendash\ विज्ञेया}~।\\
उद्धतपुरुषप्राया मा परस्पराधर्षणकृता च~॥~४३

\renewcommand{\thefootnote}{5}\footnote{भ \textendash\ उत्थापनं च}उत्थापकश्च परिवर्तकश्च सल्लापकश्च संघात्यः~।}
\end{quote}

\hrule

\vspace{2mm}
\noindent
\underline{सत्त्वोत्थानस्य} सत्त्वाधारस्य, \underline{वचनं} येषु \underline{प्रकरणेषु} काव्यखण्डेषु, तेषु वागङ्गाभिनययुक्ता सती, सत्त्वस्य सात्त्विकाभिनय\underline{स्याधिकारे} आधिक्यक्रियया सात्त्वतीवृत्तिर्युक्ता भवतीति सम्बन्धः~।\\

शृङ्गारे विषयानिमग्रं मनः,करुणे कान्दिशीकं, निर्वेदे मूढमिति तद्व्यापारो भवन्नपि क्रोधविस्मयोत्साहेष्विव न सातिशयं परिस्फुरतीति दर्शयति \underline{वीराद्भुतरौद्र}\textendash\ रसेति~। आधर्षणं वाचा न्यक्कारः~। उत्थापयति यो मानसः परिस्पन्दः स तावदुत्थापकः तत्सूचको \renewcommand{\thefootnote}{1}\footnote{व्याप्यक्रमः}व्यापारक्रम उपचारः, तथोक्तः~। तथा वेणीसंहारे भीमः\textendash\ भो भोः शृण्वन्तु भवन्तः\textendash

\begin{quote}
{\qt स्पृष्टा येन शिरोरुहेषु पशुना पाञ्चालराजात्मजा\\
येनास्याः परिधानमप्यपहृतं राज्ञां कुरूणां पुरः~।\\
यस्योरःस्थलशोणितासवमहं पातुं प्रतिज्ञातवान्\\
सोऽयं मद्भुजपञ्जरे निपतितः संरक्ष्यतां कौरवाः~॥} इति~।
\end{quote}

\noindent
परिवर्तको यथा तत्रैव भीमः\textendash\ सहदेव गच्छ त्वं गुरुमनुवर्तस्व~। अहमप्यस्त्रागारं प्रविश्यायुधसहायो भवामि~।\\

सहदेवः\textendash\ आर्थ, नेदमायुधागारम् , पाञ्चाल्याश्चतुश्शालमिदम्~। \\

भीमः\textendash\ किं नामेदं (इत्यादि यावत्) अथवा मन्त्रयितव्यैव मयापाञ्चाली~। इति

\lfoot{13}

\newpage
\lfoot{}
% ९८ नाट्यशास्त्रम्

\begin{quote}
{\na चत्वारोऽस्या भेदा विज्ञेया नाट्यतत्त्वज्ञैः~॥~४४

अहमप्युत्थास्यामि त्वं तावद्दर्शयात्मनः शक्तिम्~।\\
इति \renewcommand{\thefootnote}{1}\footnote{च \textendash\ संघर्षसमाश्रयमुत्थितं भ \textendash\ संहरणसमुत्थं तज्झैरुत्थापनं ज्ञेयम्}संघर्षसमुत्थस्तज्ज्ञैरुत्थापको ज्ञेयः~॥~४५

उत्थानसमारब्धानर्थानुत्सृज्य योऽर्थयोगवशात्\renewcommand{\thefootnote}{2}\footnote{भ \textendash\ संयोगात्}~।\\
अन्यानर्थान् भजते स चापि परिवर्तको ज्ञेयः~॥

[\renewcommand{\thefootnote}{3}\footnote{अयं श्लोको य मढादिष्वेव वर्तते}निर्दिष्टवस्तुविषयः प्रपञ्चबद्धस्त्रिहस्यसंयुक्त:~।\\
\renewcommand{\thefootnote}{4}\footnote{ढ \textendash\ संहर्ष}संघर्षविशेषकृतस्त्रिविधः परिवर्तको ज्ञेयः]~॥~४७

\renewcommand{\thefootnote}{5}\footnote{ढ \textendash\ सामर्षजो निरमर्षजोऽपि वा विविधवचन}साधर्षजो निराधर्षजोऽपि वा \renewcommand{\thefootnote}{6}\footnote{च \textendash\ विविध}रागवचनसंयुक्तः~।\\
\renewcommand{\thefootnote}{7}\footnote{ढ \textendash\ साविच्छेदालापः}साधिक्षेपालापो ज्ञेयः सल्लापकः सोऽपि~॥~४८}
\end{quote}

\hrule

\vspace{2mm}
अस्त्रागारप्रवेशपरित्यागेन पाञ्चालीदर्शनात्मककार्यान्तरसम्पादको मानसो व्यापारः परिरवर्तयति कार्यमिति, परिवर्तकवचनं तूपचारारूढम्\renewcommand{\thefootnote}{1}\footnote{रढः}~। एवमुत्तरत्रापि~।\\

सह (आ)धर्षणेन यद्वाक्यं (साधर्षं) तद्विरहितं निराधर्षे, तेन खलीकारकाद्वचनादन्यतोऽपि वा सत्, अनन्तरमधिक्षेपं वचनमभिभावकं\renewcommand{\thefootnote}{2}\footnote{आविर्भाविकं} मानसं कर्म तत्सल्लापकशब्दवाच्यम्~। यथा\textendash

\begin{quote}
{\qt अश्वत्थामा हत इति पृथासूनुना स्पष्टमुक्त्वा\\
स्वैरं शेषे गज इति किल व्याहृतं सत्यवाचा~।} (वेणी ३\textendash\ ११)
\end{quote}

\noindent
इत्यत्न सत्यवाचेति~।

\newpage
% विंशोऽध्यायः ९९ 

\begin{quote}
{\na [\renewcommand{\thefootnote}{1}\footnote{अयं चनादिषु न विद्यते}धर्माधर्मसमुत्थं यत्र भवेद्रागदोषसंयुक्तम्~।\\
साधिक्षेपं च वचो\renewcommand{\thefootnote}{2}\footnote{ढ \textendash\ क्षेपं वचनं} ज्ञेयः संलापको नाम]~॥~४९

\renewcommand{\thefootnote}{3}\footnote{भ \textendash\ यत्र}मन्त्रार्थ\renewcommand{\thefootnote}{4}\footnote{ढ \textendash\ कार्य}वाक्यशक्तया दैववशादात्मदोषयोगाद्वा\renewcommand{\thefootnote}{5}\footnote{ढ \textendash\ योगदोषाद्वा}~।\\
संघातभेदजननस्तज्ज्ञैः संघात्यको\renewcommand{\thefootnote}{6}\footnote{न \textendash\ संघातकः} ज्ञेयः~॥~५०

[\renewcommand{\thefootnote}{7}\footnote{अयं श्लोकः चादिमातृकासु न दृश्यते}बहुकपटसंश्रयाणां परोपघाताशयप्रयुक्तानाम्~।\\
कूटानां संघातो \renewcommand{\thefootnote}{8}\footnote{ढ \textendash\ ज्ञेयः ल तु}विज्ञेयः कूटसंघात्यः\renewcommand{\thefootnote}{9}\footnote{ढ \textendash\ संघातः}~॥~]~५१

इत्यष्टार्धविकल्पा वृत्तिरियं सात्त्वती मयाभिहिता~।\\
कैशिक्यास्त्वथ\renewcommand{\thefootnote}{10}\footnote{च\textendash\ इह} लक्षणमतःपरं संप्रवक्ष्यामि~।~५२}
\end{quote}

\hrule

\vspace{2mm}
(\underline{सङ्घातभेदजनन} इति) सङ्घातस्य भेदं जनयति यो युधि सं सङ्गात्यकः, सम्यक्धात्यः शत्रुवर्गो येन, सङ्घातकविषयाद्वा सङ्कात्यकः, सङ्घातभेदश्च परेण सामाद्युपायबलेन वा क्रियते~। यथा भीमो युधिष्ठिरेण साम्ना भेदितः, अतः शिखण्डिनं पुरस्कृत्य योद्धव्यश्च~। दैवात्संपद्यते, यथा द्रोणेनोक्तं सुते हते शस्त्रं त्यक्ष्यामीति~। आत्मदोषे वा स्वकटक\renewcommand{\thefootnote}{1}\footnote{{\qt कपट} स्यात्}लक्षणेन, यथा कर्णेन सह कलहायमानोऽश्वत्थामा शस्त्रत्यागं करोतीति~। अत्न च सत्त्वाधिक्यमपराध्यति {\qt कथं नामाहमेवंभूत} इति (वेण्यां तृतीयेऽङ्केऽन्ते)~।\\

\underline{अथे}त्यनन्तरम्~। \underline{अतःपरमि}त्येतेभ्यो लक्षणेभ्यः पृथग्भूतमित्यर्थः~। श्लक्ष्णः सुकुमारः श्लिष्यति हृदय इति कृत्वा~। नैपथ्यविशेषो वस्त्रमाल्यादिः

\newpage
% १०० नाठ्यशास्त्रम् 

\begin{quote}
{\na या श्लक्ष्णनैपथ्यविशेषचित्रा\renewcommand{\thefootnote}{1}\footnote{ढ\textendash\ विचित्रवेषा}\\
स्त्रीसंयुता या\renewcommand{\thefootnote}{2}\footnote{ट \textendash\ स्त्रीपुंसयुक्ता} बहुनृत्तगीता~।\\
कामोपभोगप्रभवोपचारा\\
तां कैशिकीं वृत्तिमुदाहरन्ति\renewcommand{\thefootnote}{3}\footnote{भ \textendash\ नाम वदन्ति वृत्तिम्}~।~५३

[\renewcommand{\thefootnote}{4}\footnote{श्लोकद्वयं चटडढादिषु न दृश्यते}बहुवाद्य\renewcommand{\thefootnote}{5}\footnote{ट \textendash\ काव्य}नृत्तगीता शृङ्गाराभिनयचित्रनैपथ्या~।\\
माल्यालङ्कारयुता प्रशस्तवेषा च कान्ता च~॥~५४

चित्र\renewcommand{\thefootnote}{6}\footnote{न \textendash\ पट}पदवाक्यबन्धैरलङ्कृता हसितरुदितरोषाद्यैः\renewcommand{\thefootnote}{7}\footnote{भ \textendash\ रोषयुता ज \textendash\ घोषाद्यैः}~।\\
स्त्रीपुरुषकामयुक्ता\renewcommand{\thefootnote}{8}\footnote{ज \textendash\ कालयुक्ता} विज्ञेया कैशिकीवृत्तिः]~॥~५५

\renewcommand{\thefootnote}{9}\footnote{न \textendash\ नर्मो}नर्म च नर्मस्फुञ्जो\renewcommand{\thefootnote}{10}\footnote{न \textendash\ स्फुटजो ढ \textendash\ स्फञ्जो भ \textendash\ स्पन्दो} नर्मस्फोटोऽथ नर्मगर्भश्च~।\\
कैशिक्याश्चत्वारो भेदा ह्येते \renewcommand{\thefootnote}{11}\footnote{भ\textendash\ मया}समाख्याताः~॥~५६

\renewcommand{\thefootnote}{12}\footnote{ढ\textendash\ स्थापितशृङ्गाररसं}आस्थापितशृङ्गारं विशुद्धकरणं निवृत्तवीररसम्~।}
\end{quote}

\hrule

\vspace{2mm}
\noindent
तेन चित्रा, बहु विपुलं गीतं नृत्तं च यस्याम्, कामोपभोगो रतिः ततः प्रभवो यः स श्रृङ्गारस्तद्भहुल उपचारो व्यवहारो यस्यां, सा तथोक्ता~। (कैशिक्याश्चत्वार्यङ्गानि नर्माख्यं नर्मोपपदानि च तत्र नर्मणः श्रृङ्गारस्थापकत्वं) हासप्रधा\textendash

\newpage
% विंशोऽध्यायः १०१

\begin{quote}
{\na हास्य\renewcommand{\thefootnote}{1}\footnote{ढ\textendash\ प्रपञ्च}प्रवचनबहुलं नर्म त्रिविधं विजानीयात्~॥~५७

ईर्ष्याक्रोधप्रायं सोपालम्भकरणानुविद्धं च\renewcommand{\thefootnote}{2}\footnote{भ \textendash\ लम्भं च करुणविद्धं च च \textendash\ वचनाविरुद्धं च (वचनानुविद्धं च ?)}~।\\
आत्मोपक्षेपकृतं सविप्रलम्भं स्मृतं नर्म~॥~५८

नवसङ्गमसम्भोगो रति\renewcommand{\thefootnote}{3}\footnote{भ \textendash\ समुदयवेश, ड \textendash\ समुदयवाक्यवेष}समुदयवेषवाक्यसंयुक्तः~।\\
ज्ञयो नर्मस्फञ्जो\renewcommand{\thefootnote}{4}\footnote{न \textendash\ स्फुटजो भ \textendash\ स्पन्दो ढ \textendash\ स्फञ्जो} ह्यवसानभयात्मकश्चैव\renewcommand{\thefootnote}{5}\footnote{भ \textendash\ भयानकश्चैव}~॥~५९}
\end{quote}

\hrule

\vspace{2mm}
\noindent
नता च तदेति, सामान्यलक्षणम्~। तत्र हास ईर्ष्यां वा सूचयितुं परं वोपालब्धुं परहृदयं वाक्षेप्तुमिति त्रिधा (आत्मेति) आत्मनः परकीयस्य चित्तस्योपक्षेप आत्मसमीपकरणम्~। उदाहरणम्\textendash\ वासवदत्ता (फलकमुद्दिश्य सहासं) \renewcommand{\thefootnote}{*}\footnote{एषाप्यपरा तस्य सर्मपे जाया लिखिता~। एतदप्यार्यवसन्तकस्य विज्ञानम्}एसा वि अवरा तस्स समीवे जाआ लिहिदा~। एदं वि अय्यवसन्तअस्स विण्णाणम्~।\\

द्वितीयस्योदाहरणं\textendash\ {\qt शीतांशुर्मुख} मित्यादि श्रुतवती वासवदत्ता यदा राज्ञोच्यते, {\qt प्रिये वासवदत्ते} इति तद्वचसोपालम्भं सा सहासमाह\textendash\ {\qt अय्यउत्त मा एव्वं भण} इत्यादि~।\\

तृतीयस्य सुसङ्गता (विहस्य) {\qt जादिसो तुए कामदेवो आलिहिदो मए वि तारिसी रई आलिहिदा~। ता असंभाविणी, कहेहि दाव वुत्तंत्तं~।}\\

एवं त्रिभेदं नर्माख्यमाख्याय नर्मस्फुञ्जं प्रकाशयितुमाह \underline{नवसङ्गमेति} नवसङ्गममात्र एव सम्भोगो यत्र~। कथं तस्य सङ्गमस्य सम्भोगत्वमित्याह \underline{रतिसमुदयेति~।} रतेरन्योन्यास्थाबन्धरूपायाः समुदायः स्फुटत्वं, यस्तादृशेन वेषेण वाक्येन वा योगो यत्न~। अवसाने च भयं पूर्वनायिकाकृतम्~। यथा रत्नावल्यामुदयनस्य सागरिकायाश्च नर्मणः स्फुञ्जो विघ्न इत्यर्थः~।

\newpage
% १०२ नाट्यशास्त्रम्

\begin{quote}
{\na विविधानां भावानां लवैर्लवैर्भूषितो बहुविशेषैः\renewcommand{\thefootnote}{1}\footnote{न विशेषः}~।\\
\renewcommand{\thefootnote}{2}\footnote{ब \textendash\ असमस्त}असमग्राक्षिप्तरसो नर्मस्फोटस्तु विज्ञेयः~॥~६०

विज्ञानरूप\renewcommand{\thefootnote}{3}\footnote{च सम्भावनादिभिः भ \textendash\ ससम्भावितादिभिः}शोभाधनादिभिर्नायको गुणैर्यत्र~।\\
\renewcommand{\thefootnote}{4}\footnote{च \textendash\ प्रच्छन्नैः}प्रच्छन्नं व्यवहरते कार्यवशान्नर्मगर्भोऽसौ\renewcommand{\thefootnote}{5}\footnote{भ \textendash\ गर्भः सः}~॥~६१

\renewcommand{\thefootnote}{6}\footnote{अयं च मातृकायां न वर्तते}[\renewcommand{\thefootnote}{7}\footnote{भ \textendash\ पूर्वस्थितोऽभिभूतो यत्र भवेन्नायको विषण्णः सः~। तमपि \ldots \ldots प्रयोगज्ञः~॥ ड \textendash\ पूर्वस्थितो विपद्येत यत्र चान्यतमनायकस्तिष्ठेत्~। तमपीह नर्मगर्भं वदन्ति नाट्यप्रयोगेऽस्मिन्~॥}पूर्वस्थितौ विपद्येत नायको यत्र चापरस्तिष्ठेत्\renewcommand{\thefootnote}{8}\footnote{ट \textendash\ चापरे तिष्ठेत्}~।\\
तमपीह नर्मगर्भं विद्यान् नाट्यप्रयोगेषु]~।~६२

इत्यष्टार्धविकल्पा वृत्तिरियं कैशिकी मयाभिहिता\renewcommand{\thefootnote}{9}\footnote{ट \textendash\ प्रोक्ता}~।\\
अत ऊर्ध्वमुद्धतरसामारभटीं संप्रवक्ष्यामि~॥~६३}
\end{quote}

\hrule

\vspace{2mm}
\underline{विविधा भावाः} भयहासहर्षत्रासरोषाद्याः \underline{लवैर्लवै}रित्यत एव भयादीनामंशेन भावात् स्थायित्वानुपगमात् भयानकहास्यरौद्रादिरसतापत्तिर्न सम्भवति~। श्रृङ्गारस्तु पूर्व एव {\qt जस्स किदे तुमं एत्थ आअदा से एत्थ एव्व चिव्टदि} इति सुसङ्गतोक्तौ\renewcommand{\thefootnote}{*}\footnote{रत्नावल्याम्} हासलवः, न हास्यो रसः~।\\

{\qt सहि कस्सकिदे अहं एत्थ आअदा} इत्यत्र सागरिकोक्तौ (रौद्र) लवो न तु रौद्रः~। एवमन्यत्र~। नर्मण इति तदुपलक्षितस्य श्रृङ्गारस्य स्फोटो वैचित्र्यं चमत्कारोल्लासकृतस्फुटत्वं यत्रेति~।\\

श्रृङ्गारोपयोगिभिर्विज्ञानाद्यैः \underline{प्रच्छन्नं} यत्र नायक आस्ते नवसमागमसिद्धये स नर्मगर्भः~। नर्मोपयोगिनो विज्ञानाद्या गर्भीकृता इव प्रच्छन्नतया यत्रेति, यथा प्रच्छन्नरूपो नायकः सङ्केतस्थानं गच्छति~।

\newpage
% विंशोऽध्यायः १०३ 

\begin{quote}
{\na \renewcommand{\thefootnote}{1}\footnote{भ आरभटी}आरभटप्रायगुणा तथैव बहुकपटवञ्चनोपेता\renewcommand{\thefootnote}{2}\footnote{भ धर्षणोपेता}~।\\
दम्भानृतवचनवती त्वारभटी नाम विज्ञेया\renewcommand{\thefootnote}{3}\footnote{भ\textendash\ सा ज्ञेया}~।~६४

\renewcommand{\thefootnote}{4}\footnote{श्लोकद्वयं प्रक्षिप्तमिति न व्याख्यातम्~। तयोः प्रथमः श्लोको भरतस्येति सुप्रसिद्धमुदाहृतः}[पुस्तावपात\renewcommand{\thefootnote}{5}\footnote{भ \textendash\ क्रम}प्लुतलङ्घितानि\\
च्छेद्यानि मायाकृतमिन्द्रजालम्~।\\
चित्राणि युद्धानि च यत्र नित्यं\\
तां तादृशीमारभटीं वदन्ति~॥~६५

\renewcommand{\thefootnote}{6}\footnote{ड \textendash\ या षड्गुणसंरब्धा परातिसन्धान ज \textendash\ या षाड्गुण्यारब्धा पराति. अयं श्लोको भमातृकायामपि न वर्तते}षाड्गुण्यसमारब्धा हठातिसन्धानविद्रवोपेता~।\\
लाभालाभार्थकृता विज्ञेया वृत्तिरारभटी]~।~६६

संक्षिप्तकावपातौ वस्तूत्थापनमथापि\renewcommand{\thefootnote}{7}\footnote{भ\textendash\ अपीह} संफेटः~।\\
एते ह्यस्या भेदा लक्षणमेषां प्रवक्ष्यामि~॥~६७

\renewcommand{\thefootnote}{8}\footnote{भ \textendash\ मातृ \textendash\ कायां संक्षिप्तकेति स्त्रीलिङ्गशब्दप्रयोगात्\ldots {\qt युक्ता,\ldots नेपथ्या विषया \ldots \ldots ज्ञेया} इति पाठः स्वीकृतः }अन्वर्थशिल्पयुक्तो बहुपुस्तोत्थानचित्रनेपथ्यः~।}
\end{quote}

\hrule

\vspace{2mm}
(\underline{उद्धतेति}) दीप्तरसा रौद्रादयः उद्धताः~। (\underline{आरभटेति}) आरभटानां ये गुणाः क्रोधावेगाद्यास्ते \underline{प्रायेण} बाहुल्येन यत्र, बहुभिः कपटैः यद्वश्चनं तेनोपेता~। कपटत्रयं च समवकारलक्षणे (१८ \textendash\ ७१) व्याख्यातम्~। यत एवास्यां कपटयोगोऽत एव दम्भप्राधान्यमसत्यवचनसम्भवश्च~। (\underline{संक्षिप्तकेति}) संज्ञया क्षिप्तानि वस्तूनि विषयोऽस्येति संक्षिप्तकः~। तानि वस्तूनि दर्शयति (\underline{अन्वर्थेति})~। अर्थेन प्रयोजनेनानुगताः शिल्पयुक्ताः कुशलशिल्पिविरचिताः, अर्था यत्रेति~। अत्रैव दिशं दर्शयति (\underline{बहुपुस्तेति})~। बहु विपुलं, पुस्तस्योत्थानं प्रकटत्वं

\newpage
% १०४ नाट्यशास्त्रम्

\begin{quote}
{\na संक्षिप्तवस्तुविषयो ज्ञेयः संक्षिप्तको नाम~॥~६८

भयहर्षसमुत्थानं\renewcommand{\thefootnote}{1}\footnote{भ \textendash\ विद्रुतविभ्रान्तविविधविषयं च, ड \textendash\ विद्रुतसंभ्रान्तविविधवचनं च} विद्रवविनिपातसंभ्रमाचरणम्~।\\
क्षिप्रप्रवेशनिर्गममवपातमिमं विजानीयात्~॥~६९

\renewcommand{\thefootnote}{2}\footnote{य \textendash\ नैकरसलेशयुक्तं सविद्रवं वाप्यविद्रवं वापि}सर्वरससमासकृतं सविद्रवाविद्रवाश्रयं वापि~।\\
\renewcommand{\thefootnote}{3}\footnote{य \textendash\ पश्चात् प \textendash\ कार्य}नाट्यं विभाव्यते यत्तद्वस्तूत्थापनं ज्ञेयम्~॥~७०}
\end{quote}

\hrule

\vspace{2mm}
\noindent
विचित्रं च नेपथ्यं खड्गचर्मवर्गादि यत्र पुस्तयोगे~। यथा मायाशिरोनिक्षेपे रामाभ्युदये चित्रं नेपथ्यम्, यथा (वा)श्वत्थाम्नः (वेण्याम्)~।\\

भयातिशयेन हर्षातिशयेन च क्षिप्रमेव प्रवेशनिर्गमौ यत्र पात्राणां, तथा, विद्रवो वाक्यादिकृतो विनिपातोऽवस्कन्दः ताभ्यां कृतं सम्भ्रमाचरणं आवेगप्रधाना चेष्टा यत्र, सोऽवपातः, अवपतन्त्यस्मिन् पात्राणीति~। यथा कृत्यारावणे षष्ठेऽङ्के {\qt प्रविश्य खङ्गहस्तः सप्रहारः पुरुष} इत्यतः प्रभृति यावदसौ निष्क्रान्तः~।\\

(\underline{वस्तूत्थापनमिति}) वस्तूनां बहुना(मर्थाना)मुत्थापनं प्रसङ्गागतनिबन्धनं यत्र \underline{कार्ये} तत्तथोक्तम्~। कानि वस्तुनीत्याह \underline{सर्वरसेति}~। रसशब्देन स्थायिनो व्यभिचारिणश्च तेषां संक्षेपेण कृतं करणं यत्र, विद्रवैरग्न्याद्युपप्लवैः सह, तैर्विहीनम्(च)~। यथा, तत्रैव (कृत्यारावणे) अङ्गादादभिद्रूयमाणाया मन्दोदर्या भयं, अङ्गदस्योत्साहः रावणं दृष्ट्वा तस्यैव हि {\qt एतेनापि सुरा जिता} इत्यादि वदतो हासः, रावणस्यातिक्रोधः, {\qt यस्तातेन निगृह्य बालक इव प्रक्षिप्य कक्षान्तरे} इति वदतोऽङ्गदस्य जुगुप्साहासविस्मयरसा, विध्वंसनं नाटयतीत्यत्र रावणस्य शोकः\textendash\ इत्येवं विद्रवाश्रयं वस्तूत्थापनम्~। तद्विपरीतं तु तत्रैव (कृत्यारावणे) द्वितीयेऽङ्के {\qt नेपथ्ये कलकलः} इत्यतः प्रभृति यावत्सीतां प्रति रावणस्योक्तिः\textendash\ {\qt आ लोकपालानाक्रन्दसि} इत्यादि~।

\newpage
% विंशोऽध्यायः १०५

\begin{quote}
{\na संरम्भसंप्रयुक्तो\renewcommand{\thefootnote}{1}\footnote{ड \textendash\ समायुक्तो} बहुयुद्धनियुद्धकपटनिर्भेदः~।\\
शस्त्रप्रहारबहुलः \renewcommand{\thefootnote}{2}\footnote{च \textendash\ संस्फोटो}सम्फेटो नाम विज्ञेय:~॥~७१

एवमेता बुधैर्ज्ञेया वृत्तयो नाट्यसंश्रयाः\renewcommand{\thefootnote}{3}\footnote{च \textendash\ काव्यहेतवे, ट \textendash\ नाट्यमातरः}~।\\
रसप्रयोगमासां च \renewcommand{\thefootnote}{4}\footnote{च \textendash\ गदतो मे}कीर्त्यमानं निबोधत~॥~७२

\renewcommand{\thefootnote}{5}\footnote{भ \textendash\ हास्यशृङ्गारकरुणैर्वृत्तिः स्यात्कैशिकी रसैः\textendash\ * न \textendash\ शृङ्गारे चैव हास्ये च वृत्तिः स्यात् कैशिकी द्विजाः, ड \textendash\ शृङ्गारं चैव हास्यं च वृत्तिः स्यात्कैशिकी श्रिता , ट \textendash\ शृङ्गारे चाङ्गहास्ये च कैशिकी वृत्तिरिष्यते}हास्यशृङ्गारबहुला कैशिकी परिचक्षिता~।\\
\renewcommand{\thefootnote}{6}\footnote{नभ\textendash\ सात्त्वती चैव विज्ञेया वीररौद्राद्भुताश्रया ,ट\textendash\ वीरे चाप्यद्भुते चैव वृत्तिः स्यात्सात्त्वती मता}सात्त्वती चापि विज्ञेया वीराद्भुतशमाश्रया\renewcommand{\thefootnote}{7}\footnote{ढ\textendash\ समाश्रया~।}~।~७३}
\end{quote}

\hrule

\vspace{2mm}
\noindent
भाविनो वस्तुनः समुत्थापनादपीदं तथोक्तम्~। तथा च तत्रैव (कृत्यारावणे द्वितीयेऽङ्के) ऋषीणामुक्तिः\textendash \\

दुरात्मन् , नेयं सीता स्वनाशाय कृत्येयं हियते त्वया~। इति~।सम्फेटस्योदाहरणं जटायुयुद्धादि सर्वम् (कृत्यारावणे)~।\\

अथासां वृत्तीनां संक्षिप्य स्वरूपमाह \underline{हास्यश्रृङ्गारबहुला कैशिकीति सात्त्वती} \underline{चापि विज्ञेया वीराद्भुतशमाश्रया}~। इति~। अत्र शमशब्दः शान्तरसपरिग्रह इति तद्वादिनो मन्यन्ते~। \underline{समाश्रयेत्यन्ये} पठन्ति\renewcommand{\thefootnote}{*}\footnote{भ \textendash\ मातृकापाठः कोहलानुसारी~। यथा\textendash\ वीराद्भुतप्रहसनैरिह भारती स्यात् सात्त्वत्यपीह गदिताद्भुतवीररौद्रैः~। शृङ्गारहास्यकरुणैरपि कैशिकी स्यादिष्टा भयानकयुतारभटी सरौद्रा~॥ इति कोहलः~।}

\lfoot{14}

\newpage
\lfoot{}
% १०६ नाट्यशास्त्रम् 

\begin{quote}
{\na \renewcommand{\thefootnote}{1}\footnote{भ \textendash\ भयानके च बीभत्से रौद्रे चारभटी भवेत्, ट \textendash\ रौद्रे भयरसे चापि वृत्तिरारभटी स्सृता}रौद्रे भयानके चैव विज्ञेयारभटी बुधैः~।\\
\renewcommand{\thefootnote}{2}\footnote{न \textendash\ भारती चापि विज्ञेया करुणाद्भुतरुपयोः, (ज \textendash\ संश्रया) भ \textendash\ भारती चापि विज्ञेया वीरहास्याद्भुताश्रया ट \textendash\ सर्वेषु रसभावेषु भारती संप्रकीर्तिता~।}बीभत्से करुणे चैव भारती संप्रकीर्तिता~।~७४

\renewcommand{\thefootnote}{3}\footnote{च \textendash\ मातृकायां श्लोकद्वयं न विद्यते}[न ह्येकरसजं काव्यं किंचिदस्ति प्रयोगतः~।\\
भावो वापि रसो वापि प्रवृत्तिर्वृत्तिरेव वा~।~७५

सर्वेषां समवेतानां यस्य रूपं भवेद्बहु~।\\
स मन्तव्यो रसः स्थायी शेषाः सञ्चारिणः स्मृताः]~॥~७६

\renewcommand{\thefootnote}{4}\footnote{भ \textendash\ त्रिधा विभक्तोऽभिनयो मयोक्तो च \textendash\ प्रभवः समासात्}वृत्यन्त एषोऽभिनयो मयोक्तो\\
वागङ्गसत्त्वप्रभवो यथावत्\renewcommand{\thefootnote}{5}\footnote{भ \textendash\ नयप्रयोगं~।}~।\\}
\end{quote}

\hrule

\vspace{2mm}
अस्याध्यायस्याभिनयशेषभूततां ख्यापयन्नध्यायार्थमुपसंहरति भाविनश्चार्थमासूत्रयति \underline{वृत्त्यन्त} एष इति~। वृत्तिरभिनयस्य दशरूपकात्मा विषयोऽपि \underline{अन्त} इत्यभिनयस्य वृत्तयोऽन्तरमेकदेश आहार्य इति शरीरव्यतिरिक्तं बाह्यमित्यर्यः~। नटस्य हि सत्त्वात्मा वागभिनयो व्याहरणीय एव साक्षात्प्रयत्नकृतत्वात्~। अत एव तुशब्देन ततो व्यतिरेकमाहार्यमेव विशिनष्टि~। नेपथ्यकृतं त्वाहार्य वक्ष्यामीति भूयः कृतं विस्तार्येति शिवम्~।

\newpage
% विंशोऽध्यायः १०७ 

\begin{quote}
{\na आहार्यमेवाभिनयं प्रयोगे\renewcommand{\thefootnote}{1}\footnote{न \textendash\ यथावत् भ \textendash\ तथैव}\\
वक्ष्यामि नेपथ्यकृतं तु भूयः\renewcommand{\thefootnote}{2}\footnote{जादिबान्तेषु द्वाविंशः, भमातृकाया मेकोनविंशः, जट \textendash\ एकविंशः}~॥~७७

इति भारतीये नाट्यशास्त्रे वृत्तिविकल्पनं\\
नामाध्यायो विंशः3~।}
\end{quote}

\hrule

\begin{quote}
{\qt नृसिंहगुप्तायतिनेत्थमत्र वृत्तिस्वरूपं प्रकटं व्यधायि~।\\
यस्य त्रिणेत्रेण हृदन्तरात्मवृत्तिस्वरूपं प्रकटं व्यधायि~॥}
\end{quote}

\begin{center}
इति श्रीमहामाहेश्वराचार्याभिनवगुप्ताचार्यविरचितायां\\

भारतीयनाट्यवेदगृत्तावभिनवभारत्यां विंशो\\

वृत्त्यध्यायः समाप्तिमगमत्\\

\vspace{2cm}
\rule{0.2\linewidth}{0.5pt}
\end{center}
 
\newpage
\thispagestyle{empty}

\begin{center}
\textbf{\large श्रीः}

\textbf{\LARGE नाट्यशास्त्रम्}

एकविंशोऽध्यायः\renewcommand{\thefootnote}{1}\footnote{जादिबान्तेषु त्रयोविंशोऽध्यायः भ \textendash\ द्वाविंशोऽध्यायः}\renewcommand{\thefootnote}{*}\footnote{अस्मिन्नभ्याये भिन्नमातृकासु पाठक्रमो भिन्नतया दृश्यते~। तत्र तत्र वाक्यार्थोऽपि भिद्यते~। व्याख्याया विरलत्वात् मयमातृकयोः पाठानुसारी क्रमोऽनुसृतः~॥}

\rule{0.2\linewidth}{0.5pt}
\end{center}

\begin{quote}
{\na आहार्याभिनयं\renewcommand{\thefootnote}{2}\footnote{न \textendash\ चैव} विप्रा \renewcommand{\thefootnote}{3}\footnote{न \textendash\ प्रवक्ष्यामि}व्याख्यास्याम्यनुपूर्वशः~।\\
\renewcommand{\thefootnote}{4}\footnote{भ \textendash\ प्रयोगो योऽस्य, न \textendash\ सर्व एवा प्रयोगोऽयं यतस्तस्मिन् प्रतिष्ठितः, च \textendash\ प्रयोगो यत्र}यस्मात् प्रयोगः सर्वोऽयमाहार्याभिनये स्थितः~॥~१

नानावस्थाः प्रकृतयः \renewcommand{\thefootnote}{5}\footnote{न\textendash\ पूर्व}पूर्वं \renewcommand{\thefootnote}{6}\footnote{च \textendash\ नेपथ्य}नैपथ्यसाधिता\renewcommand{\thefootnote}{7}\footnote{ड \textendash\ सूचिताः}~।\\
अङ्गादिभिरभि\renewcommand{\thefootnote}{8}\footnote{भ \textendash\ व्यक्तिरुपगच्छति}व्यक्तिमुपगच्छन्त्ययत्नतः~॥~२}
\end{quote}

\hrule

\begin{center}
अभिनवभारती \textendash\ एकविंशोऽध्यायः\\

आहार्याभिनयः
\end{center}

\begin{quote}
{\qt यस्य सङ्कल्पमात्रेण विश्वमाहार्यमद्भुतम्~।\\
तं मानसमहामूर्तिं वन्दे गिरिसुतामपि~॥}
\end{quote}

आहार्यस्य सर्वपश्चादभिधानं वागाद्यभिनयेभ्योऽस्य बहिरङ्गत्वादित्यानुपूर्व्यमिति केचित्~। तच्चासत्, आवेदितपूर्वमाहार्यस्य प्राधान्यादेव त्वद्य सर्वानुग्राहकत्वं सर्वोपजीव्यताख्यापनाय पश्चादभिधानम्~। \underline{तदेवानुपूर्वश} इत्यनेनोक्तम्~। तथा चाह \underline{यस्मात् प्रयोगः सर्वोऽय}मिति वागङ्गसत्त्वात्मक इति~।\\

अत्रैवोपपत्तिमाह \underline{नानावस्था} इत्यादि~। नानाभूता या अवस्था रति\textendash

\newpage
\fancyhead[CO]{एकविंशोऽध्यायः}
% एकविंशोऽध्यायः १०९

\begin{quote}
{\na आहार्याभिनयो नाम ज्ञेयो \renewcommand{\thefootnote}{1}\footnote{भ \textendash\ नैपथ्यजो}नेपथ्यजो विधिः~।\\
तत्र कार्य: प्रयत्नस्तु \renewcommand{\thefootnote}{2}\footnote{न \textendash\ नाट्यशोभामिहेच्छता}नाट्यस्य शुभमिच्छता~॥~३

[तस्मिन्यत्रस्तु कर्तव्यो नैपथ्ये सिद्धिमिच्छता~।\\
नाट्यस्येह त्वलङ्कारो नैपथ्यं यत्प्रकीर्तितम्~॥~]~४

चतुर्विधं तु नेपथ्यं\renewcommand{\thefootnote}{3}\footnote{ड \textendash\ नैपथ्यं} पुस्तोऽलङ्कार एव च~।\\
\renewcommand{\thefootnote}{4}\footnote{च \textendash\ नाट्याङ्ग}तथाङ्गरचना चैव ज्ञेयं \renewcommand{\thefootnote}{5}\footnote{न \textendash\ संजीवं}सज्जीवमेव च~॥~५

पुस्तस्तु त्रिविधो ज्ञेयो नानारूपप्रमाणतः~।\\
सन्धिमो व्याजिमश्चैव वेष्टिमश्च प्रकीर्तितः~।~६}
\end{quote}

\hrule

\vspace{2mm}
\noindent
शोकाद्या नानाश्रयभूताश्च याः प्रकृतयो धीरोदात्तादय उत्तमाधमप्रभृतयश्च ताः पूर्वं यत्नतो नैपथ्येन साधिताः प्रकाशिताः पश्चादङ्गादिभिर्विभागं\renewcommand{\thefootnote}{1}\footnote{विभावं} अनुभावविषयविभागं नामोपपत्तिं देशकालादिविभागं चार्पयद्भिः स्फुटतमतामानीयन्ते~। तेन समस्ताभिनयप्रयोगचित्रस्य भित्तिस्थानीयमाहार्यम्~। तथा च समस्ताभिनयव्युपरमेऽपि नैपथ्यविशेषदर्शनाद्विशेषोऽवसीयत एव~।\\

यत्त्ववस्थान्तरयोगेऽभिनयान्तरवदाहार्ये न परिवर्तते तेन\renewcommand{\thefootnote}{2}\footnote{केन} प्रत्युत तथाभूतस्येयमवस्था प्राप्तेति स्थायिसूत्रानुस्मृतिसंपादनप्रावण्याद्रसं प्रत्यन्तरङ्गत्वमाहार्यस्यावेद्यते, तथा चाश्वत्थाम्नो युद्धवीररससम्पदोपेतस्यायं शोक आयात इति तथा येन [यदि] \renewcommand{\thefootnote}{3}\footnote{तथा हि यदि}युद्धोचितोज्ज्वलधर्मपरिग्रहाद्यपासनं क्रियेतेत्यंलं बहुना~।*\\

ज्ञेयो लोके~। नैपथ्यस्य(विधिः अलङ्कारः)स इहाहार्याभिनयः नाट्यस्य तु शुभमिति सिद्धिम्~। (सन्धिमः) सन्धानं सन्धा तया निर्वृत्तः, सदलादिरूपं क्रियते(इति) सन्धिमः~। व्याजः सूत्रस्याकर्षादिरूपः क्षेपस्तेन निर्वृत्तो व्याजिमः~।

\newpage
% 110 नाट्यशास्त्रम् 

\begin{quote}
{\na \renewcommand{\thefootnote}{1}\footnote{भ \textendash\ कैलिञ्जं चर्मजं वास्त्रं, ब कैलिञ्चं चार्मणं वस्त्रं, ड \textendash\ किलिञ्जं, च \textendash\ कै\textendash\ लिञ्जं न \textendash\ कलञ्ज}किलिञ्जचर्मवस्राद्यैर्यद्रूपं क्रियते बुधैः~।\\
सन्धिमो नाम विज्ञेयः पुस्तो नाटकसंश्रयः~॥~७

व्याजिमो नाम विज्ञेयो यन्त्रेण क्रियते तु यः~।\\
वेष्ट्यते चैव\renewcommand{\thefootnote}{2}\footnote{भ \textendash\ तेन} यद्रूपं वेष्टिमः स तु संज्ञितः\renewcommand{\thefootnote}{3}\footnote{च\textendash\ वस्राद्येवेष्टिमः स तु, ब \textendash\ स ज्ञेयो वेष्टिमोद्भवः}~॥~८

शैलयानविमानानि चर्मवर्मध्वजा नगाः\renewcommand{\thefootnote}{4}\footnote{भ \textendash\ वस्त्रध्वजाश्च ये}~।\\
\renewcommand{\thefootnote}{5}\footnote{च \textendash\ यानि क्रियेत, भ \textendash\ यानि क्रियन्ते}ये क्रियन्ते हि\renewcommand{\thefootnote}{6}\footnote{न \textendash\ अत्र} नाट्ये तु स पुस्त इति संज्ञितः~॥~९

\renewcommand{\thefootnote}{7}\footnote{न \textendash\ अलंकारास्तु विज्ञेया मालाभरणसंज्ञकाः~। नानावस्त्रकृताश्चैव नानावस्थान्तरात्मकाः}अलङ्कारस्तु विज्ञेयो माल्याभरणवाससाम्~।\\
नानाविधः समायोगोऽप्यङ्गोपाङ्गविधिः स्मृतः~॥~१०

वेष्टिमं विततं चैव संघात्यं ग्रन्थिमं\renewcommand{\thefootnote}{8}\footnote{ढ\textendash\ ग्रन्थिमत्} तथा~।\\
\renewcommand{\thefootnote}{9}\footnote{च \textendash\ प्रलम्बितं}प्रालम्बितं तथा चैव माल्यं पञ्चविधं स्मृतम्~॥~११}
\end{quote}

\hrule

\vspace{2mm}
\noindent
उपरि जतुसिक्थकादिना वेष्टस्तेन निर्वृत्तो वेष्टिमः~। भावप्रत्ययान्तन्निर्वृत्तार्थे इमपं स्मरन्ति (४\textendash\ ४\textendash\ १० का)~। \underline{किलिञ्जं} भूर्जवेणुदलादि~। \underline{रूपं क्रियत} इति रूपतां नीयत इति यावत्~। \underline{यन्त्रेणेति} सूत्रादिप्रयोगेण सन्धिमादयः प्रकाराः~। कोपयुज्यन्त इत्याह \underline{शैले}त्यादि~।\\

\underline{समायोग} इति योजना; स चाङ्गेषु शिरोहस्तादिषु उपाङ्गेषु च ललाटाङ्गुल्यादिषु निर्मितः~। \underline{वेष्टिमं} तृणवेष्टनया निर्मितं बहुमालावेष्टनकृतं वा~। \underline{वितत} \textendash\ मित्यावेष्टितान्योन्यश्लिष्टमालासमूहात्मकं वस्त्रधारणभयेनोम्भितं वा~। \underline{सङ्घात्यं} वृत्तं वा आस्यच्छिद्रान्तःप्रक्षिप्तसूत्रं बहुपुष्पगुच्छोम्भितं वा~। \underline{ग्रन्थिमं} ग्रन्थिभिरुम्भितं वा~। \underline{प्रलम्बित}मिति जालादिपर्यन्तव्याप्तिकम्~। आवे\textendash

\newpage
% एकविंशोऽध्यायः १११ 

\begin{quote}
{\na चतुर्विधं तु विज्ञेयं \renewcommand{\thefootnote}{1}\footnote{ड \textendash\ देहस्य}नाट्ये ह्याभरणं बुधैः~।\\
आवेध्यं बन्धनीयं च \renewcommand{\thefootnote}{2}\footnote{ड\textendash\ प्रक्षेप्यारोप्यके तथा, भ \textendash\ प्रक्षेप्यमृतुमेवच(?)}क्षेप्यमारोप्यमेव च~॥~१२

आवेध्यं \renewcommand{\thefootnote}{3}\footnote{भ\textendash\ कुण्डलानीह तथा श्रवणभूषणैः}कुण्डलादीह यत्स्याच्छ्रवणभूषणम्~।\\
आरोप्यं हेमसूत्रादि हाराश्च विविधाश्रयाः~॥~१३

श्रोणीसूत्राङ्गदे मुक्ताबन्धनीयानि सर्वदा\renewcommand{\thefootnote}{4}\footnote{न \textendash\ निर्दिशेत्}~।\\
प्रक्षेप्यं नूपरं विद्याद्वस्त्राभरणमेव च~॥~१९

भूषणानां विकल्पं हि\renewcommand{\thefootnote}{5}\footnote{न च} पुरुषस्त्रीसमाश्रयम्\renewcommand{\thefootnote}{6}\footnote{भ \textendash\ पुरुषस्यस्त्रियोऽथवा}~।\\
नानाविधं प्रवक्ष्यामि देशजातिसमुद्भवम्\renewcommand{\thefootnote}{7}\footnote{भ \textendash\ संज्ञान्तरसमाश्रयम्}~॥~१५

चूढामणिः \renewcommand{\thefootnote}{8}\footnote{ड \textendash\ सुमकुटं, ड \textendash\ समकरः}समकुटः शिरसो भूषणं स्मृतम्~।\\
कुण्डलं \renewcommand{\thefootnote}{9}\footnote{भ \textendash\ मोचकः कीलं}मोचकं कीला कर्णाभरणमिष्यते~॥~१६

मुक्तावली \renewcommand{\thefootnote}{10}\footnote{भ \textendash\ परिसरं}हर्षकं च \renewcommand{\thefootnote}{11}\footnote{प\textendash\ ससूत्रं}सूत्रकं कण्ठभूषणम्~।\\
\renewcommand{\thefootnote}{12}\footnote{ड \textendash\ कटका, ढ \textendash\ खटका, प \textendash\ वेटिका, म केटकोऽङ्गुलि, च\textendash\ वटिका}वेतिकाङ्गुलिमुद्रा च स्यादङ्गुलिविभूषणम्~॥~१७}
\end{quote}

\hrule

\vspace{2mm}
\noindent
ध्यादीनि स्वयमेव व्याचष्टे \underline{आवेध्यं कुण्डला}दीत्यादिना~। विविधाश्रया इति लतासंख्यादिभेदेन बहुभेदा इत्यर्थः~। चूडामणिः शिरोमध्ये~। मकुटो ललाटोर्ध्वे~। \underline{कुण्डलम}धरपाल्याम्~। \underline{मोचकं} कर्णशष्कुल्या मध्यच्छिद्रे कृतम्, \underline{कीला} ऊर्ध्वच्छिद्रे उत्तरकर्णिकेति प्रसिद्धा~। \underline{हर्षकमिति} समुद्गकं सर्पादिरूपतया प्रसिद्धम्~। \underline{सूत्रकमिति} गुच्छग्रीवासूत्रादितया प्रसिद्धम्~। \underline{\renewcommand{\thefootnote}{1}\footnote{वेधिका}वेतिकेति} सूक्ष्मकटकरूपा अङ्गुलिमुद्रा पक्षिपद्माद्याकारेणोपेता~।

\newpage
% ११२ नाट्यशास्त्रम् 

\begin{quote}
{\na \renewcommand{\thefootnote}{1}\footnote{ज हस्तती, ड \textendash\ हस्तवी, ढ\textendash\ हस्तपी}हस्तली वलयं चैव बाहुनालीविभूषणम्~।\\
\renewcommand{\thefootnote}{2}\footnote{म \textendash\ रुचकाोच्चितिके कार्ये ढ \textendash\ रुचिकोच्चित्तिके कार्ये}रुचकश्चूलिका कार्या\renewcommand{\thefootnote}{3}\footnote{ड \textendash\ चैव} मणिबन्धविभूषणम्~।~१८

\renewcommand{\thefootnote}{4}\footnote{ड \textendash\ केयूरमङ्गदं, ढ \textendash\ केयूरे साङ्गदे, च \textendash\ केयूरावङ्गदे}केयूरे अङ्गदे चैव कूर्परोपरिभूषणे\renewcommand{\thefootnote}{5}\footnote{ड\textendash\ भूषणम्}~।\\
\renewcommand{\thefootnote}{6}\footnote{भ\textendash\ त्रिसरं}त्रिसरश्चैव हारश्च\renewcommand{\thefootnote}{7}\footnote{ड \textendash\ भवेत्, न चित्रं; प \textendash\ भवेद्वक्षोज} तथा वक्षोविभूषणम्~॥~१९

\renewcommand{\thefootnote}{8}\footnote{ड \textendash\ व्यालम्बिमौक्तिका हारा माल्याद्या देहभूषणम्; ढ \textendash\ व्यालम्बमुक्ताहारादिमालादेहविभूषणम्}व्यालम्बमौक्तिको हारो माला चैवाङ्गभूषणम्~।\\
\renewcommand{\thefootnote}{9}\footnote{च\textendash\ तरलं, घ\textendash\ तदलं, र \textendash\ तनुकं, ड \textendash\ तलङ्कं}तलकं सूत्रकं चैव भवेत्कटिविभूषणम्~॥~२०

अयं पुरुषनिर्योगः कार्यस्त्वाभरणाश्रयः~।\\
देवानां पार्थिवानां च पुनर्वक्ष्यामि योषिताम्\renewcommand{\thefootnote}{10}\footnote{भ \textendash\ अहं त्रयम्}~॥~२१}
\end{quote}

\hrule

\vspace{2mm}
\underline{रुचक} इति करगोलके विततः तत ऊर्ध्वे \underline{चूलिकेति} प्रसिद्धो निकुञ्चकोऽग्रबाहुस्थाने \textendash\ एतन्मणिबन्धविभूषणम्, \underline{केयुरः} कुर्परस्योर्ध्वतः तयोरूर्ध्वे त्वङ्गदे~। \underline{त्रिसरः} मुक्तालतात्रयेण~। \underline{तलकं} नाभेरधः, तस्याप्यधः \underline{सूत्रकम्}~। \underline{पुरुषनिर्योगः} औचित्यमस्य~। आभरणाश्रयः आभरणविधिरित्यर्थः~।\\

ननु सर्वः पुरुषोऽनेन भूष्यत इत्याशङ्क्याह \underline{देवानां पार्थिवानां चेति}~। शिखाव्यालः नागः ग्रन्थिभिरुपनिबद्धो मध्ये कर्णिकास्थानीयः, तस्यैव दलसन्धानतया चित्ररचनानि वर्तुलानि पत्राणि पिण्डीपत्राणि~। चूडामणिः शिरोमध्ये, ततो मकरपत्रं (मकरिका), ततो ललाटान्तमुक्ताजालिका तोरणं जालिकादिरूपेण प्रसिद्धा~। सर्पस्यैव वा शिरस एकमेव सुवर्णमुक्तामणि\textendash

\newpage
% एकविंशोऽध्यायः ११३ 

\begin{quote}
{\na शिखपाशं शिखाव्यालः\renewcommand{\thefootnote}{1}\footnote{च \textendash\ जालं, न व्यालं, भ \textendash\ शिरोव्यालं} पिण्डीपत्रं\renewcommand{\thefootnote}{2}\footnote{भ \textendash\ पिण्डयन्त्रं, य \textendash\ खण्ड\textendash\ यन्त्रं, प\textendash\ पिण्डपत्रं, ढ \textendash\ खण्डपात्रं,} तथैव च~।\\
चूडामणिर्मकरिका\renewcommand{\thefootnote}{3}\footnote{भ \textendash\ मकरको} मुक्ताजालगवाक्षिकम्\renewcommand{\thefootnote}{4}\footnote{न \textendash\ गवाक्षिका, भ \textendash\ गवाक्षिकः}~॥~२२

शिरसो भूषणं \renewcommand{\thefootnote}{5}\footnote{भ \textendash\ वापि चित्रकं शीर्षजालकम्}चैव विचित्रं शीर्षजोलकम्\renewcommand{\thefootnote}{6}\footnote{ड \textendash\ जालकम्}~।\\
\renewcommand{\thefootnote}{7}\footnote{ड \textendash\ कुण्डलं}कण्डकं \renewcommand{\thefootnote}{8}\footnote{च \textendash\ खण्डपत्रं, भ \textendash\ गण्डपत्रं}शिखिपत्रं च वेणीपुच्छः\renewcommand{\thefootnote}{9}\footnote{न \textendash\ गुच्छः, ड \textendash\ कुञ्जः सरोचकः, ढ \textendash\ कञ्जः भ \textendash\ पुच्छमधोकरः} सदोरकः~॥~२३

ललाटतिलकं\renewcommand{\thefootnote}{10}\footnote{च \textendash\ तिलकः} चैव नाना\renewcommand{\thefootnote}{11}\footnote{भ \textendash\ नील, प \textendash\ नाली, ड \textendash\ शीर्ष}शिल्पप्रयोजितम्\renewcommand{\thefootnote}{12}\footnote{च \textendash\ प्रयोजितः}~।\\
\renewcommand{\thefootnote}{13}\footnote{छ\textendash\ भ्रूकच्छोपरि, ड \textendash\ भ्रुवोश्चोपरि}भ्रुगुच्छोपरिगुच्छश्च कुसुमानुकृतिस्तथा\renewcommand{\thefootnote}{14}\footnote{ड \textendash\ भवेत्}~॥~२४

कर्णिका कर्णवलयं तथा स्यात्पत्रकर्णिका~।\\
\renewcommand{\thefootnote}{15}\footnote{न \textendash\ आवेष्टकं कर्णमुद्रा, ड आवेष्टिकः कर्णमुद्रा}कुण्डलं कर्णमुद्रा च \renewcommand{\thefootnote}{16}\footnote{च \textendash\ कर्णोत्पलकं, न \textendash\ कर्णोत्फलकं ढ \textendash\ कर्णोत्कीलक एव च, भ \textendash\ कर्णाक्षिपकः}कर्णोत्कीलकमेव च~॥~२५

नानारत्नविचित्राणि दन्तपत्राणि चैव हि\renewcommand{\thefootnote}{17}\footnote{भ \textendash\ तथा संस्कारणानि च}~।\\
कर्णयोर्भूषणं \renewcommand{\thefootnote}{18}\footnote{न \textendash\ कार्ये}ह्येतत्कर्णपूरस्तथैव च~॥~२६}
\end{quote}

\hrule

\vspace{2mm}
\noindent
चित्रितम्~। खेला (डोला ?) प्रायं शीर्षतः जोलकं भूषणं शिखिपत्रं मयूरपि ञ्छाकारो विचित्रवर्णमणिरचितः कर्णावतंसकः~। \underline{कर्णिं}केत्यादिना विकल्पतः कर्णाभरणान्यपि तु स्थानान्तरभेदात्समुच्चयेनेत्याहुः~।

\lfoot{15}

\newpage
\lfoot{}
% ११४ नाट्यशास्त्रम् 

\begin{quote}
{\na \renewcommand{\thefootnote}{1}\footnote{य\textendash\ कीलकाः न \textendash\ तिलकः पत्रलेखश्च}तिलकाः पत्रलेखाश्व भवेद्गण्डविभूषणम्~।\\
त्रिवणी चैव विज्ञेयं भवेद्वक्षोविभूषणम्~॥~२७

नेत्रयोरञ्जनं \renewcommand{\thefootnote}{2}\footnote{ड \textendash\ कार्यं}ज्ञेयमधरस्य च रञ्जनम्~।\\
दन्तानां \renewcommand{\thefootnote}{3}\footnote{ड \textendash\ विविधारागाः}विविधो रागश्चतुर्णां शुक्लतापि वा~॥~२८

रागान्तरविकल्पोऽथ\renewcommand{\thefootnote}{4}\footnote{भ \textendash\ वा यः शोभेन} शोभनेनाधिकोज्ज्वलः~।\\
\renewcommand{\thefootnote}{5}\footnote{भ \textendash\ मुख्यानां}मुग्धानां सुन्दरीणां च मुक्ताभासितशोभनाः\renewcommand{\thefootnote}{6}\footnote{भ\textendash\ शोभितशोभिताः ड \textendash\ मुक्ताभाः सितशोभनाः}~॥~२९

\renewcommand{\thefootnote}{7}\footnote{प\textendash\ आरक्ता एव दन्ताः स्युस्तथा मालि च रञ्जनम्~। स्वरागोज्ज्योतितश्च स्यात्}सुरक्ता वापि दन्ताः स्युः पद्मपल्लवरञ्जनाः~।\\
अश्मरागोद्द्योतितः स्यादधरः पल्लवप्रभः~॥~३०

विलासश्च भवेत्तासां सविभ्रान्तनिरीक्षितम्\renewcommand{\thefootnote}{8}\footnote{भ \textendash\ विभ्रान्तं च विलक्षितम् (ब \textendash\ व)}~।\\
मुक्तावली व्यालपङ्क्तिर्मञ्जरी रत्नमालिका~॥~३१

रत्नावली \renewcommand{\thefootnote}{9}\footnote{ड \textendash\ च सूत्रं च}सूत्रकं च ज्ञेयं \renewcommand{\thefootnote}{10}\footnote{भ \textendash\ कण्ठे}कण्ठविभूषणम्~।\\
द्विसरस्त्रिसरश्चैव चतुस्सरकमेवच~॥~३२

तथा शृङ्खलिका चैव भवेत्कण्ठविभूषणम्~।\\
अङ्गदं वलयं चैव बाहुमूलविभूषणम्~॥~३३

नाना\renewcommand{\thefootnote}{11}\footnote{भ \textendash\ शिल्पीकृताः च \textendash\ रत्न}शिल्पकृताश्वैव हारा\renewcommand{\thefootnote}{12}\footnote{प \textendash\ वक्षोज}वक्षोविभूषणम्~।}
\end{quote}

\hrule

\vspace{2mm}
\noindent
\underline{त्विलासः}~। तेनायमर्थः\textendash\ एवं भूतेन दन्ताधररागवैचित्र्ये सति, \underline{सविभ्रान्त\textendash }

\newpage
% एकविंशोऽध्यायः ११५ 

\begin{quote}
{\na मणिजालावनद्धं\renewcommand{\thefootnote}{1}\footnote{च \textendash\ अनुबन्धं} च भवेत् स्तनविभूषणम्~॥~३४

खर्जूरकं सोच्छितिकं\renewcommand{\thefootnote}{2}\footnote{भ \textendash\ सोपवीतं ड \textendash\ वर्जूरं खेच्छितीकं च} बाहुनालीविभूषणम्~।\\
\renewcommand{\thefootnote}{3}\footnote{च \textendash\ शङ्खः कलापी कटकं तथा स्यात्पत्रपूरकम् ड \textendash\ कटकं कलकाशाखा (शाखा च \textendash\ ढ) भ \textendash\ कलापी शाटकं}कलापी कटकं शङ्खो हस्तपत्रं \renewcommand{\thefootnote}{4}\footnote{ढ \textendash\ सु}सपूरकम्~॥~३५

मुद्राङगुलीयकं चैव ह्ङ्गुलीनां विभूषणम्\renewcommand{\thefootnote}{5}\footnote{न \textendash\ च स्यादङ्गल्याभरणं भवेत्}~।\\
\renewcommand{\thefootnote}{6}\footnote{भ \textendash\ काञ्ची मौक्तिकजालाढ्या तलकं मेखला तथा (ढ \textendash\ कुलकं मेखलं)}मुक्ताजालाढ्यतलकं\renewcommand{\thefootnote}{7}\footnote{म \textendash\ तिलकं} मेखला \renewcommand{\thefootnote}{8}\footnote{च \textendash\ काञ्च्यथापि वा}काञ्चिकापि वा~॥~३६

रशना च कलापश्च \renewcommand{\thefootnote}{9}\footnote{च \textendash\ ज्ञेयं}भवेच्छ्रोणीविभूषणम्~।\\
एकयष्टिर्भवेत्काञ्ची मेखला त्वष्टयष्टिका~॥~३७

\renewcommand{\thefootnote}{10}\footnote{ड \textendash\ रशना षोडश ज्ञेया}द्विरष्टयष्टी रशना कलापः पञ्चविंशकः\renewcommand{\thefootnote}{11}\footnote{ड \textendash\ विंशतिः}~।\\
द्वात्रिंशच्च चतुःषष्टिः शतमष्टोत्तरं तथा\renewcommand{\thefootnote}{12}\footnote{च \textendash\ षोडशाष्टौ च चतुःषष्टिः शतं तथा}~॥~३८

मुक्ताहारा भवन्त्येते देवपार्थिवयोषिताम्~!\\
\renewcommand{\thefootnote}{13}\footnote{म \textendash\ नूपुरं किङ्किणी चैव घण्टिकाजालमेवच (भ \textendash\ रत्नजालकमेव च) प \textendash\ नूपुरं किङ्किणी काञ्ची}नूपुरः किङ्किणीकाश्च घण्टिका रत्नजालकम्~॥~३९}
\end{quote}

\hrule

\vspace{2mm}
\noindent
\underline{निरीक्षितं} यत् स्मितं तद्भवेदतीवहृद्यं (विलासः) संपाद्यत इति तावत्~। \underline{तलक}मिति कवाटद्वययोजितम्~।\\

\underline{यष्टि}रिति लता~। नूपुरो जान्वधः~। किङ्किणीका घण्टिकालग्ने (ग्ना?)~। 

\newpage
%११६ नाट्यशास्त्रम् 

\begin{quote}
{\na \renewcommand{\thefootnote}{1}\footnote{न \textendash\ सघोष कटकं}सघोषे कटके चैव गुल्फोपरिविभूषणम्~।\\
\renewcommand{\thefootnote}{2}\footnote{भ \textendash\ सरत्नं कर्णिकोद्योतं}जङ्घयोः पादपत्रं स्याद\renewcommand{\thefootnote}{3}\footnote{ज \textendash\ अङ्गुलौ}ङ्गुलीष्वङ्गुलीयकम्~॥~४०

\renewcommand{\thefootnote}{4}\footnote{च \textendash\ अङ्गष्ठे च}अङ्गुष्ठतिलका\renewcommand{\thefootnote}{5}\footnote{ढ \textendash\ तलकाः न \textendash\ तिलकं प \textendash\ तिलका च}श्चैव पादयोश्च विभूषणम्~।\\
\renewcommand{\thefootnote}{6}\footnote{ज \textendash\ तथैवालक्त}तथालक्तकरागश्च नानाभक्तिनिवेशितः\renewcommand{\thefootnote}{7}\footnote{भ\textendash\ विभूषितः}~॥~४१

अशोकपल्लवच्छायः स्यात् स्वाभाविक एव च\renewcommand{\thefootnote}{8}\footnote{भ \textendash\ वा}~।\\
एतद्विभूषणं नार्या आकेशादानखादपि~।~४२

यथाभावरसावस्थं विज्ञेयं द्विजसत्तमाः\renewcommand{\thefootnote}{9}\footnote{ड \textendash\ रसावस्यां विज्ञायैव प्रयोजयेत्}~।\\
आगमश्च प्रमाणं च\renewcommand{\thefootnote}{10}\footnote{म \textendash\ वै} रूपनिर्वर्णनं तथा~॥~४३

तिश्वकर्म\renewcommand{\thefootnote}{11}\footnote{भ\textendash\ मते}मतात्कार्यं \renewcommand{\thefootnote}{12}\footnote{ड \textendash\ बुद्ध्यावापि प्रयोजयेत्}सुबुद्ध्यापि प्रयोक्तृभिः\\
न हि शक्यं सुवर्णेन मुक्ताभिर्मणिभिस्तथा~।~४४

\renewcommand{\thefootnote}{13}\footnote{ड \textendash\ स्वाधीनं चेप्सया चैव च \textendash\ स्वाधीनमतिरुढ्यैव}स्वाधीनमिति रुच्यैव कर्तुमङ्गस्य भूषणम्~।\\
\renewcommand{\thefootnote}{14}\footnote{ड \textendash\ विभावतो हि (ढ अभि) य \textendash\ विगाहता हि}विभागतोऽभिप्रयुक्तमङ्गशोभाकरं भवेत्~॥~४५}
\end{quote}

\hrule

\vspace{2mm}
\noindent
रत्नजालकं (प्रपदा)च्छादकम्~। \underline{सघोषे} सशब्दे \underline{कटके}~। (अङ्गुष्ठ) \underline{तिलका} इति विचित्ररचनाकृताः~। \underline{आकेशादिति} शिखापाशः शिखाव्याल इत्यतःप्रभृतीत्यर्थः~। \underline{अनखादिति} अलक्तकरागपर्यन्तमिति यावत्~। \underline{आगम} इत्युपादानकारणमिति यावत्~। \underline{प्रमाणमित} अङ्गुल्यादिपरिमाण (संमितत्वम्~। रूपनिर्वर्णन) मिति शुक्लादि निर्वर्तनं(यथास्वाभावम्)~। विश्वकर्मणा प्रोच्यते यादृङ्निर्दिष्टं तादृक्कार्यमित्यर्थः~। \underline{सुबुद्धयेति} लोकप्रसिद्धापीत्यर्थः~। 

\newpage
% एकविंशोऽध्यायः ११७ 

\begin{quote}
{\na यथा स्थानान्तरगतं भूषणं रत्नसंयुतम्~।\\
न तु नाट्यप्रयोगे तु\renewcommand{\thefootnote}{1}\footnote{य \textendash\ प्रयोगेषु} कर्तव्यं भूषणं गुरु~॥~४६

खेदं जनयते तद्धि सव्यायतविचेष्टनात्~।\\
गुरुभावावसन्नस्य स्वेदो मूर्छा च जायते\renewcommand{\thefootnote}{2}\footnote{ड \textendash\ प्रजायते}~॥~४७

गुर्वाभरणसन्नो हि चेष्टां न कुरुते पुनः~।\\
\renewcommand{\thefootnote}{3}\footnote{ड\textendash\ तस्मान्न सम्यक् च}तस्मात्तनुत्वचकृतं सौवर्णं भूषणं भवेत्~॥~४८

\renewcommand{\thefootnote}{4}\footnote{ड \textendash\ जतुपूर्णाल्परत्नं तु}रत्नवज्जतुबद्धं वा न खेदजननं भवेत्~।\\
स्वेच्छया भूषणविधिर्दिव्यानामुपदिश्यते\renewcommand{\thefootnote}{5}\footnote{न \textendash\ अपदिश्यते}~॥~४९

\renewcommand{\thefootnote}{6}\footnote{ड \textendash\ यदभावाद्विनिष्पन्नं}यत्नभावविनिष्पन्नं मानुषाणां विभूषणम्~।\\
\renewcommand{\thefootnote}{*}\footnote{एतौ श्लोकावग्रे श्मश्रुकर्मविधानानन् ड ढ \textendash\ प्रभृतिमातृकासु पठितौ~। कासुचित्तु वर्जितावेव ड \textendash\ मातृकायां\textendash\ वेष्टयं तथैव संघात्यं ग्रथनीयं तथैव च~। लम्बितं शोभितं चैव माल्यं पञ्चविधं स्मृतम्~॥ इति पठितम्}[वेष्टितं विततं चैव सङ्घात्यं ग्रन्थिमं तथा~॥~५०

लम्बशोभि तथा चैव माल्यं पञ्चविधं स्मृतम्~।\\
आच्छादनं बहुविधं नानापत्तनसंभवम्~।~५१

तज्ज्ञेयं त्रिप्रकारं तु शुद्धं रक्तं विचित्रितम् ]~।\\
दिव्यानां भूषणविधिर्य एष परिकीर्तितः~॥~५२}
\end{quote}

\hrule

\vspace{2mm}
\underline{पत्तनं} देशः~। शुद्धमिति शुक्लवर्णकम्~। \underline{रक्तमिति} कुसुम्भनील्याद्यन्यतमोपरक्तम्~। \underline{विचित्रमिति} बहुवर्णम्~। विभक्तिः विभागः~।

\newpage
% ११८ नाट्यशास्त्रम् 

\begin{quote}
{\na मानुषाणां च कर्तव्यो नानादेशसमाश्रयः~।\\
भूषणैश्चापि वेषैश्च\renewcommand{\thefootnote}{1}\footnote{प \textendash\ अपवेष्टैश्च} नानावस्थासमाश्रयैः~॥~५३

दिव्याङ्नानां कर्तव्या विभक्तिः स्वस्वभूमिजा\renewcommand{\thefootnote}{2}\footnote{भ स्वनिकायजा}\\
विद्याधरीणा यक्षीणामप्सरोनागयोषिताम्~॥~५४

ऋषिदैवतकन्यानां वेषैर्नानात्वमिष्यते~।\\
तथा च सिद्धगन्धर्वराक्षसासुरयोषिताम्~॥~५५

दिव्यानां नरनारीणां तथैव च शिखण्डकम्\renewcommand{\thefootnote}{3}\footnote{ड \textendash\ मानुषीणां तथैव च}~।\\
शिखापुटशिखण्डं तु मुक्ताभूयिष्ठभूषणम्~॥~५६

विद्याधरीणां कर्तव्यः शुद्धो वेषपरिच्छदः\renewcommand{\thefootnote}{4}\footnote{च \textendash\ कर्तव्यं चित्रवेषपरिच्छदम्}\renewcommand{\thefootnote}{*}\footnote{संग्रहे\textendash\ शिखाबद्धशिखण्डं च भूषां मौक्तिकभूयसीम्~। वेषं शुद्धं प्रकुर्वीत विद्याधरमृगीदृशाम्~॥ इति संवादोऽस्ति}~।\\
\renewcommand{\thefootnote}{5}\footnote{ड \textendash\ यक्षिण्यप्सरसां चैव कार्यं रत्नैर्विभूषणम्}यक्षिण्योऽप्सरसश्चैव कार्या रत्नविभूषणाः\renewcommand{\thefootnote}{6}\footnote{न \textendash\ बहुविभूषणाः}~॥~५७

\renewcommand{\thefootnote}{7}\footnote{न \textendash\ समस्तानां}समस्तासां भवेद्वेषो यक्षीणां केवलं शिखा~।\\
\renewcommand{\thefootnote}{8}\footnote{ड\textendash\ दिव्यवत्संप्रकर्तव्यं नागीनां तु विभूषणम्}दिव्यानामिव कर्तव्यं नागस्त्रीणां विभूषणम्~॥~५८

मुक्तामणि\renewcommand{\thefootnote}{9}\footnote{न \textendash\ गणप्रायं}लताप्रायाः फणास्तासां तु केवलाः\renewcommand{\thefootnote}{10}\footnote{भ \textendash\ केवलम्}}
\end{quote}

\hrule

\vspace{2mm}
\underline{तासा}मिति (श्लो \textendash\ ५९) नागयोषिताम्~। 

\newpage
% एकविंशोऽध्यायः ११९

\begin{quote}
{\na कार्यं तु मुनिकन्यानामेकवेणीधरं शिरः~॥~५९

न चापि\renewcommand{\thefootnote}{1}\footnote{ड \textendash\ भूषणं कार्यं तासामत्यर्थतो भवेत्}भूषणविधिस्तासां वेषो वनोचितः~।\\
मुक्तामरकतप्रायं मण्डनं सिद्धयोषिताम्~॥~६०

तासां चैव तु कर्तव्यं पीतवस्त्रपरिच्छदम्~।\\
पद्मरागमणिप्रायं गन्धर्वीणां विभूषणम्~॥~६१

\renewcommand{\thefootnote}{2}\footnote{ड \textendash\ वीणाहस्ताश्च कर्तव्याः कौसुम्भवसनास्तथा}वीणाहस्तश्च कर्तव्यः कौसुम्भवसनस्तथा~।\\
इन्द्रनीलैस्तु कर्तव्यं राक्षसीनां विभूषणम्~॥~६२

सितदंष्ट्रा च कर्तव्या कृष्णवस्त्रपरिच्छदम्\renewcommand{\thefootnote}{3}\footnote{न परिच्छदः म \textendash\ परिच्छदा}~।\\
वैडूर्यमुक्ताभरणाः कर्तव्या सुरयोषिताम्~॥~६३

शुकपिञ्छनिभैर्वस्त्रैः कार्यस्तासां परिच्छदः\\
पुष्यरागैस्तु मणिभिः क्वचिद्वैडूर्यभूषितैः\renewcommand{\thefootnote}{4}\footnote{ड \textendash\ भूषितः}~॥~६४

दिव्यवानरनारीणां कार्यो नीलपरिच्छदः~।\\
एवं शृङ्गारिणः कार्या वेषा दिव्याङ्गनाश्रयाः\renewcommand{\thefootnote}{5}\footnote{च\textendash\ दिव्याङ्गनासु वा}~॥~६५

अवस्थान्तरमासाद्य शुद्धाः कार्याः पुनस्तथा\renewcommand{\thefootnote}{6}\footnote{भ \textendash\ तथैव च}~।\\
मानुषीणां तु कर्तव्या नानादेशसमुद्भवाः~॥~६६}
\end{quote}

\newpage
% १२० नाट्यशास्त्रम् 

\begin{quote}
{\na \renewcommand{\thefootnote}{1}\footnote{च \textendash\ वेषास्त्वाभरणोपेतास्तांश्च सम्यङ्निबोधत}वेषाभरणसंयोगान् गदतस्तान्निबोधत~।\\
\renewcommand{\thefootnote}{2}\footnote{ड \textendash\ अवन्ति}आवन्त्ययुवतीनां तु\renewcommand{\thefootnote}{3}\footnote{प \textendash\ हि} शिरस्सालककुन्तलम्~॥~६७

गौडीनामलकप्रायं \renewcommand{\thefootnote}{4}\footnote{प \textendash\ शिखाप्रायैकवेणिकम्}सशिखापाशवेणिकम्~।\\
आभीरयुवतीनां तु द्विवेणीधर \renewcommand{\thefootnote}{5}\footnote{ढ \textendash\ धरं}एव तु~॥~६८

शिरः परिगमः कार्यो\renewcommand{\thefootnote}{6}\footnote{ढ \textendash\ परिगतं कार्यं} नीलप्रायमथाम्बरम्~।\\
तथा पूर्वोत्तरस्त्रीणां \renewcommand{\thefootnote}{7}\footnote{च \textendash\ समुद्धत प \textendash\ समुद्बद्ध}स्मुन्नद्धशिखण्डकम्~॥~६९

\renewcommand{\thefootnote}{8}\footnote{प \textendash\ आकेशधारणं ड \textendash\ आकेशं छादनं}आकेशाच्छादनं तासां \renewcommand{\thefootnote}{9}\footnote{ज \textendash\ वेष}देशकर्मणि कीर्तितम्~।\\
\renewcommand{\thefootnote}{10}\footnote{च \textendash\ तथा च}तथैव दक्षिणस्त्रीणां कार्यमुल्लेख्यसंश्रयम्\renewcommand{\thefootnote}{11}\footnote{च \textendash\ संज्ञितम्}~॥~७०

कुम्भी\renewcommand{\thefootnote}{12}\footnote{च \textendash\ पथक प \textendash\ पतक य \textendash\ पताक ज \textendash\ पथव (?)}बन्धकसंयूक्तं तथावर्तललाटिकम्\renewcommand{\thefootnote}{13}\footnote{न \textendash\ ललाटकम्}~।
% [ गणिकानां तु कर्तव्यमिच्छाविच्छित्ति मण्डनम्~॥ ] ७१

देशजातिविधानेन\renewcommand{\thefootnote}{14}\footnote{ड \textendash\ विशेषेण} शेषाणामापि कारयेत्~।\\
वेषं तथा चाभरणं \renewcommand{\thefootnote}{15}\footnote{भ \textendash\ नानावस्यान्तराश्रयम्}क्षुरकर्म परिच्छदम्~।~७२}
\end{quote}

\hrule

\vspace{2mm}
(\underline{सालकलुन्तलमिति}) अलकाः स्थाने कुन्तलाः कुञ्चिताः केशा यत्न तत्तथोक्तम्~। \underline{नीलप्रायं} वस्त्रपित्याभीरीणामेव~।\\

हृदयं व्याप्नोति हृद्यत एवेति \underline{वेषः} केशरचनादिः~। आ समन्तात् भ्रियते पोष्यते कान्तिर्येन तदाभरणं शिखाव्यालादि~। क्षुरकर्म अलकादियोजना~। \underline{परिच्छदः} विचित्रवस्त्रयोगः~। एतद्वेषादि प्रलम्भेन~।

\newpage
% एकविंशोऽध्यायः १२१ 

\begin{quote}
{\na \renewcommand{\thefootnote}{1}\footnote{ज \textendash\ मातृकायामेव}[आगमं चापि नैपथ्ये नाट्यस्यैवं प्रयोजयेत्]~।\\
\renewcommand{\thefootnote}{2}\footnote{ड \textendash\ अदेशजो हि वेषस्तु}अदेशयुक्तो वेषो हि न शोभां जनयिष्यति~॥~७३

मेखलोरसि बद्धा तु हास्यं समुपपादयेत्\renewcommand{\thefootnote}{3}\footnote{ड \textendash\ बन्धे च (तु) हास्यायैवौपजायते (ज बन्धात्तु हास्याय\ldots )म\textendash\ बद्धा या हास्यं समुपपादयेत्}~।\\
तथा प्रोषित\renewcommand{\thefootnote}{4}\footnote{भ\textendash\ कान्तानां व्यसनाभिहताश्च याः (ड \textendash\ या)}कान्तासु \renewcommand{\thefootnote}{5}\footnote{च \textendash\ मदन}व्यसनाभिहतासु च~॥~७४

वेषो \renewcommand{\thefootnote}{6}\footnote{ज स्यान्मलिनस्तासां}वै मलिनः कार्य एकवेणीधरं शिर\renewcommand{\thefootnote}{7}\footnote{च \textendash\ शिरश्चाप्येकवेणिकम्}ः~।\\
विप्रलम्भे तु\renewcommand{\thefootnote}{8}\footnote{ड\textendash\ हि} नार्यास्तु शुद्धो वेषो भवेदिह~॥~७५

\renewcommand{\thefootnote}{9}\footnote{ड \textendash\ नाना}नात्याभरणसंयुक्तो न चापि मृजयान्वितः\renewcommand{\thefootnote}{10}\footnote{न \textendash\ हि मृदायुतः}~।\\
एवं स्त्रीणां \renewcommand{\thefootnote}{11}\footnote{च \textendash\ प्रयोक्तव्या वेषा देशसमुद्भवाः (न \textendash\ ब्यो\ldots . षो\ldots वः)}भवेद्वेषो देशावस्थासमुद्भवः~॥~७६

पुरुषाणां पुनश्चैव वेषान्वक्ष्यामि तत्त्वतः\renewcommand{\thefootnote}{12}\footnote{भ \textendash\ अतःपरम्}~।\\
तत्राङ्गरचना पूर्वं कर्तव्या नाट्ययोक्तृभिः~॥~७७

\renewcommand{\thefootnote}{13}\footnote{च \textendash\ अतः}ततःपरं प्रयोक्तव्या वेषा देशसमुद्भवाः~।\\
सितो नीलश्च पीतश्च चतुर्थो रक्त एव च~॥~७८

एते स्वभावजा वर्णा यैः कार्यं त्वङ्गवर्तनम्~।\\
संयोगजाः पुनश्चान्ये\renewcommand{\thefootnote}{14}\footnote{न \textendash\ पुनस्त्वन्य} उपवर्णा भवन्ति हि~॥~७९}
\end{quote}

\hrule

\vspace{2mm}
\noindent
\underline{देशो}ऽवन्त्यादि, \underline{अवस्था} रतिशोकाद्याः~। \underline{तत्रे}ति पुरुषेष्वेव~। अङ्गनानां रूपपरिवर्तनसंपादनात्मकवर्णवर्तना कर्तव्या न स्त्रीपात्रेष्विति यावत्~। \underline{संयागजा} इति वर्णद्वयश्लेषणोत्थापिताः, उपवर्णास्तु बहुवर्णमिश्रणेनेत्यर्थः~।

\lfoot{16}

\newpage
\lfoot{}
%१२२ नाट्यशास्त्रम् 

\begin{quote}
{\na तानहं संप्रवक्ष्य्मि यथाकार्यं प्रयोक्तृभिः\\
सितनीलसमायोगे\renewcommand{\thefootnote}{1}\footnote{न \textendash\ योगात्} \renewcommand{\thefootnote}{2}\footnote{न \textendash\ कापोत इति संज्ञितः च\textendash\ कापोतो नाम जायते ग \textendash\ कापोतक}कारण्डव इति स्मृतः~॥~८०

सितपीतसमायोगात्पाण्डुवर्णः प्रकीर्तितः\renewcommand{\thefootnote}{3}\footnote{न \textendash\ इति स्मृतः}~।\\
सितरक्तसमायोगे पद्मवर्णः प्रकीर्तितः\renewcommand{\thefootnote}{4}\footnote{न \textendash\ इति स्मृतः भ \textendash\ प्रकीर्त्यते}~॥~८१

पीतनीलसमायोगाद्धरितो नाम जायते~।\\
नीलरक्तसमायोगात्कषायो नाम जायते~॥~८२

रक्तपीतसमायोगद्गौरवर्ण इति स्मृतः\renewcommand{\thefootnote}{5}\footnote{न \textendash\ गौर इत्यभिधीयते}~।\\
एते संयोगजा वर्णा ह्युपवर्णास्तथापरे\renewcommand{\thefootnote}{6}\footnote{न \textendash\ तथैव च}~॥~८३

त्रिचतुर्वर्णसंयुक्ता बहवः \renewcommand{\thefootnote}{7}\footnote{न परि}संप्रकीर्तिताः~।\\
बलस्थो यो भवेद्वर्णस्तस्य \renewcommand{\thefootnote}{8}\footnote{न\textendash\ भावो}भागो भवेत्ततः\renewcommand{\thefootnote}{9}\footnote{भ \textendash\ भावस्तस्य विधीयते}~॥~८४

दुर्बलस्य च भागौ द्वौ \renewcommand{\thefootnote}{10}\footnote{ज \textendash\ नीलवर्णादृते भवेत् न\textendash\ नीलयुक्त्या}नीलं सुक्त्वा प्रदापयेत्~।\\
नीलस्यैको भवेद्भागश्चत्वारोऽन्ये तु वर्णके\renewcommand{\thefootnote}{11}\footnote{भ\textendash\ अन्यस्त्वेकश्च निश्चितः च \textendash\ अन्यस्य तु स्मृताः}~॥~८५

\renewcommand{\thefootnote}{12}\footnote{भ\textendash\ वर्णस्य तु बलीयस्त्वं नीलस्यैव तु कीर्तितम् (च \textendash\ हि कीर्त्यते)}बलवान्सर्ववर्णानां नील एव प्रकीर्तितः~।\\
एवं वर्णविधिं ज्ञात्वा\renewcommand{\thefootnote}{13}\footnote{च \textendash\ मान} नानासंयोगसंश्रयम्\renewcommand{\thefootnote}{14}\footnote{भ\textendash\ संभोगसंभवम्}~॥~८६}
\end{quote}

\hrule

\vspace{2mm}
बलस्थ इत्यभिभवनकारी, ततोऽन्यो वर्णो द्विगुण इति, अस्यापवादमाह \underline{नीलवर्णादृते} इति~। तत्र तु यो भागविधिस्तं दर्शयति \underline{नीलस्यैक} इति

\newpage
% एकविंशोऽध्यायः १२३ 

\begin{quote}
{\na ततः \renewcommand{\thefootnote}{1}\footnote{ड \textendash\ तु वर्तना कार्या नानारूपसमाश्रया (न \textendash\ वर्ण)}कुर्याद्यथायोगमङ्गानां वर्तनं बुधः~।\\
वर्तनाच्छादनं रूपं स्ववेषपरिवर्जितम्\renewcommand{\thefootnote}{2}\footnote{च \textendash\ वर्तितम्}~॥~८७

नाट्यधर्मप्रवृत्तं तु\renewcommand{\thefootnote}{3}\footnote{ड \textendash\ धर्मीप्रवृत्तेन न \textendash\ प्रवृत्ते तु} ज्ञेयं तत्प्रकृतिस्थितम्~।\\
स्ववर्णमात्मनश्छाद्यं\renewcommand{\thefootnote}{4}\footnote{ज \textendash\ वर्णज्ञैः ड \textendash\ वर्णजैः} वर्णकैर्वेषसंश्रयै:~॥~८८

\renewcommand{\thefootnote}{5}\footnote{झ \textendash\ प्रकृतिर्यस्य}आकृतिस्तस्य\renewcommand{\thefootnote}{6}\footnote{ज \textendash\ यस्य} कर्त्तव्या \renewcommand{\thefootnote}{7}\footnote{ज \textendash\ तस्य}यस्य प्रकृतिरास्थिता\renewcommand{\thefootnote}{8}\footnote{न \textendash\ प्रकृतिमास्थिताः}~।\\
यथा \renewcommand{\thefootnote}{9}\footnote{ड\textendash\ नरः न\textendash\ जीवः}जन्तुः स्वभावं \renewcommand{\thefootnote}{10}\footnote{न \textendash\ स्थं}स्वं परित्यज्यान्यदैहिकम्\renewcommand{\thefootnote}{11}\footnote{न \textendash\ देहजम्}~॥~८९

\renewcommand{\thefootnote}{12}\footnote{ज \textendash\ परभावं प्रकुरुते भूतदेहसमाश्रितम् (न \textendash\ देहं समाश्रितः) भ \textendash\ अन्यत्स्वभावं लभते देहान्तरसमाश्रितः}तत्स्वभावं हि भजते देहान्तरमुपाश्रितः~।\\
\renewcommand{\thefootnote}{13}\footnote{ड\textendash\ वेषैश्च}वेषेण वर्णकैश्चैव च्छादितः पुरुषस्तथा~।~९०

परभावं प्रकुरुते\renewcommand{\thefootnote}{14}\footnote{ज \textendash\ प्रकुर्वीत} यस्य वेषं समाश्रितः\renewcommand{\thefootnote}{15}\footnote{ज \textendash\ उपाश्रितः}~।\\
देवदानवगन्धर्वयक्षराक्षसपन्नगाः~॥~९१}
\end{quote}

\hrule

\vspace{2mm}
\noindent
\underline{वर्णके} नीलस्य भाग इत्यर्थः~। वर्तनाशब्दं पर्यायैर्व्याचष्टे \underline{वर्तना च्छादन}मिति~। \underline{प्रकृतिस्थि}तमिति देवमानुषादिस्वभावविभागेनावस्थितमित्यर्थः~। वर्तनस्य प्रयोजनमाह \underline{यथा जन्तुः स्वभावं} स्वमिति~। \underline{जन्तुरिति} जीवात्मेत्यर्थः स च शुद्धनिर्मलानन्तचिदानन्दप्रकाशः स्वातन्त्र्यरूपं स्वमनपायिनमपि खभावं परित्यज्यान्यद् व्यतिरिक्तमपि दैहिकं देहभवंः शरीरकरणोचितं तत्स्वभावं भजते, यतो देहान्तरं तद्देहविशेष उपसमीपे आ समन्तात् श्रितः अतिनैकट्येन तदात्मवृत्त्या प्रतिपन्न इत्यर्थः~।

\newpage
% १२४ नाट्यशास्त्रम् 

\begin{quote}
{\na \renewcommand{\thefootnote}{1}\footnote{ड \textendash\ ते प्राणिन इति प्रोक्ता (ज्ञेयाः) जीवबन्धाश्च ये त्विह न \textendash\ ते स्मृताः, (न \textendash\ बद्धाश्च ते स्मृताः)}प्राणिसंज्ञाः स्मृता ह्येते जीवबन्धाश्च येऽपरे~।\\
स्त्रीभावाः पर्वताः नद्यः समुद्रा वाहनानि च~॥~९२

नानाशस्त्राण्यपि तथा विज्ञेयाः प्राणिसंज्ञया\renewcommand{\thefootnote}{2}\footnote{ढ \textendash\ संश्रयाः भ \textendash\ ज्ञेयानि तु विचक्षणैः}~।\\
शैलप्रासादयन्त्राणि चर्मवर्मध्वजास्तथा~॥~९३

नानाप्रहरणाद्याश्च तेऽप्राणिन इति स्मृताः~।\\
अथवा कारणोपेता भवन्त्येते शरीरिणः~॥~९४

\renewcommand{\thefootnote}{3}\footnote{भ \textendash\ देशभाषाश्रयोपेतं}वेषभाषाश्रयोपेता नाट्यधर्ममवेक्ष्य तु~।\\
वर्णानां तु विधिं ज्ञात्वा वयः प्रकृतिमेव च~॥~९५

\renewcommand{\thefootnote}{4}\footnote{न \textendash\ तस्मात्}कुर्यादङ्गस्य रचनां देशजातिवयःश्रिताम्\renewcommand{\thefootnote}{5}\footnote{न \textendash\ समाश्रिताम्}~।\\
देवा गौरास्तु विज्ञेया यक्षाश्चाप्सरसस्तथा~॥~९६}
\end{quote}

\hrule

\vspace{2mm}
एतदुक्तं भवति\textendash\ यथा परमात्मा स्वचैतन्यप्रकाशमत्यजन्नपि देहकञ्चुकोचितचित्तवृत्तिरूषितमिव स्वरूपमादर्शयति, तथा नटोऽपि आत्मावष्टम्भमत्यजन्नेव स्थाने लतालाद्यनुसरणाद्यायोगाद् देहस्थानीयेन वर्तनादिवेषपरिवर्तने(न) तदुचितस्वभावालिङ्गितमिव स्वात्मानं सामाजिकान् प्रति दर्शयति, प्रेक्षकपक्षे न नटाभिमानस्तत्र हि रामाभिमान इति दर्शयति~। एतदाशयेनैवास्माभिस्तत्र तत्र प्रतीतिरेव व्याख्याता रसाध्यायादौ~।\\

सजीवमाहार्यभेदं व्याचष्टे देवदानवगन्धर्वेत्यादिना, शैलप्रासादादीनि निर्जीवत्वे प्रस्तुतत्वेन परिगणितान्यपि अवस्थाविशेषेषु नाट्यधर्मेण सजीवत्वेऽपीत्याह \underline{शैले}त्यादिना, अथवा कारणोपेता इत्यादिना च~।\\

अङ्गरचनानि विभजति वर्णानामिति गौरादीनाम्~।

\newpage
% एकविंशोऽध्यायः १२५ 

\begin{quote}
{\na रुद्रार्कद्रुहिणस्कन्दास्तपनीयप्रभाः स्मृताः\renewcommand{\thefootnote}{1}\footnote{ड \textendash\ समप्रभाः}~।\\
सोमो बृहस्पतिः शुक्रो वरुणस्तारकागणाः\renewcommand{\thefootnote}{2}\footnote{ज \textendash\ वरुणोऽथ शिवस्तथा न \textendash\ वरुणोऽर्थेन्द्र एव च.}~॥~९७

समुद्रहिमवद्गङ्गाः श्वेता हि स्युर्बलस्तथा\renewcommand{\thefootnote}{3}\footnote{भ \textendash\ सिताः कार्यास्तु वर्णतः}~।\\
रक्तमङ्गारकं विद्यात् पीतौ बुधहुताशनौ~॥~९८

नारायणो नरश्चैव श्यामो नागश्च\renewcommand{\thefootnote}{4}\footnote{ड \textendash\ श्यामवर्णोऽथ} वासुकिः~।\\
दैत्याश्च दानवाश्चैव राक्षसा गुह्यका नगाः~॥~९९

पिशाचा \renewcommand{\thefootnote}{5}\footnote{ड \textendash\ यम आकाशं}जलमाकाशमसितानि\renewcommand{\thefootnote}{6}\footnote{इ \textendash\ श्यामवर्णाः} तु वर्णतः~।\\
\renewcommand{\thefootnote}{7}\footnote{च \textendash\ वसन्ति}भवन्ति षट्सु द्वीपेषु \renewcommand{\thefootnote}{8}\footnote{च \textendash\ ये नराः वर्णतस्तु ते}पुरुषाश्चैव वर्णतः~॥~१००

कर्तव्या नाट्ययोगेन\renewcommand{\thefootnote}{9}\footnote{न \textendash\ तत्वज्ञैः} निष्टप्तकनकप्रभाः~।\\
जम्बूद्वीपस्य \renewcommand{\thefootnote}{10}\footnote{न \textendash\ वर्षेषु भ \textendash\ सर्वस्य ड \textendash\ वर्षे ये}वर्षे तु नानावर्णाश्रया नराः~॥~१०१

\renewcommand{\thefootnote}{11}\footnote{भ \textendash\ उत्तराः कुरवो ये च}उत्तरांस्तु कुरूंस्त्यक्त्वा\renewcommand{\thefootnote}{12}\footnote{भ \textendash\ मुक्त्वा} ते चापि कनकप्रभाः~।\\
\renewcommand{\thefootnote}{13}\footnote{न \textendash\ भद्राश्वे}भद्राश्वपुरुषाः श्वेताः कर्तव्या वर्णतस्तथा\renewcommand{\thefootnote}{14}\footnote{भ \textendash\ ज्ञेयाः श्वेतास्ते वर्णतो बुधैः}~॥~१०२}
\end{quote}

\hrule

\vspace{2mm}
नगाः पर्वताः~। जलमाकाशमिति तदधिष्ठात्री देवतेह विवक्षिता~। \underline{जम्बूद्वीपस्य} वर्ष इति भारते~। \underline{ते चापीति} उत्तरकुरवः~। 

\newpage
% १२६ नाट्यशास्त्रम् 

\begin{quote}
{\na केतु\renewcommand{\thefootnote}{1}\footnote{ड \textendash\ मालास्तथा श्वेता भ \textendash\ मालाः पुनर्नीलाः}माले नरा नीला \renewcommand{\thefootnote}{2}\footnote{ढ \textendash\ श्वेता गौरा भवन्ति. हि (प\textendash\ वा)}गौराः शेषेषु कीर्तिताः~॥\\
\renewcommand{\thefootnote}{3}\footnote{भ नानावर्णा\textendash\ इति श्लोकस्थाने जादिबान्तेष्वादर्शषु पठितं\textendash\ {\qt नानावर्णाः स्सृता भूता वामना विकृताननाः~। वराहमेषमहिषमृगवक्त्रास्तथैव च}}नानावर्णाः स्मृता भूता \renewcommand{\thefootnote}{4}\footnote{भ \textendash\ गन्धर्वाश्च सपन्नगाः}गन्धर्वा यक्षपन्नगाः~।~१०३

विद्याधरास्तथा चैव पितरस्तु समा नराः\renewcommand{\thefootnote}{5}\footnote{च, भ \textendash\ समानवाः}~॥\\
पुनश्च भारते वर्षे॑\renewcommand{\thefootnote}{6}\footnote{भ\textendash\ सम्यक्}तांस्तान्वर्णान्निबोधत~।~१०४

राजनः पद्मयवर्णास्तु गौराः श्यामास्तथैव च~।\\
ये चापि सुखिनो मर्त्या गौराः कार्यास्तु वै बुधैः~॥~१०५

कुकर्मिणो ग्रहग्रस्ताः व्याधितास्तपसि स्थिताः~।\\
आयस्तकर्मिणश्चैव \renewcommand{\thefootnote}{7}\footnote{ड \textendash\ कुजाताश्चासिताः स्मृताः (ढ \textendash\ त्याः)}ह्यसिताश्च कुजातयः~॥~१०६

\renewcommand{\thefootnote}{8}\footnote{भ \textendash\ ओषध्यश्चापि}ऋषयश्चैव कर्तव्या नित्यं \renewcommand{\thefootnote}{9}\footnote{न \textendash\ बदरवर्णिनः}तु बदरप्रभाः~।\\
\renewcommand{\thefootnote}{10}\footnote{भ \textendash\ तपसस्विनश्च कर्तव्या}तपःस्थिताश्च ऋषयो नित्यमेवासिता बुधैः~॥~१०७

कारणव्यपदेशेन\renewcommand{\thefootnote}{11}\footnote{भ \textendash\ तथाध्यात्मेच्छ्यापि च} तथा चात्मेच्छया पुनः~।}
\end{quote}

\hrule

\vspace{2mm}
\noindent 
\underline{पितरस्तु समानरा॒} इति तुः स्वार्थे पितरो नराश्च तुल्या इत्यर्थः~। \underline{कुकर्मिण} इति कुत्सितं निन्दितं कर्म येषाम्~। \underline{आयस्तं} शरीरक्लेशबहुलं कर्म येषामित्यर्थः~। कुजातयो धीवरडोम्बाद्याः~। बदरप्रभावत्त्वेऽप्यपवादमाह तपः स्थिता असिता इति~। अनेन तु ऋषीणामापि तपोनिरतानामसितत्वमित्यपौनरुक्त्यम्~।\\

व्यापकं लक्षणमाह \underline{कारणव्यपेदेशेनीति}~। कारणं यथा क्लेशबहुला क्रिया कृष्णत्वे~। \underline{आत्मेच्छयेति} कविबुद्ध्यनुसारेणेत्यर्थः~।

\newpage
% एकविंशोऽध्यायः १२७

\begin{quote}
{\na वर्णस्तत्र प्रकर्तव्यो\renewcommand{\thefootnote}{1}\footnote{ड \textendash\ त्वन्यः प्रयोक्तव्यो न त्वन्योऽपि कर्तव्यो} देशजातिवशानुगः\renewcommand{\thefootnote}{2}\footnote{च \textendash\ तपोऽनुगः (?) ज \textendash\ वयः श्रितः घ \textendash\ वयोऽनुगः}~॥~१०८

देशं \renewcommand{\thefootnote}{3}\footnote{ज \textendash\ कालं भ \textendash\ जातिं दिशश्चैव}कर्म च जातिं च पृथिव्युद्देशसंश्रयम्\renewcommand{\thefootnote}{4}\footnote{भ \textendash\ संश्रयान्}~।\\
विज्ञाय वर्तना कार्या पुरुषाणां प्रयोगतः\renewcommand{\thefootnote}{5}\footnote{न \textendash\ वर्तनां कुर्यात्पुरुषाणां प्रयोगवित्}~॥~१०९

किरातबर्बरान्ध्राश्च द्रविडाः \renewcommand{\thefootnote}{6}\footnote{म \textendash\ काञ्चि}काशिकोसलाः\renewcommand{\thefootnote}{7}\footnote{न \textendash\ कोशलाः}~।\\
पुलिन्दा दाक्षिणात्याश्च प्रायेण त्वसिताः स्मुताः\renewcommand{\thefootnote}{8}\footnote{भ\textendash\ प्रायशो वर्णतोऽसिताः}~॥~११०

\renewcommand{\thefootnote}{9}\footnote{ढ \textendash\ शाकाः}शकाश्च यवनाश्चैव पह्लवा बाह्लिकाश्च ये\renewcommand{\thefootnote}{10}\footnote{ढ \textendash\ बहिकादयः ब \textendash\ बह्निका वाहिकास्तथा}~।\\
प्रायेण गौराः \renewcommand{\thefootnote}{11}\footnote{भ \textendash\ विज्ञेया उत्तरां चा श्रिता}कर्तव्या उत्तरा ये श्रिता दिशम्~॥~१११

पाञ्चालाः \renewcommand{\thefootnote}{12}\footnote{न \textendash\ शूरसेनाश्च}शौरसेनाश्च \renewcommand{\thefootnote}{13}\footnote{भ \textendash\ तथाचैव न \textendash\ महिषाश्च ज \textendash\ सखसाश्च}माहिषाश्चौड्रमागधाः~।\\
अङ्गा वङ्गाः कलिङ्गाश्च श्यामाः कार्यास्तु वर्णतः~॥~११२

ब्राह्मणाः क्षत्रियाश्वैव \renewcommand{\thefootnote}{14}\footnote{ड \textendash\ रक्ताः}गौराः कार्यास्तथैव हि\renewcommand{\thefootnote}{15}\footnote{ज\textendash\ सदैव हि}\\
वैश्याः शूद्रास्तथा चैव श्यामाः कार्यास्तु वर्णतः~॥~११३

एवं कृत्वा यथान्यायं \renewcommand{\thefootnote}{16}\footnote{च \textendash\ अङ्गोपाङ्गेषु}मुखाङ्गोपाङ्गवर्तनाम्\\
श्मश्रुकर्म प्रयुञ्जीत देश\renewcommand{\thefootnote}{17}\footnote{ड \textendash\ कर्मीक्रियानुगम् न \textendash\ जाति}कालवयोऽनुगम्~॥~११४

शुद्धं विचित्रं श्यामं च तथा रोमशमेव च~।}
\end{quote}

\hrule

\vspace{2mm}
तत्रेति भारते~। \underline{शुद्ध}मिति क्षुरेण, सर्वथा चासितं, \underline{श्यामं} पूर्वं क्षुरकर्म योजितमपीदानीं निवारितं तद्योजनाङ्कम्~। \underline{विचित्र}मिति क्षुरकर्त्रिकाकर्मणो\textendash

\newpage
% १२८ नाट्यशास्त्रम्

\begin{quote}
{\na भवेच्चतुर्विधं श्मश्रु नानावस्थान्तरात्मकम्\renewcommand{\thefootnote}{1}\footnote{न \textendash\ आश्रयम्}~॥~११५

शुद्धं \renewcommand{\thefootnote}{2}\footnote{भ \textendash\ च निखिलं}तु लिङ्गिनां कार्यं \renewcommand{\thefootnote}{3}\footnote{प \textendash\ महामात्र}तथामात्यपुरोधसाम्\renewcommand{\thefootnote}{4}\footnote{च \textendash\ {\qt पुरो धसाम्} इत्यनन्तरं {\qt अनिस्तीर्ण प्रतिज्ञानां} इति श्लोकः पठितः~। तत्र {\qt शुद्धं श्यामं विचित्रं} इत्युद्देशे पठितम्~।}~।\\
मध्यस्था \renewcommand{\thefootnote}{5}\footnote{न \textendash\ चैव पुरुषाः स्थानीयाश्चैव ये पुनः}ये च पुरुषा ये च दीक्षां समाश्रिताः~॥~११६

दिव्या ये पुरुषाः\renewcommand{\thefootnote}{6}\footnote{न \textendash\ चैव} केचित्सिद्धविद्याधरादयः\renewcommand{\thefootnote}{7}\footnote{न \textendash\ धराश्च ये}~।\\
\renewcommand{\thefootnote}{8}\footnote{न \textendash\ नृपतीनां कुमाराणां}पार्थिवाश्च कुमाराश्च ये च राजोपजीविनः\renewcommand{\thefootnote}{9}\footnote{न \textendash\ सेविनः}~॥~११७

श्रृङ्गरिणश्च ये मर्त्या यौवनोन्मादिनश्च\renewcommand{\thefootnote}{10}\footnote{भ \textendash\ उन्मादिताश्च} ये~।\\
तेषां विचित्रं कर्तव्यं श्मश्रु \renewcommand{\thefootnote}{11}\footnote{भ \textendash\ कर्म}नाट्यप्रयोक्तृभिः~॥~११८

अनिस्तीर्णप्रतिज्ञानां दुःखितानां तपस्विनाम्~।\\
व्यसनाभिहतानां च श्यामं श्मश्रु प्रयोजयेत्\renewcommand{\thefootnote}{12}\footnote{न \textendash\ भवेदथ}~॥~११९}
\end{quote}

\hrule

\vspace{2mm}
\noindent
त्पादनकर्मणा च रचितचित्रसंनिवेशम्~। \underline{रोम}शमिति यथोत्पन्नम्~। \underline{लिङ्गि}नामपि ब्रह्मचारिवानप्रस्थ दीनाम्~।

\begin{quote}
{\qt मध्यस्था ये च पुरुषा ये च दीक्षां समाश्रिताः}
\end{quote}

\noindent
इत्यर्धे\textendash

\begin{quote}
{\qt शुद्धं तु लिङ्गिनां कार्ये तथामात्यपुरोधसाम्}
\end{quote}

इत्यर्धस्यानन्तरं योज्यम्, लेखकदोषात्तु स्थानान्तरे दृश्यते~। \underline{मध्यस्थ} इति नोत्तमानामधमानामित्यर्थः~। \underline{यौवनोन्मादिन} इति अमात्यपुरोधसोऽ पीति भावः~।

\newpage
% एकविंशोऽध्यायः १२९ 

\begin{quote}
{\na \renewcommand{\thefootnote}{1}\footnote{भ मुनीनां}ऋषीणां तापसानां च ये च दीर्घव्रता नराः~।\\
\renewcommand{\thefootnote}{2}\footnote{प \textendash\ तथा च वैर न\textendash\ सिद्धविद्याधराणां च रोमशं च भवेदतः (ढ \textendash\ तु\ldots तु विधीयते भ \textendash\ संप्रयोजयेत् )}तथा च चीरबद्धानां रोमशं श्मश्रु कीर्तितम्~॥~१२०~॥

एवं नानाप्रकारं तु श्मश्रु कार्यं प्रयोक्तृभिः\renewcommand{\thefootnote}{3}\footnote{ड\textendash\ कर्म प्रयोजयेत्}~।\\
अत ऊर्ध्वं प्रवक्ष्यामि \renewcommand{\thefootnote}{4}\footnote{न \textendash\ वेषं योगजम्}वेषान्नानाप्रयोगजान्\renewcommand{\thefootnote}{5}\footnote{भ \textendash\ नानाश्रयोद्भवान्~। अतः परं माल्याच्छादनविधानश्लोकौ पठितौ केषुचिदादर्शेषु}~।~१२९~॥

शुद्धो विचित्रो मलिनस्त्रिविधो वेष उच्यते~।\\
तेषां \renewcommand{\thefootnote}{6}\footnote{ड \textendash\ विशेषान् व्याख्यास्ये यथाकार्यं प्रयोक्तृभिः}नियोगं वक्ष्यामि यथावदनुपूर्वशः~॥~१२२

देवाभिगमने चैव मङ्गले नियमस्थिते~।\\
तिथिनक्षत्रयोगे च विवाहकरणे तथा~॥~१२३

धर्मप्रवृत्तं यत्कर्म स्त्रियो\renewcommand{\thefootnote}{7}\footnote{ड \textendash\ कार्य स्त्रीणां भ \textendash\ किंचित् स्त्रियो} वा पुरुषस्य वा~।\\
वेषस्तेषां\renewcommand{\thefootnote}{8}\footnote{च \textendash\ तत्र} भवेच्छुद्धो ये च प्रायत्निका नराः\renewcommand{\thefootnote}{9}\footnote{भ \textendash\ उदासीनाश्च ये नराः}~॥~१२४

देवदानवयक्षाणां गन्धर्वोरगरक्षसाम्~।\\
नृपाणां कर्कशानां\renewcommand{\thefootnote}{10}\footnote{प \textendash\ कामुकानां} च चित्रे वेष उदाह्रतः\renewcommand{\thefootnote}{11}\footnote{भ \textendash\ विचित्रोऽथ उदाहृतः}~॥~१२५

\renewcommand{\thefootnote}{12}\footnote{न \textendash\ कञ्चुकिनाममात्यानां श्रेष्टिनां सपुरोधसाम् (प \textendash\ च)~। सिद्धविद्याधराणां च वणिक्छस्त्रविदामपि (ढ\textendash\ शा)}वृद्धानां ब्राह्मणानां च श्रेष्ठयमात्यपुरोधसाम्~।\\
वणिजां काञ्चुकीयानां तथा चैव तपस्विनाम्~॥~१२६}
\end{quote}

\hrule

\vspace{2mm}
शुद्ध इति शुक्लवस्त्रादिः प्रायः~। \underline{प्रायत्निका} इति प्रयत्ने भवा विनीता इत्यर्थः~।

\lfoot{17}

\newpage
\lfoot{}
%१३० नाट्यशास्त्रम् 

\begin{quote}
{\na विप्रक्षत्रियवैश्यानां स्थानीया ये च मानवाः~।\\
शुद्धो वस्त्रविधिस्तेषां कर्तव्यो नाटकाश्रयः~॥~१२७

उन्मत्तानां प्रमत्तानामध्वगानां तथैव च\renewcommand{\thefootnote}{1}\footnote{भ \textendash\ छन्नानामध्वगामिनाम्}~।\\
व्यसनोपहतानां च मलिनो वेष उच्यते\renewcommand{\thefootnote}{2}\footnote{ड \textendash\ इष्यते}~॥~१२८

शुद्धरक्तविचित्राणि वासांस्यूर्ध्वाम्बराणि\renewcommand{\thefootnote}{3}\footnote{च \textendash\ उच्चावचानि} च~।\\
योजयेन्नाट्यतत्त्वज्ञो वेषयोः शुद्धचित्रयोः~॥~१२९

कुर्याद्वेषे तु मलिने \renewcommand{\thefootnote}{4}\footnote{च\textendash\ मलिनानि}मलिनं तु विचक्षणः~।\\
\renewcommand{\thefootnote}{5}\footnote{भ मुनिनिग्रन्थशाक्यानां तथैव च तपस्विनाम्~। यतिपाशुपतानां च वेषः कार्यो व्रतानुगः}मुनिनिर्ग्रन्थशाक्येषु यतिपाशुपतेषु च\renewcommand{\thefootnote}{6}\footnote{ड \textendash\ त्रिदण्डिश्रोत्रियेषु च}~॥~१३०

व्रतानुगस्तु कर्तव्यो वेषो \renewcommand{\thefootnote}{7}\footnote{ड \textendash\ लोकानुभावतः}लोकस्वभावतः~।\\
चीरवल्कलचर्माणि तापसानां तु योजयेत्~॥~१३१

\renewcommand{\thefootnote}{8}\footnote{न \textendash\ परिव्राण्मुनिमुख्येषु तापसेषु तथैव च~। काषायवसनो वेषः कार्यस्त्वर्थवशेन वा}परिव्राण्मुनिशाक्यानां वासः काषायमिष्यते~।\\
नानाचित्राणि वासांसि कुर्यात्पाशुपतेष्वथ~॥~१३२

कुजातयश्च ये प्रोक्तास्तेषां चैव यथार्हतः~।\\
\renewcommand{\thefootnote}{9}\footnote{भ \textendash\ राजान्तःपुरकक्ष्यासु नियुक्ता ये नरा नृपैः}अन्तःपुरप्रवेशे च विनियुक्ता हि ये नराः~॥~१३३}
\end{quote}

\hrule

\vspace{2mm}
चीरमिति अवितता स्थूला च वृक्षत्वक्, वल्कलं तु तद्विपरीतम् , यथा भूर्जपत्रत्वक्, मृगादेश्चर्म~।

\newpage
% एकविंशोऽध्यायः १३१ 

\begin{quote}
{\na काषायकञ्चुकपटा: कार्यास्तेऽपि यथाविधि\renewcommand{\thefootnote}{1}\footnote{भ \textendash\ कर्तव्यास्ते प्रयोक्तृभिः~। कार्याणि कुशचीराणि वल्कलानि तथैवच~। व्रतिनां तापसानां तु ह्यन्यान्येवं विधानितु~।}~।\\
\renewcommand{\thefootnote}{2}\footnote{ज \textendash\ अवस्थान्तरमासाद्य स्त्रीणां वेषो भवेत्तथा}अवस्थान्तरतश्चैवं \renewcommand{\thefootnote}{3}\footnote{न\textendash\ सम्यक्}नृणां वेषो भवेदथ\renewcommand{\thefootnote}{4}\footnote{न \textendash\ तथा}~॥~१३४

\renewcommand{\thefootnote}{5}\footnote{भ \textendash\ सांग्रामिकस्तु वेषः स्याच्छूराणां संप्रकीर्तितः}वेषः सांग्रामिकश्चैव शूराणां संप्रकीर्तितः~।\\
विचित्रशस्त्रकवचो बद्धतूणो\renewcommand{\thefootnote}{6}\footnote{भ\textendash\ तूण, न \textendash\ तूण} धनुर्धरः~॥~१३५

\renewcommand{\thefootnote}{7}\footnote{ज \textendash\ विचित्रवेषः}चित्रो वेषस्तु कर्तव्यो नृपाणां नित्यमेव च\renewcommand{\thefootnote}{8}\footnote{च\textendash\ हि}~।\\
केवलस्तु भवेच्छुद्धो नक्षत्रोत्पातमङ्गले\renewcommand{\thefootnote}{9}\footnote{न\textendash\ मङ्गलैः}~॥~१३६

एवमेष भवेद्वेषो देशजातिवयोऽनुगः\renewcommand{\thefootnote}{10}\footnote{न \textendash\ वयोजातिगुणान्वितः}\\
उत्तमाधममध्यानां स्त्रीणां नृणामथापि च~॥~१३७

एवं वस्त्रविधिः कार्यः प्रयोगे नाटकाश्रये~।\\
नानावस्थां समासाद्य शुभाशुभ\renewcommand{\thefootnote}{11}\footnote{न कृतं ड \textendash\ कृतस्त्वथ}कृतस्तथा~॥~१३८

\renewcommand{\thefootnote}{12}\footnote{भ\textendash\ प्रतिशीर्षाणि च पुनर्नानारूपाणि योजयेत्}तथा प्रतिशिरश्चापि कर्तव्यं नाटकाश्रयम्~।\\
दिव्यानां मानुषाणां च देशजातिवयःश्रितम्\renewcommand{\thefootnote}{13}\footnote{भ \textendash\ यथावदनुपूर्वशः}~॥~१३१

\renewcommand{\thefootnote}{14}\footnote{भ \textendash\ पार्श्वगता}पार्श्वगता मस्तकिनस्तथा चैव किरीटिन:~।\\
\renewcommand{\thefootnote}{15}\footnote{न \textendash\ त्रिविधा मकुटा ज्ञेया दिव्याः पार्थिवसंश्रयाः}त्रिविधो मकुटो ज्ञेयो दिव्यपार्थिवसंश्रितः~॥~१४०}
\end{quote}

\hrule

\vspace{2mm}
(\underline{नक्षत्रेति}) नक्षत्रोत्पातप्रशमनार्थे यन्मङ्गलं, एतच्च नैमित्तिकस्य श्राद्ध\textendash

\newpage
% १३२ नाट्यशास्त्रम् 

\begin{quote}
{\na देवगन्धर्वयक्षाणां पन्नगानां सरक्षसाम्~।\\
\renewcommand{\thefootnote}{1}\footnote{न \textendash\ कार्या हि तैस्तु विहिता मकुटाः पार्श्ववर्तिनः}कर्तव्या नैकविहिता मुकुटा पार्श्वमौलयः\renewcommand{\thefootnote}{2}\footnote{ड \textendash\ मालिनः}~॥~१४१

उत्तमा ये च \renewcommand{\thefootnote}{3}\footnote{न \textendash\ दिव्याः स्युस्तेषां}दिव्यानां ते च कार्याः किरीटिनः~।\\
मध्यमा मौलिनश्चैव कनिष्ठाः \renewcommand{\thefootnote}{4}\footnote{प \textendash\ पार्श्व}शीर्षमौलिनः~॥~१४२

नराधिपानां कर्तव्या मस्तके मकुटा\renewcommand{\thefootnote}{5}\footnote{भ\textendash\ तथा मस्तकिनो} बुधैः~।\\
विद्याधराणां सिद्धानां चारणानां तथैव च~॥~१४३

\renewcommand{\thefootnote}{6}\footnote{ड \textendash\ ग्रन्थितं केशमकुटं कर्तव्यं तु}ग्रन्थिमत्केशमकृटाः कर्तव्यास्तु प्रयोक्तृभिः~।\\
\renewcommand{\thefootnote}{7}\footnote{भ\textendash\ देवदानवयक्षाणां}रक्षोदानवदैत्यानां पिङ्गकेशेक्षणानि हि\renewcommand{\thefootnote}{8}\footnote{न \textendash\ कृतानि हि}~।~१४४

\renewcommand{\thefootnote}{9}\footnote{भ\textendash\ यथा श्मश्रूणि}हरिच्छमश्रूणि च तथा \renewcommand{\thefootnote}{10}\footnote{प \textendash\ नानारूपाणि}मकुटास्यानि कारयेत्~।\\
\renewcommand{\thefootnote}{11}\footnote{ड \textendash\ उदात्ताः}उत्तमाश्चापि ये तत्र ते कार्याः पार्श्वमौलिनः~॥~१४५

\renewcommand{\thefootnote}{12}\footnote{ड\textendash\ तस्मात्तु मकुटाः श्लिष्टाः}कस्मात्तु मुकुटा: सृष्टाः प्रयोगे दिव्यपार्थिवे~।\\
केशानां\renewcommand{\thefootnote}{13}\footnote{ड \textendash\ छेदनं नेष्टं} छेदनं दृष्टं वेदवादे\renewcommand{\thefootnote}{14}\footnote{ढ \textendash\ पादे} यथाश्रुति~॥~१४६

भद्रीकृतस्य वा यज्ञे शिरसश्छादनेच्छया \renewcommand{\thefootnote}{15}\footnote{ड \textendash\ ईप्सया}~।\\
\renewcommand{\thefootnote}{16}\footnote{ड \textendash\ केशानां चाति}केशानामप्यदीर्घत्वात्स्मृतं मुकुटधारणम्~॥~१४७}
\end{quote}

\hrule

\vspace{2mm}
\noindent
देवार्चनादेरप्युपलक्षणम्~। प्रकृतिरूपं शिरः प्रतिशिरः 

\newpage
% एकविंशोऽध्यायः १३३ 

\begin{quote}
{\na सेनापतेः पुनश्चापि\renewcommand{\thefootnote}{1}\footnote{न \textendash\ अमात्यस्य} युवराजस्य चैैव हि~।\\
\renewcommand{\thefootnote}{2}\footnote{भ \textendash\ मस्तकेष्वर्धमकुटं प्रयोगे संप्रयोजयेत्}योजयेदर्धमकुटं महामात्राश्च ये नराः~॥~१४८

\renewcommand{\thefootnote}{3}\footnote{न \textendash\ अमात्यकञ्चुकिश्रेष्ठिवदूषकपुरोधसाम्}अमात्यानां कञ्चुकिनां तथा श्रेष्ठिपुरोधसाम्~।\\
\renewcommand{\thefootnote}{4}\footnote{ड \textendash\ वेष्टनं बन्धपट्टादि}वेष्टनाबद्धपट्टानि प्रतिशीर्षाणि कारयेत्~॥~१४९

पिशाचोन्मत्तभूतानां \renewcommand{\thefootnote}{5}\footnote{भ \textendash\ ताप\textendash\ सानां तथैव च}साधकानां तपस्विनाम्~।\\
अनिस्तीर्णप्रतिज्ञानां लम्बकेशं भवेच्छिरः\renewcommand{\thefootnote}{6}\footnote{भ \textendash\ केशशिरो भवेत् ढ \textendash\ केशं तु शीर्षकम्}~॥~१५०

शाक्यश्रोत्रियनिर्ग्रन्थपरिब्राड्दीक्षितेषु\renewcommand{\thefootnote}{7}\footnote{प \textendash\ भिक्षितेषु} च~।\\
शिरोमुण्डं तु कर्तव्यं यज्ञदीक्षान्वितेषु च\renewcommand{\thefootnote}{8}\footnote{न \textendash\ तु}~॥~१५१

तथा \renewcommand{\thefootnote}{9}\footnote{ढ \textendash\ वृत्तानुषङ्गेण}व्रतानुगं चैव शेषाणां लिङ्गिनां शिरः~।\\
मुण्डं वा कुञ्चितं वापि लम्बकेशमथापि वा~॥~१५२

धूर्तानां \renewcommand{\thefootnote}{10}\footnote{ड \textendash\ चापि}चैव कर्तव्यं ये च \renewcommand{\thefootnote}{11}\footnote{ड \textendash\ राजोप प \textendash\ रात्रोप}रात्र्युपजीविनः~।\\
\renewcommand{\thefootnote}{12}\footnote{ढ \textendash\ ये च शृङ्गारिणस्तेषां शिरः कुञ्चितमूर्धजम्}श्रृङ्गारचित्ताः पुरुषास्तेषां कुञ्चितमूर्धजाः~॥~१५३

बालानामपि कर्तव्यं त्रिशिखण्डविभूषितम्\renewcommand{\thefootnote}{13}\footnote{च \textendash\ शिरस्त्रिशिख\textendash\ भूषितम्}~।}
\end{quote}

\hrule

\vspace{2mm}
(\underline{वेष्टनेति}) वेष्टनार्थमाबद्धं पट्टमुष्णीषप्रायं येषु~। त्रिखण्डाश्चूलिकाः (त्रिशिखण्डम्)~।

\newpage
% १३४ नाट्यशास्त्रम् 

\begin{quote}
{\na जटामकुटबद्धं\renewcommand{\thefootnote}{1}\footnote{ड \textendash\ लम्बं} च मुनीनां तु\renewcommand{\thefootnote}{2}\footnote{च\textendash\ च} भवेच्छिरः~॥~१५४

चेटानामपि कर्तव्यं त्रिशिखं मुण्डमेव वा~।\\
\renewcommand{\thefootnote}{3}\footnote{ड \textendash\ विदूषकाणां कर्तव्यं खल्ली काकपदं तथा}विदूषकस्य खलतिः स्यात्काकपदमेव वा~।~१५५

शेषाणामर्थयोगेन देशजातिसमाश्रयम्\renewcommand{\thefootnote}{4}\footnote{न \textendash\ वयःश्रितम्}~।\\
शिरः प्रयोक्तृभिः कार्यं नानावस्थान्तराश्रयम्\renewcommand{\thefootnote}{5}\footnote{ज \textendash\ प्रयोगस्य वशानुगम्}~॥~९५६

\renewcommand{\thefootnote}{6}\footnote{ढ\textendash\ अतस्तैर्भूषणैश्चित्रै\textendash\ र्वस्त्रैर्माल्यैरथापि च (ड\textendash\ तथैव च)~। अवस्थानुकृतिः स्थाप्या प्रयोगरससंभवा}भूषणैर्वर्णकैर्वस्त्रैर्माल्यैश्चैव यथाविधि~।\\
एवं नानाप्रकारैस्तु\renewcommand{\thefootnote}{7}\footnote{प \textendash\ प्रकारांस्तु} बुद्ध्या वेषान्प्रकल्पयेत्~॥~१५७

पूर्वं तु प्रकृतिं स्थाप्य प्रयोगगुणसंभवाम्~।\\
स्त्रीणां वा पुरुषाणां वाप्यवस्थां\renewcommand{\thefootnote}{8}\footnote{ढ \textendash\ व्यवस्थां} प्राप्य तादृशीम्~॥~१५९

सर्वे भावाश्च दिव्यानां कार्या मानुषसंश्रयाः~।\\
तेषां चानिमिषत्वादि\renewcommand{\thefootnote}{9}\footnote{न\textendash\ त्वं च} नैव कार्यं प्रयोक्तृमिः~॥~९५१

इह भावरसाश्चैव दृष्टिभिः संप्रतिष्ठिताः~।\\
दृष्ट्यैव \renewcommand{\thefootnote}{10}\footnote{ड \textendash\ आपितो प \textendash\ प्रापितो}स्थापितो ह्यर्थः पश्चादङ्गैर्विभाव्यते~॥~१६०

एवं ज्ञेयाङ्गरचना नानाप्रकृतिसंभवा \\
\renewcommand{\thefootnote}{11}\footnote{ढ\textendash\ संजीवः प \textendash\ सज्जीवः}सजीव इति यः प्रोक्तस्तस्य वक्ष्यामि लक्षणम्~।~१६१}
\end{quote}

\hrule

\vspace{2mm}
(\underline{काक॒पदमिति}) काकपक्षवद्यत्न केशविच्छेदः~। 

\newpage
% एकविंशोऽध्यायः १३५ 

\begin{quote}
{\na यः प्राणिनां प्रवेशो वै \renewcommand{\thefootnote}{1}\footnote{ढ \textendash\ स संजीव इति स्मृतः}सजीव इति संज्ञितः~।\\
चतुष्पदोऽथ द्विपदस्तथा चैवापदः स्मृतः\renewcommand{\thefootnote}{2}\footnote{च \textendash\ स्मृताः}~॥~१६२

\renewcommand{\thefootnote}{3}\footnote{ड \textendash\ उरगाह्यपदो ज्ञेया द्विपदा खगमानुषाः}उरगानपदान् विद्याद् द्विपदान्खगमानुषान्~।\\
\renewcommand{\thefootnote}{4}\footnote{ज \textendash\ ग्राम्यारण्याश्च च \textendash\ ग्रामारण्याश्च ढ \textendash\ ग्राम्य मुण्डाश्च}ग्राम्या आरण्याः पशवो \renewcommand{\thefootnote}{5}\footnote{प \textendash\ ज्ञेयास्ते च चतुष्पदाः}विज्ञेयाः स्युश्चतुष्पदाः~॥~१६३

ये ते तु \renewcommand{\thefootnote}{6}\footnote{भ\textendash\ युद्धे संफेटे ह्यवरोधे}युद्धसंफेटैरुपरोधैस्तथैव च~।\\
नानाप्रहरणोपेताः प्रयोज्या नाटके बुधैः\renewcommand{\thefootnote}{7}\footnote{ढ \textendash\ नाटकाश्रये}~॥~१६४

आयुधानि च कार्याणि\renewcommand{\thefootnote}{8}\footnote{ढ \textendash\ वर्माणि} \renewcommand{\thefootnote}{9}\footnote{ड \textendash\ तज्ज्ञैः सम्यक्}पुरुषाणां प्रमाणतः~।\\
तान्यहं \renewcommand{\thefootnote}{10}\footnote{च \textendash\ संप्रवक्ष्यामि}वर्तयिष्यामि यथापुस्तप्रमाणतः\renewcommand{\thefootnote}{11}\footnote{च \textendash\ यावदनु\textendash\ पूर्वशः ज \textendash\ यथायुक्तिप्रमाणतः}~॥~१६५

\renewcommand{\thefootnote}{12}\footnote{भ \textendash\ भिण्डी}भिण्डिर्द्वादशतालः स्याद्दश कुन्तो भवेदथ\renewcommand{\thefootnote}{13}\footnote{भ \textendash\ विधीयते}~।\\
अष्टौ शतघ्नी शूलं\renewcommand{\thefootnote}{14}\footnote{ढ \textendash\ शूलश्च} च तोमरः शक्तिरेव वा~॥~१६६

अष्टौ \renewcommand{\thefootnote}{15}\footnote{ज \textendash\ तालं}ताला धनुर्ज्ञेय\renewcommand{\thefootnote}{16}\footnote{न \textendash\ आवापो प \textendash\ आमर्षोऽपि तथैव च} मायामोऽस्य द्विहस्तकः~।}
\end{quote}

\hrule

\vspace{2mm}
\underline{प्रहरणोपेता} इति युद्धोपयोगिन इत्यर्थः~। तथा च नागास्त्रे दत्ते सर्पाकृतिः प्रदर्शनाया, एवं नृसिंहास्त्रे तदाकृतिरित्यादि~। (आयुधानां प्रमाणं) दर्शयति (\underline{भिण्डिरिति}~। वज्रं\ldots ..चतुस्तालम्)~। चक्रमिति खड्गादियुद्धेऽपवारणम्~।

\newpage
% १३६ नाट्यशास्त्रम्

\begin{quote}
{\na शरो गदा च \renewcommand{\thefootnote}{1}\footnote{च\textendash\ चक्रं}वज्रा च चतुस्तालं विधीयते\renewcommand{\thefootnote}{2}\footnote{न \textendash\ भवेदथ}~॥~१६७

अङ्गुलानि त्वसिः कार्यश्चत्वारिंशत्प्रमाणतः~।\\
\renewcommand{\thefootnote}{3}\footnote{भ \textendash\ चक्रं च द्वादश ज्ञेयं}द्वादशाङ्गुलकं चक्रं ततोऽर्धं प्रास इष्यते~॥~१६८

\renewcommand{\thefootnote}{4}\footnote{न \textendash\ प्रासार्धं}प्रासवत्पट्टसं विद्या\renewcommand{\thefootnote}{5}\footnote{न \textendash\ दण्डकश्चैव विंशकः प \textendash\ दण्डकस्तस्य विंशकः}द्दण्डश्चैव तु विंशतिः~।\\
\renewcommand{\thefootnote}{6}\footnote{न \textendash\ कणयश्च भवेद्विंश\textendash\ त्यङ्गुलैः परिमाणतः}विंशतिः \renewcommand{\thefootnote}{7}\footnote{प\textendash\ कम्पणः}कणयुश्चैव ह्यङ्गुलानि प्रमाणतः~॥~१६९

षोडशाङ्गुलविस्तीर्णं \renewcommand{\thefootnote}{8}\footnote{न \textendash\ खप्पणं भ\textendash\ करवालः प्रघट्टितः ढ \textendash\ चर्मकार्यं द्विहस्तकम्~। षोडशाङ्गुलविस्तीर्णं सबलं सप्रघण्टिकम् म \textendash\ सबलं संप्रकीर्तितम्}सबलं संप्रघण्डिकम्~।\\
त्रिंशदङ्गुलमानेन कर्तव्यं खेटकं बुधैः~॥~१७०

जर्जरो दण्डकाष्ठं च तथैव प्रतिशीर्षकम्~।\\
छत्रं च चामरं चैव \renewcommand{\thefootnote}{9}\footnote{ढ \textendash\ भट्टो}ध्वजो शृङ्गार एव च~॥~१७१

यत्किंचिन्मानुषे लोके द्रव्यं पुंसां प्रयोजकम्\renewcommand{\thefootnote}{10}\footnote{ड\textendash\ प्रयोगजम् च \textendash\ प्रयोजयेत्}~।\\
\renewcommand{\thefootnote}{11}\footnote{ढ \textendash\ तत्सर्वं तूपकरणं नाट्येऽस्मिन्संविधीयते}यच्चोपकरणं सर्वं नाट्ये तत्संप्रकीर्तितम्~॥~१७२

यद्यस्य विषयप्राप्तं \renewcommand{\thefootnote}{12}\footnote{न \textendash\ तेनोक्तं ढ \textendash\ तेन प्रोक्तं तु}तेनोह्यं तस्य लक्षणम्~।\\
\renewcommand{\thefootnote}{13}\footnote{ज\textendash\ जर्जरं}जर्जरे दण्डकाष्ठे च संप्रवक्ष्यामि लक्षणम्~॥~१७३}
\end{quote}

\hrule

\vspace{2mm}
\underline{यद्यस्येति}~। यस्य शास्त्रस्य यद्विषयीभूतं पुरुषस्य वा तदनुसारेण तस्यवस्तुनो \underline{लक्षणमूह्य}मिति~। ऊह्यशब्दे भेदमाह, परिपूर्णलक्षणमुपजीव्यम्,

\newpage
% एकविंशोऽध्यायः १३७ 

\begin{quote}
{\na \renewcommand{\thefootnote}{1}\footnote{ज \textendash\ माहेन्द्रे वै ध्वजे प्रोक्तं लक्षणं विश्वकर्मणा}माहेन्द्रा \renewcommand{\thefootnote}{2}\footnote{ढ \textendash\ ये\ldots लक्षणे}वै ध्वजाः प्रोक्ता लक्षणैर्विश्वकर्मणा~।\\
\renewcommand{\thefootnote}{3}\footnote{ड \textendash\ तेषां}एषामन्यतमं\renewcommand{\thefootnote}{4}\footnote{भ \textendash\ एकतमं} कुर्याज्जर्जुरं दारुकर्मतः\renewcommand{\thefootnote}{5}\footnote{भ\textendash\ कर्मजम्}~॥~१७४

अथवा \renewcommand{\thefootnote}{6}\footnote{न \textendash\ वृक्षजातस्य म \textendash\ वृक्ष\textendash\ जातः स्यात्}वृक्षयोनिः स्यात्प्ररोहो वापि जर्जरः~।\\
वेणुरेव \renewcommand{\thefootnote}{7}\footnote{भ \textendash\ तु वै श्रेष्ठो वक्ष्यते ह्यस्य}भवेच्छ्रेवेष्ठस्तस्य वक्ष्यामि लक्षणम्~॥~१७५

श्वेतभूम्यां तु यो जातः पुष्यनक्षत्रजस्तथा~।\\
संग्राह्यो \renewcommand{\thefootnote}{8}\footnote{ज \textendash\ विधिना}वै भवेद्वेणुर्जर्जरार्थे\renewcommand{\thefootnote}{9}\footnote{ज \textendash\ जर्जरार्थं} प्रयत्नतः~॥~१७६

\renewcommand{\thefootnote}{10}\footnote{भ \textendash\ प्रमाणतः}प्रमाणमङ्गुलानां तु शतमष्टोत्तरं भवेत्~।\\
पञ्चपर्वा चतुर्ग्रन्थिस्तालमात्रस्तथैव च~॥~१७७

स्थूलग्रन्थिर्न कर्तव्यो न शाखी न च कीटवान्~।\\
न \renewcommand{\thefootnote}{11}\footnote{भ \textendash\ क्षतः क्रिमिपार्श्वश्च}कृमिक्षतपर्वा च न हीनश्चान्यवेणुभिः~॥~१७८

\renewcommand{\thefootnote}{12}\footnote{ज \textendash\ अक्तं तु मधुसार्पिभ्यां}मधुसर्पिस्सर्षपाक्तं माल्यधूपपुरस्कृतम्~।\\
उपास्य विधिवद्वेणुं \renewcommand{\thefootnote}{13}\footnote{न \textendash\ प्रकुर्यात्}गृहणीयाज्जर्जरं प्रति~॥~१७९}
\end{quote}

\hrule

\vspace{2mm}
(यथा) खङ्गलक्षणेऽप्युपजीव्यमाने लोहादिनिर्मितत्वमप्यूह्यते~। तच्च तस्मान्नाट्योपयोगरूपमूहापोहाभ्यां कर्तव्यमिति~। \underline{लक्षणानीति} (\underline{लक्षणैरिति}?) विश्चकर्ममते बहुभेदं महेन्द्रध्वजस्य लक्षणमुक्तमित्यर्थः~। न होनश्चेति अन्यवेणुसंघर्षेऽपूर्णोऽवयवतः स नेष्यत इत्यर्थः~।

\lfoot{18}

\newpage
\lfoot{}
% १३८ नाट्यशास्त्रम् 

\begin{quote}
{\na \renewcommand{\thefootnote}{1}\footnote{प \textendash\ विधिः कार्यः क्रमेणैव}यो विधिर्यः क्रमश्चैव माहेन्द्रे तु ध्वजे स्मृतः~।\\
स जर्जरस्य\renewcommand{\thefootnote}{2}\footnote{भ \textendash\ जर्जरे तु} कर्तव्य: \renewcommand{\thefootnote}{3}\footnote{च \textendash\ पुष्य ह \textendash\ पुण्य}पुष्यवेणुसमाश्रयः~।~१८०

भवेद्यो \renewcommand{\thefootnote}{4}\footnote{म \textendash\ दीर्घपत्रस्तु न \textendash\ नित्यं हि पीतपत्रस्तु}दीर्घपर्वा तु तनुपत्रस्तथैव च~।\\
पर्वाग्र\renewcommand{\thefootnote}{5}\footnote{ज \textendash\ वर्तुलः न \textendash\ मण्डलः}तण्डुलश्वैव पुष्यवेणुः \renewcommand{\thefootnote}{6}\footnote{न \textendash\ प्र}स कीर्तितः~॥~१८१

विधिरेष\renewcommand{\thefootnote}{7}\footnote{प \textendash\ एवं} मया प्रोक्तो जर्जरस्य प्रमाणतः\renewcommand{\thefootnote}{8}\footnote{च \textendash\ महात्मनः ज \textendash\ च लक्षणे}~।\\
अत ऊर्ध्वं\renewcommand{\thefootnote}{9}\footnote{भ \textendash\ परं} प्रवक्ष्यामि दण्डकाष्ठस्य लक्षणम्~॥~१८२

\renewcommand{\thefootnote}{10}\footnote{भ \textendash\ दण्डकाष्ठं तु बैल्वं स्यात्कापित्थं वांश्यमेव वा}कपित्थबिल्ववंशेभ्यो दण्डकाष्ठं भवेदथ\renewcommand{\thefootnote}{11}\footnote{ढ \textendash\ सदा ज \textendash\ काष्ठाविधिस्तथा}~।\\
\renewcommand{\thefootnote}{12}\footnote{च \textendash\ वक्रत्वेन}वक्रं चैवहि कर्तव्यं\renewcommand{\thefootnote}{13}\footnote{न \textendash\ तु तत्कार्ये} त्रिभागे लक्षणान्वितम्~।~१८३

कीटैर्नोपहतं यच्च व्याधिना न च\renewcommand{\thefootnote}{14}\footnote{च \textendash\ नैव} पीडितम्~।\\
मन्दशाखं भदेद्यच्च दण्डकाष्ठं तु तद्भवेत्\renewcommand{\thefootnote}{15}\footnote{च \textendash\ तदुच्यते}~॥~१८४

यस्त्वेभिर्लक्षणैर्हीनं दण्डकाष्ठं सजर्जरम्~।\\
कारयेत्स त्वपचयं महान्तं प्राप्नुयाद्ध्रुवम्\renewcommand{\thefootnote}{16}\footnote{भ \textendash\ कारयेत्स तु नानन्दं कदाचित्प्राप्वुयान्नरः}~।~१८५

\renewcommand{\thefootnote}{17}\footnote{न \textendash\ तथा च प्रतिशीर्षस्य न \textendash\ तथाशीर्षविधानार्थं}अथ शीर्षविभागार्थं \renewcommand{\thefootnote}{18}\footnote{भ \textendash\ किटी}घटी कार्या प्रयत्नतः\renewcommand{\thefootnote}{19}\footnote{भ \textendash\ स्वमानतः ढ \textendash\ तु मानतः}~।}
\end{quote}

\hrule

\vspace{2mm}
पुष्यवेणुं व्याचष्टे \underline{भवेद्वेणुर्दीर्घपर्वेति} लक्षणेन लेखा \underline{शीर्षविभागा} इति यत्र द्विशिरास्त्रिशिरा इत्यादि दृश्यते, यत्र वा

\newpage
% एकविंशोऽध्यायः १३९

\begin{quote}
{\na स्वप्रमाणविनिर्दिष्टा द्वात्रिंशत्यङ्गुलानि वै\renewcommand{\thefootnote}{1}\footnote{ज \textendash\ वा}~॥~१८६

बिल्वमध्येन\renewcommand{\thefootnote}{2}\footnote{ज \textendash\ कल्केन} कर्तव्या घटी \renewcommand{\thefootnote}{3}\footnote{भ\textendash\ चरि न\textendash\ शीर्ष ज \textendash\ चिर}सिरसमाश्रया~।\\
स्विन्नेन बिल्वकल्केन द्रवेण च समन्विता\renewcommand{\thefootnote}{4}\footnote{भ \textendash\ समाहिता प \textendash\ समन्वितम्}~॥~१८७

भस्मना वा तुषैर्वापि कारयेत्प्रतिशीर्षकम्\renewcommand{\thefootnote}{5}\footnote{ज \textendash\ प्रतिशीर्षाणि कारयेत्}~।\\
\renewcommand{\thefootnote}{6}\footnote{भ \textendash\ संस्राव्य तान्}संछाद्य तु ततो वस्त्रैर्बिल्वदिग्धैर्घटाश्रयैः~॥~१८८

बिल्वकल्केन चीरं तु दिग्ध्वा संयोजयेद्धटीम्~।\\
न स्थूलां \renewcommand{\thefootnote}{7}\footnote{ढ \textendash\ न तनुम् चैव न सुद्वां चैव कारयेत् न \textendash\ न च तां तन्वीं मृद्वीं नैव च॒ (च \textendash\ न) भ \textendash\ न तनुं चैव दीर्घां}नानतां तन्वीं दीर्घां चैव न कारयेत्~॥~१८९

\renewcommand{\thefootnote}{8}\footnote{भ \textendash\ सुशुष्कायां ततस्तस्यामनिलातपयोगतः छेद्यं बुधाः प्रकुर्वीत लक्षणं कृतिनिर्मितम् ढ \textendash\ शुष्कायां तु ततस्तस्यामनिलातपयोगतः}तस्यामातपशुष्काया सुशुष्कायामथापि वा~।\\
छेद्यं बुधाः प्रकुर्वन्ति विधिदृष्टेन कर्मणा~॥~१९०

सुतीक्ष्णेन तु शस्त्रेण अर्धार्धं प्रविभज्य च~।\\
स्वप्रमाणविनिर्दिष्टं \renewcommand{\thefootnote}{9}\footnote{ढ\textendash\ ललाटाकृतिकोणजम्}ललाटकृतकोणकम्~॥~१९१

अर्धाङुलं ललाटं तु कार्यं छेद्यं षडङ्गुलम्~।}
\end{quote}

\hrule

\vspace{2mm}
\noindent
निजशिर एवाच्छाद्य शिरोऽन्तरं प्रदर्श्यते~। प्रतिपादप्रतिहस्तादेष एव कल्पः~। बिल्बस्य मध्यमज्जा \underline{सिर}श्च वृक्षत्वगादिः~। कथं सा कर्तव्येत्याह \underline{स्विन्नेने}ति भस्मना तुषचूर्णेन वा सुसमाहिता छादितच्छिद्रेत्यर्थः~। न स्थुलामिति गुरुत्वभयात्~। \underline{नानता}मिति पेलवात्, छेद्यमित्यवयवच्छिद्रोत्पादनार्थम्~।

\newpage
% १४० नाट्यशास्त्रम् 

\begin{quote}
{\na अर्धार्धमङ्गुलं छेद्यं \renewcommand{\thefootnote}{1}\footnote{न \textendash\ कटे च}कटयोद्वर्यङ्गुलं\renewcommand{\thefootnote}{2}\footnote{ज \textendash\ त्र्यङ्गलं} भवेत्~॥~१९२\\
कटान्ते कर्णनालस्य \renewcommand{\thefootnote}{3}\footnote{भ \textendash\ तालस्य}छेद्यं \renewcommand{\thefootnote}{4}\footnote{च \textendash\ च विधिमङ्गुलम् न \textendash\ त्वधिकं}द्वयधिकमङ्कुलम्~।\\
\renewcommand{\thefootnote}{5}\footnote{च \textendash\ अङ्गुलं}त्र्यङ्गुलं कर्णविवरं\renewcommand{\thefootnote}{6}\footnote{भ \textendash\ विस्तारं} \renewcommand{\thefootnote}{7}\footnote{न \textendash\ सदैव}तथा स्याच्छेद्यमेव हि~॥~१९३\\
\renewcommand{\thefootnote}{8}\footnote{भ \textendash\ तस्य चैवावटः कार्यो समा वै द्वाद्वशाङ्गुला}ततश्चैवावटुः कार्या सुसमा द्वादशाङ्कला~।\\
\renewcommand{\thefootnote}{9}\footnote{भ \textendash\ घटीच्छेद्यकृतं ह्येतद्विधानं परिकीर्तितम्}घट्यां ह्येतत्सदा च्छेद्ये \renewcommand{\thefootnote}{10}\footnote{विधिसंविहितं मया}विधानं विहितं मया~॥~१९४\\
\renewcommand{\thefootnote}{11}\footnote{भ \textendash\ तस्या उपरिगाः}तस्योपरिगताः\renewcommand{\thefootnote}{12}\footnote{ढ \textendash\ ततः} कार्या मुकुटा बहुशिल्पजाः\renewcommand{\thefootnote}{13}\footnote{भ \textendash\ विविधाश्रयाः}~।\\
नानारत्नप्रतिच्छन्ना \renewcommand{\thefootnote}{14}\footnote{भ \textendash\ गुरु}बहुरूपोपशोभिताः~॥~१९५\\
तथोपकरणानीह नाट्ययोगकृतानि वै\renewcommand{\thefootnote}{15}\footnote{न \textendash\ युक्तिकृतानि च}~।\\
\renewcommand{\thefootnote}{16}\footnote{भ \textendash\ नाना\textendash\ विधान}बहुप्रकारयुक्तानि कुर्वीत प्रकृतिं प्रति~॥~१९६\\
यत्किंचिदस्मिन् लोके तु\renewcommand{\thefootnote}{17}\footnote{छ\textendash\ अथ} \renewcommand{\thefootnote}{18}\footnote{ज\textendash\ सचराचरसंकज्ञिते}चराचरसमन्विते~।\\
विहितं कर्म शिल्पं वा \renewcommand{\thefootnote}{19}\footnote{प \textendash\ तद्रूपकरणं भवेत्}तत्तदेवापकरणं स्मृतम्~॥~१९७\\
यद्यस्य \renewcommand{\thefootnote}{20}\footnote{भ \textendash\ विषयं प्राप्तं तत्तस्मिन् तत्क्वचिद्भवेत्~।\ldots करणं प्रति ढ \textendash\ विषयप्राप्यं स तस्मिस्त्वधिगच्छति~। नान्यतः \ldots करणाश्रयम्~॥ प\textendash\ विषयं प्राप्तः स}विषयं प्राप्तं तत्तदेवाभिगच्छति~।\\
नास्त्यन्तः पुरुषाणां हि नाट्योपकरणाश्रये~॥~१९८}
\end{quote}

\hrule

\vspace{2mm}
{\small \underline{अव}टुरिति कर्णशष्कुली~।}

\newpage
% एकविंशोऽध्यायः १४१ 

\begin{quote}
{\na यद्येनोत्पादितं कर्म शिल्पयोग\renewcommand{\thefootnote}{1}\footnote{च \textendash\ योगः}क्रियापि वा~।\\
\renewcommand{\thefootnote}{2}\footnote{भ \textendash\ सा तस्यैव क्रिया कार्या}तस्य तेन कृता सृष्टिः प्रमाणं लक्षणं तथा~॥~१९९

या \renewcommand{\thefootnote}{3}\footnote{प \textendash\ कार्णायसभूयिष्ठा कृता भूमिः भ \textendash\ काष्टा या च भूयिष्ठा कृता भूमिर्महत्तरा} काष्ठयन्त्रभूयिष्ठा कृता सृष्टिर्महात्मना\renewcommand{\thefootnote}{4}\footnote{ढ \textendash\ महत्तरा}~।\\
\renewcommand{\thefootnote}{5}\footnote{भ \textendash\ नाट्ययोगे न सास्माकं}न सास्माकं नाट्ययोगे कस्मात्खेदावहा हि सा~॥~२००

यद्द्रव्यं जीवलोके तु नानालक्षणलक्षितम्\renewcommand{\thefootnote}{6}\footnote{भ \textendash\ संयुतम्}~।\\
तस्यानुकृतिसंस्थानं नाट्योपकरणं भवेत्~॥~२०१

प्रासाद\renewcommand{\thefootnote}{7}\footnote{ढ \textendash\ कृत}गृहयानानि नानाप्रहरणानि च~।\\
\renewcommand{\thefootnote}{8}\footnote{भ\textendash\ किं ढ \textendash\ न शक्यानि तथा}न शक्यं तानि वै कर्तुं यथोक्तानीह लक्षणैः~॥~२०२

लोकधर्मी भवेत्त्वन्या नाट्यधर्मी तथापरा\renewcommand{\thefootnote}{9}\footnote{भ \textendash\ तथापि वा}~।\\
\renewcommand{\thefootnote}{10}\footnote{ढ प्रभावो प \textendash\ प्रभवो}स्वभावो लोकधर्मी तु \renewcommand{\thefootnote}{11}\footnote{प\textendash\ विकारो नाट्यमेव हि}विभावो नाट्यमेव हि~॥~२०३

\renewcommand{\thefootnote}{12}\footnote{न \textendash\ आयसं न च भ \textendash\ मृण्मयं न च ढ \textendash\ लोहादिभिर्न}आयसं तु न कर्तव्यं \renewcommand{\thefootnote}{13}\footnote{ढ \textendash\ नगसारमयं न च}न च सारमयं तथा~।\\
नाट्योपकरणं तज्ज्ञैर्गुरुखेदकरं भवेत्\renewcommand{\thefootnote}{14}\footnote{ज \textendash\ गुरुत्वात् स्वेदजं हि तत्}~॥~२०४

\renewcommand{\thefootnote}{15}\footnote{भ \textendash\ जतुकाष्ठमयैर्भाण्डैश्चर्मवेणुदलैस्तथा ढ \textendash\ जतुकाष्ठचर्मवस्त्रभाण्डवेणुदलैस्तथा}काष्ठचर्मसु वस्त्रेषु \renewcommand{\thefootnote}{16}\footnote{छ \textendash\ तन्तु (तनु?)}जतुवेणुदलेषु च~।}
\end{quote}

\hrule

\vspace{2mm}
\noindent
\underline{महात्मने}ति विश्वकर्मणा~। \underline{विभाव} इति भावनामात्रमित्यर्थः~।

\newpage
% १४२ नाट्यशास्त्रतम् 

\begin{quote}
{\na नाट्योपकरणानीह लघुकर्माणि\renewcommand{\thefootnote}{1}\footnote{ढ \textendash\ कर्मणि} कारयेत्~॥~२०५

चर्मवर्म\renewcommand{\thefootnote}{2}\footnote{भ \textendash\ विवर्जं च प्रासादशिबिकास्तथा} ध्वजाः शैलाः प्रासादा \renewcommand{\thefootnote}{3}\footnote{ढ \textendash\ शिखरा\textendash\ स्तथा}देवतागृहाः~।\\
हयवारणयानानि विमानानि गृहाणि च\renewcommand{\thefootnote}{4}\footnote{भ \textendash\ वा}~॥~२०६

पूर्वं वेणुदलैः कृत्वा \renewcommand{\thefootnote}{5}\footnote{न \textendash\ कृति र्भावसमाश्रया प \textendash\ ह्याकृतिं स्वाङ्गसंध्रयाम् ज \textendash\ आकृतीर्भावसंश्रयाः}कृतीर्भावसमाश्रयाः~।\\
\renewcommand{\thefootnote}{6}\footnote{भ \textendash\ चेषाकृति समाश्रयाः ( च \textendash\ नानावर्णैस्ततो वस्त्रैच्छादयेद्रूपकारणान् ढ \textendash\ ततः सुरक्तैः}ततः सुरङ्गैराच्छाद्य वस्त्रै सारूप्यमानयेत्~॥~२०७

अथवा यदि \renewcommand{\thefootnote}{7}\footnote{ज \textendash\ वर्णानां}वस्त्राणामसान्निध्यं भवेदिह\renewcommand{\thefootnote}{8}\footnote{ढ \textendash\ तद्विधानमसंभवम्}~।\\
\renewcommand{\thefootnote}{9}\footnote{ज \textendash\ तालीमयैः ढ \textendash\ तालीयजैः कोलजैर्वा}तालीयैवा किलिञ्जैर्वा श्लक्ष्णौ \renewcommand{\thefootnote}{10}\footnote{ढ \textendash\ वस्तैः न \textendash\ वस्तु}र्वस्त्रक्रिया भवेत्~॥~२०८

\renewcommand{\thefootnote}{11}\footnote{भ \textendash\ चर्मकाष्ठकृतैर्वापि तृणवेणुदलैरपि~। जतुभाण्डकृतैश्चैव नानारूपाणि कारयेत्}तथा प्रहरणानि स्युस्तृणवेणुदलादिभिः~।\\
जतुभाण्डक्रियाभिश्च नानारूपाणि नाटके~॥~२०९

\renewcommand{\thefootnote}{12}\footnote{ढ\textendash\ प्रतिपादौ प्रतिशिरः प्रतिहस्तौ प्रतित्वचम्~। तृणजैः कीलजै र्भाण्डैः सरूपाणि तु (णीह \textendash\ प) कारयेत्}प्रतिपादं प्रतिशिरः प्रतिहस्तं प्रतित्वचम्~।\\
तृणैः किलिञ्जैर्भाण्डैर्वा \renewcommand{\thefootnote}{13}\footnote{भ \textendash\ तद्रूपाणीह}सारूप्याणि तु कारयेत्~॥~२९०}
\end{quote}

\hrule

\vspace{2mm}
\underline{लघुकर्माणीति} येषु क्रियमाणेषु लाघवेन क्रिया संपद्यते~।\\

\underline{तालीयैरिति} तालपत्रैः~। \underline{भाण्डैरिति} अलाबुदलखङ्गादिभिः~।

\newpage
% एकविंशोऽध्यायः १४३ 

\begin{quote}
{\na \renewcommand{\thefootnote}{1}\footnote{भ \textendash\ यद्यस्य यादृशं कर्म तद्रूपं गुणसंयुतम्~। मृण्मयं तमुपाकृत्य यद्रृपं तत्प्रकारयेत्}यद्यस्य सदृशं रूपं सारूप्युगुणसंभवम्~।\\
मण्मयं \renewcommand{\thefootnote}{2}\footnote{ज \textendash\ तत्र}तत्तु कृत्स्नं तु नानारूपं तु कारयेत्~॥~२११

\renewcommand{\thefootnote}{3}\footnote{च \textendash\ भेण्ड}\renewcommand{\thefootnote}{*}\footnote{भाण्डः असारं लघु दारु, भेण्ड इति साधुशब्दः स्यात्~। {\qt बेण्डु} इति भाषासु प्रयोगः~।}भाण्डवस्रमधूच्छिष्टैर्लाक्षयाभ्रदलेनच~।\\
\renewcommand{\thefootnote}{4}\footnote{ढ \textendash\ नगाः}नागास्ते विविधाः कार्या ह्यतसीशणबिल्वजैः\renewcommand{\thefootnote}{5}\footnote{ढ \textendash\ चर्मवर्मध्वजास्तथा}~॥~२१२

नानाकुसुमजातीश्च\renewcommand{\thefootnote}{6}\footnote{न \textendash\ जात्यश्च} फलानि विविधानि च~।\\
भाण्डवस्त्रमधृच्छिष्टैर्लाक्षया वापि कारयेत्~।~२१३

भाण्डवस्त्रमधृच्छिष्टैस्ताम्रपत्रैस्तथैव च~।\\
\renewcommand{\thefootnote}{7}\footnote{भ \textendash\ नीलीरागेण चान्यैश्च सस्यशाकेन चैव हि ढ \textendash\ तत्सम्यङ्नीलरागेण}सम्यक्च नीलीरागेणाप्यभ्रपत्रेण चैव हि~॥~२१४

रञ्जितेनाभ्रपत्रेण \renewcommand{\thefootnote}{8}\footnote{भ \textendash\ भित्तयश्चैव कारयेत्}मणीश्चेव प्रकारयेत्~।\\
\renewcommand{\thefootnote}{9}\footnote{भ \textendash\ अपाश्रयं तथा चैषां शुक्लभेण्डेन चैव हि}उपाश्रयमथाप्येषां शुल्बवङ्गेन\renewcommand{\thefootnote}{10}\footnote{प \textendash\ वङ्गैश्च} कारयेत्~॥~२१५

विविधा मकुटा दिव्या\renewcommand{\thefootnote}{11}\footnote{भ \textendash\ दीर्घाः} पूर्वं ये गदिता मया~।\\
\renewcommand{\thefootnote}{12}\footnote{भ \textendash\ ताम्र}तेऽभ्रपत्रोज्ज्वलाः कार्या \renewcommand{\thefootnote}{13}\footnote{ड \textendash\ मणिक्यालोक भ \textendash\ मणिप्रद्योत}मणिव्यालोपशोभिताः~॥~२९६}
\end{quote}

\hrule

\vspace{2mm}
मधूच्छिष्टं सित्थकम्~।

\newpage
% १४४ नाट्यशास्त्रम् 

\begin{quote}
{\na न शास्त्रप्रभवं कर्म \renewcommand{\thefootnote}{1}\footnote{भ . प्रोक्तमेषां विधानतः}तेषां हि समुदाहृतम्~।\\
\renewcommand{\thefootnote}{2}\footnote{ड \textendash\ विचार्य }आचार्यबुध्द्या कर्तव्यमूहापोहप्रयोजितम्\renewcommand{\thefootnote}{3}\footnote{भ\textendash\ साधितम्}~॥~२१७

\renewcommand{\thefootnote}{4}\footnote{न\textendash\ एवं च \textendash\ एवमल्प }एष मर्त्यक्रिया \renewcommand{\thefootnote}{5}\footnote{भ \textendash\ कृतो }योगो \renewcommand{\thefootnote}{6}\footnote{भ \textendash\ भविष्यो यो मयोदितः}भविष्यत्कल्पितो\renewcommand{\thefootnote}{7}\footnote{न \textendash\ कथितो} मया~।\\
\renewcommand{\thefootnote}{8}\footnote{ढ \textendash\ यस्मात् ज \textendash\ तस्मात् }कस्मादल्पबलत्वं हि \renewcommand{\thefootnote}{9}\footnote{भ \textendash\ मानृषेषु}मनुष्येषु भविष्यति~।~२१८

\renewcommand{\thefootnote}{10}\footnote{भ \textendash\ मर्त्यानामल्पशक्तित्वान्न च वागङ्गचेष्टितम् ज \textendash\ मर्त्यानामपशक्तीनां न चातीवाङ्गचेष्टितम् (ड \textendash\ भवेद) }मर्त्यानामपि नो शक्या विभावाः\renewcommand{\thefootnote}{11}\footnote{न \textendash\ विभवाः} सर्वकाञ्चनाः~।\\
नेष्टाः सुवर्णरत्नैस्तु\renewcommand{\thefootnote}{12}\footnote{भ \textendash\ रत्नानि प \textendash\ रत्नेषु} मुकुटा भूषणानि वा\renewcommand{\thefootnote}{13}\footnote{न \textendash\ च}~॥~२१९

युद्धे नियुद्धे नृत्ते वा वृष्टिव्यापारकर्मणि~।\\
गुरुभावावसन्नस्य \renewcommand{\thefootnote}{14}\footnote{ज \textendash\ न व्या\textendash\ यतविचेष्टना }स्वेदो मूर्छा च जायते~॥~२२०

\renewcommand{\thefootnote}{15}\footnote{च मूर्छ्याभिहते जन्तौ प्रयोगो न भविष्यति ज \textendash\ मूर्छाखेदश्रमा\textendash\ र्तस्य}स्वेदमूर्छाक्लमार्तस्य प्रयोगस्तु विनश्यति~।\\
प्राणात्ययः कदाचिच्च भवेद्व्यायतचेष्टया\renewcommand{\thefootnote}{16}\footnote{ज \textendash\ चेष्टनात् च चेष्टिते }~॥~२२१

\renewcommand{\thefootnote}{17}\footnote{भ\textendash\ तस्माद्धि ताम्रपत्रेण मुकुटादि} प्रकारयेत्~। खच्छन्दनीलरागेण अभ्रपत्रेण चित्रितम्तस्मात्ताम्रमयैः पत्रै\renewcommand{\thefootnote}{18}\footnote{न \textendash\ अभ्रगैः}रभ्रके रञ्जितैरपि~।\\
भण्डैरपि मधूच्छिष्टैः कार्याण्याभरणानि तु\renewcommand{\thefootnote}{19}\footnote{भ\textendash\ च }~॥~२२२

एवं लोकोपचारेण स्वबुद्धिविभवेन च~।\\
नाट्योपकरणानीह बुधः सम्यक् प्रयोजयेत्~॥~२२३}
\end{quote}

\newpage
% एकविंशोऽध्यायः १४५ 

\begin{quote}
{\na \renewcommand{\thefootnote}{1}\footnote{भ \textendash\ मोक्तव्यं नायुधं रङ्गे न छेद्यं न च ताडनम्~। प्रादेशमात्रंगृह्णीयात्सं\textendash\ ज्ञार्थं शस्त्रमेव च}न भेद्यं नैव च च्छेद्यं न प्रहर्तव्यमेव च~।\\
रङ्गे प्रहरणैः कार्यं संज्ञामात्रं तु कारयेत्~॥~२२४

\renewcommand{\thefootnote}{2}\footnote{भ \textendash\ शिक्षायोगेन नाट्येऽस्मिन् विद्यायोग}अथवा योगशिक्षाभिर्विद्यामायाकृतेन वा~।\\
शस्त्रमोक्षः प्रकर्तव्यो रङ्गमध्ये प्रयोक्तृभिः~॥२२५ 

\renewcommand{\thefootnote}{3}\footnote{भ \textendash\ आयुधा\textendash\ न्यवमेतानि प्रयोज्यानि प्रयोक्तृभिः}एवं नानाप्रकारैस्तु आयुधाभरणानि च~। \\
नोक्तानि यानि च मया लोकाद्र\renewcommand{\thefootnote}{4}\footnote{भ\textendash\ लोके} ग्राह्याणि तान्यपि~॥२२६ 

आहार्याभिनयो ह्येष मया प्रोक्तः समासतः~। \\
अत ऊर्ध्वं\renewcommand{\thefootnote}{5}\footnote{भ\textendash\ परं} प्रवक्ष्यामि सामान्याभिनयं प्रति~॥२२७ }
\end{quote}

\begin{center}
\textbf{इति भारतीये नाट्यशास्त्रे आहार्याभिनयो नामैकविंशोऽध्यायः\renewcommand{\thefootnote}{6}\footnote{भ \textendash\ अध्यायो विंशः ज \textendash\ त्रयोविंशोऽध्यायः य \textendash\ द्वाविंशोऽध्यायः}~।}
\end{center}

सुधामकोक्ता विद्या हस्तलाघवादि माया चक्षुर्बन्धादिका~। एनमध्यायमुपसंहर(न्नभिनये वक्तव्यशेषमा)सूत्रयति आहर्याभिनयो ह्येषु इति शिवम्~। 

\begin{quote}
{\qt आहार्याभिनयाध्याये वृत्तिरेषा यथाक्रमम्~। \\
कृताभिनवगुप्तेन ग्रन्थिस्थानेषु तत्त्वतः~॥}
\end{quote}

\begin{center}
इति श्रीमहामाहेश्वराभिनवगुप्ताचार्यविरचितायां \\
नाट्यवेदविवृतावभिनवभारत्यामाहार्या\textendash\ \\
भिनयाध्याय एकविंशः~॥
\end{center}

\lfoot{19}

\newpage
\lfoot{}
\thispagestyle{empty}

\begin{center}
\textbf{\large श्रीः }\\

\vspace{2mm}
\textbf{\huge नाट्यशास्त्रम् }\\
\vspace{2mm}
द्वाविंशोऽध्यायः.\renewcommand{\thefootnote}{1}\footnote{भ \textendash\ एकविंशः जादिषु चतुर्विंशः चय \textendash\ त्रयोविंशः}\\
\rule{0.2\linewidth}{0.5pt}
\end{center}

\begin{quote}
{\na सामान्याभिनयो नाम ज्ञेयो वागङ्गसत्त्वजः~। }
\end{quote}

\begin{center}
अभिनवभारती \textendash\ द्वाविंशोऽध्यायः
\end{center}

\begin{quote}
{\qt भेदेनात्माभिमुखतां नयन्तं भेदकारणम्~। \\
सामान्याभिनयाकारगर्वमूर्ति शिवं नुमः~॥ }
 \end{quote}

इहान्यदित्युपरञ्जकं च अभिनयं चान्याभिनयं समं च तदिति तत्र भवः सामान्याभिनय इति परमार्थः~। कोहलमतानुसारिभिर्वृद्धैः सामानाभिनयस्तुषोढा भण्यते~। तथा हि कोहलः\textendash\ 

\begin{quote}
{\qt शिष्टं कामं मिश्रं वक्रं संभूतमेकयुक्तत्वम्~। \\
सामान्याभिनये यत् षोढा विदुरेतदेव बुधाः~॥ इति~। }
\end{quote}

तत्र सामान्यमिति साधारणमुच्यते तेन सर्वेष्वभिनयेषु यद्रूपमवशिष्टं पूर्वं नोक्तमवश्यं वक्तव्यं च कविनटशिक्षार्थं तद्येनाध्यायेनाभिधीयते ससामान्याभिनयः, सोऽभिनयेषु सामान्यभूतः साधारणभूतोऽभिनयविषय\textendash\ त्वात् स्ववाच्याभिमुख्यनयनाद्वाभिनय इति व्युत्पत्त्या~। तथा हिा त्विकस्य हावभावहेलादिना विशेषः पूर्वमनुक्तोऽभिधीयते~। विषयश्चैवं ' षडात्मकः शारीरः ' (२२\textendash\ ४१) इत्यादिना ' आलापश्च प्रलापश्च' (२२\textendash\ ४९) इत्यादि\textendash\ नाङ्गिकवाचिकयोः~।\\

ननु ' अङ्गाद्यभिनयस्यैव यो विशेषः ' (२५\textendash\ १) इत्यतः चित्राभिनयात् कोऽस्य विशेषः,उच्यते\textendash\ तत्र वागङ्गसत्त्वव्यामिश्रत्वेन चित्रता~। इह तु

\newpage
\fancyhead[CO]{द्वाविंशोऽध्यायः}
% द्वाविंशोऽध्यायः १४७ 

\noindent
प्रत्येकनियतस्यानुक्तस्य विशेषान्तरस्यभिधानमिति~। तथा हि तत्र चित्रशब्दं पठिष्यति ' अनुक्त उच्यते चित्रः' इति (25\textendash\ 2) तथा (चेह तु) सामान्याभिनयः कामोपचारः, स हि सकलप्राणिवर्गसाधारण आभिमुख्यं~। नयति च सर्व जन्तुवगमिति वागङ्सत्तलमणेन सकलेन सामान्यात्मना चाभिनयेन अभिनीयत इति~। तत्कामोपचारः स्त्रीपुरुषखभावः तदवस्थाभेदेनेहाभिनीयत इति सामान्यभिनयोऽयमध्यायः~। अत एवैतदध्यायशेषभूतकामोपचारप्रतिपादकमेवाध्यायं वैशिकोपचाराख्यं मन्तव्यम्~। तथा सामान्यामिति समानानां कर्म सामान्यं च तदभिनयनं च~। तत्तेनैकमेवाभिनयं गमयितुं यथासम्भवं बहूनामभिनयानां याभिनयक्रिया एकं तदेवाभिनयक्रियारूपं कर्म समानानां सताम्~। \\

नन्वेवमेकत्राभिनये किं बहुभिरभिनयैः~। तत्र केचिदाहु \textendash\ स्वोपस्थानेषु साध्येषूपस्कारांशो व्यापार इति~। तच्चासत्, नहि नाटकादौ सूत्रेष्विवोप\textendash\ स्कारो युक्तः~। स ह्यत्र प्रत्युत दोषाय, यथाह\textendash\ " काव्यान्यपि यदीमानि व्याख्यागम्यानि शास्तवत् "* इत्यादिना~। प्रविस्पष्टपराकरणं तत्र निमित्तं स्पष्टार्थेन वाक्यमात्रेण तत्सिद्धेः~।\\ 

तत्रोक्तं श्रीशङ्कुकादिभिः\textendash\ इह लोकानुसारिनाट्यात् लोके सुखदुःखाद्यावेशविवशो वक्ता, तत एव स्तम्भस्वेदादिभिर्बृंहितं अवधानबन्धोऽपि गुणक्रियादिस्वरूपसाहचर्याभ्याससंस्कृतः (तं ?) शब्दप्रयोगः(गं ?)तदुपचिताङ्गोपाङ्गविकारसंकीर्णमेव कुर्वाणो दृ्यते\textendash\ ३ति~। \\

(१) यत्र त्वसत्यतो वक्तव्यं तदस्य निरुक्तमष्टमेऽध्याये \textendash\ 

\begin{quote}
{\qt विभावयति यस्माद्धि नानार्थार्थप्रयोगतः~। \\
शाखाङ्गोपाङ्गसंयुक्तं तस्मादभिनयः स्मृतः~॥ }
\end{quote}

इत्येवमन्तं श्लोकं व्याचक्षाणैः~। [न तथा] सामान्यस्य समानीकृतसङलङ्गोपाङकर्मणा सतोऽभिनयनं येनालातचक्रप्रतिमता प्रयोगस्य जायते~। यथोक्तं ' प्रयोगश्चास्य कीदृश' इति, यद्वक्ष्यते\renewcommand{\thefootnote}{*}\footnote{भामहालङ्कारे, भट्टिकाव्ये च}

\newpage
% १४८ नाट्यशास्त्रम् 

\begin{quote}
{\qt शिरोहस्तकटीवक्षोजङ्कोरुकरणेषु तु~। \\
 समः कर्मविभागो यः सामान्याभिनयस्तु सः~॥} (अ २२) इति
\end{quote}

(२) सामान्य इत्यनेनाशेषाभिनयविशेषा आङ्गिकादिगता उपलक्षिताः तत्कृतोऽभिनयः~। यद्वक्ष्यति\textendash

\begin{quote}
{\qt कृत्वा साचीकृतां दृष्टिं शिरः पार्श्वे नतं तथा~।\\
तर्जनीं कर्णदेशे च बुधः शब्दं विनिर्दिशेत्~॥} इति (२२\textendash\ ७६)
\end{quote}

(३) अत्र हि दृष्टिविशेषः शिरोविशेषो हस्तविशेषश्च संभूयैकमभिनयं प्रत्येकोऽभिनयः संपद्यते~। (४) एकैकेन तु शब्दाभिनयस्य कापि मात्रा निष्पद्येत, एवमेव तृतीयपक्षादस्य विशेषः~। तत्र हि एकैकस्याप्यभिनयनेऽस्य (संभूतत्वेन) सामर्थ्यभ्~। (५) तथा विघ्नसंभावना\renewcommand{\thefootnote}{*}\footnote{षष्ठेऽध्याये रससूत्रव्याख्याने सप्तविघ्ना उक्ताः~।} विहीनसकलसाधारणस्पष्टभावसाक्षात्कारकल्पाध्यवसायसंपत्तये सर्वेषां प्रयोग इत्युक्तम्~। (६) तथाभिनय इति तद्विशेषो यत्र उच्यते, स च साधारणरूपः सामान्याभिनयः~। तथा हि\textendash\ प्रकटाक्षप्रेक्षणाद्यं यं यत्रं कुशलं प्रयोक्ता गृह्णाति तेनैव तदुचितशिरःकर्मान्तमस्य संपाद्यम् \renewcommand{\thefootnote}{$\dagger$}\footnote{इदमार्याया विकाररूपं स्यात्\textendash
\begin{quote}
{\qt प्रकटाक्षवीक्षणाद्यं यत्नं कुशलप्रयोगयोक्ता यम्~। \\
 गृह्णाति तेन तु तदुचितशिरःकर्मान्तमस्य संपाद्यम्~॥} इति स्यात्~।
\end{quote}}इति षोढा गुरूभिर्निदर्शितः~।\\

वयं तु [न] मन्महे : \textendash\ रसभावाध्याययोर्वागङ्गसत्वजास्त(त्त)द्रसभावेषु दर्शितास्ते कथं प्रयोज्या इत्ययमध्यायः~। यथा हि किराटगृहाद् गन्धद्रव्याण्यानीय गान्धिकेन समानीक्रियते अस्येयान् भाग इदं पूर्वमिति, एवमत्राध्यायेऽ भिनयाः~। तत्र श्रृङ्गारस्य प्राधान्यात् तत्रैवाभिनयानां भागयोगेन पौर्वापर्ययुक्त्या च समीकरणं सत्त्वातिरिक्त इति~। तेन सामान्यानां कर्म समानीकरणं भावनप्रायमभिनयविषयं स्वयं चाभिनयरूपं सामान्याभिनयं श्रृङ्गारमुखेन चान्यदुपनेयमिति~। तदेतत्सर्वं हृदये कृत्वा मुनिराह सामान्याभिनयो नाम ज्ञेय इति~। नाम्नैव ज्ञातुं शक्योऽन्वर्थत्वादस्येति भावः~। तत्त्तु व्याख्यातम्~। \\

नन्वेवं तत्र न किंचिदवशिष्यते वक्तव्यमित्याशङ्क्यावृत्याह सामान्याभिनयो नाम ज्ञेय इति~। नामशब्दः प्रसिद्धिद्योतकः~। तदयमर्थः\textendash\ यद्यपि

\newpage
% द्वाविंशोऽध्यायः १४९

\begin{quote}
{\na \renewcommand{\thefootnote}{1}\footnote{1 ढ \textendash\ सत्चे }तत्र कार्यः प्रयत्नस्तु नाट्यं सत्त्वे प्रतिष्ठितम्~॥ १}
\end{quote}

\hrule

\vspace{2mm}
\noindent
ज्ञेयः [स्थिर]विषये सामान्याभिनयः प्रसिद्धोऽपि वाक्यार्थबलात् , तथापि यो वागङ्गसत्त्वेभ्यो जातः तद्विषयः सामान्याभिनयो व्याख्यातः~। तत्रेति विषये तन्निरूपणायामस्माकं प्रयासः कार्य एव~। आहार्यो हि यद्यप्यभिनयान्तरेभ्यो न्यूनस्तथापि तस्य सिद्धस्वरूपत्वान्नात्रोपादानम्~। आङ्गिकादिक्रियाणां हि पूर्वापरीभूतरूपतया सम्भावनीयवि(शेष)भावनादेकीकारात्मा सामान्याभिनयो यत्रसंपाद्य एव~। आहार्यस्य तु तन्मध्ये स्थिरत्वेनावस्थानाद्यत्नसिद्ध एवासौ~। अत एवाहार्येऽपि भविष्यति सामान्याभिनयचिन्ता~। न तु सर्वथैवास्य तत्र त्यागः~। तथा हि " वागङ्गालङ्कारैः " (२२\textendash\ १४) इति लीलायां, " माल्याच्छादनविलेपनभूषणानां ' (२२\textendash\ १७) इति विच्छित्तौ, " वागङ्गाहार्यसत्त्ववेगेन " (२२\textendash\ १७) इति विभ्रमे, तस्य सातिशयनिरूपणं भविष्यति~। \\

अन्ये त्वाहुः\textendash\ आन्तरभावानपेक्ष एवाहार्यो दण्डकमण्डल्वक्षसूत्रादिर्व्रतविशेषादिमात्रं, गमयति , न तु भावं कंचित्~। उज्ज्वलो हि वेषो न रतिं गमयति नापि मलिनः शुचम्~। तदभावेऽपि हि ते भवत एव~। औचित्यमात्रं ह्येतद्रतावुज्ज्वलो वेषः, शुचि मलिन इति~। ये त्वेते गुणद्रव्यादिबाह्याभिनयाः सुखदुःखदिभावानिश्चयाश्च ते चित्तवृत्तीनां बाह्यार्थानां च कार्यकारणभावस्य नियतव्यक्तित्वाद् भावापेक्षा इति~। वागङ्गसत्त्वाभिनया अन्योन्य सहचर्यमाणाः, नत्वेवं तेष्वाहार्य इत्यस्यानुपादानक्रिया~। एतच्च न मुनेर्मतमित्यावेदितमस्माभिरुपाङ्गाभिनयाहार्याभिनयाध्याययो (८,२२) रित्यास्ताम्~।\\

नन्वेवं त्रितयनिष्ठो यद्यपि यत्नस्तथाप्यभिहितत्वेन किमिह वक्तव्यमित्याह नाद्यं सत्त्वइति~। तुशब्दः सत्त्वशब्दानन्तरं द्रष्टव्यः~।सात्त्विके त्वभिनये नाट्यं प्रतिष्ठितम्~। रसमयं हि नाट्यं रसे चान्तरङ्गः सात्विकस्तस्मात् स एवाभ्यर्हित इति तद्गतमेव वक्तव्यं पूर्वमभिधेयमित्याशयमशेषचिरन्तना आक्षेपपूर्वकं समादधतित्रिषूद्दिष्टेषु वक्तव्यं वागङ्गसत्त्वेषु नाट्यं प्रतिष्ठितमिति सोऽयमाक्षेपः~। प्रतिसमाधानं तु यदि वागङ्गजमेव स्यात् प्रयत्नं विनापि 


\newpage
% नाट्यशास्त्रम् १५०

\begin{quote}
 {\na सत्त्वातिरिक्तोऽभिनयो ज्येष्ठ इत्यभिधीयते~। \\
 समसत्त्वो भवेन्मध्यः सत्त्वहीनोऽधमः स्मृतः~॥ २ 

 अव्यक्तरूपं सत्त्वं हि विज्ञेयं भावसंश्रयम्\renewcommand{\thefootnote}{1}\footnote{1 च \textendash\ ज्ञेयं भावरसाश्रयम् भ\textendash\ \textendash\ भावनाश्रयम् }~।\\
 यथास्थानरसोपेतं रोमाञ्चास्त्रादिभिर्गुणैः~॥ ३ }
\end{quote}

\hrule

\vspace{2mm}

\noindent
सिद्धिः स्यात्, वागङ्गसत्त्वजोऽसौ सत्त्वे च नाट्यं प्रतिष्ठितम् सत्वं च मनस्समाधानजम्~। तस्माद्भूयसा प्रयत्नेन विना (न) सिध्द्यतीति~। एतत्त्तु चोद्यसममेवोत्तरं सत्त्वस्य हि प्रयत्नाधिक्यमुपयोगीति वागङ्गयोरुपादानमलमेवेतिअलमनेत~। \\

ननु कोऽत्रहेतुः सत्त्वे नाट्यं प्रतिष्ठितमित्याशङ्क्याह सत्वतिरिक्तोऽभिनय इति~। सत्त्वमिति सात्त्विकोऽभिनयः, तेन वागङ्गा[सत्त्व]भिनययोर्यत्रैकत्रैवाभिनये क्रमेण युगपद्वा प्रयुज्यते तत्र परे(रं?) सात्त्विकस्या(न्य)द्वयापेक्षयाधिक्यं भवति~। तत्प्रशस्यतमाभिनयक्रिया (ज्येष्ठा) भवति~। सुष्ठु सम्यगभिमुखीभावं सौष्ठवं नीतो भवति रसपर्यन्तत्वात्प्रीतेरिति भावः~। \\

अथ सात्त्विकोऽन्यतुल्य एव, तदभिनयन प्रशस्यं संपद्यते परमिति यावत्~। यदि त्वितरापेक्षया सात्त्विको न्यूनस्तर्हि अभिनयक्रिया खरूपेणापूर्णा संपद्यत इत्यर्थः~। सात्त्विकाभावे ह्यभिनयक्रियानामापि नोन्मीलति~। अभिनयनं हि चित्तवृत्तिसाधारणतापत्तिप्राणसाक्षात्कारकल्पाध्यवसायसंपादनमिति, अत एवोक्तं सत्त्वे नाट्यं प्रतिष्ठितमिति~। \\

अव्यक्तरूपमित्यादिकं प्रबन्धं श्रीशङ्कुकादय इत्थं नयन्ति\textendash\ कस्मात् पुनः सत्त्वं प्रयत्नातिशयमपेक्षते~। उच्यते\textendash\ रामाद्यनुकार्यगतं भावसंश्रयं तद्भावनाप्रकर्षजं रोमाञ्चादिसंपादकं यदान्तरं नाट्यस्य सत्त्वं तदव्यक्तं अस्फुटं केवलं रोमाञ्चादिभिर्गमवत्वाडुणभूतैर्वि्ञयं अन्यथा रि सुखाद्यभावे कृत एषापुद्भव इत्यरेतुकं स्यात्~। तत् सत्वं भावस्य स्थाने प्रसङ्गतो यो मुख्यो रसस्तेनोपेतं, रसेनानुकर्ये च प्रकृष्टेन यत्नेन ज्ञेयं सुखादि तस्य ये रोमाञ्चादयः कार्यास्तत्त\textendash\ 


\newpage
%द्वाविंशोऽध्यायः १५१ 

\noindent
त्साध्यभावे यतः सत्त्वात्प्रवर्त्यन्ते तन्मयः प्रयोगः कथं प्रकृष्टयत्नमन्तरेण सिध्द्येदिति तात्पर्यम्~। न केवलं [प्रकृत]रोमाञ्चादावभिनये संपाद्ये नटस्य सत्त्वमुपयुज्यते यावदङ्गनानां येऽलङ्कारास्तेष्वापि~। तथा हि ते तावत्क टककेयूरादिभ्योऽप्यभि विकारानयनेऽभिनयं (?) रूपलावण्यादिवत् स्तनकेशादिवच्च युवतिरियमिति प्रतीयते, न त्वभिनेयं तेषां किंचिदस्ति केवलमलङ्कारत्वपेषाम्~। न च प्रयोगाभिनिविष्टत्वाद्युवतेरपि प्रयोज्यास्ते प्रयोक्तुं शक्या मनस्समाधानमन्तरेण~। तत्र मनसो देहवृत्तित्वात् समाधानं सत्त्वमुपचारादेहात्मकम्~। देहे हि मनस्समाधातव्यम्, तत ईषद्विकारो भावः स एव प्रौढतायां तदतिशये च हावो हेला च~। तथा च भावः तत्र कटकादाविव हेमः स्थितः~। तत्र तु मदनानपेक्षी विकारो भावः येनाकामयमानापि तरुणी कामयमानेव लक्ष्यते, तस्यैव तु मदनापेक्षत्वेन प्रौढतायां हेलात्वमेव, यौवने क्रमादुपचीयमाने स्वात्मेन्द्रियमनःस्वास्थ्ये हावः हेला शरीरविकारः धात्वादिवैषम्यात्तु तदवसादे प्रविलय इति हेलातो भावयुक्तो भावतै(हावतै ?)वेति नानपेक्षितहेत्वन्तरा याौवनकृताः शरीरविकारा अपि प्राधान्येन वक्त्रगात्रनगता गुणा इव भावा इव नाभिनयाः~। किं त्वीषद्भिद्यमानैर्वागाद्यभिनयैर्मुखरागेण च संभवत्तया प्रतीता अप्रतीता अलङ्कारा उच्यन्ते~। तदेतदुक्तमव्यक्तरूपमित्यादिना समाख्याता बुधैर्हेला ललिताभिनयात्मिका (२२\textendash\ ११) इत्यन्तेन~। एतेभ्यस्त्वङ्गजेभ्योऽन्ये शीलकृता इति स्वाभाविका दश लीलाद्याः सत्त्वबलेनैव प्रयोज्याः~। अन्ये तु निसर्गजत्वेनायत्नजाः सप्त शोभाद्या उक्ताः (२२\textendash\ ३२) ~। तत्रैते शोभाद्याः स्त्रीगताः पुरुषगताश्चान्ये~। सर्वे चैते अतत्स्वभावेनापि नटेन सत्त्वबलात्प्रयोक्तव्या इति बहुप्रद(र्शनविलसितं) व्याख्यानं न ग्रन्थज्ञेभ्यो रोचते~। \\

तथा हि\textendash\ किमिदमनुकार्यं (र्यगतं ?) कवेः शिक्षार्थमुपदिश्यते, तथानुकतृगतं नटस्य (वा)~। प्रथमस्तावद्यदि पक्षस्तदव्यक्तरूपं सत्त्वमिति सत्त्वस्य कथं प्रयोक्तरि स्थितिः, सत्त्वाद्भावः समुत्थित इति ह्युक्तम्~। नटे च सत्त्वं, अनुकार्ये च भाव इति किं केन संगच्छेत~। स्वनुकार्ये च प्रस्तुते प्रागल्भ्य\textendash\ 

\newpage
% १५२ नाट्यशास्त्रम् 

\noindent
माधुर्ये परत्न (पात्र ?) गते उच्येते इति किमेतच्च, प्रतिलव(विलय?) क्रमेण भावदहावहेलानां परस्परकार्यकरत्वं प्रथमं तावद्याख्यातम् , तदप्यसत्~। न हि प्रतिसंहारे कारणता कार्यस्य व्यपदिश्यते~। न हि पृथिव्यादिभूतीन प्रविलयतन्मात्राणां कारणानि, तानि चाहंकारस्य, सोऽपि च बुद्धेः, सा च प्रकृतेः, प्रकारो वा तदहङ्करणमिति व्यवहारः~। प्रति (संहार इव) प्रकृतेः कार्यदशा यामपि संभवान्न पूर्वः प्रादुर्भावः तत्कथं कार्यता तदहंकारादेः कारणत्वमेतदिति चेत् सामानमेतदिहापि~। यदि हि भावो हावतां प्राप्तः सोऽपि इेलात्वं च ततो हेला विलीयते~। आख्यास्था दोषात् तदा हावः स्थित एव, न हेलया हेला परं कार्यकारणभावव्यवहारस्यावकाशः~। किं चेते देहविकाराः प्रयत्नेन निर्वर्त्या इति [साक्षिण इति] यदुच्यते तस्मिन्नाट्यस्य संसारे नाम तदस्ति यत्प्रयत्नेन निर्वर्त्या इति सात्विकाद्वैतम्~। किं च विभावानुभावब्यभिचारिव्यतिरिक्तमपि यद्यत्रोपयोगि संभवति तद्वृथैव प्रतिज्ञातं तत्संयोगाद्रसनिष्पत्तिरिति, गीतातोद्यरङ्गादिबलेनेदं व्यवस्थितं सामान्याभिनय इत्याभिधानात् अनभिनयवत्त्वे चास्य सत्त्वनिर्वर्त्यस्यापि को नाट्ये उपयोग ,कथं च सामान्याभिनयेनेत्यपरामृष्टाभिधानम्~। एतत्सर्वे मुनिमताननुप्रविष्टैः परं श्रद्धीयते नामेत्यास्तां तावत्~। \\

प्रकृतव्वाख्यानमुच्यते\textendash\ इहोक्तं सत्त्वे नाट्यं प्रतिष्टितं तेन सात्त्विकभावानुनयो वक्तव्यः तस्य च किंचिदुक्तमिति दर्शयति\underline{अव्यक्तरूपं सत्त्वं हि विज्ञेयमिति~।} इह चित्तवृत्तिरेव संवेदनभूमौ संक्रान्ता देहमापि व्याप्नोति~। सैव च सत्वमित्युच्यते~। तत्र चाव्यक्तं संवित्प्राणभूमिद्वयानिपतितं यत्सत्वं तद्भावाध्यायसंश्रयत्वेनैव विज्ञेयम्~। तस्य च ये गुणा देहपर्यन्ततां प्राप्ता घर्मरोमाञ्चादयः तेऽपि तत्रैवोक्ताः किंचित्~। यथास्थानमिति यस्य रसस्य यत् स्थानं, तद्यथा श्रृङ्गारस्य (उत्तमौ) स्त्रीपुंसौ, रौद्रस्य रक्षोदानवादिः, भयानकस्याधमप्रकृतिः तदनतिक्रमेण रसेषूपेतं सम्बद्धं तत्सत्त्वम्~। भावशब्देनात्र भावाध्यायः(उक्तः)~। 

\newpage
%द्वाविंशोऽध्यायः १५३ 

\begin{quote}
{\na{ अलङ्कारस्त\renewcommand{\thefootnote}{1}\footnote{च \textendash\ च } \renewcommand{\thefootnote}{2}\footnote{च \textendash\ सत्त्वस्था भ \textendash\ वृत्तज्ञैः}नाट्यज्ञैर्ज्ञेया भावरसाश्रया:\renewcommand{\thefootnote}{3}\footnote{क समाश्रयाः }~। }}
\end{quote}

\hrule

\vspace{2mm}

एतदुक्तं भवति\textendash\ चित्तवृत्तिरूपं यत्सत्त्वं तद्भूकायसंक्रान्तप्राणदेहधर्मतावशाद् भवदपि भावाध्याये रसाध्याये च बितत्य निरूपितमिति पुनः किं तदभिधानेन~। \\

\begin{sloppypar}
किं तस्य भूसत्त्वस्य रूपं वक्तव्यमित्या \underline{अलङ्कारास्तु नाट्यज्ञैरित्यादि~।} अयमभिप्रायः\textendash\ संवेदनरूपात्प्रसृतं यत्सत्त्वं तद्विचारितम्~। अन्यत्तु देहधर्मत्वेनैव स्थितं सात्त्विकं, यतः सात्त्त्विकेष्वेवोत्तमेषु दृश्यते, तत्र स्त्रीणामुत्तमत्वं श्रृङ्गाररसपर्यन्तमेव, पुरुषाणां तु वीररसविश्रान्तम्~। शान्तस्तु प्रधानत्वेन न प्रयोगार्ह इत्युक्तप्रायः~। स्त्रीगतेन श्रृङ्गारेण पुरुषनिष्ठेन वीरेण च सार्वलौकिकः पुमर्थो व्याप्तः~। न च सत्त्वमयमुत्तमस्त्रीरूपं विमुच्यान्यत्रामीचेष्टालङ्कारा विनिवेशं लभन्ते [न] सात्त्विकास्तावद्राजसतामसशरीरेष्ठसंभवात्~। चण्डालीनामपि रूपलावण्यसंपदो दृश्यन्ते, ननु चेष्टालङ्कारास्तासामपि भवन्त उत्रमतामेव सूचयन्ति स्ववर्गापेक्षया वा संपद्धंशादिना~। एतदुक्तं भट्टतोतेन\textendash\ न चालङ्कृतीनामत्र\renewcommand{\thefootnote}{ * }\footnote{न चेष्टालङ्कृतीनां तु\textendash\ इति स्यात् } लक्षणं महदाश्रयमिति\textendash\ ते च दृष्टाः सन्तः उत्तमेयं शृङ्गारसमुचितेति विभावादिसुविवेकविहीनं व्यभिचारिरूपदशान्तरसंस्पर्शशून्यं विशेषविरहितमेव सामान्यरूपं शृङ्गारमभिनयति(न्ति?), सामान्याभिनया न तु लावण्यादिवदनभिनेया एव(वं?) शरीरविकारा अनुभावा एव तेन विभावानुभावव्यभिचारिसंयोगादित्येवमेवैतत्~। एवं पुरुषगता अपि शोभादय उत्साहप्रकृतिरयमित्येतावन्मात्रं गमयन्तः सामान्याभिनया एव~। किं च यत्किंचिदङ्गनानां श्रृङ्गारोचितं चेष्टितमभिनीयते तत्रैव चेष्टालङ्कारा अवश्यमभिनेया इति सामान्यवत्सर्वावस्थानुयायित्वेनाभिनीयत इति (च) सामान्याभिनया एव प्रधानपुरुषस्य शोभादयः, तथैते वागङ्गसत्त्वाहार्याणि स्वभेदसहितानि यथासंभवं संभूयाभिप्रविष्टानि यथा किलिकिञ्चिते विच्छित्तौ विभ्नमे चेति सामान्याभिनया वागङ्गाहार्ययोगेऽपि च सत्त्वप्रधानतया सात्विका इत्युक्ताः~।
\end{sloppypar}

\lfoot{20}

\newpage
\lfoot{}
%१५४ नाट्यशास्त्रम्

\begin{quote}
 {\na{यौवनेऽभ्यधिकाः \renewcommand{\thefootnote}{1}\footnote{भ \textendash\ ह्यधिकाः न \textendash\ अभ्यधिकं} स्त्रीणा विकारा वक्त्रगात्रजाः~॥ 

आदौ त्रयोऽङ्गजास्तेषां\renewcommand{\thefootnote}{2}\footnote{भ \textendash\ प्रोक्ताः} दश स्वाभाविकाः परे\renewcommand{\thefootnote}{3}\footnote{न \textendash\ तथा }~। \\
अयत्नजाः \renewcommand{\thefootnote}{4}\footnote{ड \textendash\ तथा }पुनः सप्त \renewcommand{\thefootnote}{5}\footnote{न \textendash\ प्रोक्ताभावोप }रसभावोपबृंहिताः ~॥~। ५ }}
\end{quote}

\hrule

\vspace{2mm}

एवं तैरेव सामान्याभिनयैः प्रधानप्रमदापुरुषद्वारेण विश्वमेव व्याप्तम्~। ते चात्राध्याये वक्तव्याः, तदाह\textendash\ अलङ्कारास्त्विति तुर्व्यतिरेके, अन्ये भावाध्य॒य एवोक्ताः, एते तु वक्तव्याः ते तु तत्र नोक्ताः~। यत एते केवलमलंकारा देहमात्रनिष्ठाः, न तु चित्तदृत्तिरूपाः~। भाबमंश्रया इति रतिभावमात्रमभिनयन्तीत्यर्थः~। ते हि यौवने उद्रिक्ता दृश्यन्ते बाल्ये त्वनुद्भिन्ना वार्धके तिरोभूताः~। यदाह\textendash\ 

\begin{quote}
 {\qt{ यावन्त एते तरुणीजनस्य भावाः समं कुट्टमितादयोऽपि~। \\
 रात्रावदृश्यानिव तान्घटादीन्कामप्रदीपः प्रकटीकरोति~॥ इति~। } }
\end{quote}

\begin{sloppypar}
\underline{वक्त्रगात्रजा} इति देहविकारमात्ररूपा एव परं न हि यथा बाष्पादीनामन्तःप्राणभुवि कण्ठरोधादिरूपं लक्ष्यते, तथा चेष्टालङ्गाराणाम्गा त्राणि(वक्षोनितम्बा)दीनि, वक्त्रं प्राधान्यात् पुनरुपात्तम्~। तत्र देहविकाराः केचन क्रियात्मका अपि ते च प्राग्जन्माभ्यन्तरिता भावसंस्कारमात्रेण सत्त्वोद्बुद्धेन देहमात्रे सति भवन्ति, त एवाङ्गुजा उच्यन्ते, तथा भावो हावो हेला च~। अन्ये त्वद्यतनजन्मसमुचितविशिष्टविभावानुप्रवेशस्फुटीभवद्रतिभावानुविद्धे देहे परिस्फुरन्ति ते स्वाभाविकाः स्वस्माद्रतिभावात् हृदयगोचरीभूताद् भवन्तीति~। तथा कस्याश्चित् कश्चिदेव स्वभावबलाद् भवति, अन्यस्या अन्यः, कस्याश्चित् द्वौ त्रय इत्यादि, अतोऽपि स्वाभाविकाः~। \\
\end{sloppypar}

भावहावहेलास्तु सर्वा एव सर्वास्वेव ? सत्त्वाधिकासूत्तमाङ्गनासु भवन्ति ~। तथा शोभादयः सप्त~। एवमङ्गजाः स्वाभाविकाश्च क्रियाजन्मानः, अन्ये तु गुणस्वभावाः शोभादयः ते चायत्नजाः~। यत्नजाता: क्रियात्मका उच्यन्ते, (इच्छातो) यत्नस्ततो देहक्रियेति हि पदार्थविदः~। ततोऽन्येऽयत्न जाताः~। तदेतदाह\textendash\ आदौ त्रयोऽङ्गजा इति~। तेषामलङ्काराणां मध्ये~। आदा\textendash\ 

\newpage
% द्वाविंशोऽध्यायः १५५ 

\begin{quote}
 {\na{\renewcommand{\thefootnote}{1}\footnote{अतः प्रभृति श्लोकपञ्चकस्य पाठक्रमो भिन्नमातृकासु भिन्नतया दृश्यते}देहात्मकं भवेत्सत्त्वं सत्त्वाद्भावः समुत्थितः~। \\
भावात्समुत्थितो हावो हावाद्धेला समुत्थिता ~॥ ६ 

\renewcommand{\thefootnote}{2}\footnote{च \textendash\ भावो हावश्च हेला च प \textendash\ हावो भावश्च हेला च }हेला हावश्च भाविश्च परस्परसमुत्थिताः\renewcommand{\thefootnote}{3}\footnote{भ\textendash\ समुत्थिः }~। \\
सत्त्वभेदे\renewcommand{\thefootnote}{4}\footnote{भ \textendash\ सत्त्वभेदा} भवन्त्येते शरीरे प्रकृतिस्थिताः\renewcommand{\thefootnote}{5}\footnote{भ \textendash\ प्रकृतिर्हि ताः } ~॥ ७ 

वागङ्गमुखरागैश्च सत्त्वेनाभिनयेन च~। \\
कवेरन्तर्गतं भावं भावयन्भाव उच्यते~॥ ८ }}
\end{quote}

\hrule

\vspace{2mm}

\noindent
विति प्राच्यवासनानुविद्धदेहमात्रप्रभवित्वात् पूर्वमेव भवन्तीति यावत्~। भावोपबृंहिता इत्युभयशेषः~। स्वाभाविका अयत्नजा स्वरतिभावेन प्राणिता भवन्ति~। [अनुमति] पुनरिति सत्त्वानां, पुस्तचित्रालेख्यलिखितानामेव नैते भवन्ति~। तत्र त्रयाणां तावदुपक्षेपकर्तृ पीठबन्धमाह देहात्मकं भवेत्सत्त्वमिति~। शरीरस्वभावं तावत्सत्त्वं संभाव्यते उत्तमशरीरतां प्राप्तमित्यर्थः~। ततो भावः ततोऽपि हावः तस्मादपि हेला~। एवं तीव्रतरसत्त्वे देह एव~। यदा तु तथाविद्धं सत्त्वं न भवति तदा प्राक्तनरतिवासनोत्थं अत्र सहकार्यन्तरमपेक्षणीयं वर्तत इति दर्शयति\textendash\ 

\begin{quote}
 {\qt हेला हावश्च भावश्च परस्परसमुत्थिताः~। \\
सत्त्वभेदे भवन्त्येते शरीरे प्रकृतिस्थिताः~॥ इति }
\end{quote}

एकश्चशब्दोऽपिशब्दार्थे, अपरः समुच्चये~। प्रत्येकं हि समुच्चये द्योत्ये वृतीयोऽपि चः पठितव्यः स्यात्~। तदयमर्थः\textendash\ प्र(कृ)तिस्थिताः देहस्वभावमात्रापेक्षा अप्येते परस्परसमुत्थिता भवन्ति~। तथा हि\textendash\ कुमारीशरीरे प्रौढतमकुमार्यन्तरगतहेलावलोकने सति हावोद्भवो भावश्चेदुल्लासितपूर्वः, अन्यथा हि भावस्यैवोद्भवः~। एवं हावेऽपि दृष्टे भावो हेला वा~। यदा तु हावावस्थोद्भिन्ना पूर्वं परत्र च हेला दृश्यते तदा हेलातोऽपि हेला~। एवं हावाद्धावो भावाद्भाव इति च वाच्यम्~। एवं परकीयभावादिश्रवणात् तथाविधे\textendash\ 

\newpage
% १५६ नाट्यशास्त्रम् 

\begin{quote}
 {\na {[\renewcommand{\thefootnote}{1}\footnote{भ\textendash\ मातृकायामयं श्लोको न लभ्यते}भावस्यातिकृतं सत्त्वं व्यतिरिक्तं \renewcommand{\thefootnote}{2}\footnote{च\textendash\ च }स्वयोनिषु~। \\
नैकावस्थान्तरकृतं भावं तमिह निर्दिशेत्\renewcommand{\thefootnote}{3}\footnote{य\textendash\ निर्विशेत् }~॥] ९ 

तत्राक्षिभ्रूविकाराढ्यः\renewcommand{\thefootnote}{4}\footnote{भ\textendash\ विकाराभ्यां} श्रङ्काराकारसूचकः\renewcommand{\thefootnote}{5}\footnote{ड \textendash\ रससूचकः म \textendash\ संयुतः}~। \\
सग्रीवारेचको ज्ञेयो हावः\renewcommand{\thefootnote}{6}\footnote{न\textendash\ भावः} \renewcommand{\thefootnote}{7}\footnote{भ \textendash\ सत्त्व }स्थितसमुत्थितः~॥ १० } }
\end{quote}

\noindent
याभिधेयरमणीयकाव्याकर्णनादेरपि हेलादीनां प्रबोधो भवतीति मन्तव्यम्~। एतदन्योन्यसमृत्थितत्वम्~। \\

ननु यद्येते प्रकृतिस्थितास्तत एवाङ्गजास्तत्किमन्यापेक्षणेनेत्याह (वागङ्गेत्यादिना)~। वागङ्गमुखरागेणेत्यादिपाठः परं भावाध्यायश्लोको नास्य तुल्योऽर्थस्त्वन्य एव, न तु श्रीशङकुकेननार्थं(र्धे ?) एकार्थं मन्तव्यम्~। एवं चित्तवृत्तिलक्षणं देहधर्मस्येति सर्वसमतम्~। तस्मादयमर्थः\textendash\ वागङ्गमुखरागैः सत्त्वेन च लक्षितो भावः वागङ्गसत्त्वविशेष एव बालिकाया भाव इत्युच्यत इत्यर्थः~। किमपि विशेषो नेत्याह~। किं त्वन्तर्गतं वासनात्मतया वर्तमानं रसाख्यं भावं भावयन्सूययन् किं सर्वस्य नेत्याह कवेः सूक्ष्मसूक्ष्मानपि योऽर्थान् पश्यति तस्य सहृदयस्येत्यर्थः~। \\

एतदुक्तं भवति\textendash\ उत्तमाधमरूपे कुमारीद्वितये व्यवहरति (एकायाः) वाक् स्पन्दते चक्षुरादिव्यापारः क्रीडनकावहारखेदजनितमुखवैवर्ण्य बाष्पादि च पश्यतः हृदयस्य भवति तावद्विशेषोल्लासिनी अभिनयजनितेवानुमातृरूपा अपि तु विशेषाध्यवसायिनी मतिः, महतीयं काचिन्नायिकाभविष्यतीति~। तथाविधं यद्वागादेरान्तररतिवासनासद्भावसमुपनतं किंचिद्विशिष्टरूपत्वं स देहविकारविशेषो भावः~। चशब्द एक इवशब्दार्थ, अभिनयतुल्यो वागादिभिर्लक्षितो भाव॒इत्यर्थः~।\\

तत्रेति तत्पुरुष एव (उत्तमाङ्गना)पात्रलक्षणेन चोद्भ्रूतारकचिबुकग्रीवादेः सातिशयो विकाररूपो धर्मः, अत एव श्रृङ्गारोचितमाकारं सहृदयासहदयसर्वजनहृदयं सूचयतीति~। हावः\textendash\ एष हि स्वचित्तवृत्तिं परत्र जुह्वतीं ददतीं तां कुमारीं हावयति~। स्थितसमुत्थित इति स्थितः स्वयं समुत्थितः स्वेत्यु(स्वतः?) 

\newpage
% द्वाविंशोऽध्यायः १५७ 

\begin{quote}
 \renewcommand{\thefootnote}{1}\footnote{च \textendash\ य एव, न \textendash\ य एष भ \textendash\ यो वै भावः स एवैषां }यो वै हावः स एवैषा शृङ्गाररससंभवा~।\\
समाख्याता बुधैर्हेला ललिताभिनयात्मिका~॥ ११ 
\end{quote}

\hrule

\vspace{2mm}
\noindent
द्भिद्योद्भिद्य विश्राम्यन् हावः, स तु प्रसरणैकधर्मकः, तथा हि हेला स्यात् , अत एवायं सुकुमारपरिकरसब्रह्मचारीति दर्शितम्~। हावावस्थायां यत्स्वयं रतेः प्रबोधनं न मन्यते केवलं तत्संस्कारबलात्तथाविकारान् करोति~। यैर्दृष्टा तथा कल्पयति~। यदा तु रतिवासनाप्रबोधात्तां प्रबुद्धां रतिमभिमन्यते केवलं समुचितविभावोपग्रहविरहान्निर्विषयतया स्फुटीभावं न प्रतिपद्यते तदा तज्जनितो देहविकारविशेषो हेला~। 'हिल भावकारण' इति (धातुपाठे) पठ्यते~। भावस्य संबन्धादिति या प्रसरता वेगवाहित्वमित्यर्थः~। वेगेन गच्छत् हेलतीत्युच्यते लोके~। तदाह श्रृङ्गारेति~। श्रृङ्गाररसो रतिः ततो हृदये स्थिता या हेला संभवतीत्यर्थः~। तथा श्रृङ्गारस्य रस(स्य?)मानतायां यादृक्साधारणमिव रूपं तस्य संभवः संभावना या स्यात् सामाजिकशृङ्गाररसास्वादसदृशरूपैव संभावना चमत्कारमात्रप्राणा~। तथा हि तस्या यावद्विष यार्जनं किंचिदवभाति विभावविशेषापरिस्फुरणादिति वरसुन्दर(रूपोत्कीर्ण) ग्रावकल्पशैशवदशोत्तीर्णतारुण्योन्मीलना अत एव ललिता चेष्टा अभिनयरूपतामिव अस्यां विकारावेशातिशयवशात् न प्रतिलभते~। क्रमेणोदाहरणान्येषाम्\textendash\ 

\begin{quote}
 {\qt उत्तालालकभञ्जनानि कबरीभारोऽथ शिक्षारसो \\
दन्तानां परिकर्म नीविनहनं भ्रूलास्ययोग्याग्रहः~। 

तिर्यग्लोचनवल्गितानि वचसां छेकोक्तिसंक्रान्तयः\\ 
स्त्रीणां म्लायति शैशवे प्रतिकलं कोऽप्येष केलीक्रमः~॥ (विद्ध १)\\

स्मितं किंचिन्मुग्धं तरलमधुरो दृष्टिविभवः \\
 परिस्पन्दो वाचामभिनयविलासोक्तिसरसः~। 

गतानामारम्भः किसलयितलीलापरिकरः \\
स्पृशन्त्यास्तारुण्यं किमिव हि न रम्यं मृगदृशः~॥
}
\end{quote}

\newpage
% १५८ नाट्यशास्त्रम्

\begin{quote}
 {\na लीला विलोसो\renewcommand{\thefootnote}{1}\footnote{च \textendash\ विलासौ } विच्छित्तिर्विभ्रम किलिकिञ्चितम्\renewcommand{\thefootnote}{2}\footnote{2 न \textendash\ किञ्चितः}~।\\
 मोट्टायितं कुट्टमितं\renewcommand{\thefootnote}{3}\footnote{न \textendash\ कृट्टिमितं} विब्बोको\renewcommand{\thefootnote}{4}\footnote{ढ\textendash\ बिम्बोको } ललितं\renewcommand{\thefootnote}{5}\footnote{ड \textendash\ बिब्बोकललिते} तथा~॥ १२ 

 विहतं चेति विज्ञेया\renewcommand{\thefootnote}{6}\footnote{भ \textendash\ संप्रोक्ता च \textendash\ संयुक्ता} दश स्त्रीणां\renewcommand{\thefootnote}{7}\footnote{भ \textendash\ स्त्रीषु} स्वभावजाः~।\\ 
 \renewcommand{\thefootnote}{8}\footnote{ज\textendash\ अलङ्कारास्तथैतेषां लक्षणं शृणुताञ्चितम् (ड \textendash\ शृणुत द्विजाः) }पुनरेषां स्वरूपाणि प्रवक्ष्यामि पृथक्पृथक्~॥ १३}
\end{quote}
 
\hrule
 
\begin{quote}
 {\qt कुरङ्गीवाङ्गानि स्तिमितयति गीतध्वनिषु यत्\\ 
 ससखीं कान्तोदन्तं श्रुतमपि पुनः प्रश्नयति यत्~। 

अनिद्रं यच्चान्तः स्वपिति तदहो वेद्म्यभिनवां . \\
 प्रवृत्तोऽस्याः सेक्तुं हृदि मनसिजः प्रेमलतिकाम्~॥ }
\end{quote} 


अत्र हि भावान्तर्गतरतिप्रबोधमात्रमुक्तम्~। न त्वभिलाषः श्रृङ्गार इति मन्तव्यम्~। तत्परं ब्राह्मणस्योपनयनमिव भविष्यत्समस्तपुरुषार्थसद्मपीठबन्धत्वेन योषितां परमो ह्युत्सवः लोकोत्तरोऽलङ्कारः सातिशयमानन्दस्थानं परं पवित्रमित्युपश्रूयते~। यद्यपि चैते पुरुषस्यापि भवन्ति तथापि योषितां त एवालङ्कारा इति तद्गतत्वेनैव वर्णिताः~। पुंसस्तूत्साहवृत्त्यात एव परमालङ्काराः, तथा च सर्वेष्वेव नायकभेदेषु धीरत्वमेव विशेषणतयोक्तम्~। तदाच्छादितास्तु श्रृङ्गारादयः धीरललित इत्यादौ~।\\

एवं त्रीनङ्गजान् व्याख्याय स्वाभाविकान्दशोद्दिशति लीला विलास इत्यादिना~। विशिष्टविभावलाभे रतौ सविशेषत्वेन स्फुटीभूतायां तदुपबृंहणकृता देहविकारा लीलादयः शाक्याचार्यराहुलकादिभिर्यन्मतं विशेषसौक्ष्म्यादनुपलक्ष्य हेलाहावादीन् लीलादिमध्य एव पठद्भिश्चेष्टैवालङ्काराभूतेति, एतावन्मात्रे विश्रम्य सामान्येन चेष्टा अलङ्कार इति, तदयुक्तम्~। 

\newpage
% द्वाविंशोऽध्यायः १५९ 

\begin{quote}
 {\na वागङ्गालङ्कारैः शिष्टेः प्रीतिप्रयोजितैर्मधुरैः~॥ \\
 इष्टजनस्यानुकृतिर्लीला ज्ञेया प्रयोगज्ञैः ~॥ १४ 

 स्थानासनगमनानां \renewcommand{\thefootnote}{1}\footnote{च \textendash\ नेत्रभ्रूवक्त्र}हस्तभ्रूनेत्रकर्मणां चैव~।\\ 
 उत्पद्यते विशेषो यः श्लिष्टः\renewcommand{\thefootnote}{2}\footnote{भ \textendash\ विक्लिष्टः म \textendash\ यः क्लिष्टः}स तु विलासः स्यात् ~॥ १५. 

 माल्याच्छादनभूषणं\renewcommand{\thefootnote}{3}\footnote{ड \textendash\ विभूषा} विलेपनानामनादरन्यासः~।\\ 
 स्वल्पोऽपि परां\renewcommand{\thefootnote}{4}\footnote{ड\textendash\ अधिकां} शोभां \renewcommand{\thefootnote}{5}\footnote{च \textendash\ नयति हि यत्सा तु}जनयति\renewcommand{\thefootnote}{6}\footnote{भ \textendash\ सा स्यात्तु, ड \textendash\ या सा तु}यस्मात्तु विच्छित्तिः~॥ १६ } 
\end{quote}
 
\hrule

\vspace{2mm}
अत्रैषां दशानां क्मेण लक्षणान्याह वागङ्गालङ्कारैरिति~। प्रियतमगतैः प्रीत्या तं प्रति बहुमानातिशयेन स्वात्मनि योजितैः मधुरैः सुकुमारैः न तु तदीयैरेवोद्धतैः मधुरैरपि विशिष्टैः न तु कल्पितत्वेनाभिमानैः अत एवानुकृतिनोद्धट्टकरूपेणाविकृतं(तः?) बहुमानः स्वात्मनि तत्स्वात्मीकरणेन~। एते च दश प्राप्तसंभोगत्वेऽपि भावयन्त्येव~। शोभादयस्तु सप्त भाविनोप्राप्तसंभोगतायामेव~। एतान् लीलादीन् कवयो लोकवाचोऽत्र कीदृशा साङ्कर्येण प्रयुञ्जते~। यथा\textendash\ " गतेषु ललीलाञ्चितविभ्रमेषु " (कुमा १\textendash\ ३३) इति~। ' तत्रोपचारोऽन्वर्थत्वं लौकिकी प्रसिद्धिर्वा प्रमाणीकर्तव्या~। तन्त्रज्ञेरेवं पठितव्यम्\textendash\ `'गतेषु लीलाञ्चितसुन्दरेषु " इति, तदसदिति भट्टेन्दुराजशिष्याः यतो ये लीलाविभ्रमप्रभृतयो भविष्यन्त्यस्तदुचिततया तदानीं शिक्ष्यत इव~। व्याहृत्य नीयते स हि [अभिनय] प्रयोगकालो लीलादेरिति विनयस्य तु राजहंसकर्तृत्वमुत्प्रेक्ष्यते~। \\

\underline{स्थानासनेति}स्थानमूर्ध्वता, आसनमुपविष्टता~।\\

स्थानकादावप्रयत्नशिक्षितमपि श्रृङ्गारबलादुपनीय तद्रूपं विलासः, श्लिष्ट इत्यनुल्बणम्~। यथा\textendash\ \\ 

बाळे डअंणस्सु विमण्णमआसणु(?)\textendash\ इत्यादौ~। \\

\underline{स्वल्पोऽपि परामित्यल्पत्यैव} परां शोभां जनयति सौभाग्यगर्वमहिमा ह्यसौ~। यथा\textendash\ 

\newpage
% १६० नाट्यशास्त्रम्

\begin{quote}
 {\na विविधानामर्थानां वागङ्गाहार्यसत्त्वयोगानाम्\renewcommand{\thefootnote}{1}\footnote{प \textendash\ वेगेन य \textendash\ युक्तानाम्}~। \\
 मदरागहर्षजनितो\renewcommand{\thefootnote}{2}\footnote{ज \textendash\ जनितहेलो} \renewcommand{\thefootnote}{3}\footnote{च \textendash\ योऽतिशयो विभ्रमः सु मतः (ज \textendash\ प्रोक्त) }व्यत्यासो विभ्रमो ज्ञेयः~॥ १७ 

 स्मितरुदितहसितभय\renewcommand{\thefootnote}{4}\footnote{च \textendash\ रोष ड \textendash\ रोषमोह भ \textendash\ हर्षदुःख}हर्षगर्वदुःखश्रमाभिलाषाणाम्\renewcommand{\thefootnote}{5}\footnote{ड \textendash\ षङ्गाणाम्}\\ 
\renewcommand{\thefootnote}{6}\footnote{च \textendash\ सङ्कट}सङ्करकरणं हर्षादसकृत् किलिकिञ्चितं ज्ञेयम्~। १८ }
\end{quote}

\hrule

\vspace{2mm}

कच उपरब्भउ सस्सर इसिणिअत्थ(?)\textendash\ इत्यादौ~। \\

यत्तु "सरसिजमनुविद्धं शैवलेनापि रम्यं (शाकु)\textendash\ इत्युदाहृतं तदसत्~। न ह्रत्रानादरन्यासः सौभाग्यगर्वकृतः, अपि तु तपस्विसमुचितवेषपरिग्रहणप्रायमित्यलम्~। \\

\underline{विविधानामिति }योगो भेदः तेन वागादिभेदेन बहुभेदेन च बहूनां च (अर्थाना) योऽन्यथा निवेशः पूर्ववत्सौभाग्यगर्वकृतः (स) विभ्रमः, तद्यथा, वचनेऽन्यथावक्तव्येऽन्यथाभाषणम् , हस्तेनादातव्ये पादेनादानम्, रशनायाः कण्ठे न्यासः इत्यादि~। मद्येन कृतो रागः प्रियतमं प्रत्येव बहुमानो हर्षः~। सौभाग्यगर्वो यथा\textendash\ \\

\begin{quote}
 {\qt चिरिअ बन्धिअ निच्चिप्पटणिच्चिअ बद्धजम्म अदेसि सहि~। \\
 सोहग्गमत्थि ए्कें चिअरिअ किप्पिण [वेण] णाहिणेवउ(?)~॥ }
\end{quote}

 \underline{स्मितरुदि॒तहसितेति }सङ्करेण संकीर्णतया हर्षाद्गर्वाद्यत्संकरणम्~। यथा\textendash\ \\

मह\ldots तिमळळमकळति हळहळन्ति सप्पदिओ\\

(व) ल्लवच्चर इव इ अणेकत्ति (?) इत्यादौ~। \\

अत्र हि गर्वश्रमदुःखस्मितरुदितहसितानि देशीपदैः क्रमेणोक्तानि~। 

\newpage
% द्वाविंशोऽध्यायः १६१ 

\begin{quote}
{\na इष्टजनस्य कथायां लीलाहेलादिदर्शने वापि\renewcommand{\thefootnote}{1}\footnote{भ \textendash\ हेलालीलाभिदर्शने स्याताम्}~।\\ 
 तद्भाव\renewcommand{\thefootnote}{2}\footnote{भ \textendash\ भावनकृते मोट्टायितमित्यभिख्यातम्}भावनाकृतमुक्तं मोट्टायितं नाम~॥ १९ 

 केशस्तनाधरादिग्रहणादति\renewcommand{\thefootnote}{3}\footnote{भ \textendash\ आदिषु ग्रहणेष्ट्वति ड \textendash\ ग्रहणेष्ट्वति}हर्षसंभ्रमोत्पन्नम्~।\\ 
 \renewcommand{\thefootnote}{4}\footnote{ड \textendash\ कुट्टिमितं}कुट्टमितं \renewcommand{\thefootnote}{5}\footnote{भ \textendash\ अनवदते सुखं तु}विज्ञेयं सुखमपि दुःखोपचारेण~॥ २० 

 इष्टानां भावानां प्राप्तावभिमान \renewcommand{\thefootnote}{6}\footnote{च\textendash\ गर्भ ज \textendash\ गर्ह}गर्वसंभूतः~।\\ 
 स्त्रीणामनादरकृतो\renewcommand{\thefootnote}{7}\footnote{भ\textendash\ बिम्बोको} बिब्बोको नाम विज्ञेयः~॥ २१ 

 \renewcommand{\thefootnote}{8}\footnote{२२, २४ श्लोकौ भ \textendash\ मातृकायामेव दृश्येते }हस्तपादाङगविन्यासो भ्रूनेत्रोष्ठप्रयोजितः~।\\ 
 सौकुमार्याद्भवेद्यस्तु ललितं तत्प्रकीर्तितम् ~॥ २२ }
\end{quote}

\hrule
 
\vspace{2mm}
\underline{इष्टजनस्येति }कथने दर्शने वा कान्तस्य यदुत्पद्यते योषितो लीलादि तद्भावभावनवशान्मदनाङ्गमर्दपर्यन्तं तदङ्गमोडनान्मोट्टायितम्~। यथाह\textendash\ \\

सिइऊणथणसक्कियहत्थऊरू~।\textendash\ *इत्यादौ~। 

\underline{केशस्तनाधरग्रहणादिति }प्रियतमेनेति शेषः~। यथा\textendash\ 

 देशिखणंमि णअथणहिअइआपडि अपुणहे तिहिं अज्ज\textendash\ 

 णहरग्गस उकि किं दुरपसमहणाहलहबंधाहिं~। \renewcommand{\thefootnote}{*}\footnote{प्राकृतभागोऽस्पष्टार्थः }इत्यादौ~। 

 इष्टानामिति वस्त्रालङ्कारादीनामिति अनादरकृत इति तद्विषय एव योऽ 

\noindent
नादरकृतस्तद्वहुलम् (विव्बोकम्?)~। यथा\textendash\ चन्दघसि नामकोप्पेस स तु किं दु आविळं उइआ को चण्ड~।*~इत्यादौ~॥ \\

\underline{हस्तपादाङ्गविन्यास} इति कर्तव्यवशादायत एव हस्तादिकर्माणि यदैचित्र्यं स विलासः~। ललिते तु यत्न बाह्यव्यापारयोग एव न किंचिदस्ति नादातव्यबुद्धिः~। अथ च सुकुमारकरव्यापारणं न दुष्टस्य किंचित्, अथ च 

\lfoot{21} 

\newpage
\lfoot{}
% १६२ नाट्यशास्त्रम् 

\begin{quote}
 {\na [करचरणाङ्कन्यासः सभ्रूनेत्रोष्ठसंप्रयुक्तस्तु~। \\
 सुकुमारविधानेन स्त्रीभिरितीदं स्मृतं ललितम् ]~॥ 

 वाक्यानां प्रीतियुक्तानां प्राप्तानां यदभाषणम्~।\\ 
 व्याजात्स्वभावतो वापि विहृतं नाम तद्भवेत्~। २४ 

 [प्राप्तानामपि वचसां क्रियते यदभाषणां ह्रिया स्त्रीभिः~। \\
 व्याजात्स्वभावतो वाप्येत\renewcommand{\thefootnote}{1}\footnote{ड \textendash\ ह्येतत्} त्समुदाहृतं विहृतम्~॥] २५ 

 शोभा कान्तिश्च दीप्तिश्च तथा माधुर्यमेव च~।\\ 
 धैर्यं प्रागल्भ्यमौदीर्यमित्येते स्युरयत्नजाः~॥ २६ 

 रूपयौवनलावण्यैरुपभोगोपबृंहितैः~।\\
 अलङ्करणमङ्गानां शोभेति परिकीर्तिता\renewcommand{\thefootnote}{2}\footnote{ड \textendash\ यत् सा शोभेति भण्यते }~॥ २७ }
\end{quote}

\hrule

\vspace{2mm}

\noindent
तारादिकर्मेति विशेषः~। यथा\textendash\ 

कि अणिं लोपळळविअरुपकिसारे इचच्चा ए स बहुमजत्ति\ldots .. भराबहुखिसिदुपुणाखिळळविळूण अ अणळळहिखणु अ~॥ \renewcommand{\thefootnote}{*}\footnote{प्राकृतभागोऽस्पष्टार्थः }इत्यादि~।\\ 

अन्ये तु ' लड विलास ' इति (धातु)पाठं प्रमाणयन्तो विलासमेव सातिशयं ललितसंज्ञं मन्यन्ते~। \\

 \underline{वाक्यानां प्रीतियुक्तानामिति~। प्राप्तानामित्यवसरलाभेन} कथने योग्यानामित्यर्थः~। खभावत इति मौग्ध्याद्वाल्यादन्यचित्तत्वाद्वा~। \\

व्याजादिति~। व्याजादिभिर्मौग्ध्यादिभिः प्रख्यापनातिशयेनेत्यर्थः~। {तत्प्रख्यापनमपि कासांचित् स्वभाव एव~। यथा कविभिस्तु\ldots .ळिळळि\textendash\ } करन्ति अ इच्छिहि पुणच्छ मरणुकरन्ति अ \renewcommand{\thefootnote}{*}\footnote{प्राकृतभागोऽस्पष्टार्थः }\textendash\ इत्यादौ~।\\ 

 \underline{अथायत्नजा} इति~। शोभाकान्तिरित्यादि~। एषां क्रमेण लक्षणानि~। 

\underline{रूपयौवनलावण्यैरिति~।} तान्येव रूपादीनि पुरुषेणोपभुज्यमानानि छायान्तरं 

\newpage
 % द्वाविंशोऽध्यायः १६३ 
 
 \begin{quote}
 {\na विज्ञेया च तथा कान्तिः शोभैवापूर्णमन्मथा~। \\
 कान्तिरेवाति\renewcommand{\thefootnote}{1}\footnote{ज \textendash\ अथ }विस्तीर्णा दीप्तिरित्यभिधीयते ~॥~। २८ 

 सर्वावस्थाविशेषेषु दीप्तेषु ललितेषु च\renewcommand{\thefootnote}{2}\footnote{भ \textendash\ वा }~। \\
 अनुल्बणत्वं चेष्टाया\renewcommand{\thefootnote}{3}\footnote{भ \textendash\ चेष्टायां} माधुर्यमिति संज्ञितम्~॥ २९ 

 चापलेनानुपहृता \renewcommand{\thefootnote}{4}\footnote{य \textendash\ सर्वावस्थेष्वविकत्थना (" नवाक्षरपादो भुरिचि ") }सर्वार्थेष्वविकत्थना\renewcommand{\thefootnote}{5}\footnote{च \textendash\ अनुकत्थना}~। \\
 स्वाभाविकी चित्तवृत्तिर्धैर्यमित्यभिधीयते\renewcommand{\thefootnote}{6}\footnote{च \textendash\ संज्ञितम्}~। ३० 

 प्रयोगनिस्साध्वसता प्रागल्भ्यं ससुदाहृतम्~। \\
 औदार्यं प्रश्रयः प्रोक्तः सर्वावस्थानुगो बुधैः~॥ ३१ }
 \end{quote}
 
\hrule

\vspace{2mm}

\noindent
श्रयन्ति~। सा च्छाया मन्दमध्यतीव्रत्वं ऋमेण संभोगपरिशीलनादाश्रयति शोभां कान्ति दीतिं चेत्यर्थः~। आ समन्तात् पूर्णो मन्मथ इति कामोपभोगो हेतुर्यस्याः सा इत्यर्थः~। अन्यस्तु (अ)पूर्णमन्मथेति व्याचक्षाणः कन्तिदीप्तिशोभानां क्रमेण सातिशयत्वमाह~। तच्चोपक्रमविरुद्धमित्युपाध्यायाः~। \\

दीप्तेष्विति क्रोधादिषु~। चशब्द इवार्थे ललितेषु रति क्रीडादिषु यथामासृष्यं चेष्टायास्तथा दीप्तेष्वपि यत्तन्माधुर्यम्~। \\

सर्वार्थेष्विव रूपयौवनादिषु वर्गत्वाच्चेयं क्रिया रूपेभ्यः पृथगेव धीरता पठिता~। प्रयोग इति कामकलादौ चातुःषष्टिक\renewcommand{\thefootnote}{*}\footnote{बाभ्रवीयोक्ता आलिङ्गनादिचतुःषष्टिकला वात्स्ययनेन सांप्रयोगिकेऽनूदिताः~। } 
 इत्यर्थः~। यथाहुः\textendash\ 

\begin{quote}
 {\qt अन्यदा भूषणं पुंसः शमो लज्जेव योषितः~। \\
पराक्रमः परिभवे प्रागल्भ्यं सुरतेष्विव~॥ इति~। }
\end{quote}

यत्त्वनुकर्तृविषयमेतदिस्यन्यैर्व्याख्यातं तत्पूर्वमेव दूषितम्~। सर्वास्वर्मेर्ष्याक्रोधाद्यवस्थास्वपि यत्परुषवचनाद्यनुदीरणं तदौदार्यम्~। चित्तवृत्तिस्वभावा 

\newpage
% १६४ नाट्यशास्त्रम् 

\noindent
अपि केचिदेते विभावजन्यत्वाभावाद् भाववर्गे न पठिताः रसान् प्रति भावकत्वाभावाच्च इत्याहुः~। तच्चैतदयुक्तम्~। शोभाकान्तिदीप्तयः ता बाह्यरूपलावण्यगता एव विशेषाः आवेगचापलत्रासामर्षा भावा एव~। माधुर्याद्या न {चित्तवृत्तिस्वभावा इति क एषु भावत्वाशङ्कावकाशः इत्यभावो\$पि भावान्तर\textendash\ } तया तद्विशेषणतया प्रतिभासगोचर इति अतो भावरूपतैवेति चेदस्तु नामैवम्~। तथायलङ्कारत्वात्, सामान्याभिनयरूपत्वात्, बाह्यशरीरनिष्ठतापर्यवसानात्, श्रृङ्गारैकमात्रविषयत्वाच्च, अशेषरसविषयत्वात्, व्यभिचारिवर्गात् पृथक्त्वेनैषामभिधानम्~।\\ 

[न च] एतावत एवैत इत्यत्र नियमो विवक्षितः~। तेन मौग्ध्यमदभावविकृतपरितपनादीनामापि (शाक्याचार्यराहुलादिभिरभिधानं * विरुद्ध मित्यलं बहुना~।

\hrule

\vspace{2mm}

* राहुलादिभिरिति, आदिशब्देन पद्मश्रीसागरनन्दिमातृगुप्तप्रभृतये गृहीताः ~। तन्मते मौग्ध्यं यथा\textendash\ 


बाल्ये गते वचोभङ्गी रामाणां कान्तसन्निधौ~। 
हारिमोक्तिमयी या तु तन्मौग्ध्यं परिकीर्तितम्~॥ 

मुदो यथा\textendash\ 

 तारुण्यातिशयोद्भूतः सुरापानविशेषितः~। 

 विकारबहुलो यस्तु तं वदन्ति मदं बुधाः~॥ 

भावो यथा\textendash\ 

 कान्तस्य दृष्टिपथतस्तिरोधातुमिवेच्छति~। 

 लज्जयाधोमुखी मुग्धतिर्यग्विक्षिप्तलोचना~॥ 

 प्रियं पश्यत्यतिशयजातरोमाञ्चकञ्चुका~। 

 तत्क्षणोद्भूतमदनाचार्यशिक्षोपदेशतः~॥ 

 यत्तस्यां जायते चेष्टा स भावः शाक्यसंमतः~। 

विकृतम्\textendash\ 

 वचसा प्राप्तकालेऽपि प्रियया नाभिधीयते~। 

 क्रियया यदनुष्ठानं विकृतं तदुदाहृतम्~॥ 

परितपनम्\textendash\ 

 क्षणमात्रमदृष्टे या प्रिये सन्तापसन्ततिः~। 

 विरहोत्थेव कामिन्याः परितापः स कथ्यते~॥ 

आदिशब्देन विक्षेपकेलिव्याजप्रतिभेदनादय ऊह्याः~। 

\newpage
% द्वाविंशोऽध्यायः १६५ 

\begin{quote}
 {\na सुकुमारे भवन्त्येते प्रयोगे ललितात्मके~। \\
विलासललिते हित्वा दीप्तेऽप्येते\renewcommand{\thefootnote}{1}\footnote{प \textendash\ दीप्तेष्ट्वेते} भवन्ति हि~॥ ३२ 

शोभा विलोसो माधुर्यं\renewcommand{\thefootnote}{2}\footnote{ढ \textendash\ धैर्यं} स्थैर्यं गाम्भीर्यमेव च~।\\ 
ललितौदार्यतेजांसि सत्त्वभेदास्तु पौरुषाः~। ३३ 

दाक्ष्यं शौर्य\renewcommand{\thefootnote}{3}\footnote{ड \textendash\ तथा }मथोत्साहो नीचार्थेषु जुगुप्सनम्\renewcommand{\thefootnote}{4}\footnote{च \textendash\ जुगुप्सितम्}~। \\
उत्तमैश्च गुणैः \renewcommand{\thefootnote}{5}\footnote{ड \textendash\ सन्धा}स्पर्धा यतः शोभेति सा स्मृता~॥ ३४ 

\renewcommand{\thefootnote}{6}\footnote{ड \textendash\ स्थिर भ \textendash\ वीर}धीरसंचारिणी दृष्टिर्गतिर्गोवृषभाञ्चिता~।\\ 
\renewcommand{\thefootnote}{7}\footnote{ढ \textendash\ स्मृतपूर्वं तथा वाचो ज \textendash\ स्मितपूर्वस्तथा भ \textendash\ मितं दन्तप्रभालक्ष्यं}स्मितपूर्वमथालापो विलास इति कीर्तितः\renewcommand{\thefootnote}{8}\footnote{भ \textendash\ स स्मृतः}~॥ ३५ 

\renewcommand{\thefootnote}{9}\footnote{भ \textendash\ स्वभावाच्चक्षुरादीनां लीनत्वं यत्र जायते }अभ्यासात्करणानां तु श्लिष्टत्वं यत्र जायते~। \\
महत्स्वपि विकारेषु तन्माधुर्यमिति स्मृतम्~॥ ३६ }
\end{quote}

\hrule

\vspace{2mm}

\begin{sloppypar}
अथैषां सामान्याभिनयत्वमुपपादयितुमाह \underline{सुकुमारे भवन्त्येत} इति~।\underline{ ललितात्मके प्रयोगे} प्रयुज्यमाने श्रुङ्गारे यः सुकुमारोऽन्योन्योपसंभोगविप्रलम्भादि भेदः~। तत्र सर्वत्रैते न भवन्ति~। न ह्येतच्छून्यमङ्गानां चेष्टितं प्रयोगार्हम्~। योऽपि तत्र श्रृङ्गारे ईर्ष्यामर्षदीप्तिप्रकारस्तत्रापि विलासं ललितं च वर्जयित्वा अवश्यमन्येषां क्रमयौगपद्यादिना संभव इति श्रृङ्गारभेदेषु साधारणभूतो योऽभिनयस्तेन सामान्याभिनयतास्य युक्तेति तात्पर्यम्~।\\ 
\end{sloppypar}


अथ पुरुषगतानुद्दिश्य लक्षयति\underline{शोभा विलास }इत्यादि~।\underline{यतः} शरीरविकाराद् दाक्ष्यादि गम्यते सा शोभेति संबन्धः~। गोवृषभः उत्तमा गौः~। \\

\underline{महत्स्वपि विकारेष्विति}युद्धनियुद्धव्यायामादिष्वभ्यासकृतं, करणानां 

\newpage
% १६६ नाट्यशास्त्रम् 

\begin{quote}
 {\na धर्मार्थकामसंयुक्ताच्छुभाशुभसमुत्थितात्~। \\
व्यवसायादचलनं स्थैर्यमित्यभिसंज्ञितम् ~॥~। ३७ 

यस्य प्रभावादाकारा\renewcommand{\thefootnote}{1}\footnote{ड \textendash\ आकारे रोषहर्ष भयादिषु~। भावेषु नोपलभ्यं यत् ढ आकारो } हर्षक्रोधभयादिषु~।\\ 
भावेषू नोपलक्ष्यन्ते\renewcommand{\thefootnote}{2}\footnote{च \textendash\ लभ्यन्ते} तद्गाम्भीर्यमिति स्मृतम्~॥ ३८ 

अबुद्धिपूर्वकं\renewcommand{\thefootnote}{3}\footnote{ज \textendash\ पूर्वजं} यत्तु निर्विकारस्वभावजम्\renewcommand{\thefootnote}{4}\footnote{भ \textendash\ मण्डनं निर्विकारजम् च \textendash\ सुकुमारः स्वभावतः ज \textendash\ सुकुमारस्वभावजम्}~।\\ 
\renewcommand{\thefootnote}{5}\footnote{भ \textendash\ शृङ्गारसूचकं चैव }श्रृङ्गाराकारचेष्टत्वं ललितं तदुदाहृतम्\renewcommand{\thefootnote}{6}\footnote{ज \textendash\ प्रकीर्तितम् } ~॥ ३९ 

दानमभ्युपपत्तिश्च\renewcommand{\thefootnote}{7}\footnote{ड\textendash\ अवपत्तिः } तथा च प्रियभाषणम्~। }
\end{quote}

\hrule

\vspace{2mm}

\noindent
करचरणादिक्रियाणां श्लिष्टत्वं अनुल्बणत्वं यत्न शरीरविकारो स माधुर्वम्~। नटगतमेतदिति त्वसत् प्रक्रमविरोधात्, प्रयोक्तृगुणानां च प्रकृत्यध्याये भूमिकाविकल्पाध्याये च (अ ३५, ३६) वक्ष्यमाणत्वात्~। \\

धर्मादयो यस्य फलं स तैः संयुक्त इत्युक्तम्~। शुभाशुभसमुत्थितादिति~। शुभसंकल्पजा अशुभसङ्कल्पजाश्च~। अशुभादपि हि सङ्कल्पोऽस्थिरान्निवर्तते~। यथैलश्चातुर्वर्ण्यस्य सर्वस्वहारादर्धफलाद्याशयात्~।\\

अन्ये तु वीरस्यैतदनुचितमिति मत्वान्यथा व्याचक्षते शुभाशुभयोः समुत्थित इति~। तेन यच्छास्त्रोक्तमुचितं चारभ्यते~। तत्र क्रियमाणे (शुभं) सुलभतयार्थलाभः, अशुभं क्षयव्ययादिरूपकमस्तु तथापि तद्विषयाध्यवसायादविचलनं स्थैर्यं देहविकाररूपमेव~। \\

आक्रियते चित्तवृत्तिरेभिरित्याकारः, मुखरागदृष्टविकारादयः~। तेष्वपि चासत्सु कारणसामग्रयव्यभिचरितफलेति हर्षादिसंभवेऽपि यत्कृतस्तत्कृतमुखरागाद्यभावः स एव निस्तिमितदेहस्वभावो गाम्भीर्यम्~।\\

परजनविषयं दानादि चेष्टाविकाररूपमेवौदार्यय्~। स्वग्रहणन्तु लोकोक्तिमनादररूपरूपान्तु भयं नोदाहार्य(नौदार्य)नेव स्फुटयति~। अभ्युपपत्तिः परि\textendash\ 

\newpage
 % द्वाविंशोऽध्यायः १६७ 

\begin{quote}
 {\na \renewcommand{\thefootnote}{1}\footnote{भ \textendash\ स्वे जने वापरं}स्वजने च परे वापि तदौदार्यं प्रकीर्तितम्\renewcommand{\thefootnote}{2}\footnote{ज \textendash\ इति स्मृतम्}~॥ ४० 

अधिक्षेपावमानादेः\renewcommand{\thefootnote}{3}\footnote{च \textendash\ अपमानादेः भ\textendash\ अव\textendash\ मानाद्यैः} प्रयुक्तस्य परेण यत्~।\\ 
प्राणात्ययेऽप्यसहनं तत्तेजः समुदाहृतम् ~॥ ४१ 

\renewcommand{\thefootnote}{4}\footnote{च \textendash\ सत्त्वतोऽभिनयः}सत्त्वजोऽभिनयः पूर्वं\renewcommand{\thefootnote}{5}\footnote{च \textendash\ मयोक्ता (क्तो ?) द्विजसत्तमाः} मया प्रोक्तो द्विजोत्तमाः~। \\
शारीरं चाप्यभिनयं व्याख्यास्याम्यनुपूर्वशः ~॥ ४२ 

षडात्मकस्तु शारीरो वाक्यं सूचाङ्कुरस्तथा~।\\ 
शाखा नाट्यायितं चैव निवृत्यङ्कूर एव च ~॥ ४३ }
\end{quote}

\noindent
त्राणाद्यर्थिनोऽङ्गीकरणम्~। परेणेति शत्रुणा, न तु गुरुणा मित्रादिना वा ~। प्राणात्ययेऽपीति न तु नीत्यनुवर्तनेन कथंचित् देशकालाद्यनुवर्तनेन सहनपूर्वकं निर्यातनम्~। तथा च ममैव श्लोकः\textendash\ 

\begin{quote}
 {\qt मूर्ध्ना कंचन पार्थिवं धृतवता च्छिद्रैः प्रविश्यान्तरं \\
स्वं रूपं विनिगूह्य तापमतुलं दत्त्वा चिराद् भ्रंशितः~। 

यश्चङ्कुर्जडरूपतापरिचितो नीत्येव वन्हे त्वया\\ 
तन्मन्येऽत्र कथासु तिष्ठति परं निर्वन्ध्यतेजस्विता ~॥ इति~। }
\end{quote}

सात्त्विकः पूर्वमुक्त इत्यर्थस्यास्य केचिच्छङ्काशमनं प्रयोजनमाहुः\textendash\ एवं हि शंक्यते सात्त्विकप्रसङ्गेन कस्माद्रोमाञ्चादयो नोक्ता इति; तद्वारणार्थमाह पूर्वमिति भावाध्याय एव ते निरूपिता\textendash\ इति~। अयं च नार्षः, यतः सत्त्वे नाट्यं प्रतिष्ठितं, अतो वागङ्गसत्त्व इत्यत्त पश्चान्निर्दिष्टः सात्विक: सामान्याभिनयो यस्मात् पूर्वमिति आदौ प्रोक्तः ततो हेतोर्यद्विलोमक्रमेण अनुपूर्वशः आनुपूर्व्य ऋमप्राप्तं तेन क्रमेण शारीरं सामान्याभिनयं वक्ष्यते~। सत्त्वानन्तरं हि विपरीतवृत्त्या आङ्गिकस्यानन्तर्यं वागङ्गसत्व इति~। (शारीरमित्यादि) अत्र चशब्दो यस्मादर्थे, अपिशब्दस्तत इत्यत्रार्थे~। तत्रोद्देशमाहषडात्मकस्त्विति~।

\newpage
% १६८ नाट्यशास्त्रम् 

\begin{quote}
 {\na नाना\renewcommand{\thefootnote}{1}\footnote{च \textendash\ भावरसार्थैर्वृत्तनिबद्धैः कृतस्य}रसार्थयुक्तैर्वृत्तनिबन्धैः \renewcommand{\thefootnote}{2}\footnote{य \textendash\ निबद्धैः ढ\textendash\ नृत्तनिबद्धैः}कृत: सचूर्णपदैः\renewcommand{\thefootnote}{3}\footnote{भ \textendash\ पदैः स चूर्णकृतैः}~। \\
प्राकृतसंस्कृतपाठो\renewcommand{\thefootnote}{4}\footnote{न \textendash\ पाठ्यो भ \textendash\ पाठ्यैः} वाक्याभिनयो बुधैर्ज्ञेय :~॥ ४४}
\end{quote}

\hrule

\vspace{2mm}

\noindent
यः पूर्व शारीरोऽभिनयो बहुना प्रकारवैचित्र्येणोक्तः तस्यावान्तरसामान्याभिनयरूपा हि षढ्भ वन्तीति, स एव षडात्मकतयाऽवान्तरजातियोगात् सात्त्विकवाचिकाभिनयैश्च संभूय क्रमतां प्राप्तैश्च व्यामिश्रतायां सामान्याभिनयः संपाद्यत इति तुशब्दस्यार्थः~। (वाक्यमिति)वाक्यसहचिरत; शारीरोऽभिनयो वाक्यम्~। न हि शरीराभिनयमध्ये च निर्वर्तकानि गणितानीति चिरन्तनाः~। \\

शारीरस्वरा (अ २८) इति यद्व्यवहाराद्वा भावश्च शारीरत्वेन प्रसिद्धे वाक्यमेवाभिनयान्तरनिरपेक्षमिति, सद्वृत्तभाषागुणः स्फुटतमां स्वार्थप्रतीतिं यदा विधत्ते तत एव रसभावानुभावकं तदा तदेव वाक्याभिनय इति तु काव्यकौतुकग्रन्थः~। अत्र तु पाठ्यरूपः सामान्याभिनयः कथमिति चिन्त्यम्~। शरीरानुप्रवेशादितिचेत् त्वरितानपेक्ष (तदितरानपेक्ष ?) एव~। न चाभिनयशून्याभिनेयकाव्ये वाक्यं किंचिद्भवति~। न चाभिनेये वाक्याभिनयव्यवहार इत्युपाध्यायेनायमेकीयोऽभिप्रायो दर्शितः न त्वस्यायं स्वपक्ष इति भ्रमितव्यम्~। \\

तत्र वाक्याभिनयस्य लक्षणमाह\textendash\ 

\begin{quote}
 {\qt नानारसार्थयुरक्तैर्वृत्तनिबन्धैः कृतः स चूर्णपदैः~। \\
प्राकृतसंस्कृतपाठो वाक्याभिनयो बुर्धैर्ज्ञेयः ~॥ इति~।}\\
\end{quote}
 
नानारसविशेषो वाक्यार्थः तेन युक्तानि वाक्यानि यानि तानि च वृत्त\textendash\ रचितानि चूर्णपदात्मकानि वा पूनरपि संस्कृतानि वा प्राकृतानि वा तैर्युक्तः सहचरितः शारीरो वाक्येन सहैव प्रयुज्यमानः शारीरो वाक्याभिनय इति यावत्~। प्राकृतः संस्कृतश्च पाठोऽस्मिन्निति बहुव्रीहिः स चायं वाक्याभिनयश्चतुर्धा संस्कृतप्राकृतयोर्गद्यपद्यभेदात्~।\\

अत्र केचिदाहुः योऽर्थः सदैव हृदये वर्तते अत एव विमर्शानुबन्धनादिनिरपेक्ष एवं स सततं स्फुरति~। यथा भीमसेनस्य कुरुकुल\textendash\ 

\newpage
 % द्वाविंशोऽध्यायः १६९

\begin{quote}
 {\na वाक्यार्थो वाक्यं वा \renewcommand{\thefootnote}{1}\footnote{न \textendash\ सर्वाङ्गैः}सत्त्वाङ्गैः सुच्यते यदा पूर्वम्~। \\
पश्चा\renewcommand{\thefootnote}{2}\footnote{य \textendash\ वचना ड\textendash\ वाच्या}द्वाक्याभिनयः \renewcommand{\thefootnote}{3}\footnote{ड\textendash\ सा सूचा सूरिभि\textendash\ र्ज्ञेया}सूचेत्यभिसंज्ञिता सा तु~॥ ४५}
\end{quote}
 
\hrule

\vspace{2mm}

विषयः क्रोधातिशयः तद्विषये वाक्ये "चञ्चद्भुजाभ्रमित "(वेणी\textendash\ अ १) इत्यादौ पाठसमकालं भ्रुकृट्यादिमयः शारीरोऽभिनय इति~। एतच्चासत्~। सूचायां विमर्शपूर्वकवस्तुविषयायामपि प्रवृत्तायां यद्वाक्यं बध्यते तत्सहचरितोऽपि शारीरः किमिति न वाक्याभिनयः तथापि चतुर्विधवाक्याभिनययोगात् सूचादीनां बहुभेदत्वं वक्ष्याम इत्यास्तां तावत्~। \\

\begin{quote}
 {\qt वाक्यार्थो वाक्यं वा सत्त्वाङ्गैः सूच्यते यदा पूर्वम्~। \\
पश्चाद्वाक्याभिनयः सूचेत्यभिसंज्ञिता सा तु ~॥ इति~। }
\end{quote}

यदा त निपातः यदित्यत्रार्थे इह वर्तते~। तेन यैः सात्त्विकाङ्गिकैः भाविवक्तव्यं सुच्यते, येषामन्तरोऽभिनयः प्रवर्तत एव सूचाभिनयः~। " आत्मबुद्ध्या समर्थ्यार्था "निति हि न्यायो वा प्रसरेत्~। तत्रोत्तमानां बाहुल्येनाभिसंधानविचारपूर्वकं एवमिति चिरतरोऽसावभिसन्धिकालः तत्राविष्टस्यैवेहोपयोग इति\textendash\ विचार्यमाणं तथाभूतवस्तुविशेषावेशजनितेन शरीरविकारेणावश्यं भवितव्यं सममेव सूचाभिनयः~। तत्र च द्वयी गतर्विद्यते, स्थितेनैवाभिसन्धानेन यादृक्कम आक्षिप्तस्तादृशैव क्रमेण परतः शब्दोच्चारण्~। यथा ' राज्यं निर्जितशत्रु योग्यसचिवन्यस्तः समस्तो भरः' (रत्ना\textendash\ अ\textendash\ ४ ) इत्यादौ~। तत्र हि पूर्वपूर्वावान्तरवाक्यांशाभिधेयभागप्रभावित एवोत्तरोत्तरावान्तरवाक्यार्थ इति यादृगभिसन्धानक्रमस्तादृगेव तत्र क्रमः~। यत्र त्वन्यथाभिसन्धानमन्यथा च क्रमस्तत्र निर्विभागभेदकं सन्धानीयवाक्यार्थसंधानम्~। यथा (मायापुष्पके) सुग्रीवस्य\textendash\ 

\begin{quote}
 {\qt दुर्गे भूमिरमात्यभृत्यसुहृदो दाराः शरीरं धनं \\
मानो वैरिविमर्दसौख्यममरप्रख्येन सख्येन किम् (सख्योन्नतिः ?)~। 

यस्मात्सर्वमिदं प्रियाविरहितैस्तस्मादशक्ता(विरहिणस्तस्याद्य शक्ता?)वयं\\
न स्वेच्छासुलमैः पथोऽपि घटने शैलाश्मखण्डैरपि ~॥ }
\end{quote}

\lfoot{22}

\newpage
% १७० नाट्यशास्त्रम् 

\lfoot{}
 
 \begin{quote}
 {\na \renewcommand{\thefootnote}{1}\footnote{ब \textendash\ हृदयस्थैः}हृदयस्थो निर्वचनैरङ्काभिनयः\renewcommand{\thefootnote}{2}\footnote{य \textendash\ अङ्गविकारैः भ \textendash\ अभिनयैः} कृतो\renewcommand{\thefootnote}{3}\footnote{न \textendash\ कृते} निपुणसाध्यः~। \\
\renewcommand{\thefootnote}{4}\footnote{भ\textendash\ सूच्यैवाघू(पू ?)र्णकृतो ढ \textendash\ सूचेव}सूचैवौत्पत्तिकृतो विज्ञेयस्त्वङ्कुराभिनयः~॥ ४६ }
 \end{quote}

\begin{sloppypar}
अत्र हि शरीरं दारा भूमिर्धनं भृत्या दुर्गं वैरिविमर्दसुखं रामस्य मित्रमितिप्रसिद्धिरित्यभिसन्धानक्रम उचितो, निकट (पद) परामर्शक्रमेण प्रकरणा(र्थवशादिदं लब्धं) दूरं प्रसृत्य क्रमेण यद्वा इदं तावदास्तां, इदमपि तत इति न्यायेनान्तः प्रवेशः, तथापि रामस्य मित्रमितिप्रसिद्धिरित्यादिना विलोमक्रमेण भाव्यम्~। तत्र प्रथमे पक्षे वाक्यं सूच्यत इत्युक्तम्, क्रमो हि वाक्यमिति तद्विदो मन्यन्ते~। ' एको नववच \ldots . शब्दः क्रमो युध्यतः सुहृद्भि'\renewcommand{\thefootnote}{*}\footnote{एकैनैव वचः क्रमेण(?) } (?)रित्यादौ~।द्वितीयपक्षे तूक्तं वाक्यार्यः सूच्यत इति~।\\ 
\end{sloppypar}

अन्यस्त्वाह\textendash\ यदा स्वयमेव विमृश्यते तदा वाक्यार्थः सूचितः, यदा तु परवचनमाकर्ण्यते, यथा 'भो वयस्स पेक्ख पेक्ख' (रत्ना) इद्यादावुद्यानवर्णनं पूर्यते तदा तदुक्तोऽर्थः सूच्यते तस्य पश्चाद्वचनाभिनयो भविष्यति ' वयस्य सम्यगुपलक्षित 'मित्यादि, तत्र सूचाभिनये वाक्यं सूच्यत इति~। तच्चासत्परवाक्यापेक्षया हि तत्र निवृत्त्यङ्कुराभिनयः भाविवाक्यापेक्षया तु ततोऽन्यैव सूचा~। अनभिनेय एवाभिनयनलक्ष्यं प्रत्यपरिचयोऽयमपराध्यति~। किंचिद्वाक्यार्थप्रग्रहणेन किमयमर्थो न स्वीकर्तुं शक्यो येन पुनर्वाक्यशब्दोपादानं स्यात्, तस्मात्परोदीरितवाक्यार्थ एव ह्यसौ सूचितो न तु वाक्यमित्यास्ताम् ~। \\

तदिदमुक्तपूर्वं गद्यपद्येतरचतुःप्रकारवाक्यं तदर्थसूचने भेदादष्टधा सूचाभिनयादनु षोडशधा युक्तम्~। परवाक्यस्य न भिन्नाङ्गतेत्यधुनैवोपपादितम्~। 

\begin{quote}
 {\qt हृदयस्थो निर्वचनैरङ्गाभिनयः कृतो निपुणसाध्यः~। \\
मूचैवोत्पत्तिकृतो विज्ञेयस्त्वङ्कुराभिनयः ~॥ इति }
\end{quote}

अरन्यपरेऽषि वाक्ये यद्गर्भीभूतं हृदयस्थं वस्तु तन्निष्ठो योऽभिनयोऽङ्गविकारैर्वचनश्रून्यैः सम्पादितः सूचातुल्यः सोऽङ्कुरो नाम शारीरः~। यथा साग\textendash\ 

\newpage
% द्वाविंशोऽध्यायः १७१ 

\begin{quote}
 {\na \renewcommand{\thefootnote}{1}\footnote{1 च \textendash\ यस्तु भ \textendash\ यस्तु शिरोजङ्घोरुपाणिपादिभिर्विरचितो विधिवत् (~।) }यत्तु शिरो\renewcommand{\thefootnote}{2}\footnote{2 च \textendash\ भुज } मुखजङ्घोरुपाणिपादैर्यथाक्रमं क्रियते~। \\
शाखादर्शन\renewcommand{\thefootnote}{3}\footnote{3 भ \textendash\ देशित च \textendash\ दर्शित }मार्गः शाखाभिनयः स विज्ञेयः~॥ ४७ }
\end{quote} 

\begin{sloppypar}

\end{sloppypar}
\noindent
रिका\textendash\ (जाव अहं पि कुसुमाइं अवचाइअ कामदेवं पूअइस्सं)\textendash\ इत्यभिधायैतद्वाक्ये गर्भीभूतं कुसुमापचयमङ्गविकारैर्दर्शयति~। यच्चायमङ्कुरो निपुणैरेव प्रयोक्तृभिः सामाजिकैश्च साध्यः आपाद्यः चेतसा ध्यात उत्पत्त्या स्वबुद्धिकल्पनयापचितः~। यद्यपि कविवाक्यान्येवात्रोपजीव्यानि, तथापि यादृक् कुसुमापचयकर्म सागरिकायाः सौभाग्यसौन्दर्यप्रेमसाध्वसादिगर्भे न तादृशं {तापसस्य तदुभयविलक्षणं 'यः\ldots . वेद्या (?)' इत्येवमादि तत्सर्वं वचनेभ्य एवाकृष्यते तथा विस्पष्ठेन च तथाबचनतो लभ्यमेतत्~। अपि तु पर्यालोचनातिशयगम्यमिति निपुणसाध्यमित्युक्तम्~। 

अन्ये तु मूचाया उत्पत्तिभागेन तुल्योऽङ्कुरस्तस्याः प्राग्भावे वचनशून्यत्वादिति~। इदं त्वनुक्तसमासं (नं ?) निर्वचनशब्देनोक्तत्वादर्थस्य~। \\

\begin{quote}
 {\qt यत्तु शिरोमुखजङ्घोरुपाणिपादैर्यथाक्रमं क्रियते~। \\
शाखादर्शनमार्गः शाखाभिनयः स विज्ञेयः ~॥ }\\
\end{quote}

समस्तेन शाखाव्यापारेण वर्तनाप्रधानतया प्रयुक्तः शाखाभिनयः शिरोमुखजङ्घोरु चेत्यादिना कृतैकवद्भावेन द्वन्द्वपदसमूहेन पुनः (श्लोक) द्वन्द्वेन, नाट्यायितमित्यादि स्थान इत्यादि च~।\\ 

पूर्प्रविष्टस्य पात्रस्यापरपात्रं प्रविश्य तद्रूपमुदीक्षमाणस्य प्रवेशोऽपि तद्ध्रुवागानतत्सूचापरिक्रमणादिकालेन किंचिन्नाट्यमस्तीति तत्काले पूर्वपात्रेण ये समुचिता उपचाराः क्रियन्ते नाट्यायितमित्याद्यार्यायास्तात्पर्यम्~। पूर्वप्रविष्टेन पात्रेण सह सङ्गमं विधाय पश्चात्प्रविष्टस्य पात्रस्य पूर्वभविष्टपात्रपरिक्रमणादिकाले स्थानकेनैवासीनस्य तूष्णीं स्थितौ प्राप्तायामभिनयः तदपि नाट्यायितमित्यपरार्यायास्तात्पर्यमिति श्रीशङ्कुकाद्याः~। तच्चायुक्तम्~। अन्योन्यसङ्गमावधि यत्पात्रस्य चेष्टितं तदपरोदितवाक्यार्थसूचनोचितत्वं वा निर्वचनकादौचित्यमात्रादेवोपनतं वा, पूर्वत्र पक्षे निवृत्त्यङ्कुरः उत्तरत्राङ्कुरः इत्युभयं न नाट्यायितम्~। 

\newpage
% १७२ नाट्यशास्त्रम्

\begin{quote}
 {\na नाट्यायितमुपचारैर्यः क्रियतेऽभिनयसूचया\renewcommand{\thefootnote}{1}\footnote{न \textendash\ सूचना} नाट्ये~। \\
काल\renewcommand{\thefootnote}{2}\footnote{ज \textendash\ प्रहृर्ष} प्रकर्षहेतोः \renewcommand{\thefootnote}{3}\footnote{ड \textendash\ प्रवेशने सङ्गमं भ \textendash\ समागमे}प्रवैशकैः संगमो यावत्~॥ ४८ }
\end{quote}

\hrule

\vspace{2mm}

तथा हि प्रयोगकुशला एवंविधे विषये धर्मी\textendash\ लिखति इति, प्रति पालयन्नास्ते (इति), यथापुष्पापचयं नाट्यतीति~। नाट्यस्य सन्धानरूपत्वं च वाक्यं सूचादीनामपि संभवत्येव~। न वा नाट्येन नाट्यं सन्धीयत इति नाट्यायितवाचोयुक्तिरपि कथम्~। तस्मादित्थमेतद् व्याख्यातव्यम्\textendash\ इह यदा स्वप्नोऽप्येकघनो दृश्यते तन्मध्यत एव च किं दृश्यमानं परस्य स्वप्न एव जाग्रद्रूपताभापादिते स्वप्नोऽयं मया दृष्ट इति वर्ण्यते, तदा जाग्रदपेक्षया स्वप्नव्यवहारः, न तत्र पारमार्थिक इत्यौपचारिकं तदपेक्षं तस्य स्वप्नत्वमिति तस्य स्वप्नायितव्यवहारो दृष्टः~। एवमिहापि नाटय एकघनखभावे हि स्थिते तत्रैवासत्यनाट्यानुप्रवेशान्नाट्यपात्रेषु सामाजिकीभूतेषु तदपेक्षया यदन्यं नाट्यं तस्य तदपेक्षया नाट्यरूपत्वं पारमार्थिकमिति नाट्यायितमुच्यते~। तच्च द्विविधं नाट्यरूपकनिष्ठमेव वा कार्यान्तरनिष्ठं वा~। तस्य क्रमेण लक्षणमार्याद्वितयेनोच्यते~। नाट्ये यत्प्रवेशकैर्नाट्यान्तरगतैरिव पात्रैः अत एव ततः प्रविशतीत्युरैः सङ्गमः क्रियते तन्नाट्यायितम्~। कीदृशैरभिनयद्वारेण यत्सूचनं तयोपचारैः परमार्थतयोपचर्यमाणैः~। ननूभयमपि नाट्यं कस्मान्न भवति नत्वेकघनतेत्याशङ्क्याह कालपकर्षलक्षणादेतोरन्योन्यभिन्नकालत्वात् कथं तत्रैकघनता युक्तेति भावः~। यावदिति भूयस्तरं प्रबन्धं व्याप्नोतुं परिमितं वा सर्वं नाट्यायितमित्यर्थः~। तथा यावदिति स्वप्ने स्वप्नान्तरं तत्राप्यन्यत् स्वप्नान्तरमित्यादिन्यायेन वा भवत्वेक(घन)स्वप्नायितवृत्या वा सर्वथा तन्नाट्यायितम्~। तत्रास्यबहुतरव्यापिनो बहुगर्भस्वप्नायिततुल्यस्य नाट्यायितस्योदाहरणं महाकविसुबन्धुनिबद्धो वसवदत्तनाट्यधाराख्यः स समस्त एव प्रयोगः~। तत्र हि बिन्दुसारः प्रयोज्यवस्तुक उदयनचरिते सामाजिकीकृतः, असावप्युदयनो वासवदत्ताचेष्टिते~। एष चार्थः\textendash\ स्वस्मिन् सूत्ररूपके दृष्टे सुज्ञानो भवति~। अतिवैतत्यभयात्तु न प्रदर्शितः~। एकस्तु प्रदेश उदाह्रियते तत्र ह्युदयने सामाजिकीकृते सूत्रधारप्रयोगः\textendash\ "तव सुचरितैरेष जयति' इति, तत उदयनः ' कुतो मम सुचरिता ' नीति सास्रं विलपति\textendash\ 

\newpage
% द्वाविशोऽध्यायः १७३ 

\begin{quote}
 {\na स्थाने ध्रुवास्वभिनयो \renewcommand{\thefootnote}{1}\footnote{च \textendash\ यत् }यः क्रियते हर्षशोकरोषाद्यैः\renewcommand{\thefootnote}{2}\footnote{ज\textendash\ कोपाद्यैः}~। \\
भावरससंप्रयुक्तै\renewcommand{\thefootnote}{3}\footnote{भ \textendash\ संप्रयुक्तं प \textendash\ संप्रविष्टैः ढ \textendash\ संप्रयुक्तो ज्ञेयो} ज्ञेयं नाट्यायितं तदपि\renewcommand{\thefootnote}{4}\footnote{न \textendash\ तच्च}~। ४९}
\end{quote}

\hrule

\vspace{2mm}

\begin{sloppypar}
\begin{quote}
 {\qt \renewcommand{\thefootnote}{*}\footnote{कटकादय उदयनस्य भ्रातरः }एह्यम्ब किं 'कटकपिङ्गलपालकैस्तैर्भक्तोऽहमप्युदयनः सुतलालनीयः~। \\
यौगन्धरायण ममानय राजपुत्नीं हा हर्षरक्षित गतस्त्वमपप्रभावः~॥ } 
\end{quote}
\end{sloppypar}

तत्रैव बिन्दुसारः सामाजिकीभूतः परमार्थतामभिमन्यमानो " धन्या खलु (ईदृशैर्भक्तस्य) प्रलापेः" इत्युच्छ्सति~। प्रतीहारी आत्मगतं\textendash\ " अअणिदपरमत्थकळणेहिं पिच्छइ खु देवो " इत्यादि~। \\

परिमितव्यापिनो निर्गर्भस्य नाट्यायितस्योदाहरणं यथा बालरामायणे गर्भाङ्के सीतास्वयंवरे~।\\ 

एवं तावन्नाट्यरूपकनिष्टं नाट्यायितं व्याख्यातम्~। कार्यान्तरनिष्ठं' तूच्यते~। इह यदाभ्यन्तररसाविष्टता भवति तदा ध्रुवायोगाभिनयः स्वतुल्यतामापाद्यमानः परस्परमिलिताकारकतां काकतालीयेनोपनिपातात् (संभाव्यते)~। \begin{quote}
 {\qt यथा\textendash\ नळिनीदळए णीसहसुकदेहिं आतथा मुच्चइ~।\\ 
 पळइ विअब्भइ विज्जइ हंसी णळिणीवणे वि णत्थिज्जइ~॥ }
\end{quote}

\small{इत्यादौ~। तत्र हि प्रयोक्तुरेवमभिसन्धिध्रुवामभिनयेन दर्शयामीति~। किं तु प्रासादिक्यध्रुवायां गीयमानायां, "यत्र काव्येन (वाक्येन?)नोक्तं स्यात् तत्तु गीतं प्रस~।धयेत्"' (अ ३२) इति वचनात् ध्रुवार्थस्तत्नोचित आघातः, प्रयोगो हि बहुविधां मदनावस्थां नाटयतीति~। एवं भूतोऽङ्कुरस्वभावः पौर्वापर्यपर्यालोचनवशात् तथाभूत एवोपनिपतित इति, अप्रयुज्यमानापि (ध्रुवा) काकतालीयेन प्रयोगमुपांशुरूपा नाट्यमपि नाट्यमिव शासत इति तथाविधनाट्यायितत्वापादकः शारीराभिनयो नाट्यायितमिति दर्शयति स्थाने धुवास्वभिनयो यः क्रियत इति~। भावैर्व्यभिचारिभिः रसैः खस्थोिभिः ये संप्रयुक्ता आविष्टाः तत्संपादनैकमनसः प्रयोक्तारस्तैर्यो ध्रृवास्विति ध्रवार्थविषयोऽभिनयः क्रियते~। कथं, स्थाने प्रसङ्गे सति काकतालीयवशादित्यर्थः~। योऽभिनयः शारीरो नाट्यायितम्~। ननु किं प्रतिपदमभिनयता, नेत्याह हर्षादिभिरिति तत्सूचकैरङ्गोपाङ्गसत्त्वैरित्यर्थः~। तदपीति न केवलं पूर्व यावदिदमपीति~।}

\newpage
% १७४ नाट्यशास्त्रम् 

\begin{quote}
 {\na \renewcommand{\thefootnote}{1}\footnote{ड \textendash\ यस्तु भ \textendash\ यच्च}यत्रान्योक्तं वाक्यं सूचाभिनयेन योजयेदन्यः~। \\
\renewcommand{\thefootnote}{2}\footnote{ब \textendash\ तं य \textendash\ तत्संबद्ध}तत्संबन्धार्थं\renewcommand{\thefootnote}{3}\footnote{ड \textendash\ कृतं निवृत्त\textendash\ मेवाङ्कुरं विद्यात् }कथं भवेन्निवृत्त्यङ्कुरः सोऽथ~॥ ५० 

\renewcommand{\thefootnote}{4}\footnote{च \textendash\ एतेषां च स्मृता मार्गा\ldots .न्विताः \ldots निर्दिष्टाः.. त्मकाः\ldots ज \textendash\ एतेमार्गास्तु विज्ञेया..न्विताः.. निर्दिष्टाः\ldots त्मकाः }ऐतषां तु भवेन्मार्गो यथाभावरसान्वितः~। }
\end{quote}

\hrule

\vspace{2mm}

\begin{quote}
 {\qt यत्रान्योक्तं वाक्यं सूचाभिनयेन योजयेदन्यः~। \\
तत्संबन्धार्थकथं भवेन्निवृच्यङ्कुरः सोऽथ ~॥}
\end{quote}
 
अन्योक्तं वा्क्यं कथमन्यसूचाभिनये चित्तवृत्तिसूचकेनाङ्गोपाङ्गसत्त्वक्रमेण दर्शयेदित्याशङ्क्य हेतुमाह \underline{तत्संबन्धकथमिति} बीजादेर्निवृत्तिं यथाङ्कुरः सूचयति, एवं निवृत्ते वाक्ये (तदङ्कुर)यति निवृत्ङ्यकुर उक्तः~। तथा हि विदुषकेण वत्सराजे " अवि सुहयदि दे लोआणाणां ' इति पृष्टे सागरिका" सच्चं जीविदमरणाणं अन्तरं वट्टामि" इति, ततो राजा\textendash\ " सुखयतीति किमुच्यते~। कृच्छ्रेणोरूयुगं व्यतीत्य सुचिरं " इत्यादि पठति~। तस्मिन् क्रमेणाकर्ण्यमाने सागरिकाया यथाभूत(संशयोत्कण्ठारागोदयजनितो) व्यभिचारिसत्त्वयोजितः सत्त्वाङ्गोपाङ्गपरिस्पन्दो दृश्यमानो निवृत्त्यङ्कुरः~। (निवृत्त्यङ्कुरो) नाट्यायितं च वासवदत्तानाट्यधारे\renewcommand{\thefootnote}{*}\footnote{नाट्यधारशब्दो नाट्यापार इति नाट्यसार इति च इृश्यते~। नाट्यपार एव साधुः स्यात्~। } प्रतिपदं दृश्यते~। एतेषां च सर्वाभिनयैः सम्भूय वत्तित्वात् सर्वत्र चाभिनेये प्रायशः सद्भावात् सामान्याभिनयत्वं तदर्थमेव च वितत्यैतत्स्वरूपाभिधानम्, यत्पूर्वमुक्तम्\textendash\ 

\begin{sloppypar}
\begin{quote}
 {\qt अस्य शाखा च नृत्तं च तथैवाङ्कुर एव च~। 
त्रिविधं वस्त्वभिनयः\ldots \ldots \ldots \ldots \ldots ~। }
\end{quote}
\end{sloppypar}

इति तेन सहास्य यथा न विरोधस्तथैवोपपादितमुपाङ्गाभिनय इति तत एवावधार्यम्~। किं पुनरुक्ताभिधानेन~। \\

\begin{sloppypar}
एवमाङ्गिकं सामान्याभिनयमुपपाद्य वाचकमुपपादयति \underline{एतेषां तु भवेन्मार्ग} इति विषय इत्यर्थः~। वाक्यभावे यद्यप्यात्मापि शरीरो निर्विषय एव तेन यदेके शाखाङ्कुरनाट्यायितानां च वाक्यविरहितत्वं मन्यमाना एतेषामिति
\end{sloppypar}
 
\newpage
% द्वाविंशोऽध्यायः १७५ 

\begin{quote}
 {\na \renewcommand{\thefootnote}{1}\footnote{1 भ \textendash\ काव्यवस्तुविनिर्दिष्टाः म \textendash\ काव्यबन्धेषु न \textendash\ कार्याः }काव्यवस्तुषु निर्दिष्टो\renewcommand{\thefootnote}{2}\footnote{2 ढ निर्विष्टाः } द्वादशाभिनयात्मकः~॥ ५१ 

आलापश्च प्रलापश्च विलापः \renewcommand{\thefootnote}{3}\footnote{3 ज \textendash\ च ड \textendash\ अन्यः }स्यात्तथैव च~।\\ 
अनुलपोऽथ संलापं \renewcommand{\thefootnote}{4}\footnote{4 ड \textendash\ ह्यप }स्त्वपलापस्तथैव च~॥ ५२ 

सन्देशश्चातिदेशश्च निर्देशः स्यात्तथापरः\renewcommand{\thefootnote}{5}\footnote{5 ड \textendash\ च तथैवच}~। \\
उपदेशोऽपदेशश्च व्यपदेशश्च कीर्तितः ~॥ ५३ 

\renewcommand{\thefootnote}{6}\footnote{6 य \textendash\ आभाषणे }आभाषणं तु यद्वाक्यमालापो नाम स स्मृतः~।\\ 
अनर्थकं वचो यत्तु\renewcommand{\thefootnote}{7}\footnote{7 न . यच्च} प्रलापः स तु कीर्तितः\renewcommand{\thefootnote}{8}\footnote{8 च \textendash\ स प्रलापश्च कीर्तितः ड \textendash\ संज्ञितः भ\textendash\ सप्रलापः प्रकीर्तितः}~॥ ५४ }
\end{quote}

\hrule

\vspace{2mm}

\noindent
सर्वेषामित्यादि वाक्यसूचानिवृत्त्यङ्कुरमात्रविषयत्वेनैव संकोचयन्ति, ते न तत्त्वज्ञाः, सर्वोऽप्यभिनयो वाक्योपजीवनमन्तरेण नियमहेत्वभावादसमञ्जसतामभ्येति~। केवलं तत्कालिकातत्कालिकादिमात्रेण वाक्यं भिद्यतां नाम~। एतच्चोपाङ्गाभिनये वितत्योपपादितम्~। 

\underline{काव्यवस्तुष्विति} दशरूपकभेदेषु द्वादशरूपोऽभिनयात्मको वाचिकाभिनयस्य भाव इत्यर्थः~।द्वादशप्रकारानुद्दिशति आलापश्चेत्यादिना तानेव क्रमेण लक्षयति आभाषणं त्वित्यादिना~। युष्मदर्थविषयमुपदेशादिशून्यं यद्वचनं तदालाप इत्यर्थः~। ' विभ्राजसे मकरकेतनमर्चयन्ती ' (रत्ना\textendash\ १), यथा वा " जयतु भवान् " इत्यादि~। अनर्थकं वचो यत्तु स प्रलाप इति परस्परमसम्बद्धं मौर्ख्यादिवशादित्यर्थः~। यथा दरिद्रचारुदत्ते\renewcommand{\thefootnote}{*}\footnote{ दरिद्रचारुदत्तमिति मृच्छकट्यां प्रथमाङ्कस्य नाम } शकारः\textendash\ " शुणामि मल्लगन्धं, अन्धाळशच्चिदादो उण णासिआदो हिदुत्तं एळामि (श्रृणोमि माल्यगन्धम् , अन्धकारसंचितया पुनर्नासिकया (तदुक्तं) आलोक\textendash\ 

\newpage
% १७६ नाट्यशास्त्रम् 

\begin{quote}
 {\na \renewcommand{\thefootnote}{1}\footnote{च \textendash\ दुःखशोकोद्भवं यत्तु न \textendash\ करुणप्रभवं यत्तु}करुणप्रभवो यस्तु विलापः स तु कीर्तितः\renewcommand{\thefootnote}{2}\footnote{ड \textendash\ इति स स्सृतः }~। \\
बहुशोऽभिहितं वाक्यमनुलाप इति स्मृतः\renewcommand{\thefootnote}{3}\footnote{ड \textendash\ प्रकीर्तितः} ~॥ ५५ 

\renewcommand{\thefootnote}{4}\footnote{न \textendash\ उक्तप्रत्युक्त}उक्तिप्रत्युक्तिसंयुक्तः संलाप इति कीर्तितः~।\\ 
पुर्वोक्तस्यान्यथावादो ह्यपलाप इति स्मृतः ~॥ ५६ 

\renewcommand{\thefootnote}{5}\footnote{च \textendash\ त्वमिदं}तदिदं वचनं ब्रूहीत्येष सन्देश उच्यते~।\\ 
\renewcommand{\thefootnote}{6}\footnote{6 च अतिदेशस्त्वयोक्तं यत्तन्मयोक्तमिति स्मृतः~। }यत्त्वयोक्तं मयोक्तं तत्सोऽतिदेश इति स्मृतः ~॥ ५७ }
\end{quote}

\hrule

\vspace{2mm}

\begin{sloppypar}
\noindent
यामि) (मृच्छ\textendash\ १)~। \underline{करुणप्रभवो यस्तु स विलाप इति~।} करुणग्रहणं दुःखाेपलक्षणम् , तेन विप्रलम्भो५पि गृह्यते~। तत्र करुणरससंबन्धं वचनं, (दुःखे) यथा\textendash\ क्वसि प्रयच्छ मे प्रतिवचनम् ,विप्रलम्भे यथा\textendash\ बाणाः पञ्चमनोभवस्य(रत्ना \textendash\ ३) इत्यादि~। एतत्प्राधानादेव कामावस्थाविशेषो विलापः (तासु षष्ठ्यवस्था) इति वक्ष्यते~।\underline{बहुशोऽभिहितं वाक्यमनुलाप इति}~। अनुवादार्थ तदेव पुनरुच्यमानमित्यर्थः~।यथा\textendash\ द्वीपादन्यस्मादपि" (रत्ना १) इति सूत्रधारेणोक्ते नेपथ्ये यौगन्धरायणः\textendash\ एवमेतत्, दीपादन्यस्मादपीत्यनुवदति~। उक्तिप्रत्युक्तिसंयुक्तः संलाप इति~। यद्विधान्युदाहरणानि (?)~।\underline{पूर्वोक्तस्यान्यथा वादोऽप्यपलाप इति~।} यथा कृत्यारावणे गौतमीरूपच्छन्ना रामाक्रन्दितं लक्ष्मणे श्रवयितुकामा शूर्पणखा पूर्वमाह \textendash\ "अवि सुदं त '(अपि श्रुतं त्वया)~। ततः सीता (ससंभ्रमम्)` अये किंति " (अये किमिति)~। ततः सा\textendash\ ' अं वञ्चिते, सत्यं गौतमीमेव मामियं सीता जानात्वित्येवं ह्याह, " णं मए णक्क हण्णिदं, अवि सुदं ते" (इति)~। \underline{तदिदं वचनं ब्रूहीत्येष सन्देश इति~।}उदाहरणेन लक्षणमुन्नेयम्~। ततः परमुखेनान्यस्य स्ववचोऽपणं सन्देश इति~। अतिदेशस्त्वयोक्तं युक्तं मयोक्तमिति सिद्धेनासिद्धस्य तुल्यतापादनमतिदेश इत्यर्थः~। अत्रोपदेशातिदेशयोरुपमानस्य च साहित्यविषये तार्किकमीमांसकविषये विशेषप्रतिपादनं यत् टीकाकारैः कृतं तत्सुकुमारमनोमोहनं वृथा 
\end{sloppypar}

\newpage
% द्वाविंशोऽध्यायः १७७ 

\begin{quote}
 {\na \renewcommand{\thefootnote}{1}\footnote{ज \textendash\ सोऽयं ब्रवीमि यदहं निर्देशः स तु संज्ञितः (ढ\textendash\ स एकोऽहं ब्रवी\textendash\ मीति\ldots .) }स एषोऽहं ब्रवीमीति निर्देश इति कीर्तितः~। \\
व्याजान्तरेण कथनं\renewcommand{\thefootnote}{2}\footnote{भ \textendash\ करुणं } व्यपदेश इहोच्यते\renewcommand{\thefootnote}{3}\footnote{म \textendash\ प्रकीर्तितः ड \textendash\ भवेत्तु सः} ~॥ ५८ 

इदं कुरु गृहाणेति ह्युपदेशः प्रकीर्तितः\renewcommand{\thefootnote}{4}\footnote{ड \textendash\ इदमुप\textendash\ देश इति स्मृतः }~।\\ 
अन्यार्थकथनं यत् स्यात्\renewcommand{\thefootnote}{5}\footnote{ज \textendash\ तु } सोऽपदेशः प्रकीर्तितः\renewcommand{\thefootnote}{6}\footnote{ज \textendash\ इति स्मृतः}~॥ ५९

एते मार्गास्तु\renewcommand{\thefootnote}{7}\footnote{भ \textendash\ हि } विज्ञेयाः \renewcommand{\thefootnote}{8}\footnote{भ \textendash\ वाक्य }सर्वाभिनययोजकाः~।\\ 
सप्त\renewcommand{\thefootnote}{9}\footnote{च प्रकारास्तेषां च पुनर्वक्ष्यामि तत्त्वतः}प्रकारमेतेषां पुनर्वक्ष्यामि लक्षणम्\renewcommand{\thefootnote}{10}\footnote{न \textendash\ तत्त्वतः} ~॥ ६० 

प्रत्यक्षश्च परोक्षश्च तथा कालकृतास्त्रयः\renewcommand{\thefootnote}{11}\footnote{ज \textendash\ कृताश्च यः }~। \\
आत्मेस्थश्व परस्थश्र प्रकाराः सप्त एव तु\renewcommand{\thefootnote}{12}\footnote{ज \textendash\ चैव तु भ \textendash\ कीर्तिताः } ~॥ ६१}
\end{quote}

\hrule

\vspace{2mm}

\noindent
भ्रमणिकामात्रं प्रकृतानुपयोगादिहोपेक्ष्यमेव~।स एवैषोऽहं ब्रवीमीति निर्देश इति लक्षणम्~। \underline{व्याजान्तरेण कथनं} व्यपदेश इति व्याजविशेषेणेत्यर्थः~। यथा मुद्राराक्षसे क्षपणकस्य राक्षसदूषणार्थं तावन्नगरान्निर्वासनं तत्र च चाणक्येन व्याजेन वचनं कृत्य्\textendash\ अयं पापीयान् (जीवसिद्धिः) राक्षसप्रयुक्तविषकन्यया पर्वतेश्वरं घातितवान् ततो निर्वास्यते " इति~। \underline{इदं कुरु गृहाणेति ह्युपदेश} इति नियोग इत्यर्थः~। अन्यार्थकथनमपदेश इति स्वयं विवक्षितस्यान्य एव वक्तीत्यन्यकथनमित्यर्थः~। यथा भीमं प्रति सहदेवः\textendash\ एवं गुरुणा सन्दिष्टं सुयोधनस्येति (वेणी\textendash\ १)~। उपसंहरति \underline{एते मार्गास्त्विति सर्वेषु}षट्स्वपि शारीरेष्वित्यर्थः~। तथा अभिनीयन्त इत्यभिनया नाटककादिकाव्यविशेषाः तेषु, यतः सामान्येन भवन्त्यत एवैते सामान्याभिनया इति तात्पर्यम्~।

\lfoot{23} 

\newpage
% १७८ नाट्यशास्त्रम् 

\lfoot{}

\begin{quote}
 {\na \renewcommand{\thefootnote}{1}\footnote{एष इत्यादिश्लोकचतुष्टयस्य भ \textendash\ मातृकायां पाठभेदः\textendash\ कृतं मया करिष्येऽहं करोमीति च यद्वचः~। भूतं भवद् भविष्यच्च तदात्मस्थमुदाहृतम्~। स करोति कृतं तेन करिष्यति च यद्वचः~। भवद्भुतं भविष्यच्च परोक्षं परसंस्थितम्~। एष चक्रे करोत्येष करिष्यति\textendash\ च यद्वचः~। भूतं भवद्भविष्यं च प्रत्यक्षं परसंस्थितम्}एष ब्रवीमि\renewcommand{\thefootnote}{2}\footnote{ज \textendash\ ब्रवीति} नाहं भो वदामीति च यद्वचः~। \\
प्रत्यक्षश्च परोक्षश्च\renewcommand{\thefootnote}{3}\footnote{ज \textendash\ परस्थः} वर्तमानश्च तद्भवेत्~॥ ६२ 

अहं करोमि गच्छामि वदामि वचनं तव~।\\ 
\renewcommand{\thefootnote}{4}\footnote{च \textendash\ प्रत्यक्षे आत्मसंख्यश्च वर्तंमानश्च }आत्मस्थो वर्तमानश्च प्रत्यक्षश्चैव स स्मृतः~॥ ६३ 

करिष्यामि गमिष्यामि वदिष्यामीति यद्वचः~। \\
आत्मस्थश्च परोक्षश्च भविष्यत्काल एव च~॥ ६४ 

हता जिताश्च भग्नाश्च मया सर्वे द्विषद्गणाः~। \\
आत्मस्थश्च परोक्षश्च वृत्तकालश्च\renewcommand{\thefootnote}{5}\footnote{ड \textendash\ परस्थश्च वृत्तकालस्तु} स स्मृतः~॥ ६५ 

[\renewcommand{\thefootnote}{6}\footnote{एतच्छलोकचतुष्ट्यं च \textendash\ मातृकायामेव विद्यते}त्वया हता जिताश्रेति यो वदेन्नाव्यकर्मणि~।\\ 
परोक्षश्च परस्थश्च वृत्तकालस्तथैव च~॥ ६६ 

एष ब्रवीमि कुरुते गच्छतीत्यादि यद्वचः~।\\ 
\renewcommand{\thefootnote}{7}\footnote{ड \textendash\ आत्भस्थश्च परस्थश्च वर्तमानश्च स स्मृतः }परस्थो वर्तमानश्च (प्रत्यक्षश्च) भवेत्तथा~। ६७ }
\end{quote}

\hrule

\vspace{2mm}

अथात्रैव भेदान्तराण्याह आत्मस्थश्च परस्थश्चेत्यादि~। अत्रोदाहरणदिशमाह अहं करोमीत्यादि~। 
\newpage
% द्वाविंशोऽध्यायः १७९

\begin{quote}
 {\na सं गच्छति करोतीति वचनं यदुताहृतम्~। \\
परस्थं वर्तमानं च परोक्षं चेव तद्भवेत्~॥ ६८ 

करिष्यन्ति गमिष्यन्ति वदिष्यन्तीति यद्वचः~।\\ 
परस्थमेष्यत्कालं च परोक्षं चैव तद्भवेत् ] ~॥ ६९ 

हस्तमन्तरतः कृत्वा यद्वदेन्नाट्यकर्मणि~।\\ 
\renewcommand{\thefootnote}{1}\footnote{च \textendash\ मातृकायामयं श्लोकः " परेषां " इति श्लोकानन्तरं वर्तते }आत्मस्थं हृदयस्थं च परोक्षं चैव तन्मतम्~॥ ७० 

परेषामात्मनश्चैव कालस्य च विशेषणात्\renewcommand{\thefootnote}{2}\footnote{च \textendash\ विपर्ययात् }~।\\ 
सप्तप्रकारस्यास्यैव\renewcommand{\thefootnote}{3}\footnote{च\textendash\ प्रकारास्तस्यैव } भेदा ज्ञेया \renewcommand{\thefootnote}{4}\footnote{य \textendash\ ह्यनेकधा भ \textendash\ ह्यनेकशः }अनेकधा ~॥ ७१ }
\end{quote}

\hrule

\vspace{2mm}

इदानीं विषयभेदकृतमात्मस्थस्यापि पारोक्ष्यं दर्शयति \underline{हस्तमन्तरतः कृत्वेत्यादि~।} स्वगतजनान्तिकापवारितकेषु वक्तुरात्मस्थं पात्रान्तराणां चाप्रत्यक्षं नाट्यधर्मीवशादित्यात्मस्थमपि तत्परोक्षम्~। लोकधर्म्याप्यात्मस्थं परोक्षम्~। यथा\textendash\ `' सुप्तो मत्तो न्वहं किल विललाप " इति~। यदाहोत्तमविषयेऽपि चित्तव्याक्षेपादिभ्यो (लिट् भवतीति~। अन्य)भेदानां कार्येण (सम्भव)~। माह\textendash\ 

\begin{quote}
 {\qt परेषामात्मनश्चैव कालस्य च विशेषणात्~। \\
सप्तप्रकारस्यास्यैव भेदा ज्ञेया अनेकधा~॥ 

एते प्रयोगा विज्ञेया मार्गाभिनययोजिताः~।\\ 
एतेष्विह विनिष्पन्नो विविधोऽभिनयो भवेत्~॥ }
\end{quote}

इह तावद्वाक्यं द्वादशधाभिन्नं तत्राप्यात्मस्थपरस्थयोः प्रत्यक्षत्वपरोक्षत्वाभ्यां चतुर्विधत्वं चतुर्णां कालत्रयेण गुणने द्वादशभेदाः~। भूतभविष्यतो\textendash\ 

\newpage
% १८० नाट्यशास्त्रम् 

\begin{quote}
 {\na \renewcommand{\thefootnote}{1}\footnote{श्लोकार्धं चभ \textendash\ मातृकयोर्न दृश्यते }एते \renewcommand{\thefootnote}{2}\footnote{ड \textendash\ प्रयोक्तृभिः }प्रयोगा विज्ञेया \renewcommand{\thefootnote}{3}\footnote{ड \textendash\ मार्गा ह्यभिनये स्मृताः ड \textendash\ मार्गा ह्यभिनयाश्रयाः (न \textendash\ त्मिकाः) }मार्गाभिनययोजिताः~। \\
\renewcommand{\thefootnote}{4}\footnote{न \textendash\ एभिरेव ब \textendash\ एभिर्मार्गो, भ \textendash\ एभिर्मार्गैः }एतेष्विह विनिष्पन्नो \renewcommand{\thefootnote}{5}\footnote{भ \textendash\ विविधाभिनयो}विविधोऽभिनयो भवेत्\renewcommand{\thefootnote}{6}\footnote{ड \textendash\ मतः } ~॥७२ 

शिरो\renewcommand{\thefootnote}{7}\footnote{भ \textendash\ वदनपाण्यूरुजङ्घोदरकटीगतः~। समकर्मविभागो यो विविधाभिनयस्तु सः ज \textendash\ वदनहस्तोरुजङ्घोदरकटीगतः~। समकर्मविपाको यः सामान्याभिनयः (ड \textendash\ पाद\ldots .कृतः) च \textendash\ वदनहस्तोरः कटयूरुचरणाश्रयः~। समः कर्मविभागे यो विविधाभिनयः}हस्तकटीवक्षोजङ्घोरुकरणेषु तु~।\\ 
समः कर्मविभागो यः सामान्याभिनयस्तु सः ~॥ ७३ 

ललितैर्हस्तसंचारै\renewcommand{\thefootnote}{8}\footnote{च \textendash\ विन्यासैः}स्तथा मृद्वङ्गचेष्टितैः~। \\
अभिनेय\renewcommand{\thefootnote}{9}\footnote{च \textendash\ नेयं\ldots ..समन्वितम्~। }स्तु नाट्यज्ञै रसभावसमन्वितैः ~॥ ७४ }
\end{quote}

\hrule

\vspace{2mm}

\noindent
रपि प्रत्यक्षत्वं योगिप्रत्यक्षादिदृशा न भवति~। एवं द्वादशानां तावद्भिर्गुणने चतुश्चत्वारिंशदधिकं शतम्~। संस्कृतेतरभेदेन त्रिविधत्वं यदुक्तं तेन गुणने द्विपञ्चाशदधिकानि नवशतानीति (!) वाक्याभिनयस्य भेदाः~। \underline{सूचाया} वाक्यतदर्थभेदाद्द्वै गुण्यम्~। तेन एकोनविंशतिशतानि चतुरधिकानि सूचा\textendash\ भेदाः~। वाक्यतुल्या एवाङ्कुरभेदाः सूचाभेदैर्गुणनीयास्तैः, पुनः शाखाभेदा\textendash\ स्तैरपि द्विविधनाट्यायितभेदस्तावद्भिर्निवृत्त्यङ्कुरभेदा इति कोटिशतान्यनेकानि भवन्ति~। न तु यथा श्रीशङ्कुकेनोक्तं चत्वारिंशत्सहस्राणीत्यादि~। \\

एवं विशिष्टः सामान्येनाभिनीयमानः संभूयाभिनयैर्युक्तः सर्वाभिनयेषु सामान्यभूत इत्येवं यः सामान्याभिनय अस्या एकीभावनिबन्धनभूताया अलातचक्र[ मण्डल ]संनिभत्वसम्पादिकाया सामान्याभिनयक्रियायाः 

\newpage
% द्वाविंशोऽध्यायः १८१ 

\begin{quote}
 {\na \renewcommand{\thefootnote}{1}\footnote{1 च \textendash\ अनुद्भटं भ \textendash\ अनुद्भटमसंक्रान्तं}अनुद्धतमसंभ्रान्तमनाविद्धाङ्गचेष्टितम्~।\\ 
लयतालकलापात\renewcommand{\thefootnote}{2}\footnote{च \textendash\ काल ड \textendash\ कलापादि}प्रमाणनियतात्मकम्\renewcommand{\thefootnote}{3}\footnote{ड \textendash\ नियमात्मकः ढ \textendash\ आत्मजात् भ \textendash\ कालायातप्रमाणनियमान्वितम् } ~॥ ७५ 

सुविभक्तपदालापमनिष्ठुरमकाहलम्\renewcommand{\thefootnote}{4}\footnote{ड \textendash\ अना\textendash\ कुलम्}~।\\ यदीदृशं भवेन्नाट्यं ज्ञेयमाभ्यन्तरं\renewcommand{\thefootnote}{5}\footnote{ढ \textendash\ अभ्यन्तरं} तु तत् ~॥ ७६ 

एतदेव विपर्यस्तं स्वच्छन्दगतिचेष्टितम्~। \\
\renewcommand{\thefootnote}{6}\footnote{म \textendash\ अनुबद्ध च \textendash\ अति}अनिबद्धगीतवाद्यं नाट्यं बाह्यमिति स्मृतम् ~॥ ७७ 

\renewcommand{\thefootnote}{7}\footnote{च \textendash\ लक्षणाभ्यन्तरं यस्मात्तस्मादाभ्यन्तरं स्मृतम्}लक्षणाभ्यन्तरत्वाद्धि \renewcommand{\thefootnote}{8}\footnote{ड \textendash\ तदभ्यन्तरं }तदाभ्यन्तरमिष्यते~। \\
\renewcommand{\thefootnote}{9}\footnote{च \textendash\ शास्त्रार्थबाह्यभावा (त्तु) बाह्यमित्यभिधीयते}शास्त्रबाह्यं भवेद्यत्तु तद्बाह्यमिति भण्यते\renewcommand{\thefootnote}{10}\footnote{ड \textendash\ संज्ञितम् भ \textendash\ विश्रुतम्} ~॥ ७८ 

अनेन लक्ष्यते यस्मात्\renewcommand{\thefootnote}{11}\footnote{ब \textendash\ यस्मिन्} प्रयोगः कर्म चैव हि\renewcommand{\thefootnote}{12}\footnote{भ \textendash\ वा बुधैः}~। \\
तस्माल्लक्षणमेतद्धि नाट्येऽस्मिन् संप्रयोजितम्\renewcommand{\thefootnote}{13}\footnote{भ \textendash\ समुदाहृतम् ~।} ~॥७९ }
\end{quote}

\hrule

\vspace{2mm}

\noindent
प्राधान्यप्रदर्शनार्थमाह \underline{अनुद्धतमसंभ्रान्तमित्यादि~।} अनाविद्धशब्देनाङ्गिक\textendash\ विषयं सामान्यं दर्शयति~। एतदभावं दुष्टत्वमिति दर्शयन् व्यतिरेकक्रमेणापि सामान्याभिनयस्यावश्योपादेयतामाह एतदेव विपर्यस्तमित्यादिना~। \underline{आभ्य}\textendash\ न्तरमिति (स्वयमेव विवृणोति) \underline{लक्षणाभ्यन्तरत्वादिति~।}ननु वागङ्गाभि\textendash\ नयोपेत इति लक्षणं तस्य बाह्योऽप्यस्तीत्याशङ्कयाह \underline{अनेन लक्ष्यते यस्मा}\textendash\ दिति सामान्याभिनयरूपमेव लक्षणमिति तात्पर्यम्~। \underline{(संप्रयोजितमिति)} नाट्यविषयाः प्रत्येकं तावत् क्रिया एकीभावं नेयाः, एकीभावगतश्च क्रियासमूहोऽप्येकीभावं नेय इत्येतत्~। \underline{कर्मप्रयोगशब्दाभ्यां नाट्य इति }

\newpage
% १८२ नाट्यशास्त्रम् 

\begin{quote}
 {\na अनाचार्योषिता\renewcommand{\thefootnote}{1}\footnote{ड \textendash\ उदिताः भ \textendash\ अनाचार्ये हिताः ब \textendash\ अनाचार्याहिताः } ये च ये च शास्त्रबहिष्कृताः\renewcommand{\thefootnote}{2}\footnote{2 च \textendash\ बहिर्गताः}~। \\
\renewcommand{\thefootnote}{3}\footnote{च \textendash\ बाह्यस्ते (न्ते) (?) तु प्रयोज्यन्ति क्रियामात्रैः प्रयोजिते य \textendash\ बाह्यं ते तु प्रयोक्ष्यन्तेक्रियामन्यैः प्रयोजिताम् (ढ \textendash\ हेतु\ldots .मन्त्रैः) }बाह्यं प्रयुञ्जते ते तु अज्ञात्वाचार्यकीं क्रियाम् ~॥ ८० 

शब्दं स्पर्शं च रूपं च रसं गन्धं तथैव च~।\\ 
\renewcommand{\thefootnote}{4}\footnote{जइन्द्रियैः}इन्द्रियाणीन्द्रियार्थांश्च \renewcommand{\thefootnote}{5}\footnote{भ \textendash\ भावेन }भावैरभिनयेद्बुधः ~॥ ८१ 

कृत्वा साचीकृतां दृष्टिं शिरः \renewcommand{\thefootnote}{6}\footnote{य \textendash\ पार्श्वानतं, प \textendash\ पार्श्वे }पार्श्वनतं तथा~। \\
\renewcommand{\thefootnote}{7}\footnote{ड \textendash\ तर्जनी कर्णदेशे तु शब्दं त्वभिनये द्बुधः }तर्जनीं कर्णदेशे च बुधः शब्दं विनिर्दिशेत्\renewcommand{\thefootnote}{8}\footnote{8 च \textendash\ तु बुधः शब्दान् नियोजयेत् } ~॥ ८२ }
\end{quote}

\noindent
सप्तम्या कथयति\textendash\ अनेन विना स्फुटं नाट्यरूपत्वमेव साक्षात्काराध्यव\textendash\ सायरूपं रसानुप्राणितं तन्न संपद्यत इति~। व्युत्पत्तिदूरीभावं त्वनधीयानां, भट्टपुत्रादौ तदभावात्तदाह अनाचार्योषिता ये चेत्यादिना~। \\

\begin{sloppypar}
अथ संभूयाभिनयरूपत्वमेव सामान्याभिनयमाह \underline{शब्दं स्पर्शं चेत्यादि~।} इन्द्रियशब्दस्येन्द्रियार्थशब्देन स्पर्शादिविशेषणस्यासंभवम्\renewcommand{\thefootnote}{*}\footnote{इन्द्रियस्येन्द्रियार्थेन शब्दस्पर्शादिविशेषेण स्यात् संभवः\textendash\ इति पाठः स्यात्~। }, स्वविषयग्रहणा\textendash\ वेशः स्वकरणग्राह्यतावेशश्च सर्वेषामिन्द्रियाणां विषयाणां च प्रदर्श्यते, अत एवैष सामान्याभिनयनेयेऽर्थद्वयेऽभिनयस्य साधारण्यम्~। \underline{भावैरिति} क्रिया\textendash\ विशेषैरित्यर्थः~। तानाह \underline{कृत्वा साचीकृतां दृष्टिमित्यादि~।} अनेनादरवशा\textendash\ त्समस्तो भरः श्रोत्रदेशमनुयाति, यथोक्तं\textendash\ 
\end{sloppypar}


\begin{quote}
 {\qt तथाहि शेषेन्द्रियवृत्तिरासां \\
सर्वात्मना चक्षुरिव प्रविष्ठा ~॥ (रघु ७\textendash\ १२) इति}
\end{quote} 
 
चकाराच्छ्रोत्रमपि शब्दग्रहणाविष्टमनेनाभिनीतं भवतीत्याह~। एव\textendash\ मुत्तरत्र ~। 

\newpage
% द्वाविंशोऽध्यायः १८३ 

\begin{quote}
 {\na किंचिदाकुञ्चिते नेत्रे कृत्वा भ्रुक्षेपमेव च\renewcommand{\thefootnote}{1}\footnote{ज \textendash\ भ्रुवोरुत्क्षेपणेन च}~। \\
तथाऽसगण्डयोः स्पर्शात् स्पर्शमेवं विनिर्दिशेत् ~॥ ८३

कृत्वा पताकौ मूर्धस्थौ किंचित्प्रचलिताननः\renewcommand{\thefootnote}{2}\footnote{ड \textendash\ अङ्गुलिः}~।\\
निर्वर्णयन्त्या दृष्टया च\renewcommand{\thefootnote}{3}\footnote{ड \textendash\ तु } रूपं त्वभिनयेद् बुधः ~॥ ८४ 

किंचिदाकुञ्चिते नेत्र \renewcommand{\thefootnote}{4}\footnote{य \textendash\ कृत्वा फुल्लां}कृत्वोत्फुल्लां च नासिकाम्\renewcommand{\thefootnote}{5}\footnote{भ \textendash\ नाडिकाम }~। \\
एकोच्छवासेन चेष्टौ तु \renewcommand{\thefootnote}{6}\footnote{न \textendash\ चेदिष्टौ च \textendash\ चोद्दिष्ठौ ड \textendash\ हृष्टेष्टौ भ\textendash\ सहोच्छवासे चेष्टौ तु}रसगन्धौ विनिर्दिशेत् ~॥८५ 

पञ्चानामिन्द्रियार्थानां\renewcommand{\thefootnote}{7}\footnote{ब \textendash\ इन्द्रियाणां च} भावा ह्येतेऽनुभाविनः~।\\ 
\renewcommand{\thefootnote}{8}\footnote{भ\textendash\ श्रोत्रस्य तु तथैव च (च \textendash\ च हि,) (म \textendash\ जिह्वा = जिंह्वा) }श्रोत्रत्वङ्नेत्रजिङ्वानां घ्राणस्य च तथैव हि ~॥ ८६ 

\renewcommand{\thefootnote}{9}\footnote{प \textendash\ इन्द्रियार्थांश्च मनसो भावय\textendash\ न्त्यनुभाविनः }इन्द्रियार्थाः समनसो भवन्ति ह्यनुभाविनः\renewcommand{\thefootnote}{10}\footnote{भ \textendash\ भवन्तीह विभाविनः}~।\\ 
न वेत्ति ह्यमनाः किंचिद्विषयं पञ्चधागतम्\renewcommand{\thefootnote}{11}\footnote{ड \textendash\ पञ्चहेतुकम् भ \textendash\ भूतजम्} ~॥ ८७ 

\renewcommand{\thefootnote}{12}\footnote{भ \textendash\ मानसः }मनसस्त्रिविधो भावो विज्ञेयोऽभिनये बुधैः\renewcommand{\thefootnote}{13}\footnote{य \textendash\ अभिनयो बुधैः, ड \textendash\ नयं प्रति~। }~। }
\end{quote}

\hrule

\vspace{2mm}

\begin{sloppypar}
एतदुपसंहरति पञ्चानामिन्द्रियार्थानामित्यादि~। \underline{इन्द्रियार्थाः समनस} इति यथाशब्दादिग्रहणक्रियाभिः शब्दः श्रोत्रे प्रतीयते, तन्मनोऽपि तदधि\textendash\ ष्ठातृक्रमेणाधिष्ठानं कुर्वदित्यतोऽपि सामान्याभिनयः~। तदुक्तम्\textendash\ युगपद् ज्ञा\textendash\ नानुत्पत्तिर्मनसो लिङ्गमिति\renewcommand{\thefootnote}{*}\footnote{न्यायसू \textendash\ १\textendash\ १\textendash\ १६. }~। \underline{मनसस्त्रिविधो भाव} इति वैशेषिकादिदृशि मनस्संयोगजो य आत्मन इच्छाद्वेषमाध्यस्थ्यलक्षणो भावः स मनस इत्युक्तः~। 
\end{sloppypar}

\newpage 
% १८४ नाट्यशास्त्रम् 

\begin{quote}
 {\na \renewcommand{\thefootnote}{1}\footnote{च \textendash\ इष्टोऽनिष्टश्च मध्यश्च तस्याभिनय उच्यते ड \textendash\ इष्टोऽनिष्टस्तथा चैव मध्यस्थश्च तथैव हि }इष्टस्तथा ह्यनिष्टश्च\renewcommand{\thefootnote}{2}\footnote{भ \textendash\ वा } मध्यस्थश्च तथैव हि ~॥ ८८ 

\renewcommand{\thefootnote}{3}\footnote{भ \textendash\ गात्रप्रह्लादनेनेह}प्रह्लादनेन गात्रस्य तथा पुलकितेन च~।\\ 
\renewcommand{\thefootnote}{4}\footnote{ज \textendash\ \underline{नितान्तप्रक्रि\textendash\ याभिश्च} सर्वमिष्टं निरूपयेत् भ \textendash\ नितान्तास्त (तास्य ?) ढ \textendash\ आननप्रक्रियाभिश्च }वदनस्य विकासेन कुर्यादिष्टनिदर्शनम् ~॥ ८९ 

\renewcommand{\thefootnote}{5}\footnote{च \textendash\ दृष्टे (?) }इष्टे शब्दे तथा रूपे स्पर्शे \renewcommand{\thefootnote}{6}\footnote{च \textendash\ घ्राणे}गन्धे तथा रसे~। \\
इन्द्रियैर्मनसा\renewcommand{\thefootnote}{7}\footnote{भ \textendash\ मनसि } प्राप्तैः\renewcommand{\thefootnote}{8}\footnote{च \textendash\ मनसि प्राप्ते} \renewcommand{\thefootnote}{9}\footnote{ब \textendash\ सौख्यं संप्रति दर्शयेत् }सौमुख्यं संप्रदर्शयेत् ~॥ ९० 

परावृत्तेन शिरसा\renewcommand{\thefootnote}{10}\footnote{भ \textendash\ तथा पातेन चक्षुषः ~। नेत्रत्रासाञ्चिततया ड \textendash\ प्रदानेन च चक्षुषः~। नेत्रत्रासाञ्चिततया} नेत्रनासाविकर्षणैः\renewcommand{\thefootnote}{11}\footnote{य \textendash\ निकूणनैः~। }~।\\
चक्षुषश्चाप्रदानेन ह्यनिष्टमभिनिर्दिशेत् ~॥ ९१ 

\renewcommand{\thefootnote}{12}\footnote{भ \textendash\ न चाति\textendash\ मात्रं हृष्टेन न जुगुप्साकृतेन तु ड न चातिमात्रहृष्टस्तु न चात्यन्तजुगुप्सया }नातिहृष्टेन मनसा न \renewcommand{\thefootnote}{13}\footnote{च \textendash\ चात्यन्त }चात्यर्थजुगुप्सया~।\\ 
मध्यस्थेनैव भावेन मध्यस्थमभिनिर्दिशेत्\renewcommand{\thefootnote}{14}\footnote{भ \textendash\ मध्यस्थाभिनयः स्मृतः } ~॥ ९२ }
\end{quote}

\begin{sloppypar}
\noindent
कापिलदृशि तु विन्ध्यवासिनो मनस एव, ईश्वरकृष्णादिमते मनःशब्देनात्र बुद्धिः~। \underline{प्रह्लादनेन गात्रस्येत्यादिकं} विषयग्रहणक्रियास्वभीष्ठविषये निदर्श\textendash\ यितव्यम्~। अतोऽपि चास्य सामान्याभिनयत्वं द्रष्टव्यम्~।\\ 
\end{sloppypar}

एतदेव स्फुटयति \underline{दृष्टे(इष्टे ?)शब्द} इत्यादिना~। सुमुखत्वं प्रसादादि\textendash\ युक्तं वदनमित्यर्थः~। विकर्षणानि सङ्कोचनानि च~। बहुवचनेन मध्ये ग्रहण\textendash\ सिद्धये विकासासंभिन्न इति दर्शयति~। 

\newpage
% द्वाविंशोऽध्यायः १८५

\begin{quote}
 {\na तेनेदं तस्य \renewcommand{\thefootnote}{1}\footnote{ज \textendash\ च }वापीदं स एवं प्रकरोति वा\renewcommand{\thefootnote}{2}\footnote{च \textendash\ च }~। \\
परोक्षाभिनयो यस्तु मध्यस्थ इति स स्मृतः ~॥ ९३ 

\renewcommand{\thefootnote}{3}\footnote{इदमर्धं भ \textendash\ मातृकायां न दृश्यते }आत्मानुभावी योऽर्थः स्यादात्मस्थ इति स स्मृतः~।\\ 
\renewcommand{\thefootnote}{4}\footnote{च \textendash\ परार्थवर्णना याच (ज \textendash\ परस्य) ड \textendash\ परस्य वर्णनीयं च भ \textendash\ परार्थवर्णने यश्च परस्थः सोऽभिधीयते}परार्थवर्णना यत्र परस्थः स तु संज्ञितः ~॥ ९४ 

प्रायेण सर्वभावानां कामान्निष्पत्तिरिष्यते~।\\ 
स \renewcommand{\thefootnote}{5}\footnote{ज \textendash\ चेप्सा\ldots बहुधा काम इष्यते}चेच्छागुणसम्पन्नो\renewcommand{\thefootnote}{6}\footnote{भ \textendash\ निष्पन्नो } बहुधा परिकल्पितः ~॥ ९५ }
\end{quote}

\hrule

\vspace{2mm}

अथात्रैव पूर्वोक्तप्रकारसंभवं दर्शयति \underline{तेनेदमित्यादि~।} कृतं कर्तव्यमिति वाक्यशेषे भूतता भविष्यत्ता च~। आत्मनि सुखादयोऽर्थाः समवायिन इति सर्व एव ते आत्मस्थाः स्युः, रूपादीनां चान्यत्र समवायात् सदैव परस्थता स्यादि\textendash\ त्याशङ्कयाह \underline{आत्मानुभावी योऽर्थः स्यादिति~।}परशब्दसन्निधानादात्म शब्दोऽत्राहंभावास्पदे प्रत्यगात्मनि वर्तते~। तमात्मानमनुभावयति यदार्थः स आत्मस्थः~। रूपादयोऽपि चैवं भवन्तीति कथं नात्मस्थाः परसुखादयश्च नैव\textendash\ मिति कथमात्मस्थाः~।\\

\begin{sloppypar}
अथ कामोपचारस्य सामान्याभिनयत्वमुपपादयति \underline{प्रायेण सर्वभावाना}\textendash\ मिति~। \underline{कामादिति} इच्छातः~। यद्यप्यनिच्छोः किञ्चिद्भवति तदपि प्रक्तनं च कर्माधिपत्यात्~। कर्मस्थापूर्वकमिति प्रायग्रहणं व्याप्त्यर्थः कामादेव निष्प\textendash\ त्तिरित्यर्थः~। \\
\end{sloppypar}

ननु कोऽयं कामो नामेत्याह \underline{स चेच्छेति~।} न चेच्छामात्रादेव कार्यवि\textendash\ निष्पत्तिरित्याह गुणेन कार्यप्रयत्नादिना कार्यव्यापारादिसहितेन सम्पन्नः सहकृतः सर्वकार्यकारी~। नन्वेकैव चेदिच्छा कथमनेकं कार्यं प्रसूयत इत्या\textendash\ शङ्कयाह \underline{बहुधा परिकल्पितः} भूयस्यः इच्छा इति यावन्तमुदाहरति \underline{धर्मकाम }

\lfoot{24}

\newpage
page content missing 

\newpage
% द्वाविंशोऽध्यायः १८७ 

\lfoot{}

\begin{quote}
{\na \renewcommand{\thefootnote}{1}\footnote{य \textendash\ प्रायेण सर्व\ldots .नित्यशः च \textendash\ सर्वः प्रायेण ढ \textendash\ इह प्रायेण लोकोऽयं शुभमिच्छति नित्यशः }भूयिष्ठमेव लोकोऽयं सुखमिच्छति सर्वदा\renewcommand{\thefootnote}{2}\footnote{भ \textendash\ नित्यशः }~। \\
\renewcommand{\thefootnote}{3}\footnote{भ \textendash\ सुखमूलं स्त्रियश्चैव }सुखस्य हि स्त्रियो मूलं नाना\renewcommand{\thefootnote}{4}\footnote{च \textendash\ शीलधराश्च ताः}शीलाश्च ताः पुनः ~॥ ९९

\renewcommand{\thefootnote}{5}\footnote{य \textendash\ देवतासुर भ \textendash\ देवगन्धर्वदैत्यानां सयक्षोरगरक्षसाम्~। पिशाचपक्षि\ldots}देवदानवगन्धर्वरक्षोनागपतत्रिणाम्~।\\ 
पिशाचयक्षव्यालानां नरवानरहस्तिनाम् ~॥ १०० 

मृगमीनोष्ट्रमकर\renewcommand{\thefootnote}{6}\footnote{ढ \textendash\ वन }खरसूकरवाजिनाम्~।\\ 
\renewcommand{\thefootnote}{7}\footnote{ड \textendash\ महिषाश्व भ \textendash\ महिषप्रभृतीनां च }महिषाजगवादीनां तुल्यशीलाः स्त्रियः स्मृताः ~॥ १०१}
\end{quote}

\begin{sloppypar}
लक्षणं संभोगं करोतीत्याह~। उपचारोऽन्योन्यहृदयग्रहणोचितैर्व्यापारैः परि\textendash\ पूर्णः~। इह चोक्त(उत्तम)प्रकृतिर्यदि भवति तद्रसाध्यायोक्तदृशा श्रृङ्गार इत्यु\textendash\ च्यते~। एतमेवार्थमुपोद्बलयति भूयिष्ठमेव लोकोऽयं सुखमिच्छति सर्वदा इति~। भूयिष्ठमिति प्राप्तादधिकमिति यावत्~। अत एव परमानन्दलाभमन्त\textendash\ रेण न कुत्रचित् संतुष्यति लोकः~। सर्वदेति दुःखाभावमपि सुखार्थमेवेच्छति, सर्वदुःखनिवृत्तिं हि कामयते सुखेन यथेच्छमश्नीयात्, यथाभीष्टं च रमणी\textendash\ मुपभुञ्जीयादिति~। हारमपि त्यजति मृदुशीतलसमीरस्पर्शजनितसुखसिद्धयर्थ\textendash\ मेव~। अत एव शून्यरहस्यविदः कणादसुगतादिसंमतमिमं मोक्षं न रोचयन्ते प्रेक्षावतां तत्राप्रवृत्तिप्रसङ्गादिति दर्शितमित्यलं बहुना~। \\
\end{sloppypar}

\begin{sloppypar}
उपचारकृतमित्युक्तम् , तत्रोपचारज्ञानाय स्त्रीणामाशयं दर्शयति, आश\textendash\ यग्रहणपूर्वकत्वादुचितस्योपचारस्य नानाशीलाः (इति)~। शीलं सत्त्वं चैतन्यं बुद्धिपूर्वकं स्वभावो हेवाक इति पर्यायाः~। देवदानवेत्यादिना तानि शीलान्यु\textendash\ द्दिश्य यथोद्देशं लक्षयति~। आदि ग्रहणेनान्यदपि शीलमस्तीत्याह~। 
\end{sloppypar}

\newpage
% १८८ नाट्यशास्त्रम् 

\begin{quote}
 {\na \renewcommand{\thefootnote}{1}\footnote{1 ड \textendash\ स्निग्धा चाङ्गैः\ldots ..स्थिर च स्निग्धाङ्गोपाङ्गनयना भ \textendash\ स्निग्धस्वराङ्गो\textendash\ पाङ्गैश्च}स्निग्धैरङ्गैरुपाङ्गैश्च स्थिरा मन्दनिमेषिणी~। \\
अरोगा\renewcommand{\thefootnote}{2}\footnote{2 भ \textendash\ दीप्तियुक्ता च दानशील\ldots .} दीप्त्युपेता च \renewcommand{\thefootnote}{3}\footnote{3 च \textendash\ सत्यार्जवदयान्विता ज \textendash\ दानशक्त्यार्जवान्विता }दानसत्त्वार्जवान्विता ~॥ १०२ 

\renewcommand{\thefootnote}{4}\footnote{4 भ \textendash\ अपस्वेदा}अल्पस्वेदा समरता स्वल्पभुक् सुरत\renewcommand{\thefootnote}{5}\footnote{5 भ \textendash\ सुरभि}प्रिया~।\\ 
\renewcommand{\thefootnote}{6}\footnote{6 ड \textendash\ गान्धर्ववाद्या\textendash\ भिरता हृद्या देवाङ्गना }गन्धपुष्परता हृद्या देवशीलाङ्गना स्मृता ~॥ १०३ 

अधर्मशाठ्याभिरता\renewcommand{\thefootnote}{7}\footnote{7 भ \textendash\ निरता ब \textendash\ साध्यनिरता} स्थिरक्रोधातिनिष्ठुरा~।\\ 
मद्यमांसप्रिया\renewcommand{\thefootnote}{8}\footnote{8 प \textendash\ नित्य} नित्यं कोपना चातिमानिनी ~॥ १०४ 

\renewcommand{\thefootnote}{9}\footnote{9 भ \textendash\ वाचाला}चपला चातिलुब्धा च परुषा कलहप्रिया~।\\ 
\renewcommand{\thefootnote}{10}\footnote{10 च \textendash\ ईर्ष्याशीलाथ निस्नेहा शीलमासुरं }ईर्ष्याशीला चलस्नेहा चासुर \renewcommand{\thefootnote}{11}\footnote{11 भ \textendash\ सत्त्वं}शीलमाश्रिता ~॥१०५ 

\renewcommand{\thefootnote}{12}\footnote{12 भ \textendash\ सुनेत्रा कामवशगैः ज \textendash\ सुनेत्रा कामभोगा च ढ \textendash\ अनेकारामभोग्या च}क्रीडापरा चारुनेत्रा नखदन्तैः सुपुष्पितैः~। \\
\renewcommand{\thefootnote}{13}\footnote{13 ड \textendash\ स्मिताभिभाषिणी तन्वी मन्दाचारा न \textendash\ तन्वङ्गी स्मितभाषी भ \textendash\ तन्वङ्गी नित्यहृष्टा च मन्दापत्या मृजावती ज \textendash\ तन्वङ्गी स्मितभाषा च मन्दा गत्या मृदुस्तथा}स्वङ्गी च स्थिरभाषी च मन्दापत्या रतिप्रिया ~॥ १०६ 

\renewcommand{\thefootnote}{14}\footnote{14 भ \textendash\ गीतनृत्ते सदासक्ता विदग्धा सुरभिप्रिया }गीते वाद्ये च नृत्ते च \renewcommand{\thefootnote}{15}\footnote{15 ज \textendash\ नित्यं }रता हृष्टा मृजावती~। }
\end{quote}

\hrule

\vspace{2mm}

\underline{समरतेति} नातिमृद्वी नातिखरेत्यर्थः~। \underline{स्वङ्गीति} सुखसन्निवेशान्यङ्गानि यस्या इत्यर्थः~। 

\newpage
% द्वाविंशोऽध्यायः १८९ 

\begin{quote}
 {\na गन्धर्वसत्त्वा\renewcommand{\thefootnote}{1}\footnote{च \textendash\ शीला } विज्ञेया स्निग्धत्वक्केशलोचना ~॥ १०७ 

\renewcommand{\thefootnote}{2}\footnote{ड \textendash\ बृहदायत}बृहद्व्यायतसर्वाङ्गी रक्तविस्तीर्णलोचना~। \\
\renewcommand{\thefootnote}{3}\footnote{ड \textendash\ भूरि ढ \textendash\ हरि}खररोमा दिवास्वप्न4\renewcommand{\thefootnote}{4}\footnote{च \textendash\ स्वभावा ढ \textendash\ निवृत ड \textendash\ निवृत्त ज \textendash\ नियत भ \textendash\ नित्यमत्युच्च}िरतात्युच्चभाषिणी ~॥ १०८ 

नखदन्तक्षतकरी क्रोधेर्ष्या कलहप्रिया~।\\ 
\renewcommand{\thefootnote}{5}\footnote{भ \textendash\ निशापि चारशीला या}निशाविहारशीला च राक्षसं \renewcommand{\thefootnote}{6}\footnote{ब \textendash\ सत्त्वं}शीलमाश्रिता ~॥ १०९ 

तीक्ष्णनासाग्रदर्शना सुतनुस्ताम्रलोचना~। \\
नीलोत्पलसवर्णा च\renewcommand{\thefootnote}{7}\footnote{भ \textendash\ स्वल्पनिद्रा ड \textendash\ स्वप्नोद्वेगा }स्वप्नशीलाऽतिकोपना ~॥ ११० 

तिर्यग्गतिश्चलारम्भा\renewcommand{\thefootnote}{8}\footnote{भ \textendash\ चलरसा बहुसत्त्वाति (ब \textendash\ अभि)} बहुश्वासातिमानिनी\renewcommand{\thefootnote}{9}\footnote{ड \textendash\ बहुसत्त्वाभिनन्दिनी }~।\\ 
गन्धमाल्यासवरता\renewcommand{\thefootnote}{10}\footnote{ड \textendash\ माल्यादिनिरता } नागसत्त्वाऽङ्गना स्मृता ~॥१११ 

\renewcommand{\thefootnote}{11}\footnote{च \textendash\ अत्यर्थं व्यापृतास्या भ \textendash\ तन्वङ्गी दीर्घवदना}अत्यन्तव्यावृतास्या च तीक्ष्णशीला सरित्प्रिया~। \\
सुरासवक्षीररता\renewcommand{\thefootnote}{12}\footnote{भ \textendash\ क्रीतिरसा} बह्वपत्या फलप्रिया ~॥ ११२ 

नित्यं \renewcommand{\thefootnote}{13}\footnote{भ \textendash\ चासन }श्वसनशीला च \renewcommand{\thefootnote}{14}\footnote{ड \textendash\ सदा }तथोद्यान\renewcommand{\thefootnote}{15}\footnote{च \textendash\ रति }वनप्रिया~।\\ 
\renewcommand{\thefootnote}{16}\footnote{भ \textendash\ चला बहुलपा शीघ्रा}चपला बहुवाक्छीघ्रा शाकुनं\renewcommand{\thefootnote}{17}\footnote{ड \textendash\ शीलं }सत्त्वमाश्रिता ~॥ ११३}
\end{quote}

\hrule

\vspace{2mm}

\noindent
\underline{व्यावृतं} विस्तीर्णमास्यमन्तर्मुखं यस्याम्~। निष्कुटे गृहारामे चरतीति (निष्कुट\textendash\ 

\newpage
% १९० नाट्यशास्त्रम् 

\begin{quote}
 {\na \renewcommand{\thefootnote}{1}\footnote{ड \textendash\ न्यून }ऊनाधिकाङ्गुलिकरा\renewcommand{\thefootnote}{2}\footnote{च \textendash\ क्रूरा } \renewcommand{\thefootnote}{3}\footnote{भ \textendash\ रात्रिसंचरणप्रिया}रात्रौ निष्कुटचारिणी~। \\
बालोद्वेजनशीला च पिशुना \renewcommand{\thefootnote}{4}\footnote{भ \textendash\ मृदु च \textendash\ श्लिष्ट ड \textendash\ दृप्त}क्लिष्टभाषिणी ~॥ ११४

\renewcommand{\thefootnote}{5}\footnote{ड \textendash\ सुरतेषूज्झिताचारा }सुरते कुत्सिताचारा रोमशाङ्गी महास्वना\renewcommand{\thefootnote}{6}\footnote{भ \textendash\ सुनिष्ठुरा }~।\\ 
पिशाचसत्त्वा विज्ञेया मद्यमांस\renewcommand{\thefootnote}{7}\footnote{च \textendash\ रति ब \textendash\ अशन ड \textendash\ आसव }बलिप्रिया ~॥ ११५ 

\renewcommand{\thefootnote}{8}\footnote{भ \textendash\ सुप्त}स्वप्नप्रस्वेदनाङ्गी च \renewcommand{\thefootnote}{9}\footnote{भ \textendash\ प्रियशय्यासनस्थिरा}स्थिरशय्यासनप्रिया~। \\
मेधाविनी बुद्धिमती\renewcommand{\thefootnote}{10}\footnote{ड \textendash\ तु मृद्वङ्गी} मद्यगन्धामिषप्रिया ~॥ ११६ 

\renewcommand{\thefootnote}{11}\footnote{भ \textendash\ नित्यदृष्टा कृतज्ञा च स्थूलाङ्गा प्रियदर्शना ड \textendash\ चिरदृष्टे तु}चिरदृष्टेषु हर्षं च कृतज्ञत्वादुपैति सा\renewcommand{\thefootnote}{12}\footnote{च \textendash\ वा }~।\\ 
अदीर्घ\renewcommand{\thefootnote}{13}\footnote{भ \textendash\ केशिनी चैव ज्ञेया यक्षाङ्गना हि सा च \textendash\ गमना चैव ज्ञेया यक्षान्वया\textendash\ ङ्गना}शायिनी चैव यक्षशीलाऽङ्गना स्मृता ~॥११७ 

\renewcommand{\thefootnote}{14}\footnote{च \textendash\ मानापमानयोस्तुल्या \underline{परुषत्वक्कटुकाक्षरा}(ज \textendash\ च कटुस्वना) }तुल्यमानावमाना या परुषत्वक् खरस्वरा\renewcommand{\thefootnote}{15}\footnote{भ \textendash\ परुषत्वात्पटुस्वना }~।\\
\renewcommand{\thefootnote}{16}\footnote{च \textendash\ शठाकृत भ \textendash\ शाठययुक्त}शठानृतोद्धतकथा व्यालसत्त्वा च पिङ्गदृक्\renewcommand{\thefootnote}{17}\footnote{च \textendash\ पिङ्गदृग्व्यालवेशजा भ \textendash\ अथ पिङ्गली } ~॥११८ 

आर्जवाभिरता नित्यं दक्षा क्षान्तिगुणान्विता~। \\
विभक्ताङ्गी कृतज्ञा च गुरुदेवद्विजप्रिया\renewcommand{\thefootnote}{18}\footnote{च \textendash\ देवार्चने रता} ~॥ ११९ 

धर्मकामार्थ\renewcommand{\thefootnote}{19}\footnote{च \textendash\ नित्या च वश्याहंकारवर्जिता }निरता ह्यहङ्कारविवर्जिता~। }
\end{quote}

\hrule

\vspace{2mm}

\noindent
चारिणी) तच्छीलेति~। \underline{कुत्सिताचारेति} औपरिष्टकादिप्रिया~। \underline{(बुद्धिमतीति) }


\newpage
% द्वाविंशोऽध्यायः १९१ 

\begin{quote}
 {\na सुहृत्प्रिया \renewcommand{\thefootnote}{1}\footnote{भ \textendash\ तथा चैव}सुशीला च मानुषं सत्त्वमाश्रिता ~॥ १२० 

संहताल्पतनु\renewcommand{\thefootnote}{2}\footnote{भ \textendash\ धृष्टा पिङ्गायतशिरोरुहा}र्हृष्टा पिङ्गरोमा \renewcommand{\thefootnote}{3}\footnote{च \textendash\ फल }छलप्रिया~। \\
\renewcommand{\thefootnote}{4}\footnote{भ \textendash\ चलचित्तास्थिरा\ldots . रतिप्रिया}प्रगल्भा चपला तीक्ष्णा वृक्षाराम\renewcommand{\thefootnote}{5}\footnote{ड \textendash\ सरित् च \textendash\ रति }वनप्रिया ~॥ १२१ 

स्वल्पमप्युपकारं तु नित्यं या\renewcommand{\thefootnote}{6}\footnote{भ \textendash\ उपकृतं तु या नित्यं} बहुमन्यते~।\\ 
प्रसह्यरतिशीला च \renewcommand{\thefootnote}{7}\footnote{च \textendash\ कपिसत्त्वं समाश्रिता}वानरं सत्त्वमाश्रिता ~॥ १२२ 

महाहनुललाटा \renewcommand{\thefootnote}{8}\footnote{भ \textendash\ या उत्सेध}च \renewcommand{\thefootnote}{9}\footnote{ड \textendash\ मांसल }शरीरोपचयान्विता~।\\ 
पिङ्गाक्षी रोमशाङ्गी च गन्धमाल्यासव\renewcommand{\thefootnote}{10}\footnote{भ \textendash\ आमिष }प्रिया ~॥१२३ 

कोपना स्थिरचित्ता\renewcommand{\thefootnote}{11}\footnote{च \textendash\ सत्त्वा} च \renewcommand{\thefootnote}{12}\footnote{भ \textendash\ तथोद्यानरति }जलोद्यानवनप्रिया~। \\
मधुराभिरता चैव हस्तिसत्त्वा प्रकीर्तिता\renewcommand{\thefootnote}{13}\footnote{ड \textendash\ रतिप्रिया } ~॥ १२४ 

स्वल्पोदरी \renewcommand{\thefootnote}{14}\footnote{च \textendash\ भग्ननासा प \textendash\ भुग्न भ \textendash\ मन्द }भग्ननासा तनुजङ्घा \renewcommand{\thefootnote}{15}\footnote{भ \textendash\ जन }वनप्रिया~।\\
\renewcommand{\thefootnote}{16}\footnote{ड \textendash\ रक्त}चलविस्तीर्णनयना चपला शीघ्रगामिनी ~॥ १२५ 

\renewcommand{\thefootnote}{17}\footnote{भ \textendash\ परि }दिवात्रासपरा \renewcommand{\thefootnote}{18}\footnote{च \textendash\ भीरूरो\textendash\ मशा गीतलोभिनी}नित्यं गीतवाद्यरतिप्रिया~। \\
\renewcommand{\thefootnote}{20}\footnote{च \textendash\ \underline{कोपनायतसत्त्वा} च मृग\textendash\ सत्त्वाङ्गना स्मृता (ड \textendash\ स्थिर) }निवासस्थिरचित्ता च\renewcommand{\thefootnote}{21}\footnote{ भ \textendash\ भूमिनिरता } मृगसत्त्वा प्रकीर्तिता ~॥१२६}
\end{quote}

\hrule

\vspace{2mm}

\noindent
अशेषाविस्मरणं बुद्धिः प्रतिभा~। \underline{प्रसह्येति} कामुकमभियुज्येत्यर्थः~। \underline{निवास }

\newpage
% १९२ नाट्यशास्त्रम् 

\begin{quote}
 {\na दीर्घपीनोन्नतोरस्का \renewcommand{\thefootnote}{1}\footnote{भ \textendash\ रमोमन्दा (?) च \textendash\ चलनाति ड \textendash\ चपला निर्निमेषिणी }चला नातिनिमेषिणी~। \\
\renewcommand{\thefootnote}{2}\footnote{च \textendash\ बहुभृता भ \textendash\ बह्वपत्या तथा चैव }बहुश्रृत्या बहुसुता मत्स्यसत्त्वा जलप्रिया ~॥ १२७ 

लम्बोष्ठी स्वेदबहुला किञ्चिद्विकटगामिनी~।\\ 
कृशोदरी पुष्पफल\renewcommand{\thefootnote}{3}\footnote{भ \textendash\ क्षारमूल}लवणाम्लकटुप्रिया ~॥ १२८ 

\renewcommand{\thefootnote}{4}\footnote{भ \textendash\ उद्बद्ध}उद्बन्धकटिपार्श्वा च खर\renewcommand{\thefootnote}{5}\footnote{भ \textendash\ प्राया प्रियाशना}निष्ठुरभाषिणी~। \\
\renewcommand{\thefootnote}{6}\footnote{ड \textendash\ अभ्युन्नतखर }अत्युन्नतकटीग्रीवा \renewcommand{\thefootnote}{7}\footnote{च \textendash\ भवेदुष्ट्री वनप्रिया }उष्ट्रसत्त्वाऽटवीप्रिया ~॥ १२९ 

स्थूलशीर्षाञ्चित\renewcommand{\thefootnote}{8}\footnote{ड \textendash\ स्थिर भ \textendash\ शीलाञ्चित }ग्रीवा \renewcommand{\thefootnote}{9}\footnote{भ \textendash\ तीक्ष्णदंष्ठ्रा}दारितास्या महास्वना~। \\
ज्ञेया मकरसत्त्वा च\renewcommand{\thefootnote}{10}\footnote{च \textendash\ तु } क्रूरा मत्स्यगुणैर्युता ~॥ १३० 

स्थूलजिह्वोष्ठदशना\renewcommand{\thefootnote}{11}\footnote{ढ \textendash\ रसना च \textendash\ वदना } रूक्षत्वक्कटुभाषिणी~।\\ 
रतियुद्धकरी\renewcommand{\thefootnote}{12}\footnote{च \textendash\ रता } धृष्टा\renewcommand{\thefootnote}{13}\footnote{ड \textendash\ हृष्टा} नखदन्तक्षतप्रिया ~॥ १३१ 

\renewcommand{\thefootnote}{14}\footnote{भ \textendash\ सपक्ष}सपत्नीद्वेषिणी दक्षा चपला शीघ्रगामिनी~।\\ 
\renewcommand{\thefootnote}{15}\footnote{ड \textendash\ सरोषा }सरोगा बह्वपत्या च खरसत्त्वा प्रकीर्तिता ~॥ १३२ 

\renewcommand{\thefootnote}{16}\footnote{भ \textendash\ कृष्ट\textendash\ दंष्ट्रोत्कटमुखी ह्रस्वजङ्घा तथैव च}दीर्घपृष्ठोदरमुखी रोमशाङ्गी बलान्विता~। \\
\renewcommand{\thefootnote}{17}\footnote{भ \textendash\ अयं श्लोकः भ \textendash\ मातृकायां न वर्तते }सुसंक्षिप्तललाटा च कन्दमूलफलप्रिया ~॥ १३३ }
\end{quote}

\hrule

\vspace{2mm}

\noindent
इति सौधादिः~। (\underline{रतियुद्धकरीति}) रतौ युद्धमिव~। (\underline{नखेति}) नखदशनप्रका\textendash\ 

\newpage
% द्वाविंशोऽध्यायः १९३ 

\begin{quote}
 {\na कृष्णा दंष्टोत्कटमुखी \renewcommand{\thefootnote}{1}\footnote{च \textendash\ पीवरोरु}ह्रस्वोदरशिरोरुहा~। \\
हीनाचारा बह्वपत्या \renewcommand{\thefootnote}{2}\footnote{भ \textendash\ सौकरीं वृत्तिं}सौकरं सत्त्वमाश्रिता ~॥ १३४ 

\renewcommand{\thefootnote}{3}\footnote{ड \textendash\ स्फीता, ढ \textendash\ स्थिता}स्थिरा \renewcommand{\thefootnote}{4}\footnote{भ \textendash\ निचित }विभक्तपार्श्वोरुकटीपृष्ठशिरोधरा~।\\ 
\renewcommand{\thefootnote}{5}\footnote{च \textendash\ सुरूपा}सुभगा दानशीला च \renewcommand{\thefootnote}{6}\footnote{भ \textendash\ स्थूलाकुञ्चितमूर्धजा}ऋजुस्थूलशिरोरुहा ~॥ १३५ 

\renewcommand{\thefootnote}{7}\footnote{ड \textendash\ गूढा ड \textendash\ दृढा}कृशा चञ्चलचित्ता च स्निग्धवाक्छीघ्रगामिनी~। \\
कामक्रोधपरा चैव हयसत्त्वाङ्गना स्मृता\renewcommand{\thefootnote}{8}\footnote{भ \textendash\ प्रकीर्तिता} ~॥ १३६ 

स्थूलपृष्ठाक्षि\renewcommand{\thefootnote}{9}\footnote{च \textendash\ अस्थि}दशना तनुपार्श्वोदरा स्थिरा\renewcommand{\thefootnote}{10}\footnote{भ \textendash\ स्निग्धत्वङ्मेदुराचया ड \textendash\ उदरस्थिरा प \textendash\ उदरा स्थिता}~।\\ 
हरिरोमाञ्चिता\renewcommand{\thefootnote}{11}\footnote{भ \textendash\ रोमान्विता ड \textendash\ खररोमाञ्चिता रौद्रा}रौद्री लोकद्विष्टा रतिप्रिया ~॥ १३७ 

किञ्चिदुन्नतवक्त्रा च जलक्रीडावनप्रिया~।\\ 
बृहल्ललाटा सुश्रोणी\renewcommand{\thefootnote}{12}\footnote{भ \textendash\ ललाटजघना } माहिषं \renewcommand{\thefootnote}{13}\footnote{प \textendash\ शीलं }सत्त्वमाश्रिता ~॥१३८ 

कृशा तनुभुजोरस्का \renewcommand{\thefootnote}{14}\footnote{ड \textendash\ निष्टब्धेतर च \textendash\ निष्टब्धतर}निष्टब्धस्थिरलोचना~।\\ 
संक्षिप्तपाणिपादा च सूक्ष्मरोम\renewcommand{\thefootnote}{15}\footnote{च \textendash\ रुक्ष्म(सूक्ष्म? रूक्ष?)रोमा}समाचिता ~॥ १३९ 

भयशीला जलोद्विग्ना\renewcommand{\thefootnote}{16}\footnote{ड \textendash\ जडोन्मत्ता भ \textendash\ अल्पकृष्णा च} बह्वपत्या \renewcommand{\thefootnote}{17}\footnote{भ \textendash\ जन च \textendash\ धन }वनप्रिया~। \\
चञ्चला शीघ्रगमना \renewcommand{\thefootnote}{18}\footnote{भ \textendash\ अजाशीला }ह्यजसत्त्वाङ्गना स्मृता ~॥ १४० }
\end{quote}

\hrule

\vspace{2mm}

\noindent
रादि करोति स्वयं च तदात्मनि बहुमन्यते~। (\underline{संक्षिप्तपाणीति}) संक्षिप्तं परि\textendash\ 

\lfoot{25}
 
\newpage
% १९४ नाट्यशास्त्रम् 

\lfoot{}

\begin{quote}
 {\na \renewcommand{\thefootnote}{1}\footnote{ड \textendash\ उद्बद्ध}उद्बन्धगात्रनयना विजृम्भणपरायणा~। \\
\renewcommand{\thefootnote}{2}\footnote{भ \textendash\ श्वदीर्घ ड \textendash\ दीप्ताल्प च \textendash\ दीर्घान्त}दीर्घाल्पवदना स्वल्पपाणिपादविभूषिता ~॥ १४१ 

\renewcommand{\thefootnote}{3}\footnote{च \textendash\ उच्चैः स्वराल्पनिद्रा च भ \textendash\ बहुवाक्स्व ल्पनिद्रा च}उच्चःस्वना स्वल्पनिद्रा क्रोधना सुकृतप्रिया\renewcommand{\thefootnote}{4}\footnote{भचड \textendash\ लीला4 च \textendash\ बहुभाषिणी भ \textendash\ सुकृतिप्रिया}~। \\
हीनाचारा कृतज्ञा\renewcommand{\thefootnote}{5}\footnote{य \textendash\ अपकृष्टा} च श्वशीला\renewcommand{\thefootnote}{6}\footnote{भचड \textendash\ लीला} परिकीर्तिता ~॥ १४२ 

\renewcommand{\thefootnote}{7}\footnote{भ \textendash\ पृथून्नतनितम्बा च
 \textendash\ शुचिसत्त्वा ड \textendash\ नित्यशौचा}पृथुपीनोन्नतश्रोणी तनुजङ्घा सुहृत्प्रिया~।\\ 
संक्षिप्तपाणिपादा च \renewcommand{\thefootnote}{8}\footnote{च \textendash\ दृष्टारम्भा}दृढारम्भा प्रजाहिता ~॥ १४३ 

पितृदेवार्चनरता \renewcommand{\thefootnote}{9}\footnote{च}सत्यशौचगुरुप्रिया~।\\ 
\renewcommand{\thefootnote}{10}\footnote{य \textendash\ स्थिर }स्थिरा परिक्लेशसहा गवां सत्त्वं समाश्रिता\renewcommand{\thefootnote}{11}\footnote{च \textendash\ उपाश्रिता} ~॥ १४४

नानाशीलाः स्त्रियो \renewcommand{\thefootnote}{12}\footnote{भ \textendash\ ज्ञात्वा रतिसत्त्वमवेक्ष्य च~। तथोपचर्य (ब \textendash\ र्या) तत्त्वज्ञैर्यथा ता रतिमाप्नुयुः~। उपचारो (ब \textendash\ रे) यथा सत्त्वं प्रयुक्तो (ब \textendash\ क्ते) हर्षवर्धनः (ब \textendash\ ना) च \textendash\ उपाश्रिताः }ज्ञेयाः स्वं स्वं सत्त्वं समाश्रिताः~। \\
विज्ञाय च यथासत्त्वमुपसेवेत ताः पुनः\renewcommand{\thefootnote}{13}\footnote{च \textendash\ उपसर्पेद्यथागुणम् ड \textendash\ उपसर्पेत्ततो बुधः} ~॥ १४५ 

उपचारो यथासत्त्वं स्त्रीणामल्पोऽपि हर्षदः\renewcommand{\thefootnote}{14}\footnote{च \textendash\ प्रयुक्तो हर्षवर्धनः ढ \textendash\ हर्षितः}~।\\ 
महानप्यन्यथायुक्तो नैव तुष्टिकरो भवेत् ~॥ १४६ 

\renewcommand{\thefootnote}{15}\footnote{भ \textendash\ मातृकायां श्लोकद्वयं न वर्तते }यथा संप्रार्थितावाप्त्या\renewcommand{\thefootnote}{16}\footnote{ड \textendash\ या स्यात् } रतिः समुपजायते~। }
\end{quote}

\hrule

\vspace{2mm}

\noindent
मितम्~। शीलज्ञा नस्योपयोगमाह \underline{विज्ञाय च यथासत्त्वमुपसेवेतेति~।} सत्त्वानु\textendash\ सारेण सेवायाः प्रयोजनमाह \underline{उपचारो यथासत्त्वमिति~।} एवं च सतीत्येतदेव व्यतिरेकेणाह \underline{महानप्यन्यथेति~।} महानिति पूर्णः~। \underline{अन्यथेति} अयथासत्त्वम्~। 

\newpage
%द्वाविंशोऽध्यायः १९५ 

\begin{quote}
 {\na स्त्रीपुंसयोश्च रत्यर्थमुपचारो विधीयते ~॥ १४७ 

धर्मार्थं हि तपश्चर्या सुखार्थं धर्म इष्यते~।\\ 
सुखस्य मूलं प्रमदास्तासु \renewcommand{\thefootnote}{1}\footnote{प \textendash\ स्वो}सम्भोग इष्यते ~॥ १४८ 

कामोपभागो\renewcommand{\thefootnote}{2}\footnote{ड \textendash\ उपचारो } द्विविधो नाट्यधर्मेऽभिधीयते\renewcommand{\thefootnote}{3}\footnote{ड \textendash\ विधीयते न \textendash\ विविधो नाट्यधर्मो विधीयते}~।\\ 
बाह्याभ्यन्तरतश्चैव नारीपुरुषसंश्रयः\renewcommand{\thefootnote}{4}\footnote{ज \textendash\ संभवः } ~॥ १४९ 

आभ्यन्तरः पार्थिवानां \renewcommand{\thefootnote}{5}\footnote{च \textendash\ कर्तव्यः स च }स च कार्यस्तु नाटके\renewcommand{\thefootnote}{6}\footnote{भ \textendash\ नाट्यके}~। \\
बाह्यो वेश्यागतश्चैव\renewcommand{\thefootnote}{7}\footnote{ड \textendash\ कृतश्चैव च \textendash\ वेश्याङ्गनानां तु} स च\renewcommand{\thefootnote}{8}\footnote{ड \textendash\ तु } प्रकरणे\renewcommand{\thefootnote}{9}\footnote{भ \textendash\ प्रकरणो} भवेत् ~॥ १५० 

\renewcommand{\thefootnote}{10}\footnote{अयं श्लोकः च \textendash\ मातृकायां न दृश्यते भ \textendash\ वस्तु राजोपभोगस्थं}तत्र राजोपभोगं\renewcommand{\thefootnote}{11}\footnote{ज \textendash\ चारं} तु व्याख्यास्याम्यनुपूर्वशः~। \\
उपचारविधिं सम्यक्\renewcommand{\thefootnote}{12}\footnote{भ \textendash\ चैव } कामतन्त्र\renewcommand{\thefootnote}{13}\footnote{ड \textendash\ सूत्र भ \textendash\ कामं तत्र ज \textendash\ समुद्भवम्}समुत्थितम् ~॥१५१ 

\renewcommand{\thefootnote}{14}\footnote{ड \textendash\ द्विविधा }त्रिविधा प्रकृतिः स्त्रीणां नानासत्त्वसमुद्भवा~।}
\end{quote}

\hrule

\vspace{2mm}
 
\begin{sloppypar}
पुरुषार्थान्तरे कस्मादित्यव्युत्पर्त्तिर्न कृतेति वेदप्रामाण्यादि दर्शयति \underline{धर्मार्थ हीत्यादि~।} अस्येदानीं सामान्याभिनयस्य प्रकृत उपयोगं दर्शयति \underline{कामोपभोगो द्विविध इति}~। (नाट्यधर्म इति) नाट्योपाये इतिवृत्त इत्यर्थः~। अभ्यन्तरमन्तःपुरं तत्र भवः (आभ्यन्तरः)~। \underline{नाटक} इति नाटिकायां चेत्यर्थः~। \underline{अनुपूर्वश} इति (वक्ष्यमाणग्रन्थे मुनिना) स्त्रीणां त्रैविध्यम्, राजोपचारे नायि\textendash\ कायाः कामोपपत्तिः, इङ्गितं इष्टवेदनम्, कामावस्थाः, विप्रलम्भोपचारः, दूतीप्रेषणम् , सन्देशः, रत्युपायचिन्ता, [कन्यामित्रम्], प्रच्छन्नकामितम्, 
\end{sloppypar} 

\newpage
% १९६ नाट्यशास्त्रम् 

\begin{quote}
 {\na \renewcommand{\thefootnote}{1}\footnote{1 भ \textendash\ बाह्याभ्यन्तरजा चैव तथा चोभयसंश्रिता }बाह्या चाभ्यन्तरा चैव \renewcommand{\thefootnote}{2}\footnote{2 न \textendash\ बाह्या चाभ्यन्तरा\textendash\ परा च \textendash\ स्यादबाह्येऽभ्यन्तरा परा }स्याद्बाहयाभ्यन्तरापरा ~॥ १५२ 

कुलीनाभ्यन्तरा ज्ञेया बाहया वेश्याङ्गना स्मृता\renewcommand{\thefootnote}{3}\footnote{3 च \textendash\ मता }~। \\
कृतशौचा तु\renewcommand{\thefootnote}{4}\footnote{4 च \textendash\ च } या नारी सा बाहयाभ्यन्तरा स्मृता ~॥ १५३ 

अन्तःपुरोपचारे तु\renewcommand{\thefootnote}{5}\footnote{5 ड \textendash\ चारेषु} कुलजा कन्यकापि वा~।\\ 
न हि राजोपचारे तु बाह्यस्त्रीभोग इष्यते ~॥ १५४ 

आभ्यन्तरो भवेद्राज्ञो बाह्यो बाह्यजनस्य च~। \\
दिव्यवेशाङ्गनानां हि\renewcommand{\thefootnote}{6}\footnote{6 भ \textendash\ अङ्गनाभिश्च} राज्ञां भवति सङ्गमः ~॥ १५५ 

\renewcommand{\thefootnote}{7}\footnote{7 भ \textendash\ कुलटाकामिकं (ब \textendash\ का)}कुलजाकामितं यच्च तज्ज्ञेयं कन्यकास्वपि\renewcommand{\thefootnote}{8}\footnote{8 भ \textendash\ इह }~।\\ 
\renewcommand{\thefootnote}{9}\footnote{9 भ \textendash\ यच्च (या च?) वेश्या समा तत्र}या चापि वेश्या साप्यत्र यथैव कुलजा तथा ~॥ १५६ 

\renewcommand{\thefootnote}{10}\footnote{10 अयं श्लोको बभ \textendash\ मातृकयोर्न दृश्यते}इह कामसमुत्पत्तिर्नानाभाव\renewcommand{\thefootnote}{11}\footnote{11 ड \textendash\ बीज }समुद्भवा~।\\ 
स्त्रीणां वा पुरुषाणां वा उत्तमाधममध्यमा ~॥ १५७ }
\end{quote}

\hrule

\vspace{2mm}

\noindent
अन्तःपुरभोगः, [नि]वासकः इत्यादिना क्रमेणेत्यर्थः~। राज्ञामभ्यन्तर
एव नाटके भोगः, न तु राज्ञामेवेति~। वेशो गणिकानां स्थानं तत्र भवाः
वेश्याः~। (कृतशौचेति) कृतं शौचं शुद्धशीलत्वमेकान्तावरुद्धत्वेन यस्याः,
सा च वेश्या पुनर्भवा~। \underline{कन्येति} गणिका कुमार्यपीत्यर्थः~। \underline{न हि
राजोपचारे तु बाह्यस्त्रीभोग} इत्यस्यापवादमाह दिव्यवेश्येत्यादि~। यथा
पुरूरवसः उर्वश्या~। 

\newpage
% द्वाविंशोऽध्यायः १९७ 

\begin{quote}
 {\na \renewcommand{\thefootnote}{1}\footnote{1 भ \textendash\ श्रवणस्पर्शनाद्रूपादङ्गभाव}श्रवणाद्दर्शनाद्रूपादङ्गलीलाविचेष्टितैः~। \\
मधुरैश्च समालापैः\renewcommand{\thefootnote}{2}\footnote{ 2 ड \textendash\ सप्रलापैश्च ढ \textendash\ संप्रलापेश्च } कामः समुपजायते\renewcommand{\thefootnote}{3}\footnote{ब \textendash\ चीयते } ~॥ १५८ 

\renewcommand{\thefootnote}{4}\footnote{अयं श्लोको भ \textendash\ मातृकायां नास्ति}रूपगुणादिसमेतं कलादिविज्ञानयौवनोपेतम्~।\\ 
दृष्ट्वा पुरुषविशेषं नारी मदनातुरा भवति ~॥ १५९ 

ततः कामयमानानां नृणां स्त्रीणामथापि च\renewcommand{\thefootnote}{5}\footnote{भ \textendash\ वा }~।\\ 
कामभावेङ्गितानीह तज्ज्ञः समुपलक्षयेत् ~॥ १६०}
\end{quote}

कामोपचार इति प्रस्तुतस्तत्र कामस्य कथमुत्पत्तिरित्याह \underline{श्रवणादिति} सर्वत्र शेषः~। सीतायाः श्रवणाद्रावणस्य, शकुन्तलादर्शनाद् दुष्यन्तस्य~।\\ 

\begin{sloppypar}
रूपं चित्रादि प्रतिकृतिः, ततो वत्सेशस्य दृष्टेऽप्याकारे कामोऽनुत्पद्य\textendash\ मानोऽङ्गलीलालक्षणाद्विचेष्टितादुपजायते, नष्टरागप्रत्यानयनं वा ततो भवति, यथा विशाखदेवस्य निबद्धेऽभिसारिकाबन्धितके(वच्चितके?)वत्सेशस्य पद्मावती\textendash\ भट्टशबरीवेषाद्याचरणरूपाल्लीलाचेष्टितात् कामावृत्तिराख्याता~। एवं मधुरेऽ\textendash\ प्यालापे मन्तव्यम्~। माधुर्यमर्थद्वारेण स्वरूपतः~। चकारेणान्यदपि निमित्तं दर्शयति~। यथा तापसवत्सराजचरिते पद्मावतीं प्रति कामोत्पत्तिर्वत्सेश्वरस्य निमित्तमत्यन्तानुवृत्तिर्नाम~। तथा चाह\textendash\ 
\end{sloppypar}

\begin{quote}
 {\qt मयि मनः प्रणिधाय धृता जटा \\
न गणितः स्वजनो न नवं वयः~। 

अनुगमैरिति मामनुरागिणी\\ 
व्यवसितादपनेतुमिवेच्छति ~॥ इति~। (४\textendash\ ८) }
\end{quote}

\noindent
\underline{कामभावः} कामाख्यश्चित्तवृत्तिः, \underline{तत्कृतानीङ्गितानि} शरीरविकारा इत्यर्थः,

\newpage
% १९८ नाट्यशास्त्रम् 

\begin{quote}
 {\na \renewcommand{\thefootnote}{1}\footnote{एतौ भ \textendash\ मातृकायां न स्तः}ललिता \renewcommand{\thefootnote}{2}\footnote{म \textendash\ पक्ष्मविकारा}चलपक्ष्मा च \renewcommand{\thefootnote}{3}\footnote{ड \textendash\ सास्रा मुकुलि\textendash\ तेक्षणा}तथा च मुकुलेक्षणा~। \\
स्रस्तोत्तरपुटा चैव काम्या दृष्टिर्भवेदिह ~॥ १६१ 

\renewcommand{\thefootnote}{4}\footnote{अयं ढड \textendash\ मातृकयोरेव दृश्यते}[\renewcommand{\thefootnote}{5}\footnote{ड \textendash\ फुल्लितान्ता}वलितान्ता सलालित्यसंमितैर्व्यञ्जितैस्तथा\renewcommand{\thefootnote}{6}\footnote{ड \textendash\ लौकनैः}~।\\ 
दृष्टिः सा ललिता नाम स्त्रीणामर्धावलोकने ~॥ ]१६२ 

ईषत्संरक्तगण्डस्तु सस्वेदलवचित्रितः\renewcommand{\thefootnote}{7}\footnote{ड \textendash\ च स्वेदबिन्दुविचित्रितः }~।\\ 
प्रस्पन्दमानरोमाञ्चो मुखरागो भवेदिह\renewcommand{\thefootnote}{8}\footnote{ड \textendash\ तु कामजः (ढ \textendash\ दः)} ~॥ १६३ 

काम्येनाङ्गविकारेण\renewcommand{\thefootnote}{9}\footnote{भ \textendash\ विहारेण } सकटाक्षनिरीक्षितैः\renewcommand{\thefootnote}{10}\footnote{भ \textendash\ निरीक्षणैः}~। \\
तथाभरणसंस्पर्शैः\renewcommand{\thefootnote}{11}\footnote{ड \textendash\ संस्पर्शात्} कर्णकण्डूयनैरपि\renewcommand{\thefootnote}{12}\footnote{ड \textendash\ कण्डूलनादपि ढ \textendash\ कण्डूयनादपि .} ~॥ १६४ 

अङ्गुष्ठाग्रविलिखनै\renewcommand{\thefootnote}{13}\footnote{ड \textendash\ अग्रेण लिखनात् भ \textendash\ लेखनाच्चैव}स्तननाभिप्रदर्शनै\renewcommand{\thefootnote}{14}\footnote{भ \textendash\ प्रदर्शनात् }~।\\
नखनिस्तोदना\renewcommand{\thefootnote}{15}\footnote{भ \textendash\ निस्तो\textendash\ टनात् }च्चैव\renewcommand{\thefootnote}{16}\footnote{ड \textendash\ चापि } केश\renewcommand{\thefootnote}{17}\footnote{भ \textendash\ संचयनात्}संयमनादपि ~॥ १६५ 

वेश्यामेवं विधैर्भावै\renewcommand{\thefootnote}{18}\footnote{भ \textendash\ लासयेत्ब \textendash\ लालयेत्}र्लक्षयेन्मदनातुराम्~।\\ 
कुलजायास्तथा चैव \renewcommand{\thefootnote}{19}\footnote{ड \textendash\ विज्ञेयानीङ्गितानि वै (भ \textendash\ तु)}प्रवक्ष्यामीङ्गितानि तु ~॥ १६६ 

प्रहसन्ती\renewcommand{\thefootnote}{20}\footnote{च \textendash\ च नेत्राणां पतनं च परीक्षयेत्}व नेत्राभ्यां प्रततं च \renewcommand{\thefootnote}{21}\footnote{भ \textendash\ या}निरीक्षते~। \\
स्मयते \renewcommand{\thefootnote}{22}\footnote{ड \textendash\ च}सा निगूढं च \renewcommand{\thefootnote}{23}\footnote{ड \textendash\ वाक्यं}वाचं चाधोमुखी वदेत् ~॥ १६७}
\end{quote}

\newpage
% द्वाविंशोऽध्यायः १९९ 

\begin{quote}
 {\na स्मितोत्तरा मन्दवाक्या स्वेदाकारनिगूहनी\renewcommand{\thefootnote}{1}\footnote{प \textendash\ निगूहिनी ड \textendash\ निगूहना }~। \\
प्रस्पन्दिताधरा चैव चकिता \renewcommand{\thefootnote}{2}\footnote{म \textendash\ कुलजाङ्गना भ \textendash\ कुलाङ्गनाम्\ldots सुरतोत्सवाम् }च कुलाङ्गना ~॥ १६८ 

3एवंविधैः कामलिङ्गैरप्राप्तसुरतोत्सवा~।\\
\renewcommand{\thefootnote}{4}\footnote{ड \textendash\ दशावस्थागतं}दशस्थानगतं कामं नानाभावैः\renewcommand{\thefootnote}{5}\footnote{च \textendash\ भावं } प्रदर्शयेत्\renewcommand{\thefootnote}{6}\footnote{ढ \textendash\ प्रकाशयेत् } ~॥ १६९

प्रथमे त्वभिलाषः\renewcommand{\thefootnote}{7}\footnote{ड \textendash\ अभिलाषा }स्याद् द्वितीये चिन्तनं भवेत्~। \\
अनुस्मृतिस्तृतीये तु चतुर्थे गुणकीर्तनम् ~॥ १७० 

उद्वेगः पञ्चमे प्रोक्तो विलापः षष्ठ उच्यते~।\\ 
उन्मादः सप्तमे ज्ञेयो भवेद्वयाधि\renewcommand{\thefootnote}{8}\footnote{च \textendash\ अथ ड \textendash\ ततः }स्तथाष्टमे ~॥ १७१ 

नवमे जडता चैव
\renewcommand{\thefootnote}{3}\footnote दशमे मरणं भवेत्~। }
\end{quote}

\hrule

\vspace{2mm}

\underline{अप्राप्तसुरतोत्सवेति} प्राप्तसंभोगत्वे तु नैते विकाराः प्रादुर्भवन्ति~। यदा तु काम उदितस्तदादेः प्राप्तसंभोगता कामावस्थानामुदय एव तथा च प्राप्त\textendash\ संभोगतायामपि विप्रलम्भे \renewcommand{\thefootnote}{*}\footnote{कामसूत्रं\textendash\ III \textendash\ २. (कुसुमसधर्माणो हि योषितः) }कुसमसदृक्षादिमहिमानं प्राप्ते कामिजनसंभोगे भवन्त्येवैता अवस्थाः~। तथा च भट्टतोतेनोक्तम्\textendash\ 

\centering{\textbf{कामावस्था न श्रृङ्गारः क्वचिदासां तदङ्गता~। इति}}
 
\noindent
पूर्वप्राप्तसंभोगतायामपि श्रृङ्गाराङ्गतेति यावत्~। अप्राप्तसंभोगित्वेऽपि हि सागरिकावत्सराजयोर्दर्शितः श्रृङ्गारो रसाध्याये~। 

अभिलाषात्मकः कामः क्रमादीदृशीर्दशाः प्रतिपद्यत इत्याह \underline{प्रथमे त्वभि\textendash\ लाष} इत्यादि~। अत्र व्यभिचारिण एव [केचित्] कामावस्था लक्षणान्तर\textendash\ 

\renewcommand{\thefootnote}{9}\footnote{ड \textendash\ प्रोक्ता }
3 }भ \textendash\ विविधैः कामलिङ्गैश्च

 
\newpage
% २०० नाट्यशास्त्रम् 

\begin{quote}
 {\na स्त्रीपुंसयोरेष विधिर्लक्षणं च निबोधत ~॥ १७२ 

\renewcommand{\thefootnote}{1}\footnote{भ \textendash\ व्यवसाय}व्यवसायात्समारब्धः संकल्पेच्छासमुद्भवः~। \\
समागमोपायकृतः\renewcommand{\thefootnote}{2}\footnote{भ \textendash\ परः } सोऽभिलाषः प्रकीर्तितः ~॥ १७३ 

निर्याति विशति च मुहुः\renewcommand{\thefootnote}{3}\footnote{भ \textendash\ निर्गच्छति प्रविशति च } करोति चाकारमेव मदनस्य~। \\
तिष्ठति च दर्शनपथे प्रथम\renewcommand{\thefootnote}{4}\footnote{म \textendash\ स्थानस्यिते कामे}स्थाने स्थिता कामे ~॥ १७४ 

केनोपायेन संप्राप्तिः कथं वासौ\renewcommand{\thefootnote}{5}\footnote{भच \textendash\ संप्राप्यः कथं वा स ड \textendash\ संभवेत्} भवेन्मम~।\\ 
दूती6\renewcommand{\thefootnote}{6}\footnote{भ \textendash\ संपा\textendash\ दितैः ब \textendash\ संवादितैः}िवेदितैर्भावै7\renewcommand{\thefootnote}{7}\footnote{ड \textendash\ वाक्यैः}ति चिन्तां निदर्शयेत् ~॥ १७५ 

आकेक\renewcommand{\thefootnote}{8}\footnote{ड \textendash\ अक्षि}रार्धविप्रेक्षितानि वलयरशनापरामर्शः~।\\ 
\renewcommand{\thefootnote}{9}\footnote{भ \textendash\ नाभ्यूरुदर्शनं च तथा }नीवी\renewcommand{\thefootnote}{10}\footnote{ड \textendash\ नाभ्यूरूणां स्पर्शः कार्यो }नाभ्योः \renewcommand{\thefootnote}{11}\footnote{च \textendash\ संदर्शनं च }संस्पर्शनं च कार्यं द्वितीये तु ~॥१७६ }
\end{quote}

\hrule

\vspace{2mm}

\begin{sloppypar}
\noindent
योगादिह पुनस्ताः~। \underline{व्यवसायादिति} काम्यजनज्ञानं \underline{तत्संकल्पपूर्वकेच्छा} तत उद्भव उद्रिक्तत्वमस्येति \underline{समागमोपायस्य} तद्विषयस्य चिन्ता विषयस्य द्वितीयावस्थात्मनः कृतं करणं यतो वक्ष्यति हि केनोपायेन संप्राप्यत इति चिन्तनीयाद्यवस्थासहचरितं कार्यं, वाग्व्यापाराद्युचितः शब्दश्चान्यसाह\textendash\ चर्यादित्याह \underline{निर्याति विशति चेत्यादि~।} मदनस्याकार अपाक्रियते येन दृष्ट्या\textendash\ दिविशेषेण \underline{दर्शनपथ} इति तं यत्र पश्येत तेन दृश्येतेत्यर्थः~। दूतीनिवेदितै\textendash\ र्भावैः मनोरथैरित्युपलक्षणं स्वकल्पितैरपीत्यर्थः \underline{आकेकरमिति} विचलमविचलं \underline{विप्रेक्षितम्~। }
\end{sloppypar}

\newpage
% द्वाविंशोऽध्यायः २०१ 

\begin{quote}
 {\na मुहुर्मुहुर्निःश्वसितैर्मनोरथविचिन्तनैः~। \\
\renewcommand{\thefootnote}{1}\footnote{ढ \textendash\ प्रद्वेषणाच्च कार्याणां च \textendash\ प्रद्वेषणात्त्वन्य}प्रद्वेषाच्चान्यकार्याणामनुस्मृतिरुदाहृता\renewcommand{\thefootnote}{2}\footnote{भ \textendash\ अपीष्यते} ~॥ १७७ 

नैवासने न शयने धृतिमुपलभते स्वकर्मणि विहस्ता~।\\ 
\renewcommand{\thefootnote}{3}\footnote{न \textendash\ तच्चित्त}तच्चिन्तोपगतत्वात् तृतीयमेवं प्रयुञ्जीत ~॥ १७८ 

अङ्गप्रत्यङ्गलीलाभिर्वाक्चेष्टा\renewcommand{\thefootnote}{4}\footnote{भ \textendash\ वाक्येष्ट}\renewcommand{\thefootnote}{4}\footnote{भ \textendash\ वाक्येष्ट}हसितेक्षितैः\renewcommand{\thefootnote}{5}\footnote{च \textendash\ ईक्षणैः }~।\\ 
नास्त्यन्यः सदृशस्तेनेत्येतत् स्याद्\renewcommand{\thefootnote}{6}\footnote{भ \textendash\ इत्यथैतत्} गुणकीर्तनम् ~॥१७९ 

गुणकीर्तनोल्लुकसनै\renewcommand{\thefootnote}{7}\footnote{भ \textendash\ उत्ककथनैः च \textendash\ उप्लुकशनैः}रश्रुस्वेदापमार्जनैश्चापि\renewcommand{\thefootnote}{8}\footnote{ड \textendash\ अवमार्जनाद्वापि (भ \textendash\ चैव) }~।\\ 
\renewcommand{\thefootnote}{9}\footnote{भ \textendash\ दूतीविहारविस्र\textendash\ म्भणैश्चतुर्थे त्वभिनयः स्यात् च \textendash\ दूत्या}दूत्यविरहविस्रम्भैरभिनययोगश्चतुर्थे तु ~॥ १८० 

आसने शयने चापि न तुष्यति न तिष्ठति\renewcommand{\thefootnote}{10}\footnote{ड \textendash\ हृष्यति भ \textendash\ तिष्ठति न हृष्यति}~।\\ 
नित्यमेवोत्सुका च स्यादुद्वेगस्थानमाश्रिता\renewcommand{\thefootnote}{11}\footnote{च \textendash\ एव तु भ \textendash\ उद्वेगं तु विनिर्दिशेत् ड \textendash\ एव तत् } ~॥ १८१ 

चिन्तानिःश्वास\renewcommand{\thefootnote}{12}\footnote{ड \textendash\ खेदैश्च हृत्ताप }खेदेन \renewcommand{\thefootnote}{13}\footnote{भ \textendash\ हृच्छोका}हृद्दाहाभिनयेन च~। \\
कुर्यात्तदेव\renewcommand{\thefootnote}{14}\footnote{भ \textendash\ तमेवं } \renewcommand{\thefootnote}{15}\footnote{ड \textendash\ अत्यर्थं}मत्यन्तमुद्वेगाभिनयेन च\renewcommand{\thefootnote}{16}\footnote{भ \textendash\ नयं बुधः } ~॥ १८२ 

इह स्थित इहासीन इह चोपगतो मया~। }
\end{quote}

\hrule

\vspace{2mm}

\begin{sloppypar}
\underline{विहस्तेति} अशक्ता~। \underline{दूत्या अविरह} इति समासः~। \underline{उद्वेगाख्यं स्थान}\textendash\ मवस्था~। \underline{उद्वेगाभिनयो} निर्वेदे दर्शितः~। रुदितनिःश्वसितादिः पूर्वावस्थाया 
\end{sloppypar}

\lfoot{26}

\newpage
% २०२ नाट्यशास्त्रम् 

\lfoot{}

\begin{quote}
 {\na इति तैस्तैर्विलपितैर्विलापं संप्रयोजयेत्\renewcommand{\thefootnote}{1}\footnote{भ \textendash\ तु विनिर्दिशेत् ड \textendash\ \underline{विलापस्त्विति दर्शितः} (ढ \textendash\ स्त्त्वभिदर्शितः) } ~॥ १८३ 

\renewcommand{\thefootnote}{2}\footnote{अयं भ \textendash\ मातृकायां न दृश्यते}उद्विग्नात्यर्थमौत्सुक्यादधृत्या च विलापिनी\renewcommand{\thefootnote}{3}\footnote{च \textendash\ अरत्या च विलापनी ड \textendash\ रत्या चापि विलापिनी}~।\\ 
ततस्ततश्च भ्रमति विलापस्थानमाश्रिता ~॥ १८४ 

\renewcommand{\thefootnote}{4}\footnote{भ \textendash\ तत्संश्रयां ड \textendash\ तत्संश्रयकथायुक्ता च \textendash\ मातृकायामिदमर्धं न वर्तते }तत्संश्रितां कथा युङ्क्ते सर्वावस्थागतापि हि~। \\
\renewcommand{\thefootnote}{5}\footnote{च \textendash\ प्रद्वेष्टि चापरान्पुंसो यत्रोन्मादः स उच्यते}पुंसः \renewcommand{\thefootnote}{6}\footnote{ड \textendash\ प्रद्वेषितान्यश्चैष स्यात्}प्रद्वेष्टि चाप्यन्यानुन्मादः संप्रकीर्तितः\renewcommand{\thefootnote}{7}\footnote{भ \textendash\ स उन्मादो विधीयते } ~॥१८५ 

तिष्ठत्यनिमिष\renewcommand{\thefootnote}{8}\footnote{भ \textendash\ अनियम }दृष्टिर्दीर्घं निःश्वसिति गच्छति ध्यानम्~।\\
रोदिति \renewcommand{\thefootnote}{9}\footnote{भ \textendash\ विकार }विहारकाले \renewcommand{\thefootnote}{10}\footnote{य \textendash\ लास्यं}नाट्यमिदं स्यात्तथोन्मादे ~॥१८६

सामदानार्थसंभोगैः \renewcommand{\thefootnote}{11}\footnote{भ \textendash\ काम्य}काम्यैः \renewcommand{\thefootnote}{12}\footnote{च \textendash\ संपेक्षणैः}संप्रेषणैरपि~।\\ 
सर्वै\renewcommand{\thefootnote}{13}\footnote{भ \textendash\ निरन्तरकृतैस्ततो व्याधि\textendash\ र्भवेदिह }र्निराकृतैः पश्चाद्\renewcommand{\thefootnote}{14}\footnote{च \textendash\ सर्वैरवनाकारणात् (?) } व्याधिः समुपजायते ~॥१८७ }
\end{quote}

\hrule

\vspace{2mm}

\begin{sloppypar}
\noindent
उत्तरावस्थान्तरीभवतीति दर्शयति उद्विग्ना सती विलापिनी भवतीति~। \underline{सर्वा\textendash\ वस्थागतापीति} गुरुजनसन्निधावपीति, अनेनोन्मादत्वं स्फुटयति~। विहारकाल इति क्रीडोचितेषु कालेषु रोदितीत्यर्थः~। \underline{नाट्यमिति} न तु पटशकलवक्रशरावाद्य\renewcommand{\thefootnote}{*}\footnote{पटच्छेदाकाशरावाद्य\textendash\ क}\textendash\ त्रोन्मादोक्तमिति भावः~। इयत्युन्मादपर्यन्ते प्राप्ते चित्तवृत्तिभवे कामे शरीरमप्य\textendash\ न्यथाभवतीत्याह \underline{सर्वैर्निराकृतैः पश्चाद्वयाधिरिति}~। निराकृतैः विफलीभूतैः~। 
\end{sloppypar}

\end{document}