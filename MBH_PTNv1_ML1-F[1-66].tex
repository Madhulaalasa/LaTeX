\documentclass[11pt, openany]{book}
\usepackage[text={6.0in,7.7in}, centering, includefoot]{geometry}
\usepackage[table, x11names]{xcolor}
\usepackage{fontspec,realscripts}
\usepackage{polyglossia}
\setdefaultlanguage{sanskrit}
\setotherlanguage{english}
\setmainfont[Scale=0.65]{Shobhika}
\newfontfamily\s [Script=Devanagari, Scale=1]{Shobhika}
\newfontfamily\regular{Linux Libertine O}
\newfontfamily\en [Language=English, Script=Latin]{Linux Libertine O}
\newfontfamily\mbh [Script=Devanagari, Scale=0.65, Color=purple]{Shobhika-Bold}
\newfontfamily\qt [Script=Devanagari, Scale=0.65, Color=violet]{Shobhika-Regular}
\newcommand{\devanagarinumeral}[1]{
\devanagaridigits{\number \csname c@#1\endcsname}} % for devanagari page numbers
\XeTeXgenerateactualtext=1 % for searchable pdf
\usepackage{enumerate}
\pagestyle{plain}
\usepackage{fancyhdr}
\pagestyle{fancy}
\renewcommand{\headrulewidth}{0.5pt}
\usepackage{afterpage}
\usepackage{multirow}
\usepackage{multicol}
\setlength{\columnseprule}{0.5pt}
\usepackage{wrapfig}
\usepackage{vwcol}
\usepackage{microtype}
\usepackage{amsmath,amsthm, amsfonts,amssymb}
\usepackage{mathtools}% <-- new package for rcases
\usepackage{graphicx}
\usepackage{longtable}
\usepackage{setspace}
\usepackage{footnote}
\usepackage{perpage}
\MakePerPage{footnote}
\usepackage{xspace}
\usepackage{array}
\usepackage{emptypage}
\usepackage{hyperref}% Package for hyperlinks
\hypersetup{colorlinks, 
citecolor=black, 
filecolor=black, 
linkcolor=blue, 
urlcolor=black}

\renewcommand{\baselinestretch}{0.2}

\begin{document}

\thispagestyle{empty}
\begin{center}
\includegraphics[width=0.8\linewidth]{Images/page-001.jpeg}
\end{center}

% आर्यावर्तः
% विन्ध्योत्तराभारतभूमिः 
% आ समुद्रात्तु वै पूर्वादा समुद्रात्तु पश्चिमात्~। तयोरेवान्तरं गिर्योरार्यावर्त विदुर्बुधाः~॥ मनुः~।
% प्रागादर्शात् प्रत्यक्कालकवनात् दक्षिणेन हिमवन्तमुत्तरेण पारियात्रम्~। 
% एतस्मिन्नार्यावर्ते आर्यनिवासे ये ब्राह्मणाः कुंभीधान्या अलोलुपा अगृह्यमाणकारणाः किञ्चिदन्तरेण कस्याश्चिद्विद्यायाः पारंगतास्तत्रभवन्तः शिष्टाः~॥ महाभाष्यम्~।

\newpage
\thispagestyle{empty}
\begin{center}
\textbf{\s \Large श्रीमन्महर्षिपतञ्जलिविनिर्मितं}\\
\textbf{\s \huge पाणिनीयव्याकरणमहाभाष्यम्}

\vspace{4mm}
{\s श्रीमन्नागेशभट्टविरचित \textendash\ \\}

\textbf{\s भाष्यप्रदीपोद्द्योतोद्भासितेन}

\vspace{2mm}
{\s श्रीमदुपाध्यायकैयटप्रणीतेन}

\textbf{\s भाष्यप्रदीपेन}

{\s समुल्लसितम्}

\vspace{3mm}
{\small \s तत्रायम्}

{\s अष्टाध्यायीप्रथमाध्यायप्रथमपादव्याख्यानात्मको}

{\s नवाह्निकरूपः}

\vspace{2mm}
\textbf{\s प्रथमः खण्डः}

\vspace{7mm}
{\s स च}

\vspace{3mm}
{\s जोशीकुलभूषणश्रीभिकाजीशर्मात्मजेन}

\textbf{\s श्रीभार्गवशास्त्रिणा}

\vspace{2mm}
{\s टिप्पणीपाठभेदादिसंयोजनपुरःसरं}

{\s संशोधितम्}

\vspace{10mm}
\includegraphics[width=0.2\linewidth]{latex/a.JPG}

\textbf{\large \s १९८७}

\textbf{\Large \s चौखम्बा संस्कृत प्रतिष्ठान}

\textbf{\s दिल्ली}
\end{center}

\newpage
\thispagestyle{empty}
It is a reproduction of the earlier edition of 

"Nirnaya Sagax Press Bombay. 

\begin{center}
THE

\textbf{\huge VYĀKARAṆAMAHĀBHĀSYA}

\textbf{OF}

\textbf{\Large PATAÑJALI}

WITH

\textbf{THE COMMENTARY BHĀṢYAPRADīPA}

OF

\textbf{\large KAIYAṬA UPĀDHYĀYA}

AND 

THE SUPERCOMMENTARY BHĀṢYAPRADĪPODDYOTA 

OF 

\textbf{NĀGEśA BHAṬṬA}

\vspace{3mm}
VOḺUME I

NAVAHNIKA

 ( ON THE ASTĀDHYĀYĪ, CHAPTER 1, QUARTER 1 ) 

\vspace{3mm}
EDITED

WITH

NOTES AND VARIANTS

BY

BHARGAVA SASTRI BHIKAJI JOSI

\vspace{15mm}
\includegraphics[width=0.2\linewidth]{latex/a.JPG}

\textbf{1987}

\textbf{\large Chaukhamba Sanskrit Pratishthan}

Delhi ( India ) 
\end{center}

\newpage
\thispagestyle{empty}
\begin{center}
\textbf{\Large CHAUKHAMBA SANSKRIT PRATISHTHAN}

\emph{\en ( Oriental Publishers \& Distributors ) }

38 U.A., Jawaharnagar, Bungalow Road 

P.B. No. 2113 

{\large DELHI 110007}

\emph{\en Phoṅe : 236391}

\vspace{35mm}

Reprint Edition 

1987
\end{center}

\vspace{35mm}
This Publication has been brought out with the

financial assistance from Ministry of Education 

\& Culture, Govt. of India.\\

If any defect 1s found in this book, please 

return the copy by V.P.P. for the cost of 

postage to the publishers for free exchange.\\

Printed in India\\

Published by Navanita Dass Gupta for Chaukhamba 

Sanśkrit Pratishthan, 38 U.A., Jawahar Nagar, 

Bungalow Road, Delhi \textendash\ l110 007 and printed at 

shriji Mudranalaya, K \textendash\ 37/117, Gopal Mandir Lane, 

Varanasi.

\newpage
\thispagestyle{empty}
\begin{center}
\textbf{ॐ नमः शब्दब्रह्मात्मने शिवाय~।}

\textbf{\LARGE हरचरितचिन्तामणिः~।}\\
\rule{0.2\linewidth}{0.5pt}

\textbf{प्रथमं पाणिनीयव्याकरणोपज्ञात्राचार्य \textendash\ पाणिनि \textendash\ कात्यायन \textendash\ व्याडि \textendash\ प्रभृतीनाम्}

\textbf{अध्ययनदेशकालादिबोधनाय गुणाढ्यव्यरचितबृहत्कथानुवाद \textendash\ सोमदेवरचित \textendash\ }

\textbf{कथासरित्सागर \textendash\ क्षेमेन्द्ररचितबृहत्कथामञ्जरीग्रन्थानुसारि \textendash\ हरचरित \textendash\ }

\textbf{चिन्तामणेः शब्दशास्त्रावतारनामा सप्तविंशः प्रकाशः प्रदर्श्यते \textendash\ }\\

\rule{1\linewidth}{0.5pt}\\
\end{center}

\begin{multicols}{2}
\noindent
\begin{quote}
{\qt तत्तन्मोहतमोपहं पशुपते यल्लक्षणं तावकम्~।\\
सर्व पाणिनिबद्धमित्यधिगतं येनाद्वितीयोऽसि च~॥}

{\mbh यन्माहात्म्यवशादुदेति च परा वाग्वाच्यवैचित्र्यसू \textendash\ \\
स्तन्मे संविदि सामरस्यमधुना पुष्णातु लोकोत्तरम्~॥~१~॥

सन्तापसंततिकरी तावत्तिष्ठति मूर्खता~।\\
वाच्यवाचकरूपोऽयं न ध्यातो यावदीश्वरः~॥~२~॥

पूर्वं कदाचिद्भगवान्कैलासशिखरोपरि~।\\
विविधैर्विभ्रमैरासीददेव्यः सह रहोभुवि~॥~३~॥

सप्रमोदेन चित्तेन देवी शंभुमतोषयत्~।\\
उत्सङ्गमधिरोप्यैतां शम्भुर्दृष्टोऽभ्यभाषत~॥~४~॥

प्रियं करोमि सुभगे किं तवेत्यभिधीयताम्~।\\
श्रुत्वेति साब्रवीद्देवी स्मेरीकृतविलोचना~॥~५~॥

सत्यं यदि प्रसन्नोsसि किंचिदाख्याहि सत्कथाम्~।\\
कस्याऽपि वर्तते या न विदिता परमेश्वर~॥~६~॥

उक्त्वेति भूयोऽप्यवदन्नन्दिन् द्वारं निरुध्यताम्~।\\
न केनचित्प्रविष्टव्यमिहेति तुहिनाद्रिजा~॥~७~॥

नन्दिन्यथाश्रितद्वारे सावधानतया स्थिताम्~।\\
अभाषत महादेवो देवीं मधुरया गिरा~॥~८~॥

सदा सुखममर्त्येषु दुःखमैकान्तिकं नृषु~।\\
वैद्याधरं तच्चरितं सुखदुःखमयं श्रुणु~॥~५~॥

एवं यावच्छिवो देवीं बभाषे तावदागमत्~।\\
पुष्पदन्तो गणः शंभुप्रसादरसभाजनम्~॥~१०~॥

प्रवेशं प्रार्थयन्द्वारे निषिद्धः सोऽथ नन्दिना~।\\
बुभुत्सुः कारणं योगात्प्रविवेशालिविग्रहः~॥~११~॥

प्रविष्टः सर्वमशृणोत्कथितं चन्द्रमौलिना~।\\
विद्याधराणां चरितं सप्तानामद्भुतं च सः~॥~१२~॥

श्रुत्वा निर्गत्य भार्यायै स जयायै जगाद तत्~।\\
विमोहयति सर्वं हि दुर्लङ्घया भवितव्यता~॥~१३~॥

ततः सखीनां पुरतस्कथावर्णनोद्यता~।\\
देवी जयामप्यज्ञासीदुक्तिभिस्तत्कथाविदम्~॥~१४~॥

अकुप्यत्पार्वती भर्त्रे नापूर्वं कथितं त्वया~।\\
जयाऽप्येतद्विजानातीत्यभ्यधाच्च तदाऽचिरात्~॥ १५~॥}
\end{quote}

\columnbreak

\noindent
\begin{quote}
{\mbh प्रणिधायं ततः शम्भुरभाषत हिमाद्रिजाम्~।\\
श्रुता योगप्रविष्टेन पुष्पदन्तेन सा कथा~।\\
बभाषे स जयायै तु न जानात्यपरः प्रिये~॥~१६~॥

श्रुत्वेति चाद्रिजा पुष्पदन्तमानाययत्क्रुधा~।\\
शशाप चाविनीतस्त्वं भव मर्त्य इति क्रुधा~॥~१७~॥

तथेव माल्यवन्तं च गणं तत्पक्षपातिनम्~।\\
जयया प्रार्थिता देवी ततः शापान्तमभ्यधात्~॥~१८~॥

प्रभूणां करुणाहेतुर्विनयप्रार्थनैव हि~।\\ 
यक्षः कुबेरशापेन सुप्रतीकः पिशाचताम्~॥~१९~॥

आसाद्य विन्ध्याटव्यां यः काणभूतिरिति स्थितः~।\\ स्मृतजातिस्तमालोक्य पुष्पदन्तः कथामिमाम्~॥~२०~॥

यदा वर्णयते तस्तै तदा शापाद्विमोक्ष्यते~।\\
शापावसानमित्युक्त्वा विरराम हिमाद्रिजा~॥~२१~॥

दृष्टनष्टाविव गणौ तौ च तत्र वभूवतुः~।\\
कालान्तरेण कौशाम्ब्यां सोमदत्तद्विजन्मनः~॥~२२~॥

भार्यायां वसुदत्तायां पुष्पदन्तो गणोऽजनि~।\\
वररूच्य१भिधानेन स ततः समवर्धत~॥~२३~॥

अतिबालस्य चैतस्य जनकः पञ्चतामगात्~।\\
माता च तं सुतस्नेहाद्विधवा समवर्धयत्~॥~२४~॥

क्रमेणाथ स दिव्यां तां धिषणां समवाप्तवान्~।\\
अथैकदा तत्सदनं ब्राह्मणावभ्यगच्छताम्~॥~२५~॥

दूराध्वगमनश्रान्तौ स्थातुं दिवसमेककम्~।\\
तयोर्निवसतोस्तत्र मुरजध्वनिरुद्ययौ~॥~२६~॥

माता तं चाब्रवीद्भर्तुः स्मृत्वा बाष्पाकुलेक्षणा~।\\
मित्रं तव पितुर्नन्दो नटो नृत्यति पुत्र सः~॥~२७~॥

श्रुत्वैति सोऽब्रवीन्मातर्गच्छाम्यहमवेक्षितुम्~।}
\end{quote}

\noindent
\rule{1\linewidth}{0.5pt}\\

१ {\qt नाम्ना वररुचि किंच कात्यायन इति श्रुतः} कथासरित्सागरे \textendash\ १~। २~। १~॥ तस्याहं बसुदत्तायां जातः श्रुतधराभिधः~। कात्यायनो वररुचिश्चेत्यन्वर्थकृताह्वयः~। {\qt बृहत्कथामञ्जरी \textendash\ कात्या \textendash\ यमः श्रुतधरस्तथा वररुचिश्च सः~। गुणिनामग्रणीर्लोके नामभि \textendash\ रित्यभिरुच्यते} १~। ७०
\end{multicols}

\fancyhead[CE,CO]{\thepage}
\cfoot{}
\newpage
%%%%%%%%%%%%%%%%%%%%%%%%%%%%%%%%%%%%%%%%%%%%%%%%%%%%%
\renewcommand{\thepage}{\devanagarinumeral{page}}
\setcounter{page}{2}
% २

\begin{multicols}{2}
\begin{quote}
{\mbh सर्वं ते दर्शयिष्यामि नाट्यं वीक्ष्य यथाक्रमम्~॥~२८~॥

तेनेति कथिते विप्रौ तौ विस्मयमवापतुः~।\\
ततोऽब्रवीत्तौ तन्माता नात्रांशेनापि संशयः~॥~२९~॥

तच्छ्रुतं वा दृष्टं वा बालो जानात्यसाविति~।\\
प्रातिशाख्यं ततस्ताभ्यां जिज्ञासुभ्यामपठ्यत~॥~३०~॥

तथैव तद्वररुचिः पपाठ च तयोः पुरः~।\\
ताभ्यां स सहितो गत्वा वीक्ष्य नाट्यं निजे गृहे~॥~३१~॥

स्वमातुर्दर्शयामास स समग्रं तथैव तत्~।\\
सकृद्भाहिणमालोक्य प्राज्ञं वररुचिं तदा~॥~३२~॥

व्याडिरेकस्तयोर्मध्यात्तन्मातरमभाषत~।\\
द्वेवस्वामी करम्बश्च सोदरौ प्राग्बभूवतुः~॥~३३~॥

परस्परमतिप्रीतौ वीत ( १ ) सीनगरे द्विजौ~।\\
इन्द्रदत्तोsयमेकस्य तयोः सूनुरजायत~॥~३७~॥

अहं व्याडिर्द्वितीयस्य जनकः संस्थितश्च मे~।\\
इन्द्रदत्तपिता यातस्तच्छोकेन महापथम्~॥~३५~॥

अस्मन्मात्रोश्च हृदयं शोकेन त्रुटितं ततः~।\\
धने सत्यप्यनाथौ तौ गतौ विद्याभिलाषिणौ~॥~३६~॥

अभ्यर्थयन्तौ तावावां तपोभिः खामिनं गुहम्~।\\
गुहस्तत्रादिशत्स्वप्ने प्रभुरावां तपःस्थितौ~॥~३७~॥

नन्दस्य नृपतेरस्ति पुरं पाटलिपुत्रकम्~।\\
वर्षाख्यस्तत्र विप्रोsस्ति तस्माद्विद्यामवाप्स्यथः~॥~३८~॥

इति श्रुत्वा विभोर्वाक्यमावां तस्य पुरं गतौ~।\\
तत्पुरं प्राप्य पृच्छज्भ्यामावाभ्यां शुश्रुवे जनात्~॥~३९~॥

द्विजोsस्ति मूखो वर्षाख्य इति चिन्तावहं वचः~।\\
अन्विष्यद्भ्यामथावाभ्यां दारिद्र्यैकनिधिर्गृहे~॥~४०~॥

दृश्यते स्म द्विजो वर्षो ध्यानस्तिमितलोचनः~।\\
तत्पत्नी विहितातिथ्या धूसरा मलिनाम्बरा~॥~४१~॥

आवं प्रविष्टौ प्रणतौ पप्रच्छास्मत्प्रयोजनम्~।\\
अस्मन्निवेदितोदन्ता ततः साध्वी जगाद सा ?~॥~४२~॥

पुत्रयोर्युवयोरग्रे कालज्ञा ( २ ) कथयामि तद्~।\\
उक्त्वेति भर्तृवृत्तान्तकथां वक्तुं प्रचक्रमे~॥~४३~॥

बभूव शंकस्वामी नगरेsस्मिन्द्विजोत्तमः~।\\
मत्पतिस्तस्य पुत्रोsयमुपवर्षsतथा पुरा~॥~४४~॥

मूर्खश्च दुर्गतश्चायं, बुधश्च धनवांश्च सः~।\\
तेन भ्रात्रा स्वभार्याsस्य नियुक्ता निजपोषणे ?~।~४५~॥

प्रावृट् कदाचिदासन्ना तस्या आस्ते स्म योषितः~।\\
जुगुप्सितं पिष्टमयं गुह्यरूपं गुडान्वितम्~॥~४६~॥

दत्त्वा विप्राय कस्मैचिल्लभन्ते स्म ऋतुं स्त्रियः~।\\
निदाघे, शीतकाले च स्नानक्लेशासहिष्णवः~॥~४७~॥

एवं योषिज्जनाः प्रायः स्वमाचारं वितन्वते~।\\
जुगुप्सितं तद् गृह्णाति को नाम मतिमान्द्विजः~॥~४८~॥

तस्मादाददतेमूर्खा दक्षिणां धनलम्पटाः~।}
\end{quote}

\rule{0.9\linewidth}{0.5pt}

१ वेतसे पाठः~। २ {\qt का लञ्जा} इति कथा सः सा.~॥ 

\columnbreak

\begin{quote}
{\mbh सा च मद्देवरवधूरासन्नऋतुशङ्किनी~॥~४९~॥

तमन्वतिष्ठदाचारमस्मै मूर्खाय गर्विता~।\\
अयं सदक्षिणं तश्च गृहीत्वा गृहमागतः~॥~५०~॥

निर्भर्त्सितो मया मूर्खभावाच्चान्तरतप्यत~।\\
अथ स्वामिकुमारस्य सानुतापस्तपो व्यधात्~।~५१~॥

प्रभुणा तेन तुष्टेन विद्याश्चास्य प्रकाशिताः~।\\
सकृद्ग्राहिणमासाद्य शिष्यं तास्त्वं प्रकाशयेः~॥~५२~॥

इत्युक्तश्च गुहेनायं प्रमोदाद्गृहमागतः~।\\
स्वं वृत्तान्तं समग्रं च समागत्य न्यवेदयत्~॥~५३~॥

ततः प्रभृत्ययं ध्यायञ्जपंश्च सततं स्थितः~।\\
समागतौ युवां सद्यः सर्वसिद्धिर्भविष्यति~॥~५४~॥

उक्त्वेति सा वर्षवधूः साध्वी तूष्णीं बभूव च~।\\
अदायि तस्यै चावाभ्यां हेम दारिद्र्यनाशनम्~॥~५५~॥

परिभ्रम्य भुवं लब्धः सकृद्ग्राही न कुत्रचित्\\
आवाभ्यामद्य तु प्राप्तस्तव श्रुतधरः सुतः~॥~५६~॥

तदमुं खसुतं देहि विद्यासिध्द्यै यतावहे~।\\
इति व्याडिवचः श्रुत्वा जगदे वसुदत्तया~॥~५७~॥

सर्वं सङ्गतमेवैतत्प्रत्ययोऽत्रास्त्यसंशय~।\\
तथाहि जातमात्रेsस्मिन्वाक् दैवी ह्युदपद्यत~॥~५८~॥

अयं श्रुतधरो लब्ध्वा विद्या वर्षादिशेषतः~।\\
लोके व्याकरणं दिव्यं प्रतिष्ठां प्रापयिष्यति~॥~५५~॥

रोचते दि वरं यस्मान्नाम्ना वररुचिः सुधीः~।\\
प्रतिपद्येति३ सा वाणी व्यरमञ्च नभस्तलात्त्~॥~६०~॥

ततः प्रभॄति बालेsस्मिन्वर्धमाने मुहुर्मुहुः~।\\
वर्षोपाध्यायलाभाय मम चेतः प्रवर्तते~॥~६१~॥

युवाभ्यामद्य तज्ज्ञातं सन्तोषश्च परो मम~।\\
तदेष युवयोर्भ्राता नीयतां च तदन्तिकम्~॥~६२~॥

इत्याकर्ण्य वचस्तस्य व्याडिः पूर्णमनोरथः~।\\
महोत्सवविधानाय ददाति स्म निजं धनम्~॥~६३~॥

उपनीय ततो व्याडिर्वेदार्हत्वाय तं शिशुम्~।\\
जग्राह मात्रा कथमप्यनुज्ञातं सबाष्पया~॥~६४~॥

लब्ध्वा वररुचिं \textendash\ गाढप्रमोदौ तरसा ततः~।\\
व्याडीन्द्रदत्तौ वर्षस्य गुरोः प्राप्तौ निकेतनम्~॥~६५~॥

जानन्वररुचिं मूर्तं प्रसादं षण्मुखस्य सः~।\\
वर्षः सन्तोषमासाद्य तदानीमास्त निर्वृतः~॥~६६~॥

अन्येद्युः पुरतः कृत्वा तान्शुद्धे वसुधातले\\
ओङ्कारमकरोद्वर्षौपाध्यायो दिव्यया गिरा~॥~६७~॥

तदानीमस्य चत्वारः साङ्गा वेदाः सभुद्ययुः~।\\
अध्यापयितुमेतांश्च प्रावर्तत पुरःस्थितान्~॥~६८~॥

सकृच्छ्रुतं वररुचिर्व्याडिश्च द्विःश्रुतं वचः~।\\
त्रिःश्रुतं चेन्द्रदत्तः स जग्राह गुरुणोदितम्~॥~६९~॥

अपूर्वं ध्वनिमाकर्ण्य दिव्यं संजातविस्मयाः~।\\
उपासते स्म सर्वेऽपि पौरा वर्षगुरुं ततः~॥~७०~॥}
\end{quote}

\rule{0.9\linewidth}{0.5pt}

३ इति प्रतिपद्य सा वाणीत्यन्वयः~॥ 
\end{multicols}

\newpage
% ३

\begin{multicols}{2}
\begin{quote}
{\mbh स्कंन्दप्रसादं तं ज्ञात्वा तदेशनृपतिस्ततः~।\\
हर्षेण वर्षोपाध्यायमुपास्ते स्म सविस्मयः~॥~७१~॥

अथ कालेन बहवः शिष्या वर्षमुपाययुः~।\\
एकोsपि पाणिनिर्नाम जडबुद्धिरुपाययौ~॥~७२~॥

शिष्यान्तरोपहासेन सावमानः स पाणिनिः~।\\
शुश्रूषाक्लेशतो यातः कदाचित्तुहिनाचलम्~॥~७३~॥

आराध्य तपसा तत्र विद्याकामः स शंकरम्~।\\
प्राप व्याकरणं दिव्यं स च विद्यामुखं शुभम्~॥~७४~॥

तत्प्राध्याह्वयते स्मायं सर्वान्वादाय पाणिनिः~।\\
स्ववर्ग्यान्वररुच्यादीनुपहासान्स्मरन्सुधीः~॥~७५~॥

ततोsभवद्वररूचेर्वादः पाणिनिना सह~।\\
दिनानि सप्त च ययुस्तयोर्विवदमानयोः~॥~७६~॥

अथाष्टमे दिने तेन पाणिनौ निर्जिते सति~।\\
नभस्तलान्महाघोरं हुंकरोति स्म शंकरः~॥~७७~॥

श्रुत्वा शंकरहुंकारं ततः पाणिनिना जितम्~।\\
मूर्खत्वं वररुच्याद्याः प्रापुश्च प्रतिवादिनः~॥~७८~॥

ऐन्द्रं व्याकरणं नष्टं समग्रं चाभवद्भुवि~।\\
ततो वररुचिर्दुःखं विद्याविरहितो दधे~॥~७९~॥

मूर्खीभूतो वररूचिर्मुक्ताहारो विनिर्ययौ~।\\
आराधयितुमीशानं तपोभिस्तुहिनाचले~॥~८०~॥

परितुष्टो वररुचेस्तपोभिः परमेश्वरः~।\\
शास्त्रं प्रकाशयामास पाणिनीयमशेषतः~॥~८१~॥

शंकरेच्छाप्रसादेन चूर्णीकृत्याथ तच्च सः~।\\
जगाम स्वगृहं तुष्यन्नज्ञातान्वपरिश्रमः~॥~८२~॥

तत्प्राप पाणिनेः शास्त्रं वर्षः खामिकुमारतः~।\\
वर्षाव्याडीन्द्रदत्तौ च लब्धवन्तावसंशयम्~॥~८३~॥

एवं व्याकरणं दिव्यं प्रबोधायाsभ्यधात्पुनः~।\\
पाणिनीयमधिष्ठाय शरीरं परमेश्वरः~॥~८४~॥

ततो व्याडीन्द्रदत्ताभ्यां प्रार्थितो गुरुदक्षिणाम्~।\\
अङ्गीचकार वर्षः स हेमकोटिं सुनिर्मलाम्~॥~८५~॥

इन्द्रदत्तः प्रविश्याथ संस्थितस्य कलेवरम्~।\\
योगेन नन्दनृपतेर्व्याड्येsदत्त काञ्चनम्~॥~८६~॥

व्याडिः काञ्चनकोटिं तां वर्षायादत्त दक्षिणाम्~।\\
इन्द्रत्तश्च नन्दोsभूचछकटालधिया नृपः~॥~८७~॥

इन्द्रदत्तस्ततो योगे नन्दभूपालतां भजन्~।\\
मन्त्रिभावे वररुचि महाप्राज्ञं न्ययोजयत्~॥~८८~॥

किमभ्यदथ नन्दस्य चिरमादाय मन्त्रिताम्~।\\
खिन्नो वररुचिः शान्तस्तपोवनमशिश्रियत्~॥~८९~॥

एकदा स ययौ द्रष्टुंदेवीं विन्ध्यनिवासिनीम्~।\\
आराधिता च तपसा सा तं स्वप्ने समादिशत्~॥~९०~॥

गच्छ विन्ध्याटवीमद्य काणभूतिमवेक्षितुम्~।\\
इति देव्या गिरा सोsपि विन्ध्यकान्तारमाययौ~॥~९१~॥

अपश्यत्काणभूतिं स पिशाचैस्तत्र चावृतम्~।\\}
\end{quote}

\columnbreak

\begin{quote}
{\mbh जगाद स सदाचारः कथमीदृग् भवानिति~॥~९२~॥

श्रुत्वा वररूचेरेवं काणभूतिरभाषत~।\\
पादोपसंग्रहं कृत्वा सौजन्यमधुरां गिरम्~॥~९३~॥

विज्ञानं मे स्वतो नास्ति श्मशाने श्रुतमीश्वरात्~।\\
यदुज्जयिन्यां सर्वं ते कथयामि शृणुष्व तत्~॥~९४~॥

श्मशाने च कपाले च तव देव रतिः कुतः~।\\
इति सप्रणयं पृष्टः पार्वत्या शंभुरब्रवीत्~।\\
कल्पान्तसमये पूर्वमभूदेकार्णवं जगत्~॥~९५~॥

पातितः शोणितकणो विभिद्योरुं मया ततः~।\\
जले स एवाण्डमभूद्विधा गतवतस्ततः~॥~९६~॥

निरगात्पुरुषः सृष्टौ सृष्टा च प्रकृतिर्मया~।\\
ततः स निखिलं सृष्ट्वा प्रजापतिपुरःसरम्~॥~९७~॥

पितामह इति प्रोक्तो गाढाहंकारतामगात्~।\\
ततः कररुहेणाहंं तन्मूर्धानमपाटयम्~॥~९८~॥

महाव्रतं गृहीत्वा च श्मशानप्रियतां श्रये~।\\
एतद्देवि कपालात्म जगत्करतले स्थितम्~॥~९९~॥

तस्याण्डस्य कपाले द्वे रोदसी परिकीर्तिते~।\\
कपालप्रियता तेन सदैव मम वर्तते~॥~१००~॥

श्रुत्वेति शंभोस्तत्रैव स्थितवानस्मि सादरम्~।\\
ततो भयि श्रोतुकामे पार्वती पुनरब्रवीत्~॥~१०१~॥

कियता समयेनास्मान्पुष्पदन्त उपैष्यति~।\\
आकर्ण्येत्यवदद्देवीमुद्दिशन्मां त्रिलोचनः~॥~१०२~॥

कुबेरानुचरः सोsयं यक्षः प्राप्तः पिशाचताम्~।\\
अस्य स्थूलशिरा नाम मित्रमासीन्निशाचरः~॥~१०३~॥

तन्मैत्र्या धनदः शापात्पिशाचममुमाददे~।\\
प्रार्थिते दीर्घजङ्घेन भ्रात्राऽप्यस्य धनाधिपः~॥~१०४~॥

शापावसानमवदत्किंचित्कोपं परित्यजन्~।\\
पुष्पदंतात्समाकर्ण्य कथां शापावरोहिणः~॥~१०५~॥

उक्तत्वा शापावतीर्णस्य तां च माल्यवतोऽखिलाम्~।\\
गणाभ्यां सहितस्ताभ्यामेष शापाद्विमोक्ष्यते~॥~१०६~॥

पुष्पदन्तस्य शापान्तस्त्वयाऽप्येवं कृतः स्मर~।\\
श्रुत्वेति शंभोर्वचनं हृष्यन्नहमिहागतः~॥~१०७~॥

तस्मान्निर्वते शापः पुष्पदन्तागमान्मम~।\\
आकर्ण्येमां वररुचिः काणभूतेर्गिरं ततः~॥~१०८~॥

जातिं सुप्तोत्थित इव स्मृत्वा तत्क्षणमब्रवीत्~।\\
पुष्पदन्तगणः सोऽहं शृणु मत्तोऽखिलं च तत्~॥~१०९~॥

उक्त्वेति ग्रन्थलक्षाणि सप्त सप्ताब्रवीत्कथाः~।\\
आकर्ण्य ताः काणभूतिरभाषत सविस्मयः~॥~११०~॥

त्वं रुद्र एव कोऽन्यो वा कश्चिद्वेत्तीदृशीः कथाः~।\\
श्रुत्वा कथा इमाः शापो विरतो मे शरीरतः~॥~१११~॥

स्ववृत्तान्तं समाख्याहि बाल्यात्प्रभृति मे प्रभो~।\\
ततो विनीतशिरसं वृत्तान्तं स्वमवर्णयत्~॥~११२~॥

काणभूतिं वररुचिर्जन्मनःप्रभृति स्फुटम्~।}
\end{quote}
\end{multicols}

\newpage
% ४

\begin{multicols}{2}
\begin{quote}
{\mbh उक्त्वा स्ववार्तां भूयोऽपि काणभूतिं जगाद सः~॥~११३~॥

स्वास्थ्यं लभे त्वामालोक्य परं खेदमहं श्रितः~।\\
त्वदालोकनमाहात्म्यान्मम शापो निर्वर्तते~॥~११४~॥

प्रभावाद्विन्ध्यवासिन्या मयोक्ताश्च महाकथाः~।\\
क्षीणशापो वपुस्त्यक्त्वा तत्प्राग्जन्म भजाम्यहम्~॥~११५~॥

\ldots~\ldots~\ldots~\ldots~\ldots~\ldots~\ldots~ पुनरवस्थितिः~।\\

मत्पक्षपाती प्रथमं माल्यवानप्यशप्यत~।\\
तस्मै महेश्वरेणोक्ता कथनीया महाकथा~॥~११६~॥

त्वं च सम्प्रति तिष्ठेह यावदायाति तेऽन्तिकम्~।\\
त्यक्तभाषात्रयः सोऽपि गुणाढ्यो द्विजसत्तमः~॥~११७~॥

एवं वररुचिस्तत्र काणभूतेर्निवेद्य सः~।\\
देहमोक्षाय त्वरितमगाद्वदरिकाश्रमम्~॥~११८~॥

ततो वररुचिस्त्यक्त्वा योगधारणया वपुः~।\\
प्राग्जन्म तत्समासाद्य भगवद्गणतामगात्~॥~११९~॥

स गणो माल्यवान्नाम देवीशापादधोभवत्~।\\
तत्काले गुरुणा तेन प्रोक्तोऽसौ द्विजसत्तमः~॥~१२०~॥

क्रमेण विद्याः सर्वाः स समासाद्य प्रसिद्धिमान्~।\\
सुप्रतिष्ठितनामानं देशं प्राप प्रकृष्टधीः~॥~१२१~॥

सातवाहनभूपालमास्थानस्थमवैक्षत~।\\
शर्ववर्मादिभि सर्वैर्मन्त्रभिः परिवारितः~।\\
स राजा तममात्यत्वे स्तुतिपूर्वं न्यवेशयत्~॥~१२२~॥

अथाऽसौ राज्यकर्माणि चिन्तयन्मन्त्रिभावतः~।\\
ताँस्तान्नध्यापयञ्छिष्यान्गुणाढ्यः सुखमन्वभूत्~॥~१२३~॥

कदाचिदथ भूपालो वसन्ते कामिनीसखः~।\\
दिव्योद्यानावनौ वापीजले चिक्रीड सादरः~॥~१२४~॥

सपाणियन्त्रधाराभिः सिषेच वरकामिनीः~।\\
कामिन्योऽपि प्रजह्रुस्तं कटाक्षैः सह वारिभिः~॥~१२५~॥

एकदा तस्य क्रीडन्ती नितम्बस्तनगौरवात्~।\\
खिद्यमाना क्लमं प्राप वापीमध्ये विलासिनी~॥~१२६~॥

सिञ्चन्ति सलिलैर्भूपं सा जगादालसालसा~।\\
मोदकैर्नाथ मां सद्यः प्रहरेति कृतस्मिता~॥~१२७~॥

एवं तद्वचनं श्रुत्वा जलप्रदरणात्मकम्~।\\
शब्देन च्छलितो राजा मोदकैस्तां तताड सः~॥~१२८~॥

ततो विहस्य सा राज्ञी पुनरेवमभाषत~।\\
राजन्नवसरः कोऽत्र मोदकानां जलान्तरे~॥~१२९~॥

उदकैः सिञ्च मा मा त्वं मामित्युक्तं मया हि तत्~।\\
संधिमात्रं न जानासि माशब्दोदकशब्दयोः~॥~१३०~॥

न च व्याकरणं वेत्सि मूर्खस्त्वं कथमीदृशः~।\\
इत्युक्तः स तया राज्ञ्या शब्दशास्त्रविदा नृपः~॥~१३१~॥

ततश्चिन्तापरो मुह्यन्नाहारादिविवर्जितः~।\\
चिन्तास्थ इव पृष्टोऽपि नैव किंचिदभाषत~॥~१३२~॥

पाण्डित्यं शरणं वा मे मृत्युर्वेति विचिन्तयन्~।\\
शयनीये परित्यक्तगात्रः संतापवानभूत्~॥~१३३~॥}
\end{quote}

\columnbreak

\begin{quote}
{\mbh उपविश्याथ निकटे मन्त्रिणो ज्ञातमानसाः~।\\
कारणं देव कथय वर्तसे विमना इति~॥~१३४~॥

तच्छ्रुत्वाsपि तथैवास्त तूष्णीं स सातवाहनः~।\\
ततोऽवदत्सुधीः कश्चिद्वर्षैर्द्वादशभिः सह~॥~१३५~॥

ज्ञायते सर्वविद्यानां मुख्यं व्याकरणं नृप~।\\
अहं तु शिक्षयाभ्येतत्तुल्यः स्कन्देन चापरः~॥~१३६~॥

तदहं मासषट्केन देव त्वां शिक्षयामि तत्~।\\
ततः स्वामिकुमारस्य प्रसादात्तदकल्पयत्~॥~१३७~॥

शिक्षयामास राजानं सुधीः प्राप्तार्थसञ्चयः~।\\
राजा कवित्वपाण्डित्यमयीं प्राप च चातुरीम्~॥~१३८~॥

ततो गुणाढ्यस्तद्वीक्ष्य प्रतिज्ञां प्राक्तनीं स्मरन्~।\\
संस्कृतं प्राकृतं देशभाषामपि समत्यजत्~॥~१३९~॥

संत्यज्य कृतमौनत्वाद्व्यवहारानसौ ततः~।\\
निर्ययौ नगरात्तस्माद्दिदृक्षुर्विन्ध्यवासिनीम्~॥~१४०~॥

स्वप्ने स विन्ध्यवासिन्या प्रेषितस्तदगाद्वनम्~।\\
यत्र स्थितः काणभूतिः पिशाचैः परिवारितः~॥~१४१~॥

तत्राऽशृणोत्पिशाचानां परस्परमसौ कथाः~।\\
शिशिक्षे चात्र तद्भाषां भाषात्रयविलक्षणाम्~॥~१४२~॥

पिशाचभाषया तत्र मौनमोक्षैकहेतुना~।\\
स्वागतं विदधे काणभूतेर्विन्ध्याटवीस्थितेः~॥~१४३~॥

मित्रस्य रक्षसो भूतिवर्मणो दिव्यचक्षुषः~।\\
वचसा माल्यवन्तं तं गुणाढ्यं सोsभ्यगाद्वने~॥~१४४~॥

काणभूतिः कथां तस्य पुष्पदन्तोदितां ततः~।\\
अवर्णयद्गुणाढ्यस्य शापान्तसमयोत्सुकः~॥~१४५~॥

निबबन्ध गुणाढ्यस्ताश्चतुर्थ्या भाषया कथाः~।\\
सप्तैव सप्तभिर्वर्षैर्ग्रन्थलक्षाणि सप्त सः~॥~१४६~॥

मसीमटव्यामत्राप्य गुणाढ्यः स्वाङ्गशोणितैः~।\\
लिलेख ताः कथा दिव्याश्चित्रचारित्रशालिनीः~॥~१४७~॥

निबद्धास्ता गुणाढ्येन दृष्ट्वा तत्र महाकथाः~।\\
त्यक्तशापो गतिं प्राप काणभूतिर्निजां ततः~॥~१४८~॥

काणभूतेरनुचराः पिशाचास्तत्र ये स्थिताः~।\\
तेऽपि दिव्यां कथां श्रुत्वा सर्वे प्रांपुर्दिवं ततः~॥~१४९~॥

इयं बृहत्कथा पृथ्व्यां प्रसिद्धिं प्राप्य ( प्स्य ) ते कथम्~।\\
इति शापान्तसोत्कण्ठो गुणाढ्यः समचिन्तयत्~॥~१५०~॥

प्राहिणोत्तां कथां सोऽथ सातवाहनभूभुजे\\
अधिचिक्षेप राजाऽपि तां कथां मदनिष्ठुरः~॥~१५१~॥

सानुतापो गुणाढ्योऽपि वह्रिकुण्डं ततो व्यधात्~।\\
व्याख्याय पत्रमेकैकं निचिक्षेप च तत्र सः~॥~१५२~॥

देहातिवाहमुत्सृज्य तृणाम्बुमयमादरात्~।\\
अशृण्वन्साश्रवस्तत्र तां कथां मृगपक्षिणः~॥~१५३~॥

निराहारेषु शुष्यत्सु तदानीं मृगपक्षिषु~।\\
अस्वादुनि च तन्मांसे भुक्ते प्राप रुजं नूपः~॥~१५४~॥

गुणाढ्यचरितं श्रृत्वा वनेचरजनात्ततः~।}
\end{quote}
\end{multicols}

\newpage
% ५

\begin{multicols}{2}
\begin{quote}
{\mbh आजगाम स्वयं राजा तमेवोद्देशमादरात्~॥~१५५~॥

सबाष्पमृगमध्यस्थं गुणाढ्यं वनवासिनम्~।\\
प्रत्यभिज्ञाय च ततो नमश्चक्रे महीपतिः~॥~१५६~॥

अथ पृष्टवतो राज्ञः स्ववृत्तान्तं जगाद सः~।\\
भूतभाषामयैर्वाक्यैर्गुणाढ्यो विस्मयस्पृशः~॥~१५७~॥

आख्याहि निजवृत्तान्तं पुनरप्यब्रवीदसौ~।\\
दग्धानि ग्रन्थलक्षाणि षडेकमवशिष्यते~॥~१५८~॥

ग्रन्थलक्षमिदं यैका सा कथा गृह्यतां त्वया~।\\
नरवाहनदत्तस्य चरितं त्वेतदद्भुतम्~॥~१५९~॥

उक्त्वेत्यदत्त स कथाः सातवाहनभूभुजे~।\\
राजाऽपि प्रणमन्भक्त्या जग्राह निजमूर्धनि~॥~१६०~॥

अथ ज्ञात्वा स शापान्तं त्यक्त्वा योगेन विग्रहम्~।\\
माल्यवान्गणतां लेभे शिवभक्त्येकभावितः~॥~१६१~॥

वृहत्कथा सा सर्वत्र प्रसिद्धिं प्रापदद्भुता~।\\
यां स्वयं स महादेव्याश्चन्द्रमौलिरवर्णयत्~॥~१६२~॥}
\end{quote}

\columnbreak

\begin{quote}
{\mbh अनया कथया किंवा प्रसङ्गोद्दिष्टया मया~।\\
ऐद्रं व्याकरणं हित्वा पाणिनीयं व्यधाच्छिवः~॥~१६३~॥

ततःप्रभृति निःशेषशब्दज्ञानप्रकाशकम्~।\\
दिव्यं व्याकरणं भूमौ पाणिनीयं प्रसिद्ध्यति~॥~१६४~॥

पाणिनिर्भगवानेव स्वयं चन्द्रार्घशेखरः~।\\
प्रतिष्ठापयते कोऽन्यो दिव्यं व्याकरणं भुवि~॥~१६५~॥

अशेषेष्वपि शास्त्रेषु स्तूयते कैर्न पाणिनिः~।\\
ज्ञायन्ते सम्यगेवैते यत्प्रसादेन वाचकाः~॥~१६६~॥

कालान्तरेण सर्वज्ञस्याज्ञया भुजगेश्वरः~।\\
स्वजिह्वाः सफलीचक्रे बह्लीर्भाष्योपदेशतः~॥~१६७~॥

परिमितमतयः किमाचरन्तु \\
प्रसृमरसंशयखिद्यमानचित्ताः~।\\
विभुरनवधिवाच्यवाचकात्मा \\
शिव इति पश्य तमेव सर्वशक्तिम्~॥~१६८~॥}
\end{quote}
\end{multicols}

\begin{center}
इति श्रीमहामाहेश्वरजयद्रथविरचिते हरचरितचिन्तामणौ शब्दशास्त्रावतारो नाम सप्तविंशः प्रकाशः~॥\\

\vspace{3cm}
\rule{0.1\linewidth}{0.5pt}
\end{center}

\newpage
\thispagestyle{empty}
\begin{center}
\textbf{\Large पाणिनीयम्~।}\\

\rule{0.15\linewidth}{0.5pt}
\end{center}

\noindent
\rule{1\linewidth}{0.5pt}\\

\begin{multicols}{2}
तदेवं विदितमस्त्येव १सर्वेषां सुधीवराणाम् \textendash\ सर्बष्वपि व्याकरणेषु प्रांतिशाख्येषु च लोक \textendash\ २सर्ववेदसाधारणतया मूर्धन्यतमं पाणि \textendash\ \\
\rule{1\linewidth}{0.5pt}\\

१ एशियन, युरोपियन, अमेरिकन.

२ {\qt सर्ववेदपारिषदं हीदं शास्त्रम्} इति {\qt पूर्वापरप्रथम \textendash\ } ( २~। १~। ५८ ) इति सूत्रे भाष्यम्~॥

\begin{quote}
{\qt पाणिनीयं महाशास्त्रं पदसाधुत्वलक्षणम्~।\\
सर्वोपकारकं आह्यं कृत्स्नं त्याज्यं न किंचन~॥}
\end{quote}

\noindent
इति पराशरोपपुराणम्~॥ 

अस्मिंश्च पाणिनीये व्याकरणे छन्दसि ( १~। २~। ३६ ) आम्नाय ४~। ३~। ११३ {\qt निगमे} ( ७~। २~। ६४ ) मंत्रपदेषु २~। २~। २३ {\qt ऋचि ४~। १~। ९ यजुषि ६~। १~। ~। ११७ काठके ७~। ४~। ३८ साम १~। २~। ३४ आथर्वर्णिकः ६~। ४ [ १७४ संहितायाम् ( १~। २~। ३९ ) ब्राह्मणे } ( २~। ३~। ६० ) उपनिषदौ १~। ४~। ७९ अनार्षे १~। १~। १६ क्रम \textendash\ पदंन्यायमीमांसा ४~। २~। ६१ {\qt अनुब्राह्मणावत् ४~। २~। ६२ अनु \textendash\ प्रवचनादिभ्यः ५~। १~। १११ भिक्षुनट्सूत्रयोः ४~। ३~। ११० महान्\ldots\ldots मारत ६~। २~। ३८} इत्यादिग्रन्थानाम् {\qt सास्य देवता} ( ४~। २~। २४ ) इत्यधिकारे क \textendash\ शुक्र \textendash\ अपोनप्तृ \textendash\ अपांनप्तृ \textendash\ महेन्द्र \textendash\ सोम \textendash\ वायु \textendash\ ऋतु \textendash\ उषस् \textendash\ द्यावापृथिवी \textendash\ शुनासीर \textendash\ मरुत्वत् \textendash\ अग्नीषोम \textendash\ वास्तोष्पति \textendash\ गृहमेध \textendash\ अग्नि \textendash\ काल \textendash\ महाराजप्रोष्ठपदानाम् \textendash\ {\qt देवताद्वन्द्वे ( ६~। ३~। २६ ) } इ्त्यधिकारे \textendash\ अग्नि \textendash\ सोम \textendash\ वरुण \textendash\ दिव् \textendash\ पृथिवी \textendash\ उषसाम् {\qt देवताद्वन्द्वे ( ७~। ३~। २१ ) } इत्यधिकारे \textendash\ इन्द्रवरुणयोश्च देवतात्वेन, दृष्टं साम ४~। २~। ७{\qt इत्यधिकारे कलि \textendash\ वामदेवयोः तेन प्रोक्तम् ( ४~। ३~। १०१ ) } इत्यधिकारे तित्तिरि \textendash\ वरतन्तु \textendash\ खण्डिकं \textendash\ उख \textendash\ काश्यप \textendash\ कौशिक \textendash\ कलाप्यन्तेवसि \textendash\ वैशम्पायनान्ते \textendash\ वासि \textendash\ शौनकादि \textendash\ कठ \textendash\ चरक \textendash\ कलापिन् \textendash\ छगलिन् आदीनां च वर्णनं दृंश्यते~॥ एवं {\qt सारव \textendash\ ऐक्ष्वाक ( ६~। ४~। १७४ ) साकेत ( ४~। ३ १२७ ) } देशानां नवीनां कूपानां व्याकरण \textendash\ प्रातिशाख्यान्यतर \textendash\ निर्मातृणाम् \textendash\

\begin{center}
\begin{tabular}{c c}
आचार्याणाम् & ७~।~३~।~४९\\
आपिशलेः & ३~।~१~।~९२\\
उंदीचीम् & ४~।~१~।~१५३\\
एकेषाम् & ८~।~३~।~१०४\\
काश्यपस्य & १~।~२~।~२५\\
गार्ग्यस्य & ८~।~३~।~२०\\
गालवस्य & ७~।~१~।~७४\\
चाक्र्वर्मणस्य & ६~।~१~।~१३०\\
प्राचाम् & ४~।~१~।~१७
\end{tabular}
\end{center}

\noindent
\rule{1\linewidth}{0.5pt}\\

1 अनयोर्वृष्टप्रोक्तयोर्नैव विशेषः~। पूर्वं दृष्टानां ( शातानां $=$ स्मृतानाम् ) एव प्रवचनस्योत्वितत्वात्~। नश्यदृष्टं प्रोच्यते~। अत एव {\qt य एवाप्ता वेदार्थानां द्रष्टारः प्रवक्तारश्च} इति ( २~।~१~।~६७ ) 

\columnbreak

\noindent
नीयं व्याकरणमिति~॥\\

\noindent
\rule{1\linewidth}{0.5pt}\\

\begin{center}
\begin{tabular}{c c}
भारद्वाजस्य & ७~।~२~।~६७\\
शाकटायनस्य & ३~।~४~।~१११\\
शाकल्यस्य & १~।~१~।~१६\\
सेनकस्य & ५~।~४~।~११२\\
स्फोटायनस्य & ६~।~१~।~१२३
\end{tabular}
\end{center}

आचार्याणां च नामानि समुपलभ्यन्ते~॥

लिपिशब्दो {\qt दिवाविभा \textendash\ ३~। २~। २१सूत्रे} 

यवनशब्दश्च इन्द्रवरुण \textendash\ ( ४~। १~। ४९ ) सूत्रे 

शका यवनाश्च वसिष्ठविश्वामित्रयुद्धसमय एवोत्पन्ना इति रामायणतौ व्यक्तमेव प्रतीयते~। वाल्मीके रामायणनिर्माणं

\begin{quote}
{\qt रावणान्तकरो राजा रघूणां वंशवर्धनः~।\\
वास्मीकिर्यस्य चरितं चक्रे भार्ववनन्दनः~॥}
\end{quote}

इति मात्स्ये १२ अध्याये दशितं, रामचन्द्रसत्तायामेव वाल्मीकि \textendash\ मुनेराश्रम एव सीतात्याग \textendash\ लवकुशोत्पत्तिवर्णनस्य सर्वत्रोपलम्भात् श्रीरामकृतयज्ञसमय एव लवकुशकृत् \textendash\ रामचरितगानस्य वर्णितत्वात्~॥ रामावतारश्च १८५६०००० वर्षेभ्यः प्रागेत जात इति

\begin{quote}
{\qt त्रेतायुगे चतुर्विंशे रावणस्तपसः क्षयात्~।\\
रामं दाशरथिं प्राप्य सगणः क्षयमीयिवान्~॥}
\end{quote}

वायुपुराणे उत्तरार्धे ९ मेऽध्याये ४८ श्लोके स्पष्टमेव~॥

\begin{quote}
{\qt शनकैस्तु क्रियालोपादिमाः क्षत्रियजातयः~।\\
वृषलत्वं गता लोके ब्राह्मणानामदर्शनात्~॥~१०~।~४३~॥

पौण्ड्रकाश्चोड \textendash\ द्रविडाः काम्बोजा यवनाः शकाः~।\\
पारदाः पह्लवाश्चीनाः किराता दरदाः खशाः~॥~१०~।~४४~॥}
\end{quote}

इत्येवं वर्ण्यन्ते मनुस्मृतावपि यवनादयः~॥

यवना इव हूणा अपि म्लेच्छजातय एव

\begin{quote}
{\qt श्वपाकश्च तुरुष्कश्च हूणो यवन इत्यपि~।\\
लोकबाह्यस्तु यो वाजिगवाश्याचारवर्जितः~॥

कोमगिरेर्दक्षभागे मरुदेशात्तथोत्तरे~।\\
हूणदेशः समाख्यातः शूरास्तत्र रमन्ति च~॥}
\end{quote}

\noindent
इति शब्दार्थचिन्तामणौ~॥ 

एवं च रघुवंशे हूणवर्णनेऽपि कथमर्वाचीनतेति विचारयन्तु {\qt सर्व \textendash\ मर्वाचीनमेव} इति मननाग्रहग्रहानाविष्टाः सुधियः~॥

{\qt उतत्वः प्ष्यन्न ददर्श वाचम्} इति श्रुतौ चाक्षुषशानविषयत्वं वाचो लिपिसत्त्व एव संगच्छते~। पद्मपुराणेऽपि शकुन्तलोपाख्याने ५ अध्याये मुद्रायामक्षरमयच्चिह्नस्यैव वर्णनाच्च~॥

\noindent
\rule{1\linewidth}{0.5pt}\\

\noindent
सूत्रे न्यायभाष्यकृतोक्तम्~॥ अत एव {\qt ऋषिर्दर्शनात्} इति निरुक्ते {\qt आाख्या प्रवचनात्} ( १~। १ ) इति पूर्वमीमांसायां चोभयोपादानम्~॥
\end{multicols}

\fancyhead[CE,CO]{पाणिनिः~।}
\fancyhead[RO,LE]{\thepage}
\cfoot{}
\newpage
%%%%%%%%%%%%%%%%%%%%%%%%%%%%%%%%%%%%%%%%%%%%%%%%%%%%%
\renewcommand{\thepage}{\devanagarinumeral{page}}
\setcounter{page}{7}
% पाणिनिः~। ७

\begin{multicols}{2}
तत्रमूलभूत१सूत्रोपज्ञाता २नन्दमहाराजराजधानीपाटलि \textendash\ पुत्र \textendash\ वास्तव्य \textendash\ शंकरस्वामितनय \textendash\ पूर्वोत्तरमीमांसावृत्तिकार \textendash\

\noindent
\rule{1\linewidth}{0.5pt}\\

"अनेनैव व्याकरणोपज्ञात्रा भगवता पाणिनिनाचार्येण पाताल \textendash\ विजयापरनामकं जाम्बवतीविजयनामतं महाकाव्यं व्यरचि~। अत एव हारावल्यां सूक्तिमुक्तावत्यां च लिखितं राजशेखरीयं \textendash\

\begin{quote}
{\qt स्वस्ति पाणिनये तस्मै यस्य रुद्रप्रसादतः~।\\
आदौ व्याकरणं, काव्यमनु जाम्बवतीजयम्~॥}
\end{quote}

\noindent
इति पद्यम् \textendash\

\begin{quote}
{\qt पञ्चविंशतिसंयुक्तैरेकादशसमाशतैः~।\\
विक्रमात्समतिक्रान्तैः प्रावृषीदं समर्थितम्~॥}
\end{quote}

इति समाप्तिपद्यसूचितनिर्माणकाले \textendash\ ११२५ वि. सं. ( इ. स. १०४८ ) \textendash\ न रुद्रटालंकारव्याख्यात्रा नेमिसाधुना \textendash\ महाकवी \textendash\ नामप्ययशब्दपातदर्शनात् तन्निरासादरसूचनाय पुनरभियोगः \textendash\ तथाहि \textendash\ पाणिनेः पातालविजये महाकाव्ये {\qt संध्यावधूं गृह्य} करेण~। भानुः{\qt इत्यत्र गृह्य} क्त्वो ल्यबादेशः~। तथा तस्यैव कवेः \textendash\

\begin{quote}
{\qt गतेऽर्धरात्रे परिमन्दमन्दं गर्जन्ति यत् प्रावृषि कालमेघाः~।\\
अपश्यती वत्समिवेन्दुबिम्बं तच्छर्वरी गौरिव हुंकरोति~॥}
\end{quote}

इत्यत्र पश्यतीः इदं लुप्तनकारं पदम् इत्युक्तम्~॥

रायमुकुटेन \textendash\

\begin{quote}
{\qt पयःपृषन्तिभिः पृक्ता वान्ति वाताः शनैः शनैः~।}
\end{quote}

इत्यपि पाणिनीयत्वेनोपन्यस्तम्~॥

शार्ङ्गधरपद्धतावपि \textendash\

\begin{quote}
{\qt उपोढरागेण विलोलतारकं तथा गृहीतं शशिना निशामुखम्~।\\
यथा समस्तं तिमिरांशुकं तया पुरोsपि रागाद् गलितं न लक्षितम्~॥\\

क्षपाः क्षामीकृत्य प्रसभमपहृत्याभ्बु सरितां \\
प्रताप्योर्वी कृत्स्नां तरुगहनमुच्छोष्य सकलम्~।\\
क्वसंप्रत्युष्णांशुर्गत द्दतिसमालोकनपरा \textendash\ \\
स्वडिद्दीपाठोक्ता दिशि दिशि चरन्तीव जलदाः~॥}
\end{quote}

इति श्लोकद्वयं पाणिनीयत्वेन विन्यस्तम् ~॥ 

सदुक्तिकर्णामृतेऽपि \textendash\ 

\begin{quote}
{\mbh गतौ गिरेः शीतलकन्दरस्थः पारावतो मन्मथचाटुदक्षः~।\\
घर्मालसाङ्गीं मधुरापि कूजन् संवीजते पक्षपुटेन कान्ताम्~॥~५~।~२१~॥}
\end{quote}

\noindent
\rule{1\linewidth}{0.5pt}\\

1 अनयोर्नैवापशब्दत्वमिति महाभाष्यपस्पशाह्निकटिप्पण्याम \textendash\ स्माभिः प्रकाशितमेवेति गवेषणीयम्~॥

2 यदि दीर्घसमासदर्शनेनैवास्मिन् काव्ये नवीनत्वं मन्यते चेत् तर्हि {\qt विभक्तिसमीप \textendash\ जानपदकुण्ड \textendash\ बुञ्छण \textendash\ } इत्यादि सूत्रेष्वपि दीर्घसमासस्य दर्शनेन तत्रापि नजीनत्वमेव स्याद्~॥ रामायण \textendash\ भारतादिष्वपि शृङ्गारादिवर्णनेन तेषामपि प्राचीनत्वं नैव स्यात् नवीनत्वमेव स्यात्~॥ किंव सुसूक्ष्मजटकेशेन सुनुताजिनवाससा इत्यादिभाष्योदाहृतेष्वपि नवीनत्वमेव स्यात्~॥ सौमित्रि \textendash\ वात्मीकिशब्दयोः गहादि ( ४~। २~। १३८ ) गणे सीतान्वेषणशब्दस्य {\qt अधिकृत्य कृते गन्थे ४~। ३~। ८७} इत्यधिकारे इन्द्रजननादि ( ४~। ३८८ ) गणे पाठस्योपलम्भात् कौसल्याकैकेयीसुमित्रापदसाधनोपायस्य सूत्रे गणे च विद्यमानत्वात्~। एति जीवन्तमानन्दः इतिरामा \textendash\

\columnbreak

\noindent
भगवदुपवर्षसहोदरवर्षोपाध्यायाधीतशास्त्रः~।

\noindent
\rule{1\linewidth}{0.5pt}\\

\begin{quote}
{\mbh उद्बुद्धेभ्यः सुदूरं घनजनिततमःपूरितेषु द्रुमेषु\\
प्रोद्ग्रीवं पश्य पादद्वयनमितभुवः श्रेणयः फेरवाणाम्~।\\
उत्कालोकैः स्फुरद्भिर्निजवदनदरीसर्पिभिर्वीक्षितेभ्य \textendash\ \\
श्च्योतत्सान्द्रं वसाम्भः कुथितशववपुर्मण्डलेभ्यः पिबन्ति~॥~५~।~३६३~॥}
\end{quote}

\noindent
इत्यादिश्लोकाष्टकं पाणिनीयत्वेनोपन्यस्तम्~॥ 

शर्मण्यदेशनिवासिना आफ्रेचमहाशयेनापि सदुक्तिकर्णामृत स्थम् \textendash\

\begin{quote}
{\qt सुबन्धौ भक्तिर्नः क इह रघुकारे न रमते\\
धृतिर्दाक्षीपुन्ने हरति हरिचन्द्रोऽपि ह्वदयम्~।\\
विशुद्धोक्तिः शूरः प्रकृतिमधुरा भारविगिर \textendash\ \\
स्तथाप्यन्तर्मोदं कमपि भवभूतिर्वितनुते~॥}
\end{quote}

इति श्लोकमुपन्यस्य {\qt तत्र पठितो दाक्षीपुत्रो व्याकरणाचार्य एव} इति निश्चितम्~। अन्यस्य कस्यापि पाणिनेग्रन्थस्यास्मरणाच्च तद्दृढीकृतम्~॥ एवमन्येनापि शर्मण्यदेशीयेन पण्डित \textendash\ पिश्चल महोदयेन सदुरक्तिकर्णामृत \textendash\ Z. D. M. G. XXXIX 319 एष एवार्थः समर्थितः~॥

ङॉ. भाण्डारकरमहोदयस्य तु नैवैषोऽर्थ संमतो येन २तादृशो ( ओजोगुणगौडीरीतिविशिष्टः ) लेखो न कदापि पाणिनेः स्यादिति हि तस्य सिद्धान्तः~॥ सर्वोऽप्येषोऽनुवादः डॉ. पीटर पीटर्सन्महाशयैः सुभाषितावलिभूमिकायां पाणिनिपदव्याख्र्याने लिखितस्यो \textendash\ द्धृतोऽस्माभिः~॥

\begin{quote}
{\qt १ अल्पाक्षरमसंदिग्धं सारवद् विश्वतोमुखम्~।\\
अस्तोभमनवद्यं च सूत्रं सूत्रविदो विदुः~॥ \\
इति विष्णुधर्मोत्तरे पराशरोपपुराणे च~॥}
\end{quote}

२ एतन्नन्दमहाराजसमानकालकत्वमेव पाणिनेः भारतीय \textendash\ प्राचीनवृत्तप्रदर्शक ( इंडियन एंटिक्वेरी. VII \textendash\ IV ) चतुर्थखण्डेऽपि १० २ तमे पृष्ठे वर्णितमस्ति~॥

नन्दमहाराजशासनप्रारम्भसमयस्तु \textendash\ 

\begin{quote}
{\qt शतेषु षट्सु सार्धेषु त्र्यधिकेषु च भूतले~।\\
कलेर्गतेषु वर्षाणामभवन् ३कुरुपाण्डवाः~॥~ ( १~।~५१ ) ~॥}
\end{quote}

\noindent
\rule{1\linewidth}{0.5pt}\\

\noindent
यणीययुद्धकाण्डे १२९~। २ भरतवाक्यस्य अनुदात्तङितः{\qt सार्वशातुके यक्} तं सूत्रभाष्य उपन्यस्तत्वाच्च वाल्मीकीयरामायणस्य न नव्य \textendash\ त्वम्~। {\qt पश्य वानरसैन्येsस्मिन् एतदप्यस्य कापेयम्} इति भाष्योदाहरणाच्च कपिपर्यायवानराणामपि लङ्कायां युद्धकर्तृत्वं सत्यमेव~।

3 सर्वेषां कुरुपाण्डवानामेकस्मिन्नेव वर्षे जन्मासंभवात् अभवन् कुरुपाण्डवाः इत्यत्र {\qt योद्धारः} इत्यध्याहृत्य योद्धृत्वेन युद्धकाल एव सूच्यते~। अध्याहारमन्तरेणैव सुबोधतायै तु तस्य स्थाने {\qt अभवत् कौरुपाण्डवः} इति शोधितव्यं सुधीभिः~॥ कुरवः पाण्डवाश्च योद्धारो यस्मिन् संग्रामे स कौरुपाण्डवः संग्राम इत्येवं व्याख्येयम्~। दाधिमथाः~॥
\end{multicols}


\newpage
% ८ पाणिनिः~। 

\begin{multicols}{2}
\begin{quote}
{\qt आसन् मघासु मुनयः शासति पृथ्वीं युधिष्ठिरे नृपतौ~।\\
षट्द्विकपञ्चद्वि २५२६ युतः शककालस्तस्य राज्यस्य~॥}
\end{quote}

\noindent
इति बृहत्संहिताराजतरङ्गिणीतः \textendash\

\begin{quote}
{\qt शतं शतं तु वर्षाणामेकैकस्मिन् महर्षयः~।\\
नक्षत्रे निवसन्त्येते ससाध्वीका महातपाः~॥}
\end{quote}

\noindent
इति वाराहीसंहिताव्याख्याभट्टोत्पलीधृतकश्यपवाक्यतः \textendash\ 

इत्येते बार्हद्रथा भूपतयो वर्षसहस्रमेकं १००० भविष्यन्ति~॥ ४~। २३~। १३~॥ 

इत्येतेऽष्टत्रिंशदुत्तरमब्दशतं १३८ पञ्च प्रद्योताः पृथिवीं भोक्ष्यन्ति~॥~४~।~२४~।~८~॥ 

इत्येते शैशुनाभा भूपालास्त्रीणि वर्षशतानि द्विषष्ट्यधिकानि ३६२ भविष्यन्ति~॥~४~।~२४~।~१९~॥

महानन्दिनस्ततः शूद्रागर्भोद्भवोऽतिलुब्धोऽतिबलो महापद्मनामा नन्दः परशुराम इवापरोऽखिलक्षत्रान्तकारी भविष्यति~।~४~।~२४~।~२०~॥

\begin{quote}
{\mbh यावत्परीक्षितो जन्म यावन्नन्दाभिषेचनम्~।\\
एतद् वर्षसहस्रं तु ज्ञेयं पञ्चशतोत्तरम्~॥~४~।~२४~।~१०४~॥

ते तु पारीक्षिते काले मघास्वासन् द्विजोत्तम~॥\\
तदा प्रवृत्तश्च कलिद्वादशाब्दशतात्मकः~॥~४~।~२४~।~१०७~॥

यदेव भगवानू विष्णोरंशो यातो दिवं द्विज~।\\
वसुदेवकुलोन्द्भूतस्तदैवात्रागतः कलिः~॥~४~।~२४~।~१०८~॥

यावत्स पादपद्माभ्यां पस्पर्शेमां वसुंधराम्~।\\
तावत्पृथ्वीपरिष्वङ्गे समर्थौ नाभवत्कलिः~॥~४~।~२४~।~१०९~॥

गते सनातनस्यांशे विष्णोस्तत्र भुवो दिवम्~।\\
तत्याज सानुजो राज्यं धर्मपुत्रो युधिष्ठिरः~॥~४~।~२४~।~११०~॥

निमित्तानि च दृष्ट्वा स विपरीतानि पाण्डवः~।\\
याते कृष्णे चकाराथ सोऽभिषेकं परीक्षितः~॥~४~।~२४~।~१११~॥}
\end{quote}

\noindent
\rule{1\linewidth}{0.5pt}\\

1 यद्यपि र्वेकटेश्वरयंत्रालयमुद्रिते पुस्तके {\qt अष्टशतम्} इत्युपलभ्यते~। तथापि टीकयोः सार्धसहस्रवर्षोक्तिसामञ्जस्यायैवं पाठः स्थापितः~॥ तथा पाठस्य सत्यत्वे तु कलिगताब्दाः २८५३ भवन्ति~॥

2 अनेन नैव नन्दसमये पूर्वाषाढायां सप्तर्षिस्थितिः प्रतिपा्यते किंतु नन्दात् परमागमिष्यति सप्तर्षिपूर्वाषाढाश्रयणे कलिवृद्धिर्भवि \textendash\ ष्यतीत्येव प्रतिपाद्यते~। अधुना स्वातौ सप्तर्षिसंस्थितितः परं पूर्वाषाढायां सप्तर्षिस्थितौ भविष्यन्त्यां पञ्चमशताब्द्यां कलिवृद्धिसूचकमेवैतद्वाक्यम् इति सर्वं समंजसम्~॥ इति पञ्चाशतोत्तरम् इति पाठशोधकयोर्विष्णुपुराणटीकाकर्त्रोरभिप्रायः~॥ {\qt पञ्चाशदुत्तरम्}; इत्युपलभ्यमानपाठपर्यालोचनेन तु नन्दसमय एव सप्तर्षीणां पूर्वीषाढाश्रयणं निराबाधमेव~। यतो मधायां सप्तर्षिस्थितिसमये युधिष्ठिरराज्यं \textendash\ परीक्षिद्राज्यं चासीत्~। तत एकादश्यां शताब्द्यां पूर्वाषाढानक्षत्रे सप्तर्षिस्थितौ नंन्दाभिषेचनमिति क्लेशमर्थपरिवर्तनञ्च विनैव सुबोधतेति ६५ ३+ १०५०+१७०३ कलिगताब्दैषु नन्दराज्यमासीत् इति केचित्~॥ तन्न मनोरमम्~। प्रतिपदोक्तानां बार्हद्रथराज्याब्दानाम् १०० ०+ प्रद्योतराज्याब्दानाम् १ ३८ शैशुनाभराज्याब्दानाम् ३६२ योगेन तु १५०० भवन्ति तत्र कौरुर्पाण्डवयुद्धकलिगताब्दानां ६५३ राज \textendash\ तरङ्गिण्युक्तानंं भोगेन २१५३ कलिगताब्दा नन्दराज्ये भवन्ति~।

\columnbreak

\begin{quote}
{\mbh प्रयास्यन्ति यदा चैते पूर्वाषाढां महर्षयः~।\\
तदा नन्दात्प्रभृत्येष कलिर्वृद्धिं गमिष्यति~॥~४~।~२४~।~११२~॥}
\end{quote}

इत्येवं विष्णुपुराणतश्च ६५३+१०००+१३८+३६२$=$२१५३ कलिगताब्दा भवन्ति~॥

३ पुष्पपुरं पाटलिपुत्रकम् इति त्रिकाण्डशेष \textendash\ केशवशब्द \textendash\ कल्पद्रुमाः~॥ ८ {\qt पाटलिपुत्रं कुमुमपुरम्} इति हैमनाममाला~॥

अस्य पाटना ( बिहार ) मध्यरेखातोंऽशाः पूर्वे ९∘. २२ घटी १. पलानि ३३. विपलानि ४०~॥ अक्षांशा उत्तरे \textendash\ २५∘. ३७

नवम इ. स. शतकोत्पन्नराजशेखरविरचितकाव्यमीमांसायाम् \textendash\ {\qt श्रूयते च पाटलिपुत्रे शास्त्रकारपरीक्षा, अत्रोपवर्षवर्षाविहपाणिनि \textendash\ पिङ्गलाविहव्याडिः~। वररुचिपतञ्जली इह परीक्षिताः ख्यातिमुप \textendash\ जग्मुः~।} इति समुपलभ्यते~। 

४ {\qt उपवर्षौ हलभूतिः कृतकोटिरयाचितः} ति त्रिकाण्डशेष \textendash\ केशवौ~॥ अर्यं चोपवर्षोपाध्याग् उत्तरमीमांसाभाष्यकारशंकर \textendash\ भगवत्पादैः {\qt एक आत्मनः शरीरे भावात्} ( उ. मी. सू० ३~। ३~। ५३ ) इति सूत्रभाष्ये \textendash\ {\qt इत एवाकृष्याचार्येण शबरस्वामिना प्रमाणलक्षणे वर्णितम्} इत्येवं स्मृतनाम्ना पूर्वमीमांसाभाष्यकारेण शबरस्वामिना~। तत्र तत्र {\qt वृत्तिकारस्तु अन्यथेमं ग्रन्थं वर्णयांचकार इति भगवानुपवर्षः} इत्येवं भगवत्पदपूर्वकं स्मर्यते~॥ स्मर्यते चोत्तरमीमांसाभाष्यकृद्भिर्भगवच्छंकरपादैरपि ( {\qt एक आत्मनः शरीरे भावाद्} ) ( उ. मी. स ३~।~३~।~५३ ) इति सूत्रभाष्ये \textendash\ अत एव च भगवतोपवर्षेण प्रथमे तन्त्रे आत्माभिधानप्रसक्तौ {\qt शारीरिके वक्ष्यामः इत्युद्धारः कृतः} इति ग्रन्थेन~॥ त्रिकाण्डमण्डने५पि {\qt उपवर्षादिवाक्यतः} इति पुण्यपत्तनमुद्रितसंस्काररत्नमाला ४३८ पृष्ठे~॥

\noindent
\rule{1\linewidth}{0.5pt}\\

\noindent
इति निर्विवादः पन्थाः~॥ राजतरङ्गिगीस्थसार्धपदघटकस्यार्थपदस्य षटूसंख्यासंबन्धित्वनिर्णये तु ६०० \textendash\ +३००+३$=$९०३ कलिगताब्दे युद्धमभूदिति मते तु २४०३ कलिगताब्दे नन्दराज्यमासीदिति नव्यत्वाभिनिवेशिनां मतम्~॥ प्रतिपदोक्तानां बाहंद्रथराज्याब्दानां १००० प्रद्योतराजाब्दानां १३८+ शैशुनाभराज्याव्दानां ३६२ योगेन १५ ०० कलिगताब्दा भवन्ति~। अत्रैव नन्दराज्ये पूर्वाषाढायां सप्तर्षिस्थितिर्निराबाधेति जानीमः~। {\qt आसन् मघासु मुनयः शासति पृथ्वीं युधिष्ठिरे नृपतौ}~। ते तु पारिक्षिते काले मघास्वासन् द्विजो \textendash\ त्तम~। तदा प्रवृत्तश्च कलिर्द्वादशाब्दशतात्मकः~। प्रयास्यन्ति यदा चैते पूर्वाषाढां महर्षयः~। तदा नन्दात्प्रभूत्येव कलिर्वृद्धिं गमिष्यति~। इत्येतेषां संवादेन सप्तर्षीणां मघाप्रवेशासन्नकाल एव कलियुद्धयोरा \textendash\ रम्भः~। पूर्वाषाढासमाप्त्यासन्नकाल एव नन्दराज्यम् इत्येवं निश्चये प्रतिनक्षत्रं १६०० मासस्थितिकल्पनायां नैव विरोधः~। शतं सत्र्यंशवर्षाणामेकैकस्मिन्महर्षयः~। नक्षत्रे निवसन्त्येते ससाध्वीका महातपाः इत्येवं पाठेन अर्धतो न्यूनप्राप्तौ त्यज्यतेsधिकप्राप्तौ त्वेकमेव गण्यते इति गणित्तविदां नियमेन शतं शतं च इति पाठेनापि १६०० मासाम् सप्तर्षीणामेकनक्षत्रयोग इति कल्पनायां सर्वं समञ्जसम् इति दाधिमथाः~॥
\end{multicols}

\fancyhead[CO]{पाणिनिः~। वररुचिः~।}
\fancyhead[CE]{पतञ्जलिः~।}
\fancyhead[RO,LE]{\thepage}
\cfoot{}
\newpage
%%%%%%%%%%%%%%%%%%%%%%%%%%%%%%%%%%%%%%%%%%%%%%%%%%%%%
\renewcommand{\thepage}{\devanagarinumeral{page}}
\setcounter{page}{9}

% पाणिनिः~। वररुचिः~। ९

\begin{multicols}{2}
१सलातुरग्रामाभिजनः २शलङ्कुतनयो ३दाक्षीपुत्रो ४भगवान् ५पाणिनिगोत्र आहिकनामा मुनिर्गोत्राश्रयनाम्नैव प्रसिद्धो विदित एव सर्वेषाम्~॥

ततश्च तस्यैव वर्षोपाध्यायस्य प्रधानान्तेवासी कौशा \textendash\

\noindent
\rule{1\linewidth}{0.5pt}\\

१ अयं व स ( श ) लातुरनामा ग्रामोऽधुना लाहुर नाम्ना प्रसिद्धः पञ्चापदेशे पेशावरमण्डले सिन्धुनदीतः पश्चिमोत्तरे देशे अटकस्थेशनतः पक्चिमायां योजनत्रयान्तरे ( १५ भैल ) वर्तमानात् ~। उत्खण्ड ( ओहिद ) ग्रामतः पश्चिमोत्तरायां ३+१\textbackslash३ मीलान्तरे वर्तते~। शलातुरशब्दस्य कालान्तरेण वातपित्तकफविकृत्योच्चारणविकारैः सलातुर$=$हलाथुर$=$हलाहुर$=$लाहुर इत्येवं व्यत्ययोऽजनि इति ह्यूनसाङ्गकथितं केनिङ्गहैममहोदयेन तत्र सलातुरे खिस्ताब्द \textendash\ प्रवृत्ते ३५० वर्षपूर्वं पाणिनिजीवनं च वर्णितम्~।

एकेनास्मन्मित्रवर्येण तद्देशीययवनेन तु \textendash\ {\qt दशमशताब्द्यां तद्देशे लाहोरनरपतिजयपालकटकस्थित्या लाहोरनाम्ना प्रसिद्धिं गतः} इति वर्णयांचक्रे तत्र अट्टकनाम्ना प्रसिद्धस्य स्थानस्य मध्यरेखातो रेखांशाः पश्चिमायाम् ३ ∘ ३३ अक्षांशाः ३३ ∘ ५३~। ~। वर्तन्ते~॥

२ पाणिनिस्त्वाहिको दाक्षीपुत्रः शालङ्किपाणिनौ~। शालातुरीयः इति त्रिकाण्डशेषः~॥ मातुरीय ( ? ) स्तु पाणिनिः~। सालातुरीयो दाक्षेयः शालङ्किः पाणिनाहिकौ~। शिवपर्यायभक्तः इति केशवः~॥

पैलादि ( २~। ४~। ५९ ) गणे {\qt शालङ्किः} इत्युपलभ्यते~। तेन नढादि ( ४~। १~। ९९ ) गणे {\qt शलङ्कु शलङ्कं च इति गणसूत्रमिव} बाह्लादि ( ४~। १~। ९६ ) गणेsपि ताद्वशं गणसूत्रमनुमीयते~। तेन शालङ्कि \textendash\ शालङ्कायनौ सिद्धौ~॥

३ {\qt दाक्षीपुत्रस्य पाणिनेः} घुसंज्ञा ( १~।~१~।~८ ) सूत्रे भाष्यकारः~॥

४ {\qt कथं पुनरिदं भगवतः पाणिनेराचार्यस्य लक्षणं प्रवृत्तम्} इति प्रथमवार्तिकावतरणे भाष्यकृत्~॥

५ पाणिनिगोत्राणाम् \textendash\ {\qt भार्गव \textendash\ च्यावन \textendash\ आप्नवान \textendash\ और्व \textendash\ जाम \textendash\ दग्न्य} इति पञ्च प्रवरा भवन्ति~॥ इति बौधायनादिसूत्रमूलकभट्ट \textendash\ गोपीनाथदीक्षितविरचितसंस्काररत्नमालापुण्यपत्तनीयानन्दाश्रममुद्रित पुस्तके समुपलभ्यते~॥

६ इयं च कौशाम्बीनगरी रामायणे बालकाण्डे दाशरथिराम \textendash\ चन्द्रेण विश्वामित्रयज्ञविघ्नकर्तुषु रात्रिञ्चरेषु नाशितेषु पश्चाज्जनकपुरं

\noindent
\rule{1\linewidth}{0.5pt}\\

I अस्मिन् पाणिनिविषये पाश्चात्यविद्याविशारदानां विद्वद्वराणां वादप्रतिवादस्तु {\qt इंडियन एंटिक्केर} I21;102.281 310. ३६२ V \textendash\ 49. 245, 345. VI \textendash\ 107 \textendash\ 306. 317, IX \textendash\ 80. 81. 251. 305. 307. 318, X \textendash\ 77, 76, XI \textendash\ 123, XII \textendash\ 205 \textendash\ 6, XV \textendash\ 241 XVI \textendash\ 178, XXII 223 पुस्तकभागेषु पृष्ठेषु द्रष्टव्यः~॥

2 प्र० पा०

\columnbreak

\noindent
म्बीवास्तव्य \textendash\ सोमदत्तद्विजन्मात्मजन्मा वसुदत्तागर्भज पुष्पदन्तगणावतारो भगवान् कात्यं७ \textendash\ कात्यायन८गोत्र९नामा श्रुतधरापराभिधेयवररुचि१०नामा मुनिः पाणिन्युपज्ञातसूत्र \textendash\ न्यूनतापूरणमिव वार्तिक११रूपेण कृतवान्~॥

\noindent
\rule{1\linewidth}{0.5pt}\\

\noindent
निनीषुणा विश्वामित्रेण वर्णितैव भवेदन्या वेति न निश्चयः~। यतो नद्यादि ४ \textendash\ २ \textendash\ ९७ \textendash\ गणे कौशाम्बी वनकौशाम्बीत्युभयपाठात्~। {\qt कौशाम्बी वत्सपत्तनम्} इति त्रिकाण्डशेषहैमनाममालादिकोश \textendash\ प्रामाण्येन वत्सराजधानी सर्वेषु कथासरित्सागर \textendash\ बृहत्कथामञ्जरी \textendash\ रत्नावली \textendash\ प्रियदर्शिका \textendash\ प्रतिज्ञायौगन्धरायण \textendash\ स्वप्नवासवदत्ताप्रभृतिषु ग्रन्थेषु वर्णिता~॥ अधुना च कनिंगहमप्रभृतीनां पाश्चात्यानां विदुषां निश्चयोऽस्ति यत् सेयं कौशाम्बी प्रयागक्षेत्रत उपरिभागे यमुनातीर~। व षड्योजनान्तरे वर्तते~। तस्याश्च रेखांशा मध्यरेखातः पूर्वस्यां दिशि ६ ∘ २७ अक्षांशाः २५९ ∘ २० मिता वर्तन्ते~। प्रसिद्ध नाम च कोसम इति ख्यातमस्ति~। इति~।

७ {\qt प्रोवाच भगवान् कात्यः} इति~। {\qt आतोऽनुपसर्गे कः} इति ३, २, ३, सूत्रे भाष्यम्~॥

८ {\qt स्मादिविधिः पुरान्तो यद्यविशेषेण किं कृतं भवति~। न स्म पुराद्यतन इति ब्रुवता कात्यायनेनेह~॥ } इति श्लोकव्याख्यायां {\qt किं वार्तिककारः प्रतिषेधेन करोति} इति {\qt लट्स्मे} इति ३~। २~। ११५ सूत्रे भाष्यम्~॥

९ कात्यायनगोत्रस्य कपिगणे पाठः~। तेषामाङ्गिरसबार्हस्पत्यकापेयेति त्रिप्रवरत्वम्~।

१० {\qt मेधाजित् कात्यः कात्यायनश्च सः~। पुनर्वसुर्वररुचिः} इति त्रिकाण्डशेषः~॥

११ \begin{quote}
{\qt उक्तानुक्तदुरुक्तानां चिन्ता यत्र प्रवर्तते~।\\
तं ग्रन्थं वार्तिकं प्राहुर्वार्तिकज्ञा मनीषिणः~॥}
\end{quote} 

\begin{flushright}
पराशरोपपुराणम्~॥
\end{flushright}

अयं कात्यायन एव वार्तिककार इति तु लोकप्रसिद्धिमात्रम्~॥ वहवो वर्तिककाराः केचन वाक्यरूपवार्तिककाराः यथा \textendash\ क्रोष्ट्रीयाः १~। १~। ३ सूत्रे, भारद्वाजीयाः स्थानिवत्सूत्रे, सौनागाः २~। २~। १८ सूत्रे इत्यादयो नामनिर्दिष्टाः~। केचन नैव नामभेदनिर्दिष्टाः, अत एव तेषां न पौनरुक्त्यवैयर्थ्यम्~। ~। यथा {\qt चङ्परनिर्ह्रासे स्थानिवद्भावप्रतिषेधवार्तिकमुक्त्वाऽपि }णौ \textendash\

\noindent
\rule{1\linewidth}{0.5pt}\\

2 \begin{quote}
{\qt सप्त्नवत्यधीकेष्वेकादशसु शतेष्वतीतेषु\\
वर्षाणां विक्रमतो गणरत्नमहोदधिर्विहितः~॥}
\end{quote}

इति बोधितसमयगणरत्नमहोदधौ \textendash\ {\qt शालातुरीय \textendash\ शकटाङ्गज \textendash\ } इति पाणिनिं स्मरति वर्धमानः~॥

3 वार्तिकेष्वपि सूत्रत्वव्यवहारः वार्तिककारेष्वपि आचार्यत्व \textendash\ व्यवहारः {\qt आचार्याः सूत्राणि कृत्वा न निवर्तयन्ति} इति पस्पशायां भाष्यकृता कृतो न विस्मरणीयः~॥
\end{multicols}

\newpage
% १० पतञ्जलिः~। 

\begin{multicols}{2}
ततश्च तस्यैव वर्षोपाध्यायस्य सर्वच्छात्रोत्तमवररुचितो~। निकृष्ट इन्द्रदत्तत उत्कृष्टो वीतसी ( १वेतसी ) तीर्थवास्त \textendash\ व्यकरम्बवाडवनन्दनो नो २नन्दिनीतनयो व्याडिगोत्र आचार्यः

\noindent
\rule{1\linewidth}{0.5pt}\\

\noindent
चङ्युपधायाः \textendash\ इत्यत्र {\qt णेर्णिच्युपसंख्यानम्}इति वर्तिकरचनम्~॥ {\qt पूर्वत्रासिद्धे न स्थानिवत्, तस्य दोषः संयोगा \textendash\ दिलोपलत्वणत्वेषु}इत्यनेन काक्यर्थम्{\qt इत्यत्र कलोपप्राप्ति \textendash\ वारणे सिद्धेऽपि संयोगादिलोपे च यणः प्रतिषेधो वाच्यः} इत्येतत्करणम्~॥

अन्य एवासन् केचन श्लोकवार्तिककाराः~॥

अत एवास्माभिरपि वार्तिकगणनायां नैवापारि इति तदङ्कदाने प्रतिमुद्गणं व्यत्ययेपि नास्माकं दोषः~॥

अनेन कात्यायनेन भ्राजाख्याः श्लोका अपि रचिताः यदन्तर्गत एव {\qt यस्तु प्रयुङ्के कुशलो विशेषे \textendash\ } इत्ययं श्लोको व्याकरण \textendash\ प्रयोजनवर्णनावसरे भाष्यकृतोपन्यस्तः~॥

अन्येऽपि बहवो ग्रन्थाः पुस्तक \textendash\ तन्निर्मातृनामसंग्रहसूचीपत्रे केटलागस् कटलोगरम् 

\begin{tabular}{c c}
कात्यायनीयत्वेन प्रकाशिताः & वररुचीयत्वेन प्रकाशिताः\\
इष्टिपद्धति & अष्टाध्यार्थावृत्ति \\
कर्मप्रदीप & एकाक्षरकोषः \\
कारिका & कारकचक्र \\
गृह्यकारिका & कारिका \\
गृह्यपरिशिष्ट & चैत्रकुटी व्या. \\
चण्डीविधान & दशगणकारिका व्या. \\
ज्योतिष्टोम & पत्रकौमुदी \\
त्रिकण्डिकासूत्र & प्रयोगविवेक व्या. \\
नवकण्डिकाश्राद्धसूत्र & विधिसंग्रह व्या. \\
परिशिष्ट & प्राकृतप्रकाश व्या. \\
" पद्धति & फुल्लसूत्र \\
पशुबन्धसूत्र & योगशत \\
आकृतमञ्जरी & राक्षसकाव्य \\
प्रायश्चित्त & राजनीति \\
भाषिकसूत्र & लिङ्गविशेषविधि \\
भ्राजश्लोक & वररुचिकाव्य \\
मू ( मौ ) ल्याध्याय & वादतरङ्गिणी \\
रुद्रविधान & वार्तिकपाठ \\
वार्तिकपाठ & विवेकसंग्रह \\
शान्ति & शब्दलक्षण \\
" विधान & श्रुतबोध 
\end{tabular}

\noindent
\rule{1\linewidth}{0.5pt}\\

1 रघुकारः कालिदासः इति त्रिकाण्डशेषः~।

2 मेघदूत \textendash\ विक्रमोर्वशी \textendash\ मालविकाग्निमित्रज्योतिर्विदाभरणादिग्र \textendash\ न्थानां निर्मातारः कालिदासास्त्वितो भिन्ना एवेति सर्वमनवद्यम्~॥ कुमारसंभवनिर्माता कालिदासस्तु रघुवंशप्रणेतुरभिन्नोsनन्यो वेति न निश्चितम्~॥ शकुन्तलायां {\qt रसभावदीक्षागुरोर्विक्रमादित्यस्य सभायाम्} इति पाठस्तु कलिकातामुद्रितपुस्तक एवोपलभ्यते~। न युरोपमुम्बय्यादिमुद्रितपुस्तकेषु इति नैव प्रामाण्यसाधकः~। प्रामा \textendash\

\columnbreak

\noindent
४खपितृष्वसेयपाणिन्युपज्ञातशास्त्रे व्याख्यानभूतमिव लक्षश्लो \textendash\ कात्मकं संग्रहनामानं ग्रन्थं निर्मितवान्~॥ एष च ग्रन्थो नागे \textendash\ शादिभिरपि नोपलब्धः~।

\noindent
\rule{1\linewidth}{0.5pt}\\

\begin{tabular}{c c}
शिक्षा & समासपटल \\
शुल्वसूत्र & \\
स्नानविधिसूत्र & \\
शुक़यजुःप्रातिशाख्यम् & \\
श्रौत्रसूत्नम् & 
\end{tabular}

अस्यैव वररुचेः समानसमयः कश्चन गोपालसदृशो जडबुद्धि. रपि श्रीकाशीनरेश \textendash\ भीमशुक्लमहाराजपुत्री \textendash\ महाविदुषी \textendash\ वास \textendash\ न्तीपतिः संभूय श्रीकालीप्रसादतः सकलकलाविद्यापारंगमो भूत्वा कालिदासनाम्ना प्रसिद्धिं प्राप्य 1रघुवंश \textendash\ शकुन्तलादिग्रन्थान्निर्माय \textendash\ समग्रमहीमण्डलमण्डनपण्डितवरमाननीयो भूत्वा गुणिगण \textendash\ गणनायां प्रथमगणनीयो बभूव इति इण्डियन अण्टिकेरी पुस्तक \textendash\ चतुर्थभागे १०३ पृष्ठेऽपि वर्णनं 2समुपलभ्यते~।

\begin{quote}
{\mbh १ अथ वेतसिकां गत्वा पितामहनिषेविताम्~।\\
अश्वमेधमवाप्नोति गच्छेदौशनसीं गतिम्~॥~३ ( वनपर्व ) ~।~८२~।~५६

अथ सुन्दरिकातीर्थं प्राप्य सिद्धनिषेवितम्~। \\
रूपस्य भागी भवति दृष्टमेतत्पुरातनैः~॥~३~।~८२~।~५७

ततो वै ( ब्राह्मणं ) गत्वा ब्रह्मचारी जितेन्द्रियः~।\\
पद्मवर्णेन यानेन ब्रह्मलोकं प्रपद्यते~।~३~।~८३~।~५८

ततस्तु नैमिषं गच्छेत्पुण्यं सिद्धनिषेवितम्~।\\
तत्र नित्यं निवसति ब्रह्मा देवगणैः सह~॥~३~।~८२~।~५९.}
\end{quote}

इति महाभारतसमवलोकनेन ज्ञायते \textendash\ नैमिषारण्य ( २७∘.२०. ५५, उत्तराक्षांशे, ४∘.४१.४०" मध्यरेखातः पूर्वदिगंशे ) क्षेत्रतः पश्चिमोत्तरे देश एव भवेदियं चेतसी इति विभावनीयम्~॥

२ व्याडिर्विन्ध्यस्थो नन्दिनीसुतः~। मेधावी च इति त्रिकाण्डशेषः~। 

३ व्धाडिगोत्रस्यात्रिगणे पाठात् तद्गोत्राणाम् आत्रेयार्थनानस \textendash\ श्यावाश्व इति त्रिप्रवरत्वं ज्ञेयम्~॥

\includegraphics[width=0.6\linewidth]{latex/b.JPG}

५ अयं लक्षश्लोकात्मको ग्रन्ध इत्यैतिह्यम्~॥ व्याडिकृता ग्रन्थाः \textendash\ त्रिकाण्डमण्डने \textendash\ व्याडि \textendash\ वासिष्ठवाक्यतः इत्युक्तेश्च~।

\noindent
\rule{1\linewidth}{0.5pt}\\\\
णिकत्वेऽपि न तेन विशेषणेन संवत्प्रवर्तकस्योज्जयिनीनरपतेरेव ग्रहणं सिद्ध्यति~।

3 {\qt व्यडो नाम ऋषिः,} तस्यापत्यं व्याडिः इति स्वागतादिगण \textendash\ रत्नमहोदधिः~॥ क्रोड्यादिगणे {\qt व्याडिः}~।

4 तथा चोक्तं भाष्यकृता {\qt उभयप्राप्तौ \textendash\ } इति सूत्रभाष्ये \textendash\ {\qt शोभना खलु दाक्षायणस्य संग्रहस्य कृतिः} इति~॥
\end{multicols}

\fancyhead[CO,CE]{महाभाष्यनिर्माता पतञ्जलिः~।}
\fancyhead[RO,LE]{\thepage}
\cfoot{}
\newpage
%%%%%%%%%%%%%%%%%%%%%%%%%%%%%%%%%%%%%%%%%%%%%%%%%%%%%
\renewcommand{\thepage}{\devanagarinumeral{page}}
\setcounter{page}{11}

% महाभाष्यनिर्माता पतञ्जलिः~। ११

ततश्च कुण्या१दिभिराचार्यै रचितासु २वृत्तिषु क्वचिन्न्यूनतामवलोक्य गो३नर्ददेशे भगवान् ४पतञ्जलिगोत्रजन्माश्रीशेषावतारत्वेन विख्यातः कश्चिन् महाप्रतिभाप्रभावो मुनिप्रवरो ५महाभाष्यं निरमात्~॥\\
\rule{1\linewidth}{0.5pt}\\

१ {\qt कुणिना प्राग्ग्रहणम् अन्येन प्राग्ग्रहणम् भाष्यकारस्तु
कुणिमतमेवाशिश्रियत्} इति {\qt एुङ् प्राचां देशे } इति प्रदीपे कैयटः~॥\\

२ {\qt मूर्धाभिषिक्तमिति~। सर्ववृत्तिषूदाहृतत्वादिति अचः परस्मिन्निति सूत्रे कैयटः~। }वृत्तिकारोक्तं सूत्रार्थमाह \textendash\ न खल्वपीति वृत्तिकारोक्तं दूषयति \textendash\ अकच्स्वरौ त्विति $=$न बहुव्रीहौ इति सूत्रे नागेशः~॥\\

३ अयं च गोनर्ददेशः कश्मीरेषु इति प्राञ्चः~॥ अयोध्याप्रान्ते ( मध्यरेखातः ६ ∘.१०{\qt पूर्वस्याम् अक्षांशाः २७ ∘.७} उत्तरस्याम् ) इति पौरस्त्याः~॥\\

४ अथ कपीनां त्र्यार्षेयःआङ्गिरस \textendash\ माहाय्यवोरुक्षय्येति \textendash\ उरुक्षय्यवद् \textendash\ महय्युवद् \textendash\ अङ्गिरोवदिति~। तरस्वान् तिलो विदुः शालुः पातञ्जलिः, भूयसीः, द्वन्द्वकीः, जलन्द्वः कपेरष्टविधाः प्रजाः इति संस्काररत्नमालायां सत्याषाढहिरण्यकेशिसूत्रम्~॥\\

५ {\qt सूत्रार्थो वर्ण्यते यत्र वाक्यैः सूत्रानुसारिभिः~। स्वपदानि च वर्ण्यन्ते भाष्यं भाष्यविदो विदुः~॥\\
पराशरोपपुराणम्~। अत्र च महाभाष्ये }चन्द्रगुप्तसभा पुष्पमित्रसभा पुष्पमित्रो यजते, याजका याजयन्ति इति लक्ष्यप्रदानेन पुष्पमित्रराज्ये तदनन्तरं वाऽस्य निर्माणं प्रतीयतै~। तद्राज्यप्रारन्भकालश्च \textendash\ 

\begin{quote}
{\qt महापद्मः ( नाम्ना नन्दः ) पुत्राश्चैकं वर्षशतमवनीपतयो भविष्यन्ति~४~।~२४~।~२५~॥

तेषामभावे मौर्याः पृथिवीं भोक्ष्यन्ति~॥~४~।~२४~।~२७~॥

एवमेते मौर्या दक्ष भूपतयो भविष्यन्ति \textendash\ अब्दशतं सप्तत्रिंशदुत्तरम्~॥~४~।~२४~।~३२~॥

त्तेषामन्ते पृथिवीं दश शृङ्गा भोक्ष्यन्ति~॥~४~।~२४~।~३३~॥

पुष्पमित्रः सेनापतिः स्वामिनं ( बृहद्रथनामानं ) हत्वा राज्यं करिष्यति तस्यात्मजोऽग्निमित्रः~॥~४~।~२४~।~३३~॥}
\end{quote}

इति विष्णुपुराणसमवलोकनेन ६५३+१५००+१००+१३७$=$२३९० कलिचतुर्विंशशताब्द्यां प्रतीयते~॥ पुष्पमित्रसमय एव, भग्निमिन्रसमय एव वा भाष्यनिर्माणम्~॥

\begin{center}
भाष्यकारसुगृह्रीतनामधेयपुष्पमित्रराजपर्यन्तं कलिग१ताब्दाः वायुपुराणोत्तरखण्ड\\
सप्तत्रिंशा३७ ध्यायप्रकाशिताः \textendash\

\vspace{5mm}
\begin{tabular}{m{10em} m{25em} m{10em}}
प्राचां मते & & अर्वाचीनत्वाभिनिवेशिनव्यभते\\
कलिगताब्दाः B. 0. & & कलिगताब्ब्दाः B. C.
\end{tabular}

\vspace{2mm}
\begin{tabular}{c c c}
शतेपु षदसु सार्धेषु व्यधिकेषु च भूतले & \multirow{2}{*}{$\big\}$} & राजतरङ्गिणी१~।~५१~॥\\
कलेर्गतेषु वर्षाणामभवन् कुरुपाण्डवाः & & \\
\end{tabular}

\vspace{2mm}
\begin{tabular}{m{5em} m{5em} m{25em} m{5em} m{5em}}
६५३ & २४४८ & {\mbh संग्रामे भारते तस्मिन् सहदेवो निपातितः~।} & ९०३ & २१९८\\
& & {\mbh सोमाधि ( पि ) स्तस्य तनयोराजर्षि स गिरिव्रजे~॥} & & \\
७११ & २३९० &{\mbh पञ्चाशतं तथाsष्टौ च समा राज्यमकारयत्~।} & ९६१ & २१४०\\
७७५ & २३२६ & {\mbh श्रुतश्रवाश्चतुःषष्टिसमास्त्तस्य सुतोऽभजत्~॥~२९२~॥ } & १०२५ & २०७६\\
८११ & २२९० & {\mbh अयुतायुस्तु षट्त्रिंशद् राज्यं वर्षाण्यकारयत्~।} १०६१ & २०४० \\
९११ & २१९० & {\mbh समाशतं निरमित्रो मही भुक्त्वा दिवं गतः~॥~२५३~॥} & ११६१ & १९४०\\
९६७ & २१६४ & {\mbh पञ्चाशतं समाः षट् च सुकृत्त ( नेत्र ) ; प्राप्तवान् महीम्~।} & १२१७& १८९४ \\
९९० & २१११&{\mbh त्रयोविशं बृहृत्कर्मा राज्यं वर्षाण्यकारयत्~॥~२९४~॥} & १२४० & १८६१
\end{tabular}
\end{center}

\noindent
\rule{1\linewidth}{0.5pt}\\

१ अन्तरे चैव संप्राप्ते कलिद्वापरयोरभूत्~। स्यमन्तपञ्चके युद्धं कुरुपाण्डवसेनयोः~। ( महाभारते आदिपर्वणि २ अध्याये कोकः १३ ) पर्यालोचनेन कलिप्रारम्भकाल एव भारतयुद्धप्रारम्भस्य निर्णयेन शतेषु षट्सु सार्धेपु व्यधिक्केषु च भूतले~। कलेर्गतेषु वर्षाणामभवन् कुरुपाण्डवाः~। इति राजतरङ्गिणीश्लोके वर्षपदस्य दिनरात्रिपरत्वस्य ( पू० मी० ६~। ७~। १३ ) ~॥ दीर्धसत्राधिकरणानुसारेण सप्त च वै शतानि विंशतिश्च संवत्सरस्याहोरात्राः इति च ब्राह्मणं विभागेन विभागेन, ( ४~। २७~। ४ ) इति निरुक्तधृतब्रह्मणनिरुक्तयोश्चानुसरणेन दिनरात्र्योर्विभागस्य चाश्रयणेन दिनरात्र्योः समासे ३२६ जाते युद्धमभूदेत्यर्थकल्पनया विरोधपरिहारस्यावश्यकरणीयत्वेन कलिप्रथमसंवत्सर एव भारतयुद्धकल्पनायां सत्याम् ६५३ संख्यांयाः कलिगताब्दसंख्याया व्यवकलनेन B.C.वर्षसंख्यायां संकलनेन संख्याङ्कनिर्देश सत्यो मन्तव्यः~। इति दाधिमथाः~। 

\newpage
%१२ महाभाष्यनिर्माता पतञ्जलिः~। 

\begin{center}
\begin{tabular}{m{5em} m{5em} m{25em} m{5em} m{5em}}
& & {\mbh सेनाजित् सांप्रतं चापि एतां वै भुज्यते समाः~।} & & \\
१०४८& २०५३&{\mbh भोक्ष्यवते नृपतिश्चैष अष्टपञ्चाशतं समाः~॥~२९५~॥ }& १२९८ & १८०३\\ 
१०८८& २०१३&{\mbh श्रुतंजयस्तु वर्षाणि चत्वारिंशद् भविष्यति~॥~२५६~॥} & १३३८ & १७६३\\
& & {\mbh महाबाद्दुर्महाबुद्धिर्महाभीमपराकमः~।} & & \\
११२३ & १९७८ &{\mbh पञ्चत्रिंशत्तु वर्षाणि महीं पालयिता नृपः ( विप्रः ) ~॥~२९७~॥} & १३७३ & १७२८\\
११७१ & १९२० &{\mbh अष्टपश्चाशतं चाब्दान् राज्ये स्थास्यति वै शुचिः~।} & १४३१ & १६७० \\
१२१९ & १८८२ &{\mbh अष्टात्रिंशत् समाः पूर्णाः क्षेमो ( म्यो ) राजा भविष्यति~॥~२९८~॥} & १८४९ & १६३२\\
१२८३ & १८१८ &{\mbh सुव्रतस्तु चतुःषष्टी राज्यं प्राप्स्यति वीर्यवान्~। }& १५३३ & १५६८\\
१२८८& १८१३ & {\mbh पञ्च वर्षाणि पूर्णानि धर्मनेत्रो भविष्यति~॥~२९९~॥} & १५३८ & १५६३\\
१३२६& १७७५&{\mbh अष्टात्रिंशत्समा राज्यं सुव्रत ( श्रव ) स्य भविष्यति~।} & १५७६ & १५२५\\
१३८४& १७१७&{\mbh चत्वारिंशद्दशाष्टौ च दृढसेनो भविष्यति~॥~३००~॥} & १६३४& १४६७\\
१४१७ & १६८४ &{\mbh त्रयस्त्रिंशत्तु बर्षाणि राज्यं सुमतिः प्राप्स्यते ततः~।} & १६६७ & १४३४\\
१४३९ & १६६२ &{\mbh द्वाविंशतिसमा राज्यं सुच ( ब ) लो भोक्ष्यते ततः~॥~३०१~॥ }& १६८९ & १४१२\\
१४७९ & १६२२ & {\mbh चत्वारिंशत्समा राजा सुनेत्रो भोक्ष्यते ततः~।} & १७२९ & १३७२\\
१५६२ & १५३९ & {\mbh सत्यजित् पृथिवीराज्यं त्र्यशीतिं भोक्ष्यते समाः~॥~३०२~॥} & १८१२ & १२८९\\
१५६२ & १५९४ & {\mbh प्राप्येमां वीर ( विश्व ) जिच्चापि पञ्चत्रिंशद भविष्यति~।} & १८४७& १२५४\\
१६४७ &१४५४ & {\mbh अरिंं ( रिपुं ) जयस्तु वर्षाणि पञ्चाशत् प्राप्स्यते महीम्~॥~३०३~॥} & १८९७ &१२०४\\
& & {\mbh द्वाविंशच्च नृपा ह्येते भवितारो बृहद्रथाः~।} & & \\
१६५३& १४४८&{\mbh पूर्मं वर्षसहस्त्रं वै तेषां राज्यं भविष्यति~॥~३०४~॥} & १९०३ & ११९८\\
& & {\mbh बृहद्रथेष्वतीतेषु वीतहोत्रेषु वतिंपु~।} & & \\
& & {\mbh मुनिकः स्वामिनं हत्वा पुत्रं समभिषेक्ष्यति~।~३०५~॥} & & \\
& & {\mbh मिषतां क्षत्रियाणां हि प्रद्योतो मुनिको बलात्~।} & & \\
& & {\mbh स वै प्रणतसामन्तो भविष्ये नयवर्जितः~॥~३०६~॥} & & \\
१६७६ & १४२५ & {\mbh त्रयोविंशत्समा राजा भविता स नरोत्तमः~।} & १९२६& ११७५\\
१७०० & १४०१ &{\mbh चतुर्विंशत्समा राजा पालको ( बलाको ) भविता ततः~॥~३०७~॥} & १९५० & ११५१ \\
१७५० & १३५१ &{\mbh विशाखयूपो भविता नृपः पञ्चाशतं समाः~।} & २००० & ११०१\\
१७७१ & १३३० &{\mbh एकविंशत्समा राज्यमजनक ( राज्यं जनकस्य ) भविष्यति~॥~३०८~॥} & २०२१& १०८१\\
१७११ & १३१० &{\mbh भविष्यति समा विंशत्तत्सुतो वर्ति ( नन्दि ) वर्धनः~।} & २०४१& १०६०\\
*१७९१ & १३१० & {\mbh अष्टात्रिंशच्छतं भाव्याः प्राद्योताः पञ्च ते सुताः~॥~३०५~॥} & & \\
& & {\mbh दृत्वा तेषां यशः कृत्स्नं शिशुनाको ( भो ) भविष्यति~।} & & \\
& & {\mbh वाराणस्यां सुतस्तस्य संप्राप्स्यति गिरिजव्रम्~॥~३१०~॥} & & \\
१८३१ & १२७० &{\mbh शिशुनाकस्य वर्षाणि चत्वरिंशद् भविष्यति~।} & २०८१ & १०२०\\
१८६७ & १२३४ & {\mbh शक ( काक ) वर्णः सुतस्तस्य षट्त्रिंशच्च भविष्यति~॥~३११~॥} & २११७ & ९८४ \\
१८८७ &१२१४ & {\mbh ततस्तु विंशति राजा क्षेमव ( ध ) र्मा भविष्यति~।} & २१३७ & ९६४\\
१९२२ &११७९&{\mbh अजातशत्रुर्भविता पञ्चत्रिंशत्समा नृप~॥~३१२~॥} & २१७२ & ९२४\\
१९६२ & ११३९ &{\mbh चत्वारिशत्समा राज्यं क्षत्रौजाः प्राप्स्यते ततः~।} & २२१२ & ८८९\\
१९९० & ११११ &{\mbh अष्टाविंशत्समा राजा वि ( धि ) सारो भविष्यति~।~३९३~॥} & २२४० & ८६१\\
२०१५ & १०८६ &{\mbh पश्चविंशत्समा राजा दर्श ( अर्भ ) कस्तु भविष्यति~।} & २२६५ & ८३६\\
२०४८ & १०५३ & {\mbh उदायी भविता तस्मात् त्रयस्त्रिंशत् समा नृपः~॥~३१४~॥} & २२९८ & ८३३\\
& & {\mbh स वै पुरवरं राजा पृथिव्यां कुसुमाह्वयम्~।} & & \\
२०५२ & १०४९ & {\mbh गङ्गाया दक्षिणे कूले चतुर्थेऽब्दे करिष्यति~॥~३१५~॥} & २३०२ & ७९९\\
२०९४ & १००७ &{\mbh द्वाचत्वारिंशत्समा भाव्यो राजा वै नन्दिवर्धनः~।} & २३४४ & ७५७\\
२१३७ & ९६४ & {\mbh चत्वारिंशत् त्रयश्चैव महानन्दी भविष्यति~॥~३१६~॥} & २३८७ & ७१४
\end{tabular}
\end{center}

\newpage
% महाभाष्यनिर्माता पतञ्जलिः~। १३

\begin{tabular}{m{5em} m{5em} m{25em} m{5em} m{5em}}
& & {\mbh इत्येते भवितारो वै शैशुनाका ( भा ) नृपा दश~।} & &\\
*२१५३ & ९४८& {\mbh शतानि त्रीणि द्विषष्ट्यभ्यधिकानि तु~॥~३१७~॥} २४०३& ६९८\\
& & {\mbh शैशुनाका ( भा ) भविष्यन्ति तावत्कालं नृपाः परे~। } & & \\
& & {\mbh राजानः क्षत्रबान्धवा एतैः सार्धं भविष्यति~॥~३१८~॥} & & \\
& & {\mbh ऐक्ष्वाकवाश्चतुर्विशत्पाञ्चालाः पञ्चविंशतिः~।} & & \\
& & {\mbh कालकास्तु चतुर्विंशच्चतुर्विंशत्तु हैहयाः~॥~३१९~॥} & & \\
& & {\mbh द्वात्रिंशद्वै कलिङ्गास्तु पञ्चत्रिंशत् तथा शकाः~।} & & \\
& & {\mbh कुरवश्चापि षट्त्रिंशदष्टाविंशतिमैथिलाः~॥~३२०~॥} & & \\
& & {\mbh शूरसेनास्त्रयोविंशत् वीतिहोत्राश्च विंशतिः~।} & & \\
& & {\mbh तुल्यकालं भविष्यन्ति सर्व एव महीक्षितः~॥~३२१~॥} & & \\
& & {\mbh महानन्दिसुतश्चापि शूद्रायाः कालसंवृतः~।} & & \\
& & {\mbh उत्पत्स्यते महापद्मः सर्वक्षत्रान्तको नृपः~॥~३२२~॥} & & \\
& & {\mbh ततः प्रभृति राजानो भविष्याः शूद्गयोनयः~।} & & \\
& & {\mbh एकराट्र स महापद्म एकच्छत्रो भविष्यति~॥~३२३~॥} & & \\
२२४१ & ८६० & {\mbh अष्टाविंशति ( शीतिं स ) वर्षाणि पृथिवीं पालयिष्यति~।} &२४९१ &६१०\\
& & {\mbh सर्वक्षत्रान् समुद्धृत्य भाविनोऽर्थस्य वै बलात्~॥~३२४~॥} & & \\
& & {\mbh सहसा तत्सुताश्चाष्टौ समा द्वादश ते नृपाः~।} & & \\
& & {\mbh महापद्मस्य पर्याये भविष्यन्ति नृपाः क्रमात्~॥~३२५~॥} & & \\
& & {\mbh उद्धरिष्यति तान् सर्वान् कौटिल्यो वै द्विरष्टभिः~।} & & \\
२२५३& ८४८& {\mbh भुक्त्वा महीं वर्षशतं नन्देन्दुः स भविष्यति~॥~३२६~॥} २५०३ & ५९८\\
& & {\mbh चन्द्रगुप्तं नृपं राज्ये कौतिल्यः स्थापयिष्यति~।} & & \\
२२७७ & ८२४ & {\mbh चतुर्विंशत्समा राजा चन्द्रगुप्तो भविष्यति~॥~३२७~॥}& २५२७ & ५४९ \\
२३०२ & ७९९& {\mbh भविता भद्र ( बिन्दु ) सारस्त पञ्चविंशत्समा नृपः~।} & २५५२& ५४९ \\
२३३८ & ७६३& {\mbh षट्त्रिंशत्तु समा राजा अशोको भविता नृपु~॥~३२८~॥} & २५८८ & ५१३\\
२३४६ & ७५५& {\mbh तस्य पुत्रः कुनालस्तु वर्षाण्यष्टौ भविष्यति~।} & २५९६ & ५०५\\
२३५४ & ७४७& {\mbh कुनालसूनुरष्टौ च भोक्ता वै बन्धुपालितः~॥~३२९~॥} & २६०४& ४९७\\
२३६४ & ७३७& {\mbh बन्धुपालितदायादो दशाब्दानीन्द्रपालितः~।} & २६१४ & ४८७\\
२३७१& ७३०& {\mbh भविता सप्तवर्षाणि देववर्मा नराधिपः~॥~३३०~॥} & २६२१ & ४८०\\
२३७९ & ७२२& {\mbh राजा शतधरश्चाष्टौ तस्य पुत्रो भविष्यति~।} & २६२९ & ४७२\\
२३८६ & ७१५& {\mbh बृहद्रथश्च वर्षाणि सप्त वै भविता नृपः~॥~३३१~॥} & २६३६ &४६५\\
& & {\mbh इत्येते नव भूपाला ये भोक्ष्यन्ति वसुन्धराम्~।} & & \\
२३९०& ७११& {\mbh सप्तत्रिंशच्छतं पूर्णं तेभ्यस्तु गौर्भविष्यति~॥~३३२~॥} & २६४०& ४६१\\
& & {\mbh पुष्पमित्रस्तु सेनानीरुद्धत्य वै बृहद्रथम्~।} & &\\
२४५० & ६५१& {\mbh कारयिष्यति वै राज्यं समाः षष्टिं सदैव तु~॥~३३३~॥} & २७०० & ४०१
\end{tabular}

This is also accepted by Mr. M. M. Kunte B. A. in the Vicissitudes of Aryan Civilizatiou in India, page 316 ; \textendash\ \\

"The chronology of Buddha being accepted as 500 years before Christ it follows that Patanjal wrote his commentary about 600 B. C.; and that Pānini taught his pupils about 803 B. C."\\

Pandit N. Bhāshyāchārya in his The Age of Patanjali, page 33 also comes to the following conclusion ( a ) Patanjali was the author of the Mahābhāshya, a commentary 0n Pānini's Ashtā \textendash\ dhyāyi ( and also of the Yoga utras ) and ( b ) that he lived without any donbt between the 9th and l0th centuries B. C., that is, ahout the 10th century B. C."

\begin{center}
\rule{0.2\linewidth}{0.5pt}
\end{center}

\newpage
% १४ महाभाष्यनिर्माता पतञ्जलिः~। 

अत एव \textendash\ 

\begin{quote}
{\qt तदा भगवतः शाक्यसिंहस्य परनिर्वृतेः~। अस्मिन् महीलोकधातौ सार्धं वर्षशतं ह्यगात्~॥ १~। १७२~॥

बोधिसत्त्वश्च देशेsस्मिन्नेकोभूमीश्वरोsभवत्~। स च नागार्जुनः श्रीमान् षडर्हद्धनसंश्रयी~॥ १~। १७३~॥ 

अथ निष्कण्टको राजा कण्टकोत्साह्यहारदः~। अभीर्बभूवाभिमन्युः शतमन्युरिवापरः~॥ १~। १७४~॥ 

स्वनामाङ्कं शशाङ्काङ्कशेखरं विरचय्य सः~। परार्ध्यविभवं श्रीमानभिमन्युपुरं व्यधात्~॥ १~। १७५~॥ 

चन्द्राचार्यादिभिरब्ध्वा देशात्तस्मात्तदागमम्~। प्रवर्तितं महाभाष्यं स्वं च व्याकरणं कृतम्~॥ १~। १७६~॥}
\end{quote}

इत्येवं राजतरङ्गिगीसमवलोकनेन शाक्यसिंहनिर्वाणकालः पञ्चदशोत्तरषड्विंशतिशत २६१५ कलिगताब्दोत्तरं १५० वर्ष ( २६१५+ १५०$=$२७६५ कलिगताब्दकाल$=$३३६B . C. ) राज्येन अभिमन्युना महाभाष्यस्य काश्मीरेषु प्रचारकथनं संगच्छते~॥\\

पूर्वव्याकरणानुसार्येवोत्तरव्याकरणनिर्माणम् इति नियमानुसारेणैव पूर्वव्याकरणस्यावश्यकत्वेनैव \textendash\

\begin{quote}
{\qt असौ पुनर्व्याकरणं ग्रहीष्यन्सूर्योन्मुखः प्रष्टुमनाः कपीन्द्रः~।\\
उद्यद्गिरेरस्तगिरिं जगाम ग्रन्थं महद् धारयन्नप्रमेयः~॥ ७~। ३६~। ४४~॥ 

ससूत्रवृत्त्यर्थपदं महार्थं ससंग्रहं सिध्यति वै कपीन्द्रः~॥}
\end{quote}

इति वाल्मीकीयरामायणे वर्णितस्य श्रीहनुमता पठितस्य संग्रहग्रन्थस्य सत्त्वेऽपि नैव विरोधिनां शङ्का~॥ यद्वा ग्रन्थानां संग्रहं समूहमित्येवार्थो भवेत्~॥\\

यत्तु {\qt अकथितं च} इतिसूत्रभाष्ये {\qt उभयथा गोणिकापुत्रः}? इति भाष्यव्याख्यानावसरे गोणिकापुत्रो भाष्यकार दत्याहुः इति नागेशीयोद्दयोतमनुसृत्य पतञ्जलिचरितकाव्ये रामभद्रदीक्षितेन \textendash\ चिदम्बरतीर्थे पुत्रार्थं तपस्यन्त्या गोणिकाया गर्भे प्रादुरभूत् पतञ्जलिर्भाष्यकारः \textendash\ इति वर्णितम्~।\\

यञ्च कैश्चित् पण्डितवरैः \textendash\

\begin{quote}
{\qt योगेन चित्तस्य पदेन वाचां मलं शरीरस्य तु वैद्यकेन~।\\
योऽपाकरोत्तं प्रवरं मुनीनां पतञ्जलिं प्राञ्जलिरानतोऽस्मि~॥}
\end{quote}
 
इत्यनुपलब्धप्रमाणोद्भटश्लोकानुसारेण योगसूत्रकर्तृत्वमप्यस्यैव पतञ्जलेर्वर्णितम्~।\\

तदुभयमपि नैव विचारसहम्~। तथाहि तत्प्रसङ्गाच्चारायणः साधारणमधिकरणं पृथक् प्रोवाच, सुवर्णनाभः साभ्प्रयोगिकम्, घोटकमुखः कन्यासंप्रयुक्तम्, गोनर्दीयो भार्याधिकारम्, गोणिकीपुत्रः पारदारिकम्, कुचुमार औपनिषदिकम \textendash\ इत्येवं बहुभिराचार्यैस्तच्छास्त्रं खण्डशः प्रणीतमुपसन्नकल्पमभूत्{\qt इति कामसूत्रे प्रथमाध्याये शास्त्रसंग्रहे वात्स्यायनेन गोनर्दीयगोणिकापुत्रयोः पार्थक्येनोपादानात् }गोनर्दीयः पतञ्जलौ इति कोशप्रामाण्येन गोनर्दीयस्य भाष्यकारपतञ्जलिनामत्वाङ्गीकारेऽपि गोणिकापुत्रस्य भाष्यकारनामत्वे प्रमाणानुपलम्भ एव \textendash\ इति~॥\\

एवं च योगसूत्रकर्तृत्वं कपिपुत्रस्य आद्यपतज्जलेः पतञ्जलिगोत्रप्रवर्तकस्वैव~। अत एव तत्कृतयोगसूत्रस्य {\qt एतेन योगः प्रत्युक्तः ( २~। १~। ३ ) इति पाराशर्येण स्वसूत्रेषूद्वेशः कृतः~। पाराशर्यसूत्राणि }पाराशर्यशिलालिभ्यां भिक्षुनटसूत्रयोः इति पाणिनिना खतः प्राग्भवत्वेन सूचितानि~॥\\

भाष्ये \textendash\ जानकीहरणस्थस्य {\qt वरतनु संप्रवदन्ति कुक्कुटाःइति पञ्चचतुर्थांशस्य व्यक्तवाचां समुच्चारणे} इति सूत्रे उदाहरणात् जानकीहरणकर्तुः कुमारदासादर्वाचीनत्वं प्रसिद्ध्यति~। {\qt कुमारदासश्च} जानकीहरणं कर्तुं रघुवंशे स्थिते सति~। कविः कुमारदासश्च रावणश्च यदि क्षमः~॥ इत्युक्ते रघुवंशप्रणेतुः कालिदासादर्वचीनः~। एवं {\qt उदगात् कठकालापं प्रत्यष्ठाद कठकौथुमम्~। } इति रावणार्जुनीयकाव्यस्थपद्यार्धत्य {\qt अनुवादे चरणानाम्} इति सूत्रभाष्ये उदाहरणाद् भौमककवेरप्यर्वाचीनत्वं भाष्यकर्तुरवगम्यते~।

\noindent
\rule{1\linewidth}{0.5pt}\\

१ अयं च श्लोको धारेश्वरभोजराजनिर्मितशब्दानुशासनस्थो भवेत्, एतदर्थस्यैव तदीययोगसूत्रवृत्तिमङ्गलश्लोकेनोपमया सूचितत्वात्~।\\

२ {\qt सौबीराः सैन्धवा हूणाः शाल्वाः शाकलवासिनः~। महारामास्तथान्वष्ठः पारसीकादयस्तथा~॥~१७~॥}

इति विष्णुपुराणे द्वितीयेंऽशे तृतीयेऽध्याये,

\begin{quote}
{\qt असृजत् पह्ववान् पुच्छाद् प्रस्नावाद् द्राबिडाञ्शकान्~। योनिदेशाच्च यवनान् शकृतः शबरान् बहून्~॥~३५~॥

मूत्रतश्चासृजत् काञ्चीञ्शरभांस्चैव पार्श्वतः~। पौण्ड्रान् किरातान् यवनान् सिंहलान् बर्बरान् खशान्~॥~३६~॥

चिबुकांश्च पुलिन्दाश्च चीनान् हूणान् सकेरलान्~। ससर्ज फेनतः सा गौर्म्लेच्छान् बहुविधानपि~॥~३७~॥}
\end{quote}

 इति श्रीमन्महाभारते आदिपर्वणि १७७ अध्याये, 

\begin{quote}
{\qt उत्तराश्चापरे म्लेच्छाः क्रूरा भरतसत्तम~। यवनाश्चीनकाम्भोजा दारुणा म्लेच्छजातयः~॥~६५~॥

सकृद्भहाः कुलत्थाश्च हूणाः पारसिकैः सह~। तथैव रमणाश्चीनास्तथैव दशमालिकाः~॥~६६~॥}
\end{quote}

इति श्रीमन्महाभारते भीष्मपर्वणि नवमे भारतवर्षनिरूपणाध्याये च वर्णनात्प्राचीनत्वमेव हूणानामिति नैव हूणपदप्रयोगाद्रधुवं \textendash\ शस्यार्वाचीनत्वमिति~॥

\newpage
% महाभाष्यनिर्माता पतञ्जलिः~। १५ 

{\qt भाष्यकृता पद्यगन्धिगद्यस्यैवोदाहृत्वेन पद्यगन्धितयैव तस्य समस्यात्वमङ्गीकृत्य समस्यापूर्तिरेव कविभिः कृता भवेत् इति नार्वाचीनत्वसंभावनेति} शिष्टशिरोमणयः~॥\\

भविष्यपुराणे तु प्रतिसर्गपर्वणि कलियुगीयेतिहाससमुच्चये \textendash\ 

\begin{quote}
{\qt आसीत्पुरा कलियुगे पितृशर्मा द्विजोत्तमः~। वेदवेदाङ्गतत्त्वशो यमलोकभयान्वितः~॥ ३०~। ४~॥ 

सुता या विष्णुयशसो ब्राह्मणस्य तदा स्वयम्~। तामुद्वाह्य द्विजो देवीं नाम्ना वै ब्रह्मचारिणीम्~॥ ३०~। १६~॥ 

प्रियायै स रजोवत्यै ऋतुदानं करोति हि~। चत्वारश्चात्मजाश्चासंश्चतुर्वदै ( क ) क्यधारिणः~॥ ३०~। ९८~॥

ऋचस्तु तनयो व्याडिर्न्यायशास्त्रविशारदः~। यजुषस्तु सुतो जातो मीमांसो लोकविश्चुतः~॥ ३०~। १९~॥

पाणिनिः सामनस्यैव सुतोsभूच्छब्दपारगः पुत्रो वररुचिः श्रेष्ठोऽथर्वणस्य नृपप्रियः~॥ ३०~। २०~॥

ते गता मागधेशस्य चन्द्रगुप्तस्य वै सभाम्~। नृपस्तान् पूजयामास बहुमानपुरःसरैः~॥ ३०~। २१~॥

सामनस्य सुतः श्रेष्ठःपाणिनिर्नाम विश्रुतः~। काणभूतिप्रशिष्यैश्च शास्त्रज्ञैः स पराजितः~॥ ३१~। २~॥

लज्जितः पाणिनिस्तत्र गतस्तीर्थान्तरं प्रति~। स्नात्वा सर्वाणि तीर्थानि संतर्प्यं पितृदेवताः~॥ ३१~। ३~॥

केदारमुदकं पीत्वा शिवध्यानपरोऽभवत्~। पर्णाशी सप्त दिवसाञ्जलभक्षस्ततोऽ भवत्~॥ ३१~। ४~॥

ततो दशदिनान्ते स वायुभक्षो दशाहनि~। अष्टाविंशे दिने रुद्रो वरं ब्रहि वचोऽब्रवीत्~॥ ३१~। ५~॥

इति श्रुत्वा महादेवः सूत्राणि प्रददौ मुदा~। सर्ववर्णमयान्येव अइउणादिशुभानि वै~॥ ३१~। १०~॥

इत्युक्त्वाऽन्तर्दथे रुद्रः पाणिनिः स्वगृहं ययौ~। सूत्रपाठं धातुपाठं गणपाठं तथैव च्~॥ ३१~। १३~॥

लिङ्गसूत्रं तथा कृत्वा परं निर्वाणमाप्तवान्~॥ ३१ शेऽध्याये १४~॥ 

चित्रकूटे गिरौ रम्ये नानाधातुविचित्रिते~। तत्रावसन्महाप्राज्ञ उपाध्यायः पतञ्जलिः~॥ ३५~। ३~॥

काश्यां कात्यायनेनैव तस्य वादो महानभूत्~। वर्षान्ते च तदा विप्रो महाशक्तेन निर्जितः~॥ ३६~। ३~॥

लज्जितः स तु धर्मात्मा स तुष्टाव सरस्वतीम्~॥ ३६~। ४

कात्यायनं पराजित्य परां मुदमवाप्तवान्~। {\qt महाभाष्यमुदैरयत्~॥ ३६~। १२~॥}}
\end{quote}

इत्येवं समुपलभ्यते मुनीनां दीर्घायुष्ट्वेन नन्द \textendash\ चन्द्रगुप्त \textendash\ पुष्पमित्रराज्यद्रष्टत्वे का बाधा~॥ व्याडि \textendash\ पाणिनि \textendash\ वररुचीनां भ्रातृतुत्वं कल्पान्तराभिप्रायेण भवेत्~॥ 

\vspace{3cm}
\begin{center}
\rule{0.1\linewidth}{0.5pt}
\end{center}

\newpage
\thispagestyle{empty}
\begin{center}
\textbf{\Large महाभाष्यप्रदीपनिर्माता कैयटः~।}\\
\rule{0.2\linewidth}{0.5pt}
\end{center}

\noindent
\rule{1\linewidth}{0.5pt}\\

\begin{multicols}{2}
तस्यैतस्य महाभाष्यस्य व्याख्यासु१ मूर्धन्यतमा २हरिनिर्मित व्याख्यानुसारिणी जैयटात्मज३कैयटविरचिता प्रदीपाख्या व्याख्या सर्वसंमताऽस्ति~॥

\noindent
\rule{1\linewidth}{0.5pt}\\

१ तासां व्याख्यानां व्याख्याकर्तृणां च नामान्येतावत्कालपर्य \textendash\ न्तमेतावन्त्येव र्केटलॉगस् केटलोगोरम् नामके ग्रन्थे थिआडोर आफ्रेच महाशयेन प्रकाशितानि

\begin{tabular}{c c}
व्याख्यानाम& रचयितृनाम \\
शब्दबृहती & \\
महाभाष्यप्रदीपः& कैयटरचितः \\
प्रकाशः& नारायणशेषः \\
सूक्तिरत्नाकरः & नारायणशेषः कृष्णात्मजः \\
" & नृसिंहः जीवदेवात्मजः \\
महाभाष्यादर्शः & लक्ष्मणः मुरारिसूनुः \\
महाभाष्यसिद्धान्तरत्नप्रकाशः & शिवरामेन्द्रसरस्वती \\
महाभाष्यगूडार्थदीपिनी & सदाशिवः \\
महाभाष्यत्रिपदीव्याख्यानम् & भर्तृहरिः\\
महाभाष्यदीपिका & भर्तृहरिः
\end{tabular}

२ अस्य हरेः$=$भर्तृहरेः समयः श्रीपाद कृष्ण बेल्वलकर, एम्. ए. पीएच् ङी. महाशयेनाष्टमी वैक्रमी शताब्दी ( ६५० इसवी ) निर्दिष्टः~॥

३ अस्य कैयटस्य विषये वेल्वलकरमहाशयेन बम्बईमुद्रिताया वामनाचार्यकृतकाव्यप्रकाशव्याख्याया भूमिकायां भीमसेनीय \textendash\ सुधासागरनामक ( वैक्रमसंवत् १७७९$=$१७२२ इसवी निर्मित ) काव्यप्रकाशव्याख्याश्लोकेषु \textendash\ 

\begin{quote}
{\qt श्रीमान् कैयट औवटो शवरजो यच्छात्रतामागतो\\
भाष्याब्धिं निगमं यथाक्रममनुव्याख्याय सिद्धिं गतः~॥}
\end{quote}

इत्यस्य श्लोकस्यानुसारेण मम्मटानुजत्वं प्रकाशयता वैक्रमी द्वादशी शताब्दी कैयटसत्ताकाल इति प्रकाशितम्~॥ तत्र शुक्लयज्जुर् \textendash\ वेदसंहिता \textendash\ शौनकीय \textendash\ ऋक्प्रातिशाख्य \textendash\ कात्यायनीयशुक्लयजुर्वेदप्राति \textendash\ शाख्यभाष्येषु मम्मट \textendash\ जैयट \textendash\ कैयटानाम् एकस्यापि नामानुपलम्भात्~। {\qt भानन्दपुरवास्तव्यभट्टवज्रटसूनुना} इत्येवमेव उव्वटेन वर्णितत्वाच्च औव्वटस्य जैयटपुत्रत्वे मानाभाव एव~। दत्तकपुत्रत्वकल्पनाया अपि विना मूलं प्रामाणिकत्वाभावात्~। कैयट \textendash\ मम्मटाभ्यामपि अन्यतमस्य नाम्नोsप्यनुल्लेखात्परस्परपरिचयस्याप्यनुगमाभावस्यैव वक्तुं योग्यत्वाच्च~॥ त्रयाणामपि पण्डितप्रकाण्डानां परस्परं नामालेखने

\columnbreak

तस्यैतस्य जैयटात्मजेन महेश्वरान्तेवासिना कैयटेन विरचितस्य महाभाष्यप्रदीपस्य व्याख्यासु४ मध्ये शिवभट्टसुतेन सतीदेवीगर्भजेन५ हंरिदीक्षितान्तेवासिना शङ्गवेरपुराधी \textendash\ शरामतो लब्धजीविकेन पायगुण्डोपाख्यवैद्यनाथ६स्य

\noindent
\rule{1\linewidth}{0.5pt}\\

\noindent
कारणं स्वल्पान्तरसमयोत्पन्नत्वं भिन्नभिन्नदेशप्रभवत्वमेव कल्पनीयम्~॥ वेव्याख्यातुरुव्वटस्य{\qt भोजे राज्यं प्रशासति} इति लेखेन भोज \textendash\ समानकालिकत्वे ज्ञातेऽपि मम्मटकैयट्योः समयो नैव निश्चितः~॥\\

काव्यमालानवमगुच्छकमुद्रित \textendash\ श्रीमदानन्दवर्धनाचार्यविरचि \textendash\ तदेवीशतकव्याख्याता कैय्यटस्तु चन्द्रादित्यपुत्रः भीमगुप्त \textendash\ नरपतिराज्ये ४०७८ मितेषु कलिगताब्देषु ( ९७८ खिष्टाब्दे ) संबभूवेति काव्यमालापुस्तकतोऽवसेयम्~॥

४ तासां व्याख्यानाम् तत्कर्तॄणां च नामानि केटलोगस् केटलो \textendash\ गोरम् ग्रन्थे थिअडोर अक्रेच्महोदयेन प्रकाशितानि दर्शयामि \textendash\ 

अनन्तभदट्टनिर्भितम् महाभाष्यप्रदीपविवरणम्, 

ईश्वरानन्दविरचितं प्रदीपविवरणम्, 

नारायणनिर्मितं प्रदीपविवरणम्, 

नागेशविरचितः प्रदीपोद्योतः, 

हरिरामरचिता महाभाष्यप्रदीपटीका~॥

५ अयं च हरिदीक्षितो भट्टोजिदीक्षितपौत्रो वीरेश्वरपुत्रो रामाश्रमशिष्यो नागेशगुरुरस्ति~। एतन्निर्मिता ग्रन्थास्तु \textendash\ 

परिभाषाटीका, 

परिभाषोपस्कारः, 

भावार्थप्रकाशिका, 

शब्दरत्नम्

शब्दसिद्धिः 

सिद्धान्तकौमुदीटीका~॥ 

६ अयं च वैद्यनाथः पायगुण्डोपाख्यमहादेवसुतो वेणीगर्भजो नागेशशिष्यो मन्नुदेवगुरुरस्ति~॥ एतन्निर्मिता ग्रन्थास्तु \textendash\ 

अर्थसंग्रहः, 

कलामञ्जूषाव्याख्या 

काशिका परिभाषेन्दुव्याख्या 

गदा "

चिदस्थिमाला लघुशब्देन्दुव्याख्या 

छया ( भाष्यप्रदीपोद्द्योतव्याख्या ) 

परिभाषेन्दुशेखरसंग्रहः, 
\end{multicols}

\fancyhead[CO,CE]{महाभाष्यप्रदीपोद्द्योतनिर्माता नागेशभट्टः~।}
\fancyhead[RO,LE]{\thepage}
\cfoot{}
\newpage
%%%%%%%%%%%%%%%%%%%%%%%%%%%%%%%%%%%%%%%%%%%%%%%%%%%%%
\renewcommand{\thepage}{\devanagarinumeral{page}}
\setcounter{page}{17}

% महाभाष्यप्रदीपोद्द्योतनिर्माता नागेशभट्टः~। १७

\begin{multicols}{2}
\noindent
मणिरामप्रपितामहगङ्गारामस्य च गुरुणानागेशभट्टेन \\
\noindent
\rule{1\linewidth}{0.5pt}\\

भक्तितरङ्गिणी, 

प्रभा शब्दकौस्तुभटीका 

भावप्रकाशः श. र. टी. 

भूषण, 

रप्रत्याहारखण्डन, 

विवरण ( बृ. मं. टी. ) 

वृद्धशब्दरत्नशेखरः, 

सर्वमंगला, 

हरिलोचनचन्द्रिका चन्द्रालोकव्याख्या~। 

१ एतद्रचिता ग्रन्थास्त्वेते \textendash\ 

अलंकारसुधा कुवलयानन्दन्याख्या, 

अष्टाध्यायीपाठः 

आचारेन्दुशेखरः 

अशौचनिर्णयः

इष्टिकालनिर्णयः, 

कात्यायनीतन्त्रम् \textendash\ 

काव्यप्रदीपोद्दधोतः. 

गुरुमर्मप्रकाशः रसगङ्गाधरव्याख्या. 

चण्डीपाठटीका. 

चण्डीस्तोत्रप्रयोगविधिः. 

तर्कभाषाटीका$=$युक्तिमुक्तावली, 

तात्पर्यदीपिका. 

तिङन्तसंग्रहः. 

तिथीन्दुशेस्वरः. 

तीर्थेन्दुशेखरः. 

त्रिस्थलीसेतुः 

\noindent
\rule{1\linewidth}{0.5pt}\\

1.अस्य परिभाषेन्दुशेखरस्य व्याख्या कटलागस् कटलोगोरम् ग्रन्थे प्रकाशिताः \textendash\ 

\begin{tabular}{c c}
विषमी & चिद्रूपशर्मा\\
& दुर्बलाचार्य: \\
चित्प्रभा& ब्रह्मानन्दसरस्वती \\
परिभाषार्थमञ्जरी& भीमभट्टो \\
& माधवात्मजः \\
& भैरवमिश्रः \\
श्रिपथगा & राघवेन्द्राचार्यः \\
दोषोद्धारः& मन्यु ( न्न ) देवः \\
काशिका & वैद्यनाथपायगुण्डः \\
गदा & "\\
& लालविहारिशर्मा \\
& शंकरभट्टः \\
सर्वमंगला &शेषशर्मा \\
& हरिरामः
\end{tabular}

3 प्र० पा०

\columnbreak

विरचितो महाभाष्यप्रदीपोद्योत एव शिखामणीयते~॥\\
\rule{1\linewidth}{0.5pt}\\

धातुपाठवृत्तिः 

णेरणिवादार्थः 

पदार्थदीपिका. 

परमलघुमञ्जूषा 

परिभाषेन्दुशेखरः1 

पातञ्जलसूत्रवृत्तिः. 

पातज्लसूत्रवृत्तिभाष्यच्छायाव्याख्या. 

प्रभाकरचन्द्र \textendash\ तत्त्वदीपिका. 

प्रयोगसरणिः 

प्रायश्चित्तेन्दुशेखरसारसंग्रह. 

महाभाष्यप्रदीपोद्द्योतः 

रस्तरङ्गिणीटीका 

रसमञ्जरीप्रकाशः 

रामायणटीका 

लक्षणरत्नमाला ध. 

विषमपदी$=$शब्दकौस्तुभटीका. 

वेदसूक्तभाष्यम् 

वैयाकरणकारिका. 

वैयाकरणभूषणं ( ? ) 

वैयाकरण2सिद्धान्तमञ्जूषा बृहती लघ्वी च 

 व्याससूत्रेन्दुशेखरः 

शब्दरत्नम् 

शब्दानन्तसागरसमुञ्चयसुप्तिङन्तसागरसमुच्चयः 

शब्देन्दुशेखरौ 

लघुसांख्यसूत्रवृत्तिः 

सापिण्डीमञ्जरी 

स्फोटवादः~॥ 

\noindent
\rule{1\linewidth}{0.5pt}\\

\begin{tabular}{c c}
अग्बाकर्त्री & \\
वाक्यार्थचन्द्रिका &हरिशास्त्री\\
भूतिः &तात्या ( रामकृष्ण ) शास्त्री \\
विभूतिः & शिवदत्तदाधिमथः \\
जया & जयदेवमिश्रः~॥
\end{tabular}

2 वैयाकरणसिद्धान्तमज्जूषाया व्याख्याः कैट्लोगस् कैटलोगोरम्
ग्रन्थेनिर्दिष्टाः \textendash\ 

\begin{tabular}{c c}
कला &वैद्यनाथपायगुण्डः \\
कुञ्चिका& कृष्णमित्रः \\
"& जचार्यः \\
& राजाराम दीक्षितः \\
& हरिरामः \\
दिप्पणी & रामनाथः \\
बृ.म. विवरण & वैश्वनाभपायगुण्डः~॥
\end{tabular}

\begin{center}
\rule{0.2\linewidth}{0.5pt}
\end{center}
\end{multicols}

\fancyhead[CO,CE]{काशिकानिर्मातारौ वामनजयादित्यौ~~~~~~~~~~~~~~~~~~~~~~~~शब्दानुशासननिर्माता भोजदेवः~॥ }
\fancyhead[RO,LE]{\thepage}
\cfoot{}
\newpage
%%%%%%%%%%%%%%%%%%%%%%%%%%%%%%%%%%%%%%%%%%%%%%%%%%%%%
\renewcommand{\thepage}{\devanagarinumeral{page}}
\setcounter{page}{18}

% १८ काशिकानिर्मातारौ वामनजयादित्यौ शब्दानुशासननिर्माता भोजदेवः~॥ 

\begin{multicols}{2}
तदेवं वामन१ \textendash\ जयादित्याभ्यां
वृत्ति \textendash\ महाभाष्य \textendash\ नाम \textendash\ पारायण \textendash\ जुसरनन्दिन्नुज्वलदत्त२सायणमाधवरा \textendash\ यमुकुटसुगृहीतनामधेयपूर्णचन्द्रनिर्मितधातुपारायणा \textendash\

\noindent
\rule{1\linewidth}{0.5pt}\\

१ एतयोः$=$वामन \textendash\ जयादित्ययोः समयः वैक्रमे ७०७ ( ६५० इ. स. ) संवत्सरो बेल्वलकरमहाशयेन निर्णीतः~॥ 

२ उज्ज्वलदत्तस्य समयो वैक्रमसंवत्सरः १३०७$=$१२५० ३. स. बेल्वलकरेण निर्णीतः~॥ 

३ तस्याश्चैतस्याः वृत्तेर्व्याख्याः कैटलोगस कैटलॉगोरम् ग्रन्येचिकित्सा

तत्वविमर्शिनी उपमन्युनिर्मिता~॥ 

पञ्जिका 1न्यासापरनामधेया जिनेन्द्रबुद्धविरचिता 

पदमञ्जरी 2हरदत्तनिर्मिता~॥\\
न्यासस्य व्याख्या \textendash\ 

तत्रप्रदीप मैत्रेयरक्षितरचितः 

न्यासोद्दयोतः मल्लिनाथनिर्मितः~॥ \\
पदमभ्जर्या व्याख्या \textendash\ 

कुङ्कुमविकासः शिवभट्टविनिर्मितः 

मकरन्दः रघुनाथरचितः~॥

\begin{center}
\rule{0.1\linewidth}{0.5pt}
\end{center}

४ अयं च श्री भोजदेवो वैक्रम्याम् एकादशशताब्यां समभूदिति काव्यमालान्तर्गतप्राचीनलेखमालायां द्वितीयलेखतः संप्रतीयते~। अतः स लेखः प्रदर्श्यते \textendash\

\begin{quote}
{\qt जयति व्योमकेशोऽसौ यः सर्गाय बिभर्ति ताम्~।\\
ऐन्दवीं शिरसा लेखां जगद्बीजांकुराकृति्~।\\
तन्वन्तु वः स्मरारातेः कल्याणमनिशं जटाः~।\\
कल्पान्तसमयोद्दामतडिद्वलयपिङ्गलाः~॥}
\end{quote}

परमभट्टारकमहाराजाधिराजपरमेश्वर श्रीकृष्णराजदेवपादानु \textendash\ ध्यात \textendash\ परमभट्टारकमहाराजाधिराजपरमेश्वर श्रीवैरिसिंहदेव \textendash\ पादानुध्यात \textendash\ ] परमभट्टारकमहाराजाधिराजपरमेश्वर श्रीसीयक \textendash\ देवपादानुध्यात \textendash\ परमभट्टारकमहाराजाधिराजपरमेश्वर श्रीवाक्पतिराजदेवपादानुध्यातपरमभट्टारकमहाराजाधिराजपरमेश्वर श्री \textendash\ सिन्धुराजदेवपादानुध्यातपरमभट्टारकमहाराजाधिराजंपरमेश्वर \textendash\ श्रीभोजदेवः कुशली नागह्रदपश्चिमपथकान्तःपाति \textendash\ वीराणके समुपगतान् समस्तराजपुरुषान् ब्राह्मणोत्तरान् प्रतिनिवासिपट्टकिल \textendash\ जनपदादींश्च समादिशति~। अस्तु वः संविदितम् \textendash\ यथाsतीताष्टसप्त \textendash\ त्यधिकसाहस्रिक १०७८ संवत्सरे माघासिततृतीयायां रवाबुदग \textendash\

\noindent
\rule{1\linewidth}{0.5pt}\\

1 अस्य समयो बेल्वलकरेणाष्टमी ख्रैष्टशताब्दी निर्णीता~॥ 

2 अस्य हरदत्तस्य समयो वेल्वलकरेण द्वादशी ख्रैष्टशताब्दी निर्णीता~॥ 

\columnbreak

\noindent
दिविप्रकीर्णार्थसंग्रहरूपा काशिका३ नाम नूतना वृत्तिविरचिता~॥ ततश्च धारानगराधीश्वर \textendash\ भोज४देवेन भूयसो५ ग्रन्थान्निर्मात्रा शब्दानुशासनं विरचितम्~॥

\noindent
\rule{1\linewidth}{0.5pt}\\

\noindent
यनपर्वणि कल्पितहलानां लेख्ये श्रीमद्धारायाम् अवस्थितैरस्माभिः स्नात्वा चराचरगुरुं भगवन्तं भवानीपतिं समभ्यर्च्य संसारस्यासारतां दृष्ट्वा \textendash\

\begin{quote}
{\qt वाताभ्रविभ्रममिदं वसुधाधिपत्य \textendash\ \\
मापातमात्रमधुरो विषयोपभोगः~। \\
प्राणास्तृणाग्रजलबिन्दुसमा नराणां \\
धर्मः सखा परमहो परलोकयाने~॥ 

भ्रमत्संसारचक्राग्रधाराधारामिमां श्रियम्~। \\
प्राप्य ये न ददुस्तेषां पश्चात्तापः परं फलम्~॥}
\end{quote}

इति जगतो विनश्वरस्य स्वरूपमाकलय्योपरिलिखितग्रामः स्यसी \textendash\ मातृणागोचरपूति ( त ) पर्यन्तः सहिरण्यभागभोगः सपरिकरः सर्वादा \textendash\ यसमेतः ब्राह्मणधनपतिभट्टाय भट्टगोविन्दसुताय बह्वृचाश्वलायन \textendash\ शाखाय त्रिप्रवराय वैल्लवल्लप्रतिबद्धश्रीवादाविनिर्गतसधसुरसङ्गकर्णादाय मातापित्रोरात्मनश्च पुण्यशोभिवृद्धये अदृष्टफलमङ्गीकृत्य आच \textendash\ न्द्रर्कार्णवक्षितिसमकालं यावत् परया भक्त्या शासनेनोदकपूर्व प्रतिपादित इति मत्वा यथादीयमानभागभोगकरहिरण्यादिकमाज्ञा \textendash\ श्रवणविधेयैर्भूत्वा सर्वमस्मै समुपनेतव्यम्~।

सामान्यं चैतत् पुण्यफलं बुद्ध्याऽस्मद्वंशजैः, अन्यैरपि भाविभोक्तृभिरस्मत्प्रदत्तधर्मादायोऽयमनुमन्तव्यः पालनीयश्च~॥ उक्तं च \textendash\ 

\begin{quote}
{\qt बहुभिर्वसुधा दत्ता राजभिः सगरादिभिः~।\\
यस्य यस्य यदा मूमिस्तस्य तस्य तदा फलम्~॥

यानीह दत्तानि पुरा नरेन्द्रैर्दानानि धर्मार्थयशस्कराणि~।\\
निर्माल्यवान्तप्रतिमानि तानि को नाम साधुः पुनराददीत~॥

अस्मत्कुलक्रममुदारमुदाहर्द्भि \textendash\ \\
रन्यैश्च दानमिदमभ्यनुमोदनीयम्~।\\
लक्ष्म्यास्तडित्सलिलबुद्धुदचञ्चलाया \\
दानं फलं परयशःपरिपालनं च~॥}
\end{quote}

\noindent
सर्वानेतान्भाविनः पार्थिवेन्द्रान्भूयो भूयो याचते रामभद्रः~।\\
सामान्योऽयं धर्मसेतुर्नराणां काले काले पालनीयो भवद्भिः~॥\\
इति कमलदलाम्बुबिन्दुलोलां श्रियमनुचिन्त्य मनुष्यजीवितं च~। \\
सकलमिदमुदाहृतं च वुद्ध्या नहि पुरुषैः परकीर्तयो विलोप्याः~॥\\
इति~॥ संवत् १०७८ चैैत्र सुदि १४ स्वयमाज्ञामङ्गलं महाश्रीः~॥ \\
स्वहस्तोऽयं श्रीभोजदेवस्य~॥ 

५ ते च गन्थाः \textendash\ 

आदित्यप्रतापसिद्धान्तः$=$ ( ज्यौ. ) 

आयुर्वेदसर्वस्व ( वैद्य. ) 

चम्पूरामायण 

चाणक्यनीति ( ? ) 

चारुचर्या ( धर्म ) 
\end{multicols}

\fancyhead[CO,CE]{धातुवृत्तिनिर्माता माधवाचार्यः~।}
\fancyhead[RO,LE]{\thepage}
\cfoot{}
\newpage
%%%%%%%%%%%%%%%%%%%%%%%%%%%%%%%%%%%%%%%%%%%%%%%%%%%%%
\renewcommand{\thepage}{\devanagarinumeral{page}}
\setcounter{page}{19}
% धातुवृत्तिनिर्माता माधवाचार्यः~। १९

\begin{multicols}{2}
ततश्च श्रीविष्णुसर्वज्ञ \textendash\ शङ्करानन्दशिष्यो रामकृष्णस्य गुरुः सायणगोत्रः, मायणपुत्रो माधवः विद्यानगरीय 

\noindent
\rule{1\linewidth}{0.5pt}\\

तत्त्वप्रकाश्च ( शैव ) 

नाममालिका 

युक्तिकल्पतरु 

राजम~। र्तण्डः ( यो. स. वृ. ) 

~~~~~~~~~~~~~ ( वेदान्त ) 

~~~~~~~~~~~~~ ( ज्यौति. ) 

राजमृगाङ्कः ( ज्यौति. ) 

~~~~~~~~~~~ ( वैद्य. ) 

विद्याविनोद ( काव्य. ) 

विद्वज्जनविनोद ( प्र. ज्यौ. ) 

विश्रान्तविद्याविनोद ( वैद्य. ) 

व्यवहारसमुच्चय ( धर्म. ) 

शब्दानुशासनम् 

शालिहोत्र ( अश्ववै. ) 

शिवसत्त्वरत्नकालिका 

समराङ्गणसूत्रधार 

सरस्वतीकण्ठाभरण ( अलं. ) 

सिद्धान्तसंग्रह ( शैव. ) 

सुभाषितप्रबन्ध 

इत्येवं केटलोगस् केटलोगोरम् प्रकाशिताः समुपलभ्यन्ते~॥ 

\begin{center}
\rule{0.2\linewidth}{0.5pt}
\end{center}

१ इत्थं हि बुक्क \textendash\ माधवयोः समयः प्राचीनलेखमालातः प्रत्येयः \textendash\

\includegraphics[width=0.8\linewidth]{latex/c.JPG}

\columnbreak

\noindent
 ( विजयनगरीय ) बुक्कस्य१ तदुत्तराधिकारिहरिहरस्य चामात्यताग्रां चकार कारयामास च पण्डित२प्रकाण्डैर्ग्रन्थान्~॥ 

\includegraphics[width=0.5\linewidth]{latex/d.JPG}

२ बहुभिः पण्ङितप्रकाण्डैर्जीविकालोभेन ग्रन्थान् निर्माय माधवीयत्वेन ख्यातिं नीताः~। यथा हि \textendash\

\begin{quote}
{\qt तेन मायणपुत्रेण सायणेन मनीषिणा~।\\
आख्यया माधवीयेयं धातुवृत्तिर्विरच्यते~॥ 

तत्कटाक्षेण तद्रूपं दधद् बुक्कमहीपतिः~।\\
आदिशन्माधवाचार्य वेदार्थस्य प्रकाशने~॥

स प्राह नृपतिं राजन् सायणार्यो ममानुजः~।\\
सर्वं वेत्त्येष वेदानां व्याख्यातृत्वे नियुज्यताम्~॥

इत्युक्तो माधवार्येण वीरबुक्कमहीपतिः~॥

अन्वशात् सायणाचार्य वेदार्धस्प प्रकाशने~॥}
\end{quote}

अद्भुतदर्पण 

अधिकरणमाला ( पू. मी ) 

अनुभूतिप्रकाश 

अपरोक्षानुभूति 

अभिनवमाधवीय ( धर्म ) 

अष्टकटीका 

आचारमाधवीय 

आत्मानात्मविवेक 

आधानयज्ञतंत्र

आर्षेय ब्राह्मणभाष्य 

आशीर्वादपद्धतिः 

आश्वलायनदर्शपूर्णमाससूञ्रभाष्य 

उपग्रन्थसूत्रवृत्ति 

ऋग्वेदभाष्य 

ऐतरेयारण्यकभाष्य 

ऐतरेयोपनिषद्भाष्य 

कर्मकालनिर्णय 

कर्मविवेक 

कल्पभाष्य 

काठकभाष्य 

कालनिर्णय 

कुरुक्षेत्रमाहात्म्य ( ? ) 

कृष्णचरणपरिचयवृत्ति 
\end{multicols}

\newpage
% २० धातुवृत्तिनिर्माता माधवाचार्यः~। 

\begin{multicols}{2}
कैवल्थोनिपद्भाष्य 

कौषीतकीयोपनिपद्भाष्य 

गोत्रप्रवरनिर्णय 

गोभिलगृह्यसूत्रभाष्य ( ? ) 

चरणव्यूहभाष्य ( ? ) 

छान्दग्योपनिषद्दीपिका 

जातिविवेकशतप्रश्न 

जीवन्मुक्तिविवेक 

जैमिनीयन्यायमालाविस्तार 

ज्ञानखण्डभाष्य 

णत्वभेद 

ताण्ड्यब्राह्मणभाष्य 

तिथिनिर्णय 

तैत्तिरीय ( विद्या ) प्रकाशवार्तिक 

तैत्तिरीयब्राह्मणभाष्य 

~~~~~~~ \textendash\ संध्याभाध्य 

~~~~~~~ \textendash\ संहिताभाष्य 

तैत्तिरीयारण्यकभाष्य 

तैत्तिरीयोपनिषद्भाष्य 

ञ्यम्बकभाष्य 

दक्षिणामूर्त्यष्टकटीका 

दत्तकमीमांसा 

दर्शपूर्णमासप्रयोग 

~~~~~~~भाष्य 

~~~~~~~यज्ञतंत्र 

दशोपनिषद्भाष्य 

देवताध्यायभाष्य 

देवीभागवतस्थिति 

धातुवृत्तिः 

पञ्चदशी 

पञ्चरुद्रीयटीका 

पञ्चशर्वव्यात्ख्या 

पञ्चीकरण 

पराशरस्मृतिव्याख्या 

पाणिनीयशिक्षाभाष्य 

पुराणसार 

पुरुषसूक्तटीका 

पुरुषार्थसुधानिधि 

प्रमेयसारसंग्रह 

बृहदारण्यकभाष्य 

बौधायनश्रौतसूत्रव्याख्या. 

ब्रह्मगीताटीका~। 

भगवद्गीताभाष्य.

मण्डलब्राह्मणभाष्य 

\columnbreak

मन्त्रप्रश्नभाष्य. 

महावाक्यनिर्णय. 

माधवीय. ( धर्म. ) 

माधवीयभाष्य ( वेदान्त ) 

मुक्तिखण्डटीका. 

मुहूर्तमाधवीय, 

यजुर्वेदब्राह्मणभाष्य. 

यज्ञतन्त्रसुधानिधि. 

यज्ञवैभवखण्डटीका 

याज्ञिक्युपनिषद्भाष्य. 

योगवासिष्ठसारसंग्रह. 

रात्रिसूक्तभाष्य. 

रामतत्त्वप्रकाश. 

रुद्रभाष्य. 

लघुजातकटीका. 

व्यवहारमाधव. 

व्याख्या ( वेदान्त ) 

व्यासदर्शनप्रकार. 

शंकरविलास. 

शतपथब्राह्मणभाष्य. 

शतरुद्रीयभाष्य. 

शिवखण्डभाष्य. 

शिवमाहात्म्यभाष्य. 

श्रीसूक्तभाष्य. 

श्रेताश्वतरोपनिषत्प्रकाशिका. 

षड्विंशब्राह्मणभाष्य. 

संध्याभाष्य. 

सरस्वतीसूक्तभाष्य. 

सर्वदर्शनसंग्रह. 

सहस्रनामवार्तिक. 

सामब्राह्मणभाष्य. 

सामविधानब्राह्मणभाष्य. 

सामवेदभाष्य, 

सिंहानुवाकभाष्य, 

सिद्धान्तबिन्दु ( ? ) 

सूतसंहितातात्पर्यदीपिका. 

सूर्यसिद्धान्तटीका. 

स्तोमभाष्य. 

स्मृतिसंग्रह 

स्वरविग्रहशिक्षाभाष्य. 

स्वाध्यायब्राह्मणभाष्य. 

हरिस्तुतिटीका
\end{multicols}

\fancyhead[CO,CE]{प्रक्रियाकौमुदीनिर्माता रामचन्द्रः~।~~~~~~~~~~~~~~~~~~सिद्धान्तकौमुदीनिर्माता भट्टोजीदीक्षितः~।}
\fancyhead[RO,LE]{\thepage}
\cfoot{}
\newpage
%%%%%%%%%%%%%%%%%%%%%%%%%%%%%%%%%%%%%%%%%%%%%%%%%%%%%
\renewcommand{\thepage}{\devanagarinumeral{page}}
\setcounter{page}{21}

% प्रक्रियाकौमुदीनिर्माता रामचन्द्रः~। सिद्धान्तकौमुदीनिर्माता भट्टोजीदीक्षितः~। २१ 

\begin{multicols}{2}
ततःकाशिकावृत्तेरतिविस्त्तृतत्वेन च्छात्राणां दुरवधेयत्वमवलोक्य च्छात्राणामुपकाराय अनन्ताचार्यात्मज \textendash\ नरहरि \textendash\ सूनु \textendash\ कृष्णसूनुना नृसिंहजनकेन विट्ठलपितामहेन लक्ष्मीधरप्रपितामहेन अनन्तप्रपितामहेन गोपालशिष्येण रामचन्द्राचार्येण प्रक्रिया॑कौमुदी १विरचिता~॥

\noindent
\rule{1\linewidth}{0.5pt}\\

१ अस्याः प्रक्रियाकौमुद्याः समयो बेल्वलकरेण पञ्चदशी ख्रैष्ट \textendash\ शताब्दी निर्णीता~॥

तस्याश्चैतस्या व्याख्याः \textendash\ 

\begin{quote}
{\qt अमृतसृतिः वारणवनेशशास्त्रिरचिता\\
तत्त्वचन्द्रो मधुसूदननन्दनजयन्तनिर्मितः\\
प्रकाशः नरसिंहात्मज \textendash\ शेषकृष्णरश्चितः\\
प्रक्रियाञ्जनम्, विद्यानाथदीक्षितविनिर्मितम्~। \\
२प्रसादः ग्रन्थकर्तृपौत्रविद्ठलरचितः\\
सत्प्रक्रियाव्याकृतिः विश्वकर्मशास्त्रिरचिता~॥}
\end{quote}

\begin{center}
\rule{0.2\linewidth}{0.5pt}
\end{center}

२ अनेन सप्तदशख्रैष्टशताब्दीजनुषा रङ्गोजिभट्टेन. \textendash\

अद्वैतचिन्तामणि. 

" शास्त्रसारोद्धार. 

पदार्थदीपिका. 

ग्रन्था निर्मिताः~॥ 

३ अनेन भानुजीदीक्षितेन. \textendash\ 

व्याख्यासुधा ( अमरकोशव्याख्या ) रचिता~॥ 

४ अनेन भद्टोजिदीक्षितात्मजवीरेश्वरात्मजेन रामाश्रम \textendash\ शिष्येण नागेशभदट्टगुरुणा हरिदीक्षितेन. \textendash\

परिभाषाटीका. 

परिभाषोपस्कार. 

फिट्सूत्रव्याख्या. 

भावार्थप्रकाशिका. 

शब्दरत्न ( मनोरमा व्याख्या ) 

\noindent
\rule{1\linewidth}{0.5pt}\\

1 अस्याः प्रकाशव्याख्यायाः समयो बेल्वलकरेण सप्तदशीख्रैष्टशताब्दी निर्दिष्टा~॥ 

2 अस्याः प्रसादव्याख्यायाः समयो बेल्वलकरेण षोडशी ख्रैष्टशताब्दी निर्दिष्टा~॥ 

3 अस्य भूषणस्य व्याख्याः \textendash\ कैटलोगस् कैटलोगोरम् प्रकाशिताः \textendash\ 

~~~~~~~~~~~~~~~~~~~~~~कृष्णमित्रकृता, 

~~~~~~~~~~~~~~~~~~~~~~गोपालदेवकृता, 

~~~~~~~~~~~~~~~~~~~~~~रामनाथकृता, 

~~~~~~~~~~~~~~~~~~~~~~रुद्रदेवकृता,

\columnbreak

ततश्च काशिकावृत्तौ धातुपाठस्य उणादिपाठस्य फिट्स्वरपाठस्य लिङ्गानुशासनस्य च विरहेण व्याकरणसाहित्यस्यैकेनैव ग्रन्थेन छात्राणां ज्ञानसंपत्तये लक्ष्मीधरसूनुना रङ्गोजिभट्ट१सहोदरेण भा्नुजीदीक्षितस्य२ वीरेश्वरस्य३ च जनकेन हरिदीक्षितस्४य पितामहेन कौण्डभट्टस्य५पितृव्येन भट्टोजिर्भट्टे६ ( दीक्षित ) न सकलाष्टाध्याय्या७ अष्टाध्यायीक्रमेण व्याख्या \textendash\ 

\noindent
\rule{1\linewidth}{0.5pt}\\

शब्दसिद्धि 

सिद्धान्तकौमुदीटीका. 

ग्रन्था निर्मिताः~॥ 

५अनेन सप्तदशखैष्ट्शताब्दीजनुषा कौण्डभट्टेन. \textendash\ 

तर्कप्रदीपः

" रत्नभू. 

वैयाकरणभूषम् . 

" सारः

लघुवैयाकरणभूषणसारः 

सिद्धान्तदीपिका 

स्फोटवादः 

ग्रन्था निर्मिताः 

६ अनेन सप्तदशख्रैष्टशताब्दीजनुषा भद्टोजिभट्टेन \textendash\ 

अद्वैतकौस्तुभः 

आचारप्रदीपः 

आशौचनिर्णयः 

आह्निकम्. 

कारिका. 

कालनिर्णयसंग्रहः 

गोत्रप्रवरनिर्णयः 

चतुर्विंशतिमुनिवरव्याख्या. 

चन्दनधारणविधिः 

तत्त्वकौस्तुभः 

तत्त्वविवेकदीपनव्याख्या. 

\noindent
\rule{1\linewidth}{0.5pt}\\

\begin{tabular}{c c}
मतोन्मज्जनी,& वनमालिमिश्रकृता, \\
काशिका, & हरिरामकृता, \\
दर्पणः & हरिवल्लभकृतः~॥ 
\end{tabular}

4 वैयाकरणसिद्धान्तभूषणसारस्य व्याख्याः \textendash\ कटलोगस् केटलोगोरम् ग्रन्थे निर्दिष्टाः \textendash\ 

\begin{tabular}{c c}
कान्तिः & गोपालदेवकृता \\
परीक्षा & भैरवमिश्रकृता \\
& रुद्रनाथकृता \\
काशिका & हरिरामदीक्षितकृता \\
दर्पणः & हरिवल्लभविरचितः \\
दर्पणमार्जनदिप्पणी & शिवदत्तदाधिमथकृता~॥
\end{tabular}
\end{multicols}

\fancyhead[CO,CE]{वैयाकरणसिद्धान्तकौमुदीनिर्माता भट्टोजीदीक्षितः~।}
\fancyhead[RO,LE]{\thepage}
\cfoot{}
\newpage
%%%%%%%%%%%%%%%%%%%%%%%%%%%%%%%%%%%%%%%%%%%%%%%%%%%%%
\renewcommand{\thepage}{\devanagarinumeral{page}}
\setcounter{page}{22}
% २२ वैयाकरणसिद्धान्तकौमुदीनिर्माता भट्टोजीदीक्षितः~। 

\begin{multicols}{2}
\noindent
शब्दकौस्तुभ१नामा तत्सारभूता वैयाकरणसिद्धान्त \textendash\ 

\noindent
\rule{1\linewidth}{0.5pt}\\

तन्त्रसिद्धान्तदीपिका

तत्राधिकारनिर्णयः 

तर्कामृतम् 

तिथिनिर्णयः 

तिथिनिर्णयसंक्षेपः 

तिथिप्रदीपः 

तीर्थयात्राविधिः 

त्रिस्थलीसेतुसारसंग्रहः 

दशश्लोकीव्याख्या. 

धातुपाठनिर्णयः 

प्रायश्चित्तविनिर्णय. 

प्रौढमनोरमा. 

बालमनोरमा. 

भट्टोजिभट्टीय ( ध. ) 

मासनिर्णय. 

लिङ्गानुशासनसूत्रवृत्ति. 

शब्दकौस्तुभ. 

श्राद्धकाण्ड. 

संध्याभाष्य. 

सिद्धान्तकौमुदी. 

एते ग्रन्था निर्मिताः~॥ 

\noindent
७ अत एव \textendash\ 

\begin{quote}
{\qt इत्थं वैदिकशब्दानां दिङ्मात्रमिह दर्शितम्~।\\
विस्तरस्तु यथाशास्त्रं दर्शितः शब्दकौस्तुभे~॥}
\end{quote}

\noindent
\rule{1\linewidth}{0.5pt}\\

\noindent
1 उपलब्धप्रथमपादस्य टीकास्तु \textendash\ C.C.

\begin{tabular}{c c}
उद्वयोतः& विद्यानाथशुक्लरचितः \\
प्रभा& राघवेन्द्राचार्यनिर्मिता \\
प्रभा &वैद्नाथपायगुण्डनिर्मिता\\
भावप्रदीपः& कृष्णमिश्रकृतः\\
विषमपदी& नागेशकृता \\
& कृष्णमिश्रकृतः\\
शब्दकौस्तुभदूषणम्& भास्करदीक्षितकृतम्~॥
\end{tabular}

\noindent
\rule{1\linewidth}{0.5pt}\\

\noindent
 1 एतस्य बृहच्छब्दरत्नस्य व्याख्या \textendash\ C.C. 

\begin{tabular}{c c}
& भवदेवमिश्रकृता \\
& व्यासदेवमिश्रकृता~॥ 
\end{tabular}

\noindent
2 एतस्य लघुशब्दरत्नस्य व्याख्या \textendash\ C.C. 

\begin{tabular}{c c}
भावप्रकाशः &वैद्यनाथपायगुण्डः \\
रत्नप्रकाशिका& भैरवमिश्रकृता 
\end{tabular}

\columnbreak

\noindent
कौमुदी२ तद्याख्यानभूता प्रौढमनोरमा च निरमायिषत~॥

\noindent
\rule{1\linewidth}{0.5pt}\\

इति सिद्धान्तकौमुदीकृदन्तसमाप्तौ, 

\begin{quote}
{\qt फणिभाषितभाष्याब्धेः शब्दकौस्तुभ उद्धृतः~।\\
तत्र निर्णीत एवार्थः संक्षेपेणेह कथ्यते~॥}
\end{quote}

\noindent
इति भूषणकारिकायाम्~। 

{\qt अतो लोपः} इति सूत्रप्रौढमनोरमायाम् {\qt विस्तरः शब्द \textendash\ कौस्तुभे बोध्यः} इत्यस्य व्याख्याने {\qt कौस्तुभे षाष्ठे} इति शब्दरत्ने उक्तं संगच्छते~॥

\begin{center}
\rule{0.2\linewidth}{0.5pt}
\end{center}

१ अस्य समग्राष्टाध्यायीव्याख्यानभूतस्यापि शब्दकौस्तुभस्य प्रथमाध्यायप्रथम1पाद एव मिलति नाग्रिमः~॥

२ अस्याः सिद्धान्तकौमुद्या व्याख्याः \textendash\ 

\begin{tabular}{c c}
तत्त्वबोधिनी& ज्ञानेन्द्रसरस्वतीकृता \\
दाधिमथी &शिवदत्तो दाधिमथः \\
प्रौढ2मनोरमा& ग्रन्थकर्तृकृता \\
फक्किकाप्रकाशः &इन्द्रदत्तोपाध्यायकृतः \\
बालबोधः &सारस्वतव्यूढमिश्रकृतः \\
बालभनोरमा &वासुदेवदीक्षितकृता \\
मानसरञ्जनी& वल्लभकृता \\
रत्नाकरः &रामकृष्णकृतः~।\\
"& शिवरामेन्द्रसरस्वतीकृतः \\
रत्नार्णवः &कृष्णमित्रकृतः
\end{tabular}

\noindent
\rule{1\linewidth}{0.5pt}\\

2 अस्याः प्रौदमनोरमाया व्याख्याः \textendash\ C.C.

\begin{tabular}{c c}
कल्पलता &कृष्णमित्रकृता \\
कुचमर्दनम् & \\
खण्डनम् &मौनिकृष्णभट्टीय \\
" & चक्रपाणिकृतम् \\
" & जगन्नाथकृतम् \\
बृहच्छब्दरत्नम्& हरिदीक्षितकृतम् \\
लघुशब्दरत्नम्& "\\
\end{tabular}

3 एतस्य रत्नाकरस्य व्याख्या प्रदीपः~॥ 

\noindent
\rule{1\linewidth}{0.5pt}\\

लघो शेखरस्य व्याख्यास्तु C.C. 

\begin{tabular}{c c}
वरचन्द्रिका & \\
& उदयंकरः \\
& गोपालदेवः \\
चन्द्रकला& भैरवमिश्रः \\
& मल्लिनाथः \\
चिदस्थिमाला & वैद्यनाथपायगुण्डः~॥
\end{tabular}
\end{multicols}

\fancyhead[CO,CE]{सार, मध्य~। लघुकौमुदीनिर्माता वरदराजः~।}
\fancyhead[RO,LE]{\thepage}
\cfoot{}
\newpage
%%%%%%%%%%%%%%%%%%%%%%%%%%%%%%%%%%%%%%%%%%%%%%%%%%%%%
\renewcommand{\thepage}{\devanagarinumeral{page}}
\setcounter{page}{23}

% सार, मध्य~। लघुकौमुदीनिर्माता वरदराजः~। २३ 

\begin{multicols}{2}
ततश्च दैयाकरणसिद्वान्तकौमुद्या दुरवधेयतामवगत्य भट्टोजि \textendash\ भट्टान्तेवासिना दुर्गातनयतनयेन वरदराजेन सारकौमुदी \textendash\

\noindent
\rule{1\linewidth}{0.5pt}\\

\begin{tabular}{c c}
विलासः &भास्कररायकृतः\\
शब्देन्दुशेखरौ& नागेभट्टकृतौ \\
शब्दसागरः & \\
सरला & \\
& सुधाकरः\\
सुबोधिनी & जयकृष्णकृता\\
सुमनोरमा &तिरुमलकृता \\
& विश्चेश्वरतीर्थकृता~॥
\end{tabular}

\noindent
\rule{1\linewidth}{0.5pt}\\

1 तत्र बृहतः शेखरस्य व्याख्याः \textendash\ केटलोगस् कैटलोगोरम् ग्रन्थकृता प्रकाशिताः ग्रन्थकृता प्रकाशिताः \textendash\

\begin{tabular}{c c}
त्विद्रथी & \\
वृत्तिप्रदीपः& \\
उपन्यासः & \\
ज्योत्स्ना & उदयकरकृता \\
इन्दुप्रकाशः & जगद्धरकृतः \\
दोषोद्धारः & गोपालदेवकृतः
\end{tabular}

\columnbreak

\noindent
मध्यकौमुदी१ \textendash\ लघुकौमुदी२ \textendash\ गीर्वाणपदमञ्जरी \textendash\ नामानो ग्रन्था निर्मिताः~॥ 

\noindent
\rule{1\linewidth}{0.5pt}\\

१ मध्यकौमुदीव्याख्यास्तु \textendash\ 

\begin{tabular}{c c}
मध्यमनोरमा& \\
मध्यकौमुदीविलासः& 
\end{tabular}

२ लघुकौमुदीव्याख्यस्तु \textendash\ 

\begin{tabular}{c c c}
& & जयकृष्णकृता\\
लघुकौमुदीप्रकाशः & \multirow{2}{*}{$\big\}$}& \\
वैयाकरणसिद्धान्तचन्द्रोदय & &शिवदत्तदाधिमथकृतः
\end{tabular}

\noindent
\rule{1\linewidth}{0.5pt}\\

\begin{tabular}{c c}
विषमी* & राघवेन्द्राचार्यः\\
& राजारामदीक्षितः\\
& वल्लभः\\
& शंकरः\\
& शेषशास्त्री\\
& सदाशिवभट्टः\\
& हरिरामः
\end{tabular}

*विषमीप्रभृतयो व्याख्या लघोरेव वर्तन्ते उपलभ्यन्ते च, बृहतस्त्वस्माभिरनुपलब्धा अप्यनेन नामसंग्रह C.C. कर्त्रोपलब्धा भवेयुः~॥
\end{multicols}

एचमनन्तशाखतां गतं पाणिनीयमेतद् व्याकरणं गुरूपासनयैवाविकृतबुद्धित्वसंपादकम् आर्षानार्षभाषाप्रयुक्तशब्दानां शास्त्रान्वितत्वसंपादनेन एकः शब्दः शास्त्रान्वितः सुप्रयुक्तः स्वर्गे लोके कामधुग् भवति इति श्रुतिबोधितस्वर्गलोके कामधुक्त्वसंपादक भवतीति सर्वं कल्याणम्~॥

\noindent
\rule{1\linewidth}{0.5pt}\\

१ तत्र आर्षपदेन न मन्त्रब्राह्मणात्मको वेदो महतो भूतस्य निश्वसितरूपो गृह्यते अनार्षपदेन वेदभिन्नो वेदतुल्यो महावाक्यरूपवेदघटकपदविभागदर्शकः पदक्रमादिनाम्ना विख्यातो अन्थः समधिगमनीयः~॥ {\qt न लक्षणेन पदकारा अनुवर्त्याः~। पदकारैर्नाम लक्षणमनुवर्त्यम्~। यथालक्षणं पदं कर्तव्यम्} इति भ~। ष्यग्रन्थस्य यथाश्रुतार्थाङ्गीकारे तु पदमात्रसंस्कारकसूत्रनिर्माणमेव व्यर्थं स्यात्~। {\qt महो वा छन्दस्यानङोsवग्रहदर्शनात्} इत्यादिवार्तिकानां च वैयर्थ्यं स्यात्~॥ तस्माद् {\qt अनुवर्त्ये} णिजङ्गीकरणीयः~। {\qt यथालक्षणम्} इत्यस्य {\qt च} ष्णान्ता षट् इति सूत्रे प्रदीपोक्ता {\qt नैव वा लक्षणम् \textendash\ } इति सिद्धान्तभूतैव व्याख्या स्मर्तव्या, {\qt प्रयुक्तानां हीदमन्वाख्यानम्} इति भाष्यस्य संकोचे कारणाभावात्~। तथा च पदानुसार्येव लक्षणमुपज्ञातव्यम्~। पदपाठश्च गुरुसंप्रदायसिद्ध एव करणीयः इत्येव भाष्य कृतामपि सिद्धान्तः~॥ अत एव {\qt पृषोदरादीनि \textendash\ } इति सूत्रस्थभाष्येण न विरोधः~॥ कैयटस्तु भाष्यस्यास्य प्रौढिग्रन्थत्वं मत्वा यथाश्रुतार्थ प्रकाश्यानिष्टवारणायाग्रे संप्रदायानुसरणमवश्यमङ्गीकरणीयमिति सिद्धान्तितवान्~॥ भाषापदेन {\qt स्मृति \textendash\ पुराण \textendash\ काव्यादयो ग्राह्याः~। तत्र स्मृतिपुराणेषु तु छन्दोवत् कवयः कुर्वन्ति} इति वैदिका अपि शब्दाः प्रयुज्यन्ते इति सर्वमनवद्यम्~॥

\begin{center}
{\small इति संध्यापयति \textendash\ }\\

\textbf{\large शिवदत्तो दाधिमथः~॥}
\end{center}

 [ इत्येवं व्याकरणसंसारस्य संक्षेपेण परिचयः श्री म. भ. शिवदत्तकृतो विद्यर्थिनामुपयोगायान्यूनाक्षरैरुपन्यस्तः ~॥ ] 

\newpage
\thispagestyle{empty}
\begin{center}
\textbf{\Large पञ्चमावृत्तिप्रस्तावः~।}
\end{center}

\noindent
\rule{1\linewidth}{0.5pt}\\

चराचरगुरोर्भक्तवत्सलस्य विश्वनाथस्य भगवतः परमानुग्रहेण व्याकरणमहाभाष्यस्य प्रथमो भागो नवाह्विकाभिधो महता सम्भारेण पुनः सम्पाद्य वाचककरकमलेषु वितीर्यते~। इदानीं वितीर्यमाणं नवाह्निकमवलोक्य प्रश्नोsयं समुदेति यत् भाष्यप्रणयनतः प्राक् का व्याकरणस्य गतिरासीत्, किं मुनित्रयेणैव व्याकरणं प्रवर्तितं, उत ततः प्रागप्यासीत्, उत पूर्वतनव्याकरणानि चन्द्रादिप्रणीतानि प्रामादिकान्यासन्, अतो मुनित्रयेण प्रमादरहितं लोकवेदोभयोपकारकं सुगमोपायेन सम्पादितमित्यादिशङ्काकुलितं चेतः कल्लोलास्फालिते संशयसागरे विघूर्णमानमितः समाश्वासं लभते \textendash\ {\qt बृहस्पतिश्च प्रवक्ता इन्द्रश्चाध्येता दिव्यं वर्षसहस्रमध्ययनकालः}~। लघुतरं व्याकरणप्रणयनप्रकारं प्रकट्यता भाष्यकारेण प्रतिपदपाठोऽपि व्याकरणप्रकार एक उक्तः~। ज्ञायतेsनेन \textendash\ पूर्वतनैर्ऋषिप्रवरै र्नानाविधानि व्याकरणानि प्रणीतानीति~। भगवतः प्रत्यक्षदर्शिनो वाल्मीकेः \textendash\

\begin{quote}
{\qt असौ पुनर्व्याकरणं ग्रहीष्यन् सूर्योन्मुखः प्रष्टुमनाः कपीन्द्रः~।\\
उद्यद्विरेरस्तगिरिञ्जगाम ग्रन्थं महद्धारयनप्रमेयः~॥

ससूत्रवृत्त्यर्थपदं महार्थं ससंग्रहं सिध्यति वै कपीन्द्रः~॥}
\end{quote}

इत्यादिवचनैः प्राचीनव्याकरणग्रन्थानामपि सूत्रवृत्तिसंग्रह इत्यादीन्येव नामान्यासन् \textendash\ इति~। {\qt धाता यथापूर्वमकल्पयत्} इति श्रुतिमनुसरन्तो वैयाकरणाः पाणिन्यादयः सूत्राणि संग्रहादीनि च पूर्वव्याकरणानुबन्धीन्येव प्रकाशितवन्तः~। अतः पूर्वं व्याकरणमेव नासीदित्यपि न, स्वकपोलकल्पितं वा किञ्चित्पाणिन्यादिभिः कथ्यत इयमप्याशङ्का नावतरति~॥\\

पूर्वतनानि व्याकरणानि विहाय किमर्थमेतन्नूतनंविधीयत इत्याशङ्कायां द्विधाsत्र समाधातुं शक्यते~। पस्पशाह्निके भगवता वृहस्पतिश्च प्रवक्तेत्यादिना व्याकरणनिर्माणप्रकारमेकमुपपाद्य पूर्वतनेभ्यो व्याकरणेभ्यो व्यावृत्तं पाणिनीयव्याकरणविशेषगुणं प्रख्यापयितुं भाष्ये \textendash\ {\qt कथं तर्हीमे शब्दाः प्रतिपत्तव्याः} इत्याशङ्क्य {\qt किश्चित्सामान्यविशेषवल्लक्षणं प्रवर्त्यम्~। येनाल्पेन यत्नेन महतो महतः शब्दौघान् प्रतिपद्येरन्} इत्युक्तम्~। अस्य विशेषविमर्शने वक्तुं शक्यमेतत् यत् \textendash\ पाणिनेः पूर्वतनानि व्याकरणानि नैवंभूतानि, किन्तु श्रद्धावद्भिरनसूयद्भिः कर्तव्यमिति बुद्ध्या पठिष्यमाणानि बहुकालव्ययसाध्यानि आसन्निति~। तादृशानि च सुकुमारबुद्धीनां दुरवगाहानि \textendash\ इति व्याकरणागमभ्रंशमाशङ्कता भगवता पाणिनिना सुकुमारबुद्ध्यमुग्रहाय उत्सर्गापवादरूपं लघुभूतमेतद् व्याकरणमा \textendash\ विष्कृतमिति~॥ अथवा \textendash\ वाक्यपदीये भाष्यप्रणयनप्रसङ्गो यादृश उपवर्णितस्तादृश एव व्याकरणप्रणयनप्रसङ्ग इति~। तथाहि वाक्यपदीये द्वितीये काण्डे \textendash\

\begin{quote}
{\mbh प्रायेण संक्षेपरुचीनल्पविद्यापरिग्रहान्~। सम्प्राप्य वैयाकरणान् संग्रहेऽस्तमुपागते~॥\\
कृतेऽथ पतञ्जलिना गुरुणा तीर्थदर्शिना~। सर्वेषां न्यायबीजानां महाभाष्ये निबन्धने~॥~४८५~॥} इति~।
\end{quote}

अत्र चेत्थमाह पुण्यराजः \textendash\ इह पुरा पाणिनीयेऽस्मिन् व्याकरणे व्याड्युपरचितं ग्रन्थलक्षपरिमाणं संग्रहाभिधानं निबन्धनमासीत्~। तच्च कालवशात् सुकुमारवबुद्धीन् वैयाकरणान् प्राप्यास्तमुपागतम्~। तस्मात्क्लेशभीरुत्वात्संक्षेपरुचयस्ते जनाः~। तैः संग्रहाध्ययन उपेक्षिते सति अस्तं यात्तः संग्रहः इति~॥\\

द्वितीयकाण्डविषयप्रदर्शनवेलायाश्चेत्थमाह पुण्यराजः \textendash\ 

\begin{quote}
{\qt अवतारोऽपि भाष्यस्य संग्रहेऽस्तमुपागते~।\\
निबन्धहेतौ शास्त्रस्य टीकाकारेण कीर्तितः~॥

संग्रहार्थाद्यनुगुणरूपत्वं चोपपादितम्~॥} इति~।
\end{quote}

एवच्च सुकुमारबुद्धीनध्येतॄन् सम्प्राप्य ते ते व्याकरणागमास्तस्मिंस्तस्मिन्नवसरे शनैः शनैरस्तमुपयातास्तैस्तैराचार्यैः पुनः पुनः प्रकृतिमवस्थापितास्तत्तद्रूपाणि भजन्ते~। अत एव \textendash\ इन्द्रचन्द्रादयो व्याकरणकारा बहवः श्रूयन्ते~। समयप्राबल्यात् पूर्वव्याकरणेष्वस्तमुपयातेषु साध्वसाधुशब्दविषये विप्लवे समुपजाते पाणिनीयमेतत्प्रवृत्तमिति समुचितमेव समुपपद्यते~॥ 

\begin{center}
\textbf{\large मुनित्रयस्य समानासमानकालिकत्वम् \textendash\ }
\end{center}

अथ पाणिनीयव्याकरणे सूत्रवार्तिकभाष्यकारास्त्रय एव ऋषयः प्रमाणमित्ययं सार्वजनीनः सिद्धान्तः~। परं त्वेते ऋषयः कदा भूमिमिमामलंबुर्वन्ति स्मेत्येतद्विषये कालगणनापटुर्भिर्नव्यैर्भूरि पराक्रान्तं, प्रथमसंस्करणसम्पादकैः शिवदत्तपण्डितैश्च स्वीयोपोद्धाते यथाशक्त्युपवर्णितच्च~। अतस्तद्विगुपलब्धये शिवदत्तोपोद्धातः साकल्येनात्र संगृहीतः~। बहुशो बहुभिर्वर्णितेऽस्मिन्विषये विद्वज्जनानां स्वातन्त्रयमनिर्बाधं संकल्प्य सूत्रकारवार्तिककारयोः प्रायः समानकालिकत्वं सूत्रभाष्यकारयोश्चासमानकालिकत्वं युक्त्या लोकाचारेण वा यादृशमवभासते तत्प्रदर्श्थते~॥ तच्च यथा \textendash\ पाणिनीयस्य व्याकरणस्य वेदाङ्गत्वेन स्वीकारः, अष्टाध्याय्या वैदिकपाठे संग्रहश्च यदैवाभूत् तदैव वार्तिककारोपनिबद्धा योगविभागाः प्रमाणत्वेन तत्पाठे स्वक्रियन्ते~। भाष्यकारकृतयोगविभागाः सूत्रविपर्यासादयो वा न तथा स्वीक्रियन्ते~। पाणिनिना सूत्राणि दण्डकरूपेणैव प्रणीतानि, न तत्र योगविभागाः~। वैदिकपाठे च विभक्तान्येव सूत्राणि

१ प्र० पा० प्रस्ता०

\fancyhead[LE,LO,RE,RO]{}
\fancyhead[CE,CO]{\thepage}
\cfoot{}
\newpage
%%%%%%%%%%%%%%%%%%%%%%%%%%%%%%%%%%%%%%%%%%%%%%%%%%%%%
\renewcommand{\thepage}{ ( \devanagarinumeral{page} ) }
\setcounter{page}{2}
% ( २ ) 

\noindent
पठ्यन्ते~। तत्र वैदिकपाठविमर्शे स्फुटतरमेतत्प्रतिभासते \textendash\ वार्तिकोक्ता योगविभागाः सर्वेऽपि वैदिकैर्विभक्तत्वेन पठ्यन्ते \textendash\ इत्येव न, किन्तु वार्तिके ये सिद्धान्तभूता योगविभागास्त एव वैदिकैर्गृहीताः~। ये च पूर्वपक्षाभिप्रायिणस्ते वैदिकपाठे न सन्निबद्धाः~। भाष्यकृत्कृताश्च विभागाः सिद्धान्तभूता अपि वैदिकपाठे नैवोपनिबद्धा दृश्यन्ते~। तादृशानि सर्वप्रकाराणि कतिपयान्युदाहरणानि प्रदर्श्यन्ते~॥\\

यथा \textendash\ उञः~। ऊँ १~। १~। १७~। इति सूत्रे वार्तिककारेण {\qt उञ् इति योगविभागः} इति वार्तिकेन स्फुटतरं विभागः प्रतिपादितः स तथैव वैदिकपाठेऽप्यनुगृह्यते~। श्रीमता सूत्रकृता संहितयैवपठिताऽष्टाध्यायी~। तत्र वृत्तिकारादिभिस्तट्टीकाकृर्द्भिः सूत्राणामिदा \textendash\ नीमुपलभ्यमानो विभागः कृतः~। यत्र टीकाकृद्भि संशयादिनाsनवधानेन वा विभागो न प्रदर्शितस्तत्र भगवता वार्तिककारेण विभागः प्रदर्श्यते~। ते च सिद्धान्तभूताः सर्वेsपि वार्तिककृद्योगविभागा वैदिकैरनुगृह्यन्ते~। तेन वार्तिकप्रणयनान्तरमेव वेदाङ्गत्वेन प्रवेशोsष्टाध्याय्याः समजनीति विज्ञायते~। {\qt उञः~। ऊँ} इति सूत्रविषये संशयादिनैव वृत्तिकारादिभिर्योगविभागो न कृतः~। अत एव भाष्ये {\qt किमर्थो योगविभागः} इत्याशङ्क्य {\qt ऊँ वा शाकल्यस्य} इति वार्तिकमवतारितम्~। कैयटेनापि {\qt ततश्च विति ऊँ इति \textendash\ } एते द्वे एव रूपे स्याताम्~। तस्माच्छाकल्यग्रहणानुवृत्त्या आदेशविकल्पे सति {\qt त्रीणि रूपाणि सिध्यन्ति} इत्युक्तम्~। उद्योतकृता च तथैव भाष्यस्वरसादिति ऊँ इति उ इति च रूपे स्यातां न तु वितीति तदर्थं विकल्पानुवृत्तिः इति योगविभागः समर्थितः~। योगविभागफलविषये कैयटोद्द्योतयोः सुस्पष्टमुपलभ्यमानेन फलभेदेन वृत्तिकाराणां व्यामोहोऽपि एतद्विषये समर्थितप्राय एव~। वैदिकपाठेऽस्य योगविभागस्य स्फुटमुपलभ्यमानत्वात् विज्ञातुं शक्यमेतत् \textendash\ सूत्रप्रणयनानन्तरं न चिरादेव वार्तिकानामवतार इति~॥\\

ह्वः सम्प्रसारणम् ६~। १~। ३२~। इति सूत्रे *{\qt ह्वः सम्प्रसारणे योगविभागः~। ह्वः सम्प्रसारणे योगविभागः कर्तव्यः \textendash\ ह्वः सम्प्र \textendash\ सारणम्, अभ्यस्तस्य चेति} इत्येवं वार्तिककृता प्रदर्शितो योगविभागो लोकेऽप्याश्रीयते~॥\\

अचो ञ्णिति ७~। २~। ११५~। इति सूत्रे {\qt *योगविभागः सखिव्यञ्जनाद्यर्थः~। योगविभागः क्रियते सख्यर्थो व्यञ्जनाद्य्थश्च} इति वार्तिककृतोक्तम्~। पूर्वं {\qt अचो ञ्णित्यत उपधायाः }इति सूत्रमासीत्, वार्तिकेन प्रदर्शिते योगविभागे लोके विभक्ते एव ते सूत्रे पठ्येते~॥\\

औदच्च घेः ७~। ३~। ११८ \textendash\ ११९~। इत्यत्र *औत्वे योगविभागः~। औत्वे योगविभागः कर्तव्यः~। औद्भवति \textendash\ इदुद्भयाम्~। ततोऽच्च घेः~। इत्युक्तं वार्तिककृता~। सोsपि योगदिभागो लोके तथैव स्वीकृतः~॥

वार्तिकप्रदर्शिताः सर्वेऽपि योगविभागा वैदिके पाठे समाश्रीयन्त इत्येव न, किन्तु यत्र वार्तिककृता विशिष्टस्य कस्यचन शब्दस्य प्रत्ययस्य वोपसङ्ख्यानं क्रियते तदपि सूत्राङ्त्वैतैव वैदिकपाठे पापठ्यते~। यथा \textendash\ टिड्ढाणञ्द्वयसज्दघ्नञ्मात्रच्तयप्ठक्ठञ् \textendash\ कञ्क्वरपः ४~। १~। १५~। इति सूत्रे {\qt *ख्युन उपसङ्ख्यानम्~। ख्युन उपसह्ख्यानं कर्तव्यं~। आढ्यङ्करणी सुभगङ्करणी} इत्युक्तं वार्तिककृता~। अत्र हि ख्युनः सूत्रे उपसङ्क्यानस्योक्तत्वात् वार्तिककारेsपि पाणिनिवत्प्रामाण्यं विजानद्भिर्वैदिकैः {\qt टिड्ढाणञ्द्वयसज्दघ्नञ्मात्र \textendash\ च्तयप्ठक्ठञ्कञ्क्वरप्ख्युनाम्} इत्येव सूत्रं पठ्यते~। ज्ञायते चैतेन व्यवहारेण \textendash\ {\qt सूत्राणो वैदिकपाठे पदप्राप्तिसमये वार्तिकानां प्रचुरः प्रचार आसीत्} यतो वार्तिककृतमप्युपसङ्ख्यानं सूत्रेषु योजितम्~॥ अन्यान्यप्येवं प्रकाराणि बहूनि स्थलानि वैदिकपाठे दृश्यन्ते येन वैदिकपाठे वार्तिकप्रचारस्य पर्याप्त आवेशः~॥\\

ननु वार्तिकोक्ता अपि योगविभागाः साकल्येन वैदिकैर्नाश्रीयन्ते~। यथा \textendash\ नाव्ययीभावादतोऽम् त्वपञ्चम्याः २~। ४~। ८३~। इति सूत्रे {\qt नाव्ययीभावादत इति योगव्यवसानम्~। नाव्ययीभावादत इति योगो व्यवसेयः~। नाव्ययीभावादकारान्तात्सुपो लुग्भवति~। ततः \textendash\ अम् त्वपञ्चम्या इति} इत्युक्तं वार्तिककृता~। लोके च नाश्रीयतेऽयं विभाग इति वार्तिकोक्ताः सभाश्रीयन्त इत्यपि न, इति चेन्न~। ये वार्तिककारेण सिद्धान्तभूता विभागाः प्रदर्शितास्त एव लोके समाश्रीयन्ते, न तु पूर्वपक्षीभूता अपि विभागाः~। अत्र च {\qt स तर्हि योगविभागः कर्तव्यः~। न कर्तव्यः~। तुर्नियामकः} इत्यादिना योगविभागस्य विफलतां प्रदर्शयति~॥\\

अत एव व्यत्ययो बहुलम् ३~। १~। ८५~। इति सूत्रे {\qt *योगविभागः~। योगविभागः कर्तव्यः \textendash\ व्यत्ययः, ततः \textendash\ बहुलम्} इति वार्तिककृता प्रदर्शितमपि योगविभागं न सभाजयन्ति वैदिकाः~॥\\

एवमेव लृटः सद्वा शश३~। ३~। १४~। सार्वधातुके यक् ३~। १~। ६७~। युष्मदस्मदोरन्यतरस्यां खञ्च ४~। ३~। १~। प्रथमयोः पूर्वसवर्णः ६~। १~। १०२~। इत्यादिसूत्रव्याख्यानावसरे वार्तिकप्रदर्शिता अपि योगविभागाः सिद्धान्तभूता नेति नाद्रियन्ते~॥\\

एवञ्च वार्तिकोक्ताः सिद्धान्तभूता योगविभागा यत आश्रीयन्ते, अतो निश्चयेनैतद्वक्तुं शक्यं \textendash\ प्रायः सूत्रसमय एव वार्तिकावतार इति न सूत्रवार्तिकयोः कालव्यवधानमिति~॥\\

भाष्यकृत्प्रदर्शिताश्च न्यासान्तराः सूत्रभेदा योगविभागा वा लोके नाश्रीयन्ते~। यथा \textendash\ उपसर्गादनोत्परः ८~। ४~। २८~। इति सूत्रं भङ्क्त्वा भाष्यकारः {\qt उपसर्गाद्वहुलम्} इत्याह~। सिद्धान्ताभिप्रायश्चायं तथापि भाष्यप्रणयनात्पूर्वमेव वैदिकपाठस्य निश्चयात् न भाष्यसिद्धान्तः पाठे समाश्रीयते, किन्तु {\qt उपसर्गादनोत्परः} इत्येव तेषां पाठः~॥\\

इको गुणवृद्धी १~। १~। ३~। इति सूत्रे भाष्ये {\qt मृज्यर्थमिति चेद्योगविभागात्सिद्धम्~। मृजेवृद्धिरचः~। ततो ज्णिति~।} इत्येवं विभागः प्रदर्शितोsपि लोके तस्यानाश्रयणम्~।

\newpage
% ( ३ ) 

न धातुलोप आधधातुके १~। १~। ४~। इति सूत्रभाष्ये अथापि कथंचिदनवकाशो सुक् स्यादेवमपि न दोषः~। अल्लोपे योगविभागः करिष्यते~। अतो लोपः, ततो यस्य, ततो हलः, इति प्रदर्शतो योगव्रिभागो नैव लोके समाश्रितः~। अत्र {\qt अल्लोपे योगविभागः} इति भाष्ये अल्लोपसमीपे यस्य हल इत्यत्रेत्यर्थः~। अल्लोपे अतो लोपः इत्येवं विभागस्तु सावजनीन इति स नात्र प्रदर्शनविषयः, तदुपन्यासस्तु लोपपदानुवर्तनप्रदर्शनायैव~। अत एव छायायां {\qt अल्लोप इति~। तत्समीपे यस्य हलं इत्यत्रेत्यर्थः~। वटे गाव इतिवत्सप्तमी} इत्युक्तम्~॥\\

अद्सो मात् १~। १~। १२~। इति सूत्रभाष्ये अथवा योगविभागः करिष्यते \textendash\ अदसः~। अदस इदादयः प्रगृह्यसंज्ञा भवन्ति~। ततो {\qt मात्} माच्च परे ईदादयः प्रगृह्यसंज्ञा भवन्तीति इत्युक्तम्~। तादृशश्च योगविभागो लोके नाश्रितः~॥\\

ष्णान्ता षट् १~। १~। २४~। इति सूत्रभाष्ये {\qt एवं तर्हि सप्तमे योगविभागः करिष्यते \textendash\ अष्टाभ्य औश्, ततः \textendash\ षड्भ्यः, ततो लुक्} इत्येवं प्रदर्शतोsपि योगविभागो लोके नावतरति~। तदमग्रे च {\qt अथवा \textendash\ उपरिष्टाद्योगविभागः करिष्यते \textendash\ अष्टन आ विभक्तौ, ततो रायः, हलिः} इति द्विधा समाधाने समुपन्यस्तोsप्ययं विभागो वैदिकपाठसम्बन्धं नाधिगच्छति~॥\\

एवमादीनि बहून्युदाहरणानि प्रदर्शितान्यपि चिकित्सकैर्नं सिद्धान्तीभूतानीमानीति नाद्रियन्ते चेत् यत्र सिद्धान्तेऽपि विभाग एव समाश्रीयते तादृशोऽपि पाठविषयतां न लभते~॥\\

यथा \textendash\ श्लिष आलिङ्गने ३~। १~। ४६~। इति सूत्रे भाष्ये ननु चोक्तं श्लिष आलिङ्गने नियमानुपपत्तिर्विधेयभावादिति~। नैष दोषः~। योगविभागात्सिद्धम्~। श्लिषः, ततः \textendash\ {\qt आलिङ्गने} इत्येवं प्रदर्शितो विभागः सर्वैः समाश्रयणीयोsपि पाठे पदं न धत्ते~॥\\

ननु गोत्रे कुञ्जादिभ्यश्च्फञ्४~। १~। ९८~। इति सूत्रे भाष्ये एवं तर्हि स्वरे योगविभागः करिष्यते \textendash\ {\qt चितः} ६~। १~। १३३~। चितोऽन्त उदात्तो भवतीति, ततः \textendash\ तद्धितस्य, ततः \textendash\ कितः~। इति प्रदर्शितो योगविभागो लोकेप्sयाश्रीयत इत्यपि न मन्तव्यम्~। नायं भाष्यकृत्कृतो योगविभागः, वृत्तिकारादिभिर्वार्तिककारापेक्षयाऽपि पूर्वं प्रदर्शितस्य योगविभागस्य भाष्ये प्रदर्शनमात्रमत्र कृतम्~॥\\

किमिदंभ्यां वो घः ५~। २~। ४०~। इत्यत्र किमिदंभ्यां वतुपोऽप्राप्तिमाशङ्क्य {\qt अथवा योगविभागः करिष्यते~। किमिदंभ्यां~। वतुप् भवति~। ततः \textendash\ वो घः~। वश्चास्य घो भवतीति?} इत्युक्तं भाष्यकृता~। वतुप्प्रत्ययविधानाय सामर्थ्यं वा कल्पनीयम्, योग \textendash\ विभागो वा विधेयः~। तत्र सामर्थ्यकल्पनापेक्षया योगविभागस्य लघुत्वाद्युक्तोऽपि योगविभागो नाश्रीयते वैदिकैः~॥\\

कुरुगार्हपतरिक्तगुर्वसूतजरत्यश्लीलदृढरूपा पारेवडवा तैतिलकद्रूः पण्यकम्बलो दासीभाराणां च ६~। २~। ४२ इति सूत्रे भाष्ये कुरुगार्हपतरिक्तगुर्वसूतजरत्यश्लीलदृढरूपा पारेवडवा तैतिलकद्रूः पण्यकम्बलो दासीभारादीनामिति वक्तव्यम्~। इहापि यथा स्यात् \textendash\ देवहूतिः, देवनीतिः, वसुनीतिः~। तत्तर्हि वक्तव्यम्~। न वक्तव्यम्~। योगविभागः करिष्यते \textendash\ कुरुगार्हृपत \textendash\ रिक्तगुर्वसूतजरत्यश्लीलदृढरूपा पारेवडवा तैतिलकद्रूः पण्यकम्बलः~। ततः \textendash\ दासीभाराणाश्चेति~। तत्र बहुवचननिर्देशात् दासीभारा \textendash\ दीनामिति विज्ञास्यते{\qt इत्येवं योगविभाग उक्तो देवहूत्यादिलक्ष्यसिध्यर्थमुपयुज्यमानोsपि वैदिकपाठे नानुसन्धीयते~॥\\

एवमेव \textendash\ }आर्धधातुकस्येड्वलादेः, नपुंसकस्य झलचः, शीङो रुट् इत्यादिषु बहुषु स्थलेषु भाष्यप्रदर्शिता योगविभागाः सफला अपि यतो वैदिकपाठे नाश्रीयन्ते अतो भाष्यप्रणयनात्सुदीर्घ पूर्वो वैदिकपाठो बद्धमूलतामापन्न इति~। एवच्च सूत्रवार्तिककारयोर्न काल \textendash\ विप्रकर्षः, सूत्रवार्तिककारापेक्षया कियताsपि कालेन भाष्यकारो भाष्यं प्रणिनायेति वैदिकव्यवहारतो निश्चयेनावगन्तुं शक्यते \textendash\ इति~॥

\begin{center}
\textbf{\Large विषयोपन्यासः \textendash\ }
\end{center}

अथ महाभाष्यस्य भाषासौष्ठवं, कठिनतरस्यापि विषयस्य सुलभतया विवेचनं, सर्वत्र शब्दस्य शब्दार्थस्यापि वा सफलत्व \textendash\ मित्यादिगुणसमूहमवलोक्य कस्य चेतो न चित्रीयेत~। मीमांसान्यायशास्त्रादिटीकाग्रन्थानां केवल भाष्यशब्देनैव व्यवहारः, अस्य श्रीमद्भगवत्पतञ्जलिप्रणीतस्य च महाभाष्यशब्देन व्यवहारोऽपि महत्त्वबोधनायादरेण लोके क्रियते~। तथा चोक्तं पुण्यराजेन \textendash\ {\qt कृतेऽय पतञ्जलिना} ४८५ इति वाक्यपदीयकारिकाव्याख्याऽवसरे \textendash\ {\qt तच्च भाष्यं न केवलं व्याकरणस्य निबन्धनं यावत्सर्वेषां न्यायबीजानां बोद्धव्यमित्यत एव सर्वन्यायबीजहेतुत्वादेव महच्छब्देन विशेष्य महाभाष्यमित्युच्यते लोके} इति~। अस्याध्ययनाध्यापनप्रणालिरिदा \textendash\ नीमुच्छिन्नेव प्रतिभाति~। परकीयसत्तासमये परकीयभाषापराभूतप्रायेव भारतीया विद्या~। यत्र क्वचन काश्यादिविशेषस्थलेषु संज्ञामात्रेणावशिष्टा तत्रापि टीकाग्रन्थैरेव सर्वमप्यायुः परिपूर्णमिति क्वावशिष्टोऽवकाशो भाष्याध्ययनाय~। इदानीं स्वाधीने भारते प्रभविष्यति भारतीया विद्या, सुखी करिष्यति च संसारमिति न नः सन्देहः~। भारतीयानामभ्युदये समुचिते काले सुसम्पन्नस्यास्य महाभाष्यस्यावलोकनेन हृदये सम्परिवर्तते यत् महाभाष्यकारसमये का वा भारतस्य दशाssसीत्, कथञ्चैतद्देशीयाः कालं यापयन्ति स्म, कृष्यादिकं तदा कीदृशमासीत्, जातिविभागास्तदाऽऽसन्न वा, कथं वा तदानींतनानां मनुष्याणां स्वभाव आसीत्, देशे तदानीं प्रमाणादिव्यवहार आसीन्न वा, लोकव्यवहाराश्च कथमेतेषामासन् \textendash\ इति~। अस्य ज्ञानस्य प्रत्यक्षं तादृशसामग्र्यभावादशक्यमपि भाष्यप्रदत्तोदाहरणादिनाऽनुमातुं शक्यं स्यात्~। यतः स्वाभाविकोऽयं समयस्तत्कालव्यवहारप्रतिबिम्बं सामाजिकानां लौकिकेष्वाचारेष्वप्यवतरति किमुत विशिष्ट \textendash\ ग्रन्थप्रणयितृव्यवहारेषु~। यथा \textendash\ इदानींतनानां सामाजिकानां व्यवहाराः सत्याहिंसादिकाल्पनिकाचारानपि नातिक्रमन्ते, जात्यादिव्यु \textendash\

\newpage
% ( ४ ) 

\noindent
च्छेदाय सततं यतन्ते, लोककल्याणे ननु सर्वः समाज एव प्रवृत्त इति प्रतिभासयन्ति, किन्नु अधिकं! ये ये सभ्यानामिदानींतनानां व्यवहारास्तैः सर्वोsपि भारतीयानामितिहासो हस्तामलकवदिव प्रतिभाति~। तथैव भाष्यसदृशप्रेष्ठतमग्रन्थप्रणयितृव्यवहारा अपि तत्कालव्यपदेशायालमिति भाष्योदाहरणानि कानिचन विविच्य तात्कालिकीमवस्थामनुमातुं प्रत्ययते~॥

\begin{center}
\textbf{\Large वस्त्रनिमाणप्रकाराः \textendash\ }
\end{center}

तथाहि \textendash\ तद्धितार्थोत्तरपदसमाहारे च २~। १~। ५०~। इतिसूत्रभाष्ये {\qt कश्चित्तन्तुवायमाह \textendash\ अस्य सूत्रस्य शाटकं वयेति~। स पश्यति \textendash\ यदि शाटको न वातव्यः, वातव्यो न शाटकः, स मन्ये वातव्यो यस्मिन्नुते शाटक इत्येतद्भवति} इति~। एतदुदाहरणदर्शनेन वायकोsपि यस्मिन् देशे एतादृशो विचक्षणः यदि शाटको न वातव्य इत्यादि विचारयति, यत्र च सूत्रोत्पत्तिः सर्वैरपि क्रियते स्म येनायं प्रस्तुतो विचारो व्यवहारश्च श्रुतः श्रोत्रे सुखयति~। तदानीन्तनैस्तैः सूत्रमपि निष्पाद्यते वस्त्रमपि च~। सूत्रनिष्पादने विशै \textendash\ षसामग्र्या अभावात् यत्र तत्राssपामरः सर्वैरपि कर्तु शक्यं स्यात्~। न तथा वस्त्रवायनं, तच्च वायनसामग्र्यधीनमिति यत्र तत्र न कर्तुं पार्यत इति वायकसमीपे एवागमनमुल्लिख्यते भाष्ये~। अन्यथा कश्चित्कश्चिदाहेत्याद्येवोल्लिख्येत~।\\

तस्मिन् समये न केवलं सामान्यवस्त्राणामेवोत्पत्तिरासीत्, क्रिन्तु सूक्ष्मसूक्ष्मतरवस्त्राणामप्युत्पत्तौ कृतप्रयत्नास्तदीया जना दृश्यन्ते~।\\

यथा \textendash\ वर्णो वर्णेन २~। १~। ६८~। इति सूत्रे पूर्वपदातिशय इति वार्तिके {\qt किं प्रयोजनं, सूक्ष्मवस्त्रतराद्यर्थः~। यावता वस्त्राणि तद्वान्तमपेक्षन्ते, तद्वन्तं चापेक्ष्यवस्त्राणां वस्त्रैर्युगपत्स्पर्धा भवति~। यथैवायं द्रव्येषु यतते \textendash\ वस्त्राणि मे स्युरिति, एवं गुणेष्वपि यतते \textendash\ सूक्ष्मतराणि मे स्युरिति }~। एतद्दर्शनेनेदं सुस्पष्टं प्रतीयते यत् न केवलं वस्त्राणामेवोत्पत्तौ कृतकृत्यास्ते किन्तु गुणवन्ति सूक्ष्मतराणि च वायकैर्निमीयन्ते स्म, वस्त्रव्यवहारिणश्च गुणवद्वस्त्रव्यवहरणे दत्तचित्ताश्चासन्~। युज्यते चैतत्~। ये निरायदा वस्त्राणि गृह्नीयुस्ते यावन्ति गुणवन्ति सुशक्यानि तत्रैव कृतप्रयत्ना भविष्यन्ति न तु तेषां वस्त्रमात्रलाभेन समाश्वासः~॥ अन्यच्च यत्र सूक्ष्मतरवस्त्रनिष्पादनशक्तिस्तत्र सुधौतवस्त्रादिनिष्पादनमप्यावश्यकम्~। तादृशं चेदानीमैव, न तदाऽऽसीदित्यपि न मन्तव्यम्~। तदा यथा सूक्ष्मतरवस्त्रनिष्पादनै सामर्थ्यमासीत् तथैव सुधौतवस्त्रनिर्माणेsपि शक्तिरासीदिति प्रतीयते~। तच्च \textendash\ अतिशायने तमबिष्ठनौ ५~। ३~। ३५~। इत्यत्रोदाहरता भाष्यकृता स्पष्टीकृतमेतत्~। तत्र हि {\qt एवं हि दृश्यते लोके \textendash\ समाने आयामे विस्तारे पटस्यार्घोऽन्यो भवति काशिकस्य, अन्यो \textendash\ माथुरस्य~। गुणान्तरं च खल्वपि शिल्पिन उत्पादयमाना द्रव्यान्तरेण प्रक्षालयन्ति~। अन्येन शुद्धं धौतकं कुर्वन्ति, अन्येन शैफालिकम्, अन्येन माध्यमिकम्} इत्युक्तम्~। एकस्यैव वस्त्रस्य गुणोत्पादनवैशिष्ट्येन भिन्नानि मूल्यानि भवन्ति, ताश्च क्रिया भिन्नभिन्नदेशेषु भिन्ना भिन्ना आसन् \textendash\ इत्यनेन व्यवहारेण गम्यते \textendash\ काशिकस्य मूल्यमन्यत्, माथुरस्य चान्यदिति~। तत्र धौतशैफालिकमाध्यमिकाः क्रियाः प्रक्षालनवैजात्यकृता वस्त्रस्य गुणविशेषकराः~। तदेतच्च समानगुणे एव तादृशी स्पर्धा आसीत्, न गुणभेदकृता~। शुक्ल गुणस्य यन्मूल्यं तदेव कृष्णगुणस्येति नासीत्~। तदग्रे क्रियमाणे चापि गुणग्रहणे समानगुणग्रहणं कर्तव्यम्, शुक्लात् कृष्णे माभूत्~। न कर्तव्यम्~। समानगुणे एव स्पर्धा भवति, न त्याढ्याभिरूपौ स्पर्धेते~। इत्यनेन स्पष्टीकृतम्~। एवञ्च तदानीं सूक्ष्मतराणि सुधौतानि बहुमूल्यवन्ति वस्त्राणि सुलभान्यासन्~॥\\

यथेदानीमपि यन्त्रनिर्माणयुगे कश्मिश्चिद्यन्त्रे लघुमूल्यानि वस्त्राणि भवन्ति, कस्मिंश्चिद्यन्त्रे बहुमूल्यानि~। धावनविशेषप्रकारैरपि मूल्ये तरतमभावोऽनुभूयते जनैः~। यादृश एवे दानीन्तरतमभावः स्पर्धा वा तादृश्येव तदानीमपि~। एतावता वस्त्रव्यवहारश्च तदसुलभतर आसीदित्यवगम्यते~॥\\

साम्प्रतिके काले वस्त्रव्यवहारो यथा व्युच्छिन्न इव दृश्यते तत्रापि महताकारणेन भवितव्यम्~। इदानीं वस्त्रोत्पत्तिर्महद्यन्त्राधीना, तानि च यन्त्राणि प्रदेशविशेषेष्वेवावतिष्ठन्ते, न तु सर्वत्र~। ये च तदुत्पादनसमर्थास्तत्र तेषां काचन स्पर्धैव नास्तीति निरुत्साहास्ते यद्भविष्यति तद्भविष्यतीति वादिनः खोदरपूर्ती कदाचिद्वस्त्राणि निष्पादयेयुः, कदाचिदन्यद्वा किञ्चिदुत्पादयेयुः~।\\

अन्यच्च यन्त्रेष्वपि द्विविधमाधिपत्यमिदानीं दृश्यते~। एकत्र च धनिकायत्तम्, अन्यत्र च कर्मकरायत्तम्~। तेन गुणवानप्येको भृत्य उभाभ्यां खामिभ्यामेकस्मिन्नेव समये बिरुद्धदिक्स्थे कर्मणी आज्ञाप्येयातां तदोभयोरपि न करोति, तथा यन्त्रयुगे समापन्नमिव दृश्यते~। प्राक्तने च काले एतन्नासीत्, यतस्तत्र जातिनिर्बन्धादिना नियतान्येव कर्माण्यासन्~। वायकास्तावद्वस्त्राण्येव वयन्ति, न ते कदाचिदपि अन्यदीयकर्मणा व्यवहरन्ति~। यदि चेत्ते न वस्त्राणि कुर्युस्तदाsकृत्रिमनिर्वन्धादिनाsन्यकर्मकरणेऽशक्ताः स्वोदर \textendash\ पूरणेसमर्था भवेयुः~। यत्र जातिनिर्बन्धो दृढो भवेत् तत्र स्वीयकर्मकरणे उत्साहवन्तः स्वकर्तव्यमिति बुध्या तत्र प्रवर्तमानाः स्वे खे कर्मणि निरता जना भवन्ति~। तत्रापि भिन्नदेशस्थानां सजातीयानां कर्मणि विशेषकौशलादिकमवलोक्य परस्परं पास्पर्धमाना उत्कृष्टगुणवन्तो भवन्ति~। यत्र स्पर्धैव नास्ति तत्रेदानीं शर्कराविषये यादृशीमवस्थामात्मा नीतः स्वकीयैः स न कदापि विस्मर्तव्यः~। यतन्ते च स्वार्थप्रवणा राजकीयकार्यकरा नीतिमन्त्तोऽपि पुरुषा अन्यत्रापि स्पर्धा विलोपयितुम्~। निदर्शनञ्चात्र राष्ट्रीयकरणमिषेण व्युच्छिन्नव्यवहाराणां कतिपयानानामेवालं स्यात्~। न च राजाज्ञया कस्मिंश्चित्कर्मणि सहजत्वं निर्मातुं शक्यम्, न वा नेदं ते सहजं कर्मेति वा विधातुं शक्यम्~॥

\newpage
% ( ५ ) 

नन्वेवं चेत्, राजाज्ञया न किश्चिदनिष्टं निष्पाद्यत इति चेज्ञ~। व्रिभ्रमवुद्धयो हि मनुष्याः~। ते राजाज्ञया किंकर्तव्यविमूढाः संशयितचित्तवृत्तयो न सहजं कुर्वन्ति, न वाऽऽज्ञप्ते कर्मणि सावधाना भवन्ति~। एवश्चेदृशराजाज्ञया देशविप्लव एव सर्वथाsभिसमीक्ष्यः स्यात्~। अत एवेदृशं विधानं युक्तं यत् तत्रत्यसमाजस्य तत्स्वाभाविकं स्यात् \textendash\ इति~॥

\begin{center}
\textbf{\Large लौकिको व्यवहारः \textendash\ }
\end{center}

लोकव्यवहारनिदर्शक समर्थः पदविधिः २~। ९~। १~। इति सूत्रे भाष्यकृता समीचीनमुदाहरणं दत्तम्~। तद्यथा \textendash\ {\qt एवं हि दृश्यते लोके भिक्षुकोऽयं द्वितीयां भिक्षां समासाद्य पूर्वा न जहाति, सञ्चयायैव प्रवर्तते} इति~। भिक्षुको हि धनसंग्रहे लुब्धः कथं वा पूर्वां भिक्षां जह्यात्~। अनेनोदाहरणप्रदर्शनेन समाजस्य स्वाभाविकं रूपं सन्दृश्यते~॥ ५६प्राग्रीश्वरान्निपाताः १~। ४~। ५६ इत्यत्र किमर्थ रेफाधिक ईश्वरशब्दो गृह्यते लोकत एतत्सिद्धम्, तद्यथा आ वनान्तात् आ उदकान्तात् प्रियं पान्थमनुव्रजेदिति य एव प्रथमो वनान्त उदकान्तश्च ततोऽनुव्रजन्ति~। लौकिकं चातिवर्तते~। द्वितीयं तृतीयञ्च वनान्तमुदकान्तमनुव्रजन्ति तस्माद्रैफाधिक ईश्वरशब्दो प्रग्रहीतव्यः इति~। अत्र हि रीश्वरादिति किमुक्तं, ईश्वरादित्येव कथं नोच्येतेत्याशङ्ख्य, आ वनान्तादा उदकान्तात्प्रियं पान्थमनु \textendash\ ब्रजेदिति विधाने नियमाभावाद्वितीयं तृतीयं च वनान्तमनुगच्छन्ति, तथाऽत्रापि प्रथमेश्वरशब्दं विहाय द्वितीयपर्यन्तमपि अनुत्व्रजिष्य \textendash\ न्तीति तदर्थ विकृतमिति तद्भाष्यार्थः~। ग्रामाद्भानान्तरं गच्छतः पान्थस्यानुव्रजनं उदकान्तं वनान्तं वा कर्तव्यमित्याचार आसीत्तदा \textendash\ नीम्~। ग्रामाद्वहिरपि मध्येमार्ग विपुला उदकान्ता विपुलानि वा वनानि यत्र सम्भवन्ति तत्रैवेदमुचितं स्यात्~। तत्रापि द्वितीयं तृतीयं वा वनान्तं गच्छन्तीति दर्शनेन च नियतं तस्मिन् काले बहूनि वनानि संरक्षितानि निर्मितानि वा भवेयुः~। पान्थानामुपभोगाय धार्मिकैर्निर्मिता बहवो जलाशया वा भवेयुरिति सुजलां सुफलामिति भारतभूस्वरूपं दृष्टिपथमायाति~॥\\

तदानींतनाश्चधार्मिका राजदण्डे समादरयुक्ताश्च लोका इत्यग्रिमोदाहरणदर्शनेन प्रतीयते~। यथा \textendash\ वारणार्थानामीप्सितः १~। ४~। २७~। इत्यत्र किमुदाहरणं ? माषेभ्यो गां वारयति~। भवेद्यस्य माषा न गावस्तस्य माषा ईप्सिताः स्युः, यस्य तु खलु गावो न माषाः कथं तस्य माषा इप्सिताः स्युः~। तस्यापि माषा एवेप्सिताः~। आतश्चेप्सिताः, यदेभ्यो गां वारयति~। पश्यत्ययं यदीमा गावस्तत्र गच्छन्ति ध्रुवः सस्यविनाशः, सस्यविनाशेsधर्मश्चैव राजभयञ्च~। स बुध्या सम्प्राप्य निवर्तयति इति~। क्षेत्रे गवां सम्मर्देन सस्यविनाशः स्यात्, विनष्टे च सस्ये क्षेत्रस्वामिनः क्लेशेनाधर्मः स्यादिति येषां बुद्धिस्तेषां खलु भारतीयानां कथममङ्गलाशंसनमपि सम्भवति~। गो सञ्चारयितुः पांशुलपादस्य यत्रेदृशं धर्मबन्धनं तत्र सुनिविष्टानां नागरिकाणां किमु वक्तव्यम्~। वस्तुत अरण्यै संस्थि \textendash\ तस्य क्षेत्रस्य रक्षणमपि धर्म एव करोतीति धर्गो रक्षति रक्षितः इति वचनं कथमन्यथा मन्तव्यम्~। {\qt अधर्मश्चैव राजभयञ्च} इत्युभयमप्युक्तं परंतु तत्र राजभयस्य न तादृशः समादरो यादृशो धर्मस्य, अत एव तस्य पूर्वमुक्तिः~। एतदपि तत्रावधेयं \textendash\ तदा सस्यविनाशेऽपि राजकीयः प्रतीकारः समर्थतर आसीत् न निर्विण्णः, येनारण्यकेनापि राजभयात्क्षेत्ररक्षणं क्रियते~। अत्रापि न राजा रक्षति किन्तु धर्म एव रक्षतीत्येव सत्यम्~। यद्यसत्याश्रयणं कर्तुं शक्यते तदा गाश्चारयित्वा क्षेत्रविनाशेsपि न मदीयगोभिस्तत्क्षेत्रं नाशितम्, अहं तु तस्मिन्दिने तत्क्षेत्रं नैव दृष्ट्वानित्याद्युक्तिभिःस्वसंरक्षणं कर्तुं शक्यम्~। राजभयं चैति प्रतिपादयता भाष्यकारेण जनानां धर्मानुसन्धानात् सत्यप्रियत्वमाविष्क्रियते, तदैव च राजभयं च सम्भाव्यते~। अन्यथाsधार्मिके राज्ये यावन्तः शासनीया जनास्तावन्तो रक्षापुरुषान् विधायापि चौरैराततायिभिश्च प्रत्यहं राजधान्यां कतिपयमनुष्या हन्यन्ते, कानिचिच्च गृहाणि दह्यन्ते, केचिच्च मुष्यन्ते \textendash\ इत्यादि त्वनुभूयत एव व्यवहारकुशलैः~।\\

धर्मबन्धनादेव \textendash\ अर्थवदधातुरप्रत्ययः प्रातिपदिकम् १~। २~। ४५~। इत्यत्र तद्यथा लोके \textendash\ आढ्यमिदं नगरं गोमदिदं नगरमित्युच्यते~। न च सर्वे तत्राढ्या भवन्ति सर्वं वा गोमन्तः~। यस्य हि यद्द्रव्यं भवति स तेन कार्यं करोति~। यस्य च या गावो भवन्ति स तासां क्षीरं धृतमुपभुङ्कते, अन्यैरेतद्वृमप्यशक्यम्~। इत्युक्तं सङ्गच्छते~। धर्मबन्धनाभावेऽन्यस्यापि द्रव्यमन्येनोपभुज्यमानं प्रत्यहमुपलभ्यते, यस्य च गावो भवन्ति तद्दुग्धमन्यैरुपभुज्यमानमेव प्रत्यहं प्रतिपाल्यते~। एतदुदाहरणदर्शनेन नगरेऽपि गोपालनं बहूनामपि तदा सुकरमासीदिति गम्यते~। इदानीमपि नगरे गाः पालयन्ति ता दुग्धेनोपजीवनार्थं न स्वोपभोगार्थमिति विशेषः~। अत एव अनुदात्त ङित आत्मनेपदम् १~। ३~। १२~। इत्यत्र सुष्ठु खल्बभिहितं {\qt न सल्वप्यन्यत्प्रकृतमनुवर्तनादन्यद्भवति, नहि गोधा सर्पन्ती सर्पणादहिर्भावति} इति~। स्वातन्त्र्यमात्रप्राप्तिलाभेन चौरा दुष्टाः शठा वा न नागरिकपद्वीमर्दन्ति, न वा ते शुद्धाचाराः सम्भवन्तीति~।\\

वह्वायाससाध्ये लघुनि फले न तदानीन्तनानां प्रवृत्तिरासीदिति नवाह्निकस्य द्वितीयाह्निके {\qt सैषा महतो वंशस्तम्वाल्लट्वाऽनुकृष्यते इति वदता प्रदर्श्यते~। अयञ्च हयवरट् ( शिवसूत्रम् ) सूत्रस्थो युक्तिवादत्तदानींतनानां व्यवहारे याथार्थ्यं गमयति~। तद्यथा \textendash\ ॰एकश्चक्षुष्मान् दर्शने समर्थस्तत्समुदायश्च शतमपि समर्थम्, एकश्च तिलतैलदाने समर्थस्तत्समुदायश्चखार्यपि तैलदाने समर्था~। येषां पुनरवयवा अनर्थकाः समुदाया अपि तेषामनर्थकाः~। एकोऽन्धो दर्शनेऽ समर्थस्तत्समुदायश्च कतमप्यसमर्थम्, एका च सिकत~। तैलदानेऽसमर्था तत्समुदायश्च खारीशतमप्यसमर्थम्} इति~। पर्याप्तमेतन्निदर्शनं तदानींतनानाम्~। राजनीतिविचारे राजनीतिशून्यानां सहस्राधिकानां, धर्मसभायां संशयितचित्तानां लक्षादप्यधिकानां, लोकोपकारविचारसमयै बहुसंख्ययाऽप्युपस्थितानां स्वार्थप्रवणानां तदानींतने काले शून्यमेव मूल्यमासीत्, न ततोsधिकम्~। तत्र च राजनीतिज्ञा धर्मशास्त्रविदो निःस्वार्था एवाधिकृताः सम्मन्यन्ते

\newpage
% ( ६ ) 

\noindent
स्म \textendash\ इत्याविष्कृतं भगवता~॥ अत एव श्रमात्रैकनिष्ठानां मंदिरप्रवेशविवाहादिविषयाणां चर्चाप्रसङ्गे शास्त्रगन्धमपि दुर्गन्धिं मन्यमानानामन्धप्रायाणामभिप्रायेण शास्त्रविरुद्धाचरणमिदानींतनानां नातीव शोभते~। बहूनामन्धानां समवाये न चक्षुष्मताऽप्यन्धेन भवितव्यं न वा बधिराणां समवाये कर्णवताsपि पिको हातव्यः~॥\\

सुगृहीतनाम्नो भाष्यकारस्य समये स्वे स्वे कर्मण्यभिरता जना आसन्निति कृन्मेजन्तः १~। १~। ३९~। इत्यत्र {\qt न हि भिक्षुकाः सन्तीति स्थाल्यो नाधिश्रीयन्ते, न च समृगाः सन्तीति यवा नोप्यन्ते} इत्यादितो ज्ञायते~। न हि विघ्नाः सन्तीति कार्याणि नारभ्यन्ते इत्येव तेषां दृढा मतिः, न तु विघ्नाः सन्तीति स्वाभिमतं कार्यमपि सन्त्यज्यावस्थीयत इति~॥\\

व्यवहारकौशल्यमपि तदानीन्तनानां नमुने ८~। २~। ३~। इति सूत्रे अथवा द्विगता अपि हेतवो भवन्ति~। तद्यथा \textendash\ आम्राश्च सिक्ताः पितरश्च प्रीणिता भवन्ति~। तथा वाक्यान्यपि द्विगतानि दृश्यन्ते \textendash\ श्वेतो धावति \textendash\ अलम्बुसानां यातेति~। अथवा वृद्धकुमारीवाक्यवदिदं द्रष्टव्यम्~। वृद्धकुमारी इन्द्रेणोक्ता वरं वृणीष्वेति सा वरमवृणीत पौत्रा मे बहुक्षीरघॄतमोदनं कांस्यपात्र्यां भुज्जीरन्निति~। न च तावदस्याः पतिर्भावति कुतः पुत्राः कुंतो वा पौत्राः कुतो गावः कुतो धान्यम्~। तत्रानया एकेन वाक्येन पतिः पुत्रा गावो धान्यमिति सर्वं संगृहीतं भवति~। इत्युपन्यस्यता प्रकटितम्~। कस्मिंश्चित्समवाये किंजातीयो घावति, किंवर्णो धावति, को धावतीति बहुभिः पृष्टे चतुर उत्तरयति \textendash\ श्वेतो धावत्पलम्बुसानां यातेति एकेनैवोत्तरेण सर्वेषामपि समाधिर्विधीयते~। अनुभीयते चैतेन भाष्य \textendash\ प्रणयनकाले व्युत्पन्नाः सुचतुरा धीरा धार्मिकाश्च जना भारते निवसन्ति स्मेति~॥

\begin{center}
\textbf{\Large कृषिः \textendash\ }
\end{center}

तदानींतनानां कृषिविषये कीदृशं ज्ञानमासीत्, कथं वा तेषां कर्षणम्, के वा कर्षका इत्यादिविषये भाष्यप्रदत्तोदाहरणमात्रादेव यदुपलभ्यते तदेतत् \textendash\ वर्णो वर्णेन २~। १~। ६८~। इत्यत्र इह हि सर्वं मनुष्या अल्पेनाल्पेन महतो महतोऽर्थानाकाङ्क्षन्ति~। एकेन माषेण शतसहस्रम्~। एकेन कुद्दालपदेन खारीसहस्रम् इत्युक्तम्~। सर्वषामयं स्वाभाविको धर्मो यदल्पेन प्रयत्नेनाल्पेन मूल्येन वा महती कार्यसिद्धिरपेक्ष्यते \textendash\ इति~। एकमाषसुवर्णेन वस्त्रादीनां शतं सहस्रमपीच्छन्ति, अथवा एकेन माषेणोप्तेन सहस्रसङ्ख्याकान् माषानपीच्छन्ति.~। तथा कुद्दालपदेन \textendash\ खनित्राग्रस्थितपरिमितधान्येन \textendash\ तादृशधान्यवापेन, अथवा खननार्थं भूम्यां पातितः कुद्दालो यत्परि \textendash\ मितं क्षेत्रं व्याप्नोति तत्परिमितेन क्षेत्रेण खारीणां सहस्रमिच्छन्ति~। वस्तुतः कर्षकाणामयं व्यवहारः \textendash\ एकेन हलेतैकस्मिन्दिवसे यावत् \textendash\ क्षेत्रं क्रष्टुं शक्य तावदेव तदुच्यते \textendash\ एकहलं क्षेत्रमिति, तथैकेन कुद्दालेनैकस्मिन्दिवसे यावत्क्षेत्रं कृष्टं स्यात्तत्कुद्दालपदमिति~। एकेन कुद्दालपदेन \textendash\ एककुद्दालपरिमितक्षेत्रेण खारीसहस्रमपीच्छन्तीति तदर्थः~। एकेन कुद्वालपरिमितक्षेत्रेण {\qt खारीसहस्रमिच्छन्ति} एवमुक्तौ सम्भावनीयायामपि एतावता क्षेत्रेण इयद्धान्यं भवति, इयद्धान्यं चापेक्षते \textendash\ इत्येतादृशं यत्क्षेत्रविषयकं संशोधनं तत्तदानीमासीदिति निश्चयेन प्रतीयते~। तत्र खारीपरिमाणञ्च \textendash\ 

\begin{quote}
{\qt पलं प्रकुञ्चकं मुष्टिः कुडवस्तच्चतुष्टम्~।\\
चत्वारः कुडवाः प्रस्थश्चतुः प्रस्थमथाढकम्~॥

अष्टाढको भवेद्रोणो द्विद्रोणः सूर्प उच्यते~।\\
सार्द्धसूर्पो भवेत्खारी द्विद्रोणा गोण्युदाहृतम्~।\\
तामेव भारं जानीयाद्वाहो भारचतुष्ट्यम्~॥}
\end{quote}

इत्यभियुक्तोक्तवचनात् \textendash\ सेटकस्य पादः \textendash\ पलम्, चत्वारि पलानि \textendash\ कुडवः स एव सेटकः ( भाषायां शेर इत्युच्यते ), चत्वारश्च कुडवाः \textendash\ श्रस्थं ( पायली ), चत्वारः प्रस्थाः \textendash\ आढकं, ( चतुर्णामाढकानां भाषायां {\qt मण} इत्युच्यते ) अष्टाढकः \textendash\ द्रोणः ( दोन मण ), द्विद्रोणः \textendash\ सूर्पः, सार्धसूर्पः \textendash\ खारी ( सहा मण ), द्वाभ्यां द्रोणाभ्यामधिका खारो \textendash\ द्विद्वेणा \textendash\ गोणी, गोणी \textendash\ भारः, चत्वारो भाराः \textendash\ वाहः \textendash\ इति प्रायः परिमाणानि भवन्ति~। अनेन खारी \textendash\ सहा मण ( भाषायां ), खारीसहस्रं \textendash\ भाषायां मणसंज्ञकानां षट्सहस्राणामेकेन कुद्दालपरिमितक्षेत्रेणोत्पादनमिच्छन्ति~। कुद्दालपरिमितं क्षेत्रमपि यथा \textendash\ एकहलं क्षेत्रं भाषायां {\qt एकर} इत्याख्यात् किंचिदेव न्यूनं प्रायो व्यवह्रियते~। एकहलमूल्यं कार्षिकैरेवं व्यवह्रियते~। द्वौ वृषभौ एकश्च हलवाद् इति त्रयो भवन्ति, एकस्य हलवाहस्य या एकदिनस्य भृतिस्तत्त्रिगुणा भॄत्तिर्हलस्य~। तथैव एकेन हलेन यावत्क्रष्टुं शक्यं तस्य तृतीयो भाग एकेन हलवाहेनापि कुद्दालेन क्रष्टुं शक्यः~। एवञ्च भाषायां {\qt एकर} इत्याख्यस्य तृतीयो भागः कुद्दालपरिमितं क्षेत्रं स्यात्~। इदानीं एकर परिमिते क्षेत्रे शालीनां निष्पत्तिः खारी \textendash\ चत्तुष्टयपरिमिता भवतीति द्वादशद्रोणा उत्पद्यन्त इति राजकीयकर्मचराणामभिमतम्~। कुद्दालपरिमिते क्षेत्रे चतुर्णा द्रोणानां निष्पत्तिः स्वीकृतप्राया~। चत्वारो द्रोणाश्च भाषायामष्ट {\qt मण}संज्ञका भवन्ति~। तत्र तदानींतनैः खारीसहस्रस्योत्पत्तिरभिहिता~॥ मनुष्यख्खवभावनिरीक्षणे यदेव प्रत्यक्षं तत्समीपे एव कल्पनाव्यवहारः~। यथा \textendash\ दरिद्रः शतमिच्छति, शती च सहत्रमिच्छति, न तु दरिद्रः कदापिकोटिमीहते~। साहित्ये रसाभिव्यक्तिः सहृदयस्यैव भवतीति संकेतस्तथा कल्पनाया व्याप्तिरपि सहृदयैरेवाकलनीया~। सहृदयैर्विचार्य माणे सामान्यो मनुष्यस्तावद्राजानं दृष्ट्वा अहं राजा स्यां इत्येवाकलयति, प्रधानामात्यक्च {\qt अहं सावभौमः स्याम्} इति कल्पयेत् तेनैव च स सुखी भवति~। तथाsत्राप्यत्युक्तिकल्पनायां यत्र शतस्योत्पत्तिस्तत्र सहस्रं परिकल्प्येत, न तु यत्रैकमेव भवति तत्र सह

\newpage
% ( ७ ) 

\noindent
स्रस्य कल्पनाऽभिमता~। एवश्च खारीसहस्रमित्यनेन खारीशतं तु मन्तव्यमेव~। एकस्मिन् कुद्दालपरिमिते क्षेत्रे यत्रेदानीं चत्वारो द्रोणा भवन्ति तत्र खारीशतं ( त्रिशतसंख्याका द्रोणाः ) भवन्ति स्मेति अनुमीयते~। अथवा खारीशतविषये तेषां प्रयत्नस्तु अवश्यं वक्तव्यः~। ईदृशी धान्योत्पत्तिस्तदाssसीदित्यभिप्रायग्रहणे तदानींतना जनाः सुखिन आसन्नित्यत्र न कोsपि विवादः स्यात्~॥

\begin{center}
\textbf{\Large अदेवमातृका कृषिः \textendash\ }
\end{center}

तथा \textendash\ शालीनामुत्पादने यादृशस्तेषां व्यवहारस्तेनैतज्ज्ञायते \textendash\ न वर्षास्वेव शालीनुत्पादयन्ति, किंतु अवर्षायां हेमन्तादावपि शालीनुत्पादयन्ति स्मेति~। मीनाति मिनोतिदीङाल्यपि च६~। १~। ५०~। इति सूत्रे {\qt अन्यार्थमपि प्रकृतमन्यार्थ भवति~। तदथा \textendash\ शाल्यर्थं कुल्याः प्रणीयन्ते ताभ्यश्च पानीयं पीयत उपस्पृश्यते च शालयश्च भाव्यन्ते} इति व्यवहारदर्शनात् शाल्यर्थ कुल्यानां बहुप्रचार आसीदिति निश्चयेनावगन्तुं शक्यम्~। अत एव बहुगणवतुडति संख्या १~। १~। २४~। सूत्रादौ बहुत्र {\qt शाल्यर्थ कुल्याः प्रणीयन्ते} इत्युक्तिर्भाष्यकृतां सामञ्जस्येन सङ्गच्छते~। कुल्याशब्दश्च कृत्रिमनदीवाचकः, {\qt कुल्याल्पा कृत्रिमा सरित्} इत्यमरात् कुल्या \textendash\ कृत्रिमा वारिप्रवाहिकाऽल्पा नदी भवति~। तथाविधानां कुल्यानां शाल्यर्थं निर्माणेन समृद्धशालिमान् देशो भविष्यतीति प्राचीनव्यव \textendash\ हारदर्शनात् अनुमीयते~॥

\begin{center}
\textbf{\Large शालिप्रकाराः \textendash\ }
\end{center}

शालीनां प्रकारा अपि भिन्नास्तत्तद्देशविशेषणविशिष्टा भिन्नरुचयश्च यथेदानीं दृश्यन्ते तथा तदानीमप्यासन्~। यथेदानीं केचिच्च शालयोऽल्पेन कालेन भवन्ति केचिच्च महता तद्वत् तदानीमपि~। तच्च यथा \textendash\ वर्णो वर्णनेति सूत्रे एवः लोहितशालिमान् ग्रामः इत्युदाहृतम्~। केचिच्च शालयो लोहितवर्णा भवन्ति, अपरे न~। कस्मिश्चिंद्ग्रामे विशिष्टवर्णाः शालयः सम्भवन्ति~। क्वचिच्च ग्रामः सर्वबीजी भवति, कश्चिच्च न तथेत्येतदपि तस्मिन्नेव सूत्रे सर्वबीजी ग्रामः इत्युदाहरणादवसीयते~।

\begin{center}
\textbf{\Large कर्षकाणां ऋणं }
\end{center}

धनित्वच्च तदानीं धान्येनापि व्यवहरन्ति स्मेति तृतीया तत्कृतार्थेन गुणवचनेन २~। १~। २९~। इति सूत्रे {\qt धान्येन धनवान्} इत्युदाहरणेन प्रदर्शितम्~। न केवलं धनेनैव धनी भवति किन्तु धान्येन~। तस्मिन् समये कर्षकाश्च ऋणेन व्याप्ता इत्येवमुदाहरणं न दृष्टिगोचरं सम्पूर्णेऽपि भाष्ये~। इदानींतनीयेन व्यवहारेण ये ये नगराद्वहिर्ग्रामे घोषे वा वर्तन्ते ते सर्वेऽपि ऋणिनोsवश्यं भवेयुः~। ये ग्रामे वसन्ति ते नानाविधानि वस्तूनि निष्पादयन्ति नागरिकान् द्रव्यव्ययेन विक्रीणन्ति च \textendash\ इत्थमासीत्पुरातनो व्यवहारः~। इदानीं तत्सर्वं विपरीतमिव सज्ञातम्~। सामान्यजनानां ग्रामीणानां प्रत्यहं येषामुपयोगस्ते हि दीपशलाका, बनस्पतिः ( चहा ), वस्यं, और्णेयं, दीपतैलं, गुडः शर्करा वा एते पदार्था झटिति वुद्धिगोचरा भवन्ति~। तत्र गुडः शर्कराभावेsपि ग्रामे एव सम्भवतीति तद्विषये न काऽपि चिन्ता~। परन्तु दीपशलाकादीन्पदार्थान् क्रेतुं न धान्यस्योपयोगः, द्रव्यमेव तत्रापेक्ष्यते~। दीपशलाकानिर्माणेन बहूपकृतं खलु वैदेशिकैर्नागरिकाणाम्, ग्रामीणानां कृते तद्विरुद्धमिव सञ्जातम्~। यतो दीपशलाकाभावे करीषाणां वैपुल्यात् अग्नीन् गृहे संरक्ष्य दीपशलाकाकार्यमनायासेनैव सम्पत्स्यते~। इदानीं दीपशलाकापराक्रमश्चैवं वर्णनीयः \textendash\ एकेन पणेन दीपशलाकाशतं लभते, तेनैकसप्ताह \textendash\ पर्यन्तं कार्यं निर्वहति~। एकसप्ताहपर्यन्तमग्निसंरक्षणे बहूनां करीषाणां व्ययः कालहानिश्चेत्यालोच्य तादृशपदार्थनिर्माणेन विस्मितबुद्ध \textendash\ यस्तत्पदार्थव्यवहारे कटिबद्धा वैदिशिकानां गुणगौरवे दत्तचित्ताश्च विस्मृतस्ववृत्तयो दास्यभावेन सहात्मसमर्षणमपि चक्रुः~॥ ( चहा ) वनस्पतिसेवनमपि वैदेशिककृपालब्धमेव, तत्र तु द्रव्यव्यय आवश्यकः, यतः सा वनस्पतिर्न यत्रकुत्रापि क्षेत्रे सम्भवतीति~॥ ततश्च वस्त्रं, तत्तु इदानीं यन्त्राधीनं कथं द्रव्याभाव उपलभ्येत~॥ दीपतैलं तु ( राकाइल ) इतरतैलापेक्षया लधुमूल्यं दीपशलाकावत् ग्रामादिषु एकेन पणेन सप्ताहपर्यन्तं प्रकाशकमिति कृत्वैवान्यान् तैलानभिभावयति~। धान्येन धनिनां हस्ते समये समये द्रव्यसंग्रहः सम्भवति, परंतु एवमपि केचन दिवसाः सम्भवन्ति \textendash\ यदद्य पणोsपि नास्तीति~। त एव तेषां ऋणकारका दिवसाः~। तेषु संचीयमानं ऋणं धान्यानां सभये समागते लघुवुद्धित्वान्मोहवशाद्वा न प्रतिक्रियत इति ते ऋणिनः~॥\\

भाष्यकारसमये तु दीपशलाकादीनामापणक्रेतव्यानां नामापि नासीदिति ग्रामवासिनां ऋणग्रहणसमय एव नाभूदिति सुखिनस्ते तदा सम्प्रवृत्ताः~। अत एव ऋणोदाहरणसमयेऽपि ग्रामीणऋणस्य न क्वापि तैश्चर्चा कृता~। इदानीं स्वातन्त्रयावसरे समागते यदि केचन सुसंस्कृतभारतीयबुद्धिवैभवशालिनो नेतारोsस्मद्भाग्यवशात् संयुज्येरन् तदा कोsवकाशो ग्रामीणानां क्लेशनिर्हरणे~। पाश्चात्य \textendash\ वादविप्लुष्टमतयो नैतत्कार्यं कर्तु प्रभविष्यन्तीति सुनिश्चितमेव~। अत एव कर्षकऋणापाकरणचिकित्सानिर्णये {\qt तदृणमेव न देयम्} इत्येव निश्चयो गृहीतस्तैः~। एतच्च नावलोकितं \textendash\ कर्षकाणामुत्तमर्णा अपि प्रायः कर्षका एवेति तेषां द्रव्यनाशे कर्षकाणामेव द्रव्यनाश इति~। अनायासेन लब्धं राज्यमपि यथा पालयितुं न शक्यते तथाsनायासेन प्राप्तमनृणित्वमपि तथैव~। अथ च यावदृणकारणसामप्री तदवस्थैव तावदृणापाकरणं मूलं विहाय शाखासिञ्चनमिव दरीदृश्यते~॥

\begin{center}
\textbf{\Large कर्षकाः के ?}
\end{center}

अवसरेऽस्मिन् प्राक् के कृषकपदवाच्याः शिष्टसम्मता इत्यवलोकवमपि सुसम्बद्धं स्यात्~। तथाहि \textendash\ हेतुमति च ३~। १~। २६~। इत्यत्र णिच्प्रत्ययोत्पत्त्यनुत्पत्तिविचारवेलायां कृष्यादिषु चानुत्पत्तिर्वक्तव्या~। एकान्ते तूष्णीमासीन उच्यते पञ्चभिर्हलैः कृषतीति~। तत्र भवितव्यं पश्चभिर्हलैः कर्षयतीति~। कृष्यादिषु चानुत्पत्तिर्नाना क्रियाणां कृष्यर्थत्वात्~। कृष्यादिषु चानुत्पत्तिः

\newpage
% ( ८ ) 

\noindent
सिद्धा~। कुतः~। नानाक्रियाणां कृष्यर्थत्वात्~। नाना क्रियाः कृषेरर्थाः~। नावश्यं कृषतिर्विलेखने एव वर्तते~। किं तर्हि प्रतिविधानेऽपि वर्तते~। यदसौ भक्तबीजबलीवर्दैः प्रतिविधानं करोति स कृष्यर्थः~। आतश्च प्रतिविधानेsपि वर्तते, यदहरेवासौ न प्रतिविधत्ते तदहरेव तत्कर्म न प्रवर्तते~। इत्युक्तम्~॥\\

कृष्यादिधातोर्णिजनुत्पत्तिर्वक्तव्या, यतस्तूष्णीमासीनेऽपि \textendash\ अकिञ्चित्कुर्वाणेऽपि देव्रदत्ते {\qt पञ्चभिर्हलैः कर्षयति देवदत्तः} इति लोका वदन्ति, वस्तुतस्त्र भृत्यादिद्वारा कर्षणात् {\qt पञ्चभिर्हलैः कर्पयति देवदत्तः} इति वक्तुं युक्तम्~। एवञ्च यत्र णिच् प्राप्तस्तत्राकरणात् कृष्यादिषु चानुत्पत्तिर्वक्तव्येति~। ततश्च वार्तिककार आह \textendash\ पञ्चभिर्हलैः कृषति देवदत्त इत्यत्र णिचः प्राप्तिरेव नेति न ततोऽनुत्पत्तिर्वक्तव्या~। तत्र हेतुमाह \textendash\ नानाक्रियाणां कृष्यर्थत्वादिति~। कृषधातोर्नं विलेखनमेवार्थः, किन्तु विलेखनप्रेरणं विलेखनोपकारकरणं विलेखनविषये यावत्प्रतिविधानकरणच्च~। तथा च यत्र स्वयं कृषति तत्र देवदत्तः कृषति, यत्र देवदत्तस्य भृत्याः कृषन्ति तत्रापि देवदत्त एव कृषति, यत्र च देवदत्तस्यांशभागिनः कृषन्ति तत्रापि देवदत्तस्य प्रतिविधानर्कर्तृत्वाद्देवदत्तः कृषतीति भवत्येव~। क्वचिद्देवदत्तो भक्तेन प्रतिविधानं करोति, क्वचिच्च बीजदानेन प्रतिविधानं करोति, क्वचित् हलबलीवर्दादिभिः प्रतिविधत्ते इति तदर्थं~। एवञ्च तदानीन्तने समये कर्षकपदवाच्याः \textendash\ यः स्वयं कृषति, यो भृत्यद्वारा, यो भागदानेन, यश्च बीजबलीवर्दीदिसम्पादनेन, ते सर्वेऽपि कर्षका एवेति निर्णयः~। कृषिश्चैवं पदार्थोsस्ति य एकेन न निर्वर्तयितुं शक्यः, बहूनां साहाय्यसमवधाने एव तस्य सम्पन्नत्वात्~। कर्षकार्थविचारवेलायां यद्येतदनवलोक्य यः कृषति स कर्षक इत्येवोपादाने न तेनैकेन समीचीना कृषिः स्यात्, न च धान्यसमृद्विश्च स्यादिति~। अत एव भाष्ये {\qt भक्तबीजबलीवर्दैः प्रतिविधानं करोति} इत्यत्र नात्मनेपदप्रयोगः~। अन्यथा प्रतिविधानफलस्य कर्तृगामित्वात् तत्र {\qt स्वरितजितः कर्त्रभिप्राये क्रियाफले} इत्यात्मनेपदमेव युक्तं स्यात्~। इह तु भागादिना प्रतिविधानफलं यथा कर्तृगामी तथाऽन्यत्रापि गच्छतीति नात्मनेपदप्रयोगः~॥

\begin{center}
\textbf{\Large कृषिकर्माणि \textendash\ }
\end{center}

कर्षकाणां क्रियाप्रवृत्याद्यपि क्वचिदुपवर्णितम्, तद्यथा \textendash\ कर्मवत्कर्मणातुल्यक्रियः ३~। १~। ८७~। इत्यत्र कथं ज्ञायते भिद्यते कुसूलेनेति~। न चान्यः कर्ता दृश्यते क्रिया चोपलभ्यते~। किञ्च भो विग्रहवतैव क्रियायाः कर्त्र भवितव्यम्, न पुनर्वातातपकाला अपि कर्तारः स्युः~। भवेत्सिद्धं यदि वातातपकालानामन्यतमः कर्ता स्यात्~। यस्तु खलु निवाते निरभिवर्षे अचिरकालकृतः कुसूलः स्वयमेव भिद्यते तस्य नान्यः कर्ता भवति अन्यदतः कुसूलात्~। यद्यपि तावदत्रैतच्छक्यते वक्तुं यत्रान्यः कर्ता नास्ति~। इह तु कथं \textendash\ लूयते केदारः स्वयमेवेति~। यत्रासौ देवदत्तो दात्रहस्तः समन्ततो विपरिपतन्दृश्यते~। अत्रापि याsसौ सुकरता नाम तस्या नान्यः कर्ता भवति अन्यदतः केदारात्~। इति~। अयमर्थः \textendash\ भिद्यते कुसूलः खयमेवेत्यत्र कुसूलस्य कथं कर्तृत्वमित्याक्षिप्य न न्वान्यः कर्ता दृश्यते क्रिया चोपलभ्यतेऽतः कोऽपि कर्ताऽवश्यमाश्रयणीयः स न्चान्याभावात्कुसूल एव~। विभ्रहवतैव \textendash\ शरीरिपौव केनचन कर्त्री भवितव्यमिति न, किन्तु वातातपादयोऽपि कर्तीरः~। एवमपि यत्र वातवर्षादिकं नास्ति नूल्च कुसूलः स्वयमेव भिद्यते तत्र कुसूल एव कर्ता~। कुसूलविषये भवतु तथा, तत्रान्यः कश्चित् कर्ता नास्तीति~। गृत्र चान्यः कर्ता दृश्यते तत्रापि लूयते केदारः स्वयमेवेति कथं? तत्र दात्रहस्तो देवदत्त इतस्ततो धावन् दृश्यते तत्र देवदत्त एव कर्तेति वक्तुं युक्तम्~। तत्रायं समाधिः \textendash\ अत्र या सुकरता तस्याः कर्ता नान्य इति केदार एव तस्याः कर्तेति निर्णयः~। एवं च लूयते केदारः, भिद्यते कुसूल इत्यादौ केदारादीनां कर्तृत्वमुपपदाते इति~॥ कुसूलः \textendash\ अर्धभित्तिसदृशो मृत्तिकाभिः पाषाणैर्वा क्षेत्रादीनां संरक्षणार्थं निर्मीयते~। यदा दात्रखनित्रहस्तो देवदत्तः केदारं लुनाति तदा सौकर्यातिशयात् देवदत्तस्य परिक्लेशो न, केदारश्च लूनो भवति तदा कर्मणः वर्तृत्वविवक्षायां लयते केदार इति भवति~। यदा क्षेत्रेऽधिकजलसंचयो जायते तच्च शालीनां नावश्यकं तदा नालिकयैकया क्षेत्राज्जलं तनिष्कास्यते~। यदाऽधिकजलसंचयात् दात्रसंसर्गमात्रेण नालिका जायते तत्र देवदत्तस्य न कर्तत्वं किन्तु केदारस्येवेति तदर्थः~। एवञ्च यथेदानीं जलसंचयापाकरणे कर्षका दक्षा भवन्ति तथा प्रागपि स व्यवहारस्तथैवासीदिति ज्ञायते~॥

\begin{center}
\textbf{\Large गवादीनामुपचारः \textendash\ }
\end{center}

छदिरुपधिवलेर्ञ् ५~। १~। १३~। इत्यत्र {\qt इहार्थ तावद्वालेयास्तण्डुलाः}~। आर्धभ्यो वत्सः~। ऋषभार्थो घासः~। गुणान्तरयुक्ता हि तण्डुल्वा वालेयाः~। गुणान्तरयुक्तवत्स आर्षभ्यः~। इति वदता वहुगुणविशिष्टास्तण्डुला आसन्, केचिद्व्यर्थमुपयुज्यन्ते, केचिच्च राजभोजनाः, केचिद्गुणान्तरविशिष्टा इति सूचितम्~। आर्षभ्यो वत्स इत्यनेन \textendash\ अयं वत्सो वृषभत्वे योग्य इति परीक्ष्य तादृश एव वृषभकरणे नियुज्यते~। एवं पश्चादीनां चिकित्सा परीक्षा चासीत्, न केवलं गोर्जातो बृषभ इति स्वीकारः~। तथा ऋमभार्थो घास इत्यनेन च गवामुपयुक्तो घासोsन्यः, वत्सोपयुक्तोsन्यः, वृषभार्थोऽप्यन्य इति घासादीनां पशूनाञ्च गुणावगुणदर्शनेन निश्चयस्य प्रथाऽऽसीद्, घासमात्रलाभेन ते न सन्तुष्टा इति व्यज्यते~॥\\

पूर्वचत्स्सनः १~। ३~। द२~। इति सूत्रे {\qt गोः सक्थनि कर्णे वा कृतं लिङ्गं गोरेव विशेषकं भवति न गोमण्डलस्य} इत्युदाहरणेन गवां प्रत्यभिज्ञानादिकं नियतमासीदिति गम्यते~॥ तया \textendash\ अञ्चोऽनपादाने ८~। २~। ४९~। इति सूत्रे {\qt एवं तर्हिअश्वे रङ्कः~।} अङ्कश्च प्रकाशनम्~। अङ्किता गाव इत्युच्यते, अन्याभ्यो गोभ्यः प्रकाशन्ते इत्युदाहरता गव्रां परिसङ्ख्यानादिकंसुचारुरूपेणासीदिति प्रकाश्यते~॥\\

षष्टिकाः पष्टिरात्रेण पच्यन्ते ५~। १~। ९०~। इति सूत्रे ये षष्टिरात्रिसमूहेन पच्यन्ते ते ब्रीहयः षष्टिका इत्युच्यन्ते ( इदानीं भाषायां साठीः शब्द्रेन तेषां व्यवहारः ) ~। तत्र वार्तिककार आह \textendash\ षष्टिके संज्ञाग्रहणम् *षष्टिके संज्ञाग्रहणं कर्तव्यम्~। मुद्गा अंपि हि

\newpage
% ( ९ ) 

\noindent
षष्टिरात्रेण पच्यन्तै तत्र माभूदिति~॥ अत्र वार्तिककारस्यायमाशयः \textendash\ मुद्गा यद्यपि षष्टिरात्रेण पच्यन्ते तेषामपि {\qt षष्टिकाः} इत्येदमभिधानं स्यात्, लोके तच्च नेष्यत्ष्यत इति तत्र संज्ञाग्रहणं कर्तव्यम्~। संज्ञाग्रहणे कृते षष्टिरात्रेण ये पच्यन्ते व्रीहयस्त्रैव षष्टिकाः इति शब्द्व्यवहारः~। तेन मुह्याः षष्टिरात्रेण पच्यमाना अपि नैतां संज्ञां लभन्ते~। एवञ्च षष्टिका इति ब्रीह्रीणामेव संज्ञेति तदभिप्रायः~॥

\begin{center}
\textbf{\Large क्षेत्रपरीक्षा \textendash\ }
\end{center}

यथा थान्येष्वनेके प्रकारा गुणभेदादिना प्रवर्तन्ते तथा क्षेत्राहीनामपि विचिकित्साऽऽसीत्~। तथा चोच्यते \textendash\ कृभ्वस्तियोगे रुस्पद्यकररि च्विः ५~। ४~। ५०~। इति सूत्रे {\qt सम्पदयन्ते यवाः, सम्पद्यन्ते शाल्य इति~। सम्पद्यन्तेsस्मिन् क्षेत्रे शालय इति~। } अस्मिन् क्षेत्रे शाल्यो भविष्यन्ति, अस्मिंश्च यवा भविष्यन्तीत्यनेनास्मिन् क्षेत्रे यवापेक्षया समीचीनाः शालयः सम्भविष्यन्तीत्यर्थो गम्यते~। एतच्च तदैव युज्येत यदा क्षेत्राणं धान्यादीनां विशिष्तोऽभ्यासः सम्पत्स्येत~। स तथाऽतीदित्यनुभावयत्येतद्वाक्यम्~।

\begin{center}
\textbf{\Large क्षेत्रमितिः \textendash\ }
\end{center}

ईदृशं तत्र कृषिकर्मं प्रवर्तते तत्र क्षेत्राणां मर्यादा स्वामिसम्बन्धोऽप्यावश्यकः, तदभावे न्यूनतैवास्यार्थस्य स्यात्~। तदर्थमत्रा \textendash\ वधेयम् \textendash\ मपर्यन्तस्य ७~। २१९१~। इति सूत्रे अयमन्तशब्दोऽस्त्यैव सह तेन वर्तते~। तद्यथा \textendash\ मर्यादान्तं देवदत्तस्य क्षेत्रम्, सह \textendash\ मर्यादयेति गम्यते~॥ अस्ति प्राक् तस्मात्वर्तते~। तद्यथा \textendash\ नद्यन्तं देवदत्तस्य क्षेत्रमिति, प्राक नद्या इति गम्यते~। तद्यः सह तेन वर्तते तस्येदं ग्रहणं यथा विज्ञायेत~। नैतदस्ति प्रयोजनम्~। सर्वत्रैवान्तशब्दः सह तेन वर्तते~। अथ कथं नद्यन्तं देवदत्तस्य क्षेत्रामिति ? नद्याः क्षेत्रत्वे सम्भवो नास्तीति कृत्वा प्राक् नद्या इति गम्यते इत्युक्तम्~। अत्र क्षेत्राणां मर्यादाऽप्युक्ता~। {\qt मपर्यन्तस्य} सूत्रे पर्यन्तस्येति किमर्थमुच्यते, भान्तस्येत्येव वक्तव्यमित्याक्षिप्य अन्तशब्दस्य व्यभिचारदर्शनात् परिभ्रहणं कृतमित्युक्तम्~। अन्तशब्दस्य व्यभिचारश्च मर्यादान्तं देवदत्तस्य क्षेत्रमित्युच्यमाने मर्यादाऽपि क्षेत्रावयव इत्येव प्रतीयते नतु मर्यादां विहाय क्षेत्रमित्वर्थः प्रतीयते, नद्यन्तं क्षेत्रमित्युच्यमाने च नदीं विद्दाय क्षेत्रमित्वेव ज्ञायते~। ततश्च क्वचिदन्तशब्दस्तेन सहेत्यर्थस्य बोधकः क्वचित्तं विहायेत्यर्थस्येति दृष्टापचारोsन्तशब्द इति परिग्रहणमित्युक्तेऽपि पूर्वपक्षिणा तत उच्यते \textendash\ अन्तशब्दः सर्वत्रादृष्टापचार एव, नद्यन्तं देवदत्तस्य क्षेत्रमित्यादौ नद्याः क्षेत्रत्वासम्भवान्न नद्यन्तपदेन नद्याः क्षेत्रे समावेश इति तद्भावः~। अत्र देवदत्तस्य क्षेत्रमिति क्षेत्रादीनां स्वामि \textendash\ सम्बन्धो बोधितः~। इदं च क्षेत्रं देवदत्तस्य, न यज्ञदत्तस्येत्यर्थं प्रमाणमपि राज्याधिकारिभिः स्थापितमेव भवेत्~। अन्यथा देवदत्तस्येत्यर्थं यदि न राजा प्रमाणं तदितरः कः शक्नुयादेवं वक्तुम्, अतो देवदत्तस्येत्यर्थे राजाऽपि प्रमाणमासीत्तदेत्यभ्युपगन्तव्यमेव~। तथा मर्यादान्तमित्यनेनेदानीं यथा क्षेत्रसीम्नि पाणाणाद्युपन्यस्यते तथा तदानीमपि मर्यादाकरणविधिः प्रतीयते~। एवञ्च मर्यादापरिपालितं स्वामिसम्बद्धं पर्याप्तन्नोत्पादकं तदार्नीं क्षेत्रमासीदित्यायातमनेन~॥

\begin{center}
\textbf{\Large धान्यमूल्यम् \textendash\ }
\end{center}

तथा धान्यादीनां स्थितिप्रलयाववेक्षणीयावेवास्ताम्, येन नाशो विलोक्यते तेन तदर्थं कश्चित्प्रयल्नः क्रियत एव~। धान्यस्य नाशः स्थितिश्च \textendash\ ल्वादिभ्यः ८~। २~। ४४~। सूत्रे पूञो विनाशे इति वक्तव्यम्~। पूना यवाः~। विनाश इति किमर्थम्~। {\qt पूतं धान्यम्~।} इत्यनेन प्रदर्शिता~। धान्यं यदि नष्टं चेत् पूनमित्युच्यते, अनष्टश्चेत्पूतमित्युच्यते \textendash\ इति शब्दद्वैविध्यमेव निर्दिष्टम्~। येन संशयलेशोऽपि न भवेत्~। यथेदानीं वस्तुतो विचार्यमाणे, व्यवहारे धान्यस्य मूल्यं नास्ति~। मूल्यं च तदेवोच्यते यस्मिन् कस्मिंश्चिदपि क्षणे आपणे नीतं वस्तु विनिमयेन व्यवह्रियेत न तथा धान्यमिति वस्तुतो धान्यस्य मूल्यमिदानी न संरक्षितम्~। तदानीं तु अध्यर्थशूर्पम्, अर्थपञ्चमशर्पम् इत्यादिभाष्योदाहरणदर्शनात् धान्येनैव ग्रामीणानां प्रत्यहं सर्वोऽपि व्यवहारः सेत्स्यतीति न तदर्थ तेषां परमुख \textendash\ प्रेक्षणमिति व्यवस्थितिः~। अत एव धान्यस्यापि व्यवहारे मूल्यमासीदिति वक्तुं युज्यते~॥\\

एवमेव चतुर्थीतदर्थार्थबलिहितसुखरक्षितैः २~। १~। ३६~। इति सूत्रे{\qt महाराजार्थो बलिः स महाराजार्थो भवति, अश्वघासः, हस्तिविधा, ब्राह्मणार्थ पयः, ब्राह्मणार्थः सूपः, ब्राह्मणार्था यवागूः} इत्याद्युदाहरणैरिदं समर्थ्यतै \textendash\ यो राजग्राह्यो भागः सोऽपि धान्यरूपेणैव, न तु द्रव्येणेति महाराजार्थो बलिरित्येतत्सूचयति~। धान्यरूपेण बलेनिर्धारणान्न धनिकमुखापेक्षा कर्षकाणामिति व्यव \textendash\ हारसौलभ्यं भवति~। अश्वधासो हस्तिविधेत्यनेन समुचितोपायवन्तः कर्षका यद्यस्य रुचितं तत्तेभ्यो दत्वा निर्वृतास्तिष्ठन्ति, न हस्ति \textendash\ भोज्यं गोभ्यो ददते, नवाsश्वभोज्यं हस्तिभ्यो दत्वा कृतार्थमात्मानं मन्यन्ते~। कृषिसंरक्षणोपाये पशुसंरक्षणं मूलभूतमिति गवादीनां भक्ष्यविशेषतिवोनोचितप्रबन्धाः पशवः संरक्ष्यन्ते स्मेति कृषेरुकारः समर्थितो भवति~। अत एव {\qt गोभिर्वपावान्} इति तृतीया \textendash\ तत्कृतार्थेन गुणवचनेन २~। ९~। २९~। इत्यत्रोक्तमुदाहरणं सङ्गच्छते~। गोसम्बन्धिदध्याद्युपयोगाद्वपावत्वं तदैव स्याद्यदा गोदुग्धस्य प्रषुर उद्गमः स्यात्, तदथ्च गवादीनां दुग्धवाहाद्युपयोगिनां सर्वतः संरक्षणं विधीयते स्मेत्यवगम्यते~॥\\

कर्तरि कर्मव्यतिहारे १~। ३~। ९४~। सूत्रे कर्मव्यतिहार इत्युच्यमाने इहापि प्रसज्येत \textendash\ देवदत्तस्य धान्यं व्यतिलुनन्तीति~। इह च न स्यात \textendash\ व्यतिलुनते व्यतिपुनते इति इत्युदाहरता मूल्येन कियाव्यतिहारेण वा कर्षकाणां कार्यं समर्थयता न सामुदायिकी कृषिः, न वा एकमात्रकर्तृका कृषिरिति समर्धितम्~। देवदत्तस्य धान्यमित्यादिभाष्यस्यायमर्थ: \textendash\ सूत्रे धान्यस्य कर्मणो व्यतिहारसत्रैव स्यात्, यत्र क्रियाव्यतिहारस्तत्र न स्यात्~। तथाच देवदत्तस्य धार्न्यं लवनात्प्राक् क्षेत्र एव व्यवस्थितं\\

२ प्र० पा०प्रस्ता०

\newpage
% ( ६० ) 

\noindent
कश्चित्क्रीणाति तत्र कर्मणो धान्यस्य व्यतिहारात् देवदत्तस्य धान्यं व्यतिलुनन्तीत्यत्न्नात्मनेपदं स्यात्~। यत्र च देवदत्तेन कर्तुमुचिता लवनक्रिया वेतनेन पणनादिना वाsन्येन क्रियते तत्र न स्यादिति~। देवदत्तस्य धान्यं व्यतिलुनत इत्यत्रान्यस्तस्य क्रियां करोति न त्वत्रकर्मणो व्यतिहार इत्यत्रात्मनेपदं भवति, एतादृशक्रियाव्यतिहारेण द्रव्याभावेsपि कृषेः सर्व कार्य सेत्स्यति सर्बेऽपि सुखिनो भवन्ति~। अत्रार्थे यदि मूल्यस्यैव निर्भरः स्यात् तदा समये समये सर्व कार्यं न निष्पद्येत, समयमूल्यं तु कृषिव्यवहारे परमोन्नतमिति परमुखप्रेक्षिणां कर्षकाणामसमये क्रियमाणं कार्यमक्रियमाणमेवेति महती हानिः स्यात्~॥\\

देवदत्तस्य धान्यं व्यतिलुनन्तीति व्यवहारस्तु कृषेः परमोन्नतिं प्रदर्शयति~। धान्यं लत्वा पूत्वाऽऽपणे च नीत्वा विक्रेयमित्येतत्तु परिमितधान्यवतां व्यवहारविषयम्~। अन्ततो गत्वा यदि विक्रेयमेव यद्धान्यं तद्यदि लवनादिव्यवहारतः प्रागेव विक्रीतं भवेत् तदा तावता समयेनान्यत्कार्यं सम्पत्स्येतेत्यादिव्यवहारः सम्पन्नानामेव शोभते, न तु दरिद्राणां~। परिमितधान्यवन्तस्तु तदपि कुर्वत इति गतिवुद्धिप्रत्यवसानार्थशब्दकर्माकर्मकाणामणि कर्ता स णौ १~। ४~। ५२ इति सूत्रे {\qt नयति देवदत्तः, नाययति देवदत्तेन~।} वहन्ति वलीवर्दा यवान्, वाहयति वलीवर्दान् यवान्~। भक्षयन्ति यवान् बलीवर्दाः, भक्षयति बलीवर्दान् यवान्~। मासं शेते देवदत्तः इत्युदाहरणैः प्रदर्शितम्~। परिमितधान्यवांस्तु स्वयमेव नयनानयनं करोति, ततोऽप्यधिको वलीवर्दैर्नयति~। यत्र तु अधिकं धान्यं पुष्टिकामाश्च बलीवर्दानां तत्र \textendash\ भक्षयति वलीवर्दान् यवानित्यपि सम्भाव्यते~॥

\begin{center}
\textbf{\Large क्षेत्रादीनां विशेषः \textendash\ }
\end{center}

खलादिवर्णनमपि {\qt खल्यवं, खलेवुसं, लूनयवं, पूनयवं, पूयमानयवम्} इत्यादिना तिष्ठद्गुप्रभृतीनि च २~। १~। १६~। इति सूत्रे समुचितं दृश्यते~। अत्र हि खलेयवमित्यादिना कृषिकर्मणः परिज्ञानं यथावद्भवति~। यवाः क्षेत्रतः खल आयाता इति खलेयवमित्यनेन बोध्यते~। खलेबुससमित्यनेन क्षेत्रतः खल आनीय पूता यवा गृहं गताः खले बुसमात्रमवशिष्टमिति ज्ञायते~। लूनयवमित्यनेन क्षेत्रे एव यवा लूनाः सन्तिष्ठन्ति खले नायाता इति बोध्यते~॥ कुद्दालपदेन खारीसहस्त्रमित्यनेनापि विशेषोऽयं बोध्यते \textendash\ यत्र हलेन कर्षणं न सम्भवतिपर्वतादौ, तत्र कुद्दालेनापि धान्यमुत्पादयन्तीति~। इदानीमपि पार्वतीपरिचरणपवित्रितभूभागस्य ऋष्याश्रमसन्निधानस्य भारतवैभवस्य हिमाचलस्योपत्यकायामधि \textendash\ वसन्तो जनाः कुद्दालेमैव कर्षन्तीति श्रूयते~। एवच्च तदा सुसम्पन्ना कृषिः सुखयति स्म लोकानिति विज्ञाने न कोऽपि प्रत्यवायः~॥

\begin{center}
\textbf{\Large धान्यसञ्चयः \textendash\ }
\end{center}

धान्यादीनां सञ्चयस्तेषां वितुषीकरणादिकमपि तात्कालिकमेवं प्रतीयते \textendash\ क्षिप्रवचने लृट् ३~। २~। १३३ सूत्रे {\qt अवश्यं खल्वपि कोष्ठगतेष्वपि शालिषु अवहननादीनि प्रतीक्ष्याणि} इत्यादिना~॥ पूर्वं {\qt शाल्यर्थं कुल्याः प्रणीयन्ते} इत्यादिनाऽदेवमातृका कृषिः प्रतिपन्ना, विशालेऽस्मिन् भारते यत्र कुल्यादिकरणासम्भवस्तत्र देवमातृकत्वमपि प्रतीयते तत्रैव सूत्रे \textendash\ {\qt देवश्चेद्वृष्टो निष्पन्नाः शालयः}~। कश्चिदाह देवश्चद्बृष्टः सम्पत्स्यन्ते शालय इति~। स उच्यते \textendash\ मैवं वोचः, सम्पन्नाः शालय इत्येवं ब्रूहि इत्याद्युदाहरणैः~। अन्यच्च वृष्ट्यनन्तरं शालीनामुत्पत्तिर्निश्चिता, न तत्र रोगादिना चौरादिव्यतिक्रमाद्वा नाशः स्यादिति रोगादिप्रतीकारोपाया राजादिसंरक्षणश्च तत्र सन्नद्धमिति द्योतयति~॥ \textendash\ 

\begin{center}
\textbf{\Large ग्रामनगरादिविभागः \textendash\ }
\end{center}

ननु सम्पन्ना कृषिरिति स्वीकारेऽपि तदा देशनगरग्रामघोषादीनां निर्माणमासीत्, उत यत्र तत्र वा यथा कथञ्चिन्निवासः~। न देशविभागाः, न देशग्रामयोस्तारतम्यम्, न मार्गादीनां निर्माणमित्याद्याशङ्कादीनामवसरः स्यात्, यदि भाष्यकृता देशग्रामविषयकाणि उदाहरणानि नोपन्यस्तानि भवेयुः~। तद्विषयकाणि च बहून्युदाहूरणानि यत्र तत्र प्रदर्शितानि, तत्र कतिपयान्यत्र प्रस्तूयन्ते~। पृषोदरादीनि यथोपदिष्टम्६~। ३~। १०९~। इत्यत्र् {\qt उञ्चार्य हि वर्णानाह \textendash\ उपदिष्टा इमे वर्णा इति~। कैः पुनरुपदिष्टाः ? शिष्टैः~। के पुनः शिष्टाः ? निवासतश्चाचारतश्च~। स चाचार आर्यावर्तं एव~। कः पुनरार्यावर्तः ? प्रागदर्शात् प्रत्यकालकवनात् दक्षिणेन हिमवन्तमुत्तरेण पारियात्रम्~। एतस्मिन्नार्यावर्तं निवासे ये ब्राह्मणाः कुम्भीधान्या अलोलुपा अगृह्यमाणकारणाः किश्चिदन्तरेण कस्याश्चिद्विद्यायाः पारङ्गतास्तत्रभवन्तः शिष्टाः~। इत्येवं शिष्टलक्षणं हितं, तदवसरे आर्यावर्तलक्षणं दक्षिणेन हिमवन्तमुत्तरेण पारियात्रं \textendash\ हिमवद्विन्ध्ययोरमध्ये कालकवनादर्शयोर्मध्ये च यो देशः स आर्यावर्त इत्युच्यते~। कालकवनं \textendash\ प्रयागः, आदर्शः पर्वतः, पूर्वस्यां दिशि आदर्शः पर्वतः पश्चिमस्यां च कालकवनं \textendash\ प्रयागो दक्षिणस्यां हिमवान् उत्तरस्याच्च विन्ध्य इति चतुर्णामादर्शादीनां मध्ये यो देशः स आर्यावर्त इत्युच्यते, तस्मिन्नार्यावर्ते निवसन्तः कुम्भ्यामेव येषां धान्यं ते अलुब्धाः स्वाध्यायोsध्येतव्य इति बुध्यैव विद्यापारंगतास्ते ब्राह्मणाः शिष्टाः~। तादृशशिष्टैरुपदिष्टो मार्ग एव कल्याणकरो न त्वन्यैरुपदिष्टः~। अनेकनगरग्रामक्षेत्रारामनदीपर्वतविशिष्टस्य भूभागस्य देश इति संज्ञामाचक्षते पुरातत्त्वविदः~॥ तस्यैतस्यार्यावर्तस्य जगतीतलभूषणभूतस्य मानचित्रम्प्रागुपनिबद्धमवलोकनीयम्~॥ अन्यत्रापि आर्यावर्तलक्षणं पुनः पुनरुपरिष्टं दृश्यते~। शूद्राणामनिरवसितानाम् २~। ४~। १०~। इति सूत्रे केऽनिरवसिता केच निरवसिताः शूद्ना इति विचारप्रस्तावे {\qt कः पुनरार्यावर्तः ?प्रागादर्शात्प्रत्यक्कालकवनात् दक्षिणेन हिमवन्तमुक्तरेण पारियात्रम्} इति पूर्ववदेव लक्षणमार्यावर्तस्योक्तम्~। एतादृशदेशविशिष्टा रचनास्तदानीमासन्निति ज्ञायते~॥ विशिष्टलिङ्गो नदीदेशो ग्रामाः

\newpage
% ( ११ ) 

\noindent
२~। ४~। ७~। इति सूत्रे {\qt अग्रामाः} इति यदुच्यते तत्र ग्रामप्रतिषेधे नगरप्रतिषेधो वक्तव्यः इति वार्तिककृतोक्तम्~। नावश्यको ग्रामप्रति \textendash\ षेधः, आवश्यकश्च नगरप्रतिषेध इति तदाशयः~। अतश्च वार्तिककारसमये ग्रामनगरयोर्महान् भेदःसंप्रत्येति~। तदुदाहरणेषु {\qt शौर्यं च केतवता च शौर्यकेतवते}~। जाम्बवं च शालूकिनी च जाम्बवशालूकिन्यौः इत्यत्र शौर्यं \textendash\ नगरं, केतवता \textendash\ ग्रामः, जाम्बवं \textendash\ नगरं, शालूकिनी \textendash\ ग्रामः~। तत्र एवं तर्हि आर्यनिवासादनिरवसितानाम्~। कः पुनरार्यनिवासः ? ग्रामो घोषो नगरं संवाह इति~। एवमपि ये महान्तः संस्त्यायास्तेष्वभ्यन्तराश्चण्डाला मृतपाश्च वसन्ति तत्र चण्डालमृतपा इति न सिध्यति~। इत्युक्त्या ग्रामो घोषो नगरः संवाहा निवासविशेषाः प्रतीयन्ते~।\\

तत्र ग्रामश्च प्रसिद्ध एव, घोषः \textendash\ प्राधान्येन गोमहिष्यादीनां शालास्तदङ्गत्वेन चेतरशाला यत्र सन्ति, संवाहः \textendash\ वणिजामापणप्रधानः ( बाजारपेठ भाषा )~। ग्रामलक्षणं च आद्यन्तवदेकस्मिन् १~। १~। २१~। सूत्रे लोके शालासमुदायो ग्राम इत्युच्यते~। भवति चैतदेकस्मिन् एकशालो ग्राम इति~। विषम उपन्यासः~। ग्रामशब्दोऽयं बह्वर्थः~। अस्त्येव शालासमुदाये वर्तते, तद्यथाग्रामो दग्ध इति~। अस्ति वाटपरिक्षेपे वर्तते \textendash\ ग्रामं प्रविष्ट इति~। अस्ति च मनुष्येषु वर्तते \textendash\ ग्रामो गतो ग्रामं आगत इति~। अस्ति सारण्यके ससीमके सस्थण्डिलके वर्तते, तद्यथा \textendash\ ग्रामो लब्ध इति~। तद्यः सारण्यके ससीमके सस्थण्डिलके वर्तते तमभिसमीक्ष्यै \textendash\ तत्प्रयुज्यते \textendash\ एकशालो ग्राम इति~। ? एतादृशमुक्तम्~। ग्रामं प्रविष्ट इत्युच्यमाने शालां प्रविष्ट इति न भवति किन्तु ससीमकः सारण्यकः सस्थण्डिलकः सर्वोsपि ग्राम एवेति सीमाप्रविष्टे ग्रामं प्रविष्ट इति व्यवहारः~। ग्रामश्च लघुरेकशालोsपि भवति महांश्चानेकशालोऽपि~। सारण्यके ससीमके सस्थण्डिलक इत्याद्युपवर्णनात् नैसर्गिकसीमासन्निवेशे विशेष निर्भरः प्रतीयते~॥\\

ग्रामनगरयोर्भेदोऽपि लोकव्यवहारतः सुस्पष्टःप्राचां ग्रामनगराणाम् ७~। २~। १४~। इत्यत्र {\qt नगरग्रहणं किमर्थ, न प्राचां ग्रामाणामित्येव सिद्धम्~। न सिध्यति~। अन्यो हि ग्रामः, अन्यन्नगरम्~। कथं ज्ञायते एवं हि कश्चित्कंचित्पृच्छति \textendash\ कुतो भवानागच्छति ग्रामात्~। स आह \textendash\ न ग्रामात् नगरादिति~। ननु च भोः य एव ग्रामस्तन्नगरम्~। कथं ज्ञायते ? लोकतः~। ये हि ग्रामे विधयो नेष्यन्ते साधीयस्ते नगरे क्रियन्ते~। तद्यथा \textendash\ अभक्ष्यो ग्राम्यकुक्कुटः, अभक्ष्यो ग्राम्यसूकर इत्युक्ते सुतरां नागरोऽपि न भक्ष्यते~। तथा ग्रामे नाध्येयमिति साधीयो नगरेऽपि नाधीयते~। तस्माद्य एव ग्रामस्तन्नगरम्~। कथं यदुक्तम् \textendash\ एवं हि कश्चित्पृच्छति कुतो भवानागच्छति ग्रामात्, स आह \textendash\ न ग्रामान्नगरादिति~। संस्त्यायविशेषमसौ आचष्टे~। संस्त्यायविशेषा ह्येते ग्रामो घोषो नगरं संवाह इति} इत्युक्तवता भाष्यकृता प्रकाशितः~। ग्रामशब्दो हि जननिकायनिवासमात्रे प्रसिद्धः, निवासविशेषेsपि च, समुदायेsपि प्रसिद्धो भूतग्रामः \textendash\ इन्द्रियग्राम इत्यादौ~। अनेकार्थप्रत्यायकं ग्रामशब्दमाश्रित्यैतद्याख्यानं भाष्ये~। अभक्ष्यो ग्राम्यकुक्कुटोऽभक्ष्यो ग्राम्यसूकरो ग्रामे नाध्येयमित्यत्र जननिकायनिवासमात्रे प्रसिद्धस्य ग्रहणम्~। अत एव ग्राम्यकुक्कुटो न भक्षणीय इत्यनेन नागरोऽपि न भक्ष्यते इत्याद्युपपद्यते~। ग्रामे नाध्येयमित्यत्रापि जननिकायनिवासमात्रे प्रसिद्धस्यैव ग्रहणम्~। कुतो भवान् ग्रामादागत इत्यादौ संस्त्यायविशेषमसौ पृच्छतीत्यालोच्य न ग्रामान्नगरादित्युत्तरं संगच्छते~। संस्त्यायः \textendash\ जनसंनिवेशः~। भूतग्रामादौ केवलसमुदायवाचकस्यैव ग्रामशब्दस्य प्रसिद्धिः~। संस्त्यायविशेषविवक्षायां ग्रामो नगराद्भिन्न इत्यस्मादुपन्यासात्प्रतीयते~॥\\

ग्रामाश्च नगराणां देशानामपि वाsवयवाः सम्बन्धिनो वा भवन्ति~। तस्येदम् ४~। ३~। १२०~। इति सूत्रे स्वे ग्रामजनपद \textendash\ मनुष्येभ्य इति वार्तिकोदाहरणे स्रौघ्नो ग्रामः, माथुरो ग्राम इति स्रुघ्ननगरस्य यो ग्रामः स स्रौघ्नो ग्राम इच्युते, मथुरानगर्या यो ग्रामः स माथुरो ग्राम इति~। ये नगरसमीपे ग्रामा भवन्ति नगर एव वा येषामुपजीवनं ते तत्तन्नगरशब्दसम्बन्धित्वेन व्यवह्नियन्ते~। तथा देशवाचकशब्दसम्बन्धित्वेनापि ग्रामाणां व्यवहारः {\qt आङ्गकः, वाङ्गकः} इत्यादिशब्दैर्भवति~। अङ्गदेशस्य यो ग्रामः स आङ्गकः~। वङ्गदेशसम्बन्धी यो ग्रामः स वाङ्गकशब्देन व्यवह्नियते~॥ निवासाभिजनशब्दयोरपि भिन्नो व्यवहारः सोऽस्य निवासः ४~। ३~। ८९~। इति सूत्रे {\qt निवासाभिजनयोः को विशेषः निवासो नाम यत्र सम्प्रत्युष्यते~। अभिजनो नाम यत्र पूर्वैरुषितम्} इत्यादिना प्रतिपादितः~॥ देशनगरादीनां व्याख्यानान्यपि कृतानि दृश्यन्ते~। तस्य व्याख्यान इति च व्याख्यातव्य नास्नः ४~। ३~। ६६् इत्यत्र {\qt पाटलिपुत्रस्य व्याख्यानी सुकोसला} यादृशं पाटलिपुत्रं व्याख्यानग्रन्थे प्रसिद्धमुपवर्ण्यते तादृशी सुकोसला नगरीत्युपवर्णनेन नगराणां परिचयपुस्तिकाप्रकारास्तेषां मानचित्राणि वा निर्मीयन्ते स्मेति सम्भवति~। शब्दश्वैतावान् रूढो यः शब्दग्रन्थेषु व्याख्यानमिति भवति, नगरेषु नगरव्याख्यानमिति न भवतीत्युपवर्णयता {\qt अथ क्रियमाणेsपि व्याख्यातव्यनाम्नो ग्रहणे कस्मादेवात्र न भवति~। अवयवशो ह्याख्यानं \textendash\ व्याख्यानं, पाटलिपुत्रमपि अवयवशःआचष्टे \textendash\ ईदृशा अस्य प्राकाराः, ईदृशा अस्य प्रासादा इति~। सत्यमेवमेतत्~। क्वचित्तु काचित्प्रसृततरा गतिर्भवति~। शब्दग्रन्थेषु चैषा प्रसृततरा गतिर्भवति \textendash\ निरुक्तं व्याख्यायते व्याकरणं व्याख्यायत इत्युच्यते, न कश्चिदाह \textendash\ पाटलिपुत्रं व्याख्यायत इति} इत्युपन्यस्तम्~। नगरस्यावयवशो ह्याख्यानेऽपि नगर \textendash\ व्याख्यानमिति तत्र व्याख्यानशब्दो न प्रभवति~। अनेन नगराणां वर्णनमवयवशः सविस्तरमितरनगरसाम्यञ्च प्रदर्श्य निःसन्दिग्धं कृतमासीदित्युपलभ्यते~॥\\

देशान्तरगमनागमनप्रकारा मनुष्याणां नित्यमेव भवन्ति, ये व्यवसायार्थ देशदेशान्तरं गच्छन्ति क्वचिच्च चिरमधिवसन्ति 

\newpage
% ( १२ ) 

\noindent
ते तत्र प्रायभवाः, ये चतत्र नित्यम्भवन्ति ते तत्रभवा इति~। यथा ये मुम्बापुर्यां व्यवसायार्थमागत्य तत्र चिरं निवसन्ति ते तत्र प्रायभवा मरुदेशस्था गुर्जरा उत्तरदेशीयाश्च~॥ ये च नित्यं निवसन्ति मुम्बापुर्यां ते तत्रभवा इति व्यवहारस्तथा तदानीमपि व्यवहार आसीदिति प्रतीतिः प्रायभवः ४~। ३~। ३९~। इति सूत्रे यो हि राष्ट्रे प्रयेण भवति तत्र भवोsसौ भवति, तत्र तत्र भव इत्येव सिद्धम्~। न सिध्यति~। अनित्यभवः \textendash\ प्रायभवः, नित्यभवः \textendash\ तत्रभवः~। स्रौघ्नो देवदत्तः, स्रौघ्नाः प्रासादाः, स्रौघ्नाः प्राकारा इति इत्यनेन वर्णनेन ज्ञायते~। अत्र राष्ट्रशब्दो देशसमानार्थो व्यवहृतः~॥ \\

तस्मिन् समये प्रसिद्धानि ग्रामनगरनामानि अव्ययात्त्यप् ४~। २~। १०४~। सूत्रादौ भाष्यकृता बहुश उक्तानि \textendash\ आरात् \textendash\ करव \textendash\ तीर \textendash\ कास्तीर \textendash\ चणाररूप्य \textendash\ दासरूप्य \textendash\ शिवपुर \textendash\ शाकल \textendash\ निलीनक \textendash\ सौसुकं \textendash\ निषाहकर्षू \textendash\ दाक्षिकर्षु \textendash\ त्रिगर्त \textendash\ गर्ग \textendash\ वत्स \textendash\ मालव \textendash\ सांकाश्य काम्पिल्य \textendash\ पातालप्रस्थ \textendash\ काश्चीपुर \textendash\ नान्दीपुर \textendash\ वातवह \textendash\ कौक्कुटीवह \textendash\ ब्राह्मणक \textendash\ आष्टक \textendash\ ऋषिक \textendash\ जिह्वव \textendash\ मौञ्ज इत्यादीनि नानाविधानि विचित्राणि च मनसो मोदकराणि संदृश्यन्ते~। पस्पशाह्निके {\qt सर्वै देशान्तरे} इति व्याख्यानावसरे सर्वे खल्वप्येते शब्दा देशान्तरेषु प्रयुज्यन्ते, न चैवोपलभ्यन्ते~। उपलब्धौ यत्नः क्रियताम्~। महान् शब्दस्य प्रयोगविषयः \textendash\ सप्तद्वीपा वसुमती त्रयो लोकाः} इत्याद्युक्तवतो भाष्यकारस्य अनेकदेशग्रामाद्यभिधानप्रस्तावो न चित्रकरः, किन्तु स्वाभाविक एव~। अइउण् सूत्रव्याख्यानावसरे {\qt तानेव शाटकानाच्छादयामः, ये मथुरायाम्~। तानेव शालीन् भुञ्ज्महे, ये मगधेषु~। तदेवेदं भवतः कार्षापणं, यन्मथुरायां गृहीतम्~। अन्यस्मिंश्चान्यस्मिन् रूपसामान्यात्तदेवेदमिति भवति} इत्यादिवचनैद्वौ यात्रिणौ स्वानुभवं परस्परमुपयादयतः~। एतेन च यात्रायामभिसुचिः खानुभव \textendash\ सम्पादनस्वभावश्च प्रतिपादितो भवति~। यात्राभिरुचिवैशिष्ट्यादेव च चार्थे द्वन्द्वः २~। २~। २९~। सूत्रे \textendash\

\begin{quote}
{\qt उपास्नातं स्थूलसिक्तं तूष्णींगङ्गं महाह्रदम्~।\\
द्रोणञ्चेदशको गन्तुं मा त्वा ताप्तां कृताकृते~॥}
\end{quote}

इत्युक्तिः सङ्गच्छते~॥ 

\begin{center}
\textbf{\Large मार्गप्रबन्धः \textendash\ }
\end{center}

कालाध्वनोरत्यन्तसंयोगे २~। ३~। ५~। इति सूत्रे {\qt शय्यते} क्रोशम् इत्युदाहरणात् रथादिवाहनैर्गच्छन् पुरुषः कोशं शेते इत्यभिव्यक्तिबलात् अनेके पन्थानः शिल्पिभिः कृता व्यवतिष्ठन्त इति प्रतीतिः~। तथा तत्रैव {\qt क्रोशं कुटिला नदी कोशं रमणीया वनराजिः} इत्युदाहरणात् वनराज्या सुशीतलाः सुगमाश्च पन्थान आसन्ननायासेन पान्था गमागमं कुर्वत इति~। अत एव {\qt कुतो भवानागच्छति ? पाटलिपुत्रादागच्छामि~।} गवीधुमतो निःसृत्य सांकाश्यं चत्वारि योजनानि~। दक्षिणतो ग्रामस्य उत्तरतो ग्रामस्य उपरिष्टाद्ग्रामस्य इत्यादिनिर्देशास्तत्रतत्रोपलभ्यमानाः संगच्छन्ते~॥\\

नगराणां निर्देशः क्वचिन्नदीसहचरितोऽपि दृश्यते~। यस्य चायामः २~। १~। १५~। इति सूत्रे अनुगङ्गं वाराणसी, अनुगङ्गं हास्तिनपुरम्, अनुशोणं पाटलिपुत्रम्~। यस्य चायाम इत्युच्यते~। गङ्गा चाप्यायता, वाराणस्यप्यायता, तत्र कुत एतत् गङ्गया सह समासो भविष्यति, न पुनर्वाराणस्येति~। एवं तर्हि लक्षणेन इति वर्तते~। गङ्गा चैव लक्षणं न वाराणंसी~। अथवा {\qt यस्य चायामः} इत्युच्यते~। गङ्गा चाप्यायता वाराणस्यप्यायता~। तत्र प्रकर्षगतिर्विज्ञास्यते \textendash\ साधीयो यस्यायाम इति~। साधीयश्च गङ्गायाः, न वाराणस्याः~। इत्ययामवन्ति महान्ति नगराणि प्रदर्शयन्ते~। अस्य भाष्यस्य चायमाशयः \textendash\ यत्सम्बन्धि आयामबोधकोsनुः स सेन सुबन्तेनोपस्थितत्वात् षष्ष्यन्तेन समस्यत इत्यर्थं गङ्गाऽपि दीर्घा वाराणस्यपि, यथाऽनुगङ्गं वाराणसीति भवति तथा \textendash\ अनुवाराणसि गङ्गेति कथं नेत्याशङ्क्य पूर्वसूत्रादत्र लक्षणेनेति पदमनुवर्तते, तेन लक्षणवाचकेन समस्यत इत्यर्थात् अनुगङ्गमित्येव भवति, यतो हि गङ्गा लक्षणम्~। अथवा वाराणसी आयता गङ्गाsप्यायता, तत्र वाराणस्यपेक्षया गङ्गाऽयतेति अनुगङ्गमित्येष भवतीति~। एतानि नगराणि सम्पन्नानि सुदीर्घाणि महान्ति गङ्गादिभिः प्लावितानि च शोभन्ते स्म~। इदानीमपि तानि नगराणि प्रधानान्येव वर्तन्ते~। यद्यपि पाटलिपुत्रहास्तिनपुरयोः नामविपरिवर्तनेsपि महत्वं तयोस्तदवस्थमेव~। अनुमीयते च पुरातने काले इमान्यतिविस्तृतान्येव नगराण्यासन्निति~। यत इमानि महान्ति नगराण्यासन्नतस्तत्र नगरप्रबन्धकैः प्रयत्नैरपि भवितव्यमेव, तद्विषये किञ्चिद्वक्तव्यमिति नोदाहरणतः सुस्पष्टं प्रतीतमस्माभिः~॥\\

एतावत्तु प्रतीयते \textendash\ तत्समयेऽपि नूत्ननगरग्रामादिनिर्माणमपि सञ्जातमिति~। अत एव न वेतिविभाषा १~। १~। ४३ सूत्रे {\qt आमो भवता गन्तव्यो नवः, प्रत्यग्र इति गम्यते} इत्युक्तं सङ्गच्छते~। नगरयोः परस्परं विप्रकर्षोsपि प्रतिपादित एतद्बोधयति \textendash\ तत्रत्या नागरिका विशिष्टयुणाधाने बद्धपरिकरा नगराणामैश्वर्यं बहुतरमवीवृधन्निति~। अत एव ध्रुवमपायेsपादानम् १~। ४~। २४~। इत्यत्र इह च सांकाश्यकेभ्यः पाटलिपुत्रका अभिरूपतरा इति यस्तैः साम्यं गतवान् भवति स एतत्प्रयुङ्क्ते इत्युक्तमुपपद्यते~। सांकाश्यक \textendash\ पाटलिपुत्रयोः समगुणत्वात् साम्यमवगत्य पाटलिपुत्रे कश्चित् विशेषं पश्यति स एतत्प्रयुङ्क्ते इति तदर्थः~। तथा {\qt ग्रामादागच्छति शकटेन} ग्रामादागच्छन् कंसपात्यां पाणिनौदनं भुङ्क्ते इत्युक्तिग्रामीणानां व्यवहारं बोधयति~॥

\begin{center}
\textbf{\Large वैद्यकप्रबन्धः \textendash\ }
\end{center}

नगरेषु ग्रामेषु वा वैद्यकप्रबन्धः कीडृश इत्यवलोकने मुक्तकण्ठेनैवं वक्तव्यम् \textendash\ इदानीं परिदृश्यमानो वैद्यकव्यवहारस्तद्व्यवहार \textendash\

\newpage
% ( १३ ) 

\noindent
विमर्शने न दृष्टिगोचरो भवति~। भवन्ति च {\qt नड्वलोदकं पादरोगः, पादरोगनिमित्तमितिगम्यते~। दधित्रपुसं प्रत्यक्षोज्वरः~। } इत्येवविधा व्यवहारा दृष्टिगोचराः~। रोगोत्पत्त्यनन्तरं तदुपशमोपायाश्चेदानीन्तनानाम्, रोगोत्पत्तिरेव प्रतिबध्यते प्राचीनैः~। एतादृशवैलक्षण्य इदमेव वक्तु पार्यते \textendash\

\begin{quote}
{\qt अहरहर्नयमानो गामश्वं पुरुषं पशुम्~।\\
वैवस्वतो न तृप्यति सुराया इव दुर्मदी~॥}
\end{quote}

इति चार्थे द्वन्द्वः~। २~। २~। २९~। इति सूत्रोक्तवचनानुसरणमेव~। रोगे रोगे चिकित्साया नवीनाविष्कारमाविष्कुर्वद्भिः \textendash\ मस्तकशूले इदमौषधं, उदरव्याधाविदं, कर्णोपशमनार्थमिदमित्यादिव्याध्युत्पत्त्यनन्तरमेव चिकित्साविद्भिः पाश्चात्यानुसरणशालिभिर्वैद्यैः {\qt यथा दुर्मदी ( अल्पेन पानेन यस्य मदो न भवति बहुतरं पानमपेक्ष्यते सः ) सुरया न तृप्यति यमराजोsपि अहरहर्नयमानो न तृप्यति तथा वैद्यराजोऽपि रोगैर्न तृप्यति \textendash\ नवीनरोगाविष्करणे दत्तचित्तो भवति} प्रयत्यते, न तेन तादृशो लोकोपकारो न तथा व्याधिनिवृत्तिः~॥

\begin{center}
\textbf{\Large रोगप्रतिबन्धार्थं दृष्टादृष्टोपायाः \textendash\ }
\end{center}

ननु यमराजेनैते पास्पर्धमाना नवीनाविष्करणैस्तत्साहाय्यमेव कुर्वते~। बहुवित्तव्ययसाध्यश्चार्यं मार्गो नवीनानाम्~। वस्तुतस्तु प्राचीनप्रणाल्यां चिकित्सा दृष्टाऽदृष्टविषयिका द्विविधाऽप्यासीदिति आमः २~। ४~। ८१~। इति सूत्रे {\qt चक्षुष्कामं याजयांचकार} इत्याद्युदाहरणैर्दृश्यते~। कश्चिद्देवदत्तोऽन्धत्वं प्राप्तो दृष्टोपायैर्निस्तरगमनभिपश्यन् अदृष्टोपायार्थ यतत इत्यादि प्रकथनमपीदानीं दोषायतनं स्यात्~। वस्तुतस्तु व्याधयो द्विविधा दृष्टकारणका अदृष्टकारणकाश्च~। तत्र दृष्टकारणा दृष्टैरेवोपायैः शाम्यन्तीति न विवादः~। अदृष्टकारणाश्च कुष्ठादिसदृशा न दृष्टोपायैः कदापि शाम्यन्तीति अदृष्टोपायकरणं न दोषाय~। अत एव दृष्टकारणकान् रोगान्निवर्तयितुं रोगे रोगे औषधमित्युपायो न शिष्टसम्मतः किन्तु रोगकारणान्निवर्तयितुं तादृश आचार एवानवद्यः पन्था इति तदभिप्रायः~। अत एव तादृशापत्तिपरिहारार्थमाचारविशेषं श्रुत्यादिभिः प्रतिपादितं तदानीन्तनानां सुपरिचितमेव, उदाहरणमुखेन जनसांमुख्यमुपनयति भगवान् भाष्यकारः~। तदथा \textendash\ चतुर्थ्यर्थे बहुलं छन्दसि २~। ३~। ६२~। इति सूत्रे {\qt या खर्वेण पिबति तस्यै खर्वो जायते, यां मलवद्वाससं सम्भवन्ति यस्ततो जायते सोsभिशस्तः, यामरप्ये तस्यै स्तेनः, यां पराचीं तस्यै ह्वीतमुख्यपगल्भः, या स्नाति तस्या अप्सुमारुकः, याऽभ्यङ्के तस्यै दुक्षर्मा, या प्रलिखते तस्यै खलतिरपस्मारी, याऽङ्क्ते तस्यै काणः, या दतो धावते तस्यै श्यावदन्, या नखानि निकृन्तते तस्र्यै कुनखी, या कृणत्ति तस्यै क्लीबः, या रज्जुं सृजति तस्या उद्वन्धुकः, या पर्णेन पिबति तस्या उन्मादुको जायते} इति~। अनेन मातृदोषविशेषेण बालकस्य रोगोत्पत्तिं विभावयन्नीरोगप्रसवार्थं व्यतिरेकेणाचारविशेषमुपशिक्षयति~॥ रजस्वलाऽऽचारप्रस्तावेऽयमुपन्यासस्तैत्तिरीयश्रुतौ प्रसिद्धः~। अनेनोपन्यासेनेदानीन्तना जना बहुशिक्षितव्या भवेयुः, आचाराश्चैते धार्मिका इति कृत्वा निर्धर्मकैस्पेक्ष्यन्ते, आचारविद्वेषिणश्चाल्पबुद्धयो निर्मर्यादाः पैशाचादीन्धर्मानुपस्तुवन्ति, साम्यवादमभिकामयमानाश्च नैसर्गिके रजःप्रवर्तने कीदृशोऽयं रजस्वालनामाचारो देहाद्युपदण्डनश्चेत्यभिमन्यमाना आचारविच्छेदेन स्वीयं नवनीतादपि सुकोमलं हृदयं ख्यापयन्ति~। एवमनेकप्रकारैरुपहन्यमानोऽपि भारतीयाचारः पूर्ववत् यावन्नाश्रियेत, परमुखप्रेक्षिभिः शिशुबुद्धिभिर्यावदुपेक्ष्येत, स्वार्थवञ्चनाप्रवणैर्यावज्जनानां बुद्धिभेदः परिकल्प्येत तावद्विनिपात एव सांमुखीनः, का कथाsभ्युन्नतेरिति खलु न विस्मर्तव्यन्तृत्रभवद्भिर्जननेतृभिः~॥\\

इदानीं दृश्यमानानि रुग्णसेवासदनानि तदाऽऽसन्नवेति विचार्यमाणे सुस्पष्टन्तादृ्शार्थप्रतिपादकानि वाक्यानि नोपलभ्यन्त इत्येव वक्तव्यं स्यात्~। व्याधितानां वर्णनञ्च कवचिदुपलभ्यत एव~। तद्यथा \textendash\ भूवादयो धातवः १~। ३~। १~। इति सूञ्रे {\qt एवं हि कश्चि \textendash\ त्पृच्छति \textendash\ किमवस्थो देवदत्तस्य व्याधिरिति, स आह \textendash\ वर्द्धते, अपर आह \textendash\ अपक्षीयते, अपर आह \textendash\ स्थितः } इति~। अनेकक्रियाभिरुत्तराणि ह्येवं बोधयन्ति \textendash\ कस्यचन व्याधिः स्थितः, कस्यचन वर्धते, कस्य चनापक्षीयत इति बहुषु व्याधितेषु मिलितेषु चिकित्सकेन पृष्टोऽन्यश्चिकित्सको वक्तीति रुग्णालयवर्णनेन भाव्यमिति~॥\\

तथा \textendash\ अनुदात्तङित आत्मनेपदम् १~। ३~। १२~। सूत्रे {\qt एतिजीवन्तमानन्दः, नास्य किश्चिदुजति} इत्युदाहरणे खस्थोऽयं देवदत्तो नास्य किश्चिदुजति, जीवन्तं पुरुषमानन्दः प्राप्नोतीति तदर्थ यतनीयं नोपेक्षा करणीयेति चिकित्सकानां व्यवहारं बोधयतः~॥ अस्मदो द्वयोश्च १~। २~। ५९ इत्यत्र {\qt इदं मे अक्षि सुष्ठु पश्यति, अयं मे कर्णः सुष्ठु शृणोति, अनेनाक्ष्णा सुष्ठु पश्यामि, अनेन कर्णेन सुष्ठु शृणोमि} इत्युदाहरणानि नेत्रकर्णादिचिकित्सकसविधे तद्व्याधितानामुत्तराण्यपि सम्भवन्ति~। एतेन तदानीन्तने काले चिकित्साप्रकाराः शुश्रूषाविशेषाश्च समवर्तन्तेत्यनुमाने न हेत्वाभासाः प्रभवेयुः~॥\\

चिकित्साप्रकारा भिन्ना भिन्ना आसन्नित्यप्यवगन्तुं शक्यते नानोर्ज्ञः १~। ३~। ५८~। इति सूत्रे औषधस्यानुजिज्ञासते इत्युदा \textendash\ हरणावलोकनेन~। औषधस्यानुजिज्ञासत इत्यस्य हि औषधेन प्रवर्तितुमिच्छतीत्यर्थः~। एवश्चान्यचिकित्साकरणानन्तरमिदानीमोषधीभि \textendash\ श्चिकित्सितुमिच्छतीत्यनेन विविधाश्चिकित्साप्रकारा ज्ञायन्त इति सिद्धम्~। अस्मिन् विषये प्राचीनानामभिमतोऽर्थः \textendash\ सदाचरणेन रोगो \textendash\ त्पत्तिरेव प्रतिबन्धनीया, एवम्प्रयतमानस्यापि यदि दैववशात्कश्चिद्व्याधिरुत्पत्स्येत तत्कारणान्येव चिकित्सितव्यानि न तु परकीयां चिकित्सापद्धतिमवलम्ब्य स्वात्मनाशवत् द्रव्यनाशोऽपि विधेय इत्येव~॥

\newpage
% ( १४ ) 

\begin{center}
\textbf{\Large नगरादीनां निर्माणम् \textendash\ }
\end{center}

इदानीं नवराष्ट्रनिर्माणप्रसङ्गो राजकीयपुरुषाणां सांमुख्यमादधाति~। एतादृशे समये प्रस्तुतो विषयोऽपि जनकल्याणमतिभिर्नोपेक्षणीय एव~। यादृशेनोपायेन सर्वेषां जनानामत्यन्तं हितं स्यात्तादृश एवोपाय आश्रयणीय इत्येतन्न विवादास्पदम्~। अत एवेतिहासविद्धिः परमहितैषिभिः परिवर्तिनि संसारेऽस्मिन् येनोपायेन कुशलिनो जनाः सम्पत्स्यन्ते स उपायः समादरणीयश्चेत्, एतद्देशप्राचीनपरम्परादर्शनादेव कुशलम्, न परकीयदेशाचचारमुखप्रेक्षणेन~। नवराष्ट्रनिर्माणप्रसङ्गोऽपि भाष्यकृता कस्यच दः ५~। ३~। ७२~। इत्यत्र नवग्रामकं, नवराष्ट्रकं, नवनगरकं इत्युदाहरता प्रतिपादितः~। ग्रामो नतो विधीयते स्म, नवनगरनिर्माणदक्षा अपि तदानीन्तनाः पुरुषाः, नवराष्ट्रनिर्माणेऽपि वद्धपरिकराः सफलाश्चेति सर्वमनेन प्रतीयते~॥\\

नगरादीनां निर्माणे यत्र विशेषतोऽवधेयं तत् स्थानेऽन्तरतमः १~। ९~। ५०~। सूत्रे भगवता महता कौशल्येन प्रतिपादितम्~। तदित्थम् \textendash\ समाजेषु समाशेषु समवायेपु चास्यतामित्युक्ते नैव कृशाः क्रृ्ैः सहासते, न पाण्डवः पाण्डुभिः~। येषामेव किञ्चिदर्थकृतमान्तर्यं तैरेव सहासते~। यथा गावो दिवसं चरितवत्यो यो यत्याः प्रसबो भवति तेन सह शेरते~। तथा यान्येतानि गोयुक्तानि संघुष्टकानि भवन्ति तान्यन्योन्यमपश्यन्ति शब्दं कुर्वन्ति~॥ एवं तावच्चेतनावत्सु~॥ अचेतनेष्वपि यथा \textendash\ लोष्टः क्षिप्तो बाहुवेगं गत्वा नैव तिर्यग्गच्छति नोर्ध्वमारोहति पृथिवीविकारः पृथिवीमेतच्छत्यान्तर्यतः~॥ तथा \textendash\ या एता आन्तरिक्ष्यः सूक्ष्मा आपस्तासां विकारो धूमः~। स धूम आकाशे निवाते नैव तिर्यग्गच्छति नार्वागवरोहति~। आब्विकारोsप एव गच्छत्यान्तर्यतः~॥ तथा \textendash\ ज्योतिषो विकारोऽर्चिराकाशदेशे निवाते सुप्रज्वलितं नैव तिर्यग्गच्छति नार्वागवरोहति~। ज्योतिषो विकारो ज्योतिरेव गच्छत्यान्तर्यतः{\qt इति~। }नैव कृशाः इत्यस्य व्याख्यानवेलायां गुणप्रमाणकृतसादृश्यपरिहारेण विद्याद्यर्थकृतमान्तर्य लोक आश्रीयत इत्यर्थः इति व्याख्यातं कैयटेन~। प्रदीपव्याख्यानावसरे {\qt विद्यादीति} इति प्रतीकमुपादाय {\qt आदिनोत्सवाशनाद्यर्थित्वब्राह्मणत्वादिजातिधनादिसङ्ग्रहः} इत्युक्तं नागेशभद्टेन~। अनेन प्रघट्टकेन चेतनाचेतनसाधारणधर्मप्रतिपादकेन येषां किञ्चिदर्थकृतमान्तर्य तेषामेवैकत्र निवासः सम्भवति~। अर्थकृतमान्तर्यश्च समानोत्सवत्वेन समानाशित्वेन समानेच्छावत्वेन समानजातित्वादिना वा भवति न विरुद्धधर्मवत्वेनेति तदनुगुणा \textendash\ नामेवैकन्नगरं राष्ट्रं ग्रामो वा भवति~। तथा च विरुद्धसंस्कारवतां विद्वेषस्वभावानां क्रूराणां जडाभिमानिनामनिच्छयाऽवकाशे प्रदत्तैऽपि अनुनयसम्पादनं विफलमेवेत्यवधेयम्~।\\

यथेदानीं विशिष्टवर्गीयाः पुरुषा उपपत्त्यनुपपत्तिमवलोक्य यत्र तत्र वसन्ति यतस्ततौ वाऽन्यत्र गच्छन्ति तद्वत्तदानीमपि विशिष्टवर्ग्या इतस्ततो गच्छन्तीति गतिश्च १~। ४~। ५९~। इत्यत्र विगताः सेचका अस्माद्ग्रामात् \textendash\ विसेचको आमः, प्रगता नायका अस्माद्ग्रामात् \textendash\ प्रनायको ग्रामः इत्यादिनोपवर्ष्यते~। तत्र सेचनकार्यभावात्, अन्यत्र सेचनमूल्यधिकं मन्यमानानां वाऽन्यत्रगमनं सम्भवति तेषाम्~। कार्यवशाद्गामे नेतारः समापन्नाः कार्यावसानेऽन्यत्र गच्छन्ति, समुचितनेतृत्वकरणदक्षा अभ्युञ्नत्यपेक्षयाऽन्यत्र गच्छन्ति, अथवा सर्वेऽपि यत्र नेतारस्तत्रापीदं वक्तुं योग्यम्~॥

\begin{center}
\textbf{\Large नगरादीनां संज्ञाः \textendash\ }
\end{center}

ग्रामनगराणां नामान्यपि तत्रत्यविशिष्टवस्त्वपेक्षया समाजापेक्षयैव वा निर्वर्त्यन्ते स्मेति मतोश्च बह्वजङ्गात् ४~। २~। ७२~। इस्यत्र {\qt मालावतामयं निवासो मालावतम्, श्ववान्, उदुम्बरावती, मशकावती} इत्युदाहरणैः प्रतीयते~। यत्र मालावन्त एव प्राधान्येननिवसन्ति तन्नगरं मालावतम्, यत्र शोभनाः श्वान उत्पद्यन्तै, श्वानश्च यत्र तत्र भवन्ति, ग्रामेऽस्मिन् विशिष्टा व्याघ्र इव शूरा मृगयानिपुणा रक्षकाश्च श्वानो दृश्यन्ते इति श्ववान् ग्राम इत्युच्यते~। यस्याः परिसरे क्षा उदुम्बरा बहवः सन्ति, यत्र वा देहल्यो बहवः सन्ति सोदुम्बरावतीत्न्वर्थ नाम ( उदुम्बरस्तु देहल्यां वृक्षमेदे च पण्डवे इति मेदिनी ) ~। एवश्च गुणविशेषात् अभिजनविशेषाद्वा ग्रामादीनामभिधानानि प्रवर्तन्ते स्म~। ग्रामश्च प्रायः सर्वजातिविशिष्ट एव परिपूर्णो भवतीति व्यवहारस्य \textendash\ एच इक् हस्वादेशे १~। १~। ४८~। इत्यत्र भूयस एव ग्रहणानि भविष्यन्ति~। तद्यथा \textendash\ ब्राह्मणग्राम आनीयतामित्युच्यते~। तत्र चावरतः पञ्चकारुकी भवति इत्युदाहरणेनोपपत्तिः~। यत्र ब्राह्मणग्रामोsयमित्युच्यते तत्राप्यन्ये निवसन्त्येव न तु केवलं ब्राह्मणा निवसन्तीति ब्राह्मणग्राम इतिभवति~। एवमेव क्षत्रियग्रामो वैश्यग्रामः शूद्रग्राम इत्यादीनामुपपत्तिरनुसन्धेया, न तु क्षत्रियग्राम इत्युक्ते क्षत्रिया एव निवसन्ति,पितरजका गृह्यन्ते, ब्राह्मणप्रधानग्रामेऽते सुखेन निवसन्त्येव~। अत एव आकडारसूत्रे अब्राह्मणको देशः, अवृषलको देशः, वीरपुरुषको ग्रामः इत्यादीन्युदाहरणानि सामञ्जस्येन सङ्गच्छन्ते~॥

\begin{center}
\textbf{\Large मार्गाः \textendash\ }
\end{center}

योजनं गच्छति ५~। १~। ७४~। इति सूत्रे {\qt क्रोशशतादभिगमनमर्हति कौशशतिको भिक्षुः, योजनशतादभिगमनमर्हति यौजनशतिको गुरु} इत्यादिना देशे निवसन्तः पुरुषा योजनशतादागच्छन्तं गुरुमधिगच्छन्तीति सुगमाः सानुबन्धाः सूपन्यस्ताश्च पन्थानो बोध्यन्ते~। अत एव मार्गाणां प्रकारविशेषा उत्तरपथेनाहृतञ्च ५~। १~। ७७~। इत्यत्रानेकान्युदाहरणानि प्रदर्शयता बोध्यन्ते~। तानि यथा \textendash\ वारिपथेन गच्छति \textendash\ वारिपथिकः, वारिपथेनाहृतम् \textendash\ वारिपथिकम्, जङ्गलपथेन गच्छति \textendash\ जाङ्गलपथिकः,

\newpage
% ( १५ ) 

\noindent
जङ्गलपथेनाहृतं जाङ्गलपथिकम्,~। स्थलपथेन गच्छति \textendash\ स्थालपथिकः, स्थलपथेनाहृतं \textendash\ स्थालपथिकम्, कान्तारपथेन गच्छति कान्तारपथिकः, कान्तारपथेनाह्वतं \textendash\ कान्तारपथिकम्, अजपथेन गच्छति \textendash\ आजपथिकः, आजपथिकम्, शङ्कुपथेन गच्छति \textendash\ शाङ्कुपथिकः,शांकुपथिकम् इत्यादीनि~। अनेनोदाहरणजालेन तदा मार्गाणामनेके प्रकारास्तेषु भिन्नाः संव्यवहाराश्चासन्निति सुभगाः सुसम्पन्नाः सुभिक्षाः सम्पन्नपानीया देश भारतभूमिमलङ्कुर्वन्ति स्मेति विशेषणानां चाजातेः १~। २~। ५२ इत्यत्र {\qt सुभिक्षः सम्पन्नपानीयो बहुमाल्यफलो देशः} इत्युदाहरणदानेन समर्थयति भगवान् भाष्यकारः~॥

\begin{center}
\textbf{\Large जन्मना जातिः \textendash\ }
\end{center}

ईदृशे सुभिक्षे सम्पन्नपानीये बहुमाल्यफले देशे मनुष्याः किं समानधर्मिणः समानरूपाः समानगुणविशिष्टा वा वसन्ति स्म, उत विभिन्नजातीया विभिन्नरूपा विभिन्नगुणविशिष्टा वेति विचार्यमाणे सर्वेऽप्यत्र सुखेन निर्वहन्ति स्मेति वक्तुं पार्यते~। जातयश्चापि यथा सम्भवं यत्र तत्रोदाहरणेषु प्रतिपन्ना उ्लेखनीया एवेति जातिलक्षणं प्रथमतः प्रदर्श्यते~। अतिशायने तमबिष्ठनौ ५~। ३~। ५५~। इति सूत्रे {\qt यदप्युच्यते जातेर्नेति वक्तव्यमिति~। न वक्तव्यम्~। जननेन या प्राप्यते स्वा जातिः~। न चैतस्यार्थस्य प्रकर्षाप्रकर्षौ स्तः~।} इत्यनेन जन्मना या प्राप्यते सा जातिरिति गुणोपष्टम्भादिना न जातिरिति प्रतिपादयति~। यथा काकाज्जातमात्रै प्राणिनि काकत्वं, शुकाज्जातमात्रे च शुकत्वं, मनुष्याजातमात्रे च मनुष्यत्वं, तथा शूद्राज्जातमात्रे शूद्रत्वं, क्षत्रियाज्जाते च क्षत्रियत्वमिति जातमात्रे वस्तुनि तत्तज्जातिमत्वमस्त्येवेति~। भाष्यार्थश्च \textendash\ तमबिष्ठनौ प्रत्ययौ जातेर्न भवन्तीति वक्तव्यमित्याक्षिप्य या जननेनैव प्राप्नोति तत्र कः प्रसङ्गः प्रकर्षाप्रकर्षयोरिति जातेर्नैति न वक्तव्यमिति~। जातेनेति वक्तव्यमित्यस्योदाहरणप्रसङ्गे {\qt वृक्षोऽयम्, प्लक्षोऽयम्} इत्यु \textendash\ दाहृतम्~। यत्र तृणानि लता गुल्मानि वा वर्तन्ते न वृक्षास्तत्र महान्तं सुच्छायं महत्फलं वृक्षमेकमुपलभ्य वृक्षोsयमित्युच्यते~। तत्र लतावीरुधादिभ्यो वृक्षेऽतिशयः सम्प्रतीयते, स एव व्यक्तावुपलभ्यमानोऽतिशयो दक्षत्वजातावप्युपलभ्यत इति जातिवाच्चकाद्वृक्षशब्दात्तमबिध्ठनौ प्राप्नुतस्तदर्थां जातेर्नैति वक्तव्यमिति सम्प्रतिपन्नं, जन्मना प्राप्यमाणे वस्तुनि न कश्चिदतिशय इति समाधत्ते~। कश्चिद्राजगृह उत्पद्यते, कश्चिद्धिक्षुगृहे, तयोर्बालकयोः को गुणातिशयः ? एवञ्च जन्मना जातो ब्राह्मणो जन्मना जातः क्षत्रियो जन्मना जातश्च शूद्व इत्यनयोर्न कोsपि दृश्यो विशेषः, केवलज्जातिभेददर्शनात्तेषां भिन्नाः संस्कारा भवन्तीत्येव~। यथा काकाज्जाते शिशौ न स्तनपानसंस्कारः काकेन क्रियते न वा गोवत्से काकवत् मुखेनान्नरससंस्कारं गौर्विधत्ते तथा क्षत्रियबालके भिन्नाः संस्काराः, शूद्रबालके चान्ये, ब्राह्मणबालके चान्ये \textendash\ इति जातिभेदनिमित्तं संस्कारभेद इत्येव, न कश्चित्तेषां बालकानां दृश्यः प्रकर्षश्चाप्रकर्षश्च भाष्यकारसमये केषाञ्चनेष्टः~॥\\

अत एव पुंयोगादाख्यायाम् ४~। १~। ४८~। इति सूत्रे \textendash\ त्रीणि यस्यावदातानि विद्या योनिश्च कर्म च~। एतय्छिवं विजानीहि ब्राह्मणाग्र्यस्य लक्षणम्~॥ इत्यादीनि ब्राह्मणादीनां लक्षणान्युपसंगृहीतानि भाष्ये~॥\\

अथ संस्कारभेदोऽपीदानीमस्माभिर्नाभिप्रेयते, ये संस्कारा ब्राह्मणबालकेषु क्रियन्ते त एव क्षत्रियबालकेषु अन्येषु वेत्या \textendash\ ग्रहश्चेत, क्षत्रियबालका ब्राह्मणसंस्कारैः संस्क्रियमाणा अजसमूहे पाल्यमानो व्याघ्रशिशुर्वकं दृष्ट्वाऽजशब्दमिव शत्रु द्वष्ट्वा यज्ञोपवीतं प्रदर्शयेयुः~। एवश्च जन्मना जातिमभिमन्यमानास्तदानींतनीयाः सुखेनैव कालं व्यतिचक्रमुः~॥\\

\begin{center}
\textbf{\Large संस्कारादिभिर्न जात्युत्पत्तिः \textendash\ }
\end{center}

केवलं क्षत्रियसंस्कारैः संस्क्रियमाणाः सर्वेऽपि क्षत्रिया भवेयुरित्यपि नेत्यग्रिमोपन्यासेन समवधेयम्~। अनृष्यानन्तर्ये बिदादिभ्योsञ् ४~। १~। १०४~। इत्यत्र अनृष्यानन्तर्य इत्युच्यते तत्रेदं न सिध्यति \textendash\ कौशिको विश्वामित्र इति~। किं कारणम्~। विश्वामित्रस्तपस्तेपे नानृषिः स्यामिति, तत्र भवान् ऋषिः सम्पन्नः~। स पुनस्तपस्तेपे नानृषेः पुत्रः स्यामिति, तत्रभवान् गाधिरपि ऋषिः सम्पन्नः~। स पुनस्तपस्तेपे नानृषेः पौत्रः स्यामिति, तत्रभवान् कुशिकोsपि ऋषिः सम्पन्नः~। तद्देतदृष्यानन्तर्यम्भवति~। इत्युक्तम्~। क्षत्रियो विश्वामित्रो ऋषित्वसम्पत्तये तपश्चचार तपसा स ऋषिः समभूत्~। एवमेव तपसा गाधिकुशिकावपि ऋषी समभूताम्~। विरुद्धसंस्कारमात्रेण जातिविपरिवर्तनं न नैव भवतीति तेषां मतिः~। अन्यथा चाण्डालेष्वपि ब्राह्मण्यमाधातुं कृतप्रयत्नं कञ्चन धर्मपरिवर्तनक्षमं महामहोपाध्यायमाहूय ब्राह्मण्यमनुभूयमानो विश्वामित्रस्तपश्चरणक्केशसंवरणादिकं नाकरिष्यत्~। एवञ्च तपः सामर्थ्यमपि तदा नितरां सम्मन्यते स्मेत्यपि निर्विवादम्~। तदानीं जातिभेदसत्वादेव आदेच उपदेशेऽशिति ६~। १~। ४५ इति सूत्रे तथा ह्यर्थो गम्यते लोके \textendash\ अब्राह्मणमानयेत्युक्ते ब्राह्मणसदृशमानयति, नासौ लोष्तमानीय कृती भवति इति व्यवहारः सामञ्जस्येनोपद्यते~॥

\begin{center}
\textbf{\Large जातिभेदाः \textendash\ }
\end{center}

जातिभेदसमाश्रयणादेव च प्रत्यभिवादेऽशूद्रे८~। २~। ८३~। इति सूत्रे भो राजन्यविशां वेति वक्तव्यम्~। राजन्य \textendash\ इन्द्रवर्माहं भोः~। आयुष्मानेधीन्द्रवर्मा ३ न्~। विश्इन्द्रपालितोऽहं भोः~। आयुष्मानेधीन्द्रपालिता ३ इत्यादिना राजन्यवैश्ययोर्विकल्पेन, शूद्बब्राह्मणयोश्च नित्यं प्लुतं त्रिधाय शर्मवर्मपालितदासास्तत्तजातिविशेषा अपि प्रदर्शन्ते~। ब्राह्वाणादीनां लक्षणञ्चाकारि भाष्यकृता

\newpage
% ( १६ ) 

\noindent
तेन तुल्यं क्रिया चेद्वतिः ५~। १~। ११७~। इत्यत्र सर्व एते शब्दा गुणसमुदायेषु वर्तन्ते \textendash\ ब्राह्मणः क्षत्रियो वैश्यः शूद्र इति~। गुणसमुदाये, एवं ह्याह \textendash\

\begin{quote}
{\qt तपःश्रुतश्च योनिश्चेत्येतद्ह्मणकारकम्~।\\
तपःश्रुताभ्यां यो हीनो जातिब्राह्मण एव सः~॥}
\end{quote}

तथा गौरः शुच्याचारः पिङ्गलः कपिलकेश इत्येतानभ्यन्तरान् ब्राह्मण्ये गुणान् कुर्वन्ति~। समुदायेषु च शब्दा वृत्ता अवयवेष्वपि वर्तन्ते~। एवमयं ब्राह्मणशब्दः समुदाये वृत्तोद्वयवेष्वपि वर्तते इत्यादिलक्षणानि ब्राह्मणादीनां तत्तज्जातिविशिष्टानामुक्तानि विराजन्ते प्राचीनानाम्~। अत्रहि तेन तुल्यं क्रिया चेदित्युक्तौ तृतीयान्तेन ब्राह्मणादिशब्देन क्रियायाः सामानाधिकरण्ये वतिप्रत्ययो विधीयते~। ब्राह्मणेन तुल्यमधीते क्षत्रिय इत्यादौ ब्राह्मणक्षत्रिययोरध्ययनक्रिया तुल्या प्रतीयते, तत्र क्रियया सामानाधिकरण्यमनुपपन्नमिति प्रत्ययो न स्यादित्याक्षिप्य समुदाये वृत्तः शब्दा अवयवेष्वपि वर्तन्त इति अवयवबोधकशब्देनापि तेषां सामानाधिकरण्यम्, तैलं भुक्तमित्यादौ यथा समुदायेsपि वर्तमानस्तैलशब्दो भुक्ता या मात्रा तत्रापि व्यवह्रियते तथा क्रियादावपि अवयवे ब्राह्मणशब्दस्य सत्वेन न सामानाधिकरण्यस्यानुपपत्तिः~। अत एव शौचाचारादिविहीने केवलजन्मना जातमात्रेऽपि तत्तच्छब्दव्यवहार उपपद्यते~। गौरः शुच्याचार इत्यत्र गौरपदेन गौरवर्णविशिष्टो न, किन्तु \textendash\

\begin{quote}
{\qt अष्टवर्षा तु या दत्ता श्रुतशीलसमन्विते~।\\
सा गौरी तत्सुतो यस्तु स गौरः परिकीर्तितः~॥}
\end{quote}

इति धर्मशास्त्राभिप्रायेण गौरीपुत्र एव गौरपदवाच्यः~। तथा च स्त्रीणामभ्युन्नतिकामैरपि विवाहवयोमर्यादाकरणं शास्त्रविरूद्धं विशुद्धवर्णोत्पत्तिप्रतिबन्धकमिति दूरतः परिवर्जनयमेवेति~॥

एतादृशमेव लक्षणं ब्राह्मणादीनां नञ् २~। २~। ६~। इति सूत्रे विस्तरेणोक्तमिति तत एवावगन्तव्यम्~॥

\begin{center}
\textbf{\Large उच्चनीचभावविषये \textendash\ }
\end{center}

पूर्वनिपातप्रकरणे \textendash\ अल्पाच्तरम् २~। २~। ३४~। इति सूत्रे ब्राह्मणक्षत्रियवैश्यशूदाः सहोच्चारिताः क्व निपतन्तीति विचारावसरे {\qt वर्णानामानुपूर्व्येण पूर्वनिपाता भवन्ति \textendash\ इति वक्तव्यम्~। ब्राह्मणक्षत्रियविट्शूद्राः} इत्यादिना वेदादौ येन क्रमेण तेषामुत्पत्तिः कथिता तेनैव क्रमेण पूर्वनिपातोsपि कर्तव्य इत्यादि प्रतिपादयन् भगवान् न तेषां श्रेष्ठत्वमुच्चत्वं वाssह, किन्तु वेदे ब्राह्मणोत्पत्तिः पूर्वमभिहितेति ब्राह्मणस्य पूर्वनिपात इत्येव~। यदि तेषां पूज्यत्वापूज्यत्वे प्रसिद्धे समभूतां तदा तस्मिन्नेव सूत्रे {\qt अभ्यर्हितञ्च~।} अभ्यर्हितश्च पूर्वं निपतीति वक्तव्यम्~। मातापितरौ~। श्रद्धामेधे इत्यनेन मातापित्रोर्मातुरेवाभ्यर्हितत्वात् \textendash\ पित्रपेज्ञया पूज्यत्वात् पूर्वनिपातो विधयते तेनैव ब्राह्मणादीनामपि पूर्वनिपाते सिद्धे {\qt वर्णानामानुपूर्व्येण} इति वार्तिकप्रणयनम् स्वव्याख्यानेन तद्ंबृहणश्चोभयमपि विफलमेवाभविष्यत्~। निरर्थकमेवेदमुच्यत इति चेत्, अर्धमात्रालाघवेन पुत्रोत्सवं मन्यन्ते वैयाकरणा इत्याद्यापामरप्रसिद्धो वैयाकरणाचारो ननु अत्र प्रमाणीकर्तव्यः~। {\qt दर्भपवित्रपाणिः प्राङ्गमुख उपविश्य महता यत्नेन सूत्रं प्रणिनाय, तत्र वर्णेनाप्यनर्थकेन न भवितव्यम्} इत्याद्युपदिशन् भाष्यकारो वर्णानामानुपूर्व्यैणेति समग्रं वचनमेव निरर्थकं व्याक्रियमाणः संभवेदित्याशङ्कनमपि साहसमात्रमेव स्यात्~॥ एवश्च जातिभेदप्रबन्धे महति वर्तमानेsपि न कश्चित्तेषामुच्चनीचादिव्यवहार आसीत्, न वा कश्चिद्विरोधोऽप्यासीदिति सुनिश्चितं प्रतीयते~। पस्पशाह्निके ब्राह्मणेनावश्यं शब्दा ज्ञेयाः इत्याद्युपन्यासेऽपि ब्राह्मणशब्दः संस्कारमात्रे प्रवृत्तस्वैवर्णिकान् ग्राहयति~। तत्र त्रैवर्णिकैरवश्यं शब्दा ज्ञेया इतरेषां शब्दाज्ञाने न कश्चित्प्रत्यवायः~। यदि शब्दज्ञानजिज्ञासा, न कश्चित्प्रतिबध्नाति~। जिज्ञासाभावे न कश्चिदुपरुणद्धि च~। एवश्च जात्यादिसंव्यवहारेण बहूपकृतं समाजस्य नानिष्टं किञ्चिदित्येवावसीयते जात्यादिविचारविमर्शनेन~॥

\begin{center}
\textbf{\Large सामाजिकव्यवहारः \textendash\ }
\end{center}

तदानींतनानां सामाजिकव्यवहारेणैतत्सुस्पष्टं प्रतिपद्येतेति सोऽप्यवश्यं विमर्शनीयः~। जीवति तु वंश्ये युवा ४~। १~। १६३ इति सूत्रे {\qt वृद्धस्य च पूजायाम्~। वृद्धस्य च पूजायां युवसंज्ञा वक्तव्या~। तत्रभवन्तो गार्ग्यायणाः, तत्रभवन्तो वात्स्यायनाः~। का पुनरिह पूजा ? युवत्वं लोके ईप्सितं पूजेत्युपचर्यते~। तत्र भवन्तश्च युवत्वेनोपचर्यमाणाः पूजिता भवंति~। } इत्युदाहरणं तदानीन्तने व्यवहारे प्रकाशमापादयति~। वृद्धसंज्ञकाद्युवसंज्ञकाच्च भिन्नाः प्रत्यया विद्यन्ते, यथा च भवति \textendash\ गार्ग्यो गार्ग्यायण इति~। तत्र वार्तिकेनानेन पूजायां कर्तव्यायां वृद्धसंज्ञकस्यापि युवसंज्ञा भवति, भवति च तस्मात्तादृशः प्रत्ययः~। यतो युवत्वं लोके सर्वैः काम्यते अतो युवत्वेन व्यवहारः पूजेत्युपचर्यते~। अतो यत्र गार्ग्यशब्दव्यवहारः समापन्नस्तत्र माननीये पूज्ये गार्ग्ये गार्ग्यायणशब्दव्यवहारः कर्तव्य इत्यादिशतो वररुचिपतञ्जली~॥\\

यस्य च वस्तुतो यूनो गार्ग्यायणस्य दुर्वृत्तं पश्यति जनस्तत्र वास्तविके गार्ग्यायणेऽपि गार्ग्यशब्दव्यवहारमनुमनुतः \textendash\ अपत्यं पौत्रप्रभृतिगोत्रम् ४~। १~। १६२ सूत्रे {\qt जीवद्वंश्यञ्च कुत्सितम्~। जीवद्वंश्यश्च कुत्सितं गोत्रसंज्ञं भवतीति वक्तव्यम्~। गार्ग्यस्त्वमसि जाल्म, वात्स्यस्त्वमसि जाल्म} इत्यादिना~। वस्तुतो युवसंज्ञकस्यापि दुर्वृत्तादिविशेषान्न युवसंज्ञकेन व्यवहारः किन्तु गार्ग्यस्त्वमसि

\newpage
% ( १७ ) 

\noindent
जाल्मेति गार्ग्यशब्देन व्यवहारं बोधयति~। षष्ट्यब्दिपूर्तिकरणतत्परैरप्येतदवर्धेयं \textendash\ तादृशपूतिकरणेन न पूजा संप्रतिपद्यते, किन्तु वृद्धस्त्वमसि, प्रत्यवसेया व्यवहारा इत्येव बोध्यते~। यथा निमन्त्रितसमाजे पण्डितापण्डितसमवाये आगच्छन्तमपण्डितमपि स्वागताध्यक्षो महतोत्साहेनाचारविशेषं {\qt आगन्तव्यं महाभागैः, भवदागमनेन पूजिता वयम्} इत्यादि प्रतिपादयति, अपण्डितश्च पूजाहमात्मानं मन्यते तथैवायं व्यवहारः सम्प्रतिपद्यत इति पाश्चात्याचारचतुराः षष्ट्यब्दिपूजकाः पूज्याश्च इतो नावदधत इत्येव कुशलम्~॥ कुत्सा \textendash\ व्यवहारः क्वचिदुपदर्शितोऽपि दृष्ठिपातमर्हत्येव~। तथाहि \textendash\ {\qt का पुनरिह कुत्सा ? पितृतो लोके व्यपदेशवताsस्वतन्त्रेण भवितव्यम्~। य इदानीं पितृमान् खतन्त्रो भवति स उच्यते \textendash\ गार्ग्यस्त्वमसि जाल्म~। न त्वं पितृतो व्यपदेशमर्हसीति} इत्यादिना यत्र यः पितुराज्ञां तिरस्करोति स कुत्सितो भवति किमुत \textendash\ इति विचारणीयं सुधीभिः~॥ पितृपितामहादीनां ऋषीणामाज्ञाव्यतिक्रमे \textendash\ इति विचारणीयं सुधीभिः~॥

\begin{center}
\textbf{\Large विद्यार्थिनः \textendash\ }
\end{center}

समाजस्य प्राणभूता विद्यार्थिनः, तद्विषये {\qt कमण्डलुपाणि छात्रमद्राक्षीत्} इत्याद्युक्तयो बहुश उपलभ्यमाना विद्यार्थिनः सुचरित्राः स्वाध्यायशालिनो ब्रह्मचर्यरताः समाजेन समादृता देशभूषणा भवन्ति स्मेति बोधयन्ति~। ये च दुश्चरिता अनध्ययनवन्तस्ते कठोरतरं निर्भर्त्सिता भवन्तीति ध्वाङ्क्षेण क्षेपे२~। १~। ४२~। इति सूत्रे {\qt यथा तीर्थकाका न चिरं स्थातारो भवन्ति~। एवं यो गुरुकुलानि गत्वा न चिरं तिष्ठति स उच्यते तीर्थकाक इति~।} इत्यनेन ज्ञायते~। शालां गत्वोपस्थितिं परिज्ञाप्य ये शालायामापणे वा बाभ्रम्यमाणास्तिष्ठन्ति त उच्यन्ते तीर्थकाका इति हि सामयिकोऽर्थः~॥ यथा छात्राणां स्तुतिनिन्दे प्रवर्तेते न तथाऽध्यापकानाम्~। अत्रेदमपि बीजं स्यात \textendash\ ये प्रयोजनमनपेक्ष्य कर्तव्यमिति बुध्या शिष्यान् पाठयन्ति ते कथं निन्दनीया भवेयुः~। यथेदानीं गुरुकुलानां व्यवस्था सम्प्रवर्तते न तदानीं तादृशी आसीत्~। तदा च ग्रामे ग्रामे विद्याव्यसनिनः कर्तव्यबुध्याऽध्यापनवृत्ति सम्पादयन्तः सर्वानपि शिशूनध्यापयन्ति स्मेति ज्ञायते~। यथेदानीं पाठशालाकरणमपि व्यवसायः, यथा वा महता व्ययेनापि न तत्त्वाध्यापनं, न तदानीमेतत्किम \textendash\ प्यासीदिति प्रतीयते~। तदा गुरुशिष्यसम्बन्धश्च छत्रादिभ्यो णः ४~। १~। ६२~। इति सूत्रे किं यस्य छत्रधारणं शीलं स छात्रः ? किञ्चातः ? राजपुरुषे प्राप्नोति~। एवं तर्हि उत्तरपदलोपोsत्र द्रष्टव्यः \textendash\ छत्रमिव छत्रम्~। गुरुश्छत्रम्~। गुरुणा शिष्यश्छत्रवत् छाद्यः, शिष्यैण गुरुश्छत्रमिव परियाल्यः~। {\qt इत्येवमुपवर्णित आस्ते~।} न तदा केचन नेतारः शिष्याणां बुद्धिभेदमकुर्वन्~। न वा गुरूनुपहसन्तस्ते आत्मनः प्रतिष्ठां शिष्यसमुदायाग्रेऽवदन्, येनात्मप्रतिष्ठाबोधनेन स्वार्थो न सिध्यति परार्थश्च हीयते~। तदा चैवमुपवर्णितः सम्बन्धो गुरुशिष्ययोः समवस्थित आसीदिति प्रतिपादयति~॥ एतादृशी चिन्ता तु दृश्यते \textendash\ अनद्यतने लुद् ३~। २~। १५~। इति सूत्रे {\qt अयं तु कदाऽध्येतः य एवमनभियुक्त इति~। अध्येतेवायमध्येता नायमध्येष्यते} इति~। य एवमभ्यासहीनः तस्याध्ययनं कथं स्यादिति तदर्थः~। तथा विद्यार्थिनां शीतादिनिवारणदक्षा अपि तदानीन्तनाः \textendash\ हेतुमति च ३~। ११~। २६~। सूत्रे {\qt भिक्षा वासयति, कारीषोऽग्निरध्यापयति~। भिक्षाश्चापि प्रचुरा व्यञ्जनवत्यो लभ्यमाना वा सम्प्रयोजयन्ति, तथा कारीषोऽग्निनिर्वाते एकान्ते सुप्रज्वलितोऽध्ययनं प्रयोजयति} इत्यनेन ज्ञायते~॥ चतुर्थी० २~। १~। ३~। इत्यत्रापि {\qt छात्राय रुचितम्, छात्राय खदितम्} इत्यादिना छात्रोपकारपरान् तदानींतनान्बोधयति~॥ खट्वाक्षेपे २~। १~। २५~। इति सूत्रे {\qt क्षेप इत्युच्यते कः क्षेपो नाम ? अधीत्य स्नात्वा गुरुभिरनुज्ञातेन खट्वारोढव्या~। य इदानीमतोsन्यथा करोति स उच्यते खट्वारूढोsयं जाल्मो नातिव्रतवान्} इत्यनेन गुर्वनुज्ञां विनैव योsग्नीनुपसंगृह्णाति स एवं निन्द्यते \textendash\ खट्वारूढोऽर्यं जाल्म इति~। तदानीं गुर्वनुज्ञां विना धर्म्यों विवाहोऽपि निन्द्यते किमुत कन्यासंसर्गश्छात्रस्य क्षमेत वा~॥ छात्राणां बुध्याद्युपलब्धिरपि परीक्ष्येतेति तद्धितश्चासर्वविभक्तिरिति सूत्रे {\qt समानमीहमानानां चाधीयानानां च केचिदर्थैर्युज्यन्त, अपरे न~। तत्र किमस्माभिः कर्तुं शक्यम्~। स्वाभाविकमेतत्} इति प्रतीयते~। अनेन चोपष्टम्भेन तदा गुरुकुलानि महान्ति सर्वाध्ययनप्रकारविशेषाण्यासन्नित्यवगन्तुं युक्तम्~॥\\

वृद्धिरादैच् १~। १~। १~। सूत्रे {\qt एवं हि दृश्यते लोके य उदात्ते कर्तव्येsनुदात्तं करोति खण्डिकोपाध्यायस्तस्मै चपेटां ददाति} इति दर्शनात्, एकस्मिन् गुरुकुलेsनेकेsध्यापकाः समये समये पाठयन्ति~। कश्चित् खण्डिकोपाध्यायः, कश्चिद्वर्गाध्यापकः, कश्चिदृगध्यापक इति तत्तद्विषयाध्यापकाः खण्डिकोपाध्यायपदोपादानेन बोध्यन्ते~। अथर्ववेदे पादस्य खण्डिकापदेन न व्यवहारः, तदध्यापकश्च खण्डिकोपाध्यायः~॥\\

ननु {\qt खण्डिकोपाध्यायः शिष्याय चपेटां ददाति} इत्युदाहरणदर्शनात् सुसंस्कृतेनाधुनिकेन समाजेन विनिन्दिता बालानां ताडनादिप्रकारास्तदा प्रचलिता आसन्निति ग्राम्यत्वं तेषां प्रदर्शितं भवतीति चेन्न~। शिष्योपकारार्थं क्वचित्तादृशं ताडनमपि गुणायैव भवतीति \textendash\ वाक्याहेरामन्त्रितस्यासू्या० ८~। १~। ८~। इति सूत्रोक्तेन व्यवहारेण प्रतीयते~। तत्र हि {\qt असूयाकुत्सनयोः कोपभर्त्स} \textendash\ नयोश्चैकार्थत्वात् पृथकत्वनिर्देशानर्थक्यम्~। असूया \textendash\ कुत्सनमित्येकोऽर्थः~। कोपो भर्त्सनमित्येकोर्थः~। एकार्थत्वात्पृथक्त्वनिदे \textendash\ शोऽनर्थकः~। न ह्यनसूयम् कुत्सयति, न चाप्यकुपितो भर्त्सयते~। ननु च भोः अकुपिता अपि दृश्यन्ते दारकान् भर्त्सयमानाः~। अन्ततस्ते तां शरीराकृतिं कुर्वन्ति या कुपितस्य भवति~। एवं तर्ह्याह

\begin{quote}
{\qt सामृतैः पाणिभिर्घ्नन्ति गुरवो न विषोक्षितैः~।\\
लालनाश्रयिणो दोषास्ताडनाश्रयिणो गुणाः~॥}
\end{quote}

३ प्र०पा०प्रस्ता० 

\newpage
% ( १८ ) 

इत्युक्तम्~। एवश्चाल्पबुद्धीनां बालानामुत्पथप्रतिपन्नानां चपेटादिप्रकारेण स्वल्पेन दण्डेन दण्डयन्तः परमार्थत उपकारमेव कुर्वन्तीति सुसंस्कृतैराधुनिकैरवधेयम्~॥\\

तादृशादध्यापकात् अन्तर्धानमपि कृतापकारिणां शिष्याणां \textendash\ अन्तर्द्धौ येनादर्शनमिच्छति १~। ४~। २८~। सूत्रे {\qt उपाध्यायादन्तर्धत्त इति~।} पश्यत्ययं यदि मामुपाध्यायः पश्यति ध्रुवं मे प्रेषणमुपालम्भो वेति स बुध्या सम्प्राप्य निर्वर्तते इत्यमेन ज्ञायते~। अविनीतस्य पाठशालातो निष्क्रमणं दण्डादिनोपालम्भो वा प्रदीयते स्मेति पाठशालासम्बन्धं सुनियतं सुचारुरूपञ्च निर्वर्तयितुं स्वाधी नप्रयत्नास्तत्पराः कुलपतय इत्यध्ययनाध्यापनपद्धतिः सुचारुरूपेत्यविवादम्~॥\\

विद्याश्च नानाविधाः \textendash\ काश्चन विद्यान्तपदेन बोध्यन्ते, काश्चिच्च लक्षणान्तपदेन सूत्रान्तपदेन वेति वेदव्याकरणमीमांसादिशास्त्रेष्वेवैते परिनिष्ठितास्तेषु शास्त्रेष्वेव विवदमानाः समवर्तेरन्नित्यपि न मन्तव्यम्~। ऋतूक्थादिसूत्रान्ताठ्ठक् ४~। २~। ६०~। इति सूत्रे तदधीते तद्वेदेत्यधिकारे विधाध्ययनशीलानां निर्देशेन वेदव्याकरणमीमांसादिविद्यावत् यन्त्रतन्त्रनीतिक्रीडादिलौकिकविद्याकलादि \textendash\ स्वरूपमप्युपदर्शयति भाष्यकारः~। तत्र हि \textendash\ {\qt वायसविद्यिकः, सार्पविद्यिकः, गौलक्षणिकः, आश्वलक्षणिकः, पाराशरकल्पिकः. पारयाज्ञिकः, वार्तिकसूत्रिकः, सांग्रहसूत्रिकः, आङ्गविद्यः, क्षात्रविद्यः, धार्मविद्यः, त्रैविद्यः, उक्थान्यधीते \textendash\ औक्थिकः, यज्ञमधीते याज्ञिकः, ऐतिहासिकः, सर्ववेदः, सर्वतन्त्रः, पञ्चकल्पः, द्वितन्त्रः, आनुसुकः, लक्ष्यिकः, लाक्षणिकः, चान्दनगन्धिकः, शतपथिकः, षष्टिपथिकः} इत्यादौ वायसविद्यादिविद्यास्तत्रोपलक्षणत्वेन गृहीता दृश्यन्ते~। येन राज्यव्यवस्थायामुपयुक्ताः लौकिकनिर्वाहाय च प्रतीताः पदार्थविज्ञानादिषु निष्णाता विद्यार्थिन एकस्यामेकस्यां विद्यायां कृतप्रयत्नाः संशोधनदक्षा अप्यासन्निति प्रतीयते~। तासु च बह्वयः प्रायो लुप्तप्राया इदानीमवगम्यते

\begin{center}
\textbf{\Large विहाराः \textendash\ }
\end{center}

विद्यार्थिनां विहारादिवर्णनमपि न वेति विभाषा सूत्रे अभिजानासि देवदत्त यत्कश्मीरेषु वत्स्यामः, यत्कश्मीरेष्ववसाम, यत्तत्रौदनं भोक्ष्यामहे यत्तत्रौदनमभुञ्जमहि~। अभिजानासि देवदत्त कश्मीरान् गमिष्यामः कश्मीरानगच्छाम, तत्रौदनं भोक्ष्यामहे तत्रौदनमभुञ्जमहि इत्यादिनोपपादितं सन्दृश्यते~। तथा \textendash\ लुङ् २~। ११०~। इति सूत्रे {\qt गमिष्यामो घोषान् पास्यामः पयः~। शयिष्यामहे पूतीकतृणेषु} इत्यादिना विहारादिवर्णनं बहुशो व्याख्यातं सामञ्जस्येन संगच्छते~॥ भारतभूललामभूतः कश्मीरस्तदाऽपि तादृश एव ललामभूत आसीत्, तत्रत्यमन्नमपि स्वादु मृदु रुचिरं विशदं चासीद्यत्तदभिलाषेण कश्मीरगमनं, तद्गुणावकृष्टैः पश्चात्तत्स्मरणं वोभयमप्यत्र प्रदर्शितम्~। कश्मीरसम्भवा जलवायवः, विचित्रा भूभागाः, हिमाच्छन्नानि शिखराणि, अन्यत्रानुपलभ्यमाना विचित्राः फलादिजातयः, सुसम्पन्ना वैशिष्ट्याधायकाश्च तण्डुलाः सर्वमपीदमिदानीमास्ते~। केवलं तत्र विद्यायाः पारंगताः साहित्यादिशास्त्रप्रणेतारः सकुशला जना इदानीं नोपलभ्यन्त इति हा हन्त कालाय तस्मै नमः~॥

\begin{center}
\textbf{\Large समाजसेवाविषये \textendash\ }
\end{center}

सामाजिकसुखसम्पत्तये तदानीमपि प्रयत्नशीला जना आसन्निति छदिरुपधिबलेर्ढञ् ५~। १~। १३~। इति सूत्रप्रदर्शितेनैकेनो \textendash\ दाहरणेन सम्यक् सम्प्रतीयते~। तत्र हि {\qt उपानदर्थस्तिलकल्कः} इत्युदाहृतम्~। तेनोपानव्द्यवहारिणः तिलकल्कसंज्ञकं ( बूटपालिश ) व्यवहरन्ति स्म~। उपानहश्चोपलेपनसमर्थाः सुरूपा उपलेपनेन संस्क्रियमाणा व्यवहारसाधना आसल्निति नागरिकव्यवहारातिशयं च्यञ्जयन्ति~॥ \\

जनाः संव्यवहारे लोलुपा अलोलुपाः स्वार्थैकप्रवणा निःस्वार्थाश्च सर्वदैवोपलभ्यन्ते~। तत्र निःस्वार्थजनबाहुल्ये निःस्वार्थनेतृषु वा सत्सु स्वार्थप्रवणा अपि जना न व्यतिक्रामन्तीति समाजः सुचारुरूपेणावतिष्ठते~। तदा च कश्चित्स्वार्थप्रवणो वाच्यतामुपयाति~। अत एव समर्थः पदविधिः २~। १~। १~। इति सूत्रे {\qt राजविशिष्टाया गोः क्षीरेण सह समासो भवति, न केवलायाः~। किं वक्तव्यमेतत् ? नहि~। कथमनुच्यमानं गंस्यते~। यथैवायं गवि यतते न च क्षीरमात्रेण सन्तोषं करोति, एवं राजन्यपि यतते \textendash\ राज्ञो या गौस्तस्या यत्क्षीरमिति} इत्युदाहृतमुपपद्यते~। यथेदानीं अभिनन्दनपत्रपेटिकाक्रयणादिषु अहमहमिकया जनः सम्प्रवर्तते, यथा वा हस्ताक्षरोपलब्धये यतते, यथा च राजपुरुषादीनामागमने स्वागतसम्भारसम्भृतावुपलभते तथैव तदानीमपि गोक्रयणवेलायां राज्ञो गां बहुवित्तव्ययेनापि राज्ञ इयमित्यभिनिवेशात्क्रीणाति, राजसामीप्योपलब्धये यतते चेति प्रतीयते~। राजसम्बन्धाभिलाषुक इदानीं सम्पूज्यते तदानीञ्च वाच्यतामुपयाति स्मेति महान् भेदः~। अत एवेदानीममात्यादीनां राजकीयपुरुषाणामागमनोत्सवे अहमहमिकया स्वागतसमारम्भो विधीयते, तत्रोदासीन एवेदानीं वाच्यतामुपयाति, स एव दण्ड्यो भवति चेति सर्वजनानुभवसिद्धे वस्तु नाधिकं प्रकाशमपेक्षते~॥\\

भाष्यकारसमये विचक्षणः खाधीनक्रियश्च समाज आसीदित्यत्र न कश्चित्सन्देहहेशोऽपि~। ये च समाजसेवकान् दम्भरहितान् परोपकारार्थमेव यतमानान् स्वार्थरहितान् देशहितार्थमेव दम्यपदव्यधिष्ठितान् मन्यन्ते तेषां समाजो ननु गर्तायामापन्न इवावगन्तव्यः~। अत्रार्थं समाजसेवकाभिधानां कियन्मूल्यं प्राचीनैर्निर्द्धारितमासीदित्यवश्यमेवावगन्तव्यम्~। तच्च यथा \textendash\ हेतुमति चेति सूत्रे प्रवृत्तिर्ह्युभयत्रानपेक्ष्येव किञ्चिद्भवति देवदत्ते चादित्ये च~। नेह कश्चित्परोऽनुग्हीतव्य इति प्रवर्तते~। सर्व इमे स्वभूत्यर्थ प्रवर्तन्ते~। ये तावदेते गुरुशुश्रूषवो नाम तेऽपि खभूत्यर्थं प्रवर्तन्ते \textendash\ पारलौकिकञ्च नो भविष्यति, इह च नः प्रीतो गुरुरध्या

\newpage
% ( १९ ) 

\noindent
पयिष्यतीति~। यथा य एते दासाः कर्मकरा नाम तेऽपि स्वभूत्यर्थं प्रवर्तन्ते \textendash\ भक्तञ्चैलञ्च लप्स्यामहे परिभाषाश्च न नो भविष्यन्ति~। तथा य एते शिल्पिनो नाम एतेsपि स्वभूत्यर्थं प्रवर्तन्ते \textendash\ वेतनञ्च लप्स्यामहे मित्राणि च नो भविष्यन्तीति इति~। अत्र हि आदित्या \textendash\ दिदेवदत्तान्ताः सर्वेऽपि परानुग्रहकारिणो गुरोरादेश इति न निःस्वार्थाः प्रवर्तन्ते, न परोपकारः खकीयोsयं धर्म इति वा प्रवर्तन्ते~। केवलं स्वार्थमात्रप्रवृत्ता ह्येते स्वार्थसाधनायैव प्रवर्तन्ते~। प्रवृत्तिर्हि \textendash\ आदित्यादीनां प्रवृत्तिः उभयत्र \textendash\ आदित्ये देवदत्ते च अनपेक्ष्यकिंचित् \textendash\ गुर्वाद्यादेशमन्याभिप्रायं वाsनपेक्ष्य साभिप्रायमात्रोदेश्यकैव भवति~। तमभिप्रायमेव स्पष्टमुपपादयति \textendash\ नेह कश्चित्परोग्रहीतव्य इत्यादिना~। यश्चादित्यः केवलं लोककाम एव लोकान् प्रकाशयतीति निर्विवादं सर्वैः समर्थ्यते तमेव पूर्वमुपलक्षीकृत्य भगवान् भाष्यकारो लोकरहस्यं समाजस्थितिस्थापकं निःशङ्कमुपशिक्षयति~। यदेतत्तत्वं न प्राचीनैर्नवीनैर्भविष्य \textendash\ त्कालिकैश्च कदापि विस्मर्तव्यमिति~। भगवान् भास्करोsपि यत्र न लोककामायैव प्रवर्तते, तत्र किमु मानवाः सेन्द्रियाः शालिफलाद्यु \textendash\ पभुञ्जानाः सत्तत्वविभुखाः स्वाभिप्रयेणैव व्यवह्रियमाणा ऐहिकाभिप्राया लोककामायैव स्वार्थं विहाय प्रवर्तन्त इत्याग्रहः समाजाधःपातायैवेत्यवधेयम्~॥

\begin{quote}
{\qt निमित्तेभ्यः प्रवर्तन्ते सर्व एव स्वभूतये~।\\
अभिप्रायानुरोधोऽपि स्वार्थस्यैव प्रसिद्धये~॥}
\end{quote}

इति हरिकारिकाऽपि तमेवार्थं सुस्पष्टतयाऽभिव्यनक्ति~। यः कश्चित्पराभिप्रायानुरोधेन प्रवर्तमान उपलभ्येत सोऽपि स्वार्थ \textendash\ प्रसिद्धिकाम एव तदर्थं यतते न परानुग्रहकाम्ययेति निर्विवादम्~॥ अत एव प्रामाणिका न लोकवचनेष्ववतिष्ठन्ते, किन्तु स्वतःप्रमाणेन वेदशास्त्रवचनानुरोधेनैव प्रवर्तन्ते~॥ {\qt न हि फलमनुद्देश्य मन्दोऽपि प्रवर्तते} इति न्यायमनुरुध्य वस्तुता विचार्यमाणे राजाज्ञापालनं निर्धर्मकानां नानिवार्यम्, किन्तु राजाज्ञावञ्चनमेव तेषां धर्मः स्यात्~। यतो राजाज्ञापालनं हि धर्मो {\qt नाविष्णुः पृथिवीपतिः} इति वादिनामेवास्ते, अत एव ते भारतीया राजाज्ञापालनतत्परा न परस्परं व्यत्यघ्नन्, न वा तदा राजाज्ञा \textendash\ भङ्गमकुर्वन्~। तदभावे हि {\qt सर्वैः सह भ्रातृभावेनावस्थेयं} इत्याज्ञप्तमपि किमर्थं तथाऽवस्थेयमिति प्रश्नो व्याकुलीकरोति चेतः~। यत्रापि {\qt हिंसा न कर्तव्या, तदुपलब्धौ हि राजदण्डः स्यात्} इत्यनुशासने हिंसां कृत्वाsन्येन कृता सा हिंसा न मयेत्युद्धोष्यमाणे राजदण्डपरामर्शभीरपेयात्, स्याच्च हिंसयेष्टसिद्धिरिति अहिंसामयेऽस्मिन् समये हिंसाविहारिण एव सर्वत्र विहरन्ते~। अन्यच्च {\qt अस्मद्गुरूपदेष्टया अहिंसयैव सर्वस्य समाजस्याभ्युन्नतिः स्यादिति स एव मार्गः सर्वत्र सञ्चालनीयः, ये तत्र प्रतीपा भविष्यन्ति या वा संस्था अहिंसाविरोधिन्यो भविष्यन्ति तास्ताश्च वा वयं निर्दयमुपहिंसामः} इत्युक्तिमहिंसाssचार्ययाणां भैरवभेरीरवसदृशामाकर्ण्य सर्वपक्षेभ्यः सुदूरमवस्थितानां धर्मैकशरणानां भारतीयानां कम्पन्ते चेतांसि, यतोऽद्यापि क्रूराणामविवेकिनाञ्च न परस्परं शान्तः कलह इति~॥\\

एवञ्च तदानीं धार्मिका जना आसन्, येन हिंसाचौर्यादीनां संशयोऽपि न भवेत्~। {\qt उपकार्याभावे उपकाराभावः} इति न्यायेन सम्पन्नतरेsस्मिन् देशे को वा विपन्नः स्यात् यमासाद्योपकारिणः प्रभवेयुः~। अन्यच्च \textendash\ उपकार्यो भूत्वा जीवनं ननु मरणमेवेत्यापामर \textendash\ प्रसिद्धमेतद्देशसंस्कृतिसम्बद्धं चास्ते, समाजसेवकानां स्वरूपं च पूर्वं निर्दिष्टं सर्वजनप्रसिद्धामति उपकारिण एव स्वचारितार्थ्याय उपकार्यनिर्माणे न प्रयत्नशीलास्तदानीमिति सर्वमप्यवदातम्~॥ 

\begin{center}
\textbf{\Large ऋणादानादयो व्यवहाराः \textendash\ }
\end{center}

एतादृशे सुपरिष्कृते समाजे ऋणादानादिव्यवहारो नासीदिति वक्तुमप्यशक्यम्~। तत्र ऋणदानादाने न कश्चिद्विरोधः, वृद्धिग्रहणं तु सर्वथा विनिन्दन् सूत्रकार आह \textendash\ प्रयच्छति गर्ह्य ४~। ४~। ३~। इति~। तत्र हि भाष्ग्रकारः \textendash\ {\qt अयुक्तोऽयं निर्देशः, यदसावल्पं दत्वा बहु गृह्णाति तद्गर्ह्यं~।} कथं तर्हि निर्देशः कर्तव्यः ? प्रयच्छति गर्ह्यायेति~। स तर्हि तथा निर्देशः कर्तव्यः~। न कर्तव्यः~। तादर्थ्यात्ताच्छब्द्यं भविष्यति \textendash\ गर्ह्यार्थं गर्ह्यम्~। द्विगुणं मे स्यादिति प्रयच्छति \textendash\ द्वैगुणिकः, त्रैगुणिकः अत्र वृद्धेर्वृधुषि भावः \textendash\ वार्धुषिकः इत्याह~। अत्र प्रयच्छति गर्ह्यमिति निर्देशोsयुक्तः, एवं हि अधर्मणे प्राप्नोति, उत्तमर्णे चेष्यते, तर्हि प्रयच्छति गर्ह्यायेत्याक्षिप्य न वक्तव्यमित्याह~। तादर्थ्यात् \textendash\ यत् ऋणं दीयते तद्वृध्यर्थं, सा बुद्धिरेव गर्ह्या तदर्थं दीयमानं ऋणं गर्ह्यमिति यो गर्ह्यं प्रयच्छति \textendash\ यो वृध्यर्थं ऋणं प्रयच्छति ततस्मात्प्रत्यय इत्युपपन्नमिति तदर्थः~। अनेन प्रघट्टकेन वृद्धिग्रहणमेव निषिध्यते, ऋणदानं न निषिध्यते~। वृद्धिग्रहणनिषेधात् न कर्षकाणां काचिद्वि्प्रतिपत्तिर्न वा तत्र किञ्चिद्विधानमप्यपेक्षते~॥\\

ननु वृद्धिग्रहणाभावे कः किमर्थं ऋणं दद्यादिति चेत्, यो द्विगुणार्थं प्रयच्छति स द्वैगुणिक इत्यादिना परिमितवृध्यतिरिक्ता वृद्विरेव निषिद्धेति तदभिप्रायः~। अत एव योऽल्पवृद्धिदानमङ्गीकृत्य परकीयं धनं गृहीत्वा वृध्यर्थमन्यस्मै प्रयच्छति स वार्धुषिक इत्युच्यते~। तेन हि स्वलाभार्थं बह्वी वृद्धिरेव ग्राह्या, अन्यथा तस्य लाभो न स्यादिति तस्य निन्दा प्रतीयते~। एवञ्च अनुचितवृध्यादानं निषिद्धमासीद्व्यवहारे \textendash\ इति प्रतीयते~॥\\

देशविशेषेण आचारविशेषा न्यायविशेषा विहारादिविशेषा अपि प्रतिभासन्ते~। यथा \textendash\ अरण्यान्मनुष्ये ४~। २~। १२९~। इति सूत्रे {\qt अत्यल्पमिदमुच्यते मनुष्य इति~। पथ्यध्यायन्यायविहारमनुष्यहस्तिष्विति वक्तव्यम्~। आरण्यकः पन्थाः, आरण्यकोध्यायः, आरण्यको न्यायः, आरण्यको विहारः, काच्छको मनुष्यः, काच्छकं हसितं, काच्छकं जल्पितम्, काच्छकं स्मितम्} इत्यादिना कच्छदेशोद्भवो मनुष्यो भिन्नः, तस्य हसितं जल्पितं वाऽन्यत्, आरण्यको न्यायो भिन्न इत्यादि व्यवहारविशिष्टं प्रकथ्यते~॥\\

\newpage
% ( २० ) 

भक्ष्यप्रकारा अपि सुसंस्कृता विशिष्टा वा आसन्निति तत्र जातः ४~। ३~। २५~। इति सूत्रे जातादिष्वेव घादयो यथा स्युः, इह मा भूवन् \textendash\ तत्रास्ते तत्र शेते इति~। यदि नियमः क्रियते, दार्षदाः सक्तवः \textendash\ औलृूखलो यावक इति न सिध्यति~। संस्कृत्तमित्येवं भविष्यति~। भवेत्सिद्धं \textendash\ दार्षदाः सक्तव इति~। इदं तु न सिध्यति \textendash\ औलूखलो यावक इति~। संस्कृतं नाम तद्भवति यत्तत एवापकृष्याभ्यवह्रियते~। न च यावक उलूखलादेत्रापकृष्याभ्यवह्रियते~। अवश्यं रन्धनादीनि प्रतीक्ष्याणि इत्यनेन प्रसिध्यति~। जातादिष्वेव घादयो भवन्तीति नियमे दार्षदादीनि न सिध्यन्ति~। ननु तत्र {\qt संस्कृतम्} इत्यनेनैव प्रत्ययसिद्धिः, तेन हि {\qt दार्षदाः} इत्येव सिध्येत्~। यतः संस्कृतं तद्भवति तत एव भक्ष्यते, यथा दृषदि पिष्टाः सक्तवो भक्षणार्हा भवन्ति औलूखलो यावकश्च न तादृश इति उलूखलेन संस्कृत इति न भवतीति तदर्थः~। प्रघट्टकेनानेन भक्ष्याणां विविधाः प्रकाराः प्रचलिता इति सुस्पष्टमेव~॥\\

नदीतरादीनां प्रसिद्धिस्तदाऽऽसीदिति घसंज्ञासूत्रे {\qt घसंज्ञायां नदीतरे प्रतिषेधो वक्तव्यः~। नद्यास्तरो नदीतरः} इत्यु. दाहरणेन ज्ञायते~॥\\

भृत्यादिना कर्मकरा अपि तदानीमासन्, तद्भृत्यवलोकनेन च वस्तूनां मूल्यमपि निर्धारयितुं शक्यम्~। स्वरितञितः कर्तृभिप्राये क्रियाफले १~। ३~। ७२~। इति सूत्रे याजका यजन्ति गा लप्स्यामह इति, कर्मकराः कुर्वन्ति पादिकमहर्लप्स्यामह इति इत्यादिना गवादिलाभार्थं याजका यजन्ति, भूतिलाभार्थ कर्मकराः कुर्वन्तीति ज्ञायते~। भृत्यस्य पादिकं वेतनमासीत्, पादिकश्च पणस्य ताम्रमुद्रायाश्चतुर्थोंऽशः \textendash\ इति तावती भूतिरासीत्~। तदा पणपादेन निर्वाहः सुसम्बद्ध इत्यप्याक्षिप्यते~। अन्यथा तादृशेनाल्पमूल्येन कर्मकरो न प्रवर्ततेति~॥

\begin{center}
\textbf{\Large राजकीयम् \textendash\ }
\end{center}

राजकीयप्रबन्धेऽपि \textendash\ कारनाम्नि च प्राचां हलादौ ६~। ३~। १०~। इत्यत्र अविकटे उरणो दातव्यः \textendash\ अविकटोरणः इत्यनेन अवीनां समुदायो निर्वाहार्थं यैः पाल्यते तत्र नियतसंख्याके समुदाये राजकीयदेयमुरणो दातव्य इति नियमो दृश्यते~। चारुरूपश्चायं नियमो येन अविसमुदायपालानां अविदाने न तादृशी हानिर्यादृशी द्रव्यदाने स्यात्~। ऊर्णार्थमविव्यवसायश्च सुचारुरूपेण परिवर्तित आसीदित्यवगम्यते~॥ राज्ञां युद्धादिवर्णनं बहुत्रायातमपि पङ्क्तिविंशति० ५~। १~। ५९~। इति सूत्रे इह कदाचिद्गुणः प्राधान्येन विवक्षितो भवति~। तद्यथा \textendash\ पञ्चोsडुपशतानि तीर्णानि, पञ्चवध्रीशतानि तीर्णानि, अश्वैर्युद्धम्, असिभिर्युद्धम्~। न चासयो युध्यन्ते, असिगुणाः पुरुषा युध्यन्ते, गुणस्तु खलु प्राधान्येन विवक्षितो भवति इत्यादिवर्णनं विशेषतामाधत्ते~॥ मद्रात्परिवापणे ५~। ४~। ६७~। इत्यत्र {\qt क्षेमे सुभिक्षे कृतसञ्चयानि पुराणि राज्ञां विनयन्ति कोपम्} इत्युदाहरणेनापि बलवतां राज्ञां पारस्परिके प्रहरणेऽपि तत्र प्रजानां कश्चिदुपद्रवो नासीत्, रणाङ्गणे युद्धम्, कुशलिनश्च प्रजा नगरे इति राज्ञां युद्धप्रसङ्गे सर्वेऽपि संव्यवहरन्ते स्मेति~॥ {\qt गर्गाः शतन्दण्ड्यन्ताम्~। अर्थिनश्च राजानो हिरण्येन भवन्ति न च प्रत्येकं दण्डयन्ति} इत्युदाहरणं बहुषु स्थलेषूपलभ्यमानं राज्ञामलोलुपत्वं धर्मानुबन्धित्वञ्च द्योतयति~। हिरण्येनार्थिनोsपि राजानो न पदवाक्यप्रमाणानतिक्रामन्ते, न स्वार्थानुसारेण प्रत्येकं दण्डयन्तीति भावः~॥ {\qt अमात्यानां राज्ञा सह समवाये पारतन्त्र्यं व्यवाये स्वातन्त्र्यम्} इति कारके १~। ४~। २३~। इति सूत्रोपात्तमुदाहरणं राज्ञाममात्यानाश्चैकमत्यमेव प्रतिपादयति~॥\\

ये त्वेते राजकर्मिणो मनुष्यास्तेषां कश्चित्कंचिदाह \textendash\ कटं कुर्विति~। स आह \textendash\ नाहं कटं करिष्यामि घटो मयाऽऽहृत इति~। तस्य क्रियामात्रमीप्सितम्~। इति कर्तुरीप्सिततमं कर्म १~। ४~। ४९~। सूत्रे समुदाहृतं राजकीयकर्मचारिणां संव्यवहारं बोधयति~। राजकीयकर्मकराश्च क्रियामात्राभिनिविष्ठा न फलसिद्धौ बद्धादरा {\qt यङ्भविष्यति तद्भवतु} इत्येतद्भावभाविता भवन्ति, घटो मयाऽऽहृत इति मया काच्चित्क्रिया कृता कार्यसिद्धिर्भवतु वा मा वा, नाहं कटङ्करिष्यामीति वदन्ति~। लोकराज्ये तु खाधीनशासना अमात्यादयोsपि स्वेच्छया व्यवहरन्तः क्रियामात्रं प्रतिपादयन्ति~। दैववशात्सम्पन्ने कार्ये मयैतत्सम्पादितम्, कार्यविप्रतिपत्तौ तु भ्रष्टाचाराः सर्वेऽपीत्यधिकारन्यासेन समर्थयन्ति चात्मानम्~॥ राजकीयकर्मकरैः कथं वर्तनीयमित्येतद्विषयेऽपि {\qt जहदप्यसौ स्वार्थं नात्यन्ताय जहाति, यः परार्थविरोधी स्वार्थस्तं जहाति~। तद्यथा \textendash\ तक्षा राजकर्मणि प्रवर्तमानः स्वं तक्षकर्म जहाति, न तु हिक्कितश्वसितकण्डूयितानि} इति समर्थसूत्रे प्रतिपाद्यते~। राजकर्मणि प्रवर्तमानाः पुरुषाः स्वीयेन कर्मणा राजकर्मण उपरोधं वारयेयुरित्येव राज्ञोऽभिप्रायः~। यानि च स्वोन्नतिसाधकानि कर्माणि राजकर्मणः प्रतिरोधं न कुर्वारन् तत्सम्पादने न काचिद्धानिर्नान्यायश्चेति~॥ राजकीयोद्भासविषयप्रकाशकानि भाष्ये बहून्युदाहरणानि प्रदर्शनीयान्यपि स्थानाभावादिङ्मात्रप्रदर्शनेनैव सन्तोष्टव्यम्~॥

\begin{center}
\textbf{\Large स्त्रीसम्मानः \textendash\ }
\end{center}

स्त्रीसमुदाचारः कथमासीत्तदानीं सोऽप्यवलोकनीयः~। एको गोत्रे ४~। १~। ९३~। इत्यत्र {\qt काशकृत्स्ना प्रोक्ता मीमांसा काशकृत्स्नी तामधीते काशकृत्स्ना ब्राह्मणी} इति; अनुपसर्जनात् ४~। १~। १४~। सूत्रे {\qt काशकृत्स्नीमधीते काशकृत्स्ना ब्राह्मणी अत्र प्राप्नोति~। नैष दोषः~। अध्येत्र्यामभिधेयायामण ईकारेण भवितव्यम्} इत्याद्युदाहरणैः स्त्रीणां शिक्षैव नासीत् ताश्च केवलं भोगसाधनीभूता अस्वतन्त्रा इत्याद्याक्षेपा निर्मूलाः~। तदा गृहिण्यो मीमांसादिशास्त्रेष्वपि ग्रन्थकर्त्र्य आसन्, न तु गृहं गृहमटमाना भृत्यकर्मण्येव नि्युक्ताः~। अत एव \textendash\ अणिञोरनार्षयोः० ४~। १~। ७८~। इति सूत्रै {\qt औदमेध्यायाश्छात्राः} इत्युदाहरणं संगच्छते~॥ विप्रतिषेध \textendash\ सूत्रे{\qt अस्ति प्राधान्ये वर्तते, तद्यथा \textendash\ परमियं ब्राह्मण्यस्मिन् कुद्म्बे, प्रधानमिति गम्यते} इत्युदाहरणेन कुटुम्बप्रधान्यमपि तासा \textendash\

\newpage
% ( २१ ) 

\noindent
मासीदिति निर्विवादं वक्तुं शक्यते~॥ करणसंज्ञासूत्रे {\qt अभिरूपायोदकमानेयमभिरूपाय कन्या देया इति~। न चानभिरूपे प्रवृत्तिरस्ति तत्र अभिरूपतमाय इति गम्यते} इति; नमः स्वस्ति० २~। ३~। १६~। इति सूत्रे अलमिति पर्याप्त्यर्थग्रहणं कर्तव्यम्~। इह माभूत् \textendash\ अलं कुरुते कन्याम् इति चोदाहरणाभ्यां स्त्रीदाक्षिण्यं सुरूपमिव प्रतीयते~॥

\begin{center}
\textbf{\Large प्रमाणव्यवहारः \textendash\ }
\end{center}

मानविषये \textendash\ 

\begin{quote}
{\qt ऊर्ध्वमानं किलोन्मानं परिमाणं तु सर्वतः~।\\
आयामस्तु प्रमाणं स्यात् संख्या बाह्या तु सर्वतः~॥

भेदमात्रं ब्रवीत्येषा नैषा मानं कुतश्चन~।}
\end{quote}

इति आर्हादगोपुच्छ० ५~। १~। १९~। इति सूत्रेऽन्यत्र च विस्तरश उक्तमिति प्रमाणप्रमेयादिकल्पना सूक्ष्मतरदर्शिनां भूषण \textendash\ भूतैवासीदिति गम्यते~। तेन प्रामाणिक एव व्यवहारः संप्रवृत्त इति सिद्धम्~॥\\

व्यवहारेऽपि कथं दक्षा आसन्निति {\qt लोकतो व्यवहारं दृष्ट्वा गुणस्य निर्ज्ञानम्~। तद्यथा \textendash\ पतुरयं ब्राह्मण इत्युच्यते यो लघु \textendash\ नोपायेनार्थान् साधयति~। पटुकल्पोऽयमित्युच्यते यो न तथा साधयति} इति \textendash\ ईषदसमाप्तौ० ५~। ३~। ६७~। सूत्रे; वृषलरूपोऽयं, अप्ययं पलाण्डुना सुरां पिबेत्~। चोररूपोऽयं, अप्ययमक्ष्णोरञ्जनं हरेत्~। दस्युरूपोऽयं, अप्ययं धावतो लोहितं पिबेत् इति प्रशंसायां रूपम् ५~। ३~। ६६ सूत्रे च सम्प्रदर्शितव्यवहारेण ज्ञातुं शक्यम्~॥ गुणिनां दुष्टानाञ्च स्वभावपरिज्ञाने सकुशलास्तदानींतना अनेन प्रघट्टकेन ज्ञायन्ते~॥ व्यवहारपरिज्ञानेऽन्यान्यप्युदाहरणानि यथा \textendash\ {\qt तत्सामीप्यात्तु व्यञ्जनमपि तद्रुणमुपलभ्यते~। यथा \textendash\ द्वयो रक्तयोर्वस्त्रयोर्मध्ये शुक्लं वस्त्रं तद्गुणमुपलभ्यते~। बदरपिटके रिक्तको लोहकंसस्तद्भुण उपलभ्यते} इति नीचैरनुदआत्तः १~। २~। ३०~। इति सूत्रे {\qt कटं करोति भीष्ममुदारं शोभनं दर्शनीयम्} इति अनभिहितस्सूत्रे प्रतिपादितानि समाजस्य याथार्थ्यं गमयन्ति~। बदरपिटके \textendash\ बदरपेटिकायां \textendash\ प्रेषणार्थ कृतायां रिक्तकः \textendash\ घर्षणेनोज्वलीकृतः लोहकंसः \textendash\ आदर्शस्तद्गुण उपलभ्यत इति तदर्थः~॥ पेटिकाश्चारुरूपाः कृत्वाऽऽदर्शसंयुताश्च विधाय तासु विक्रयार्थं प्रेषणीयानि बदरादीनि फलानि निधाय ताः पेटिका बहिः प्रस्थाप्यन्ते इत्यनेन व्यवहारकरणे कौशल्यमेधामुपजनयन्ति~॥

\begin{center}
\textbf{\Large वाहनानि \textendash\ }
\end{center}

वाहनादिप्रकारा अपि निपातैर्यद्यदिहन्त० ८~। १~। ३०~। इति सूत्रे यत्कूजति शकरट, यती कूजति शकटी, यन् रथः कूजति इति; यद्धितु परं० ८~। १~। ५६~। इति सूत्रे {\qt आरोहन्ति द्वस्तिनं मनुष्या आरोहयति हस्ती स्थलं मनुष्यानः; न वेतिविभाषासूत्रे }अभ्यवहारयति सैन्धवान्, विकारयति सैन्धवैः इत्यादिभिः हस्त्यश्चगवादीनां वाहनेषु नियतमुपयोग आसीदिति प्रत्यभिज्ञानम्~। अहणकशास्रे {\qt तमेवाध्वानं कश्चिदाशु गच्छति कश्चिच्चिरेण कश्चिच्चिरतरेण~। रथिक आशुगच्छति आश्विकश्चिरेण पदातिश्चिरतरेण} इत्युदाहरणं वाहनक्रियायामुपयुज्यमानो रथ एव तत्र प्रधानमासीदिति गम्यते~॥\\

मार्गेषु गमनागमनप्रकारोऽपि हेतुहेतुर्मतोर्लिङ् ३~। ३~। १५६~। इति सूत्रे सुस्पष्टं प्रतिपादितः~। तत्र हि {\qt दक्षिणेन चेद्यायात् न शकटं पर्याभवेत्~।} दक्षिणेन चेद्यास्यते न शकटं पर्याभविष्यति इत्युदाहरणं {\qt यदि देवदत्तो मार्गं दक्षिणभागेन यास्यति तदा शकटाद्रक्षितः स्यात्} इत्यर्थकमुक्तम्~। अनेन वाहनानां पादचारिणामपि केचन नियमा आसन्निति ज्ञायते~॥\\

योजनं गच्छति ५~। १~। ७४~। इति सूत्रे {\qt क्रोशशतं गच्छति क्रौशशतिकः, योजनशतं गच्छति यौजनशतिकः, वारिपथेन गच्छति वारिपथिकः इत्यनेन क्रोशशतगमनसमर्थानि योजनशतगमनसाराणि वाहनानि प्रयुज्यन्ते स्म~॥} विचित्राणि अनैक्प्रकाराणि बाहनान्यपि तस्येदम् ४~। ३~। १२०~। इति सूत्रे {\qt आश्वरथम्, औष्ट्ररथम्, गार्दभरथम्} इत्युदाहरणैः प्रदर्श्यन्ते~। जलवाहनान्यपि नापरिचितानीति वारिपथिकेन ज्ञायते~॥ क्रीडोऽनुसं० १~। ३~। २१~। इत्यत्र {\qt संक्रीडन्ति शकटानि} इत्यनेन वाहनानां स्पर्धा प्रतीयते~। तथा तत्रैव {\qt पैतृकमश्वा अनुहरन्ते, मातृकं गावोsनुहरन्ते} इत्येतत् वाह्यानां स्वभावान् द्योतयति~॥ ग्राम्यपशुसं० १~। २~। ७३~। सूत्रे {\qt गाव उत्कालितपुंस्का वाहाय च विक्रयाय च~। कः पुनरहति अग्राम्याणां पुंस उत्कालयितुं ये ग्रहीतुमशक्याः, कुत एव वाहाय च विक्रयाय च~। कः पुनरर्हति अपशूनां पुंस उत्कालयितुं येऽशक्या वाहाय च विक्रयाय च} इति वाहविषये विशिष्टं व्यवहारं प्रतिपादयति~। प्रत्यभिज्ञायते चानेन येsपशवः, ये वाऽऽरण्यका वाद्वायानुपयुज्यमानाः, तेषु एतादृशं व्यवहारमनाचरन्तः सम्यक् दर्शिन आसन्निति~॥ 

\begin{center}
\textbf{\Large शब्दव्यवहारः \textendash\ }
\end{center}

शंब्दव्यवहारेऽपि कथं दक्षाश्चैत इत्यवलोकनेन प्रीणन्ति सहृदया इति दिङ्र्मात्रमुदाह्रियते~। ह्रस्वे ५~। ३~। ८६~। इत्यत्र {\qt अल्पं घृतं \textendash\ अल्पं तैलमित्युच्यते, न कश्चिदाह \textendash\ ह्रस्वं घृतं ह्रस्वं तैलमिति~। ह्रस्वः पटः ह्रस्वः शाटक इत्युच्यते, न कश्चिदाह \textendash\ अल्पः पटः अल्पः शाटक इति} इति; समाने रक्ते वर्णं गौर्लोहित इति भवति अश्वः शोण इति~। समाने च काले वर्णं गौः कृष्ण इति भवति अश्वो हेम इति, समाने च शुक्ले वर्णे र्गौः श्वेत इति भवति अश्वः क \textendash\ इति पितामात्रः १~। २~। ७०~। इत्यत्र; बुद्धौ कृत्वा सर्वाश्चेष्टाः कर्ता धीरस्तत्वन्नीतिः शब्देनार्थान्वाच्यान्दृष्ट्वा बुद्धौ कुर्यात्पौर्वापर्यम्~॥ इति बुद्धिविषयमेव शब्दानां

\newpage
% ( २२ ) 

\noindent
पौर्वापर्यम्~। य एष मनुष्यः प्रेक्षापूर्वकारी भवति स पश्यति अस्मिन्नर्थेऽयं शब्दः प्रयोक्तव्यः, अस्मिंस्तावच्छब्देsयं तावद्वर्णस्ततोऽयं व्यवहारः इति संहितासंज्ञासूत्रे प्रयुज्यमाना व्यवहाराश्च शब्दप्रावीण्यं गमयन्ति~। अत एवैते शब्दप्रमाणका इत्यभ्युपगम्यते~। भाषाशब्दानां संव्यवहाराश्च देशभेदेन भिन्नाः संस्कृतेन साम्यमुपगच्छन्तः {\qt गावी गोणी गोता गोपोतलिका} इत्येवमादयस्तदानीं राष्ट्रभाषात्वेन गीर्वाणवाण्याः सुप्रचार आसीदित्यवगमयन्ति~॥\\

परिमितमत्रस्थानम्, महांश्चायं विषयः~। एतावता प्रबन्धेन दिङ्मात्रदर्शनमेव यथा कश्चिद्भवेन्न वेति संशयितचित्ताः साहसेनैव प्रवृत्ताः स्वबुद्धिबलानुरूपं यत्किञ्चित्स्वरूपं विद्वज्जनमनोविनोदार्थमेतदुपस्थापयामः भाष्यप्रदत्तोदाहरणेभ्यस्तत्कालच्चित्रं यथावदुपपादयितुं शक्यं युक्तश्चेत्यैतावान् समाश्वासो यदि तत्र भवतां पारदृश्वनां मनांसि प्रमोदयेत् तदा सफलपरिश्रमा वयमिति सम्भावयामः~॥ 

\begin{center}
\textbf{\Large स्वभावपरिचयः \textendash\ }
\end{center}

मनुष्यस्वभावपरिचायकान्युदाहरणान्यप्यवश्यमवलोकनी \textendash\ यानि, तानि च सुबहून्यप्यत्र कतिपयान्येवोदाह्रियन्ते~। शीतोष्णाभ्यां० ५~। २~। ७२ इति सूत्रे {\qt किं यः शीतं करोति स शीतकः, यो वा उष्णं करोति स उष्णकः ? किञ्चातः ? तुषारे आदित्ये च प्राप्नोति~। एवन्तर्हि शीतमिव शीतम्, उष्णमिवोष्णम्~। य आशु कर्तव्यानर्थाश्चिरेण करोति स उच्यते शीतक इति~। यः पुनराशु कर्तव्यानर्थानाश्चैव करोति स उच्यते \textendash\ उष्णक इति~। इति; अयः शूल० ५~। २~। ७६ सूत्रे }किं योsयःशूलेनान्विच्छति स आयःशूलिकः, किञ्चातः ? शिवभागवते प्राप्नोति~। एवं तर्हि अयःशूलमिव \textendash\ अयःशूलम्~। यो मृदुनोपायेनान्वेष्टव्यानर्थान् रभसेनान्विच्छति स उच्यते \textendash\ आयःशूलिकः इति चोदाहरणे उष्णकं शीतकं आयःशूलिकञ्च जनमुपस्थापयतः~। तत्र प्रसङ्गात् शिवस्य भगवतो भक्ता आयसेन त्रिशूलेन भिक्षामन्विच्छन्तीत्यपि प्रदर्शितम्~। केचिच्चोष्णका वर्तन्ते केचिच्च मन्दा अपर आततायिन इति दर्शनादेते सर्वेऽपि प्रकारा निन्दिता लोके, तद्गुणविशिष्टाश्च ये विहरन्ति तेऽपि वाच्या भवन्तीत्याशयः~॥\\

उपमानं शब्दार्थ० ६~। २~। ८०~। इत्यत्र {\qt पुष्पहारी फलहारो वृकवञ्ची वृकप्रेक्षी कोकिलाभिव्याहारी गर्दभोच्चारी साध्वध्यायी विलम्बिताध्यायी} इत्याद्युदाहरणानि नानाविधान् मनुष्यस्वभावान् परिचाययन्ति~॥\\

दूराद्धूते च ८~। २~। ८४ इत्यत्र {\qt दूरशब्दश्चायमनवस्थितपदार्थकः}~। तदेव हि किश्चित्प्रति दूरं किञ्चित्प्रति अन्तिकं भवति.~। एवं हि कश्चित्कश्चिदाहृ \textendash\ एष पार्श्वेतः करकस्तमानयेति~। स आह \textendash\ उत्थाय गृहाण दूरं न शक्ष्यामीति~। अपर आह \textendash\ दूरं मथुरायाः पाटलिपुत्रमिति~। स आहृ \textendash\ न दूरमन्तिकमिति~। एवमेष दूरशब्दोऽनवस्थितपदार्थकः इति; कृत्यचः ८~। ४~। २९~। इति सूत्रे {\qt निर्विण्णोऽहमनेन वासेन} इति; अट्कुप्वाङ्० ८~। ४~। २~। इति सूत्रे {\qt गर्गैः सह न भोक्तव्यमिति प्रत्येकञ्च न भुज्यते समुदितैश्च इत्येतैश्च }गर्गैः सह न भोक्तव्यम् इति सामुदायिक आदेश इत्याद्याः स्वभावविशेष्राः प्रतीयन्ते~। भिन्नरुचिहिं लोक इतिन्यायेन यदेवै \textendash\ कस्य समीपं तदेवान्यस्य दूरमिति स्वभाववैचित्र्यं प्रदर्शयति~॥ {\qt यदुदुम्बरवर्णानां} घटीनां मण्डलं महत्, वेदान्नो वैदिकाः शब्दाः, खेदात्स्त्रीषु प्रवृत्तिर्भवति समानश्च खेदविगमो गम्यायाञ्चागम्यायाञ्च इत्यादीनि प्रसिद्धतराणि स्वभाववैचित्र्यदर्शन्युदाहरणानि लौकिकानां स्वभावपरिचये वैदुष्यातिशयं द्योतयन्ति~॥ समर्थसूत्रे {\qt त्वं तिष्ठ शङ्कुलया खण्डो धावति मुसलेन, किं त्वं करिष्यसि शङ्कुलया खण्डो विष्णुमित्र उपलेन} एते वाक्ये \textendash\ {\qt त्वं तिष्ठ शङ्खछया न प्रयोजनं मुसलेन कृतः खण्डो धावति~। हे विष्णुमित्र त्वं शङ्कुलया किं करिष्यसि, उपलेनैव कृतः खण्डः} इत्यर्थके वक्तुः स्वभाववैचित्र्यं प्रदर्शयतः,स्वल्पे वस्तुनि दुर्बोधोक्तिम्प्रदर्श्य स्वभाववक्रतामुपपादयतश्च~॥\\

अनेकमन्यपदार्थ २~। २~। २४~। इति सूत्रे {\qt इदं तावदयं प्रष्टव्यः \textendash\ अथेह देवदत्तस्य भ्राता इति कः षष्ठ्यर्थ इति~। तत्रैत्स्यात् \textendash\ एकस्मात्प्रादुर्भाव इति~। एतञच वार्तम्~। तद्यथा \textendash\ सार्थिकानामेकप्रतिश्रये उषितानां प्रातरुत्थाय प्रतिष्ठमानानां न कश्चित्परस्परमभिसम्बन्धो भवति~। एवंजातीयकं भ्रातृत्वं नाम~। अत्र चेद्युक्तः षष्ठर्थौ दृश्यते, इहापि युक्तो दृश्यताम्} इत्यनेन प्रघट्टकेन भगवान् भाष्यकारः सार्थिकेष्वपि भातृबद्यवहारमुपदिशन् तदानीन्तनानां स्वभाववैशिष्ट्यमुपपादयति~॥\\

अत्र भाष्योदाहरणान्येव प्रायोऽर्थसम्पत्तये समाहृतानि~। भाष्योदाहरणेषु च विद्याः, कलाः, सामाजिकव्यवहाराः, राजकीय \textendash\ व्यवहाराः, धर्मव्यवहाराः, स्त्रीणामधिकाराः, प्रमाणप्रमेयादिव्यवहाराः, वाहनप्रकाराः, आरण्यकन्यायाः, शब्दव्यवहारा इत्यादयोऽनेके विषयाः संगृहीता दृश्यन्ते~। तत्र कतिपयान्युदाहरणान्यासाद्य प्रतीतमर्थ विद्यार्थिनामुपयोगायात्रोपनिबद्धं सुचिरमालोक्य सफलयन्तु तत्रभवन्तो विद्याव्यसनिनः~॥ एतावांश्चार्थ उपलभ्यते \textendash\ तदानींतना जनाः सर्वासु विद्यासु कलासु चाभिविनाताः सामाजिकव्यवहारेषु निष्णाता अन्यायप्रतीकारबुद्धयः प्रमाणादीन्निर्णीय तदनुसारेण संविधानमाचरन्तः शब्दव्यवदारकुशला अधर्मभीरवः शराः प्राणपणेनापि सत्यसंरक्षणदक्षाः परोपकारिणश्चातुर्वर्ण्यव्यवस्थामर्यादामपरित्यजन्तोsपि परस्परं वैरभावमुत्सृज्य भ्रातृभावेन व्यवहरणशीलाः स्त्रीसम्मानसंरक्षणदक्षा वनधान्यपश्चादिभिरुपेताः सुखिनो धार्मिकाश्चासन्निति~। तथाsचिरोद्गतस्वातन्त्र्यरसा प्लाविता इदानीन्तना भारतीया वुद्धिवैभवशालिनः प्राज्यफलमूलोदकाः सम्पन्नतरगोरसाः सुविद्याश्चातुर्वर्ण्यव्यवस्थापालनदक्षा धर्मनिष्ठा आचन्द्रार्कं सुखिनो भूयासुरिति सम्प्रणम्य श्रीविश्वेश्वरचरणौ सम्प्रार्थयामः~॥

\begin{center}
\textbf{\Large एतत् संस्करणविषये \textendash\ }
\end{center}

एतावत्सर्वं यथावद्वोधयितुं परिमितपत्रोऽयं निबन्धो न समर्थ इति भाष्यावलोकनमेव वरमिति सम्प्रार्थ्य एतत्संस्करणविषये

\newpage
% ( २३ ) 

\noindent
द्वित्राः शब्दा विनिवेद्यन्ते \textendash\ म. म. प. शिवदत्तदाधिमथा भाष्यस्य विभज्य प्रकाशनमारभन्~। शिष्यशिक्षायै परमोपयोगोऽनेनैवंविधेन व्याख्यानेनेति भृश समर्थयामः~। अत एव स्वीयसंस्करणप्रसङ्गे राजलक्ष्मीकारा गुरुप्रसादशास्त्रिणोsपीमां व्यवस्थामङ्गीकृत्यैव स्वाभिप्रायमाविश्चक्रुः~। प्रयत्नश्चायं शिवदत्तपण्डितानां तृतीयाध्यायान्त एवावसितः~। एतादृशविभागश्च शिवदत्तैश्छायाकृत्कृत एवोपलब्ध इति द्वितीयाह्निके हयवरट् सूत्रे १४० तमे पृष्ठे {\qt अनुवर्तते विभाषा शरोचि} इति श्लोकवार्तिकावतरणे ११ एकादश संख्याकछायाटिप्पनीतो बोध्यते~। तत्र हि समाधानसाधकसिद्धान्तिवार्तिकम् इत्यवतरणम्~। तत्रत्यं {\qt सिद्धान्तीति} इति प्रतीकमुपादाय छायाप्रवृत्तिः~। {\qt सिद्धान्तीत्यस्य \textendash\ सिद्धान्तिभाष्यसहायवार्तिकमित्यर्थः} इत्युक्त्या छायाकारेण कृतोsयं विभागः सावतरणः शिवदत्तैरुपलब्धः शिष्योपकाराय यथावदुपन्यस्त इति ते धन्यवादार्हाः~। छायाकृत्कृतत्वाद्विभागश्चायं सानुबन्धः प्रामाणिकश्चेति न संशयलेशोऽपि~। तृतीयाध्यायान्ते परमास्पदमुपगताः शिवदत्तचरणा इति अवसितोsयमुपक्रमो निर्णयसागराधिपतिप्रोत्साहनेन गुरुचरणमनुसन्धाय काशीभूषणानां सुहृदां विद्वच्चरणानां साहाय्यमास्थाय चास्माभिरुपक्रान्तश्चतुर्थपञ्चमषष्ठाध्यायान् विद्वज्जनकरकमलगतान् व्यधत्तेति नाविदितम्~॥ जगन्मण्डूकं ग्रसितुं व्यात्तानने समरव्याले अवसितप्राये च सर्वव्यवहारे मुद्रणव्यवहारोऽपि शिथिल इव सञ्जातः परमसौभाग्येन स्वतन्त्रेऽस्मिन् भारते कथञ्चिदिदानीं समाश्वासमाधक्ते~। अत्रावसरे च दुर्लभानि नवाह्निकपुस्तकानि समालोक्य छात्रैः पुनः पुनः सम्प्रेर्यमाणा निर्णयसागरमन्त्रिणो नवाह्निकप्रकाशने दत्तचित्ता मां व्यजिज्ञपन्~। भाष्याध्ययनसमये प्रदीपोद्योतच्छायानां टीकाग्रन्थानां गुरूपदिष्टमभिप्रायमाकलयन्तं शिवदत्तरघुनाथपण्डितकृतास्तत्र तत्र समुपलभ्यमाना \textendash\ श्चिन्त्यत्वोक्तय एवास्य संस्कारकरणे मां समयुञ्जीरन्~। शिवदत्तपण्डितैः स्वाभिप्रायाविष्करणदक्षैरनुमितानि वार्तिकानि \textendash\ प्रत्याख्याता उपसंगृहीताश्च भाष्यभागा \textendash\ यथा मनांसि अतुदन् न तथा रघुनाथपण्डिन्ति~। तानां चिन्त्यत्वोक्तयः~। ता हि उद्द्योतच्छाये एव सम्बध्नवार्तिकभाष्याभागानामनुमानन्तु भाष्यमेव संकीर्णं विदधीतेति प्रामाणिकपुस्तकानुगृहीतं शुद्धं समुचितपाठप्रमेदपूरितं चिन्त्यत्वोक्तिनिराकरणसमुह्लासितं नवाह्निकं शिष्याणामुपकारार्थं विद्याभिवृद्धये निर्णयसागरसश्चालकाः सोपस्कारं प्रकाशयन्ति~॥ अत्रादर्शपुस्तकानां साहाय्यं षष्ठाध्यायप्रस्तावोपनिबद्धमेवास्ते इति तद्विवरणं द्विरुक्तं स्यात्~। सुगृहीतनामधेयानं नागपुरवासिनां म. म. कृष्णशास्त्रि \textendash\ घुले महोदयानां श्रीमत्सदाशिवभट्टपरिशीलितं प्रदीपोद्द्योतसहितं पुस्तकरत्नमेवात्रावतारितमिति नात्युक्तिः~। पुण्यपुरनिवासिनां विद्याव्यसनिनां बेलवलकरोपाह्वानां श्रीपादशर्मणां साहाय्येन प्राचीनतरं अत्र प. संज्ञया व्यवहृतं भाण्डारकरसंस्थात उपलब्ध \textendash\ म्पुस्तकमधिकं ततः समुपयोजितम्~। पुस्तकश्चैतत्सम्पूर्ण भाष्यस्यैव~। नात्र प्रदीपोद्द्योतौ~। पाठभेदाश्च सुपुष्कलाः~। अस्य द्वितीयाह्निकान्ते {\qt एतद्भाष्यपुस्तकं यतिश्रीः ७ अनन्तसरस्वती एतेषाम् स्वस्ति~। संवत् १५४५ वर्षे श्रावणशुदि १५ बुधे श्रीवृद्धनगरे तृतीयाह्निकादौ बहुत्र ओं नमो गणपतये} इत्याद्युल्लखो दरीदृश्यते~॥\\

काशीस्थराजकीयसंस्कृतपाठशालाध्यापकाः श्रीरघुनाथपण्डिता द्वितीयाद्यावृत्तिमुद्रणावसरे श्रीशिवदत्तसाहाय्यमारचन्~। साहाय्यविधानेन महदुपकृतमिति नात्र संशयलेशः~। छायामुद्द्योतच्च यत्र तत्र चिन्त्यत्वेन समुपबृंहणशीलांस्तान् {\qt नवाह्निके प्रायः प्रामादिका आशयानवबोधमूलिकास्ताश्चिन्त्यत्वोक्तयः} इति भृशं विज्ञापयामः~।\\

यथा \textendash\ अणुदित्सवर्णस्येति सूत्रे ह्वस्वसम्प्रत्ययादिति चेदिति वार्तिकविवरणे {\qt उच्चार्यमाणः शब्दः सम्प्रत्यायको भवति, न सम्प्रतीयमानः} इति भाष्यव्याख्याने रहसि पुस्तकमीक्षमाणस्यातिसूक्ष्मोच्चारणमस्तीत्यनन्तरं पक्षान्तरमाहोद्योते \textendash\ ५५०तमे पृष्ठे २१ पंक्तौ \textendash\ {\qt यद्वा लिपेरेव चेष्टादिवत्सङ्केतेन बोधकत्वम्~। अत एव लिपौ शब्दत्वभ्रमो बालानाम्~। यद्धा लिपावनादेः शब्दतादात्म्याध्यासाद्घोधकत्वम्} इत्युक्तम्~। तत्र {\qt बोधकत्व इति पदं विब्रियमाणै रघुनाथपण्डितैः }इदं चिन्त्यम्~। लिपौ शब्दाध्यासेऽप्युच्चार्यत्वासम्भवात् इत्युक्तम्~। वस्तुतस्तु तत्रोच्चारणाभावेऽपि बोध उपपादितः {\qt लिपौ शब्दत्वबाधज्ञानवतां पण्डितानामपि अन्तःकरणादावात्मत्वप्रत्यये भ्रमत्वञ्जानतामनादिसिद्धारोपेणैव व्यवहारवद्बोधोsपि} इत्यनेनोद्द्योतकृता~। अत्रापि रघुनाथाश्चिन्त्यत्वं वदन्ति~। तादृशानि स्थलानि क्वचिद्विवेचितानि, कानिचिच्च निराकृतानीत्युदारान्तःकरणैः क्षन्तव्यमिति विज्ञापयामः~॥\\

राजलक्ष्मीकारा गुरुप्रसादास्तु ततोऽप्यधिकतराः कैयटनागेशौ स्वशिष्यसम्मितावनुमिन्वते~। ते हि अथ शब्दानुशासनम् इत्यत्रैव प्रथमग्रासे मक्षिकापातमिव {\qt परे तु शास्त्रारम्भप्रयोजनसूचकं भगवतः कात्यायनस्य वार्तिकमेतत्~। भाष्यकारस्यैतद्व्याकरणप्रयोजनप्रदर्शनवाक्यं इति वदन्तौ कैयटनागेशौ तु भ्रान्तावेवेतीत्याहुः} इति वदन्तो भाष्यकारमपि अधिक्षिपन्ति~। भाष्यकृता {\qt सिद्धे शब्दार्थसम्बन्धे \textendash\ } इति व्याख्याने {\qt माङ्गलिक आचार्यो महतः शास्त्रौघस्य मङ्गलार्थं सिद्धशब्दमादितः प्रयुङ्क्ते} इत्यादिवदता सिद्धे शब्दार्थसम्बन्ध इत्येव वार्तिकारम्भः प्रतिपादितः~। न चैते भाष्यकारं गणयन्ति, का कथा प्रदीपोद्द्योतयोः ? वस्तुतः शिवदत्तप्रसादलभ्यां छायां राजलक्ष्मीरूपेण विपरिणमयन्तो विपरिणमनबुद्धिमपि शिथिलां तत्र तत्र प्रदर्शयन्त एते उपेक्षणीया एव~॥\\

पं० रघुनाथगुरुप्रसादवत् अपरोsपि कश्चित् भाष्ये कृतभूरिपरिश्रमं सर्वतन्त्रस्वतन्त्रं श्रीमन्नागेशभट्टमात्मौपम्येनोदरम्भरिपण्डितम्मन्यते~। तेन पण्डितम्मन्येन ईशवीये १९३९ तमे वर्षे रसगङ्गाधर आमूलचूलं संस्कृतो निर्णयसागरेण मुद्रितश्च~। तस्य पुस्तकस्य ७२ तमे पृष्ठे समाध्युदाहरणं \textendash\

\begin{quote}
{\qt स्वर्गनिर्गतनिरर्गलगङ्गातुङ्गभङ्गुरतरङ्गसखानाम्~।\\
केवलामृतमुचां वचनानां यस्य लास्यगृहमास्यसरोजम्~॥}
\end{quote}

\newpage
% ( २४ ) 

इत्युक्तम्~। तद्विवरणश्च {\qt अत्रारोहः प्रथमेऽर्धे~। तृतीयचरणे त्ववरोहः} इति ग्रन्थकृतैवोक्तम्~। {\qt तृतीयचरणे} इत्यस्य व्याख्यानं {\qt तृतीयचरण इति बहुव्रीहिः, द्वितीयेऽर्धे इत्यर्थः} इति नागोजीभद्टेनोक्तम्~। तत्र जयपुरसंस्कृतकलिजाध्यापकेन मज्जुनाथोपाह्वेन भट्टश्री मथुरानाथशास्त्रिणा ५ तृतीयश्चरणो यस्मिन्, उत्तरार्धे इत्यर्थः~। विचार्योsत्र नागेशमहोदयः, यस्य लास्येत्यादौ बन्धशैथिल्यस्याप्रतीतेः~। इति टिप्पण्या सम्भावितम्~। अत्र {\qt बन्धगाढत्वशिथिलत्वयोः क्रमेणावस्थानं समाधिः} इति समाधिलक्षणं विधाय प्रथमेऽर्धेबन्धगाढत्वमुक्त्वा तृतीयचरणे बन्धशैथिल्यमाह जगन्नाथः~। गाढत्वमोजोविवरणे लक्षितप्रायमित्थम् \textendash\ {\qt संयोगपरह्रस्वप्राचुर्यरूपं गाढत्वमोजः~। उदाहरण \textendash\ अयं पततु निर्दयं दलितदृप्तभूभृद्गलस्स्वलद्रुधिरघस्मरो इत्यादि} इति~। संयोगपरहस्वप्राचुर्यरूपं गाढत्वं वदता संयोगपरह्रस्वप्राचुर्याभाववत्त्वं शिथिलत्वमित्युक्तप्रायमेव~। तथा च स्वर्गनिर्गतेति पद्ये पूर्वार्धे संयोगपरह्रस्वप्राचुर्यात् बन्धगाढत्वं निराकुलं, मथुरानाथस्यापि सम्मतं च~। तृतीये चरणे केवलामृतमुचां वचनान इत्यत्र संयोगपरह्रस्वस्यैवाभावात् प्राचुर्याभावोऽपि नेति शिथिलत्वं नास्ति \textendash\ इति तृतीयचरणे त्ववरोहः इति मूलग्रन्थस्यासांगत्यवारणाय {\qt तृतीयचरणे इति बहुब्रीहिः} इति व्याख्यातं नागेशेन, चतुर्थचरणे {\qt यस्य लास्यगृहमास्यसरोजम्} इत्यत्र संयोगपरःःस्वस्यैकस्य {\qt यस्य} इत्यत्र सत्वेनापरस्य संयोगपरह्रस्वस्यासत्वेन प्राचुर्याभाव उपपन्न इति बन्धशैथिलस्योपपप्तिः~। स्वटिप्पण्यां {\qt यस्य लास्येत्यादौ बन्धशैथिल्यस्याप्रतीतेः} इति वदन् मथुरानाथः सुकविरपि महाविद्यालयाध्यापकोऽपि प्रथितयशस्कोऽपि श्रीमन्नागेशोक्तौ यत्र तत्र चिन्त्यत्वं प्रब्रुवन् {\qt यस्मिन् कुले त्वमुत्पन्नः} इत्युक्तिं स्मारयति~। वस्तुतस्तु ब्रह्मसूत्रन \textendash\ हानौ तूपायन शब्दशेषत्वात्०३~। ३~। २६इत्यस्य व्यख्यानावसरे भामतीव्याख्यान \textendash\ प्रसङ्गे कल्पतरुकारेण यदुक्तं \textendash\ {\qt पदवाक्यप्रमाणाब्धेः परं पारमुयेयुषः~। वाचस्पतेरित्यर्थेsप्यबोध इति साहसम्~॥ } इति, तदेव {\qt इत्यर्थेऽपि नागेशस्यावोध इति साहसम्} इति सम्परिवर्त्य पण्डितत्रयाणामप्युत्तरयितुं युक्तमित्यलं वरदोषाविष्करणेन~॥ नागेशग्रन्थप्रकाशनसमये तदीयाक्षेपनिराकरणं धर्म इति बुद्धयैव परदोषाविष्करणं, नान्यथेति यदत्रातिकान्तं तन्नौद्धत्येनेति भृशमभ्यर्थयामः~॥\\

अस्मदीयटिप्पणे क्वचित् विस्तरेण प्रपञ्चितमेतत्प्रस्तावनायाम् इत्युक्तिः प्रस्तावनाविषयविपर्यासात्स्थानाभावाच्च न पारितेति उदाराशयैस्तत्रभवद्भिः क्षन्तव्यमिति विज्ञापयामः~।\\

अत्र माननीयानामुक्तिविवेचनाय तदन्ते ( छाया ) ( दा.म. ) ( र.ना. ) इत्याद्युल्लेखो विनिवेशितः~। यत्र नास्ति तादृशः समयस्तदस्मदीर्यं विद्यार्थिनामुपयोगायैव, न पाण्डित्यप्रदर्शनायेत्यवगन्तव्यम्~॥\\

अपि चात्र त्रुटीनाम्बाहुल्येऽपि गुणैकपक्षपातिनो धीर! न तत्र दत्तचित्ता इत्यवधार्यापि पूर्वपुस्तकानुसरणशीलनस्वभावादन \textendash\ वधानाच्च वार्तिकक्रमाङ्कानाम्परिभ्रंश उद्वेजयत्येवैति तदर्थ नाप्राप्ता क्षन्तव्यम् इत्युक्तिस्तदितरविज्ञाताविज्ञातत्रुटिविषयेऽप्यवधेयेति विज्ञाप्यते~॥\\

आदर्शपुस्तकप्रदानेनोपकर्तृणामधमर्णा वयम्~। विल्सनमहाविद्यालये संस्कृतप्रधानाध्यापकानां श्रीमतां वेकणकरोपाह्वहरिपण्डितानां सर्वविधसाहाय्येनोपकृता वयम्~। यद्यपि निर्णयसागरास्थानपण्डितैरधिकारिभिश्च प्रकाशने बहूपकृतः वयं तथापि टङ्कसंयोजकहिराजीप्रमृतीनां संयोजनकौशलेन सञ्जाताश्वासा निश्चिन्ताश्चेति सत्यम्~। ग्रन्थान्ते ससूत्रवार्तिकपाठ उपन्यस्तः~। ततश्च विद्यार्थिनामुपयोगाय संशयितशब्दानां कोशोऽपि दत्तः~। ते च शब्दाः कम्मन् पृष्ठे वर्वन्त इति ज्ञानाय पृष्ठाङ्कोऽपि प्रदर्शितः~। अनल्पमतिभिरुपक्रमणीयेऽस्मिन् कर्मणि विद्वज्जनानुग्रहाद्यदस्माभिरुपक्रान्तत्सफलयन्तु तत्रभवन्तः~॥

\begin{quote}
{\qt यदत्र शौष्ठवं किञ्चित्तद्गुरोरेव मे नहि~।\\
यदत्रासौष्ठवं किञ्चित्तन्ममैव गुरोर्नहि~॥}
\end{quote}

इति मधुसूदनसरस्वतीनिष्यन्दैरत्रत्यसत्यतत्वप्रकथनेन पण्डितमण्डलीसमनुरागं सम्प्रार्थयति \textendash\ \\

\begin{tabular}{m{10em} m{30em} m{20em}}
& & काशीविश्वविद्यालयविनेयः \textendash\ \\

मकरसङ्क्रमणम्,& \multirow{2}{*}{$\Big\}$}& म. म. जयदेवमिश्रचरणान्तेवासी \textendash\ \\

शाके १८७२ & & भार्गवशास्त्री~। 
\end{tabular}

\begin{center}
\includegraphics[width=0.15\linewidth]{latex/e.JPG}
\end{center}

\newpage
\thispagestyle{empty}
\begin{center}
\textbf{\LARGE श्रीगणेशाय नमः~।}\\

\textbf{\Large छायाग्रन्थस्य मङ्गलमुखेन प्रारम्भः \textendash\ }
\end{center}

\begin{quote}
{\qt पतञ्जलिमुमाकान्तकमनीयविभूषणम्~। सहस्रवदनाम्भोजं सुवासितजगद्गुणम्~॥\\
योगिहृत्पद्मवसनं मूलमात्रविदारणम्~। स्फोटाभिधं स्फोटमूलं रूपकारणकारणम्~॥\\
वैद्यनाथः पायगुण्डो ध्यात्वा ध्यात्वा परं गुरुम्~। भाष्यप्रदीपसूद्द्योतान् व्याचष्टे गूढसंविदे~॥\\
शास्त्रं निर्णायकं न्याय्यकत्वालापं ( १ ) मुनिसत्तमाः~। सूत्ररूपेण कुर्वन्ति शिवस्यैवाज्ञयैव तु~॥\\
अल्पाक्षरमसंदिग्धं सारवद् विश्वतोमुखम्~। अस्तोभमनवद्यं च सूत्रंसूत्रविदो विदुः~॥\\
मु॑नयश्च मनुष्याश्च प्रसादादेवं शूलिनः~। सूत्रव्याख्यां प्रकुर्वन्ति वार्तिकानां स्वरूपतः~॥\\
उक्तानुक्तदुरुक्तानां चिन्ता यत्र प्रवर्तते~। तं ग्रन्थं वार्तिकं प्राहुर्वार्तिकज्ञा मनीषिणः~॥\\
सूत्रार्थ भाष्यरूपेण यथावद्दर्शयन्ति च~॥ \\
सूत्रार्थो वर्ण्यते यत्र वाक्यैः सूत्रानुका ( सा ) रिभिः~। स्वपदानि च वर्ण्यन्ते भाष्यं भाष्यविदो विदुः~॥\\
प्रसादादेव रुद्रस्य भवानीसहितस्य तु~। कुर्वन्ति केचिद् व्याख्यानं भाष्यस्यैव तपोबलात्~॥\\
पदच्छेदः पदार्थश्चं विग्रहो वाक्ययोजना~। आक्षेपश्च समाधानं व्याख्यानं षड्विधं मतम्~॥\\
स्वबुद्ध्यधीतभाष्यार्थसंग्रहेणैव १वाऽथवा~। विस्तरेण प्रकुर्वन्ति केचित् प्रकरणात्मना~॥\\
शास्त्रैकदेशसंसिद्धं शास्त्रकार्यान्तरे स्थितम्~। आहुः प्रकरणं नाम शास्त्रभेदविचक्षणाः~॥\\
सूत्रभाष्यादिभिः शास्त्रं साक्षाद्वेदनसाधनम्~। श्रोतव्यं स्वगुरोः स्वात्मस्वरूपप्रतिपत्तये~॥\\
आत्मलाभात् परं नास्ति किंचिन्मात्रमपि द्विजाः~। पराशरपुराणे वै स्पष्टमेतन्निबोधत~॥\\
जीणैरनुक्तमत्रोक्तं सूत्राद्यर्थप्रतीतये~॥ \\
विष्णुधर्मोत्तरे प्रोक्तं मार्कण्डेयेन धीमता~। वज्रं प्रति यथा सर्वं प्रोच्यतेऽथ २समा अणु~॥}

{\mbh तत्राशायुक्तमद्वैधं दीप्तं गम्भीरशब्दवत्~। क्वचिन्निरुक्तसंयुक्तं वाक्यमेतत्स्वयंभुवः~॥~१~॥

यत्किचिन्मिश्रसंयुक्तं युक्तं नामविभक्तिभिः~। प्रत्यक्षाभिहितं यच्च तदृषीणां वचः स्मृतम्~॥~२~॥

नैगमैर्विविधैः शब्दैर्निपातबहुलं च यत्~। न चापि सुमहद्वाक्यं ३ऋषिकाणां वचः स्मृतम्~॥~३~॥

अविमृष्टपदं ज्ञेयमृषिपुत्रवचो नृप~। भूतभव्यभवज्ज्ञानं जन्मदुःखविकुत्सनम्~॥~४~॥

मित्राणां तद् भवेद्वाक्यं ४गर्भेष्वार्षप्रवर्तनम्~। आज्ञायुक्तं तु वचनं तथा हेतुविवर्जितम्~॥~५~॥

राजर्षीणां तु विज्ञेयं बह्वर्थ बहुविस्तरम्~। बह्वभिधानं बह्वर्थं देवतानां प्रकीर्तितम्~॥~६~॥

बह्वभिधानमल्पार्थ दानवानां प्रकीर्तितम्~। अल्पाभिधानमल्पार्थं गन्धर्वाणां तथा भवेत्~॥~७~॥

दुर्बोधं विषमं चैव राक्षसानां प्रकीर्तितम्~। गूढाक्षरं तु यक्षाणां ऽकिन्नरैरुक्तवत्तथा~॥~८~॥

नागानामतिविस्पष्टं पुनरूक्तसमन्वितम्~। रागद्वेषसमायुक्तं हेतुमत्पौरुषं भवेत्~॥~९~॥}
\end{quote}

\vspace{-7mm}
\begin{flushright}
इति चतुर्थोऽध्यायः~॥ 
\end{flushright}

\begin{quote}
{\mbh अल्पाक्षरमसंदिग्धं सारवद्विश्वतोमुखम्~। अस्तोभमनवद्यं च सूत्रं सूत्रविदो विदुः~॥~१~॥

उत्सर्गेणापवादेन द्विविधं तत्प्रकीर्तितम्~। सूत्रेष्वेव हि तत् सर्वं यद्वृत्तौ समुदाहृतम्~॥~२~॥

सूत्रं ६व्युदासश्च तथा तथोदाहरणं नृप~। प्रत्युदाहरणं चैव चतुरङ्गं प्रकीर्तितम्~॥~३~॥

वाक्यं चैवाथ वाक्यार्थः पदार्थः पदमेव च~। चतुरङ्गमिदं वेद७ तथैवान्यत्प्रकीर्तितम्~॥~४~॥

प्रतिज्ञा हेतुदृष्टान्तावुपसंहार एव च~। तथा निगमनं चैव पञ्चावयवमिष्यते~॥~५~॥ 

आरम्भोऽथापि संबन्धः सूर्यार्थस्तद्विशेषणम्~। चोदकं८ परिहारश्च व्याख्या सूत्रस्य षड्विधा~॥~६~॥

विस्तरोक्तं मतिं हन्ति समासोक्त न गृह्यते~। समासविस्तरौ हित्वा वक्तव्यं यद्विवक्षितम्~॥~७~॥

अपार्थ व्याहतं चैव पुनरुक्तं तथैव च~। तथा विभिन्नसंस्थानं युक्तिहीनं विवर्जयेत्~॥~८~॥

क्रमभेदो विभक्तश्च गुरुसूत्रं तथैव च~। अभिधानस्य चान्यत्वं नैतानि स्युरकारणात्~॥~९~॥}
\end{quote}

\noindent
\rule{1\linewidth}{0.5pt}

\begin{quote}
{\qt गुरुप्रसादलब्धार्थः शिवदत्तः सतां मतः~। छायादिग्रन्थानालोच्य सारं संगृह्य यत्ततः~॥}
\end{quote}

१ संग्रहेण संक्षेपेणैव वा प्रकुर्वन्ति~। अथवा विस्तरेण प्रकुर्वन्नीति योजना~। ( र. ना. ) २ {\qt समासतः} इति पाठः~। ३ {\qt मृचीकानाम्} इति पाठः~। ( दाधिमथाः ) ४ {\qt गर्भेषु च प्रवर्तनम्} इति पाठो दृश्यते~। ( र. ना. ) ५ वच उच्यते इति शेषः~। ( र. ना. ) ६ विधि \textendash\ सूत्रं, व्युदासो निषेधसूत्रमित्यर्थः~। ( र. ना. ) ७ वेद इत्यस्य स्थाने चेतीति पाठ उपलभ्यते~। ( र. ना ) ८ {\qt चोदना} शति पाठान्तरम्~। 

\fancyhead[RE,LO]{[ १ अ. १ पा. १ आह्निके}
\fancyhead[CE,CO]{\thepage}
\cfoot{}
\newpage
%%%%%%%%%%%%%%%%%%%%%%%%%%%%%%%%%%%%%%%%%%%%%%%%%%%%%
\renewcommand{\thepage}{\devanagarinumeral{page}}
\setcounter{page}{2}

% २ [ १ अ. १ पा. १ आह्निके 

\begin{quote}
{\mbh १पूर्वं कृत्वा पदच्छेदं समासं तदनन्तरम~। समासे तु कृते पश्चादर्थं ब्रूयाद्विचक्षणः~॥~१०~॥

सूत्रार्थश्च पदार्थश्च हेतुश्च २क्रमशस्तथा~। निरुक्तमथ विन्यासो व्याख्या योगस्य षडविधा~॥~११~॥

उपोद्धातः प्रथमतः पदार्थः पदविग्रहः ( हौ ) ~। अविमर्शः प्रत्यवस्था व्याख्या तन्त्रस्य षड्विधा~॥~१२~॥

प्रत्यक्षमनुमानं च तथाप्तवचनं भवेत्~। त्रिभिः प्रमाणैः ३संयुक्तं तन्त्रं प्रामाण्यमर्हति~॥~१३~॥

प्रत्यक्षाभासमप्युक्तं तथा वै मृगतृष्णिका~॥~१४~॥

आभासमनुमानं च बाष्परेणुचयौ यथा~। धूमशङ्कां जनयतो नचैतदुपलब्धये~॥~१५~॥

आप्तत्वं दर्शयित्वा तु यदनाप्तेन भाषितम्~। तदाप्तवचनाभासं वदन्ति नृपसत्तम्~॥~१६~॥

रागद्वेषविनिर्मुक्त आाप्त इत्यभिधीयते~। नैरुक्तं द्विविधं विद्धि सिद्धमौत्पत्तिकं तथा~॥~१७~॥

निर्वक्तव्यं तु तत्सिद्धमर्थसिद्धिस्तु सर्वदा~। तत्र त्वौत्पत्तिकं सर्वं गौरश्वः पुरुषो यथा~॥~१८~॥

दशधा सु ४गणो ह्येष तं वक्ष्यामि यथामतम्~॥~१९~॥

गौणो जैमिक्षिको भाक्तः सांवादः कार्तकस्तथा~। औपचारश्च सांबन्धः क्रैयिको यौगिकैच्छिकौ~॥~२०~॥

गुणात्तु गौणं पश्यामो महदित्येवमादयः~। निमित्ततस्तु नैमित्तं खण्डमुञ्जादिकं यथा~॥~२१~॥

भक्त्या भाक्तो नृसिंहः स्यादमातरि च मातृवत्~। समं वदन्ति यस्मिन् स सांवादो जित्वरी यथा~॥~२२~॥

कार्तकः स्यात्तु कृतको देवदत्तादिको यथा~। औपचारश्चोपचारात्तैलं पाशं तुला यथा~॥~२३~॥

संबन्धतस्तु सांबन्धश्छत्री मीमांसको यथा~। संयोगादपि सायोगः क्रियायाः क्रैयिकः स्मृतः~॥~२४~॥

धटकृल्ल्गुडैश्चैद्यो५ जेयः शाकटिक६स्तथा~। एवं परेषां च परं स्यान्नैगमनिघण्टुकम्~॥~२५~॥

७साध्योऽर्थो न प्रसिध्येत साधनानां विधिं शृणु~। यद्विपन्नं तु दृश्येत तत्साध्यं साधनैः ८परैः~॥~२६~॥

आत्मेन्द्रियमनोर्थानां संयोग उपदिश्यते~। शानदेशादिसंदि ( वि ) ष्टं प्रत्यक्षमिति तद्भवेत्~॥~२७~॥

बुद्ध्या शरीरयुक्ता ( भूता ) त्मा यश्च युक्तोऽनुमीयते~। सोऽग्निर्धूमाद्यथा विद्यादनुमानं तदिष्यते~॥~२८~॥

वेदविद्याऽविरुद्धं यत्सा स्मृतिः शिष्टसंमता~। अप्रत्यक्षफलं विद्यात्साधनं शास्त्रसंशितम्~॥~२९~॥

द्योः सदृशयोरेकं निदिष्टं साधयेत्परम्~। गवयादिव गोः सिद्धिरुपमानं तदिष्यते~॥~३०~॥

९साध्योऽर्थश्चेदनुक्तोsर्थाद्वाक्ये यस्मिन्प्रकल्प्यते~। सार्थापत्तिर्दिवा भुङ्के देवदत्त इतीति यत्~॥~३१~॥

निपातनाद्योगविभागदर्शनाद्गुरूपदेशादनुवार्तिकादणि~। स्वातन्त्र्यसिद्धेः१० परतन्त्रदर्शनात्प्रसाधयेल्लक्षणतोऽथषड्विधा~॥~३२~॥}
\end{quote}

\vspace{-7mm}
\begin{flushright}
इति विष्णुधर्मोत्तरे तृतीयखण्डे पञ्चमोऽध्यायः~॥
\end{flushright}

अधिकरणं योगः पदार्थो हेत्वर्थं उद्देशो निर्देश उपदेशोऽपदेशः प्रदेशोऽतिदेशोरपवर्गो वाक्यशेषोऽर्थापत्ति प्रसङ्ग एकान्तोऽनेकान्तः पूर्वपक्षो निर्णयो विधानं विपर्ययोऽतिक्रान्तवीक्षणमनागतवीक्षणं संशयो व्याख्यानमनुमतं स्वसंज्ञा निर्वचनं दृष्टान्तो नियोगो विकल्ल्प: समुच्चय ऊह्यमिति~॥ तत्र यमर्थमधिकृत्योच्यते तदधिकरणम्~॥ येन वाक्यार्थो युज्यते स योगः~॥ योऽर्थोऽविकृतसूत्रपदे स पदार्थः~॥ यदन्यतो युक्तिमदर्थस्य साधनं स हेत्वर्थः~॥ समासवचनमुद्देशः~॥ विस्तरवचनं निर्देशः~॥ एवमेवेत्युपदेशः~॥ अनेन कारणेनेत्यप \textendash\ देशः~॥ प्रकृतस्यानागतेन साधनं प्रदेशः~॥ अतिक्रान्तेनातिदेशः~॥ अभिप्रायानुकर्षणमपवर्गः~॥ येनार्थः परिसमाप्यते पदेनाहार्येण स वाक्यशेषः~॥ यदकीर्तितमर्थादापाद्यते सार्थापत्तिः~॥ प्रकरणाभिहितोऽर्थः केनचिदुपोद्वातेन पुनरुष्यमानः प्रसङ्गः~। सर्वत्र यस्तथा स एकान्तः~॥ क्वचिदन्यथा क्वचित्तथा सोऽनेकान्तः~॥ प्रतिषेधवचनं पूर्वपक्षः~॥ तस्योत्तेर्वचनं निर्णयः~॥ प्रकरणानुपूर्व्यं विधानम्~॥ तस्य प्रातिलोम्यं विपर्ययः~॥ इत्युक्तमित्यतिक्रान्तवीक्षणम्~॥ परत्र वक्ष्यमीत्यनागतवीक्षणम्~॥ उभयतो हेतुदर्शनं संशयः~॥ तत्रातिशयवर्णना व्याख्यानम्~॥ परमतमप्रतिषिद्धमनुमतम्~॥ परैरसंमतः शब्दः स्वसंज्ञा~॥ लोके प्रतीतमुदाहरणं निर्वचनम्~॥ तद्युक्तिनिदर्शनं दृष्टान्तः~॥ इदमेवेति नियोगः~॥ इदं वेदं वेति विकल्पः~। ? इदं चेदं चेति समुच्चयः~॥ यदनिदिष्टं युक्तिगम्यं तदूह्यम्~॥ इति~।\\

प्रयोजनं संशयनिर्णयौ च व्याख्याविशेषो गुरुलाघवं च~। कृतव्युदासोऽकृतशासनं च स ११वार्तिको धर्मगुणोऽष्टकश्च~॥

\vspace{-7mm}
\begin{flushright}
इति विष्णुधर्मोत्तरे तृतीयखण्डे षष्ठोऽध्यायः~॥ 
\end{flushright}

\begin{center}
यद्यपि नोच्चितमिदं दुर्बुद्धेर्मम मन्दतः~। तथापि श्रीशदयया प्रवृत्तोऽहं समीरितः~॥
\end{center}

\noindent
\rule{1\linewidth}{0.5pt}\\

१ सर्व इति पाठः~। २ ऋमयुक्तं व्याख्यानमित्वर्थः~। ( र. ना. ) ३ संयुक्तैस्तन्त्री प्रा \textendash\ ? इति पाठभैदः~। ४ गणोद्गीतं तद्वक्ष्यामि यथा गतम् इति पाठः~। ५ लगुडच्छेद्य इति पाठान्तरम्~। ( र. ना. ) ६ पूर्वार्धेन क्रैयिकोदाहरणम्~। ( र. ना. ) ७ {\qt सुखार्थो न प्रसिद्धश्चेत} इति पाठः~। ८ साधनैरिमैः इति पाठः~। ९ {\qt साध्योऽर्थश्चेदतत्कार्यो वाक्ये} इति पाठः~। १० स्वतन्त्रसिद्धेरिति पाठः, स्वशास्त्र \textendash\ सिद्धेरित्यर्थः~। ( र. ना. ) ११ वृत्तौ भव इत्यर्थः~। ( दाधिमथाः ) 

\newpage
\thispagestyle{empty}
\begin{center}
\includegraphics[width=0.15\linewidth]{latex/f.JPG}\\

\textbf{\large श्रीमद्भगवत्पतञ्जलिमुनिविरचितं}\\

\textbf{\LARGE महाभाष्यम्~।}\\

\rule{0.2\linewidth}{0.5pt}

\textbf{\large कैयटप्रणीतप्रदीपप्रकाशितम्~।}

\textbf{\large नागोजीभट्टविरचितोद्योतसहितम्~॥}

\textbf{\large प्रथमाध्याये प्रथमपादे प्रथमम् \textendash\ }

\textbf{पस्पशाह्निकम्~। }

 ( प्रदीपः ) 
\end{center}

\begin{quote}
{\mbh सर्वाकारं निराकारं विश्वाध्यक्षमतीन्द्रयम्~। सदसद्रूपतातीतमदृश्यं माययावृतैः~॥~१~॥

ज्ञानलोचनसंलक्ष्यं नारायणमजं विभुम्~। प्रणम्य परमात्मानं सर्वविद्याविधायिनम्~॥~२~॥

पुरुषाः प्रतिपद्यन्ते देवत्वं यदनुग्रहात्~। सरस्वतीं च तां नत्वा सर्वविद्याऽधिदेवताम्~॥~३~॥

पदवाक्यप्रमाणानां पारं यातस्य धीमतः~। गुरोर्महेश्वरस्यापि कृत्वा चरणवन्दनम्~॥~४~॥

महाभाष्यार्णवावारपारीणं विवृतिप्लवम्~। यथागमं विधास्येऽहं कैयटो १जैयटात्मजः~॥~५~॥

भाष्याब्धिः क्वातिगम्भीरः क्वाहं मन्दमतिस्ततः~। छात्राणामुपहास्यत्वं यास्यामि पिशुनात्मनाम्~॥~६~॥

तथापि हरिबद्धेन सारेण ग्रन्थसेतुना~। क्रममाणः शनैः पारं तस्य प्राप्ताsस्मि पङ्गुवत्~॥~७~॥}
\end{quote}
 
\begin{center}
 ( उद्योतः ) 
\end{center}

\begin{quote}
{\mbh नत्वा साम्बशिवं देवीं वागधिष्ठानिकां गुरून्~। पाणिन्यादिमुनीन् वन्द्यान् पितरौ च सतीशिवौ~॥~१~॥

नागेशभट्टो नागेशभाषितार्थविचक्षणः~। हरिवीक्षितपादाब्जसेवनावाप्तसन्मतिः~॥~२~॥

याचकानां कल्पतरोररिक्षहुताशनात्~। शृङ्गवेरपुराधीशाद्रामतो लब्धजीविकः~॥~३~॥

नाविस्तीर्णं न विस्तीर्णं मध्यानामपि बोधकृत्~। भाष्यप्रदीपव्याख्यानं कुर्वेऽहं तु यथामति~॥~४~॥}
\end{quote}

भाष्यं ब्याचिकीर्षुः शिष्टाचारप्राप्तं मङ्गलं शिष्यशिक्षायै निबध्नाति \textendash\ सर्वति~। क्त्वान्तानां पञ्चमश्लोकोत्तरार्धेनान्वयः~। सर्वाकारत्वं सर्वोपादानकारणत्वान्मृदादिवत्~॥ कार्यातिरिक्तदृश्यस्वरूपाभावात् \textendash\ निराकारत्वम्~॥ कर्तृत्वात \textendash\ विश्वाध्यक्षत्वम्, {\qt स ऐक्षत} इति श्रुतेः~॥ अतीन्द्रियत्वं \textendash\ विगतेन्द्रियत्वेन, {\qt पश्यत्यचक्षुः} इत्यादिश्रुतेः~। सर्वप्रत्यक्षीकरणं तु स्वरूपचैतन्येनैव~॥ इन्द्रियाविषयत्वेन वाऽतीन्द्रियत्वम्~॥ २सदसद्रूपत्वं अनिर्वचनीयता, तामतीतं, तद्रहितमित्यर्थः~। एवं च सदेवेति भावः~॥ नन्वीदृशमस्माभिः कुतो न गृह्यत इत्यत आह \textendash\ अदृश्यमित्यादि~। १~॥ माययावृतैरित्यस्य कृत्यमाह \textendash\ ज्ञानेति~। ३तत्त्वमस्यादिमहावाक्यजनिताखण्डाद्वितीयसच्चिदानन्दब्रह्माकारा वृत्तिःज्ञानं तदेव

\noindent
\rule{1\linewidth}{0.5pt}\\

 ( छाया ) उद्द्योते \textendash\ प्रारिप्सितप्रतिबन्धकशमनाय कृतं समुचितस्वेष्टदेवतानत्यात्मकं मङ्गलं शिष्यशिक्षायै व्याख्यातृश्रोतृणामनुषङ्गतो मङ्गलाय च क्रमशो नि्वध्नाति \textendash\ नत्वेति~। पद्योक्तसर्वद्वितीयान्तानामत्रान्वयः~। अस्य च {\qt कुर्वे} इत्यत्रान्वयः~॥ वागेति~। सरस्वतीमित्यर्थः~। इष्देवतानत्युत्तर समुचितसरस्वतीनत्युत्तरं समुचितमुनिनतिमाह \textendash\ गुरूनित्यादि~। गुरून् \textendash\ जगद्गुरुन्~॥ गुरुनत्युत्तरं मुनिनस्ययुक्तेर्भेदेन व्याख्यानमयुक्तमिति बोध्यम्~॥ पित्रादिस्वर्गायाह \textendash\ पितराविति~॥ चः \textendash\ सर्वषां पूज्यानां समुञ्चायकः~॥ स्वकीर्तय आह \textendash\ नागेशेति~। भद्टः \textendash\ सकलशास्त्रज्ञः~। स्पष्टं चेदं वेणीसंहरणव्याख्यायाम्~॥ नागेशेति~। फणिभाषितभाष्याब्धिसारज्ञ इत्यर्थः~॥ एतच्छास्त्रोप \textendash\

\noindent
\rule{1\linewidth}{0.5pt}\\

\begin{quote}
{\qt गौरीविश्वेश्वरौ नत्वा प्रथमाध्यायवर्तिनाम्~। भाष्यप्रदीपोद्योतानाम्प्रभाम्पुष्णाति भार्गचः~॥}
\end{quote}

१जैय्यटा इति अ. पाठः~। 

२ {\qt सदसद्गूपता \textendash\ } प्रपञ्चधर्मः~। एतञ्च स्त्रियाम्{\qt इति सूत्रे वक्ष्यते} इति च. पुस्तकपाठः~। 

३ अ, पुस्तके तत्वमस्यादिचाक्य \textendash\ इति महाशब्दरहितः पाठः~। 

\fancyhead[RE]{[ १ अ. १ पा. १ आह्निके}
\fancyhead[LO]{शास्त्रप्रयोजनाधिकरणम् ] }
\fancyhead[LE,RO]{\thepage}
\fancyhead[CE]{उद्द्योतपरिवृतप्रदीपप्रकाशितमहाभाष्यम्~।}
\fancyhead[CO]{महाभाष्यप्रदीपोद्दयोतव्याख्या छाया~।}
\cfoot{}
\newpage
%%%%%%%%%%%%%%%%%%%%%%%%%%%%%%%%%%%%%%%%%%%%%%%%%%%%%
\renewcommand{\thepage}{\devanagarinumeral{page}}
\setcounter{page}{4}

% ४ उद्द्योतपरिवृतप्रदीपप्रकाशितमहाभाष्यम्~। [ १ अ. १ पा. १ आह्निके

\noindent
लोचनं, तेन संप्रेक्ष्यमित्यर्थः~॥ नारायणमित्यादि~। सर्वनरसमूहान्तर्यामिणमित्यर्थः~॥ सर्वविद्या \textendash\ वेदरूपा, तद्विधायिनम्~॥~२~॥ वचनसंदर्भविशेषरूपग्रन्थे समुचितत्वाद्वागधिदेवतां नमति \textendash\ पुरुषां इति~। देवत्वं$=$पाण्डित्यम्~॥~३~॥ संप्रदायशुद्धिं द्योतयन् गुरुं नमति \textendash\ पदेति व्याकरणमीमांसातर्काणामित्यर्थः~। धीमतः \textendash\ ऊहापोहकुशलस्य~॥~४~॥ महाभाष्येति~। व्याख्यातृत्वेऽप्यस्येष्ट्यादिकथनेनान्वाख्यातृत्वादितरभाष्य वैलक्षण्येन महत्त्वम्~॥ अवारपारयोर्भवम्~। प्लूयतेऽनेनेति प्लवं \textendash\ सेतुम्~॥ यथागमम्~। संप्रदायमनतिलङ्घयेत्यर्थः~॥~५~॥ औद्धत्यं परिह्ररति \textendash\ भाष्येति~। मयूरव्यंसकादिसमासो रूपकं चेति बोध्यम्~॥ पिशुनत्वं \textendash\ दुष्टत्वम्~।~६~॥ नन्वेवं कथं प्रवृत्तिरत आह \textendash\ तथापीति~। तत्रत्यसारांश एवात्र निबद्ध इति भावः~॥ शनैरित्यनेनात्यन्तग्रन्थासंकोचः~। पङ्गुवदित्यनेन व्याख्येयांशापरित्यागो व्याख्यानस्याविस्तृतत्वं च बोध्यते~॥~७~॥

\noindent
\rule{1\linewidth}{0.5pt}\\

योगितया प्राधान्येन वाऽऽह \textendash\ हरिदीक्षितेति~॥ सदा बुद्धिस्थैर्याय तदुपयोगिनस्तस्य स्वर्गावाप्तये चाह \textendash\ याचकेति~॥ कक्षं \textendash\ तृणम्~॥ रामतः \textendash\ रामसिंहवर्मतः~। दानशूरत्वं रणशूरत्वं उभयसाधकं लक्ष्मीवत्त्वं च क्रमेणात्र प्रतिपादितम्~॥ तत्राभिधेयं प्रतिज्ञातुं प्राचीनव्याख्यानामनादेयत्वायाह \textendash\ नावीति~॥ अपिना कनिष्ठपरिग्रहः~॥ भाष्येति~। द्वन्द्वोत्तरं षष्ठीतत्पुरुषः~॥ प्राचीनव्याख्याकर्तृवैलक्षण्यसूचकः {\qt तु} \textendash\ शब्दः~॥ १तत्त्वं चोक्तरीत्या बोध्यम्~॥ यथामति \textendash\ इत्यनेनाज्ञाननिरासः~॥\\

क्त्वान्तेति~। तदर्थानामित्यर्थः~। नमःपदेनैव तत्प्रतीतौ तस्य कार्याव्यवहितपूर्वक्षणवृत्तित्वरूपकारणत्वबोधनायोपादानम्~। अनेन ग्रन्थारम्भे मङ्गलस्यावश्यकर्तव्यत्वं सूचितमिति बोध्यम्~॥ उत्तरार्धेनेति~। तदुत्तरार्धप्रतिपाद्यप्रधानवाक्यार्थक्रिययेत्यर्थः~॥ विरोधपरिहाराय मतभेदं कल्पितमुपेक्ष्य वास्तवं प्रकारमाह \textendash\ सर्वोपेति~। {\qt यतो वा \textendash\ } इत्यादिश्रुतेः~। तथा च बहुब्रीहिः, तत्पुरुषो वा~॥ मृदेति~। २मध्यमणिन्यायेनान्वेति~॥ तत्राद्ये साधर्म्ये दृष्टान्तः, अग्रे वैधर्म्ये~॥ तदन्यस्वरूपसत्त्वादाह \textendash\ दृश्येति~। बहिरिन्द्रियवेद्येत्यर्थः~। तथा \textendash\ चात्रापि प्राग्वत्समासद्वयं बोध्यम्~॥ प्राग्वदाह \textendash\ कर्तृत्वादिति~॥ श्रुतेरिति~। अनेन बहुब्रीहिः सूचितः~॥ अतीन्द्रियत्वं नेन्द्रियाविषयत्वेन, अदृश्यमित्यस्य वैयर्थ्यापत्तेरत आह \textendash\ विगतेन्द्रीति~॥ अचक्षुः \textendash\ चक्षुरिन्द्रियशून्यः~। आदिना \textendash\ {\qt स शृणोत्यकर्णः} इत्यादिः परिग्रहः~॥ नन्वेवं प्रागुक्तश्रुतिविरोधोऽत आह \textendash\ सर्वेति~॥ विश्वं साक्षात्कुर्वन्नपि नास्मदादिवदिन्द्रियद्वारा किं तु तथेति भावः~॥ {\qt यतो वाचः} ति श्रुत्यनुरोधेन विनिगमनाविरहादाह \textendash\ इन्द्रियेति~। तस्य वैयर्थ्यं परिहरिष्यते~॥\\

प्रदीपे \textendash\ अस्य कारणत्वेऽपि न मायातुल्यत्वमित्याह \textendash\ सदिति~॥ अत्र दण्डी \textendash\ {\qt अनेन तद्वदनिर्वचनीयत्वं निरस्तम्~। तथा हि \textendash\ सदसच्छब्दौ मिथोविलक्षणवाचकौ~। तथा च तज्जाsनिर्वचनीयता~। यद्वा \textendash\ सत् \textendash\ पृथ्व्यादित्रयम्, असत् \textendash\ वाय्वाकाशद्वयम्} इति~। तन्न, {\qt नानिर्वचनीयस्य ख्यातिस्तदभावात्} इति सांख्यसूत्रेण तन्निरासात्~। अत एव {\qt असत्तु मृगतृष्णावत्} इत्येवोक्तम्~। न त्वनिर्वचनीयत्वमितीति {\qt स्त्रियाम्} इति सूत्रे भाष्ये~। तत्र सत्त्वमधिष्ठानब्रह्मगतमारोपितम्, असत्त्वं वास्तवम्~। विस्तरस्तत्रैव बोध्यः~॥\\

उद्द्योते \textendash\ तदेतद्ध्वनयन्नाह \textendash\ सदित्यादि~॥ नन्विति~। अनेन वैयर्थ्यमपि परिहृतम्, दृशधातोर्ग्रहणपरत्वात्~॥~१~॥ सर्वेषां न परोक्षमिति सूचनायोक्तं विशेषणमित्याह \textendash\ माययेति~॥ बहुव्रीहावनन्वयादेराह \textendash\ तदेवेति~॥ सर्वेति~। तथाच {\qt नारायणो} विष्णुरजो ब्रह्मा विभुर्महेश्वरः इति तेषां पालनकर्तत्वमपि तदधिष्ठितत्वेनेति बोध्यम्~॥ तत्फलमाह \textendash\ सर्वेति~॥ तद्याचष्टे \textendash\ वेदेति~। तत्रादेः श्रुतिविरुद्धस्याप्रमाणत्वात् विरुद्धस्य तेनैवासंग्रहात्त्यागः~॥~२~॥ {\qt पुरुषा देवत्वम्} इति कठिनकोमलशब्दोल्लेखेन वज्रस्यापि द्रवीभावो यतः किमस्येति सूचितंम्~॥ वस्तुतो देवत्वस्याभावादाह \textendash\ पाण्डित्यमिति~॥~३~॥ गुरुमिति~। अनेन सर्वविद्यालभ एकस्मादेवेति सूचितम्~॥ वैयर्थ्यं परिहरति \textendash\ ऊहेति~॥\\

कैयटे \textendash\ अपिः \textendash\ चार्थे~॥~४~॥ अभिधेयं प्रतिजानीते \textendash\ महेति \textendash\ कैयटे~॥\\

ख्यातृत्वादिति~। अविरोधादिति भावः~॥ महत्त्वमिति~॥ एतेन ग्रन्थतोऽर्थतश्च महत्वमिति विवरणोक्तमपास्तम्~॥ तद्विना तेन समासः, गुरुकुलवद्वृत्तिरिति सूच्ययन्नाह \textendash\ अवारेति~॥ प्लवनकर्तृत्वेन ३तत्त्वस्य नौकादौ सत्त्वेऽपि तस्यावारपारीणत्वासंभवादाह \textendash\ प्लूयेति~॥ इति \textendash\ इतिहेतोः~। रूपकमुभयत्र~॥ श्रुत्याद्यनतिक्रमस्यानुपयुक्तत्वादाह \textendash\ संप्रेति~॥~५~॥\\

रूपकमिति~। एतेनातिगम्भीरत्वस्य साधारणधर्मस्योपादानात्कथमुपमितसमास इत्याशङ्क्य तेन तत्त्वं न विवक्षितं किंत्वन्यथेति दीक्षितद्युक्तमपास्तम्~॥ सूचकत्वस्याफलत्वादाह \textendash\ दुष्टत्वमिति~॥~६~॥ सारपदस्वारस्यमाह \textendash\ तत्रत्येति~॥~७~॥

\noindent
\rule{1\linewidth}{0.5pt}\\

१ तत्वं \textendash\ ~। प्राचीनव्याख्यावैलक्षण्यम्~। उक्तरीत्या \textendash\ नाविस्तीर्णमित्यादिरीत्या~। २ मध्येति~। दृश्यस्वरूपाभावादित्यत्राप्यन्वयः~। ३ तत्त्वस्य \textendash\ प्लवत्वस्य~। ४ घप्रत्ययं विनैवावारपाराभ्यां सह महाभाष्यार्णवस्य समासः~। घप्रत्ययरूपवृत्तिस्तु अवारपारशब्दादेव~। ( र. ना. ) अत्र घप्रत्ययोपादानं प्रमादः, खप्रत्ययं विनैवेत्यादि वक्तुमुचितम्~। {\qt अवारपारशब्दाभ्यां खप्रत्ययः}इति राजलक्ष्मीकारोक्तिः प्रामादिकी अवारपारशब्दात् खप्रत्ययः इति वक्तव्यम्~। विशिष्टात्प्रत्यये शैषिकोऽण् स्यादिति भेदेन {\qt अवारपारयोर्मवम्} इत्युद्योत उच्यते~।

\newpage
% शास्त्रप्रयोजनाधिकरणम् ] महाभाष्यप्रदीपोद्दयोतव्याख्या छाया~। ५

\begin{multicols}{2}
\begin{center}
 ( शास्त्रप्रयोजनाधिकरणम् ) 

 ( भाष्यम् ) 

\textbf{\large अथ शब्दानुशासनम्~।}
\end{center}

{\qt अथेत्ययं शब्दोऽधिकारार्थः प्रयुज्यते~। शब्दानु \textendash\ शासनं नाम शास्त्रमधिकृतं वेदितव्यम्~॥}

 ( प्रदीपः ) भाष्यकारो विवरणकारत्वात् व्याकरणस्य सा \textendash\ क्षात्प्रयोजनमाह \textendash\ अथ शब्दानुशासनमिति~। प्रयोजनप्रयोजनानि तु रक्षोहादीनि पश्चाद्वक्ष्यन्ते~॥\\

१स्ववाक्यं व्याख्यातुं तदवयवमथशब्दं तावत् व्याचष्टे \textendash\ अथेत्ययमिति~। इतिशब्दोऽथशब्दस्य स्वरूपेऽवस्थापनाय प्रयुक्तः~। एवं हि पदान्तरैः सामानाधिकरण्येन संबन्धे सति {\qt अथ}शब्दो व्याख्यातुं शक्यते~। स्वरूपेsऽवस्थितश्च सर्वनाम्ना परामृश्यते \textendash\ {\qt अयं} इति~॥ शब्द इति~। स्वरूपकथनं विस्पष्टप्रतिपत्त्यर्थम्~॥ अधिकारार्थ इति~। अधिकारः$=$प्रस्तावः, द्योत्य२ \textendash\ \\

\noindent
\rule{1\linewidth}{0.5pt}\\

ननु {\qt वृद्धिरादैच} इत्यत आदावेवोच्यतामत आह \textendash\ कैयटे [ प्र० १मपङ्क्तौ ] व्याकरणस्येति~॥ तथा चोपोद्धाततया समस्तशा \textendash\ स्त्रारम्भे कथनस्य योग्यतयाऽत्राभिधानं युक्तमिति भावः~॥\\

कैयटे [ २ यप० ] प्रयोजनमाहेति~। प्रयोजनं घ्रुवन् शास्त्रा \textendash\ रम्भप्रतिज्ञामाहेत्यर्थः~॥ अनेन मूलानुक्तमपि व्याख्यात्राऽवश्यं दर्श \textendash\ नीयमिति सूचितम्~॥\\

इतिशब्दस्यादिहेतुप्रकारसमाप्त्याद्यर्थकत्वात्पदार्थविपर्यासकृत्त्वाच्चा ह\textasciitilde [ प्र० ४ र्थप० ] इतीति~॥ \\

अथाधिकारार्थं इत्युक्तेsनन्तरं कश्चिच्छब्दस्तदर्थं इति गम्येतात आह \textendash\ [ प्र० ६ ष्ठप० ] अथशब्द इति~॥\\

अधिकारो न प्रतिपादनम्, तथाऽकृतत्वादत आह \textendash\ कैयटे \textendash\ [ प्र० ९ प० ] प्रस्ताव इति~॥\\

ननु वाचकत्वमेव किं नात आह \textendash\ [ प्र० १० प० ] निपा \textendash\ तानां चेति~॥\\

कैयटे [ प्र० ११ प० ] संपद्यते$=$वाच्यो भवति~॥ एतेन {\qt प्रस्तूयते इति शाब्दः, ४अयमार्थः} इत्यपास्तम्~॥ तमिति~। तत्रैव शब्दानुशासनपदार्थ प्रतिपादयन्निति शेषः~॥\\

\noindent
\rule{1\linewidth}{0.5pt}\\

१ स्ववाक्यमिति~। {\qt अथ शब्दानुशासनम्} इति स्वग्रन्थस्य {\qt अथेत्ययं \textendash\ } इत्यादिना स्वयमेव व्याख्यानं कुर्वन्ति भाष्यकाराः~। एतच्च नागोजीभट्टैरप्यनुमन्यते~। शब्दकौस्तुभे च {\qt व्याकरणस्य विषयं भगवान् भाष्यकारः प्रादर्शयत् \textendash\ अथ शब्दानुशासनमिति} इत्युक्तम्~। अन्येषामपि वैय्याकरणवृद्धानामत्रार्थ एव सम्मतिः~। यदि च {\qt अथ शब्दानुशासनं} इति वचनं भाष्यकारीयं नेत्यभिमन्येत तदा {\qt अथे \textendash\ त्ययं} इत्यतो भाष्यारम्भः स्यात्~। तत्रत्यस्याथशब्दस्येतिशब्दपर \textendash\ त्वेन स्वरूपमात्रार्थकत्वान्न मङ्गलार्थत्वमिति न्यूनता स्यात्~। अतोऽपि भाष्यवचनमेवैतदिति निश्चीयते~। एवञ्च केनचित् राजलक्ष्मीसंज्ञकं टिप्पणमारचयता परेःतु शास्त्रारम्भप्रयोजनसूचकं भगवतः काल्यायनस्य

\columnbreak

त्वेनास्य प्रयोजनमित्यर्थः~। निपातानां च द्योतकत्वं वाक्यपदीयेनिर्णीतम्~॥ अथशब्दस्याधिकारार्थत्वे यो वाक्यार्थः संपद्यते तं दर्शयति \textendash\ शब्दानुशासनमिति~। अनेकक्रियाविषयस्यापि शब्दानुशासनस्य आरभ्यमाणता {\qt अथ}शब्दसन्निधाने प्रतीयते~। व्याकरणस्य चेदमन्वर्थ नाम \textendash\ {\qt शब्दानुशासनं} इति~। अत्र चाचार्यस्य कर्तुः प्रयोजनाभावादनुपादानादुभयप्राप्त्यभावात् न {\qt उभयप्राप्तौ कर्मणि} इत्यनेन षष्ठी, अपि तु {\qt कर्तृकर्मणोः कृति} इत्यनेनेति {\qt कर्मणि च} इति समासप्रतिषेधाप्रसङ्गात् इध्मप्रव्रश्चनादिवत्समासः~॥\\

( उद्द्योतः ) यद्यपि पाणिनिना {\qt ब्राह्मणेन निष्कारणः} इत्यादिश्रुतेः संध्योपासनादाविव प्रत्यवायपरिहारार्थितयैवोत्तमाधिकारिप्रवृत्तिसंभवात् प्रयोजनं नोक्तम्~। वार्तिककृताऽपि {\qt शास्त्रपूर्वके प्रयोगे धर्मः} इत्यनेन मध्यमाधिकारिणः परमप्रयोजनं दर्शितम्~। ३भगवांस्तु विवरणकारत्वान्मन्दाधिकारिणां प्रवृत्तः प्ररोचकप्रयोजन \textendash\ प्रतिपत्तिप्रवणत्वात् व्यवहितेषु स्वर्गापूर्वादिष्वनाश्वासात् साक्षात्प्रयो \textendash\ \\

\noindent
\rule{1\linewidth}{0.5pt}\\

कैयटे \textendash\ [ प्र० १३ प० ] भ्यमाणतेति~। {\qt वर्तमानसामीप्ये} इति भूते लट्~॥ स्वरीत्याssह \textendash\ सन्नीति~। तत्रैव~।\\

कैयटे [ प्र० १५ प० ] प्रयोजनेत्ति~। प्रयुज्यते येन तत्तथा, प्रयोगसाधकफलभावादित्यर्थः~। नन्विन्द्रादिव्यावृत्तिः फलमत आह \textendash\ तत्रैव \textendash\ अनुपेति~॥ अनुक्तेश्चेत्यर्थः~। सत्यपि फले {\qt शब्दा \textendash\ नामिदमनुशासनं नार्थानाम्} इत्यर्थनिवृत्तौ तात्पर्यात्सतोऽपि कर्तुर \textendash\ विवक्षितत्वेन ऽतत्संभवादिति यावत्~॥\\

 [ उ० २ यपङ्क्तौ ] श्रुतेरिति~। वेदाध्ययनवद्व्याकरणाध्ययने नित्यतया प्राप्ते इति शेषः~॥ एवेन प्रयोजनप्रतिपत्तिनिरासः~। तदुक्तं भट्टैः \textendash\ 

\begin{quote}
{\qt तस्माद्विज्ञायमानत्वान्नोक्तं शास्त्रकृता स्वयम्~।\\
शास्त्रेण सर्वशब्दानामन्वाख्यानप्रयोजनम्~॥} इति~॥
\end{quote}

 [ उ० ३ यप० ] अपिरत्वर्थे~। परमेति~। पार्यन्तिकेत्यर्थः~॥ परम्परया प्रयोजनमिति यावत्~॥\\

 [ उ० ५मप० ] समाधत्ते \textendash\ तथापीति~॥ भगवान् \textendash\ भाष्यकारः~॥ ननु ताभ्यामनुक्तं किमित्युच्यतेऽत आह \textendash\ विवरणेति~॥ प्रवृत्तेरिति~। ननु न फलकथनं प्रव्र्त्तावङ्गम्, किं तु तद्वत्ताज्ञानम्~।\\

\noindent
\rule{1\linewidth}{0.5pt}\\

\noindent
वार्तिकमेतत्~। भाष्यकारस्यैतव्द्याकरणप्रयोजनप्रदर्शनवाक्यं \textendash\ इति वदन्तौ कैयटनागेशौ तु भ्रान्तावेवेतीत्याहुः इति वदता अत्रार्थे प्रमा \textendash\ णोपन्यासमकुर्वता च {\qt घटं भिन्द्यात् \textendash\ } इति न्याय एवावलम्बितः \textendash\ ~। अन्यथाऽखण्डगुरुपरम्पराप्राप्तभाष्यतत्त्वानां वैय्याकरणवृद्धानां निरा म्बमधिक्षेपं कः प्रामाणिकः कुर्यात् \textendash\ इति~॥

२ द्योतकत्वेनाथशब्दस्य प्रयो \textendash\ ? इति अ. पुस्तके पाठान्तरम्~। 

३ {\qt तथापि भगवान् विवरणः} इति च. पाठः~। 

४ अयं \textendash\ भाष्योक्तः {\qt शब्दानुशासनं नाम} इत्यादिः, आर्थः \textendash\ तात्पर्यार्यः~। न तु वाच्य इत्याशयः~।

५ उभयप्राप्त्यभावसंभवादिल्यर्थः~। ( र. ना. ) 
\end{multicols}

\newpage
% ६ उद्द्योतपरिवृतप्रदीपप्रकाशितमहाभाष्ये \textendash\ [ १अ.१पा.१आह्निके

\begin{multicols}{2}
\noindent
जनं शब्दव्युत्पत्तिलक्षणं वदतीत्याह \textendash\ भाष्यकार इति~। तत्र शब्दज्ञानरूपे प्रयोजन उक्ते विषयोsप्युक्त एव, शास्त्रजन्यज्ञानवि \textendash\ षयस्यैव शास्त्रविषयत्वात् \textendash\ इति वोध्यम्~। तत्जिज्ञासुरधिकारी, प्रतिपाद्यप्रतिपादकभावः संबन्ध इति \textendash\ अर्थादेव लब्धम्~॥ ननु कानि

\noindent
\rule{1\linewidth}{0.5pt}\\

\noindent
तद्द्ध्याचार्यप्रवर्तनेनानुमित्यात्मकं सुलभमिति वेदाध्ययनादौ माणवकस्येवात्र प्रवृत्तिः सूपपादा~। तदुक्तम् \textendash\ 

\begin{quote}
{\qt १हितकारिभिरेवासौ ज्ञायमानः प्रयोजकः~।\\
कर्त्राऽविज्ञायमानोऽपि नैव स्यादप्रयोजकः~॥} इति~॥
\end{quote}

किंचाचार्यो यद्याप्तस्तहिं तत्प्रवर्तनयैव २तदव्यभिचारेण सा~। अन्यथा नाप्तस्तर्हि शतशः कथ्यमानेऽपि तस्मिन्विप्रलम्भकत्वाशङ्कया न सा~। तस्माद्यत्र३तत् व्यक्तम्, तत्र सा लोकानाम्~। नन्विदमेवावश्यं वाच्यं ग्रन्थादाविति चेत्~। भाष्य आप्तत्वफलवत्त्वयोः सामान्यतो ज्ञानेsपि प्रवृत्तिविशेषे तद्विशेषज्ञानमेवाङ्गम्, {\qt सामान्ये सामान्यं विशेषे विशेषः प्रयोजकः} इति न्यायात्~। न हि किंचिदस्तीति अन्यजिज्ञासुरन्यत्र प्रवर्तते~। अत एव सर्वत्र तद्वदित्येव नोच्यते किंतु तद्विशेष एव कथ्यते \textendash\ {\qt अथातो धर्मजिज्ञासा} इत्यादिना~॥ एवं चात्र नियमेन प्रवृत्त्यर्थमस्यासाधारणं तद्वाच्यमेव~॥ एतेन {\qt विषयं प्रादर्शयत्} इति कौस्र्तुभाद्युक्तमपास्तमिति दिक्~॥ तदेतद् ध्वनयन्नाह \textendash\ प्ररोचकेति~॥ प्रवणत्वात् \textendash\ तदधीनत्वात्~॥ ननु विश्वजिन्न्यायेन स्वर्ग एव फलमास्तामत आह \textendash\ व्यवहितेष्विति~। व्यवहितफलेष्वित्यर्थः~॥ क्वचित्तथैव पाठः~॥ वस्तुत आद्य एव पाठः~। स्वर्गे तत्त्वस्य सत्त्वात्~। अपूर्वे तत्साधनत्वात्तत्त्वं बोध्यम्~॥\\

न्यूनतो परिहरति \textendash\ [ उ० ७ मप० ] तत्रेति~। प्रयोजनादीनां मध्ये भाष्यवाक्ये वेत्यर्थः~। तथा च त्रिमुनिसंग्रहात्मकव्याकरणस्योच्यमानमिदं प्रयोजनं स्वकृतिमपि गोचरयति {\qt स्वाध्यायोऽध्येतव्यः} इतिवत्~॥ एतेन अस्वातन्त्र्यात् तस्य तदुक्त्या स्वस्यापि तद्वत्तासिद्धिरिति नारायणोक्तमपास्तम्~। इष्ट्यादिकथनेन स्वातन्त्र्यात्~॥

\noindent
\rule{1\linewidth}{0.5pt}\\

१ असौ प्रयोजनपदार्थों हितकारिगि शास्त्रकारैर्ज्ञायमान \textendash\ एवास्म \textendash\ दादिप्रवृत्तिः प्रयोजको भवति~। अध्ययनकर्त्राऽविज्ञायमानोऽपि तस्या अप्रयोजको न भवतीत्यर्थः~। ( र. ना. ) 

२ तदेति~। फलवत्ताया अव्यभिचारेण सा \textendash\ प्रवृत्तिरित्यर्थः~। 

३ आप्तत्वं तत्पदार्थः~। ( र. ना. ) 

४ तदाद्युभयेति पाठो भाति~। आनन्तर्यतदादिमङ्गलेत्येतदुभयार्थ \textendash\ कत्वनिरासायेत्यर्थः~। मङ्गले आनन्तर्यादित्वं च मङ्गलानन्तरारम्भेत्यादिकोशप्रसिद्धपाठापेक्षया~। तेषामसंभवादित्यस्य आरम्भातिरिक्तानामसंभवादित्यर्थः~। इति यथाकथंचिद्योजना~। वस्तुतस्तु प्रमाद एवायम्~। ( र.ना. ) ~। वस्तुतस्तु प्रमाद एवायं इति लेखनं र. ना. पण्डितानां छायाग्रन्थानवबोधमूलकमेव~। तथा हि \textendash\ {\qt आनन्तर्याद्यर्थक \textendash\ त्वनिरासाय}इत्युद्योतव्याख्यावसरे छायाकारेण आदिपदग्राह्या: मङ्गलानन्तरा \textendash\ इति कोशोक्ता इति निश्चित्य {\qt र्थकत्वेति}इति प्रतीकमुपादाय तस्य {\qt तदुभेत्यर्थः} इति व्याख्यानं कृतम्~। अथ शब्द \textendash\ व्याख्यानेन निरासश्च प्रश्नानन्तरकार्त्स्यानामेव कर्तव्यः, न मङ्गला \textendash\ रम्भयोः~। तत्रोह्द्योते आदिपदग्राह्याः के इति जिज्ञासायां तदुभेत्यर्थः

\columnbreak

\noindent
पुनर् इत्यादिना प्रयोजनानि वक्ष्यति, तेन पौनरुत्त्यमत आह \textendash\ साक्षादिति~॥ तदेवाह \textendash\ प्रयोजनप्रयोजनानीति~। व्याख्यातुमित्यस्य शब्दानुशासनमित्यादिना \textendash\ इति शेषः~॥ तावत्$=$प्रथमम्~॥ व्याचष्टे इति~। पदार्थज्ञानायानन्तर्याद्यर्थकत्वनिरासाय चेति शेषः~।

\noindent
\rule{1\linewidth}{0.5pt}\\

विषयत्वादिति~। घटकत्वाच्चेत्यपि बोध्यम्~॥ इति \textendash\ इति तु~॥ साक्षादितीति~। अव्यवधानेन ज्ञायमानत्वादिति भावः~॥\\

 [ उ० १२ प० ] तदेवाहेति~। तद्विशेषणकृत्यमेवाहेत्यर्थः~। प्रंयोजनेति~। तस्यापि किं प्रयोजनमिति जिज्ञासायामित्यादिः~॥ पदार्थेति~। पदार्थधियो वाक्यार्थधीहेतुत्वात्~॥ नन्वानन्तर्यमर्थः प्रसिद्ध एवात आह \textendash\ आनन्तर्याद्येति~। आदिना {\qt मङ्गलानन्तरारम्भ \textendash\ प्रश्नकार्त्स्न्येष्वथो अथ} इति कोशोक्तपरिग्रहः~॥ र्थकत्वेति~। ४तदु \textendash\ भेत्यर्धः~। तेषामसंभवात्~। तथा हि \textendash\ कोट्यन्तरानुक्तेः प्रश्नार्थो न~। अर्थानुशासनस्यापि सत्त्वेन बाधात् सर्वार्थो न~। {\qt सर्वेषां शब्दानाम्} इति व्याख्याने {\qt केषां शब्दानाम्} इत्यवतरणग्रन्थासंगतिः~। प्राक्कस्यचिदप्रकृतत्वेनासंभवादानन्तर्यार्थत्वमपि न, प्रागुक्तानन्त \textendash\ र्यपरत्वस्य तत्र तत्र संभवेऽपि अनुक्तानन्तर्यपरत्वेऽव्यवस्थापत्तेः~। विषयादिकथनेनैव तद्विषयजिज्ञासानन्तर्य शास्त्राध्ययनादेर्न्यायसिद्धं सूचितमेवेति तदुक्त्यानर्थक्याच्चेति~॥ ननु धर्मजिज्ञासायां वेदाध्य \textendash\ यनानन्तर्यवदत्र कुतो नेति चेन्न~। वैषम्यात्~। तत्र तथात्वे एका \textendash\ धिकारिकल्पनालाघवं चातुर्वर्ण्याधिकारनिवृत्तिश्च प्रयोजनम्~। नैव \textendash\ मत्र, व्याकरणस्य वेदाङ्गत्वश्रुतेः प्रधानाधिकारिण एवाङ्गेऽप्यधिकारात्~। नापि ऽप्रमाणम्~। तथा हि \textendash\ विना वेदाध्ययनं तदर्थविचारो न संभवतीति तदानन्तर्यमाक्षिप्यते, अतादृशस्यार्थानवधारणे तत्र जिज्ञासानुदयात्; नैवमत्र, अनधीतवेदस्यापि लौकिकवैदिकप्रयोगाङ्ग \textendash\ भूतसाधुत्वज्ञानायैतदध्ययनसंभवात्~। तथा च वक्ष्यति \textendash\ पुरा कल्प इत्यादि~॥ अत एव चात्रानन्तर्यकल्पना विरुद्धा~॥ किं च वेदाध्यय \textendash\ नसमकालताsपि स्मर्यते \textendash\ अनध्यायेप्वङ्गान्यधीयीत इति~।

\noindent
\rule{1\linewidth}{0.5pt}\\

इति व्याख्यानेन तान् परिगणयति ग्रन्थकारः~। निरासश्च अनन्तरार्थस्य आदिपदग्राह्य \textendash\ उभयोः \textendash\ प्रश्चकार्त्स्न्ययोश्च कर्तव्य इति तत्तात्पर्यम्~। निरासे हेतुश्च तेषामसम्भवात् इति~। तेषां \textendash\ प्रश्नानन्तरकार्त्स्न्यानामसम्भवादिति हि तदर्थः~। तेषामसम्भवे कारणमप्युक्तं {\qt तथा हि} इत्यादिना~। एतेनोद्योताभिप्रायप्रकाशकस्यास्य ग्रन्थस्य {\qt प्रमादोऽयं} इति वदन्तः परास्ताः~। टिप्पण्यां तदाद्युभयः इति पाठविपर्यासमाश्रित्य मङ्गलार्थग्रहणक्लेशश्च विफलो विरुद्धश्च, स्वरूपतो मङ्गलार्थत्वमथशब्दस्य सर्वेषामभिमतमेव~। अत एवाग्रे दध्यादिवन्मङ्गलत्वमपीत्युभयार्थमथशब्दः प्रयुज्यते इत्युक्तमु्ष्योते~। {\qt तेषामसम्भवात्} इत्यस्य व्याख्यानं \textendash\ आरम्भातिरिक्तानामसम्भवात् \textendash\ इति तु तुच्छमेव, छायायां येषामसम्भवस्तेषां तथाहीत्यादिना सुस्पष्टं प्रतिपादनात्~। मङ्गलरूपार्थस्यानिरासाच्च~। एवञ्चानधिकारचर्चैवेयं र.ना .महाशयानामिति प्रतिभाति~॥\\

५ प्रमाणं \textendash\ अर्थापत्तिरूपम्, तस्या अभावमुपपादयति \textendash\ तथा हीत्यादिना~।
\end{multicols}

\newpage
% शास्त्रप्रयोजनाधिकरणम् ] महाभाष्यप्रदीपोद्द्योतव्याख्या छाया~। ७

\begin{multicols}{2}
\noindent
भाष्यत्वादेव च स्ववाक्यव्याख्यानमिति भावः~॥ नन्विति विना \textendash\ ऽप्यथशब्दोsधिकारार्थ इत्यैवोच्यतामत आह \textendash\ एवं हीति~। पदा \textendash\ न्तरैः$=$शब्दपद \textendash\ अधिकारार्थदप \textendash\ प्रयुज्यत इतिपदैः~॥ व्याख्या \textendash\ तुमिति~। अधिकारार्थत्वेनेति शेषः~॥ {\qt अथेतिशब्दोऽधिकारार्थः} इत्युक्ते आनन्तर्याद्यर्थकत्वमप्रमाणमिति भ्राम्येदतः {\qt अयं} इति भाष्ये~। योऽयमस्मिन् वाक्ये स इत्यर्थः~॥ विस्पष्टेति~। {\qt नवेति \textendash\ } इत्यादावर्थपरताया दर्शनेन तथा भ्रमं निवर्तयितुमित्यर्थः~॥ ननु निपातानां द्योतकत्वादभिधेयवाच्यर्थशब्दोsनुपपन्नोऽत आह \textendash\ अधिकार इति~॥ प्रस्ताव इति~। प्रारम्भ इत्यर्थः~॥ स चार्थाद्वयाख्यायमानर्य~। अर्थ \textendash\

\noindent
\rule{1\linewidth}{0.5pt}\\

\noindent
अत उर्ध्वमित्यादिमनुश्च~॥ अन्यत्रापि प्रधानमध्येऽङ्गानामनु \textendash\ ष्ठानं दृश्यते \textendash\ इति~॥\\

 [ उ० १६ प० ] एवेन {\qt अयम्} इत्यस्य निरासः~॥ एवं हीतीति~। यत एवं सतीत्यर्थः~। १ किंचानन्तरं कश्चिच्छब्द इत्येव प्रतीयेत~। किं च सर्वत्र स तदर्थं एवेत्यैव प्रतीयेत, तत्तु न, उक्तहेतोः~॥ तत्तु {\qt अथशब्द इत्युक्तावन्योऽयमेव वा} \textendash\ इति संदेहापत्तिः~। तन्न, इतेरभावेऽर्थपरत्वेन स्वरूपपरत्वे चान्योपस्थापकाभावेन च संदे \textendash\ हानुदयात्~। इति विना समासोऽपि दुर्लभः, तदेतद् ध्वनयन् {\qt स्वरूपेऽवस्थानस्य प्रयोजनान्तरमाह \textendash\ २स्वरूपं इति} इति विवर \textendash\ णोक्त्यसांगत्यं प्रतिपादयंश्चाह \textendash\ अथेतीति~॥ अथेत्यधिकारार्थ इत्युक्तेऽप्येवं बाधो व्याख्यानासंगतिश्च, अन्यत्रोक्तार्थप्रतीतेरिति भावः~। अत एव {\qt तदवयवम्} इति प्रागुक्तं कैयटे~॥ \\

 [ उ० १९ प० ] योऽयमिति~॥ शास्त्रारम्भप्रतिज्ञापरवाक्ये उपात्त इतिना स्वरूपेऽवस्थापितः, स इतीत्यर्थः~। शब्दपदादिसामानाधिकरण्यात्तस्य पुंस्त्वम्~॥ एतेन स्वरूपपरत्वान्नपुंसकत्वमुचितभित्यपास्तमं~॥ एतेन कैयटे [ ७प० ] चस्त्वर्थं उक्तफलार्थो न तदर्थ इति सूचितम्~॥ [ २० प० ] परताया इति~। इतिसमभिव्याहृतस्येति शेषः~॥ लोकेऽर्थपरत्वस्यौत्सर्गिकत्वेन दृष्टान्तवैषम्यादाह \textendash\ भ्रममिति~॥\\

 [ उ० २३प० ] स्तुतिभ्रमनिरासाय तदर्थमाह \textendash\ प्रारम्भ इति~॥ मानस्येति~। अथशब्दानुकासनमित्युक्तेः~॥\\

सिद्धान्ताविरोधाय प्रकृतानुरोधेन चो यव्युत्क्रमे \textendash\ इत्याह \textendash\ [ उ० २४ प० ] निपातानामिति~॥ उपसंर्गभिन्नानामित्यर्थः~॥ चेन वाचकत्वसमुच्चयः~। तदाह \textendash\ उपेत्यादिना~॥ एतेनक्रियान्वयिनामेव तेषां ३तत्त्वस्य भ्वादिसूत्रे भगवता व्यवस्थापितत्वेन सर्वेषां ४तत्त्वे मानाभावेनात्र ५तत्त्वे दोषाभावेन वाचकपदासमभिव्याहारे तत्त्वस्य दुर्वचत्वेनार्थस्य द्योत्यत्वादिभेदेन त्रैविध्येन तदसमभिव्याहारे तथा व्याख्यानानौच्त्येन च चिन्त्योऽयं कैयट इति रत्नप्रकाशोक्तमपास्तम्~। उक्तहेतोः~। अनन्यलभ्य इति न्यायेन तत्त्वस्यायुक्तत्वात्~। वाचकस्याध्याह्रियमाणादेः सत्त्वात्~। सिद्धान्तस्य वक्ष्यमाणत्वाच्चेति दिक्~॥ तद्भेदनिवेशफल \textendash\ माह \textendash\ [ उ० २६ प० ] उपेति~॥ [ उ० २८ प० ] अत एव \textendash\ तेषु

\noindent
\rule{1\linewidth}{0.5pt}\\

१ भाष्ये इति \textendash\ अयम् इत्यनयोः शब्दयोरभावेsथशब्दस्यान \textendash\ न्तर्यार्थकत्वेन योऽर्थः स्यात्तमाह \textendash\ किञ्चेति~॥\\

२ प्रदीपऽस्य प्रतीकस्य सत्ताऽवगम्या~। ( दाधिमथाः ) 

\columnbreak

\noindent
शब्दः प्रयोजनवाचीति भावः~॥ निपातानां च द्योतकत्वमिति~। निपातानां द्योतकत्वं चेत्यर्थः~। उपसर्गातिरिक्तनिपातानांद्योतकत्वा \textendash\ चकत्वोभयस्वीकारस्य {\qt अव्ययं विभक्ति \textendash\ } इतिसूत्रे भाष्ये स्पष्टत्वात्~। उपसर्गाणां तु द्योतकत्वमेवेति स्पष्टं गतिर्गतौ इति सूत्रे भाष्ये \textendash\ इति तत्रैव निरूपयिष्यामः~। अत एव साक्षात्क्रियते गुरुः इत्यादि संगच्छते~॥ प्रारम्भक्रियाविषयत्वद्योतकस्यापि अथशब्दस्यान्यार्थ नीयमानदध्यादिवत् मङ्गलत्वमपीत्युभयार्थमथशब्दः प्रयुज्यत इति फलितम्~॥

\noindent
\rule{1\linewidth}{0.5pt}\\

तादृशेषूभयसत्त्वेन वाचकत्वाङ्गीकारादैव~॥ अत्रत्यं तत्त्वं कौस्तुभे मत्कृततद्व्याख्याने च स्पष्टम्~॥\\

ननु मङ्गलार्थत्वस्यापि संभवादधिकारार्थ इत्ययुक्तमत आह \textendash\ प्रयुज्यत इति भाष्ये~॥ अनेन तत्सिद्धिः~॥ तत्फलितमादृ \textendash\ [ उ० २९ प० ] प्रारम्भेति~। भूतेत्यादिः~॥ अनेनाधिकारादिपदानां तादृशकर्मप्रत्ययान्तत्वं सूचितम्~॥ अत एव [ भाष्ये ] {\qt अधिकृतम्}इत्युक्तिसंगतिः~॥ [ उ० ३० प० ] \textendash\ मङ्गलत्वमिति~॥ स्वरूपेणैवेति भावः~॥ अर्थतस्त्वधिकारार्थता, तदाह \textendash\ पीत्युभयेति~॥ अथशब्द इति~। अन्यथा {\qt अधिकृतं शब्दानुशासनम्} इत्येव वदेत्~। अर्थतो मङ्गलत्वे तत्त्वं न स्यात्, सकृत्प्रयुक्तत्वात्~। तथा च श्रुत्या मङ्गलार्थप्रयुक्तस्य तस्य व्याख्यानमुचितमेवेत्येवमुक्तमिति भावः~॥ ननूक्तभाष्यादुभयसिद्धौ वाचकत्वमेवात्र कुतो न, ततश्र बाधकसत्त्वे एव द्योतकत्वाङ्गीकारेण प्रकृते तदभावात्~। साक्षात्क्रियते \textendash\ इत्यादौ तु वाधकमिति न तत्त्वम्~॥ न च प्रमाणानां सामान्ये पक्षपातेन निपातत्वेनैव द्योतकता, प्रमाणस्य सामान्यावच्छेदेन प्रवृत्तौ बाधकाभावसहकृतस्यैव लाघवस्य बीजत्वेन प्रकृते वाचक \textendash\ पदसमभिव्याहाराभावरूपबाधकस्य सत्त्वेनाथशब्दभिन्ननिपातत्वेन तस्या विशेष्यीकरणेनादोषात्~। परेषां नामार्थयोरितिव्युत्पत्तौ तन्नि \textendash\ वेशवत्~॥ एवमन्यत्रापि वोध्यम्~॥ अन्यादृशद्योतकत्वस्य चागतावङ्गीकारात्~॥ यत्तु \textendash\ 

\begin{quote}
{\qt चतुर्विधे पदे चात्र द्विविधस्यार्थनिर्णयः~।\\
क्रियते संशयोत्पत्तेर्नोपसर्गनिपातयोः~॥

तयोरर्थाभिधाने हि व्यापारो नैव विद्यते~।\\
यदर्थद्योतकौ तौ तु वाचकः स विचार्यते~॥}
\end{quote}

इति भद्टैरक्तम्~। तन्निपातांशे बाहुल्याभिप्रायम्, न तु सार्व \textendash\ त्रिकम्~॥ वाक्यपदीयादौ निपातानां द्योतकत्वकथनमप्येवमेव~॥ न चार्धजरतीयता, द्योतकत्वस्यापि तु स्वादावसत्त्वेन द्योतकत्वमद्यो \textendash\ तकत्वमित्येवं तत्त्वस्य स्वमतेऽपि सत्त्वात्~। अथेत्यस्यैवानन्तर्यीदि \textendash\ वाचित्वं कचित्प्रारम्भद्योतकत्वमिति तत्त्वाच्च~॥ न च तस्य वाच्यत्वे गौरवेणान्यलभ्यत्वेन लाघवाद्द्योतकत्वमेव युक्तम्, अन्यस्य वाच \textendash\ कस्योपस्थापकाभावेन तदभावात्~। प्रत्युत वाचकत्वद्योतकत्वोभय \textendash\ कल्पनजगौरवाच्च~॥ न चान्यायश्चानेकार्थत्वम्, उक्तकोशात्त \textendash\

\noindent
\rule{1\linewidth}{0.5pt}\\

३ द्योतकत्वस्य~। ( र. ना. ) ४ द्योतकत्वे ( र . ना. ) \\

५ वाचकत्वे~। ( र. ना. ) ६ द्योतकत्वस्य~। ( र. ना. ) 
\end{multicols}

\newpage

% ८ उद्द्योतपरिवृप्रदीपप्रकाशितमहाभाष्ये [ १अ.१पा.१आह्निके

\begin{multicols}{2}
परे त्वत्र वाचकत्वमेवेच्छन्ति~। {\qt अधिकृतं} इति च तदर्थं वदन्ति~। भाष्ये {\qt प्रयुज्यते} इत्यनेन वर्तमाननिर्देशेन सर्वलोक \textendash\ सिद्धमस्याधिकारार्थत्वं, न तु बृद्ध्यादिवत्पारिभाषिकमिति ध्वनितम्~॥ ननु१विवरणे {\qt अधिकृतं} इत्यधिकं वाचकपदाभावात् द्योतकतानु \textendash\ पपत्तिश्च \textendash\ इत्यत आह \textendash\ अनेकेति~। २उच्यते \textendash\ श्रूयते \textendash\ क्रियते \textendash\ इति~॥ प्रतीयत इति~। एतावतैव द्योतकत्वमिति भावः~॥ ननु भावसाधनस्यानुशासनपदस्य शास्त्रपदेन सामानाधिकरण्यमनुपपन्नमत आह \textendash\ व्याकरणस्य चेति~। अन्वर्थत्वात्प्रयोजनाभिधानोपपत्तिः~। अनुशि \textendash\ ष्यन्तेः$=$असाधुशब्देभ्यो विविच्य ज्ञाप्यन्तेsनेनेति करणल्युडन्ततया शास्त्रपदेन सामानाधिकरण्यमिति भावः~। इदमेव ध्वनयितुं भाष्ये

\noindent
\rule{1\linewidth}{0.5pt}\\

\noindent
स्यैव न्याय्यत्वात्~॥ न च द्योतकत्वेनैव कोशोपपत्तिः, भवेदेवमौ \textendash\ पसंदानिकशक्तिद्योतकत्वमिति पक्षे, नह्यत्रोपसंदानम्~। प्रादेस्तात्प \textendash\ र्यग्राहकत्वमप्युपसंदानेनैव~। न च प्रकरणादिवत्तात्पर्यग्राहकत्वमेवद्योतकत्वम्, तस्यागतिकगतित्वस्योक्तत्वात्~। किंचैवं सति \textendash\ तद्वद \textendash\ त्रापि कोशो न स्यात्, स्याद्वाऽत्रेव तत्रापि \textendash\ इति कोशानुपपत्ति \textendash\ स्तदवस्थैव~॥ अथान्यलभ्यत्वसहकारेण कोशोsन्यथा नेय इति चेत्, तर्ह्यक्षादिपदानामप्यनेकार्थं शक्तिर्न स्यात्, तत्रापि लाघवादेकत्र शक्तिरन्यत्र लक्षणेति सुवचत्वात्~। अन्यलभ्यत्वाभावस्योक्तत्वाच्च~॥ तस्मादथशब्दस्य प्रारम्भद्योतकत्वं कैयटोक्तं चिन्त्यम्~। किंचाथशब्दो द्योत्यत्वेन बोध्यपर एवास्तु, बाधकाभावात्~। अत एव प्रारम्भं द्योतयन्नथशब्द इति हरदतेनोक्तम्~॥ एवं च प्रयोजनपरत्वमपि तस्य चिन्त्यम् \textendash\ इति चेत् ; अत एव सिद्धान्तमाह \textendash\ [ उ० ३१ प० ] परे त्वत्रेति~। अथशब्दे \textendash\ इत्यर्थः~॥ एवेन द्योतकत्वनिरासः~॥ नन्वेवं तस्य प्रस्तूयत इति साध्यार्थपरत्वेन नामपदानां सिद्धार्थपरत्वहानिरत आह \textendash\ अधीति~॥ यथैवं सति न पूर्वविरोधस्तथोक्तम्~। हिरुगादावेवमेवाङ्गीकाराच्च~। तेषां क्रियासामानाधिकरण्येन क्रियाविशेषकत्वेऽपि तदवाचकत्वाच्च~। अव्ययेषु नामत्वस्यासत्त्वाच्चेति दिक्~॥ पारीति~। मयेदानीं परिभाष्यत इत्यर्थः~॥\\

यत्तु \textendash\ नन्वधिकारार्थोऽप्यथशब्दः किमिति प्रयुज्यतेऽत आह \textendash\ अनेकेति \textendash\ इति ६विवरणकृत्~। तन्न; तत्फलस्य प्रागुक्तत्वात्~। स्वपूर्वविरोधाच्च~। तद्ध्वनयन्नाह \textendash\ [ उ० ३३ प० ] नन्विति~॥ कैयटरीत्याऽऽह \textendash\ वाचकेति~। अस्य मध्यमणिन्यायेन पूर्वत्राप्यन्वयः~॥ इति \textendash\ इत्यादि~॥\\

तद्रीत्यैवाह \textendash\ [ उ०३६प० ] एतेति~॥ वाचकपदासमभिव्याहारेऽपि प्रकरणादिवत्तात्पर्यग्राहकत्वेनैव क्रियाविशेषाक्षेपकत्वरूपं द्योतकत्वं \textendash\ {\qt प्रादेशं विलिखति} इत्यादिवदित्यर्थः~॥ प्रादेरिवोपसंदानेन द्योतकत्वस्यैव एवेन व्यवच्छेदः~॥ यद्यपि लक्षणया तस्यापि संभवः, तथापि

\noindent
\rule{1\linewidth}{0.5pt}\\

१ विवरणे \textendash\ स्वग्रन्थविवरणे \textendash\ भाष्ये~। 

२ अनेकक्रियाविषयत्वं प्रतिपादयति \textendash\ उच्यते \textendash\ इत्यादिना~। 

३ ज. पुस्तके {\qt तत्र चाचार्यस्येति} इति प्रतीकमुपात्तम्~। 

४ अस्तु वा यथाकथंचिदुभयप्राप्तिस्तथापि न क्षतिः, उभयप्राप्ता \textendash\ विति सूत्रेऽविशेषेण विभाषेति पक्षस्यापि वक्ष्यमाणतया नियमाप्रवृत्तिपक्षे आचार्यस्य शब्दानुशासनमिति प्रयोगसंभवात्~। शेषलक्षणा षष्ठी

\columnbreak

\noindent
नामपदोपादानम्~। नामनामिनोरभेदात् {\qt नाम \textendash\ शास्त्रम्} सामा \textendash\ नाधिकरण्यम्~॥ अनेनैवापभ्रंशानामविषयत्वं सूचितं ध्वनीनां च~॥ {\qt केषाम् \textendash\ } इति प्रश्नस्तु \textendash\ लौकिकमात्रविषयं शाकटायनादिशास्त्रम \textendash\ धिकृतम्, उत वैदिकमात्रविषयं प्रातिशाख्यम् \textendash\ इति \textendash\ इति परे~॥ ननु {\qt कर्मणि च} इति निषेधात्कथं कर्मषष्ठ्यन्तेन समासोऽत आह \textendash\ ३अत्र चाचार्यस्येति~। उभयोरुपादान एवोभयप्राप्तिरिति आत्ममाने \textendash\ ? इति सूत्रे भाष्ये स्पष्टम्~॥ {\qt कर्तृकर्मणोः \textendash\ } इति षष्ठ्यामपिसमासनिषेधः कुतो नेत्यत आह \textendash\ इध्मेति~। चस्य \textendash\ इति \textendash\ अर्थकतया कर्मणीत्युच्चार्य विहितषष्ठ्यैव समासनिषेध इति भावः~॥ भाष्ये शास्त्र \textendash\

\noindent
\rule{1\linewidth}{0.5pt}\\

\noindent
व्याख्यानभाष्यविरोधान्न तद्युक्तम्~। सिद्धान्ते तु तस्य सोऽर्थ एवेति प्रागुक्तम्~। भावसाधनेति~। अस्य व्याकरणप्रयोजनाभि \textendash\ धायकत्वादिति भावः~॥ अनुशेति~। किं च शास्त्रपदेन कस्य ग्रहणम् ? किंच रूढिर्योगो योगरूढिर्वेत्यपि बोध्यम्~॥ ननु नाम \textendash\ पदेन रूढेरेव प्रतीतौ {\qt प्रयोजनमाह} इति कथमवतरणम् ? अतआह \textendash\ अन्वर्थेति~। तद्धटकयोगार्थादित्यर्थः~॥ अनुपूर्वकशासे \textendash\ र्विविच्यज्ञापने७दृष्टत्वस्यान्यत्र प्रसिद्धत्वादाह \textendash\ असाध्विति~॥ नन्वेवं नामपदासंगतिरत आह \textendash\ इदमेवेति~। अन्वर्थत्वाद्येवेत्यर्थः~॥ एतेन {\qt नामशास्त्रम् इति समस्तमित्यपास्तम्~। आख्यात \textendash\ शास्त्रत्वस्यापि सत्त्वाच्च~। एतेन नामेति प्रसिद्धौ, शब्दान्वाख्यानार्थकत्वेन प्रसिद्धं} इति नारायणोक्तमपास्तम्~॥ शास्त्रत्वं चास्यानुपदं स्फुटीभविष्यति~॥ [ उ० ४२ प० ] अनेनैवेति~। तदन्वर्थत्वादि \textendash\ बोधकनामशब्दोलेखेनैर्वेत्यर्थः~। तत्त्वस्यात्राभावादिति भावः~॥ नन्वेवं प्रश्नासंगतिरत आह \textendash\ केषामितीति~॥ इतीति~। इतिपूर्वग्रन्थार्थसंशयाशयक इत्यर्थः~। तथा च {\qt लौकिकानामेव, उत वैदिकानामेव} इत्याशयकः प्रश्न इति भावः~॥ कैयटाद्युक्तयसांगत्यध्व \textendash\ ननायाह \textendash\ [ उ०४४शप० ] परे इति~। भाष्यानुयायिन इत्यर्थः~॥\\

 [ उ०४५शप० ] कथमिति~॥ शब्दानुशासनमित्यत्रेति भावः~। एतदेव ध्वनयन्नाह \textendash\ [ उ०४६शप० ] उभयोरिति~। प्राप्तिग्रहणात्, आश्चर्यो गवां दोहोsगोपेन इत्यादौ यथेति भावः~॥ इध्मेतीति~। एवं च तत्रापि प्रसङ्गापत्तेः. इति हेतुस्तत्रार्थं बोध्यः~॥ अपां स्रष्टेत्यादेरपि तेनैव सिद्ध्या {\qt कर्तरि च} इति {\qt कर्मणि च} इति सूत्र इति शेषः~। निपातानामनेकार्थत्वादिति भावः~॥ कर्मणीति~।

\noindent
\rule{1\linewidth}{0.5pt}\\

\noindent
वास्त्विति शब्दकौस्तुभे स्पष्टम्~॥ ( दाधिमथाः ) \\

५ सांक्रामिकी शक्तिरित्यर्थः ( र. ना. ) \\

६ विवरणकृत् \textendash\ प्रदीपकृत्~। \\

७ अनुपूर्वो हि शासिर्विविच्य ज्ञापने दृष्टस्तद्यथा संषूषन् विदु \textendash\ षायन् योऽञ्जसानुशाससीति ( ऋ.सं.६~। ५४~। १ ) पद मञ्जर्यां कौस्तुभे च स्पष्टम्~। ( दाधिमथाः ) 
\end{multicols}

\newpage
% शास्त्रप्रयोजनाधिकरणम् ] महाभाष्यप्रदीपोद्द्योतव्याख्या छाया~। ९

\begin{multicols}{2}
\noindent
मित्यस्य {\qt कर्तव्यत्वेन} इति शेषः~। {\qt विवरणकारत्वात् व्याख्यातव्य \textendash\ त्वेन} इति शेष उचित इत्यन्ये~॥ शास्त्रत्वं चास्य साधुत्वरूपैक \textendash\ प्रयोजनसंबद्धत्वाद्वोध्यम्~॥

\begin{center}
\textbf{ ( आक्षेपभाष्यम् ) }
\end{center}

केषां शब्दानाम् ? 

 ( प्रदीपः ) शब्दशब्दस्य सामान्यशब्दत्वात् विना प्रकरणा \textendash\ दिना विशेषेऽवस्थानाभावात् तन्त्रीशब्दकाकवाशितादीनामनु शासनप्रसङ्ग इति मत्वा पृच्छति \textendash\ केषामिति~। उत्तरपदार्था \textendash\ न्तर्गतस्यापि पूर्वपदार्थस्य बुद्ध्या प्रविभागात् प्रत्यवमर्शः~। यथा राजपुरुष इत्युक्ते कस्य राज्ञः ? इति~॥

\noindent
\rule{1\linewidth}{0.5pt}\\

\noindent
सप्तम्येकवचनान्तमीत्यर्थः~॥ नियमशास्त्राणां विधिरूपेण प्रवृत्ते \textendash\ राह \textendash\ विहितेति~॥ कर्तव्येति~॥ स्वकृतिनिष्ठसाध्यत्वस्य तत्राप्यारोप इति भावः~॥ आरोपाभावकृतलाघवादाह \textendash\ विवरणेति~। इदं च सूत्रादिवत्स्वकृतेरप्यस्तीति भावः~॥ यद्यपि व्युत्पाद्यतया शब्दा एव प्रस्तुताः, न शास्त्रम् ; तथापि तत्प्रतिपादकत्वाच्छास्त्रमेव प्रस्तुतमुच्यते~। शास्त्रत्वं चास्य योगरूढ्या~। पदद्वयादिव्याख्यानेन प्रागुक्तभाष्यलक्षणसत्ता सूचिता~॥ न च सिद्धसाधुशब्दबोधकस्यास्य कथं शास्त्रत्वम्,

\begin{quote}
{\qt प्रवृत्तिर्वा निवृत्तिर्वा नित्येन कृतकेन वा~।\\
पुंसां येनोपदिश्येत तच्छास्त्रमभिधीयते~॥}
\end{quote}

इत्युक्तेरिति वाच्यम्~। तस्यासार्वत्रिकत्वात्~। एकफलसंबद्धार्थस्य कार्त्स्न्येन प्रतिपादकस्यैव हि सर्वत्र शास्त्रपदार्थत्वात्~॥ अन्यथान्यायादीनां सिद्धार्थपराणामतत्त्वापत्तेः~। तदभिप्रेत्याह \textendash\ शास्त्रत्वमिति~॥ प्रवृत्त्याद्युपदेशकत्वमपि सुवचम्~॥ एतद्वोधिताः साधवः, तैरेव भाषितव्यम्, नासाधुभिः? इत्युपदेशादिति बोध्यम्~॥\\

भाष्ये \textendash\ वेदितव्यमिति~। शिष्यैरिति शेषः~॥ यद्यपि {\qt भवति} इत्येवाध्याहार्यम्, तस्य सिद्धार्थपरत्वात्, तथापि यागादौ संकल्पवत् शास्त्रारम्भप्रतिज्ञारूपसंकल्पविशेषबोधनायास्याध्याहारः~॥ अनेन \textendash\ आरब्धं मध्ये दुरत्यजम्, त्यागे च विगानप्रायश्चित्ते \textendash\ इति सूचितम्~॥\\

 [ उ० १मप० ] ननु तपरसूत्रे शब्दानां ध्वनिरिति व्यङ्ग्यव्यञ्जकयोः शब्दशब्दस्य प्रयोगादन्वर्थत्वाच्चानतिप्रसङ्गोऽत आह \textendash\ लोकेति~। त्रैविद्यवृद्धेत्यर्थः~॥ ध्वनाविति~। अनेनापरिस्फुटाकारादिवर्णसकलशब्दपरिग्रहात्काकवाशितादीनां पृथगनुक्तिः~॥ उक्तरीत्या तत्रानतिप्रसङ्ग एवेति सूचनमेवमुक्तेः फलम्~। तथा च शुद्धरूढः शब्दशब्दः~। तपरसूत्रप्रयोगस्तु प्रकृताशयक इति भावः~॥\\

 [ उ० ४र्थप० ] विशेषेषु \textendash\ ब्रीहिविशेषेषु~॥ वाचकत्वस्य सिद्धान्तेऽपभ्रंशसाधारण्यादाह \textendash\ साधाविति~॥ प्रकरणात् \textendash\ यागप्रकरणात्~॥

\noindent
\rule{1\linewidth}{0.5pt}\\

१ अ. पुस्तके {\qt ननु}शब्दो नास्ति~। \\

२ उत्तरपदार्थः संसृष्टो यत्रेति पाठो भाति~। ( र. ना. ) वस्तुतस्तु एतदप्यज्ञानविलसितमेव~। तथा हि \textendash\ उदयोते {\qt ननूत्तरपदार्थसंसृष्ट} \textendash\ पूर्वपदार्थस्य कथं इत्यादिग्रन्थेन {\qt उत्तरपदार्थान्तरगतस्यापि पूर्वपदा \textendash\ र्थस्य इति प्रदीपग्रन्थोsवतार्यते~। अवतरणे संसृष्टपदप्रवेशः प्रदीपे उत्तरपदार्थ} इत्यत्र कर्मधारयं सूचयति~। छायाकारस्तत्रेत्थमाशङ्कते \textendash\ उत्तरः पदार्थो यत्रेति बहुब्रीह्याश्रयणेनापि प्रदीपः सुयोजस्तहिं संसृ \textendash\ \\

२ प्र०पा० 

\columnbreak

 ( उद्द्योतः ) सामान्यशब्दत्वादिति~। लोकव्यवहारे ध्वना \textendash\ वपभ्रंशे च शब्दशब्दप्रयोगादिति भावः~॥ एतेन ध्वनिव्यङ्ग्य एव तपरसू्त्रादौ शब्दशब्दप्रयोगादिदं चिन्त्यमित्यपास्तम्~। ननु {\qt ब्रीहीन् प्रोक्षति} इत्यादौ सामान्यशब्दस्याप्यपूर्वीयेषु विशेषेषु प्रवृत्तिवत् वाचके साधौ तद्विशेषे प्रवृत्तिर्भविष्यतीत्यत आह \textendash\ विनेति~। ब्रीहिशब्दस्य तु प्रकरणाद्विशेषे प्रवृत्तिरिति भावः~। वाशितादीनाम् \textendash\ इति आादिनाऽपभ्रंशसंग्रहः~॥ १ननूत्तरपदार्थसंसृष्टपूर्वपदार्थस्य कथं सर्वनाम्ना परामर्शोऽत आह \textendash\ उत्तरपदार्थान्तर्गतस्येति~। ततः पृथगुपस्थित्यविषयस्येत्यर्थः~॥ बुद्ध्या \textendash\ मानस्या~। यद्वा वाक्यजन्यबुद्ध्या~॥\\

\noindent
\rule{1\linewidth}{0.5pt}\\

 [ उ०६ष्ठप० ] ननु तन्त्र्यादिशब्दानां न तत्त्वप्रसङ्गः, प्रकृत्यादिविभागस्य तत्रासंभवात्~। शकुनागमगान्धर्ववेदादिभिरेव तन्निर्णयाच्च~॥ अत एव कैयटे {\qt मत्वा} इत्युक्तिरत आह \textendash\ वाशितेति~॥ बहुवचनं तु शब्दभेदाभिप्रायम्~॥\\

 [ उ०७मप० ] ननु मिलितैकदेशस्य विशेष्यस्य {\qt सर्वनाम्नां प्रधा \textendash\ नपरामशित्वं} इति नियमाङ्गीकारेण परामर्शसंभवेऽपि गुणीभूतस्य वृत्त्याद्यन्तर्गतस्य परामर्शों नेत्याशयेन शङ्कते \textendash\ ननूत्तरपदेति~॥ प्रकृताभिप्रायमेतत्~। अप्रधानस्य न स इति तत्त्वम्~॥ संसृष्टेति~। तदेकार्थीभावापन्नेत्यर्थः~॥ यद्यप्युत्तरः२ पदार्थो यत्रेति बहुब्रीहिणा यथाश्रुतं सुयोजम्; तथापि पूर्वपदार्थेत्यस्य वैयर्थ्यापत्तिः, नियमेन वृत्त्यलाभेन प्रकृतार्थासिद्धिः, वाक्ये तथा सत्त्वेऽपि तदङ्गीकारश्चेति तस्य दुर्वचत्वेन तथा वाच्ये तदन्तर्गतत्वं दुर्वचमत आह \textendash\ [ उ०८प० ] तत इति~। उत्तरपदार्थादित्यर्थः~॥ समुदितादुभयोरेवैवोपस्थितिरिति भावः~॥ तदीयशाब्दबुद्ध्याऽतत्त्वादाह \textendash\ मानस्येति~॥ ननु नियमेन तत्सत्वे न मानमत आह \textendash\ यद्वेति~॥ वाक्येति~। लौकिकं त्वार्थमित्युक्तम्~। न त्वश्वकर्णादिवच्छुद्धरुढम्~। येनाविग्रहत्वं स्यात्~॥ अस्याश्च नियमेन तत्र सत्त्वम्~। भाष्ये {\qt शब्दानाम्} इति तु तत्परामृष्टार्थस्य प्रदर्शनं स्पष्टत्वाय, न स्वतन्त्रम्~। अत एवोत्तैरे३ तदनुक्तिः~। अत एव {\qt दशैते राजमातङ्गास्तस्यैवामी तुरङ्गमाः} इत्यादौ न दोषः~॥ यद्वा \textendash\ अयं दोषस्तदा स्यात् यदि {\qt केषाम्} इत्येवोक्तं स्यात्~। तदा हि शब्दानामित्यस्यानुषङ्गः~। स चोक्तरी \textendash\ त्याऽनुपपन्नः~। अत्र तु विशेष्यं शब्दानामिति कण्ठत उच्यते तत्तु तत्परमित्यन्यत्~॥ संनिहितत्वं न बुद्धिस्थत्वात्~। एवं च तथानि \textendash\ र्दिष्टानामेव तेन परामर्शो न तु तत्र गुणीभूतानाम्~। अत एव पुनस्तदुक्तिसाफल्यम्~। दण्डिमिश्रावप्येवम्~॥ एतेन \textendash\ {\qt तस्मिन्नद्रौ} इत्यत्राद्रिपदोक्तिर्व्याख्याता~। अन्यथोक्तरीत्या न स्यात्~। पूर्वमते \textendash\

\noindent
\rule{1\linewidth}{0.5pt}\\

\noindent
ष्टपदप्रवेशेनोद्द्योतकृतां कर्मधारयसूचनं किमर्थमिति~। यदि प्रदीपे बहुब्रीहिराश्रियेत तदा तत्रोपात्तस्य {\qt पूर्वपदार्थस्थ इत्यस्य} वैय्यर्थ्यं स्यादित्युत्तराशयः~। अत्र {\qt उत्तरपदार्थः संसृष्टो यत्र} इत्येतत्पर्यन्तस्य पाठकल्पनकेशानुस्मरणस्य क उपयोगः ? कथञ्च तादृशपाठान्तरकल्पनेन शङ्कासमाधाने \textendash\ इति त एव र. ना. महाशयाः प्रष्टव्याः~।\\

३ उत्तरे \textendash\ लौकिकानां वैदिकानाञ्चेत्यत्र~।\\
\end{multicols}

\fancyhead[RE]{[ १ अ. १ पा. १ पस्पशाह्रिकै}
\fancyhead[LO]{शास्त्रप्रयोजनाधिकरणम् ]}
\fancyhead[LE,RO]{\thepage}
\fancyhead[CE]{उद्द्योतपरिवृतप्रदीपप्रकाशितमहाभाष्यम्~।}
\fancyhead[CO]{महाभाष्यप्रदीपोद्दयोतव्याख्या छाया~।}
\cfoot{}
\newpage
%%%%%%%%%%%%%%%%%%%%%%%%%%%%%%%%%%%%%%%%%%%%%%%%%%%%%
\renewcommand{\thepage}{\devanagarinumeral{page}}
\setcounter{page}{10}

% १० उद्द्योतपरिवृतप्रदीपप्रकाशितमहाभाष्यम् [ १ अ. १ पा. १ पस्पशाह्रिकै

\begin{multicols}{2}
\begin{center}
\textbf{ ( समाधानभाष्यम् ) }
\end{center}

लौकिकानां वैदिकानां च~। तत्र लौकिकास्तावत् \textendash\ गौः \textendash\ अश्वः \textendash\ पुरुषः \textendash\ हस्ती \textendash\ शकुनिः \textendash\ मृगः \textendash\ ब्राह्मण इति~॥\\

वैदिकाः खल्वपि \textendash\ १शन्नो देवीरभिष्टये, इषे त्वोर्जे अग्निमीळे पुरोहितम्, अग्न आयाहि वीतये \textendash\ इति~॥\\

 ( प्रदीपः ) सिद्धान्तवादी तु व्याकरणस्य वेदाङ्गत्वात्साम \textendash\ र्थ्याद्विशेषावगतिरिति मत्वाऽऽह \textendash\ लौकिकानामिति~। लोके

\noindent
\rule{1\linewidth}{0.5pt}\\

\noindent
ऽप्यत्र न दोषस्तुल्यत्वात्~। निर्दयत्वकर्कशत्वादिबोधनेनानाश्रयणी \textendash\ यत्वादिव्यञ्जकतया महाशयत्वाद्यर्थान्तरसंक्रमिततया वा तस्य साफल्याच्च~॥ एवं सर्वत्र बोध्यम्~॥ द्वितीयपक्षे {\qt दशैते इत्यादौ साम \textendash\ र्थ्यात्तस्य परामर्शः आनर्थक्यात्तदङ्गषु} इति न्यायेन {\qt सविशेषेण हि इति न्यायेन च~॥} नच्चैवमपि {\qt शब्दानां} इत्यस्य पदत्वेन वाक्येत्ययुक्तमिति वाच्यम्, {\qt वाक्ये वाक्यैकदेशः} इति न्याये \textendash\ नास्यापि तत्त्वव्यवहारात्~। साकाङ्क्षपदस्थलविशेषे एकपदेऽपि तत्त्वस्य जयदेवोक्तत्वाच्च~॥ अत एव प्रश्नोत्तरयोः सिद्धान्तरीत्याश \textendash\ यप्रदर्शकोक्तिसंगतिः~। तदेतदभिप्रेत्योभयसाधारणमुदाहरति \textendash\ \\

 [ ९ पष्टे ४ प० प्रदीपे ] कैयटे \textendash\ यथेति~॥\\

 [ भा० ] लौकिकानामिति~। न च वेदाङ्गत्वात् {\qt वैदिकानाम्} इति प्राग् वक्तव्यम्~। तज्ज्ञानमेव हि मुख्यं फलम्, लौकिकज्ञानं त्वानुषङ्गिकम्~। न च वैदिकसंप्रदायतस्तत्सिद्धिः, दोषेण तस्याप्य \textendash\ न्यथात्वसंभवात्~। शाखाऽऽनन्त्येन तद्वत् प्रातिशाख्यस्य क्वचिदुत्सा \textendash\ दात् \textendash\ इति वाच्यम्~। तद्व्दादरसूचनाय तथोक्तेः, लौकिकस्य लोकवेदसाधारण \textendash\ लोकमात्रवृत्त्युभयवृत्तित्वेन व्यापकत्वाच्च~। विपरीतं

\noindent
\rule{1\linewidth}{0.5pt}\\

१ शन्नोदेवीरिति~। अत्र वेदारम्भप्रतीकान्येव भाष्ये उक्तानि \textendash\ इति व्याकरणसिद्धान्तसुधानिधिकार आह~। तत्रैतदुक्तम् \textendash\ सर्ववैदिक \textendash\ शब्दानां व्युत्पाद्यत्वलाभाय प्रथमप्रतीकोपादानम्, वैदिकैकदेशमात्र \textendash\ व्युत्पादकप्रातिशाख्योपेक्षयाsस्य व्याकरणस्य वैलक्षण्यसूचनाय च प्रथमप्रतीकोपादानम् \textendash\ इति~॥ अथ कमेण ऋग्यजुःसामाथर्वणां प्रदर्शनमकृत्वा अथर्वयजुर्ऋकसाम्नां {\qt शन्नो} इत्यादिक्रमेण प्रदर्शयतो भगवतः क आशयः ? उच्यते \textendash\ त्रेतासाध्यानुश्राविके कर्मणि अथर्ववेदस्य अभिचारादिनिराससाधनीभूतानां मन्त्रविशेषाणां कर्मविशेषाणाञ्च तत्र सद्भावेनादौ तस्योपयोगात् प्रथममुपस्थितिः, तत उपक्रमप्रभृतिप्रधानाध्वर्युकर्मतन्त्रप्रदर्शकस्य भित्तिस्थानीयस्य यजुर्वेदस्य, ततश्चित्रस्थानीययोऋर्क्सामयोर्मध्ये प्रकृतिभूतदर्शपूर्णमासहोत्रोपयोगिनो ऋग्वेदस्य, ततः सौमायुपयोगिनः सामवेदस्येति उपस्थितिक्रमानुसारेण तथा प्रदर्शनमिति अस्मत्परमगुरुचरणाः श्रीबालशास्त्रिण इति गुरूपदेशादवगम्यते~। वया सि. काराश्च वैदिकोदाहरणप्रतीकचतुष्टये वैदिकमात्रशब्दघटितत्वं {\qt शन्नोदेवीः} इत्यस्यैवेति तस्यैव प्राथम्यम्~। अन्यत्र तु लौकिका एव शब्दा इति~। उचितमप्येतत्~। ननु {\qt अग्निमीळे इत्यत्र ईळे} इति ळकारो वैदिक एव नायं लोके प्रयुज्यते, सत्यम्; ईळे इत्यत्र वर्णमात्रव्यत्ययः, शन्नेत्यत्र तु प्रकृतिप्रत्यययोः कार्यं तथेति तस्योपादानम्~।

\columnbreak

\noindent
विदिता इति {\qt लोकसर्वलोकाभ्यां }इति ठञ्~। अथवा भवार्थे अध्यात्मादित्वाट्ठञ्~। वेदे भवाःवैदिकाः~। २वैदिकानामपि लौकिकत्वे प्राधान्यख्यापनाय पृथगुपादानम्~। यथा \textendash\ ब्राह्मणा आयाता वसिष्ठोsप्यायात इति वसिष्ठस्य~। तेषां तु प्राधान्यं यत्नेनापभ्रंशपरिहारात्~। अथवा भाषाशब्दानामेव लौकिकत्वमिति भेदेन निर्देशः~॥ तत्र लोके पदानुपूर्वीनियमाभावात्पदान्येव दर्शयति \textendash\ गौरश्व इति~॥ वेदे त्वानुपूर्वानियमाद्वाक्यान्युदाहरति \textendash\ शन्न इति~॥

\noindent
\rule{1\linewidth}{0.5pt}\\

\noindent
तु न, अग्रिमग्रन्थविरोधापत्तेः~। अत एव ब्राह्मणवसिष्ठन्यायप्रवृत्ति सूच्वनार्थत्वाच्च~॥ संकोचे मानाभावेन लोकशब्दस्य सर्वलोकवृत्तित्वेन तन्त्र्यसिद्धत्वेन साधूनामेव ग्रहणं नापभ्रंशानां, प्रतिदेशं तेषां भिन्नत्वात् \textendash\ इति सूचनार्थत्वाच्चेति भावः~॥\\

 [ प्र १मप० ] प्रकरणाभावेऽपि संयोगाद्यन्तर्गतसामर्थ्यसत्त्वाद्वि \textendash\ शेषनिश्चय इत्याशयेनाह \textendash\ कैयटे \textendash\ सिद्धान्तेति~॥ [ प्र २ यप० ] कैयटे \textendash\ अत एवाह \textendash\ लोके विदिता इति~। प्रसिद्धा इत्यर्थः~॥ वैदिका इत्यनुरोधेन संभवादाह \textendash\ अथवेति~। अत एवाह तत्र \textendash\ वेदे इति~। एतत्कथने तात्पर्यम्~। \textendash\ अत एव पुनः प्रत्ययानुक्तिः~॥\\

 [ प्र०६मप० ] प्राधान्यमुपपादयति कैयटे \textendash\ तेषां त्विति~। वैदि \textendash\ कानां प्राधान्यं त्वित्यर्थः~॥ बोध्यमित्यग्रे शेषः~॥ वसिष्ठप्राधान्य \textendash\ वैलक्षण्याय \textendash\ {\qt तु}~॥ यत्नेन \textendash\ सर्वथा~। अन्यथा प्रत्यवायापत्तेः~॥\\

 [ प्र० ७मप० ] युक्तयन्तरसंभवेन विनिगमनाविरहादाह \textendash\ अथ \textendash\ वेति कैयटे~॥

\noindent
\rule{1\linewidth}{0.5pt}\\

\noindent
अथ च {\qt अभिष्टये} इति द्वितीयस्यापि शब्दस्य वैदिकमात्रत्वमिति तदेव पूर्वं प्रयुज्यते \textendash\ इति~॥\\

२ {\qt वैदिकानां लौकिकत्वेsपि} इति च छ. क. पाठः~।\\

३ अथ सर्वत्र प्रामाणिकग्रन्थेषु प्रायः ऋग्, यजुः, साम, आथ \textendash\ र्वणमित्येव क्रम उपलभ्यते~। तथा च अग्निमीळे, इषे त्वोर्जे त्वा, अग्न आयाहि, शं नो देवीरभिष्टये, इत्येवं क्रमेणैवोदाहरणानां वक्तव्यत्वे किमिह क्रमत्यागो भाष्यकृता सर्वप्रमाणभूतेन कृतः \textendash\ इति चेत्~। शृणु \textendash\ ऋग्यजुःसामवेदानां पद्यगद्यगीतिबाहुल्येन मल्लग्रामवत्प्राप्तसमाख्यानां नगरवदविशिष्टसमाख्यस्यार्ववेदस्य प्रागभिधाने स्वत्रेच्छानामाचार्याणामिच्छैव विनिगमनाभावात्प्रमाणम्~॥ विशेषजिज्ञासा चेत्, शृणु \textendash\ सर्वेषां वेदानां केवलं यज्ञार्थतया यज्ञापेक्षितक्रमेणैवाध्ययनादिकमुचितम्~। यज्ञेवरणक्रमश्चेत्यमुचितमनुष्ठीयते {\qt ब्रह्मा त्वो वदति जातविद्याम्} ८~। २~। २४~। ५ इति ऋक्संहितामन्त्रव्याख्यावसरे {\qt ब्रह्रैको जाते जाते विद्यां वदति~। ब्रह्मा सर्वविद्यः सर्वं वेदितुमर्हति} ब्रह्मा परिवृढ्ः श्रुततो ब्रह्मा परिवृढः सर्वतः इति निरुक्ते यास्कवचनेन ब्रह्मैव विद्वान् भृग्वङ्गिरोवित् सम्यग \textendash\ धीयानश्चरितब्रह्चर्योऽन्यूनातिरिक्ताङ्गोsप्रमत्तो यज्ञं रक्षति~। तस्य प्रमादाद्यदि वाऽप्यसांनिध्याद्यधा भिन्ना नौरगाधे मह \textendash\
\end{multicols}

\newpage
% शास्त्रप्रयोजनाधिकरणम् ] महाभाष्यप्रदीपोह्व्योतव्याख्या छाया~। ११

\begin{multicols}{2}
 ( उद्द्योतः ) वेदाङ्गत्वादिति~। तत्त्वं चागमसिद्धम्~। अनेन तैस्ततन्त्रीशब्दादीनामविषयता सूचिता~॥ न च वेदाङ्गत्वाद्वैदिका एव विषयाः स्युः~। तत्र तत्र {\qt भाषायां छन्दसि} इत्याद्युक्त्योभयोरपि विषयत्वात्~॥ ननु चिदानन्दस्वरूपब्रह्मातिरिक्तस्य सर्वस्य लोकपदार्थत्वात् वेदस्यापि लोकान्तर्गतत्वात् भेदेनाभिधानं न युज्यत इत्यत आह \textendash\ वैदिकानामिति~। {\qt वैदिकानां लौकिकत्वेऽपि इति पाठः}~॥ त्तेषां त्विति~। लौकिकानां याज्ञे कर्मण्येव १तत्परिहारः, तेषां तु सर्वत्रेति भावः~॥ भाषाशब्दानामिति~। भाषा$=$प्रयोज्यप्रयोजक \textendash\ वृद्धव्यवहारः, तत्र प्रयुज्यमानानामित्यर्थः~। एवञ्च वेदमात्रान्तर्गत \textendash\

\noindent
\rule{1\linewidth}{0.5pt}\\

 [ उ० १ मप० ] चागमेति~। {\qt ब्राह्मणेन} इत्यादीत्यर्थः~॥ ननु नास्येष्टसाधकत्वं लौकिकासंग्राहवत्वादत आहृ \textendash\ अनेनेति~॥ आदिनाऽपभ्रंशसंग्रहः~। तथा चातिप्रसङ्गनिवारणपूर्वकाभिमतैकदेश \textendash\ साधकत्वात् सद्धेतुरेवायम्~। अत एव {\qt मत्वा} इत्युक्तमिति भावः~॥ अत एवाशङ्कते \textendash\ न चेति~॥\\

 [ उ०४र्थप० ] यत्तु कृष्णः \textendash\ {\qt प्रयोगचोदताभावा्यैकत्वमवि \textendash\ भागात्} इत्यनेन य एव लौकिकास्त एव वैदिका इति जैमिन्युक्त \textendash\ त्वाद्भेदोक्तिरयुक्तेति~। तन्न, सखी कर्णैभिरित्यादीनां मिथो व्यावृत्तत्वात~। तदेतद् ध्वनयन्नाह \textendash\ ननु चिदेति~॥ पाठ इति~। अनेन {\qt कत्वेन} इति पाठो निरस्तः~। {\qt कानामपि} इति पाठे अपेस्तत्रान्वयः~। यथाश्रुते नैष्फल्यं ग्रन्थयोरसामञ्जस्यं चेत्यपि बोध्यम्~॥\\

 [ उ०७मप० ] तदाह \textendash\ लौकीति~। अर्थबोधायोच्चारणप्रसङ्गे इति शेषः~॥ यद्वा नित्यसाकाङ्क्षत्वात् {\qt तत्} इत्यत्रैकदेशेऽन्वयः~॥ तस्यानुषङ्गोऽग्रे~॥\\

 [ उ० ९मप ] ननु तत्र सर्वेषां तत्त्वादेव व्यवच्छेद्यासंभवोऽत आह \textendash\ एवञ्चेति~॥ तस्योक्तार्थकत्वे चेत्यर्थः~॥ मात्रपदेन गौरश्व इत्यादीनामपि तत्त्वं ध्वनितम्~। तेन नाग्रिमग्रन्थविरोधः~। तदाह \textendash\ अत एवेति~॥ नन्वेवं शं नः इत्यादीनामपि तत्त्वमुचितमिति तदुक्तिरयुक्ता~। न च {\qt देवीः, इळे} इत्यादिकं छान्दसं

\noindent
\rule{1\linewidth}{0.5pt}\\

\noindent
त्युदके संप्लवेत्~। तस्माद्यजमानो भृग्वङ्गिरोविदभेव ब्रह्माणं वृणुयात्~। स हि यज्ञं तारयति इति ( उत्तः २~। २~। ५ ) {\qt अथ \textendash\ र्वाङ्गिरोभिर्ब्रह्मत्वम्} ( पू० १~। ३~। १~। २ ) इति च गोपथब्राह्मणव \textendash\ चनेन ब्रह्मवरणस्यैवादावावश्यकत्वेन अथर्ववेदाध्ययनस्यैव ब्रह्मत्वसाधकतया पूर्वमथर्ववेदोपादानम्~। ततश्चाध्वर्युसमाख्याबलाद्यजुर्वेदस्य प्राधान्याद्यजुरुपात्तम्~। साम्नामृगाश्रितत्वात्प्राधान्येन सामतः पूर्वमृग्वेदोपादानम् \textendash\ इति दिक्~॥ यद्वा \textendash\ भृग्वपरनामकस्याथर्वमुनेः {\qt ब्रह्मा देवानां प्रथमं संबभूव विश्वस्य कर्ता भुवनस्य गोप्ता~। स ब्रह्मविद्यां सर्वविद्याप्रतिष्ठामथर्वाय ज्येष्ठपुत्राय आह} इति मुण्डकोपनिषदि ज्येष्ठपुत्रत्वोक्त्या {\qt यज्ञैरथर्वा प्रथमः पथस्तते} ( १~। ६~। ४~। ५ ) इति ऋक्संहितायां यज्ञप्रक्रियाप्रधमप्रकाशकत्वोक्त्या त्वाथर्वप्रोक्तवेदस्याभ्यहितत्वात्प्रथमोपादानम्~। तदन्तर्गतभागेषु विशिष्टलक्षणेषु पद्यगद्यगीत्यात्मकेषु गद्यस्य सुद्देयत्वात्पूर्वं यजुरुपादानम्~। ततः पादरचनात्मकपथलक्षणान्वितत्वादृगुपादानम् इत्याहुर्दाधिमथाः~॥

\columnbreak

\noindent
कर्णेभिर्देवासो गृभ्णामीत्याद्यतिरिक्तपरता लौकिकशब्दस्य, तैस्तद्वयवहारादर्शनात्~। अत एव गौरश्वः इत्यादीनां लौकिकोदाहरण \textendash\ २ त्वसंगतिः~। लोके स्वरानादरात् वेदे तदादराच्च स्वरविशिष्टानां वेदमात्रविषयतामभिप्रेत्य शन्न इत्यादीनां वैदिकोदाहरणत्वसङ्गतिरिति भावः~॥ [ भाष्ये \textendash\ लौकिकास्तावदिति~। तावच्छब्दोऽनुक्रमे~। पूर्वं लौकिका उदाह्रियन्ते, पश्चाद्वैदिका इत्यर्थः~। वैदिकानां प्रधानत्वेऽपि लौकिकानां पूर्वनिर्देशस्तद्वदादरसूचनार्थः~॥ लोके पदानामुदाहरणत्वोक्तौ वेदे वाक्यानामेव सत्वोक्तावाशयमाह \textendash\ तत्रेति~। पदान्तरसमभिव्याहारेण खरविशेषाच्च वेदे वाक्यानामुदाहर \textendash\ 

\noindent
\rule{1\linewidth}{0.5pt}\\

तत्रास्तीति वाच्यम् ,तादृशानामेव वाच्यत्वेन तथोक्तेरयुक्तत्वादत आह \textendash\ लोके इति~॥ मात्रपदेन लोकव्यवच्छेदः~॥\\

 [ भा० १० पृ० १ मप० ] तत्रेति~। तयोर्मध्य इत्यर्थः~॥ [ उ० १६ शप० ] प्रधानेति~। व्याकरणस्य वेदाङ्गत्वात्~। एवंचाभ्य हितत्वं तेषाम्~॥ तद्वत् वैदिकवत्~॥ आदरेति~। अत एव तत्पद \textendash\ घटितोक्तिसाफल्यम्~। अत एव च तद्वत्तेषाम्ंः एकः शब्दः इत्यादिश्रुत्युक्तफलवत्तयाऽभ्यहितत्वम्~। ३अन्यथाऽतत्त्वापत्त्या तत्त्वा \textendash\ पत्तिरिति भावः~॥ ४पदानां \textendash\ पदानामेव~॥ तत्रेतीति~। तयोर्मध्य इतीत्यर्थः~॥ लोकेsपि क्वचित्पदानुपूर्वीनियमसत्वादाह \textendash\ पदान्तरेति~॥ च आनुपूर्वीनियमसमुच्चायकः~। तेन तत्र तस्य सार्वत्रिकत्वं सूचितम्, न तु लौकिके तदभाववत् असार्वत्रिकत्वम्~। अपिस्तु बोधसमुच्चये इति भावः~॥ अस्यापि भेदोक्तिहेतुत्वं बोध्यम्~।\\

अत्रेदं बोध्यम् \textendash\ गौरित्यादीनां वैदिकत्वेऽपि लोके प्रयुज्यमान \textendash\ तया प्रागुक्तरीत्या वा लौकिकत्वात्तत्वोक्तिः~॥ स्तरषादीनां ( ? ) तत्त्वेपि नोक्तिर्वेदानुपकारकत्वात्~। तस्युत्पादनं तु प्रासङ्गिकम्~॥ अत एव गौरित्यादय एवोदाहृताः~॥ नचैवमपि घटपदादय एव कुतो नोक्ता इति वाच्यम्, अशोकवनिकान्यायात्~। गवाद्यर्थानां मङ्गलत्वेन प्रातर्दृश्यत्वस्य धार्मशास्त्रे उक्तत्वेन तत्प्रतिपादकत्वाच्छब्दस्यापि मङ्गलत्वमिति आद्युल्लेखाच्च~। पुरुषशब्दश्चानादिपुरुषपरः~। शकुनि शब्दोऽपि चाषादिपर इति दिक्प्रदर्शनम्~। तत्रापि दृष्टादृष्टभुक्ति \textendash\

\noindent
\rule{1\linewidth}{0.5pt}\\

१ तत्परिहारः \textendash\ अपभ्रंशपरिहारः, त्तेषा \textendash\ वैदिकशब्दानाम्~।\\

२ यद्यपिगव्यादयोsपिलोके विदितास्तथापि न ते सर्वलोकविदिताः, प्रतिदेशं भिन्नत्वादपशब्दानां~। लोकशब्दश्चायं सर्वस्मिल्लोके वर्तते सङ्कोचकाभावात्~। अतः सर्वलोकप्रसिद्धानां गवादीनामिति पदमञ्जर्यांस्पष्टम्~॥\\

३ लौकिकानां पूर्वनिर्देशस्य तद्वदादरसूचनार्थत्वाभावे~। उभयत्रापि तत्त्वं वैदिकत्वम्~। ( र. ना. ) वस्तुतस्तु एवंव्याख्यानमुचितम् \textendash\ अन्यथा \textendash\ अभ्यर्हितत्त्वभावे आदराभावे वा अतत्त्वापत्या \textendash\ अभ्यर्हित \textendash\ त्वानापत्या तत्त्वापत्तिः \textendash\ परनिर्देशापत्तिरिति~। व्याकरणस्य वेदाङ्गत्वादवैदिकानामभ्यर्हितत्वमिति वैदिकशब्दानामेव पूर्वनिर्देशप्रसङ्ग इत्याशयः~।\\

४ एवं चैतच्छायाग्रन्थपर्यालोचनेन {\qt भाष्ये लौकिकास्तावदिति} इत्यादिग्रन्थो {\qt लोके पदानाम्} इत्यतः प्राक् समुचितो लेखकप्रमादाद् {\qt इत्युत्तराशयं परे} इत्युत्तरं पतित आसीत् तस्मादस्माभिर्योग्यस्थाने स्थापित इति बोध्यम्~॥ ( दा. म. ) 
\end{multicols}

\newpage
% १२ उद्द्योतपरिवृतप्रदीपप्रकाशितमहाभाष्ये~। [ १ अ. १ पा. १ पस्पशाह्निके

\begin{multicols}{2}
\noindent
णत्वोक्तिरित्यपि बोध्यम्~॥ लोकवेदसाधारणतया पाणिनीयशब्दानुशासनस्येवात्राधिकृतत्वम्, न तु शाकटायनादिव्याकरणस्येत्युत्तराशयं परे~॥

\begin{center}
\textbf{ ( जिज्ञासाभाष्यम् ) }\\

\textbf{अथ गौरित्यत्र कः शब्दः ?}
\end{center}

 ( प्रदीपः ) अयं गौः \textendash\ अयं शुक्लः इति शब्दार्थयोरभेदेन लोके 

\noindent
\rule{1\linewidth}{0.5pt}\\

\noindent
मुक्तिफलकार्थकत्वेनादौ गौरित्युक्तिः~। अत एव {\qt गौः शान्तिरजा शान्तिरश्चः शान्तिः पुरुषः शान्तिर्ब्रह्म शान्तिर्ब्राह्मणः शान्तिः} इति तैत्तिरीयब्राह्मण उक्तम्~॥ नचैवमपि गौरित्यादिवत् शम् इत्यादीनामपि लौकिकत्वेनैव तत्त्वोक्तिरुचिता, बहूनां तत्त्वादिति वाच्यम्~। उक्तोत्तरत्वात्~॥ नचैवमपि कर्णेभिरित्यादय एवोदाहार्या इति वाच्यम्, प्रायेण लोके प्रयुज्यमाना एव वेदे प्रयुज्यमाना इति सूचनार्थत्वात्~॥ अत एव {\qt देवीरभिष्ट्ये ईळे} इत्यपि तत्रोक्तम्~। देवीरितिं प्रथमाबहुवचनम्~। अभिष्टये इत्यत्रेकारलोपः~। ईळे इत्यत्र लत्वम्~॥ नचैवमपि प्रसिद्ध ऋगादिक्रमेण तत्वोक्तिरुचितेति वाच्यम्, तथैव तदुपादेयमित्यत्र मानाभावात्~। भाष्यकारस्याथार्वणत्वाच्च~। शुद्धच्छान्दसयोस्तत्र सत्त्वाच्च~॥ नचान्त्यपक्षद्वयेऽग्रेषु क्रमोक्तौ किं बीजमिति वाच्यम्~। आध्वर्यवादेऋत्विजश्च क्रमेण तथोक्तेः~॥ यद्वा मङ्गलार्थक२सर्वपदघटितत्वेनादौ तदुक्तेः~। अग्रे तूक्तैव रीतिः~॥\\

 [ उ० १९शप० ] उक्तरीत्या कैयटस्य चिन्त्यत्वात्स्वसिद्धान्तरीत्या आह \textendash\ लोकवेदेति~॥ लौकिकवैदिकशब्देत्यर्थः~॥ अत्रइत्यस्यात्रा न्वयः~॥ यद्वा \textendash\ अत्र \textendash\ भाष्यवाक्ये~। अग्रे {\qt प्रतिपाद्यते} इति शेषः~। करणस्याथिकरणत्वविवक्षायां सप्तमी~॥\\

भाष्ये \textendash\ {\qt गौरित्यत्र} इत्याद्युपलक्षणं सर्वेषां तस्य च~॥\\ 

 [ प्र १मप० ] कैयटे \textendash\ इतीति~। इत्यादीत्यर्थः~। तेन क्रियादेरपि संग्रहः~॥\\

 [ प्र०२यप० ] कैयटे \textendash\ दर्शनादिति~। तथा च संदेह इति भावः~॥ अत एवाह तत्र \textendash\ निर्धारेति \textendash\ 

\noindent
\rule{1\linewidth}{0.5pt}\\

१ अथ गौरिति~। भाष्यस्यायमर्थः \textendash\ शब्दानुशासनमिति प्रति \textendash\ ज्ञासमनन्तरं {\qt गौः} इति विज्ञाने प्रतिभासमानाये पदार्थास्तेषु घटकत्वेनं विद्यमानः कः शब्दशब्दाभिधेयः पृच्छ्यत इति~। शास्त्रप्रतिज्ञाव्यवहितोत्तरो भाष्यकृन्निष्ठव्यापारस्तद्बोध्याः गौरितिविज्ञानविषयपदार्थाधिकरणिका शब्दशब्दाभिधेयविषयिका जिज्ञासेति शाब्दवोधः~। पृच्छधातोर्जिज्ञासाविषयव्यापारवाचित्वम्~। शब्द इति प्रथमान्तानुसारेण {\qt पृच्छ्यते} इति कर्मप्रत्ययान्तस्यासञ्जनम्~। एवञ्च शब्द इति पदोत्तर \textendash\ मितिशब्दाभावेऽपि जिज्ञासायां कर्मत्वेन तस्यान्वयः सुलभः~॥\\

२ शंपदेति पाठो भाति~। ( र. ना. ) इदमयुक्तम्, {\qt शन्नोदेवीर \textendash\ भिष्टये} इति प्रतीके सर्वेषां शब्दानां मङ्गलवाचित्वात् {\qt सर्व}पदघटितपाठस्यैव युक्ततरत्वात्~॥\\

३ शब्दविषयकेत्यर्थः~। ( र. ना ) ४ अयं गौरितीति शेषः~। ( र. ना. ) ५ अत्यक्षभूतो योsयं गौरिति व्यवहारस्तेन शब्दाभेदस्य साधनमित्यर्थः ( र. ना. ) ६ अयं गौरितिप्रयोगादर्थभिन्नतया शब्द \textendash\ विषयकज्ञानाभावेेनेत्यर्थः~। ( र. नाः. ) ७ शब्दार्थयोर्भेदरूपार्थाङ्गी \textendash\ 

\columnbreak

\noindent
व्यवहारदर्शनाच्छब्दखरूपनिर्धारणाय पृच्छति \textendash\ अथेति~। गौःइति विज्ञाने प्रतिभासमानेषु वस्तुषु कः शब्द इत्यर्थः~॥\\

 ( उद्द्योतः ) अभेदेन लोके इति~। अभेदेनैवेत्यर्थः~। शब्द \textendash\ परत्वाभिप्रायेण {\qt अस्य गौः} इति प्रयोगाभावादिति भावः~। पुरोवर्ति \textendash\ व्यक्तिं पश्यतो वाचकजिज्ञासयाकोsयमिति प्रश्ने अयं गौः {\qt अयं वर्णः शुक्लः?} इत्युत्तरस्थले संनिहितमुद्दिश्य तादात्म्येन शब्दविधेयताप्रतीते \textendash\

 [ प्र० ३ यप० ] प्रागुक्ते {\qt गौः} इति स्वरूपे कः शब्दः इति नार्थः तत्स्वरूपस्य श्रोत्रग्राह्यत्वेन तत्र तत्त्वनिर्णयेन संदेहाभावेन शङ्काऽनु \textendash\ दयात्~। किं तु ज्ञानाकारोऽयम्~। तथा च तस्यार्थादाधारता बोध्या~। नचैवमत्रपदानर्थक्यमिति वाच्यम्, तस्य प्रतिभासमानेष्वि \textendash\ त्यर्थकत्वात्~। तस्य च विषयभूतेष्वित्यर्थः~। तदेतदभिप्रेत्याह \textendash\ गौ \textendash\ रितीति कैयटे~। शाब्दे प्रत्यक्षादौ चेत्यर्थः~॥\\

 [ प्र० ३ यप० ] कैयटे \textendash\ वस्तुषु~। तेषां मध्ये~। तद्विविक्तशब्द \textendash\ तत्त्वज्ञानात्~॥\\

 [ उ० १मप० ] गुणगुण्याद्योरिवाभेदेनापि \textendash\ इत्यर्थवारणाय एवान्तर्भावेण व्याख्याने बीजमाह \textendash\ शब्देति~॥ णास्यय गौरितीति~। {\qt घटः शुक्ल इति वहो गौरिति} इत्यन्यादृशस्त्वस्त्येवेत्यनुपदं स्फुटीभविष्यति~॥ ननु शाब्दज्ञानजनकतया प्रयोगाभावेन ४कथं प्रत्यक्षभूतऽतथाव्यवहारसाधनम् ? किंच अयं गौः इति ज्ञानं ज्ञातशब्दकीयम्, उताज्ञातशब्दकीयम् ? आद्येतत्पदव्राच्यः सः इ्त्यर्थेनैषां सिद्धेः~। अन्त्ये \textendash\ अज्ञानेन तस्यैवासंभवादिति चेत्, नः तथा शाब्दज्ञानाभावेन तत्परत्वाभावेन भेदनिरासेनार्थात्तयोरभैदे सिद्धे ज्ञातशब्दकीयत्वेनैव तत्र तथैवोपपत्तौ ७तथार्थाङ्गीकारे मानभाव इत्याशयात्~। अत एवात्र द्रढयितुं पृथग् हेत्वन्तरमाह \textendash\ पुर इति~॥ तदाकारस्तु \textendash\ अयमित्येव \textendash\ इत्यादिः~। संनिहितमिति~। पिण्डमित्यर्थः~। सांनिध्यं च देशतः~॥ प्रतीतेरिति~। अन्यथा लोके जायमानायास्तस्या असंगतिः स्पष्टैवेति भावः~॥ किंचैवमभावे भेदेन तयोरुक्तिरुचितेति बोध्यम्~॥ नन्विदमयुक्तम्, शब्दस्यार्थे शक्त्यङ्गीकारेण तस्या अन्यरूपाया अन्यत्रोक्तत्वेन कथमुक्तार्थ \textendash\

\noindent
\rule{1\linewidth}{0.5pt}\\

\noindent
कारे इत्यर्थः~। ( र. न. ) वस्तुतस्तु ननु शाब्दज्ञानजनकतया इत्यारभ्य {\qt मानाभाव इत्याशयात्} इत्यन्तस्य छायाग्रन्थस्यायमभि \textendash\ प्रायः \textendash\ गोशब्दस्य शाब्दबोधजनकतया पिण्डवाचित्वेन अस्य गौः इति प्रयोगोऽनन्वितार्थत्वान्न भवति~। अभावेन प्रत्यक्षस्य वस्तुनः साधनंतु अन्यत्रादृष्टचरमिति {\qt अस्य गौरिति प्रयोगाभावादिति भावः} इत्युद्द्योतग्रन्थोsनुपपन्नश्च~। अन्यच्च {\qt अयं गौः} इति प्रयोगो यदि शब्दप्रतीतिमता क्रियते तदा गोशब्दवाच्योsयमित्यर्थेनापि स उपपद्यते शब्दानभिज्ञेन च तादृशप्रयोगासम्भव एवेत्याशङ्क्य तथा शाब्दज्ञानाभावेन \textendash\ भेदेन शाब्दबोधाभावेन {\qt अयं गौः इत्यस्य} तत्पद \textendash\ वाच्यः सः? इस्येतत्परत्वाभावेन शब्दार्थयोरभेदे सिद्धे {\qt ज्ञातशब्र्कीयत्वेनैव अयं गौः} इत्यत्राभेदबोधोपपत्तौ तथार्थाङ्गीकारे \textendash\ भेदार्थाङ्गीकारे ( तत्पदवाच्यः स इत्यर्थाङ्गीकारे ) मानाभाव इति~। एवञ्च प्रयोगाभावेन शाब्दबोधाभावस्तेन च भेदाभावः साध्यते, न तु प्रयोगाभावेन व्यवहारः साध्यते~। निरस्ते च भेदेऽभेदः सिध्यत्येवेति अभेदव्यवहारोपपत्तिरिति उद्द्योतः सुसङ्गतः~।
\end{multicols}

\fancyhead[RE]{[ १ अ. १ पा. १ पस्पशाह्रिकै}
\fancyhead[LO]{}
\fancyhead[LE,RO]{\thepage}
\fancyhead[CE]{उद्द्योतपरिवृतप्रदीपप्रकाशितमहाभाष्यम्~।}
\fancyhead[CO]{महाभाष्यप्रदीपोद्दयोतव्याख्या छाया~।}
\cfoot{}
\newpage
%%%%%%%%%%%%%%%%%%%%%%%%%%%%%%%%%%%%%%%%%%%%%%%%%%%%%
\renewcommand{\thepage}{\devanagarinumeral{page}}
\setcounter{page}{13}
% महाभाष्यप्रदीपोद्द्योतव्याख्या छाया~। १३ 

\begin{multicols}{2}
\noindent
रिति तात्पर्यम्~॥ शब्दार्थयोस्तादात्म्यमेव शक्तिः~। स्पष्टं चेदं पातज्जलभाष्ये इति मञ्जूषायामस्माभिर्हर्यादिसंमततया व्युत्पादितम्~। अत एव जात्यादिव्यक्त्योरिव {\qt रामेति द्व्यक्षरं नाम मानभङ्गःपिनाकिनः} इत्यादौ शब्दार्थयोरपि अभेदेन व्यवहारः~॥ अथ गौरितिभाष्ये~। अथ इत्यस्य {\qt पृच्छ्यते} इति शेषः~॥ किं पृच्छ्यते, तदाह \textendash\ गौरित्यन्त्रेत्यादि~॥ कः शब्द इति~। कः शब्दशब्दाभिधेय इति प्रश्नः~॥ प्रतिभासमानेषु \textendash\ शब्दजातिव्यवत्यादिषु~॥ १नन्वेवं गुणक्रिययोः शब्दत्वाशङ्काऽनुपपन्ना~। न हि ते अपि {\qt गौः इति शब्दजन्यबोधे} भासेते इति त्वेत्, न~। गुणगुणिनोः क्रियाक्रियावतोश्चा \textendash\ 

\noindent
\rule{1\linewidth}{0.5pt}\\

\noindent
लाभस्ततोऽत आह \textendash\ शब्दार्थेति~॥ यद्यपि वाच्यवाचकभावा \textendash\ परपर्यायं संबन्धान्तरमेव तयोः शक्तिः~। तद्ग्राहकं चेतरेतराध्या \textendash\ समूलकं तादात्म्यं सङ्केतापरपर्यायम्, तथापि तस्यापि पदनिष्ठश \textendash\ क्त्युपकारकत्वाच्छक्तिरिति व्यवहारः~। तदभिप्रेत्याह \textendash\ तादात्म्य \textendash\ मिति~। तच्च तद्भिन्नत्वे सति तदभेदेन प्रतीयमानत्वम्~। अभेद \textendash\ स्याध्यस्तत्वान्न तयोविरोधः~। तत्र भेदस्योद्भूतत्वविवक्षया अस्यार्थ \textendash\ स्यायं वाचकः? {\qt तस्य वाचकः प्रणवः} इत्यादौ षष्ठी~। अभेदस्योद्भूतत्वविवक्षया तु प्रथमा~॥ अत एवार्थे शब्दधर्मत्वव्यवहारः \textendash\ {\qt पद्यं श्रुतम्} \textendash\ अथार्थं शृणु {\qt पद्यमुक्त्वा अर्थं वदति} इत्यादिः अत्यन्त \textendash\ भेदेऽपुरुषयोरिव तव्यवहाराभावात्~। नाप्यत्यन्तभेदे सः~। घटे स्वधर्मत्वव्यवहाराभावात्~। कः शब्दः कोsर्थ इति प्रश्ने घट इत्ययं शब्दो घट इत्ययमर्थः \textendash\ इत्येकाकारोत्तरदर्शनात्तयोरध्याससिद्धिः~। अयमध्यास आदिव्यवहारकृदीश्वरकृत एव~। अत एव {\qt स्थितोऽस्य} वाचकस्य वाच्येन सह संबन्धः~। संकेतस्तूक्तरूप ईश्वरस्य स्थितमेवाभिसंबन्धमभिनयति यथाऽवस्थितः पितापुत्रयोः संबन्धः संकेतेनावद्योत्यते \textendash\ अयमस्य पिता, अयमस्य पुत्रः इत्यादि पातज्जलभाष्ये उक्तम्~॥ अत एव च हरिः \textendash\

\begin{quote}
{\qt }
\end{quote}
इन्द्रियाणां स्वविषयेष्वनादिर्योग्यता यथा~।\\
अनादिरर्थै शब्दानां संबन्धो योग्यता तथा~॥

संबन्धिशब्दे संबन्धो योग्यतां प्रति योग्यता~।\\
समयाद्योग्यतासंविन्मातापुत्रादियोगवत्~॥

सति प्रत्ययहेतुत्त्वं संबन्धे उपपद्यते~।\\
शब्दस्यार्थे यतस्तस्मात्संबन्धोsस्तीति गम्यते~॥

इति~॥ तदाह \textendash\ स्पष्टमिति~॥ व्युपादितमिति~। शक्तिवाद इति भावः~॥ उक्तार्थं द्रढ्यति \textendash\ अत एवेति~॥ तादात्म्यस्यैव शक्तित्वाङ्गीकारादेवेत्यर्थः~॥ कादिना गुणक्रियापरिग्रहः~।

\noindent
\rule{1\linewidth}{0.5pt}\\

१ ननु गौरिति विज्ञाने प्रतिभासमानेषु पदार्थेषु कः शब्द इत्या \textendash\ शङ्क्य {\qt यत्तर्हिं तदिङ्गितं} \textendash\ यत्तर्हिर्शुक्लो नीलः इति गुणक्रिययोः शब्दत्वमाशङ्क्यते भाष्ये तदनुपपन्नम्, तयोस्तत्राप्रतिभासात्~। अतस्तदनुपपन्नं प्रसाधयति \textendash\ नन्वेवमित्यादिना~। गुणस्य गुणिनाऽर्थेनाभेदात् अर्थाभिन्नस्य शब्दस्यापि तदभिन्नाभिन्नस्यैति न्यायेन गुणेना \textendash\ भेद इति तस्याप्युपस्थितिरित्याशयः~।\\

२ ननु {\qt तदभिन्ना \textendash\ } इति न्यायेन गुणस्य शब्देनाभेदे प्रसाधि \textendash\ तेऽपि गोशब्दे उच्चारिते गुणक्रिययोरप्रतीतिरनुभवसिद्धेति सिद्धान्तमत \textendash\ 

\columnbreak

\noindent
भेदात्~। शब्दार्थयोश्च तत्त्वात् {\qt तदभिन्नाभिन्नस्य} इति न्यायेन तच्छङ्कोपपत्तेः~। यथा \textendash\ परमकारणाभिन्नकार्यकारणकस्य परमकारणेनाप्यभेदः~। यद्वा {\qt गुणसमूहो द्रव्यम्} इति पक्षे {\qt तस्य भावः} इति सूत्रस्थभाष्यसंमते समूहस्य गोशब्दवाच्यत्वेऽवयवभूतगुणादीनामपि तद्वाच्यत्वमिति शङ्का बोध्या~। २ अत्र {\qt गौरिति विज्ञाने} इति सामान्योक्त्या तदाकारप्रत्यक्षादिज्ञानेऽपि गुणादीनां सामान्यरूपेण भानम्~। अत एव प्रत्यक्षदृष्टेऽपि आम्रफलादौ रसविशेषादिजिज्ञासा, विशेषजिज्ञासायाः सामान्यज्ञानपूर्वकत्वात् \textendash\ इत्यन्ये~। निरूपितं चैतन्मज्जूषायाम्~॥

\noindent
\rule{1\linewidth}{0.5pt}\\

\noindent
अग्रिमादिना {\qt गर्वभङ्गो भार्गवस्य शौर्यभङ्गश्च} वालिनः~॥ हिरण्यपूर्व कशिपुं प्रचक्षते~॥ {\qt बृद्धिरादैच् ओमित्येकाक्षरं ब्रह्म} इत्यस्य च परिग्रहः~॥\\

 [ उ० ९ मप० ] पृच्छतीतिकैयटोक्तिसूचितमाह \textendash\ अथेत्यस्येति~। आनन्तर्यार्थकस्येत्यर्थः~॥ किं पृच्छ्यत इति~। इति शङ्कायामिति शेषः~॥ तदित्ति~। सामान्यप्रश्नविषयभूतं विशिष्य तथै \textendash\ वाहेत्यः~॥ भाष्ये ३इतिस्तु नोक्तः, ततस्यार्थपरत्वेन शब्दपरत्वाभावेन तस्यानुपयोगात्~। ४अभेदस्तु तद्विनापि सूपपादः~॥ गौरित्यत्रैत्युक्त्याऽन्यत्रान्यैवेयमाशङ्केति निरासाय प्रकृतसंगतये चाह \textendash\ कः शब्द इति~। तद्घटकशब्द इत्यर्थः~। स्फुटत्वाय ५तस्यार्थपरत्वमिति सूचनाय च प्रश्न इत्यत्र पुनरुक्तिः~॥ {\qt इत्यर्थः इति क्वचित्पाठः~॥}\\

 [ उ० ११ शप० ] अत एव व्याचष्टे \textendash\ शब्देति~। अत्रोपस्था \textendash\ पकत्वात्, विशेषणत्वात्, गुणादयाधारत्वाच्च क्रमोक्तिर्बोध्या~॥\\

 [ उ० ११ शप० ] एवम् \textendash\ उक्तभाष्यस्योक्तार्थकत्वे~॥ [ उ० १३ शप० ] भासेते इति~। तेन शब्देन तेषां तादात्म्याभावादिति भावः~॥\\

 [ उ० १३ शप० ] गुणेति~। घटः शुक्लः [ गौश्चलः ] इत्यादिप्रतीतेस्तयोस्तत्वसिद्धिः~। \textendash\ श्च तत्वात् \textendash\ अभेदाच्च~। अत्र न्यायसत्ते मानमाह \textendash\ यथेति~॥ त्वस्यापि व्यवहारे तत्सिद्धा यथा तेन,६ तस्य तेनाभेदप्रतीतिरित्यर्थः~॥ ननु वक्ष्यमाणरीत्या तयोर्विषयतावत्त्वमत आह \textendash\ [ १६ शप० ] यद्वा गुणेति~॥ निरन्तरत्वरूपायुतसिद्धत्ववदवयव \textendash\ विशेषानुगतसामान्यविशेषरूपगुणसमूह् इत्यर्थः~॥ [ १६ शप० ] पक्ष इति~। अयुतसिद्धावयवविशेषानुगतः समूहो द्रव्यमिति पतज्जलिरिति पातञ्जलभाष्योक्तसिद्धान्तपक्ष इत्यर्थः~॥ तस्य भाव इतीति

\noindent
\rule{1\linewidth}{0.5pt}\\

\noindent
माह \textendash\ अत्र गौरितीति~।\\

३ अथ पदोत्तरमिति शेषः~। ( र. ना. ) ४ अथ पदार्थानन्तरस्य प्रच्छवात्वर्थैन सहेति शेषः~। तद्विनाऽपि~। इतिपदंविनाऽपीत्यर्थः~। ( र. ना. ) {\qt अथ पदोत्तरं} इति लेखस्तु प्रामादिकः, तत्रानुपयोगात्~। एवञ्च {\qt कः शब्दः} इत्युत्तरमितिशब्दो नोक्त इत्येव छायातात्पर्यम्~।\\

५ अथशब्दस्येत्यर्थः~। ( र. ना. ) एतदप्यनुचितमेव~। तस्यार्थ \textendash\ परत्वं इत्यस्य शब्दशब्दस्यार्थपरत्वमिति छायाsभिप्रायः~।

६ प्रकृत्याद्युपादानकत्वेनेत्यर्थः~। ( र. ना. ] 
\end{multicols}

\newpage
% १४ उद्ह्योतपरिवृतप्रदीपप्रकाशितमहाभाष्यम्~। [ १ अ. १ पा. १ पस्पशाह्निके

\begin{multicols}{2}
\begin{center}
\textbf{ ( प्रश्नभाष्यम् ) }
\end{center}

किं यत्तत्सास्नालाङ्गूलककुदखुरविषाण्यर्थरूपम्, स शब्दः ?\\

 ( प्रदीपः ) तान्येव वस्तूनि क्रमेण निर्दिशति \textendash\ किं यत्तदिति~। उद्दिश्यमानप्रतिनिर्दिश्यमानयोरेकत्वमापादयन्ति सर्वनामानि पर्यायेण तल्लिङ्गमुपाददत इति कामचारतः स शब्दः इति पुंलिङ्गेन निर्देशः~॥\\

 ( उद्द्योतः ) यत्तत् इति समुदायो यद्वृत्तार्थे वर्तते, तस्यैव पुनस्तद्वृत्तेन सःइत्यनेन परामर्शः~। यद्वा प्रसिद्धौ~। ननु {\qt वत्}इत्यस्य

\noindent
\rule{1\linewidth}{0.5pt}\\

\noindent
तत्र हि यस्य गुणस्य भावात् \textendash\ इति वार्तिकव्याख्यावसरे \textendash\ किं पुनर्द्रव्यम्, के पुनर्गुणा इति प्रश्न शब्दस्पर्शरूपरसगन्धा गुणाः, ततोsन्यद्३ द्रव्यम् इत्युक्तम्~। अत एव उक्ते इति विहायाह \textendash\ संमते इति~॥ इदमुपलक्षणम् \textendash\ {\qt }स्त्रियाम् संख्याया अवयवे इति सूत्रभाष्ययोरपि~॥ तत्र ह्युक्तम् \textendash\ {\qt द्रव्ये च भवतः कः संप्रत्ययः ?~। गुणसमुदायो द्रव्यम्} {\qt कं च प्रत्यवयवो गुणः~॥ ?~॥ समुदायं प्रति} इत्युक्तम्~॥ अत्र गुणशब्देनावयवा धर्माश्च ~॥ अत एव व्रीहिषु प्रोक्षणजन्यसंस्कारस्यांशत्वावयवत्वादिना व्यवहारः समन्वयसूत्रे भामत्याम्~॥ समूहसमूहिनोर्मिथो भेदामेदाभ्युपगमः~। अत एव {\qt मृद् घटः मृदि घटः इत्यादिसंगतिर्गुणगुण्याद्योरिव~।} अत एव घटः शुक्लो घटे रूपमित्यादिसंगतिस्तदभिप्रेत्याह \textendash\ [ पृ०१३ उ०१७शप० ] गुणादीनामिति~॥ आदिना क्रियाजातिपरिग्रहः~॥ सिद्धान्तमतमाह \textendash\ [ उ० १८ शप० ] अत्रेति~॥ भाष्यकैयट्योरित्यर्थः~॥ अादिनाऽनु \textendash\ मित्यादिसंग्रहः~॥ [ १९ शप० ] सामान्येति~। रूपत्वादिनेत्यर्थः, विशेषस्तु तत्तदिन्द्रियग्राह्य एवेति भावः~॥ [ २० शप० ] भानमिति~। इति सूचितमिति शेषः~॥ उक्तं द्रढयति \textendash\ त एवेति~। तेषां तत्र तथा भानादेवेत्यर्थः~॥ [ उ० २० शप० ] आदिना गन्धविशेषादिपरिग्रहः~॥ [ पृ०१३उ० २ १ शप० ] निरूपीति~। तत्रैवेति भावः~॥\\

भाष्ये \textendash\ सास्नेत्यादिविषाणान्तसमाहारद्वन्द्वप्रकृतिकेनिनाऽवयवी उच्यते~॥ यत्तु \textendash\ सास्नेत्यादि द्रव्यान्तरावयवस्याप्युपलक्षणम्, क्रियागुणव्यक्तीनाँ बह्नीनामग्रे उक्ते \textendash\ अत्र द्रव्यान्तरोदाहरणस्यार्थोचित्यात \textendash\ इति कृष्णः~। तन्न, गौरित्यत्र भासमानविषये शकङ्कासत्त्वेन तत्र शुक्लादीनां सर्वेषां तत्त्वसंभवे द्रव्यान्तरस्य चातथात्वेन वैलक्षण्यात्~॥ अ्त एव सामान्यविषये सामान्योक्तावपि तत्र वर्तमानस्यापि द्रव्यत्वादेrन तत्त्वाशङ्का~। भाष्ये \textendash\ स शब्द इति~। स शब्दशब्दाभिधेयः किमित्यर्थः~॥\\

 [ प्रदीपे १ मप० ] कैयटे \textendash\ क्रमेणेति~। वक्ष्यमाणबीजक्रमे \textendash\ णेत्यर्थः~॥ निर्दिशतीति~। तत्र प्रश्नवाद्येव तात्त्विकं तन्मुखादुत्तर \textendash\

\noindent
\rule{1\linewidth}{0.5pt}\\

१ भाष्ये \textendash\ नेत्याह इत्युत्तरम्, तत्र हेतुः \textendash\ द्रव्यं नाम तत् \textendash\ इति~। अथवा {\qt न, इत्याह~। द्रव्यं नाम तत्} इति भाष्याक्षरविन्यासः~। एवं विन्यस्ते {\qt न} इत्युत्तरं, तत्र हेतुः \textendash\ इत्याह \textendash\ इति~। पूर्व शब्दानुशासनमित्याह न द्रव्यानुशासनमित्याहेति तदर्थः~। द्रव्यस्य चक्षुग्राह्यत्वात् शब्दस्य श्रोत्रग्राह्मत्वान्न शब्दशब्देन तत्प्रतीयत इत्यर्थः~।

\columnbreak

\noindent
नपुंसकत्वेन तच्छब्देऽपि नपुंसकलिङ्गमेवोचितमत आह \textendash\ उद्दिश्य \textendash\ मानेति~। सिद्धवत्कीर्तनमुद्देशः~। प्रतिपाद्यतया विधेयतया वा कीर्तनं प्रतिनिर्देशः~। यथा \textendash\ {\qt शैत्यं हि यत्सा प्रकृतिर्जलस्य} इत्यादौ स्त्रीलिङ्गनिर्देशः~॥ तल्लिङ्गमिति~॥ तयोलिङ्गमित्यर्थः~॥ 

\begin{center}
\textbf{ ( उत्तरभाष्यम् ) }
\end{center}

नेत्याह,१ द्रव्यं नाम तत्~॥\\

 ( प्रदीपः ) नेत्याहेति~। भिन्नेन्द्रियग्राह्यत्वान्न द्रव्यं शब्द इति प्रतीतम्, अपि तु द्रव्यं \textendash\ इति~। यदि न द्रव्यानुशासनं विवक्षितमभविष्यत् अथ द्रव्यानुशासनम् इत्येवावक्ष्यत्२~॥

\noindent
\rule{1\linewidth}{0.5pt}\\

\noindent
मवगन्तुं फलान्तराशयं तन्मुखान्निरसितु च प्रश्नविषयतया तान्येव तथा विशिष्य निर्दिशतीत्यर्थः~॥\\

 [ प्र० ३ यम० ] यौगपद्यासंभवादाह \textendash\ पर्यायेणेति~॥ कैयटे इतिर्हेतौ~। स च कामचारे~॥\\

 [ उ० १मप० ] {\qt यत्तत्} इति न प्रतीकम्, किंतु व्याख्यानान्तर्ग \textendash\ तम्~। समुदायः \textendash\ सर्वनामसमुदायः~। द्वयोरपि समुदायादिशब्द \textendash\ प्रयोगः~॥ यद्वृत्तेति~॥ यदा वृत्तं निष्पन्नं विभक्त्यन्तमात्रं यदिति तस्यैवार्थे इत्यर्थः~॥ अत एवाह \textendash\ तस्यैवेति~॥ यत्तदोर्नित्यसंबन्धात्~॥ {\qt सोsहमाह} ते वयं ब्रूमः इत्याद्यनुरोधेनाह \textendash\ [ उ० २ यप० ] यद्वेति~॥ {\qt प्रसिद्धौ तत्} इति पाठः~। तथा च पृथगन्वयः~॥ यद्वा यदित्यनूद्यमानस्य बुद्यारूढत्वाय प्रकृतार्थकतच्छब्दप्रयोगः~॥ अग्रे प्राग्वत्~॥ अत एवाह \textendash\ यदित्येति~। नपुंसकलिङ्गार्थामिधायि \textendash\ त्वादिति भावः~॥ [ ३ यप० ] वोचितमिति~। तयोरेकविषयत्वादिति भावः~। वस्तुतोऽसिद्धत्वेऽपि तत्त्वेन तस्य तत्त्वायाह \textendash\ [ ४ र्थप० ] सिद्धवदिति~॥ प्रतिपाद्यत्वस्याव्यापकत्वादाह \textendash\ विधेयेति~॥\\

 [ उ० ] एवं क्व दृष्टमित्यत्राह \textendash\ [ ५ मप० ] यथेति~॥ स्रीलिङ्गेति~। सेति स्त्रीलिङ्गेत्यर्थः~। ननु तस्य लिङ्गं इत्यर्थे परामर्शासंभवः {\qt पर्यायेण} इत्यस्य वैयर्थ्यमिष्टासिद्धिश्चात आह \textendash\ [ उ० ६ष्ठप० ] तथोरिति~। प्रक्रान्तयोरित्यर्थः~॥\\

सिद्धान्ती उत्तरयति \textendash\ [ भाष्ये \textendash\ ] नेति~॥ तत्र हेतुमाह \textendash\ इत्याहेत्यादि~॥ यत इत्यादिः~। आहेति वर्तमानसामीप्ये लट्~। यतः \textendash\ {\qt शब्दानुशासनम्} इत्यवोचत्~। न तु द्रव्यम् इत्यर्थः~। नाम निश्चये प्रसिद्धौ वा~॥ यद्वा \textendash\ तटस्थः सिद्धान्त्युत्तरमनुवदति \textendash\ नेत्याहेति~। इतिर्नञर्थस्य कर्मत्वद्योतकः~॥ तत्र हेतुमाह स एव \textendash\ द्रव्यमिति~॥ यत इत्यादिः~॥ एवमग्रे सर्वत्र~॥\\

 [ दक्षिणभागे प्र० ] \textendash\ यत इत्यस्यार्थमाह \textendash\ कैयटे \textendash\ भिन्ने \textendash\ न्द्रियेति~॥ शब्दग्राहकश्रोत्रान्यचक्षुग्राह्यत्वादित्यर्थः~॥ तत्र द्रव्यं तत्तवेनाभिमतम्~॥ नामपदार्थमाह \textendash\ [ २ यप० ] प्रतीतमिति~॥ तहि केन धर्मेण तत्प्रसिद्धमित्यत्राह \textendash\ अपि त्विति~। गुणाश्रय \textendash\ त्वात् तत्समूहत्वाद्वेत्यर्थः~॥

\noindent
\rule{1\linewidth}{0.5pt}\\

२ अवक्ष्यदिति~। यदि शब्दशब्देन द्रव्यं प्रतीयते, अतश्च द्रव्यानुशासनमित्येव तदर्थः; तदा स्पष्टप्रतिपत्तये अथ द्रव्यानुशास \textendash\ नम् इत्येव भाष्यकृतोक्तं स्यादित्यर्थः~।\\

३ समूहसमूहिनोर्भेदमादायेदमिति बोध्यम्~। ( र. ना. ) 
\end{multicols}

\end{document}