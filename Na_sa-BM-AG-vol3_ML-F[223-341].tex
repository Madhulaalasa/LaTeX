\documentclass[11pt, openany]{book}
\usepackage[text={4.65in,7.45in}, centering, includefoot]{geometry}
\usepackage[table, x11names]{xcolor}
\usepackage{fontspec,realscripts}
\usepackage{polyglossia}
\setdefaultlanguage{sanskrit}
\setotherlanguage{english}
\setmainfont[Scale=1]{Shobhika}
% \defaultfontfeatures[Scale=MatchUppercase]{Ligatures=TeX} 
% \newfontfamily\sanskritfont[Script=Devanagari,Mapping=devanagarinumerals]{Shobhika}
\newfontfamily\s[Script=Devanagari, Scale=0.8]{Shobhika}
\newfontfamily\regular{Linux Libertine O}
\newfontfamily\en[Language=English, Script=Latin]{Linux Libertine O}
\newfontfamily\na[Script=Devanagari, Scale=1.1, Color=purple]{Shobhika-Bold}
\newfontfamily\qt[Script=Devanagari, Scale=1, Color=violet]{Shobhika-Regular}
\newcommand{\devanagarinumeral}[1]{%
	\devanagaridigits{\number \csname c@#1\endcsname}} % for devanagari page numbers
%\usepackage[Devanagari, Latin]{ucharclasses}
%\setTransitionTo{Devanagari}{\s}
%\setTransitionFrom{Devanagari}{\regular}
\XeTeXgenerateactualtext=1 % for searchable pdf
\usepackage{enumerate}
\pagestyle{plain}
\usepackage{fancyhdr}
\pagestyle{fancy}
\renewcommand{\headrulewidth}{0pt}
\usepackage{afterpage}
\usepackage{multirow}
\usepackage{multicol}
\usepackage{mdframed,lipsum}
\usepackage{wrapfig}
\usepackage{vwcol}
\usepackage{microtype}
 \usepackage{amsmath,amsthm, amsfonts,amssymb}
\usepackage{mathtools}% <-- new package for rcases
\usepackage{graphicx}
\usepackage{longtable}
\usepackage{setspace}
\usepackage{footnote}
\usepackage{perpage}
\MakePerPage{footnote}
%\usepackage[para]{footmisc}
%\usepackage{dblfnote}
\usepackage{xspace}
\usepackage{array}
\usepackage{emptypage}
\usepackage{hyperref}% Package for hyperlinks
\hypersetup{colorlinks,
citecolor=black,
filecolor=black,
linkcolor=blue,
urlcolor=black}

\newcommand\blfootnote[1]{%
 \begingroup
 \renewcommand\thefootnote{}\footnote{#1}%
 \addtocounter{footnote}{-1}%
 \endgroup
}

\mdfdefinestyle{MyFrame}{%
 linecolor=black,
 outerlinewidth=2pt,
 %roundcorner=20pt,
 innertopmargin=4pt,
 innerbottommargin=4pt,
 innerrightmargin=4pt,
 innerleftmargin=4pt,
 leftmargin = 4pt,
 rightmargin = 4pt
 %backgroundcolor=gray!50!white}
 }

\begin{document}
\cfoot{}
\fancyhead[CE]{नाट्यशास्त्रम्}
\fancyhead[CO]{द्वाविंशोऽध्यायः}
\fancyhead[LE,RO]{\thepage}
\renewcommand{\thepage}{\devanagarinumeral{page}}
\setcounter{page}{203}

% द्वाविंशोऽध्यायः २०३ 

\begin{quote}
{\na मुहयति हृदयं \renewcommand{\thefootnote}{1}\footnote{भ \textendash\ क्वापि हि गच्छति शिरश्च ड \textendash\  दह्यत्यङ्गं}क्वापि प्रयाति शिरसश्च वेदना तीव्रा~।\\
न धृतिं चाप्युपलभते ह्यष्टममेवं प्रयुञ्जीत\renewcommand{\thefootnote}{2}\footnote{भ \textendash\ अष्टमवामस्थिते कामे च \textendash\  त्वभिनयेत्}~॥~१८८

\renewcommand{\thefootnote}{3}\footnote{भ \textendash\  किञ्चिद्ब्रवीति नो पृष्टा}पृष्टा न किंचित् प्रब्रूते न श्रृणोति न पश्यति~।\\
\renewcommand{\thefootnote}{4}\footnote{ड \textendash\  तूष्णीं हाकष्टभाषा च नष्टचित्ता जडा स्मृता भ \textendash\  विक्षिप्तनिस्सहाङ्गा च नष्टचिन्ता जडस्मृतिः}हाकष्टवाक्या तूष्णीका जडतायां गतस्मृतिः~॥~१८९

अकाण्डे दत्तहुंकारा तथा प्रशिथिलाङ्गिका~।\\
श्वासग्रस्तानना चैव जडताभिनये भवेत्~॥~१९०

\renewcommand{\thefootnote}{5}\footnote{ड \textendash\  प्रतीकारैः कृतैः सर्वैः (भ \textendash\  स्मृतैः)}सर्वैः कृतैः प्रतीकारैर्यदि नास्ति समागमः~।\\
कामाग्निना प्रदीप्ताया \renewcommand{\thefootnote}{6}\footnote{ड \textendash\  भवेत्तु}जायते मरणं ततः~॥~१९१

एवं स्थानानि कार्याणि कामतन्त्रं समीक्ष्य तु\renewcommand{\thefootnote}{7}\footnote{भ \textendash\  अवेक्ष्य तु च \textendash\  तन्त्र\textendash\ समीक्षया}~।\\
\renewcommand{\thefootnote}{8}\footnote{भ \textendash\  अप्राप्तानि हि कामस्य}अप्राप्तौ यानि काम्यस्य\renewcommand{\thefootnote}{9}\footnote{ड \textendash\  कामस्य} वर्जयित्वा तु नैधनम्~॥~१९२

\renewcommand{\thefootnote}{10}\footnote{च \textendash\  त्रिविधैः}विविधैः पुरुषोऽ\renewcommand{\thefootnote}{11}\footnote{ड \textendash\  हि}प्येवं विप्रलम्भसमुद्भवैः\renewcommand{\thefootnote}{12}\footnote{भ \textendash\  अनुसूचकैः}~।\\
भावैरेतानि कामस्य नानारूपाणि योजयेत्\renewcommand{\thefootnote}{13}\footnote{ड \textendash\  कारयेत्}~॥~१९३}
\end{quote}

\hrule

\vspace{2mm}
\noindent
मुह्यतीत्यत्र हेतुः, यतो हृदयं नावतिष्ठति~। \underline{हा कष्ट}मिति एतद्व्यतिरेकेण \underline{तूष्णीका} एतदेव भाष्यत इति यावत्~। \underline{प्रतीकारैरिति} समागमोपायैरित्यर्थः~। \underline{नैधनं} मरणम्~। \underline{नानारूपाणीति} अवस्था इत्यर्थः~।

\newpage
% २०४ नाट्यशास्त्रम् 

\begin{quote}
{\na \renewcommand{\thefootnote}{1}\footnote{भ \textendash\  मातृकायां श्लोकचतुष्टयं न वर्तते}एवं कामयमानानां स्त्रीणां नृणामथापि वा~।\\
सामान्यगुणयोगेन युञ्जीताभिनयं बुधः~॥~१९४

चिन्तानिःश्वासखेदेन \renewcommand{\thefootnote}{2}\footnote{ड \textendash\  देहस्यायासनेन च}हृद्दाहाभिनयेन च\\
तथानुगमनाच्चापि तथैवाध्व\renewcommand{\thefootnote}{3}\footnote{च \textendash\  अर्ध}निरीक्षणात्~॥~१९५

आकाशवीक्षणाच्चापि तथा दीनप्रभाषणात्\\
स्पर्शनान्मोटनाच्चापि तथा \renewcommand{\thefootnote}{4}\footnote{च \textendash\  पापाश्रय ड \textendash\  चोपाश्रय}सापाश्रयाश्रयात्~॥~१९६

एभिर्नानाश्रयोत्पन्नैर्विप्रलम्भसमुद्भवैः~।\\
कामस्थानानि सर्वाणि भूयिष्ठं संप्रयोजयेत्~॥~१९७

\renewcommand{\thefootnote}{5}\footnote{ड \textendash\  वासो भ \textendash\  वासश्चन्दन}स्रजो भूषणगन्धांश्च गृहाण्युपवनानि च~।\\
\renewcommand{\thefootnote}{6}\footnote{भ \textendash\  भूयिष्ठं दह्यमानो हि}कामाग्निना दह्यमानः शीतलानि निषेवते~॥~१९८

\renewcommand{\thefootnote}{7}\footnote{ड \textendash\  प्रसह्यमानः}प्रदह्यमानः कामार्तो \renewcommand{\thefootnote}{8}\footnote{भ \textendash\  नवस्थानसमन्वितः}बहुस्थानसमर्दितः~।\\
प्रेषये\renewcommand{\thefootnote}{9}\footnote{च \textendash\  कामदूतीं तु स्वावस्थादर्शनं प्रति भ \textendash\  कामदूतीं तु निजावस्थां प्रदर्शयेत् कामुको\ldots निवेदिनीम्}त्कामतो दूतीमात्मावस्थाप्रदर्शिनीम्\renewcommand{\thefootnote}{10}\footnote{ड \textendash\  प्रशंसिनीम्}~॥~१९९

सन्देशं चैव दूत्यास्तु प्रदद्यान्मदनाश्रयम्~।}
\end{quote}

\hrule

\vspace{2mm}
पुरुषस्य सुलभोपायत्वान्मध्य एव समागमः शक्यक्रियः, न तु योषिता\textendash\ मित्याशयेन कामावस्थाः स्त्रीषूपदिष्टाः, पुरुषेष्वतिदिष्टाः~।\\

\underline{स्रजो भूषणगन्धांश्चेत्या}दिना यदुक्तं तद्यथायोगमवस्थातुं योज्यम्~।

\newpage
% द्वाविंशोऽध्यायः २०५

\begin{quote}
{\na \renewcommand{\thefootnote}{1}\footnote{भ \textendash\  इयं तस्य त्ववस्था हि निवेद्या प्रश्रयादिति ड \textendash\  तस्याप्यवस्थेति निवेद्यं प्रश्रयादिभिः}तस्येयं समवस्थेति कथयेद्विनयेन सा~॥~२००

\renewcommand{\thefootnote}{2}\footnote{च \textendash\  अथ वेदित}अथावेदित\renewcommand{\thefootnote}{3}\footnote{भ \textendash\  भावार्थैरभ्युपायं}भावार्थो रत्युपायं विचिन्तयेत्~।\\
अयं विधिर्विधानज्ञैः कार्यः प्रच्छन्नकामिते\renewcommand{\thefootnote}{4}\footnote{भ \textendash\  कामितैः}~॥~२०१

\renewcommand{\thefootnote}{5}\footnote{भ \textendash\  तत्र राजोपचारस्य व्याख्यास्याम्यनुपूर्वशः~। विधिमाभ्यन्तरं}विधिं राजोपचारस्य पुनर्वक्ष्यामि तत्त्वतः~।\\
अभ्यन्तर\renewcommand{\thefootnote}{6}\footnote{ड \textendash\  गतान्\ldots तन्त्रे समुत्थितान्}गतं सम्यक् कामतन्त्रसमुत्थितम्~॥~२०२

सुखदुःखकृतान् भावान् नानाशील\renewcommand{\thefootnote}{7}\footnote{भ \textendash\  शिल्प}समुत्थितान्~।\\
यान्यान् प्रकुरुते राजा तान्स्तान् लोकोऽनुवर्तते~॥~२०३

न दुर्लभाः \renewcommand{\thefootnote}{8}\footnote{भ \textendash\  तु ते राज्ञां स्त्रीसंभोग\textendash\ कृता गुणाः च \textendash\  दुर्लभः .\ldots कृतो गुणः ड \textendash\  नृपाणां तु स्त्रियो ह्याज्ञाकृता गुणाः}पार्थिवानां स्त्र्यर्थमाज्ञाकृता गुणाः~।\\
दाक्षिण्यात्तु समुद्भूतः कामो रतिकरो भवेत्~॥~२०४ 

बहुमानेन \renewcommand{\thefootnote}{9}\footnote{भ \textendash\  चोरीणां}देवीनां वल्लभानां भयेन च~।}
\end{quote}

\hrule

\vspace{2mm}
\noindent
प्रच्छन्नकामितं नायिकान्तरेभ्यः सन्ध्रियमाणमित्यर्थः~।\\

\begin{sloppypar}
ननु राज्ञां किमशक्यं येन दूतीप्रेषणादिप्रयासमनुभवन्तीत्याह न दुर्लभाः पार्थिवानामित्यादि~। आज्ञाकृता गुणा उपायाः उप प्रलम्भादुदाहरणात् अय इत्यर्थः~। राज्ञां ये उपाया इति सम्बन्धषष्ठी तेन बलार्थेऽभिमतनिषेधः~। \underline{दाक्षिण्यादित्युभयविषयम्}, पूर्वनायिकासु च दाक्षिण्यं, अभिलषणीय\textendash\ प्रमदाकृतं च दाक्षिण्यम्, तस्या राजविषयं प्रेमेत्यर्थः~। कुतः प्रच्छन्नत्वमित्याह \underline{बहुमानेनेति} कर्मषष्ठी~। \underline{देव्यो} हि प्राप्ताभिषेका अवश्यं राज्ञा बहुत्वेनोत्कृष्ट\textendash
\end{sloppypar}

\newpage
% २०६ नाट्यशास्त्रम् 

\begin{quote}
{\na प्रच्छन्नकामितं \renewcommand{\thefootnote}{1}\footnote{भ \textendash\  राज्ञः}राज्ञा कार्यं परिजनं प्रति~॥~२०५

यद्यप्यस्ति नरेन्द्राणां\renewcommand{\thefootnote}{2}\footnote{च \textendash\  नृपाणां तु} कामतन्त्र\renewcommand{\thefootnote}{3}\footnote{भ \textendash\  कामद्रव्यं}मनेकधा~।\\
प्रच्छन्नकामितं यत्तु तद्वै रतिकरं भवेत्~॥~२०६

\renewcommand{\thefootnote}{4}\footnote{भ \textendash\  संभाव्यते भयं यत्र यतश्चैव निवार्यते~। दुर्लभं यत्तु तत्रासौ कामो रतिकरो भवेत्}यद्वामाभिनिवेशित्वं \renewcommand{\thefootnote}{5}\footnote{च \textendash\  यतश्चैव}यतश्च विनिवार्यते~।\\
दुर्लभत्वं च यन्नार्याः \renewcommand{\thefootnote}{6}\footnote{ड \textendash\  कामिनः सा रतिः परा}सा कामस्य परा रतिः~॥~२०७

राज्ञामन्तः पुरजने दिवासंभोग इष्यते~।\\
वासोपचारो यश्चैषां\renewcommand{\thefootnote}{7}\footnote{भ \textendash\  चैव} स रात्रौ परिकीर्तितः~॥~२०८}
\end{quote}

\hrule

\vspace{2mm}
\noindent
त्वेन माननीयाः~। अनेकधेति विवाहितावरुद्धेत्यादिनेत्यर्थः~। वामाभिनिवे\textendash\ शित्वमिति {\qt सुलभावमानी हि मदन} इति विघ्नः, तथाप्यभि\textendash\ लष्यमाणं वस्तु प्राप्तं चेत् कोऽभिलाषः, तेन प्राप्तं प्राप्तमपहारितमिव गतं, गतं प्राप्तमिवेत्येवम्~। (दुर्लभत्वमित्यादि) पराक्रमेण [विद्धि] विष्णुरयं काम उत्तमतमां प्रीतिं प्रतिप्रतनोति न ह्यत्र यायामिव (भयादिव?) निवृत्तिः साध्या, अपि तु भौमात्मकं सुखं भोगस्त्वसति कामे तेन (केन?) प्रत्युत संभवनीयः\renewcommand{\thefootnote}{$\ddagger$}\footnote{व्याख्येयमस्फुटा भ्रष्टपतिताक्षरत्वात्~।}~। रतिरिति तद्धेतुत्वादित्यर्थः~।\\

अथ स्पष्टकामितमाह राज्ञामन्तःपुरजने (दिवा)संभोग इति~। संभोगः परस्परावलोकनप्रणयकलहसङ्गीतकादि~। यथोक्तम्\textendash\ {\qt तद्वक्त्रेन्दुविलोक\textendash\ नेन दिवसो नीतः}\renewcommand{\thefootnote}{*}\footnote{पूर्णः श्लोकः\textendash
\begin{quote}
{\qt तद्वक्त्रेन्दुविलोकनेन दिवसो नीतः प्रदोषस्तथा \\
तद्गोष्ठयैव निशापि मन्मथकृतोत्साहैस्तदङ्गार्पणैः~।\\
तां संप्रत्यपि मार्गदत्तनयनां द्रष्टुं प्रवृत्तस्य मे\\
बद्धोत्कण्ठमिदं पुनः किमथवा प्रेमा समाप्तोत्सवः~॥} तापस १\textendash\ १६.
\end{quote}} इति~। अन्तःपुरजनोऽत्र ऊढा, पुनर्भूः, अवरुद्धग\textendash\ णिका, कन्यकाप्रेष्याविषयं तु प्रच्छन्नकामितमुक्तम्~।

\newpage
% द्वाविंशोऽध्यायः २०७

\begin{quote}
{\na \renewcommand{\thefootnote}{1}\footnote{भ \textendash\  परिपाट्या कुलार्थे च नवप्रसवसंगमे}परिपाट्यां फलार्थे वा \renewcommand{\thefootnote}{2}\footnote{ड \textendash\  नव\ldots च}नवे प्रसव एव वा~।\\
दुःखे चैव प्रमोदे च षडेते वासकाः स्मृताः\renewcommand{\thefootnote}{3}\footnote{ड \textendash\  इति~।}~॥~२०९

उचिते वासके स्त्रीणामृतुकालेऽपि वा नृपैः\renewcommand{\thefootnote}{4}\footnote{भ \textendash\  बुधैः}~।\\
\renewcommand{\thefootnote}{5}\footnote{डभनम \textendash\  द्वेष्याणां}प्रेष्याणामथवेष्टानां\renewcommand{\thefootnote}{6}\footnote{भ \textendash\  अवचेष्टानां} कार्यं\renewcommand{\thefootnote}{7}\footnote{च \textendash\  वेश्यानामपि कर्तव्यमिष्टानां} चैवोपसर्पणम्~॥~२१०}
\end{quote}

\hrule

\vspace{2mm}
प्रभूतान्तरपूर्वस्य वासकवृत्तान्तमाह\textendash

\begin{quote}
{\qt परिपाट्यां फलार्थे वा नवे प्रसव एव वा~।\\
दुःखे चैव प्रमोदे च षडेते वासकाः स्मृताः~॥} इति
\end{quote}

\noindent
परिपाटिर्यथाकल्पितानुपूर्वी अस्या एकेन भिन्नेन वारः, अस्या द्वाभ्या\textendash\ मित्यादि~। तदपवादमाह \underline{फलार्थ} इति ऋताविति यावत्~। \underline{नव} इति नवत्वे प्रसवे वृत्ते चिरविरहखिन्ना सुखायितं \underline{दुःखे} तदीयबन्धुव्यापत्त्या दुःखिता आश्वासनीयेति~। \underline{प्रमोद} इति तदीयपुत्रोत्सवादौ 'उत्सवो हि माननीय' इत्यु\textendash\ क्तम्~। वासयति तत्र स्थाने रात्रमिति वासः~। अत्र उचितः कामोपचारः फलार्थ इत्यस्य हेतोः सर्वापवादकत्वं दर्शयितुं धर्मवृत्तिना राज्ञा परिचार्यो द्वेष्या दुर्भगा अपि सेव्या इति निरूपयितुमाह\textendash

\begin{quote}
{\qt उचिते वासके स्त्रीणामृतुकालेऽपि वा नृपैः~।\\
प्रेष्याणामपि सर्वासां कार्यं (चैवोपसर्पणम् )~॥} इति
\end{quote}

\noindent
आर्तवकालो हि भूयानपि \renewcommand{\thefootnote}{*}\footnote{परतः}फलतः परमिति भवति~। यथोक्तम्\textendash

\begin{quote}
{\qt ऋतुः षोडश तत्राद्याश्चतस्रो दशमात्परा\\
त्रयोदशी च निन्द्याः स्युरयुगाः कन्यकोद्भवाः~।\\
षष्ठ्यष्टमी च दशमी द्वाभ्यां वर्णैश्च साधिका\\
युग्मा पुत्राय रात्रिः स्यात्~॥} इति~।
\end{quote}

\newpage
% २०८ नाट्यशास्त्रम् 

\begin{quote}
{\na तत्र वासकसज्जा च\renewcommand{\thefootnote}{1}\footnote{ड \textendash\  वा} विरहोत्कण्ठितापि वा~।\\
स्वाधीनभर्तृका चापि\renewcommand{\thefootnote}{2}\footnote{ड \textendash\  चैव} कलहान्तरितापि वा\renewcommand{\thefootnote}{3}\footnote{च \textendash\  तथा}~॥~२११

खण्डिता विप्रलब्धा वा तथा प्रोषितभर्तृका~।\\
तथाभिसारिका चैव ज्ञेयास्त्वष्टौ तु नायिकाः\renewcommand{\thefootnote}{4}\footnote{भ \textendash\  इत्यष्टौ नायिकाः स्मृताः}~॥~२१२

उचिते वासके या तु रतिसंभोगलालसा~।\\
\renewcommand{\thefootnote}{5}\footnote{च \textendash\  मङ्गलं}मण्डनं कुरुते हृष्टा सा वै वासकसज्जिका~॥~२१३

अनेककार्यव्यासङ्गाद्यस्या नागच्छति प्रियः~।\\
\renewcommand{\thefootnote}{6}\footnote{च \textendash\  तस्यानुगम ड \textendash\  अनागमन भ \textendash\  तदनागम}तदनागतदुःखार्ता विरहोत्कण्ठिता तु सा\renewcommand{\thefootnote}{7}\footnote{न \textendash\  मता}~॥~२१४

\renewcommand{\thefootnote}{8}\footnote{भ \textendash\  सुरतविरसैर्बद्धा (?) ड \textendash\  सुरतैश्चरितैः}सुरतातिरसैर्बद्धो यस्याः पार्श्वे तु नायकः\renewcommand{\thefootnote}{9}\footnote{च \textendash\  पार्श्वगतः प्रियः~। सामोदे गुणसंयुक्ता}~।}
\end{quote}

\hrule

\vspace{2mm}
\noindent
तत्रापि नक्षत्रविशेषपरिवर्जनम्~। पुत्रश्च राज्ञां मुख्यफलम्, यथाह {\qt प्रजायै गृहमेधिनाम्} (रघु\textendash\ १) इति~। अत्र तु वृद्धपशुव्यो (पशवो?)वदन्ति\textendash \\

मासपसूआ\ldots \ldots (षण्)मासगब्भिणी एकदिअहज्जरमुहे\ldots \ldots ~।\renewcommand{\thefootnote}{*}\footnote{अपूर्णा चास्फुटार्थेयं गाथा कोक्कोकवचनस्य मूलं स्यात्~। यथा\textendash\  

\begin{quote}
{\qt रङ्गादिश्रान्तदेहा चिरविरहवती मासमात्रप्रसूता \\
 गर्भालस्या च नव्यज्वरयुततनुका त्यक्तमानप्रसन्ना~। \\
 स्नाता पुष्पावसाने नवरतिसमये मेघकाले वसन्ते\\
 प्रायस्संपन्नरागा मृगशिशुनयना स्वल्पसाध्या रते स्यात्~॥}
\end{quote}} (इति यत्तन्निषिद्धं स्मृतौ वैद्यके च~।)\\

वासकभावाभावभावितान्नायिकाभेदान्दर्शयति \underline{अत्र (तत्र?) वासक\textendash\ सज्जेति}~। एताः क्रमेण लक्षयति \underline{उचिते वासक} इत्यादिना~। उचितः पूर्वोक्तेन नयेनायातः~। \underline{लालसा} साभिलाषा~।

\newpage
% द्वाविंशोऽध्यायः २०९
\begin{quote}
{\na \renewcommand{\thefootnote}{1}\footnote{भ \textendash\  सामोदगुणसंप्राप्ता}सान्द्रामोदगुणप्राप्ता भवेत् स्वाधीनभर्तृका~॥~२१५

ईर्ष्याकलहनिष्क्रान्तो यस्या नागच्छति प्रियः~।\\
\renewcommand{\thefootnote}{2}\footnote{भ \textendash\  अमर्ष}सामर्षवशसंप्राप्ता\renewcommand{\thefootnote}{3}\footnote{च \textendash\  संतप्ता} कलहान्तरिता भवेत्~॥~२१६

\renewcommand{\thefootnote}{4}\footnote{भ \textendash\  अस्मि\textendash\ न्नवोचिते}व्यासङ्गादुचिते यस्या वासके नागतः प्रियः~।\\
\renewcommand{\thefootnote}{5}\footnote{ड \textendash\  तस्यानागम च \textendash\  तदनागमनार्ता तु खण्डितेत्यभिधीयते}तदनागमदुःखार्ता खण्डिता सा प्रकीर्तिता~॥~२१७

यस्या \renewcommand{\thefootnote}{6}\footnote{च \textendash\  दूत्या प्रियः\ldots गत्वा ड \textendash\  दूतीप्रियं}दूतीं प्रियः प्रेष्य दत्त्वा संकेतमेव वा~।\\
नागतः कारणेनेह\renewcommand{\thefootnote}{7}\footnote{भ \textendash\  ऐव} विप्रलब्धा तु सा भवेत्\renewcommand{\thefootnote}{8}\footnote{भ \textendash\  स्मृता}~॥~२१८

\renewcommand{\thefootnote}{9}\footnote{च \textendash\  गुरुकार्या\textendash\ न्तरवशाद्यस्या विप्रोषितः}नानाकार्याणि सन्धाय यस्या वै प्रोषितः \renewcommand{\thefootnote}{10}\footnote{भ \textendash\  अभिसंबन्धाद्यस्या विप्रोषितः ड \textendash\  अर्थसंबन्धै\textendash\ र्यस्या विप्रोषितः}प्रियः~।\\
सा(प्र?)रूढालककेशान्ता भवेत् प्रोषितभर्तृका~॥~२१९

\renewcommand{\thefootnote}{11}\footnote{भ \textendash\  या निर्लज्जेन संबद्धा (प \textendash\  या नैर्लज्येन)}हित्वा लज्जां \renewcommand{\thefootnote}{12}\footnote{ड \textendash\  समा\textendash\ कृष्टा}तु या श्लिष्टा मदेन मदनेन च\renewcommand{\thefootnote}{13}\footnote{च ड \textendash\  वा प \textendash\  या}~।\\
अभिसारयते कान्तं सा भवेदभिसारिका~॥~२२०}
\end{quote}

\hrule

\vspace{2mm}
\noindent
स्वाभिलाषा~। \underline{आमो}दगुणो हर्षः सौभाग्याभिमानश्च~। \underline{व्यासङ्गा}दित्यन्य\textendash\ नारीविषयादित्यर्थः (प्र)रूढाः प्रलम्बीभूता अलकाः, \underline{केशान्तश्च} कबरीभारः~। प्ररूढ एकवेणीभूते यस्याः~। अन्ये त्वकृतकर्मतया केशान्ते ललाटे रोम्णामुद्भेद\textendash\ मुत्प्ररूढं वर्णयन्ति~। मदो मद्यकृतः, चकाराद् द्वयं वदन्मदनस्यैव प्राधान्यमाह~। अभिसरः सहायः तस्य व्यापारेण प्रियतममतिक्रामति, {\qt तत्करोति तदाचष्टे तेनातिक्रामति धातुरूपं} चेति स्पष्ट (सृ\textendash\ टेः ?) इत्यस्य वृद्धिः~। कुप्यन्ती

\lfoot{27}

\newpage
\lfoot{}
%२१० नाट्यशास्त्रम् 

\begin{quote}
{\na आस्ववस्थासु \renewcommand{\thefootnote}{1}\footnote{भ \textendash\  सर्वासु}विज्ञेया नायिका नाटकाश्रया~।\\
एतासां \renewcommand{\thefootnote}{2}\footnote{भ \textendash\  संप्रवक्ष्यामि कामतन्त्रप्रयोजनम्}चैव वक्ष्यामि \renewcommand{\thefootnote}{3}\footnote{ड \textendash\  यथा\textendash\ योगं प्रयोक्तृभिः ड \textendash\  यथायोज्यं}कामतन्त्रमनेकधा~॥~२२१

चिन्तानिःश्वासखेदेन\renewcommand{\thefootnote}{4}\footnote{च \textendash\  खेदैश्च} \renewcommand{\thefootnote}{5}\footnote{च \textendash\  हृदया भ \textendash\  हृत्ताप}हृद्दाहाभिनयेन च~।\\
\renewcommand{\thefootnote}{6}\footnote{ड \textendash\  सखीनां संप्रलापैश्च}सखीभिः सह संलापै\renewcommand{\thefootnote}{7}\footnote{भ \textendash\  निज}रात्मावस्थावलोकनैः~॥~२२२

ग्लानिदैन्याश्रुपातैश्च रोषस्यागमनेन च~।\\
\renewcommand{\thefootnote}{8}\footnote{य \textendash\  निर्भूषणा न \textendash\  विभूषणा च \textendash\  निर्भूष\textendash\ णाङ्गी विसृजा}निर्भूषणमुजात्वेन दुःखेन रुदितेन च~॥~२२३

\renewcommand{\thefootnote}{9}\footnote{भ \textendash\  खण्डितां\ldots लब्धां\ldots तामपि~। \ldots कान्तां च भावैरेतैः प्रदर्श\textendash\ येत्}खण्डिता विप्रलब्धा वा कलहान्तरितापि वा~।\\
तथा प्रोषितकान्ता च \renewcommand{\thefootnote}{10}\footnote{डच \textendash\  भावैरेवं}भावानेतान् प्रयोजयेत्~॥~२२४

विचित्रोज्ज्वलवेषा तु\renewcommand{\thefootnote}{11}\footnote{च \textendash\  च} प्रमोदोद्दयोतितानना~।\\
उदीर्णशोभा च तथा\renewcommand{\thefootnote}{12}\footnote{च \textendash\  अतिशया} कार्या स्वाधीनभर्तृका~॥~२२५}
\end{quote}

\hrule

\vspace{2mm}
\noindent
प्रशान्तेत्यादयस्तु नायिकाभेदे नेहोक्ताः, तेषां वासकाभावनिवृत्तत्वाभावात्~। नाटकाश्रया इति \underline{नाट्यविषया} इत्यर्थः~।\\

तत्र विप्रलम्भजीवितं श्रृङ्गारेण वासकाभावभावितनायिकाभेदाश्रयं प्रथमं दर्शयति \underline{चिन्तानिःश्वासखेदेनेत्या}दिना~। सामान्यभूते सकलहृदयसंवा\textendash\ दिनि मदनोपचारे चायमभिनय इत्यपि सामान्याभिनयः~। \underline{भावानेता}निति पठितानुभावव्यभिचारिसमुचिता ये भावा व्यभिचारिस्थायिरूपा(णामि) हापठितमप्यनुभावजातं प्रदर्शयति यावत्~। वासकसज्जा अभिसारिका च स्वावसरे स्वाधीनभर्तृकयैव तुल्ये इति पृथङ्नोक्ते~।

\newpage
% द्वाविंशोऽध्यायः २११ 

\begin{quote}
{\na \renewcommand{\thefootnote}{1}\footnote{भ \textendash\  वेश्यानां कुलजाया वा प्रेष्याया वा ड \textendash\  वेश्याया\ldots जाया वा प्रेष्याया वाथवा नृपैः प \textendash\  वेश्यायां कुलजायां च}वेश्यायाः कुलजायाश्च प्रेष्यायाश्च प्रयोक्तृभिः~।\\
एभिर्भावविशेषैस्तु कर्तव्यमभिसारणम्~॥~२२६

समदा मृदुचेष्टा च तथा परिजनावृता~।\\
नानाभरणचित्राङ्गी गच्छेद्वेश्याङ्गना शनैः\renewcommand{\thefootnote}{2}\footnote{ज \textendash\  जनैः}~॥~२२७

संलीना स्वेषु गात्रेषु त्रस्ता \renewcommand{\thefootnote}{3}\footnote{च \textendash\  विप्रेक्षित ड \textendash\  अवनमित भ \textendash\  विक्षिप्तलोचना}विनमितानना~।\\
अवकुण्ठनसंवीता गच्छेत्तु\renewcommand{\thefootnote}{4}\footnote{च \textendash\  च भ \textendash\  गच्छेत} कुलजाङ्गना~॥~२२८

मदस्खलितसंलापा विभ्रमोत्फुल्ललोचना~।\\
आविद्धगतिसंचारा गच्छेत्प्रेष्या समुद्धतम्\renewcommand{\thefootnote}{5}\footnote{भ \textendash\  प्रेष्याङ्गना तथा (ड \textendash\  नया)}~॥~२२९

\renewcommand{\thefootnote}{6}\footnote{भ \textendash\  रहोगतं तु संप्राप्ता च \textendash\  स्यादयं शयितो व्यक्तं ड \textendash\  गत्वा सा वदति व्यक्तं}गत्वा सा चेद्यदा तत्र\renewcommand{\thefootnote}{7}\footnote{ज \textendash\  व्यक्तं} पश्येत्सुप्तं प्रियं तदा\renewcommand{\thefootnote}{8}\footnote{च \textendash\  यदा भ \textendash\  सदा ड \textendash\  ततः}~।\\
\renewcommand{\thefootnote}{9}\footnote{भ \textendash\  प्रिया यथोत्थापयति तथा वक्ष्याम्यहं पुनः}अनेन तूपचारेण \renewcommand{\thefootnote}{10}\footnote{ड \textendash\  कुर्याद्वै प्रति}तस्य कुर्यात्प्रबोधनम्~॥~२३०

अलङ्कारेण कुलजा वेश्या गन्धैस्तु शीतलैः~।\\
प्रेष्या \renewcommand{\thefootnote}{11}\footnote{ड \textendash\  अथ वस्तु}तु वस्त्रव्यजनैः \renewcommand{\thefootnote}{12}\footnote{च \textendash\  बोधयेच्छयितं प्रियम्}कुर्वीत प्रतिबोधनम्~॥~२३१}
\end{quote}

अथाभिसारणास्वरूपमाह \underline{वेश्याया} इत्यादि~। \underline{भावविशे}षैरिति वस्तु\textendash\ विशेषैरित्यर्थः~। संलापो दूत्या सख्या सहोक्तिप्रत्युक्तिः~। (\underline{अलङ्कारेणेति}) अलङ्कारशब्दो नूपुरादि(वि)शिष्टः [तत्]~।

\newpage
% २१२ नाट्यशास्त्रम्

\begin{quote}
{\na \renewcommand{\thefootnote}{1}\footnote{ड \textendash\  मातृकायामधिकः पाठः\textendash\ निर्भर्त्सनपरं प्रायस्तथाश्वसनपेशलम्~। निष्ठुरं मधुरं चैव सखीनामपि जल्पनम्}कुलाङ्गनानामेवायं \renewcommand{\thefootnote}{2}\footnote{च \textendash\  प्रीक्तः}नोक्तः कामाश्रयो विधिः~।\\
सर्वावस्थानु\renewcommand{\thefootnote}{3}\footnote{ड \textendash\  स्थान}भाव्यं हि\renewcommand{\thefootnote}{4}\footnote{च \textendash\  तु} यस्माद्भवति नाटकम्~॥~२३२

\renewcommand{\thefootnote}{5}\footnote{च \textendash\  भय ड \textendash\  न च}नवकामप्रवृत्ताया क्रुद्वाया वा समागमे~।\\
\renewcommand{\thefootnote}{6}\footnote{भ \textendash\  नानोपायैः समाधाय}सापदेशैरुपायैस्तु वासकं संप्रयोजयेत्\renewcommand{\thefootnote}{7}\footnote{च \textendash\  संप्रकल्पयेत्}~॥~२३३

नानालङ्कारवस्त्राणि गन्धमाल्यानि चैव हि\renewcommand{\thefootnote}{8}\footnote{भ \textendash\  यानि यानि च माल्यानि धूपगन्धाम्बराणि च}~।\\
\renewcommand{\thefootnote}{9}\footnote{{\qt प्रिया} आरभ्य\ldots {\qt समुद्भवे} इत्यन्तस्य पाठस्य स्थाने भ मातृकायां तु\textendash\ हर्षात् सुखानि वस्तूनि निषे\textendash\ वेत स्मरातुरः~। नार्या नित्यं विशेषेण प्रमोदरससंभवः चड \textendash\  नित्यं सुखान्युदात्तानि}प्रियायोजितभुक्तानि निषेवेत मुदान्वितः\renewcommand{\thefootnote}{10}\footnote{ड \textendash\  मदान्विता य \textendash\  मुदान्विता च \textendash\  सेवेत मदनान्वितः}~॥~२३४}
\end{quote}

\hrule

\vspace{2mm}
ननु राज्ञां वेश्यादिसंभोगो निह्नवकारापायसङ्कुलत्वात् किमिहानेन दर्शि\textendash\ तेनेत्याशङ्कयाह \underline{सर्वावस्थानुभाव्य हि कार्यं} नाटकमिति नाट्य इति यावत्~। सर्वावस्था अनुभाव्या प्रत्यक्षायमाणत्वं नेया, यत्र प्रकरणादौ वेश्यासंभोगो\textendash\ ऽप्यस्तीत्युक्तं दशरूपके (अध्या\textendash\ १८) यदि तर्हि सर्वतः प्रकारो नाटके प्रदर्श्यस्तदा वासके यदा प्रकटप्रकार उक्तः~।\\

एवं व्याजप्रकारोऽपि वाच्य इत्याशयेनाह \underline{नवकामप्रवृत्ताया} इति~। नवे पुरुषसम्भोगे या प्रवृत्ता प्रथमसमागमनीया वेश्या पूनर्भूर्गान्धर्वविवाह\textendash\ विवाह्या कन्या वा तस्याः समागमे चिकीर्षिते, सव्याजैः उपायैः वासको व्याजश्च यथा तथैव प्रकटं नाङ्गीकरोति, नायिकान्तरेभ्यश्च झटिति भीरुर्नायको भवति~। अवरुद्धापि खण्डिता कलहान्तरिता वा वासकमनङ्गीकुर्वती व्याजतो\textendash\ ऽर्ङ्गीकार्यते~। तत्र नायिकाहृदयग्रहणोचितविदग्धतानिमित्तं नायकस्योपचा\textendash\ रमाह \underline{नानालङ्कारेत्यादि}~। 

\newpage
% द्वाविंशोऽध्यायः २१३

\begin{quote}
{\na न तथा भवति मनुष्यो\renewcommand{\thefootnote}{1}\footnote{च \textendash\  विशेषो}\\
मदनवशः कामिनीमलभमानः~।\\
द्विगुणोप\renewcommand{\thefootnote}{2}\footnote{ढ \textendash\  चार}जातहर्षो भवति\\
यथा सङ्गतः प्रियया~॥~२३५

विलासभावेङ्गित\renewcommand{\thefootnote}{3}\footnote{ड \textendash\  एधित}वाक्य\renewcommand{\thefootnote}{4}\footnote{ड \textendash\  काव्य}लीला\textendash \\
\renewcommand{\thefootnote}{5}\footnote{च \textendash\  विशेषमाधुर्य}माधुर्यविस्तारगुणोपपन्नः~।\\
परस्परप्रेमनिरीक्षितेन\\
समागमः कामकृतस्तु\renewcommand{\thefootnote}{6}\footnote{ड \textendash\  गतश्च} कार्यः~॥~२३६

\renewcommand{\thefootnote}{7}\footnote{ड \textendash\  नार्याऽप्यथविशेषेण प्रमोदरससंभवः~।}ततः प्रवृत्ते मदने उपचारसमुद्भवे~।\\
\renewcommand{\thefootnote}{8}\footnote{भ \textendash\  उपचारस्तु}वासोपचारः कर्तव्यो नायकागमनं प्रति~॥~२३७}
\end{quote}

\hrule

\vspace{2mm}
ननु प्राप्ता चेत् प्रमदा तर्हि निवृत्तकामः, तत् कोऽयमुपचारप्रतिभर इत्याशङ्कयाह \underline{न तथा} भवतीति~। मदनः कामभवः इति काम एव~। मदि हर्ष\textendash\ ग्लपनयोरिति पठति, तेन समागमे द्विगुणीभवति कामः, ततश्चोपचारे यत्र उचित इत्याह \underline{विलासभावेति}~। विलासः स्त्रीणां च व्याख्यातः {\qt स्थाना\textendash\ सनगमनाना}मिति, (२२\textendash\ १५) (स्त्रियाः), {\qt धीरसंचारिणी दृष्टिः (२२\textendash\ ३५) इति (पुरुषस्य) च~। तत्प्रधानभावाः चेष्टितानि~। \underline{इङ्गितं} प्रेमसूचका व्यापाराः व्याख्याताः} काम्येनाङ्गविकारेणे त्यादि, {\qt चकिता भवेत्} इत्यन्तेन (२२\textendash\ १६७\textendash\ १७२)~।\\

वाक्यलीलेति अङ्गासाध्यादिषूक्ता~। माधुर्यं {\qt सर्वावस्थाविशेषे} ष्विति

\newpage
% २१४ नाट्यशास्त्रम् 

\begin{quote}
{\na गन्धमाल्ये\renewcommand{\thefootnote}{1}\footnote{च \textendash\  माल्यं} गृहीत्वा तु चूर्णवास\renewcommand{\thefootnote}{2}\footnote{भ \textendash\  वासान्}स्तथैव च~।\\
\renewcommand{\thefootnote}{3}\footnote{च \textendash\  स्थापयेन्नायककृते कुर्याच्चात्मप्रसा\textendash\ धनम् भ \textendash\  आदर्शः संगृहीतव्यः कृतार्थापि पुनः पुनः (ड \textendash\  कृतार्थो)}आदर्शो लीलया गृह्यश्छन्दतो वा पुनः पुनः~॥~२३८

वासोपचारे नात्यर्थं भूषणग्रहणं भवेत्~।\\
रशनानूपुरप्रायं स्वनवच्च\renewcommand{\thefootnote}{4}\footnote{भ \textendash\  तु} प्रशस्यते\renewcommand{\thefootnote}{5}\footnote{ड \textendash\  चैव यद्भवेत्}~॥~२३९

नाम्बरग्रहणं रङ्गे न स्नानं \renewcommand{\thefootnote}{6}\footnote{च \textendash\  नानुलेपनम्}न विलेपनम्~।\\
नाञ्जनं नाङ्गरागश्च\renewcommand{\thefootnote}{7}\footnote{च \textendash\  रागं च} \renewcommand{\thefootnote}{8}\footnote{च \textendash\  न च केशो\textendash\ पसंग्रहः भ \textendash\  न केशरचनं ड \textendash\  स्तनकेशग्रहौ न च}केशसंयमनं तथा~॥~२४०

\renewcommand{\thefootnote}{9}\footnote{ड \textendash\  नापावृता}नाप्रावृता नैकवस्त्रा न रागमधरस्य तु\renewcommand{\thefootnote}{10}\footnote{भ \textendash\  च}~।\\
उत्तमा मध्यमा वापि \renewcommand{\thefootnote}{11}\footnote{ड \textendash\  प्रयोक्तव्याङ्गना च \textendash\  प्रकुर्यात्}कुर्वीत प्रमदा क्वचित्~॥~२४१

अधमानां भवेदेष \renewcommand{\thefootnote}{12}\footnote{भ \textendash\  विधिः प्रकृतिसंभवः}सर्व एव विधिः सदा~।}
\end{quote}

\hrule

\vspace{2mm}
\noindent
(२२\textendash\ २९)नृणां च, {\qt अभ्यासात् करणानां} (२२\textendash\ ३६)इति~। एषां विस्तारगुणेन कविवर्णनाकृतेनोपपन्न एतस्मादुपचारत्सम्यगुद्भवो यस्य~। \underline{चूर्णः} पटवासः~। \underline{वासो} वस्त्रम्~। \underline{नात्यर्थमिति} खेदावहत्वात्~। रात्रावदृश्यत्वात्तु दरिद्राणां दीपशून्यं वासगृहं दृष्टवतां हेतुत्वेनु स्वनवत्त्वं, \underline{चेति} हेतौ यतः प्रशस्तेन सुरतली\textendash\ लोपयोगिना शब्देन युक्तम्~। \underline{नाम्बरग्रहणं रङ्गे} इत्यादिना प्रसङ्गतो निषेध\textendash\ माह~। वासोपचारसिद्धये तु रङ्गेऽप्येतदुपचारोपयोगि कर्तव्यमिति हि कोप\textendash

\newpage
% द्वाविंशोऽध्यायः २१५

\begin{center}
{\na \renewcommand{\thefootnote}{1}\footnote{च \textendash\  तासामपि ह्यसभ्यं यन्न तत्कार्यं प्रयोक्तृभिः~। प्रमदाभिर्नरैर्वापि नानाभावं तु नाटके भ \textendash\  तासामपि तु संभोगो न कार्यस्तु प्रयोक्तृभिः~। वेश्यादीनां तु नारीणां नराणां चापि नाटके}कारणान्तर\renewcommand{\thefootnote}{2}\footnote{ड \textendash\  सामान्यं}मासाद्य तस्मादपि न कारयेत्~॥~२४२

प्रेष्यादीनां च नारीणां नराणां वापि नाटके~।\\
भूषणग्रहणं \renewcommand{\thefootnote}{3}\footnote{भ \textendash\  कृत्वा पुष्पाणां ग्रहणं भवेत्}कार्यं पुष्पग्रहणमेव च~॥~२४३

\renewcommand{\thefootnote}{4}\footnote{ड \textendash\ निर्युक्तमण्डना चापि (भ \textendash\  निर्वृत्त\ldots वापि}गृहीतमण्डना \renewcommand{\thefootnote}{5}\footnote{च \textendash\  किंचित्}चापि प्रतीक्षेत प्रियागमम्~।\\
\renewcommand{\thefootnote}{6}\footnote{इदं श्लोकार्धं च ड \textendash\  योरेव वर्तते}लीलया मण्डितं वेषं कुर्याद्यन्न विरुध्यते~॥~२४४

\renewcommand{\thefootnote}{7}\footnote{ज \textendash\  वासोपचारं कृत्वैवं नायिका नायकागमम्}विधिवद्वासकं कुर्यान्नायिका नायकागमे~।\\
\renewcommand{\thefootnote}{8}\footnote{च \textendash\  वीक्षमाणा प्रियपश्यं (?) शृणुयान्नाडिकाध्वनिम् ड \textendash\  शृणुयान्नाडिकाघोषं प्रतीक्षेदासनस्थिता (भ\textendash\ आसनस्था समुत्सुका}प्रतीक्षमाणा च ततो नालिकाशब्दमादिशेत्~॥~२४५}
\end{center}

\hrule

\vspace{2mm}
\noindent
संभावनयेति नाम्बरग्रहणमिति सामान्योक्तावपि प्रकारादुत्तमाया मध्यमायाश्चेति गम्यते~। अधमायाश्च भवत्येतद्रङ्गेऽपि~। एवं निषेधमधमासु पुनर्विधिमुक्त्वा प्रकृत\textendash\ मुपचारमेवाभिसन्धत्ते \underline{भूषणग्रहणमि}त्यादि~। उक्तपूर्वेऽत्रोपचारे आदरं दर्शयति \underline{विधिवद्वासकं कुर्या}दिति वासकोचितमुपचारमित्यर्थः~।\\

ननु समाप्त उपचारे यदि प्रियो नागतः किं कुर्यादित्याह \underline{प्रतीक्षमाणेति}~। नालिकाशब्दमादिशेदिति इयांश्च कालो गतः किमित्यादि न प्राप्तः स्यादिति,

\newpage
% २१६ नाट्यशास्त्रम् 

\begin{quote}
{\na श्रुत्वा तु नालिकाशब्दं\renewcommand{\thefootnote}{1}\footnote{च \textendash\  नादं ड \textendash\  घोषं} नायकागमविक्लबा~।\\
\renewcommand{\thefootnote}{2}\footnote{ड \textendash\  वेपन्ती सन्नहृदया तोरणाभिमुखी व्रजेत्}विषण्णा वेपमाना च गच्छेत्तोरणमेव च~॥~२४६

\renewcommand{\thefootnote}{3}\footnote{च \textendash\  वामेन तोरणं ग्राह्यं (भ \textendash\  प्राप्य)}तोरणं वामहस्तेन कवाटं दक्षिणेन च\renewcommand{\thefootnote}{4}\footnote{च \textendash\  तु}~।\\
\renewcommand{\thefootnote}{5}\footnote{च \textendash\  हस्तेन संमुस्वीभूय उदीक्षेत प्रियागमम् (भ \textendash\  आवेष्ट्य पश्चात्तु) ड \textendash\  हस्तेनाधोमुखी भूयः प्रतीक्षेत प्रियागमम्}गृहीत्वा तोरणाश्लिष्टा संप्रतीक्षेत नायकम्~॥~२४७

\renewcommand{\thefootnote}{6}\footnote{च \textendash\  शुभाशङ्कां भयं चैव कुर्यात् तोरणसंस्थिता भ \textendash\  सशङ्का चैव रूपं च कुर्यात् तोरणमाश्रिता}शङ्कां चिन्तां भयं चैव प्रकुर्यात्तोरणाश्रिता~।\\
अदृष्ट्वा रमणं नारी विषण्णा च\renewcommand{\thefootnote}{7}\footnote{च \textendash\  तु} क्षणं भवेत्~॥~२४८

दीर्घं चैव \renewcommand{\thefootnote}{8}\footnote{भ \textendash\  तु निश्वस्य}विनिःश्वस्य \renewcommand{\thefootnote}{9}\footnote{भ \textendash\  अश्रु चैव ड \textendash\  आस्रं चैव ढ \textendash\  आस्यं चैव}नयनाम्बु निपातयेत्~।\\
\renewcommand{\thefootnote}{10}\footnote{भ \textendash\  आर्तं}सन्नं च हृदयं कृत्वा \renewcommand{\thefootnote}{11}\footnote{भ \textendash\  विमुञ्चेत्}विसृजेदङ्गमासने~॥~२४९

व्याक्षेपाद्विमृशेच्चापि नायकागमनं प्रति~।\\
तैस्तैर्विचारणोपायैः\renewcommand{\thefootnote}{12}\footnote{च \textendash\  विचारणैश्चापि(ड \textendash\  श्चैव)} शुभाशुभसमुत्थितैः\renewcommand{\thefootnote}{13}\footnote{भ \textendash\  समन्वितैः}~॥~२५०

गुरुकार्येण\renewcommand{\thefootnote}{14}\footnote{ड \textendash\  कार्यैश्च} मित्रैर्वा मन्त्रिणा\renewcommand{\thefootnote}{15}\footnote{च \textendash\  मन्त्रिणां भ \textendash\  मन्त्राणां} राज्यचिन्तया~।\\
\renewcommand{\thefootnote}{16}\footnote{न \textendash\  अनुबद्ध प \textendash\  सानुबन्धः}अनुबद्धः प्रियः किं नु\renewcommand{\thefootnote}{17}\footnote{च \textendash\  तु} वृतो\renewcommand{\thefootnote}{18}\footnote{च \textendash\  धृतो} वल्लभयापि वा~॥~२५१}
\end{quote}

\hrule

\vspace{2mm}
\noindent
ततोऽप्यनागतेऽयं विधिरित्याह \underline{श्रुत्वा तु नालिकशब्द}मिति \underline{तैस्तैर्विचारणोपायै}रिति यदुक्तं तत्रोदाहरणम्~। अनागतोपायानाह गुरुकार्येणेत्यादि अनुभाववर्ग\textendash

\newpage
% द्वाविंशोऽध्यायः २१७ 

\begin{quote}
{\na \renewcommand{\thefootnote}{1}\footnote{ड \textendash\  आकारं दर्शयेदेव (ड \textendash\  च्चापि तम् ) इदं श्लोकार्धं भ \textendash\  मातुकायां न वर्तते}उत्पातान्निर्दिशेच्चापि शुभाशुभसमुत्थितान्~।\\
निमित्तैरात्मसंस्थैस्तु स्फुरितैः स्पन्दितैस्तथा~॥~२५२

शोभनेषु तु\renewcommand{\thefootnote}{2}\footnote{ड \textendash\  च} कार्येषु निमित्तं वामतः स्त्रियाः~।\\
\renewcommand{\thefootnote}{3}\footnote{च \textendash\  दुरुक्तेषु तु कार्येषु (भ \textendash\  विज्ञेयं)}अनिष्टेष्वथ सर्वेषु निमित्तं दक्षिणं भवेत्~॥~२५३

सव्यं नेत्रं ललाटं \renewcommand{\thefootnote}{4}\footnote{च \textendash\  च भ्रूरध\textendash\ रोष्ठः (भ\textendash\ ष्टं) ड \textendash\  च भ्रूरथोष्ठं}च भ्रूनासोष्ठं तथैव च \\
ऊरुबाहुस्तनं चैव स्फुरेद्यदि समागमः~॥~२५४ 

\renewcommand{\thefootnote}{5}\footnote{च \textendash\  अतोऽन्यथा स्पन्दमाने (ड \textendash\  नं}एतेषामन्यथाभावे \renewcommand{\thefootnote}{6}\footnote{ड \textendash\  अनिष्टं चापि (भ \textendash\  दुरुक्तं) ज \textendash\  अनिष्टं च च \textendash\  दुरितं दक्षिणे भवेत्}दुर्निमित्तं विनिर्दिशेत्~। \\
दर्शने दुर्निमित्तस्य मोहं गच्छेत्क्षणं ततः~॥~२५५ 

\renewcommand{\thefootnote}{7}\footnote{भ \textendash\  अप्राप्ते चैव कर्तव्यः प्रिये गण्डाश्रितः करः (च \textendash\  र्पितः}अनागमे नायकस्य \renewcommand{\thefootnote}{8}\footnote{ड \textendash\  हस्तो गण्डाश्रितो भवेत्}कार्यो गण्डाश्रयः करः~।\\
\renewcommand{\thefootnote}{9}\footnote{भ \textendash\  प्रसाधने त्ववज्ञानं}भूषणे चाप्यवज्ञानं रोदनं च समाचरेत्~॥~२५६ 

\renewcommand{\thefootnote}{10}\footnote{भ \textendash\  ततश्चेच्छोभनं पश्येत् (च \textendash\  श्च शो)}अथ चेच्छोभनं तत्स्यान्निमित्तं \renewcommand{\thefootnote}{11}\footnote{च \textendash\  वै प्रियागमे}नायकागमे~।\\
सूच्यो नायिकयासन्नो गन्धाघ्राणेन नायकः~॥~२५७}
\end{quote}

\hrule

\vspace{2mm}
\noindent
स्त्वसूयाखेदार्थचिन्तादिव्यभिचारिसंभवो यथायोगं योज्यः~। \underline{उत्पातः} सह\textendash\ साऽशुभसूचको महाभूतपरिस्पन्दः स च परस्थः~। स्फुरितं चलनं स्पन्दितमे\textendash\ तदितित्वेते स्वदेहस्थे इति भेदेनोक्ते~। \underline{नेत्र}मित्यादि \underline{सव्यं} वाममिति मन्तव्यम्

\lfoot{28}

\newpage
\lfoot{}
% २१८ नाट्यशास्त्रम् 

\begin{quote}
{\na दृष्ट्वा चोत्थाय संहृष्टा\renewcommand{\thefootnote}{1}\footnote{च \textendash\  हृष्टाङ्गी} प्रत्युद्गच्छेद्यथाविधि\renewcommand{\thefootnote}{2}\footnote{च \textendash\  हि नायकम्}~।\\
\renewcommand{\thefootnote}{3}\footnote{सार्धश्लोकस्य स्थाने भ\textendash\  मातृ\textendash\ कायां "न तथा भवति" (श्लो\textendash\ २३५) ग्रन्थोऽस्ति ; चभब \textendash\  मातृकासु सार्धश्लोको न वर्तते}ततः कान्तं निरीक्षेत प्रहर्षोत्फुल्ललोचना~॥~२५८ 

सखीस्कन्धार्पितकरा कृत्वा स्थानकमायतम्~।\\
दर्शयेत ततः कान्तं सचिह्नं सरसव्रणम्~॥~२५९ 

यदि स्यादपराद्धस्तु \renewcommand{\thefootnote}{4}\footnote{भ \textendash\  नरः ड \textendash\  ततः}कृतस्तैस्तैरुपक्रमैः~।\\
उपालम्भकृतैर्वाक्यै\renewcommand{\thefootnote}{5}\footnote{भ \textendash\  अभिभाष्यः स (च \textendash\  तु)}रुपालभ्यस्तु नायकः~॥~२६० 

मानापमानसम्मोहैरवहित्थभयक्रमैः\renewcommand{\thefootnote}{6}\footnote{च \textendash\  अव\textendash\ हित्थैर्यथाक्रमम् ड \textendash\  क्लमैः}~।\\
वचनस्य समुत्पत्तिः स्त्रीणामीर्ष्याकृता\renewcommand{\thefootnote}{7}\footnote{भ \textendash\  कृते} भवेत्~॥~२६१ 

विस्रंभस्नेहरागेषु \renewcommand{\thefootnote}{8}\footnote{ड \textendash\  संमोहे}सन्देहे प्रणये तथा~।\\
परितोषे च \renewcommand{\thefootnote}{9}\footnote{च \textendash\  हर्षे}घर्षे च दाक्षिण्याक्षेपविभ्रमे\renewcommand{\thefootnote}{10}\footnote{च \textendash\  पातने प \textendash\  विस्मये}~॥~२६२}
\end{quote}

\hrule

\vspace{2mm}
\noindent
तथा बाहू~। \underline{अन्यथाभाव} इति दक्षिणस्फुरणे~। \underline{प्रत्युद्गच्छेदिति} नायिकेति प्रथ\textendash\ मया योज्यम्~।\\

एवमियता वासकसज्जया वागङ्गसत्त्वव्यामिश्रः सामान्याभिनयः प्रिय\textendash\ संप्राप्त्यवधिर्दर्शितः~। अथ खण्डितादीनां संदर्श्यते \underline{यदि स्यादपराद्ध}स्त्विति \underline{तैस्तै}रिति मानावहित्थवस्त्रभङ्गाभिष्यन्दादिभिः~। अन्यथाभाषणे कोपादुचिते तदपवादमाह \underline{विस्रम्भेत्या}दि कस्मिंश्चित् कार्ये निरूप्ये विस्रम्भसात्मिका कार्या, शरीरापाटवादि पश्यतो वार्ताप्रश्नादौ स्नेहः, तद्वशाद्योऽनुरागो लक्षितः,

\newpage
% द्वाविंशोऽध्यायः २१९ 

\begin{quote}
{\na धर्मार्थकामयोगेषु\renewcommand{\thefootnote}{1}\footnote{ब \textendash\  योग्येषु} प्रच्छन्नवचनेषु च\renewcommand{\thefootnote}{*}\footnote{स्मरणे च परीक्षणे\textendash\ इति व्याख्यातृपाठः स्यात्.}~।\\
हास्ये \renewcommand{\thefootnote}{2}\footnote{भ \textendash\  कुसुमिते}कुतूहले चैव संभ्रमे व्यसने तथा~॥~२६३ 

\renewcommand{\thefootnote}{3}\footnote{भ \textendash\  हास्योपस्थानसंप्राप्तौ दोषप्रक्षेप\textendash\ निह्नवे (च \textendash\  दोषोप)}स्त्रीपुंसयोः क्रोधकृते पृथङ्मिश्रे तथापि वा~। \\
अनाभाष्योऽपि संभाष्यः प्रिय एभिस्तु कारणैः~॥~२६४ 

यत्र स्नेहो \renewcommand{\thefootnote}{4}\footnote{च \textendash\  भयं तत्र यत्रेर्ष्या तत्र मन्मथः (ड \textendash\  मदनस्ततः) भ \textendash\  तत्र हर्षो यत्रेर्ष्या मदनस्ततः}भवेत्तत्र हीर्ष्या मदनसम्भवा~।\\
चतस्रो योनयस्तस्याः\renewcommand{\thefootnote}{5}\footnote{भ \textendash\  तत्र} कीर्त्यमाना निबोधत~॥~२६५ 

वैमनस्यं व्यलीकं च विप्रियं मन्युरेव च~।\\
एतेषां \renewcommand{\thefootnote}{6}\footnote{च \textendash\  च समुत्पत्तिं प्रयोगं च निबो\textendash\ धत भ \textendash\  चैव वक्ष्यामि यद्यत्तत्र विधीयते}संप्रवक्ष्यामि लक्षणानि यथाक्रमम्~॥~२६६ }
\end{quote}

\hrule

\vspace{2mm}
\noindent
कृतापराधः सत्यतो न वेति सन्देहः, प्रणयः प्रार्थना, अपत्याद्यभ्युदयः परितोषः कलास्थित्यादि (कलाशिल्पादि ?) निकृतिपक्षेण स्पर्धा संघर्षः, पक्षस्य सखी\textendash\ जनादेर्यदा कार्यः प्रियेण सम्पाद्यः पश्यति तदा सखीकार्यविवादविषयाद्दा\textendash\ क्षिण्यम्, इन्द्रजालादिकृते विस्मयः, धर्मार्थकामोपयोगिनि व्रतगृहकृत्यकौमुदी\textendash\ हर्षोत्सवादाववश्यं कर्तव्ये चाप्रियं प्रति स्मरणाकर्तव्या, तेन वा स्मरणम्, उपालम्भद्वारेण किमयं वदेदिति परीक्षणम् , यथा नायकोऽत्यादरात् किंचि\textendash\ त्पृच्छति तत्कुतूहलम्, सर्वो यदा हसति तदा हास्यम्, अग्न्याद्युत्पातः संभ्रमः, वसु(बन्धु ?)वियोगादि व्यसनम्, एतेष्वनाभाषणे सर्वथैव निस्स्ने\textendash\ हता स्यात्~। \underline{क्रोधकृत} इति युगपदित्यर्थः~। \\

अपराद्ध इत्युक्तं तत्रापराधं दर्शयितुमुपक्रमं करोति \underline{यत्र स्नेह} इति~। योनयो हेतवश्चत्वारः\textendash\ वैमनस्यं, व्यलीकं, विप्रियं, मन्युरिति, तान् क्रमेण

\newpage
% २२० नाट्यशास्त्रम् 

\begin{quote}
{\na निद्राखेदालसगतिं\renewcommand{\thefootnote}{1}\footnote{भ \textendash\  रतिं} सचिह्नं सरसव्रणम्~।\\
एवंविधं प्रियं दृष्ट्वा वैमनस्यं \renewcommand{\thefootnote}{2}\footnote{भ \textendash\  विधीयते}भवेत् स्त्रियाः~॥~२६७

\renewcommand{\thefootnote}{3}\footnote{च \textendash\  तीव्रासूयितवचनाद्रोषाद्बहुशः प्रकम्पमानोष्ठी भ \textendash\  निद्राघूर्णितनयने रोषस्फुरितोष्ठकम्पितापाङ्गया ड \textendash\  निद्रासू\textendash\ यितवदना रोषाद्बहुशः प्रवेपमानाङ्गी}निद्राभ्यसूयितावेक्षणेन रोषप्रकम्पमानाङ्ग्या~।\\
साध्विति सुष्ठ्विति \renewcommand{\thefootnote}{4}\footnote{च \textendash\  वाक्यैः शोभनमित्यभिनयं युज्यात् भ \textendash\  वाक्यैः शोभन इत्येवं}वचनैः शोभत इत्येवमभिनेयम्~॥~२६८

बहुधा वार्यमाणोऽपि\renewcommand{\thefootnote}{5}\footnote{ड \textendash\  अवधीर्यमाणो} यस्तस्मिन्नेव दृश्यते\renewcommand{\thefootnote}{6}\footnote{ड \textendash\  तत्रैव हि तिष्ठति (भ \textendash\  दृश्यते)}~।\\
\renewcommand{\thefootnote}{7}\footnote{ड \textendash\  संहर्षात्तत्र मात्सर्याद्व्यलीकमुपजायते (च \textendash\  तु भवेत् स्त्रियाः) भ \textendash\  संहर्षेतु त्वमात्सर्ये\ldots ततः~।}संघर्षमत्सरात्तत्र व्यलीकं जायते स्त्रियाः~॥~२६९ 

कृत्वोरसि वामकरं दक्षिणहस्तं\renewcommand{\thefootnote}{8}\footnote{च \textendash\  रुषा विधुन्वाना} तथा विधुन्वन्त्या~।\\
चरणविनिष्ठम्भेन\renewcommand{\thefootnote}{9}\footnote{च \textendash\  विनिक्षेपेण च तस्मिन् कुर्वति साभिनयम्} च कार्योऽभिनयो व्यलीके तु\renewcommand{\thefootnote}{10}\footnote{भ \textendash\  व्यलीककृतः}~॥~२७०

जीवन्त्यां त्वयि जीवामि दासोऽहं त्वं च मे प्रिया~।\\
उक्त्वैवं योऽन्यथा कुर्याद्विप्रियं तत्र जायते\renewcommand{\thefootnote}{11}\footnote{च \textendash\  तद्विप्रियमिति स्त्रियाः भ \textendash\  एतद्वै विप्रियं भवेत्}~॥२७१

दूतीलेखप्रतिवचनभेदनैः क्रोधहसितरुदितैश्च~।\\
विप्रियकरणेऽभिनयः सशिरःकम्पैश्च कर्तव्यः\renewcommand{\thefootnote}{12}\footnote{च \textendash\  कम्पः प्रयोक्तव्यः}~॥~२७२}
\end{quote}

\hrule

\vspace{2mm}
\noindent
लक्षयति \underline{निद्राखेदालसग}तिमित्यादिना (सुरते) गतया निद्रया~। (\underline{असूयितेति}) यदसूयितेनावेक्षणं तेन~। (बहुधेति) यतो नायिकातो वार्यते तस्यामेव दृश्यत इति सम्बन्धः~। (संघर्षेति) सम्यक् कृतो घर्षः संघर्षः~। (\underline{उक्त्वैवमि}त्यादि)

\newpage
% द्वाविंशोऽध्याय २२१ 

\begin{quote}
{\na प्रतिपक्षसकाशात्तु यः सौभाग्यविकत्थनः\renewcommand{\thefootnote}{1}\footnote{ड \textendash\  विकत्थनैः}~।\\
उपसर्पेत् सचिह्नस्तु मन्युस्तत्रोपजायते\renewcommand{\thefootnote}{2}\footnote{च \textendash\  च मन्युस्तत्र भवेत् स्त्रियाः}~॥~२७३

वलयपरि\renewcommand{\thefootnote}{3}\footnote{च \textendash\  वर्तनेन च सशिथिल\ldots भ \textendash\  वर्तनेन च तथा समुत्क्षेपणेन}वर्तनैरथ सुशिथिलमुत्क्षेपणेन रशनायाः~।\\
मन्युस्त्वभिनेतव्यः सशङ्कितं बाष्पपूर्णाक्ष्या\renewcommand{\thefootnote}{4}\footnote{च \textendash\  मोक्षैश्च}~॥~२७४ 

दृष्ट्वा स्थितं प्रियतमं\\
\renewcommand{\thefootnote}{5}\footnote{भ \textendash\  साशङ्कं ड \textendash\  अशङ्कितं}सशङ्कितं सापराधमतिलज्जम्~। \\
ईर्ष्यावचनसमुत्थैः \\
 खेदयितव्यो \renewcommand{\thefootnote}{6}\footnote{भ \textendash\  अपि}ह्युपालम्भैः~॥~२७५

न च निष्ठुरमभिभाष्यो\renewcommand{\thefootnote}{7}\footnote{च \textendash\  अतिभाष्यो भ \textendash\  प्रवाच्यो} \\
 न चाप्यतिक्रोधनस्तु परिहासः\renewcommand{\thefootnote}{8}\footnote{भ \textendash\  अति\textendash\ क्रद्धया स परिभाष्यः च \textendash\  परिहार्यः}~।\\
बाष्पोन्मिश्रैर्वचनै\renewcommand{\thefootnote}{9}\footnote{भ \textendash\  वाक्यैः}\textendash \\
 रात्मोपन्याससंयुक्तैः~॥~२७६}
\end{quote}

\hrule

\vspace{2mm}
\noindent
प्रतिज्ञातस्यापरिपालनं विप्रियम्~। दूतलेखादिमुखेन यानि प्रसादनार्थं प्रति\textendash\ वचनानि तेषां भेदः दूषणमनङ्गीकरणम्~। (\underline{वलये}त्यादि) मन्युना तत्क्षण एव तनुत्वं गात्रे भवतीति वलयानां परिवर्तनं, वाससो रशनायाश्चोत्क्षेप इति~।\\

केचित्सहृदयास्त्वाहुः\textendash\ मन्युना सखीजनमध्य एवास्याबहुमानलक्षणा भवत्यप्रतिपत्तिरिति तत्कृतानि च वलयस्य परिवर्तनं योज(लोच?)न भ्रमणादिकं,

\newpage
% २२२ नाट्यशास्त्रम् 

\begin{quote}
{\na मध्याङ्गुल्यङ्गुष्ठाग्रविच्यवात्पाणिनोरसि कृतेन\renewcommand{\thefootnote}{1}\footnote{भ \textendash\  अधरस्थेन}~।\\
उद्वर्तितनेत्रतया \renewcommand{\thefootnote}{2}\footnote{भ \textendash\  तथा प्रततवीक्षणेनापि (ड \textendash\  णाच्चापि)}प्रततैरभिवीक्षणैश्चापि~॥~२७७ 

कटिहस्तविवर्तनया\renewcommand{\thefootnote}{3}\footnote{ड \textendash\  निविष्टतया न \textendash\  निवर्तनया} विच्छिन्नतया तथाञ्जलेःकरणात्\renewcommand{\thefootnote}{4}\footnote{भ \textendash\  स्फुरदधरतया तथा जडीभावात् च \textendash\  अङ्गु\textendash\ लैः करणात्}~।\\
\renewcommand{\thefootnote}{5}\footnote{भ \textendash\  अर्ध}मूर्धभ्रमणनिहञ्चितनिपातसंश्लेषेणाच्चापि\renewcommand{\thefootnote}{6}\footnote{च \textendash\  निहंचं वियतः संदर्शनाच्चापि भ \textendash\  निपात\textendash\  संदर्शनाच्चापि (न \textendash\  निधात ड \textendash\  नखादि)}~॥~२७८

अवहित्थवीक्षणाद्वा\renewcommand{\thefootnote}{7}\footnote{च \textendash\  क्षणेन च ड \textendash\  क्षणैश्चापि भ \textendash\  क्षणादपि} अङ्गुलिभङ्गेन\renewcommand{\thefootnote}{8}\footnote{भ \textendash\  भङ्गाच च \textendash\  साङ्गुलिभङ्गेन} तर्जनैर्ललितैः\renewcommand{\thefootnote}{9}\footnote{भ \textendash\  तर्जनाच्चापि ड \textendash\  लिखितैः}~।\\
\renewcommand{\thefootnote}{10}\footnote{भ \textendash\  एषोऽभिनयः प्रयोक्तव्यः (ड \textendash\  अस्य) च \textendash\  अयमेतेष्वभिनयः}एभिर्भावविशेषैरनुनयनेष्वभिनयः कार्यः~॥~२७९ 

शोभसे साधु दृष्टोऽसि गच्छ त्वं किं\renewcommand{\thefootnote}{11}\footnote{च \textendash\  कस्मात्} बिलम्बसे~।\\
मा\renewcommand{\thefootnote}{12}\footnote{च \textendash\  सां भ \textendash\  मा मा तिष्ठ}मां स्प्राक्षीः प्रिया यत्र\renewcommand{\thefootnote}{13}\footnote{भ \textendash\  तव या हृदयोत्थिता (च \textendash\  दि संस्थिता)} तत्र या ते हृदि स्थिता~॥~२८० 

गच्छेत्युक्त्वा परावृत्य विनिवृत्तान्तरेण तु\renewcommand{\thefootnote}{14}\footnote{च \textendash\  पुनः प्रतिनिवृत्य च भ \textendash\  परावृत्ता विनिवृत्योत्तरेण तु}~।\\
केन चिद्वचनार्थेन प्रहर्षं योजयेत्पुनः\renewcommand{\thefootnote}{15}\footnote{च \textendash\  संप्रयोजयेत्}~॥~२८१}
\end{quote}

\hrule

\vspace{2mm}
\noindent
रशनोत्क्षेपणं यन्त्रोत्क्षेपणं चेति~। \underline{मध्याङ्गुल्यङ्गुष्ठाग्रविच्यवा}दिति तत्स्थं मूलं व्यापारं स्मारयति~। (\underline{कटीति}) कटीपार्श्वगतस्य हस्तस्य विवर्तमानाया \underline{अञ्जले}\textendash\ र्हस्तस्य पताकाभ्यां तु \underline{संश्लेषादि}त्यस्य विश्लेषणम्, अवहित्थेन गाम्भीर्येण यद्वीक्षणं, तेन चायतं स्थानकमाक्षिप्तम्~। तत्र हि तस्य विनियोग उक्तः (१२\textendash\ १६४)~। निरपेक्षभावता संभवतीत्याह गच्छेत्युक्त्वेति~। विनिवृत्तिप्राधान्य\textendash

\newpage
% द्वाविंशोऽध्याय २२३ 

\begin{quote}
{\na रभसग्रहणाच्चापि\renewcommand{\thefootnote}{1}\footnote{भ \textendash\  वापि} हस्ते वस्त्रे च मूर्धनि~। \\
कार्यं \renewcommand{\thefootnote}{2}\footnote{ड \textendash\  प्रणमनं}प्रसादनं नार्या ह्यपराधं\renewcommand{\thefootnote}{3}\footnote{ड \textendash\  अपराद्धं\ldots च} समीक्ष्य तु~॥~२८२ 

हस्ते वस्त्रेऽथ केशान्ते नार्याप्यथ गृहीतया~। \\
कान्तमेवोपसर्पन्त्या\renewcommand{\thefootnote}{4}\footnote{भ \textendash\  अपस\textendash\ र्पन्त्या} कर्तव्यं मोक्षणं शनैः~॥~२८३ 

गृहीतयाथ केशान्ते हस्ते वस्त्रे\renewcommand{\thefootnote}{5}\footnote{च \textendash\  वस्त्रेषु}ऽथवा पुनः~।\\
\renewcommand{\thefootnote}{6}\footnote{भ यथा प्रियो न पश्येद्धि स्पर्शे ग्राह्यस्तथा स्त्रिया (ब \textendash\  र्शो) च \textendash\  ग्राह्यः स्पर्शस्तथा नार्या न पश्येद्दयितो यथा}हुं मुञ्चेत्युपसर्पन्त्या वाच्यः स्पर्शालसं प्रियः~॥~२८४ 

पादाग्रस्थितया नार्या \renewcommand{\thefootnote}{7}\footnote{च \textendash\  तथावा\textendash\ कुञ्चिताङ्गया (ढ \textendash\  थैवा)}किंचित्कुट्ठमितोत्कटम्~। \\
अश्वक्रान्तेन कर्तव्यं केशानां मोक्षणं शनैः~॥~२८५ 

\renewcommand{\thefootnote}{8}\footnote{च \textendash\  विमुच्य ढ \textendash\  आमुच्य}अमुच्यमाने केशान्ते \renewcommand{\thefootnote}{9}\footnote{ज \textendash\  नार्या सस्वेद}संजातस्वेदलेशया~।\\
हं हु मुञ्चापसर्पेति वाच्यः\renewcommand{\thefootnote}{10}\footnote{भ \textendash\  वासः} स्पर्शालसाङ्गया~॥~२८६ 

गच्छेति रोषवाक्येन गत्वा प्रतिनिवृत्य च~।\\
केन चिद्वचनार्थेन \renewcommand{\thefootnote}{11}\footnote{भ \textendash\  वासो च \textendash\  आलापं संप्रयोजयेत् ड \textendash\  चालापं योजये\textendash\ त्पुमान्}वाच्यं यास्यसि नेति च~॥~२८७ 

विधूननेन हस्तेन\renewcommand{\thefootnote}{12}\footnote{च \textendash\  हस्तस्य} हुंकारं \renewcommand{\thefootnote}{13}\footnote{ड \textendash\  स्त्री}संप्रयोजयेत्~।\\
स चावधूनने \renewcommand{\thefootnote}{14}\footnote{ड \textendash\  कार्यं शपथैर्वाच्यः भ \textendash\  कुर्या\textendash\ च्छपथान् व्याजमेव च च \textendash\  कार्यः शपथे}कार्यः शपथैर्व्याज एव च~॥~२८८ }
\end{quote}

\hrule

\vspace{2mm}
\noindent
मुत्तरं येनासौ विलम्बते~। \underline{अपराध}मिति, अल्पेऽपराधे मूर्धनि मध्ये हस्ततले \underline{भूषित}वस्त्रे~। \underline{व्याज} इति व्रतोपवासोऽद्य मया प्रस्तुत इत्यादि~।

\newpage
% २२४ नाट्यशास्त्रम् 

\begin{quote}
{\na अक्ष्णोः संवरणे\renewcommand{\thefootnote}{1}\footnote{ज \textendash\  संवननं म \textendash\  संवरणं} कार्यं पृष्ठतश्चोपगूहनम्~।\\
नार्यास्त्वपहृते वस्त्रे \renewcommand{\thefootnote}{2}\footnote{ज \textendash\  नीवी भ \textendash\  अधश्छादनं}दीपच्छादनमेव च~॥~२८९ 

तावत् \renewcommand{\thefootnote}{3}\footnote{भ \textendash\  छादयितव्यः}खेदयितव्यस्तु यावत्पादगतो\renewcommand{\thefootnote}{4}\footnote{ड \textendash\  हतो} भवेत्~।\\
ततश्चरणयोर्याते\renewcommand{\thefootnote}{5}\footnote{भ \textendash\  पाते तु च \textendash\  पाते} कुर्याद्द्तीनिरीक्षणम्~॥~२९० 

\renewcommand{\thefootnote}{6}\footnote{भ \textendash\  संयोज्य स्पर्शनं च \textendash\  स्पर्शस्य ग्रहणं कृत्वा ज \textendash\  उत्क्षिप्य}उत्थाप्यालिङ्गयेच्चैव नायिका नायकं ततः~।\\
\renewcommand{\thefootnote}{7}\footnote{भ \textendash\  उत्थाप्य विधिना}रतिभोगगता\renewcommand{\thefootnote}{8}\footnote{ज \textendash\  हता} हृष्टा शयनाभिमुखी व्रजेत्\renewcommand{\thefootnote}{9}\footnote{च \textendash\  भवेत्}~॥~२९१ 

एतद्गीतविधानेन सुकुमारेण योजयेत्~।\\
यदा श्रृङ्गारसंयुक्तं रतिसंभोगकारणम्~॥~२९२ 

यदा चाकाशपुरुष\renewcommand{\thefootnote}{10}\footnote{भ \textendash\  आकारमात्रेण ड \textendash\  पुरुषं}परस्थवचनाश्रयम्~। \\
भवेत्काव्यं\renewcommand{\thefootnote}{11}\footnote{ड \textendash\  कार्यं च काव्ये} तदा ह्येष \renewcommand{\thefootnote}{12}\footnote{भ \textendash\  कार्यस्तु}कर्तव्योऽभिनयः स्त्रिया~॥~२९३ 

यदन्तःपुरसंबन्धं \renewcommand{\thefootnote}{13}\footnote{ड \textendash\  कार्यं च \textendash\  कार्यं नाटकसंश्रयम्}काव्यं भवति नाटके~। \\
श्रृङ्गाररससंयुक्तं \renewcommand{\thefootnote}{14}\footnote{च \textendash\  तदापि}तत्राप्येष विधिर्भवेत्~॥~२९४}
\end{quote}

\hrule

\vspace{2mm}
\begin{sloppypar}
ननु वस्त्रापहारशयनादि रङ्गे निषिद्धमिति (किं) तेनोक्तेन, सत्यं लास्यदाने (स्थाने?) तु तस्योपयोगः~। इह तु कामोपचारप्रसङ्गादित्युक्तं, तदाह \underline{एतद्गीतविधानेनेति}, न च नाट्येऽस्य सर्वात्मनानुपयोगः तथाहि यथा भाणकादौ~। \underline{आकाशपुरुष} इति (प्रविष्टपात्रेण) भावितः प्रधानोऽप्रविष्टः पुरुषो भवति तदा इदमिदं मया दृश्यत इति ब्रूयात्~। एवं परस्थं वा यदा व्याख्या\textendash\ यते न च प्रत्यक्षत्वेनैतदुपयुज्यत इति दर्शयति \underline{यदन्तः पुरसंबन्ध}मिति~। 
\end{sloppypar}

\newpage
% द्वाविंशोऽध्यायः २२५ 

\begin{quote}
{\na न कार्यं शयनं रङ्गे नात्यधर्मं विजानता\renewcommand{\thefootnote}{1}\footnote{भ \textendash\  नाट्यधर्मी तु पश्यता}~।\\
केनचिद्वचनार्थेन \renewcommand{\thefootnote}{2}\footnote{भ \textendash\  तस्यच्छेदं प्रयोजयेत् च \textendash\  छेदमत्र}अङ्कच्छेदो विधीयते~॥~२९५ 

\renewcommand{\thefootnote}{3}\footnote{च \textendash\  यदा स्वपेदर्थवशात्}यद्वा शयीतार्थवशादेकाकी सहितोऽपि वा~।\\
\renewcommand{\thefootnote}{4}\footnote{च \textendash\  चुम्बनालिङ्गनादीनि रङ्गमध्ये न कारयेत् भ \textendash\  चुम्बनं लोकधर्मं च}चुम्बनालिङ्गनं चैव तथा गुह्यं च यद्भवेत्~॥~२९६ 

दन्तच्छेद्यं नखच्छेद्यं नीवीस्रंसनमेव च~। \\
\renewcommand{\thefootnote}{5}\footnote{च \textendash\  स्तनाधर}स्तनान्तरविमर्दं च रङ्गमध्ये न कारयेत्~॥~२९७ 

भोजनं सलिलक्रीडा तथा लज्जाकरं च यत्~।\\
एवंविधं भवेद्यद्यत्तत्तद्रङ्गे न कारयेत्\renewcommand{\thefootnote}{6}\footnote{भ \textendash\  योजयेत्}~॥~२९८ 

\renewcommand{\thefootnote}{7}\footnote{च \textendash\  पितृ}पितापुत्रस्नुषाश्वश्रू\renewcommand{\thefootnote}{8}\footnote{भ \textendash\  पूज्यैः}दृश्यं यस्मात्तु नाटकम्~। \\
तस्मादेतानि सर्वाणि वर्जनीयानि यत्नतः~॥~२९९ 

वाक्यैः सातिशयैः \renewcommand{\thefootnote}{9}\footnote{भ \textendash\  श्राव्यैः}श्रव्यैर्मधुरै\renewcommand{\thefootnote}{10}\footnote{ड \textendash\  न च च \textendash\  न तु}र्नातिनिष्ठुरैः~। \\
हितोपदेशसंयुक्तैस्तज्ज्ञः कुर्यात्तु\renewcommand{\thefootnote}{11}\footnote{च \textendash\  प्राज्ञः कुर्वति ड \textendash\  जननैस्तज्ज्ञैः कार्यं तु}नाटकम्~॥~३००}
\end{quote}

\hrule

\vspace{2mm}
\begin{sloppypar}
नन्वेवं सर्वमत्र प्राप्तमिति ननूत्पलचेटादौ दृश्यते शयनमित्याशङ्कयाह \underline{यद्वा शयी}तेति नात्र शयननिषेधस्तात्पर्यम्, अपि तु चुम्बनादिनिषेध इति भावः~। \underline{पितापुत्रेत्यादि}~। ततश्च रसो भज्येत, स हि साधारणान्योन्यानुप्रवेशमाण इति प्रत्यर्पि(प्रतिपदं?)वदामः~। एतत्सामान्याभिनयमध्ये वाचिकोऽप्यभिनयो\textendash
\end{sloppypar}

\lfoot{29}

\newpage
\lfoot{}
% २२६ नाट्यशास्त्रम् 

\begin{quote}
{\na [एवमन्तःपुरकृतः कार्यस्त्वभिनयो बुधैः]~। \\
समागमेऽथ नारीणां वाच्यानि मदनाश्रये~॥~३०१ 

प्रियेषु वचनानीह यानि तानि निबोधत~। \\
प्रियः\renewcommand{\thefootnote}{1}\footnote{भ \textendash\  कान्तस्तथा नाथो दासः}कान्तो विनीतश्च नाथः स्वाम्यथ जीवितम्~॥~३०२

नन्दनश्चेत्यभिप्रीते\renewcommand{\thefootnote}{2}\footnote{म \textendash\  अतिप्रीतो भ \textendash\  अभिहितो ड \textendash\  अभिप्रेतो च \textendash\  अभिजने} वचनानि भवन्ति हि~।\\
दुःशीलोऽथ\renewcommand{\thefootnote}{3}\footnote{भ \textendash\  स्यात्} दुराचारः शठो वामो विकत्थनः\renewcommand{\thefootnote}{4}\footnote{भ \textendash\  विरूपकः}~॥~३०३ 

निर्लज्जो निष्ठुरश्चैव \renewcommand{\thefootnote}{5}\footnote{च \textendash\  प्रियं क्रोधे\textendash\ ऽभिनिर्दिशेत् ड \textendash\  प्रायः क्रोधे विधीयते (ढ \textendash\  अभि) क्रोधवाक्या भवन्ति हि}प्रियः क्रोधेऽभिधीयते~। \\
यो विप्रियं न कुरुते \renewcommand{\thefootnote}{6}\footnote{च \textendash\  नानायुक्तं भ \textendash\  न चायुक्तप्रभाषणम्}न चायुक्तं प्रभाषते~॥~३०४ 

\renewcommand{\thefootnote}{7}\footnote{भ \textendash\  तथाऽवक्र}तथार्जवसमाचारः \renewcommand{\thefootnote}{8}\footnote{भ \textendash\  प्रिय इत्युच्यते हि यः (ड \textendash\  बुधैः) च \textendash\  प्रिय इति}स प्रियस्त्वभिधीयते~।\\
अन्यनारीसमुद्भूतं चिह्नं \renewcommand{\thefootnote}{9}\footnote{च \textendash\  यत्र}यस्य न दृश्यते~॥~३०५ 

\renewcommand{\thefootnote}{10}\footnote{भ \textendash\  देहे वाप्यधरे वापि}अधरे वा शरीरे वा स कान्त इति भाष्यते~।\\
\renewcommand{\thefootnote}{11}\footnote{ड \textendash\  संक्रुद्धोऽपि}संक्रुद्धेऽपि हि यो नार्या नोत्तरं प्रतिपद्यते\renewcommand{\thefootnote}{12}\footnote{ड \textendash\  नोत्तरोत्तरभाषणम् भ \textendash\  प्रतिभाषते}~॥~३०६ 

परुषं वा\renewcommand{\thefootnote}{13}\footnote{भ \textendash\  यो}न वदति\renewcommand{\thefootnote}{14}\footnote{भ \textendash\  स दास इति कीर्तितः}विनीतः साऽभिधीयते~।\\
हितैषी रक्षणे \renewcommand{\thefootnote}{15}\footnote{भ \textendash\  सक्तो च \textendash\  युक्तो}शक्तो न मानी न च मत्सरी~॥~३०७}
\end{quote}

\hrule

\vspace{2mm}
ऽस्तीत्याशयेनाह \underline{प्रियेषु वचना}नीति~। 

\newpage
% द्वाविंशोऽध्यायः २२७ 

\begin{quote}
{\na \renewcommand{\thefootnote}{1}\footnote{ड \textendash\  सम्यक्}सर्वकार्येष्वसंमूढः \renewcommand{\thefootnote}{2}\footnote{भ \textendash\  स स्वामी परिकीर्तितः ड \textendash\  यः स स्वामीति कीर्तितः}स नाथ इति संज्ञितः~। \\
सामदानार्थसंभोगैस्तथा लालनपालनैः~॥~३०८ 

नारीं \renewcommand{\thefootnote}{3}\footnote{भ \textendash\  सज्जयते ड \textendash\  संभजते}निषेवते यस्तु स \renewcommand{\thefootnote}{4}\footnote{डभ \textendash\  नाथ इति संज्ञितः}स्वामीत्यभिधीयते~। \\
नारीप्सितैरभिप्रायैर्निपुणं शयनक्रियाम्~॥~३०९ 

करोति यस्तु संभोगे \renewcommand{\thefootnote}{5}\footnote{ड \textendash\  जीवितः सोऽभिसंज्ञितः}स जीवितमिति\renewcommand{\thefootnote}{6}\footnote{च \textendash\  जीवित इति} स्मृतः~।\\
कुलीनो धृतिमान्दक्षो दक्षिणो वाग्विशारदः~॥~३१० 

श्लाघनीयः सखीमध्ये नन्दनः \renewcommand{\thefootnote}{7}\footnote{भ \textendash\  नाम स स्मृतः}सोऽभिधीयते~।\\
एते वचनविन्यासा रति\renewcommand{\thefootnote}{8}\footnote{ड \textendash\  स्मृति}प्रीतिकराः स्मृताः~॥~३११ 

तथा चाप्रीतिवाक्यानि \renewcommand{\thefootnote}{9}\footnote{भ \textendash\  वदतो}गदतो मे निबोधत~।\\
निष्ठुरश्चासहिष्णुश्च\renewcommand{\thefootnote}{10}\footnote{भ \textendash\  यो} मानी धृष्टो विकत्थनः~॥~३१२ 

\renewcommand{\thefootnote}{11}\footnote{भ \textendash\  उत्तरोत्तरवादी च}अनवस्थितचित्तश्च दुःशील इति स स्मृतः\renewcommand{\thefootnote}{12}\footnote{च \textendash\  कथ्यते}~।\\
ताडनं बन्धनं चापि यो विमृश्य समाचरेत्~॥~३१३ 

तथा परुषवाक्यश्च दुराचारः स तन्यते\renewcommand{\thefootnote}{13}\footnote{च \textendash\  उच्यते भ \textendash\  संज्ञितः}~।\\
वाचैव मधुरो यस्तु कर्मणा नोपपादकः\renewcommand{\thefootnote}{14}\footnote{भ \textendash\  नोपपादयेत्}~॥~३१४

\renewcommand{\thefootnote}{15}\footnote{च \textendash\  योषितां}योषितः किञ्चिदप्यर्थं स शठः परिभाष्यते~।}
\end{quote}

\hrule

\vspace{2mm}
\noindent
\underline{नारीप्सितैरि}ति न तु स्वोचितैरिति यावत्~। (रतिप्रीतीति) रतौ सत्यां 

\newpage
% २२८ नाट्यशास्त्रम् 

\begin{quote}
{\na वार्यते यत्र यत्रार्थे \renewcommand{\thefootnote}{1}\footnote{च \textendash\  तं तमेव भ \textendash\  तमेव कुरुतेऽसकृत्}तत्तदेव करोति यः~॥~३१५

\renewcommand{\thefootnote}{2}\footnote{भ \textendash\  भवेदभिनिवेशी च \textendash\  विपरीतनिषेवी}विपरीतनिवेशी च स वाम इति संज्ञितः~।\\
\renewcommand{\thefootnote}{3}\footnote{य \textendash\  सरसो व्रण}सरसव्रणचिह्नो यः स्त्रीसौभाग्यविकत्थनः~॥~३१६

\renewcommand{\thefootnote}{4}\footnote{च \textendash\  अभि}अतिमानी तथा स्तब्धो \renewcommand{\thefootnote}{5}\footnote{भ \textendash\  स विरूप}विकत्थन इति स्मृतः~।\\
वार्यमाणो दृढतरं यो नारीमुपसर्पति~॥~३१७

सचिह्नः सापराधश्च स निर्लज्ज इति स्मृतः\renewcommand{\thefootnote}{6}\footnote{भ \textendash\  विभाष्यते}~।\\
\renewcommand{\thefootnote}{7}\footnote{च \textendash\  सापराधस्तु}योऽपराद्धस्तु सहसा\renewcommand{\thefootnote}{8}\footnote{च \textendash\  रभसात् भ \textendash\  रहसा} नारीं सेवितुमिच्छति~॥~३१८

अप्रसादनबुद्धि\renewcommand{\thefootnote}{9}\footnote{भ \textendash\  वृत्तिः}श्च \renewcommand{\thefootnote}{10}\footnote{भ \textendash\  स धृष्ट इति संज्ञितः}निष्ठुरः सोऽभिधीयते~।\\
\renewcommand{\thefootnote}{11}\footnote{भ \textendash\  वाच्यावाच्येष्वविन्यासाः प्रियाप्रियविभा\textendash\ षिताः~। तां तामवस्थामासाद्य विपरीता भवन्ति हि}एते वचनविन्यासाः प्रियाप्रियविभाषिताः~॥~३१९

\renewcommand{\thefootnote}{12}\footnote{ड \textendash\  नानावस्थां समासाद्य विपरीतां समाचरेत्}नर्तकीसंश्रिताः कार्या बहवोऽन्येऽपि नाटके~।\\
एष गीतविधाने तु सुकमारे विधि\renewcommand{\thefootnote}{13}\footnote{च \textendash\  मारविधिः, ड \textendash\  मारो विधिः भ \textendash\  मारो भवेद्विधिः}र्भवेत्~॥~३२०

\renewcommand{\thefootnote}{14}\footnote{च \textendash\  शृङ्गारे ड \textendash\  शृङ्गाररति}श्रृङ्गाररससंभूतो रतिसंभोगखेदनः~।\\
यच्चैवाकाशपुरुषं परस्थवचनाश्रयम्~॥~३२१}
\end{quote}

\hrule

\vspace{2mm}
\noindent
या प्रीतिः परितोषः, रतौ क्रोधोऽपि हि भवति परितोषश्च सुतादावुभयमप्यु\textendash\ पात्तम्~। \underline{सहसे}त्यप्रसाद्य~। \underline{अन्येऽपीति} उदाहरणमेतदित्यर्थः~।\\

पूर्वोक्तमेवोपसंहरति \underline{एष गीतविधान} इति श्रृङ्गारे रसे \underline{आकाशपुरुषादौ}~।

\newpage
% द्वाविंशोऽध्यायः २२९ 

\begin{quote}
{\na श्रृङ्गार \renewcommand{\thefootnote}{1}\footnote{ड \textendash\  रस}एवं वाच्यं स्यात्तत्राप्येष क्रमो\renewcommand{\thefootnote}{2}\footnote{ड \textendash\  विधिः} भवेत्~।\\
यद्वा पुरुषसंबन्धं कार्यं भवति नाटके~॥~३२२

श्रृङ्गार\renewcommand{\thefootnote}{3}\footnote{ड \textendash\  रति}रससंयुक्तं तत्राप्येष क्रमो भवेत्~।\\
एवमन्तःपुरगतः प्रयोज्योऽभिनयो भवेत्~॥~३२३

दिव्याङ्गनानां तु विधिं व्याख्यास्याम्यनुपूर्वशः~।\\
नित्यमेवोज्ज्वलो वेषो नित्यं प्रमुदितं मनः~॥~३२४

\renewcommand{\thefootnote}{4}\footnote{च \textendash\  सुखकालः सदा नित्यं}नित्यमेव सुखः कालो \renewcommand{\thefootnote}{5}\footnote{च \textendash\  देवीनां}देवानां ललिताश्रयः~।\\
\renewcommand{\thefootnote}{6}\footnote{च \textendash\  ईर्ष्या न स्यान्न}न चेर्ष्या नैव च क्रोधो नासूया न प्रसादनम्\renewcommand{\thefootnote}{7}\footnote{च \textendash\  प्रसाधनम्}~॥~३२५

\renewcommand{\thefootnote}{8}\footnote{ड \textendash\  दृश्यते दिव्यपुंसां हि (च \textendash\  देव)}दिव्यानां दृश्यते पुंसां श्रृङ्गारे योषितां तथा\renewcommand{\thefootnote}{9}\footnote{च \textendash\  प्रति}~।\\
ये भावा मानुषाणां स्युर्यदङ्गं यच्च चेष्टितम्~॥~३२६

\renewcommand{\thefootnote}{10}\footnote{भ \textendash\  तत्सर्वं मानुषीं प्राप्य कार्यं दिव्यैरपि द्विजाः (च \textendash\  कर्तव्यं दैवतैरपि)}सर्वं तदेव कर्तव्यं दिव्यैर्मानुषसङ्गमे~।\\
यदा मानुषसंभोगो\renewcommand{\thefootnote}{11}\footnote{च \textendash\  संयोगो} दिव्यानां योषितां भवेत्~॥~३२७}
\end{quote}

\hrule

\vspace{2mm}
\begin{sloppypar}
\noindent
\underline{एव}मिति प्रीत्या कोपेन वा, यदा \underline{वाच्यवचनं} स्यात् \underline{तत्राप्येष} एव वचन \underline{क्रमः~। पुरुषसंबन्धमिति} प्रत्यक्षपुरुषयुक्तमित्यर्थः~। दिव्यवेश्याङ्गनाभिस्तु राज्ञां भवति संभोग इति तत्र सामान्याभिनयमाह नित्यमेवेत्यादि~। अत्र श्लोकद्वये यद्यपीत्यध्याहारेण \underline{ये भावा} इत्यत्र च तथापीत्यध्याहारेण सङ्गतिः कार्या~।
\end{sloppypar}


\newpage
% २३० नाट्यशास्त्रम् 

\begin{quote}
{\na \renewcommand{\thefootnote}{1}\footnote{च \textendash\  सर्व एव तदा कार्याः ड \textendash\  तदा सर्वं प्रकर्तव्यं}तदा सर्वाः प्रकर्तव्या \renewcommand{\thefootnote}{2}\footnote{च \textendash\  भावा मानुष\textendash\ संश्रयाः}ये भावा मानुषाश्रयाः]\\
शापभ्रंशात्तु दिव्यानां तथा चापत्यलिप्सया\renewcommand{\thefootnote}{3}\footnote{च \textendash\  अङ्गनानां यदा भवेत् भ \textendash\  शापभ्रंशावतीर्णानां तथा चापत्यमि\textendash\ च्छताम्}~॥~३२८

\renewcommand{\thefootnote}{4}\footnote{च \textendash\  मानुषैः सहसंयोगः}कार्यो मानुषसंयोगः श्रृङ्गाररससंश्रयः\renewcommand{\thefootnote}{5}\footnote{ड \textendash\  तथाचैवोपसर्पणम् च \textendash\  देव}~।\\
पुष्पैर्भूषणजैः शब्दैरदृश्या\renewcommand{\thefootnote}{6}\footnote{च \textendash\  विप्र भ \textendash\  अदृश्यासु विलोभयेत् भ \textendash\  अदृश्यात्रापि या भवेत्}पि प्रलोभयेत्~॥~३२९

पुनः संदर्शनं दत्त्वा क्षणादन्तरिता\renewcommand{\thefootnote}{7}\footnote{च \textendash\  अन्तर्हिता भ \textendash\  पुनरन्तर्हिता} भवेत्~।\\
\renewcommand{\thefootnote}{8}\footnote{भ \textendash\  दिव्याभरण}वस्त्राभारणामाल्याद्यैर्लेखसंप्रेषणैरपि~॥~३३०

ईदृशैरुपचारैस्तु\renewcommand{\thefootnote}{9}\footnote{च \textendash\  अभ्युपगमैः भ \textendash\  अभ्युपायैस्तु} \renewcommand{\thefootnote}{10}\footnote{य \textendash\  समं मान्यस्तु}समुन्माद्यस्तु नायकः~।\\
उन्मादनात्समुद्भूतः\renewcommand{\thefootnote}{11}\footnote{च \textendash\  समुत्पन्नः} कामो रतिकरो भवेत्~॥~३३१

स्वभावोपगतो यस्तु नासावत्यर्थभाविकः\renewcommand{\thefootnote}{12}\footnote{च \textendash\  भावकः ड \textendash\  लोके नास्त्यसौ डम्बभावितः भ \textendash\  गतं चापि नात्यर्थमधिको भवेत्}~।\\
एवं राजोपचारो हि कर्तव्योऽभ्यन्तराश्रयः~॥~३३२}
\end{quote}

\hrule

\noindent
विप्रलम्भो हि जीवितेऽभिमान इति भावः~। \underline{समुन्माद्य} इत्यत्र हेतुमाह \underline{उन्मादनादिति} एतच्च विक्रमोर्वश्यां स्फुटमेव दृश्यतां इति शिवम्~।

\newpage
% द्वाविंशोऽध्यायः २३१ 

\begin{quote}
{\na बाह्य\renewcommand{\thefootnote}{1}\footnote{च \textendash\  अभ्युपचारं च व्याख्यास्याम्यथ}मप्युपचारं तु प्रवक्ष्याम्यथ वैशिके~।}
\end{quote}

\begin{center}
\textbf{इति भारतीये नाट्यशास्त्रे सामान्याभिनयो}\\
\textbf{नामाध्यायो द्वाविंशः\renewcommand{\thefootnote}{2}\footnote{ड \textendash\  चतुर्विंशोऽध्यायः च \textendash\  त्रयोविंशतितमः भ \textendash\  एकविंशोऽध्यायः}~।}
\end{center}

\hrule

\begin{quote}
{\qt सामान्याभिनयः सोऽयं ग्रन्थिस्थानेषु सङ्गतः~।\\
कृतोऽभिनवगुप्तेन शिवस्मरणशालिना~॥}
\end{quote}

\begin{center}
इति महामाहेश्वराभिनवगुप्तविरचितायां नाट्यवेदवृत्तावभिनवभारत्यां सामान्याभिनयो द्वाविंशः~॥\\

\vspace{4cm}
\rule{0.2\linewidth}{0.5pt}
\end{center}

\newpage
\thispagestyle{empty}

\begin{center}
\textbf{\large श्रीः}\\

\vspace{2mm}
\textbf{\huge नाट्यशास्त्रम्}\\

\vspace{2mm}
त्रयोविंशोऽध्यायः.\renewcommand{\thefootnote}{1}\footnote{भ \textendash\  द्वीविंशः, य \textendash\  चतुर्विंशः, जादिषान्तेषु\textendash\ पञ्चविंशः}\\

\rule{0.2\linewidth}{0.5pt}
\end{center}

\begin{quote}
{\na विशेषयेत्कलाः सर्वा यस्मात्तस्मात्तु वैशिकः\renewcommand{\thefootnote}{2}\footnote{ब \textendash\  वैशिकम् च \textendash\  वैशिके}~।\\
\renewcommand{\thefootnote}{3}\footnote{न \textendash\  वेश्योपचरणाद्वापि ड \textendash\  वेशोपचारतो वापि}वेशोपचारे साधुर्वा वैशिकः\renewcommand{\thefootnote}{4}\footnote{बभ \textendash\  वैशिकं} परिकीर्तितः\renewcommand{\thefootnote}{5}\footnote{बभ \textendash\  कीर्तितम् ड \textendash\  समुदाहृतः (ढ \textendash\  तम्)}~॥~१

\renewcommand{\thefootnote}{6}\footnote{च \textendash\  यस्तु}यो हि सर्व\renewcommand{\thefootnote}{7}\footnote{ड \textendash\  गुण} कलोपेतः सर्वशिल्पविचक्षणः\renewcommand{\thefootnote}{8}\footnote{च \textendash\  प्रयोजकः}~।\\
स्त्रीचित्तग्रहणाभिज्ञो\renewcommand{\thefootnote}{9}\footnote{च \textendash\  ग्राहकश्चैव} वैशिकः स भवेत्पुमान्~॥~२}
\end{quote}

\hrule

\begin{center}
अभिनवभारती\textendash\ त्रयोविंशोऽध्यायः
\end{center}

\begin{quote}
{\qt पुंसामशक्तापि तदेकभावमादर्शयन्ती बहुभावपूर्णा~।\\
वेश्यामतिर्निर्वृतिधाम यत्स्था तस्मै नमस्तात्परमेश्वराय~॥}
\end{quote}

सामान्याभिनयशेष एव वैशिक इत्युपसंहृतं वृत्तपूर्वेऽध्याये\textendash\ बाह्य मप्युपचारं तु प्रवक्ष्याम्यथ वैशिके\textendash\ इति, वैशिको वक्तव्य इति सङ्गति~। तदुपक्रममाणो\ldots माद्येन तावन्निरुक्तमाह \underline{विशेषये}दिति~। विशेषणं जानाति, तेनातिकामयतीति च धात्वर्थो लक्षणमिति हि तद्विदो वैशेषिका (वैशिकाः?)~। (वैशिकः) वेश्याकामुकः, स च सर्वान् कामान् विशेष\textendash\ यत्यतिवैदग्ध्यात्~। अथ व्याकरणोचितमस्य निर्वचनमाह \underline{वेश्योपचारे साधु}र्वेति~। वेशो वेश्या उपचारस्तत्रभव इत्यर्थः~। भवार्थमेव विभजति साधुरित्य\textendash\ नेन~। तस्मादसौ कलासु विशेषज्ञ इत्याह \underline{यो हि सर्वकलोपेत} इति~।

\newpage
\fancyhead[CO]{त्रयोविंशोऽध्यायः}
% त्रयोविंशोऽध्यायः २३३

\begin{quote}
{\na गुणास्तस्य तु विज्ञेयाः स्वशरीरसमुत्थिताः\renewcommand{\thefootnote}{1}\footnote{च \textendash\  समुद्भवाः}~।\\
आहार्याः सहजाश्चैव त्रयस्त्रिंशत्समासतः~॥~३\\
शास्त्रविच्छिल्प\renewcommand{\thefootnote}{2}\footnote{च \textendash\  शील}सम्पन्नो रूपवान् प्रियदर्शनः~।\\
विक्रान्तो \renewcommand{\thefootnote}{3}\footnote{च \textendash\  वृत्तिमान् ज \textendash\  मतिमान्}धृतिमाश्चैव\renewcommand{\thefootnote}{4}\footnote{य \textendash\  वाग्मी} वयोवेष\renewcommand{\thefootnote}{5}\footnote{य \textendash\  गुण}कुलान्वितः~॥~४\\
सुरभिर्मधुरस्त्यागी सहिष्णुरविकत्थनः~।\\
अशङ्कितः प्रियाभाषी चतुरः शुभदः\renewcommand{\thefootnote}{6}\footnote{च \textendash\  सुभगः य \textendash\  शुभगः} शुचिः~॥~५\\
कामोपचारकुशलो\renewcommand{\thefootnote}{7}\footnote{च \textendash\  कृतज्ञो} दक्षिणो देशकालवित्~।\\
\renewcommand{\thefootnote}{8}\footnote{च \textendash\  अदीन}अदीनवाक्यः\renewcommand{\thefootnote}{9}\footnote{य \textendash\  दक्षश्च श्रुतिमान्मतिमांस्तथा}स्मितवान् वाग्मी दक्षः प्रियंवदः~॥~६\\
\renewcommand{\thefootnote}{10}\footnote{च \textendash\  अलुब्धः}स्त्रीलुब्धः संविभागी च श्रद्धधानो दृढस्मृतिः\renewcommand{\thefootnote}{11}\footnote{ढ \textendash\  व्रतः}~।\\
गम्यासु चाप्यविस्रम्भी मानी चेति\renewcommand{\thefootnote}{12}\footnote{च \textendash\  चैव} \renewcommand{\thefootnote}{13}\footnote{ढ \textendash\  स}हि वैशिकः~॥~७\\
\renewcommand{\thefootnote}{14}\footnote{य \textendash\  अनुरक्तः}अनुयुक्तः शुचिर्दक्षो\renewcommand{\thefootnote}{15}\footnote{ढ \textendash\  दान्तः} दक्षिणः प्रतिपत्तिमान्~।\\
भवेच्चित्राभिधायी\renewcommand{\thefootnote}{16}\footnote{च \textendash\  छिद्रा\textendash\ पिधायी (प \textendash\  भि ढ \textendash\  वि) ज \textendash\  छिद्रपिधायी न \textendash\  चित्रविधायी ब \textendash\  छिद्रावघाती} च \renewcommand{\thefootnote}{17}\footnote{च \textendash\  वयस्याः\ldots गुणाः}वयस्यस्तस्य\renewcommand{\thefootnote}{18}\footnote{ढ \textendash\  षड्गुणाः} तदगुणः~॥~८\\
विज्ञानगुणसम्पन्ना \renewcommand{\thefootnote}{19}\footnote{भ \textendash\  कथिका ज \textendash\  रङ्गोपजी\textendash\ विनी चापि प्रतिपत्तिविचक्षणा (ढ \textendash\  ना)}कथिनी लिङ्गिनी तथा~।}
\end{quote}

\hrule

\vspace{2mm}
{\footnotesize आहार्याः शास्त्रज्ञतादयः~। सहजा रूपलावण्यादयः~। \underline{गम्यासु चाप्यविस्रम्भीति} सहसैव नाभियुक्तः, अपि तु स्फुटभावमन्वेष्यति~। \underline{भवेत् चित्राभिधायीति} वक्रोक्तिकुशलः~। \underline{तस्येति} वैशिकस्य~। कथिनी बृहत्कथादिलम्भ [न] कथनाकर्णनकुशला~। लिङ्गिनी चित्रकरी~। प्रातिवेश्या}

\lfoot{30}

\newpage
% २३४ नाट्यशास्त्रम् 

\begin{quote}
{\na \renewcommand{\thefootnote}{1}\footnote{ढ \textendash\  प्रतिवेश्या}प्रातिवेश्या सखी दासी कुमारी कारुशिल्पिनी\renewcommand{\thefootnote}{2}\footnote{ढ \textendash\  दारुशिल्पिका}~॥~९

धात्री पाषण्डिनी चैव तथा रङ्गोपजीविनी\renewcommand{\thefootnote}{3}\footnote{ढ \textendash\  दृत्यस्त्वीक्षणिकास्तथा}~।\\
प्रोत्साहनेऽथ\renewcommand{\thefootnote}{4}\footnote{च \textendash\  नेषु} कुशला\renewcommand{\thefootnote}{5}\footnote{च \textendash\  कुशलां\ldots कथां\ldots दक्षिणां\ldots ज्ञाम्\ldots हां\ldots त्रां\ldots दूतीं\ldots धां कुर्यात्} मधुरकथा दक्षिणाथ\renewcommand{\thefootnote}{6}\footnote{ढ \textendash\  च}कालज्ञा~॥~१०

\renewcommand{\thefootnote}{7}\footnote{य \textendash\  लटहा}लडहा संवृतमन्त्रा दूती त्वेभिर्गुणैः कार्या\renewcommand{\thefootnote}{8}\footnote{च \textendash\  दूतीमेवंविधां कुर्यात्}~।\\
तयाप्युत्साहनं\renewcommand{\thefootnote}{9}\footnote{य \textendash\  प्रोत्साहनं} कार्यं \renewcommand{\thefootnote}{10}\footnote{च \textendash\  अनुरागानुकीर्तनम् ढ \textendash\  नानादर्शन}नानादर्शितकारणम्~॥~११

यथोक्तकथनं चैव तथा भावप्रदर्शनम्~।\\
\renewcommand{\thefootnote}{11}\footnote{भ \textendash\  जननी च \textendash\  मृजारूप\textendash\ वयोऽपेतमर्थवन्तं जडं तथा~। दूतं वाप्यथ दूतीं वा न कुर्याद्वैशिकाश्रये}न जडं \renewcommand{\thefootnote}{12}\footnote{ड \textendash\  रूपवन्तं च}रूपसम्पन्नं नार्थवन्तं न चातुरम्~॥~१२}
\end{quote}

\hrule

\vspace{2mm}
\noindent
निकटावसथस्था~। पाषण्डिनी व्रतिनी~। रङ्गोपजीविनी रजकस्त्री चारणस्त्री~। \underline{प्रोत्साहने कुश}लेत्यादीनि सर्वासां विशेषणानि~। उतः सह प्रोत्साहः प्रोत्सा\textendash\ हनमिति द्वौ णिचौ~। प्रोत्साहयति नायिका तु नायकस्तया प्रोत्साहयति संमुखीकारयतीत्यर्थः~।\\

तस्या व्यापारान्तरमाह \underline{यथोक्तेति} संमुखीकरणं सन्देशार्पणं काम्याया भावपरीक्षणं चेति द्वितयमनया कार्यमित्यर्थः~। \underline{जडः} करणीयं न शक्नोति कर्तुं प्रत्युत्पन्नमतिसाध्यानि कृत्यानीत्याह~। रूपेणार्थेन वा युक्तः स्वार्थ\textendash\ तामाहरेत्~। आतुरो हि दृश्यमान एव जुगुप्सां जनयति, स च रतेर्निरपेक्ष इत्यातुरो न कामदूतः~। नानादर्शितकारणं कृत्वा भोत्साहनमित्युक्तं तानि

\newpage
% त्रयोविंशोऽध्यायः २३५ 

\begin{quote}
{\na दूतं \renewcommand{\thefootnote}{1}\footnote{ज \textendash\  वापि हि दूतीं वा}वाप्यथवा दूतीं बुधः कुर्यात्कदाचन\renewcommand{\thefootnote}{2}\footnote{य \textendash\  कथञ्चन}~।\\
कुलभोगधनाधिक्यैः कृत्वाऽधिकविकत्थनम्\renewcommand{\thefootnote}{3}\footnote{ढ \textendash\  आधिक्यं कार्यं चैव विकत्थनम् (च \textendash\  वावि) य \textendash\  विकल्पनम्}~॥~१३

\renewcommand{\thefootnote}{4}\footnote{ज \textendash\  आभिः}दूती निवेदये\renewcommand{\thefootnote}{5}\footnote{य \textendash\  काम्यं ढ \textendash\  कार्यमर्थानां च प्रभाषणम्}त्काममर्थांश्चैवानुवर्णयेत्~।\\
\renewcommand{\thefootnote}{6}\footnote{ढ \textendash\  नवकाम}न चाकामप्रवृत्तायाः \renewcommand{\thefootnote}{7}\footnote{च \textendash\  क्रुद्धायां वा समागमः}क्रुद्धाया वापि सङ्गमः~॥~१४ 

\renewcommand{\thefootnote}{8}\footnote{य \textendash\  नानापायैः ढ \textendash\  नानोपायैः}नानुपायः प्रकर्तव्यो दूत्या हि\renewcommand{\thefootnote}{9}\footnote{च \textendash\  अभिपुरुषाश्रये य \textendash\  तु} पुरुषाश्रयः~। \\
उत्सवे रात्रिसञ्चार उद्याने \renewcommand{\thefootnote}{10}\footnote{य \textendash\  ज्ञाति}मित्रवेश्मनि~॥~१५ 

धात्रीगृहेषु सख्या वा \renewcommand{\thefootnote}{11}\footnote{च \textendash\  क्षये सखीगेहे}तथा चैव निमन्त्रणे\renewcommand{\thefootnote}{12}\footnote{ढ \textendash\  निमन्त्रणैः}~।\\
व्याधितव्यपदेशेन शून्यागारनिवेशने\renewcommand{\thefootnote}{13}\footnote{ढ \textendash\  समाश्रये}~॥~१६

\renewcommand{\thefootnote}{14}\footnote{ढ \textendash\  एवं समागमः कार्यो नृणां}कार्यः समागमो नॄणां \renewcommand{\thefootnote}{15}\footnote{य \textendash\  एषु च \textendash\  एष}स्त्रीभिः प्रथमसङ्गमे~।\\
एवं समागमं कृत्वा सोपायं विधिपूर्वकम्\renewcommand{\thefootnote}{16}\footnote{य \textendash\  नानोपायविधानजम् प \textendash\  स्वोपायं विधिसंमितम्}~॥~१७ 

\renewcommand{\thefootnote}{17}\footnote{य \textendash\  अनुरक्तं विरक्तं च चिह्नैः समुपलक्षयेत् ड \textendash\  चिह्नैः समुपलक्षयेत्}अनुरक्तां विरक्ता वा \renewcommand{\thefootnote}{18}\footnote{ब \textendash\  लिङ्गाचारैस्तु}लिङ्गाकारैस्तु लक्षयेत्~।}
\end{quote}

\hrule

\vspace{2mm}
\noindent
कारणान्याह \underline{कुलभोगे}त्यादि~। \underline{उत्सव} इति स्वगृह एव रात्रिचारप्रधानो य उत्सवः~। प्रथमसङ्गम इति गान्धर्वविवाहे वेश्यापुनर्भूसङ्गमे चेत्यर्थः~।

\newpage
% २३६ नाट्यशास्त्रम् 

\begin{quote}
{\na स्वभावभावातिशयैर्नारी या मदनाश्रया\renewcommand{\thefootnote}{1}\footnote{ड \textendash\  या नारी मदनार्दिता}~॥~१८

करोति निश्रृतां\renewcommand{\thefootnote}{2}\footnote{च \textendash\  अनिभृतां ज \textendash\  अनिभृतं} लीलां नित्यं\renewcommand{\thefootnote}{3}\footnote{च \textendash\  ज्ञेया} सा मदनातुरा~।\\
\renewcommand{\thefootnote}{4}\footnote{च \textendash\  गुणान् सखीमुद्वदति य \textendash\  गुणान् सखीनामाख्याति}सखीमध्ये गुणान् ब्रूते स्वधनं च प्रयच्छति\renewcommand{\thefootnote}{5}\footnote{य \textendash\  प्रददाति च}~॥~१९

\renewcommand{\thefootnote}{6}\footnote{य \textendash\  संपूजयति च पूजयत्यपि मित्राणि शत्रुपक्षं च निन्दति}पूजयत्यस्य मित्राणि द्वेष्टि शत्रुजनं सदा\renewcommand{\thefootnote}{7}\footnote{ड \textendash\  तथा}~।\\
\renewcommand{\thefootnote}{8}\footnote{य \textendash\  समागमं प्रार्थयते हृष्टे हृष्यति चाधिकम् च \textendash\  चैवहि हृष्यति ड \textendash\ दृष्ट्वा हृष्यति चाधिकम् ड \textendash\  च}गमागमे सखीनां या हृष्टा भवति चाधिकम्~॥~२०

तुष्यत्यस्य कथाभिस्तु\renewcommand{\thefootnote}{9}\footnote{प \textendash\  अन्यकथाभिश्च} सस्नेहं च निरीक्षते~।\\
सुप्ते \renewcommand{\thefootnote}{10}\footnote{च \textendash\  च}तु पश्चात् स्वपिति \renewcommand{\thefootnote}{11}\footnote{य \textendash\  प्रथमं परिबुध्यते~। परिक्लेशांश्च सहते चुम्बिता प्रतिचुम्बति}चुम्बिता प्रतिचुम्बति~॥~२१

उत्तिष्ठत्यपि पूर्वं च तथा क्लेशसहापि च\renewcommand{\thefootnote}{12}\footnote{ढ \textendash\  वा ज \textendash\  रतिक्लेशं सहत्यपि}~।\\
\renewcommand{\thefootnote}{13}\footnote{य \textendash\  सभा दुःखे सुखे च स्यान्न क्रोधमुपयाति च}उत्सवे मुदिता या च व्यसने या च दुःखिता~॥~२२

एवंविधैर्गुणैर्युक्ता त्वनुरक्ता तु सा स्मृता\renewcommand{\thefootnote}{14}\footnote{य \textendash\  ज्ञेया रक्तेति वैशिकैः ढ \textendash\  यानुरक्ता तु सा भवेत् च \textendash\  रक्ता ज्ञेया हि वैशिकी}~।\\
विरक्तायास्तु चिह्नानि\renewcommand{\thefootnote}{15}\footnote{ड \textendash\  लिङ्गानि} चुम्बिता नाभिचुम्बति\renewcommand{\thefootnote}{16}\footnote{य \textendash\  चुम्बितास्यं प्रमार्जति (च \textendash\  र्ष्टि हि)}~॥~२३}
\end{quote}

\hrule

\vspace{2mm}
(\underline{स्वभावेति}) स्वभावे भावे सुरते येऽतिशया नखरदनसहिष्णुतादयस्तै\textendash\ रुपलक्षिता अनुरक्तेति संबन्धः~। \underline{मित्राणि शत्रुजन}मिति नायकस्येति शेषः~। पश्चात्संवेशनं पूर्वमभ्युत्थानं च~। तेन विना किमत्र सुखमिति दर्शयति \underline{उत्सवे}

\newpage
% त्रयोविंशोऽध्यायः २३७ 

\begin{quote}
{\na \renewcommand{\thefootnote}{1}\footnote{य \textendash\  करोत्यनिष्टां च कथां}अनिष्टां च कथां ब्रूते प्रियमुक्तापि कुप्यति~।\\
\renewcommand{\thefootnote}{2}\footnote{य \textendash\  मित्राणि चास्य प्रद्वेष्टि शत्रुपक्षं प्रशं\textendash\ सति (ड \textendash\  तस्य शत्रुं)}प्रद्वेष्टि चास्य मित्राणि भजतेऽरिजनं तथा~॥~२४

शेते पराङ्मुखी \renewcommand{\thefootnote}{3}\footnote{ड \textendash\  चैव शय्यायां (य \textendash\  स्था)}चापि शयने पूर्वशायिनी~।\\
सुमहत्युपकारे\renewcommand{\thefootnote}{4}\footnote{ड \textendash\  चारे}ऽपि न तोषमुपयाति च\renewcommand{\thefootnote}{5}\footnote{य \textendash\  तुष्यति कथंचन}~॥~२५

क्लेशं न सहते \renewcommand{\thefootnote}{6}\footnote{च \textendash\  वापि}चापि तथा कुप्यत्यकारणात्\renewcommand{\thefootnote}{7}\footnote{ज \textendash\  अकारणे}~।\\
\renewcommand{\thefootnote}{8}\footnote{ज \textendash\  यस्यामेवं विकारास्तु (ड \textendash\  प्र प \textendash\  प्रकारस्तु)}या स्यादेवंप्रकारा तु विरक्तां तां विनिर्दिशेत्~॥~२६

\renewcommand{\thefootnote}{9}\footnote{य \textendash\  हृदयग्रहणं चैव तथा ड \textendash\  हृदयग्रहणानि स्युः व्यापारस्य विचेष्टितम् (च \textendash\  अस्यां व्यापारचेष्टितम्)}हृदयग्रहणोपायमस्या व्यापारचेष्टितम्~।\\
अर्थप्रदर्शनं चैव \renewcommand{\thefootnote}{10}\footnote{च \textendash\  तदा सद्भावदर्श\textendash\ नम्~। अर्थोपन्यास एवं स्यादपन्यासस्तथैव हि (ड \textendash\  अर्थदानं\ldots \ldots च) भ \textendash\  परि\textendash\ त्यागोऽथ नितरां}उपदानं पुनर्भवेत्~॥~२७

\renewcommand{\thefootnote}{11}\footnote{ड \textendash\  व्याधितायाः परित्यागो भावोपक्षेप एव च (न \textendash\  पीडि) \ldots दारिद्र्यात्\ldots }अकारणमुपन्यासस्तथैव व्याधितापि च~।}
\end{quote}

\hrule

\vspace{2mm}
\noindent
\underline{व्यसन} इति नायकस्य~। \underline{अनिष्टां कथां ब्रूते} इति पर्वतादपि (पूर्वकृतामिति ?)~। हृदयं गृह्यते यैरुपायैः अस्या इति रक्ताया \underline{व्यापारचेष्टित}मिति तदीय हृदय\textendash\ ग्रहणव्यापारतात्पर्यत्वं कामतन्त्रे चेष्टितम्~। अर्थस्य प्रदर्शनमिदं ममास्तीति~। \underline{उपन्यासः} (उपदानं ?) अर्थस्य, दास्यामीति~। \underline{उपन्यासः} अन्यमुखेन काचि\textendash\ दनुरक्तस्याङ्गनास्तीति कथनम्~। (\underline{व्याधितेति}) विचित्रा आधयो यस्य तस्य भावः~। ततो हेतोरसेवनम्~। विचित्राभिप्रायदर्शनव्याजेन तन्निकटादपसर्प\textendash

\newpage
% २३८ नाट्यशास्त्रम्

\begin{quote}
{\na व्याजात्त्यागोऽथ निकटाद्भावोपक्षेप एव च~॥~२८

दारिद्र्याद्वयाधितो दुखात्पारुष्याद्दुःश्रवा\renewcommand{\thefootnote}{1}\footnote{य \textendash\  दुःश्रुतात्}त्तथा~।\\
प्रवासगमनादेव\renewcommand{\thefootnote}{2}\footnote{ड \textendash\  माना च \textendash\  गमनोन्मानात्} ह्यतिलोभादतिक्रमात्~॥~२९

\renewcommand{\thefootnote}{3}\footnote{ड \textendash\  अतिवेलागम\textendash\ त्वाच्च}अतीवाभिगमाच्चापि तथा विप्रियकारणात्\renewcommand{\thefootnote}{4}\footnote{ड \textendash\  सेवनात्}~।\\
एभिः स्त्री पुरुषो वापि कारणैस्तु विरज्यते~॥~३०

भावग्राहीणि नारीणां कार्याणि मदनाश्रये~।\\
\renewcommand{\thefootnote}{5}\footnote{ड \textendash\  या न च प्रीयते न \textendash\  यैर्न कुप्यति या नारी क्रुद्धो वापि प्रसीदति}तुष्टिमेति यथा नारी प्राप्यते पुरुषैरथ~॥~३१

लुब्धामर्थप्रदानेन कलाज्ञानेन पण्डिताम्~।\\
चतुरां लडहत्वेन\renewcommand{\thefootnote}{6}\footnote{ड \textendash\  क्रीडनत्वेन च \textendash\  चैव चातुर्यैः} ह्यनुवृत्त्या च\renewcommand{\thefootnote}{7}\footnote{ड \textendash\  तु म \textendash\  तु कामिनीम्} मानिनीम्~॥~३२

[भूषणग्रहणाच्चापि श्रृङ्गारमुखरो\renewcommand{\thefootnote}{8}\footnote{ढ \textendash\  मुखतो} भवेत्\renewcommand{\thefootnote}{9}\footnote{श्लोकार्धं ज \textendash\  ज्ञढडमातृकास्वेव वर्तते}]~।\\
पुरुषद्वेषिणीमिष्टैः कथायोगैरुपक्रमैः\renewcommand{\thefootnote}{10}\footnote{ज \textendash\  कथाभिः परिसान्त्वयेत्}~॥~३३}
\end{quote}

\hrule

\vspace{2mm}
\noindent
णम् इति यावत् (\underline{व्याजात्परित्याग} इति)~। एषोऽन्यत्र रागीत्यन्यमुखेनाभिधानं भावोपक्षेपः~। रक्ताया अप्येतानि विरागकारणानीत्याह \underline{दारिद्र्या}दित्यादिभ्यः, अपत्यमरणादेः, अश्राव्यत्वं यद्वचनं पारुष्यं ततो यत एवं तेना\textendash\ स्यापि रागं रक्षेदिति \underline{भावग्राहीणी}ति~। \underline{प्राप्यत} इति सेव्यत इति यावत्~। \underline{पुरुषैरि}ति कुशलैरिति भावः~।\\

\begin{sloppypar}
अस्याः कथं तुष्टिरित्याह \underline{लुब्धामि}त्यादि~। \underline{पण्डिता}मिति कलाविदाम्~। \underline{लडह}त्वेन प्रागल्भ्येन~।
\end{sloppypar}

\newpage
% त्रयोविंशोऽध्यायः २३९ 

\begin{quote}
{\na \renewcommand{\thefootnote}{1}\footnote{ज \textendash\  बालामपि क्रीडनकैर्भीरुमाश्वासचाटुभिः}उपक्रीडनकैर्बालां \renewcommand{\thefootnote}{2}\footnote{ड \textendash\  भीतां}भीरुमाश्वासनेन च~।\\
गर्वितां नीचसेवाभिरुदात्तां शिल्पदर्शनैः~॥~३४

सर्वासामेव नारीणां त्रिविधा प्रकृतिः स्मृता~।\\
उत्तमा मध्यमा \renewcommand{\thefootnote}{3}\footnote{य \textendash\  चैव तृतीया चाधमा स्मृता}नीचा वेश्यानां तु स्वभावजाः\renewcommand{\thefootnote}{4}\footnote{ज \textendash\  निबोधत}~॥~३५

या विप्रियेऽपि तिष्ठन्तं प्रियं वदति नाप्रियम्\renewcommand{\thefootnote}{5}\footnote{य \textendash\  न वदत्यप्रियं प्रियम्}~।\\
\renewcommand{\thefootnote}{6}\footnote{य \textendash\  न चिरं क्रोधमायाति दोषं प्रच्छादयत्यपि (न \textendash\  षान्) ड \textendash\  अदीर्घ}न दीर्घरोषा च तथा \renewcommand{\thefootnote}{7}\footnote{ड \textendash\  कला\textendash\ शिल्प}कलासु च विचक्षणा~॥~३६

\renewcommand{\thefootnote}{8}\footnote{य \textendash\  काम्यते पुरुषैर्या तु कुलभोगधनाधिकैः (च \textendash\  शोभा ड \textendash\  आदिकैः)}शीलशोभाकुलाधिक्यैः\renewcommand{\thefootnote}{9}\footnote{म \textendash\  आधिक्याः} पुरुषैर्या च काम्यते~।\\
कुशला कामतन्त्रेषु दक्षिणा रूपशालिनी\renewcommand{\thefootnote}{10}\footnote{च \textendash\  धारिणी}~॥~३७

गृह्णाति कारणाद्रोषं \renewcommand{\thefootnote}{11}\footnote{य \textendash\  गतेर्ष्या प्रब्रवीति न \textendash\  दोषं गतेर्ष्या प्रब्रवीति}विगतेर्ष्या ब्रवीति च~।\\
कार्यकालविशेषज्ञा \renewcommand{\thefootnote}{12}\footnote{य \textendash\  सुभगा}सुरूपा सा स्मृतोत्तमा~॥~३८

\renewcommand{\thefootnote}{13}\footnote{य \textendash\  पुंसः कामयते या तु पुरुषैर्या च काम्यते (च \textendash\  तु)}पुरुषैः काम्यते या तु तथा कामयते च तान्~।\\
कामोपचारकुशला प्रतिपक्षाभ्यसूयिनी\renewcommand{\thefootnote}{14}\footnote{ज \textendash\  अत्यसूयिका}~॥~३९}
\end{quote}

\hrule

\vspace{2mm}
\underline{नीचसेवाभि}रिति पादस्पर्शनादिभिः~। \underline{शिल्पदर्शनै}रिति विस्मयहेतुभि\textendash\ रित्यर्थः~। प्रतिपदमशक्यो भेदसंग्रह इत्याशयेनाह \underline{सर्वासामेवेति} उत्तममध्य\textendash\ माधमानां प्रत्येकमिति यावत्~।

\newpage
% २४० नाट्यशास्त्रम् 

\begin{quote}
{\na ईर्ष्यातुरा\renewcommand{\thefootnote}{1}\footnote{य \textendash\  च च वा ब \textendash\  ईर्ष्यार्थनात्तु} त्वनिभृता \renewcommand{\thefootnote}{2}\footnote{च \textendash\  क्षण}क्षीणक्रोधाति\renewcommand{\thefootnote}{3}\footnote{य \textendash\  क्रोधा च ड \textendash\  क्रोधाभि}गर्विता~।\\
क्षणप्रसादा\renewcommand{\thefootnote}{4}\footnote{य \textendash\  प्रसाद्या} या चैव सा नारी मध्यमा स्मृता~॥~४०

\renewcommand{\thefootnote}{5}\footnote{च \textendash\  अस्थाने}अस्थानकोपना या तु \renewcommand{\thefootnote}{6}\footnote{य \textendash\  दुःशीला चाति}दुष्टशीलातिमानिनी~।\\
चपला परुषा चैव\renewcommand{\thefootnote}{7}\footnote{च \textendash\  परुषा प्रतिकूला च} दीर्घरोषाधमा स्मृता~॥~४१ 

सर्वासां नारीणां यौवन\renewcommand{\thefootnote}{8}\footnote{च \textendash\  लाभा भवन्ति (यन \textendash\  लाभात्) ड \textendash\  लीलाश्चतस्रः स्युः}भेदाः स्मृतास्तु चत्वारः~।\\
नैपथ्यरूपचेष्टागुणेन\renewcommand{\thefootnote}{9}\footnote{य \textendash\  गुणैस्तु ड \textendash\  शोभागुणैस्तु} श्रृङ्गारमासाद्य~॥~४२

पीनोरुगण्डजघनाधरस्तनं\renewcommand{\thefootnote}{10}\footnote{म \textendash\  जघनं स्तनाधरं} कर्कशं रतिमनोज्ञम्~।\\
\renewcommand{\thefootnote}{11}\footnote{न \textendash\  सुरतं प्रति सोत्साहं प्रथमं तद्यावनं विद्यात् (ज \textendash\  ज्ञेयम् )}श्रृङ्गारसमुत्साहं प्रथमं तद्यौवनं ज्ञेयम्~॥~४३

गात्रं पूर्णावयवं पीनौ च पयोधरौ नतं\renewcommand{\thefootnote}{12}\footnote{य \textendash\  कृशं} मध्यम्~।\\
कामस्य \renewcommand{\thefootnote}{13}\footnote{च \textendash\  पारभूता}सारभूतं यौवनमेतद् द्वितीयं तु~॥~४४

सर्वश्री\renewcommand{\thefootnote}{14}\footnote{य \textendash\  सम्भृतं रतिकरमुन्मादनं बहुगुणाढयम् (च \textendash\  संपूर्णं)}संयुक्तं रतिकरणोत्पादनं रतिगुणाढ्यम्~।\\
कामाप्यायितशोभं यौवनमेतत्तृतीयं तु~॥~४५}
\end{quote}

\hrule

\vspace{2mm}
(\underline{सर्वासामिति}) प्रथमं यौवनं यावद्विंशति~। एवं त्रिंशच्चत्वारिंशत्पञ्चाशदिति विभागः~। अन्ये तु षोडशपञ्चविंशतिपञ्चत्रिंशत्पञ्चचत्वारिंशदिति विभाग\textendash\ माहुः~।

\newpage
% त्रयोविंशोऽध्यायः २४१ 

\begin{quote}
{\na \renewcommand{\thefootnote}{1}\footnote{अयं श्लोको भबयोर्न वर्तते}नवयौवने \renewcommand{\thefootnote}{2}\footnote{ड \textendash\  अप्यतीते}व्यतीते तथा द्वितीये तृतीयके वापि\renewcommand{\thefootnote}{3}\footnote{ढ \textendash\  तृतीयजे चापि}~।\\
\renewcommand{\thefootnote}{4}\footnote{प \textendash\  कामस्य}श्रृङ्गारशत्रुभूतं यौवनमेतच्चतुर्थं तु~॥~४६

\renewcommand{\thefootnote}{5}\footnote{य \textendash\  निर्मांस}अम्लानगण्डजघनाधरस्तनं \renewcommand{\thefootnote}{6}\footnote{च \textendash\  शुष्कलम्बितकपोलम् ड \textendash\  गलितगात्र\textendash\ लावण्यम् (ढ \textendash\  शेष)}किञ्चिदूनलावण्यम्~।\\
\renewcommand{\thefootnote}{7}\footnote{य \textendash\  कामे मन्दोत्साहं ब \textendash\  कामं प्रति सोत्साहः ड \textendash\  कामे च निरुत्साहं}कामं प्रति नोच्छवासं यौवनमेतच्चतुर्थं तु~॥~४७

नात्यर्थं क्लेशसहा न \renewcommand{\thefootnote}{8}\footnote{न \textendash\  च कुप्यति हर्षमेति सा पत्युः (त्यै\textendash\ य)}कुप्यति न हृष्यति स्त्रीभ्यः\renewcommand{\thefootnote}{9}\footnote{ड \textendash\  प्रति स्त्रीषु}~।\\
\renewcommand{\thefootnote}{10}\footnote{ज \textendash\  सौम्यगुणेष्वासक्ता (न \textendash\  भौग्ध्य प \textendash\  मौर्ख्य)}सौख्यगुणेष्ववसक्ता नारी नवयौवना ज्ञेया~॥~४८

किंचित्करोति मानं किंचित्क्रोधं\renewcommand{\thefootnote}{11}\footnote{य \textendash\  कोपं} च मत्सरं चैव~।\\
क्रोधे च भवति तूष्णीं यौवनभेदे द्वितीये तु~॥~४९

रतिसंभोगे दक्षा प्रतिपक्षासूयिनी\renewcommand{\thefootnote}{12}\footnote{ब \textendash\  सूयिका ढ \textendash\  प्रतिपन्नासूयिनी} रतिगुणाढया\renewcommand{\thefootnote}{13}\footnote{च ड \textendash\  गुणाढ्या}~।\\
\renewcommand{\thefootnote}{14}\footnote{च \textendash\  अतिधृत}अनिश्रृतगर्वितचेष्टा \renewcommand{\thefootnote}{15}\footnote{न \textendash\  वेषा}नारी ज्ञेया तृतीये तु~॥~५०

\renewcommand{\thefootnote}{16}\footnote{य \textendash\  पुरुषग्रहण (च \textendash\  प्रहण?)}चित्तग्रहणसमर्था कामाभिज्ञा \renewcommand{\thefootnote}{17}\footnote{च \textendash\  हि ड \textendash\  अप्यमत्सर}त्वमत्सरोपेता~।\\
\renewcommand{\thefootnote}{18}\footnote{ड \textendash\  अविरहमिच्छति नित्यं नारी ज्ञेया चतुर्थे तु}अविरहितमिच्छति सदा पुरुषं नारी चतुर्थे तु~॥~५१}
\end{quote}

\hrule

\vspace{2mm}
एषूपचारभेदमाह \underline{नात्यर्थ}मिति क्लेशः दशनादिकृत्यं नातीव सहते~। \underline{रतिगुणाढ्या} कामतन्त्रप्रयोगप्रगल्भेत्यर्थः~। \underline{अविरहितमिति} भावैस्त्रिभिरपि~।

\lfoot{31}

\newpage
\lfoot{}
% २४२ नाट्यशास्त्रम् 

\begin{quote}
{\na यौवन\renewcommand{\thefootnote}{1}\footnote{य \textendash\  लम्भा ह्येते}भेदास्त्वेते विज्ञेया नाटकेषु चत्वारः~।\\
पुनरेव \renewcommand{\thefootnote}{2}\footnote{य \textendash\  च पुरुषगुणान् कामिततन्त्रे (ढ \textendash\  तु)}तु पुरुषाणां च कामतन्त्रे प्रवक्ष्यामि~॥~५२

चतुरोत्तमौ तु मध्यस्तथा\renewcommand{\thefootnote}{3}\footnote{य \textendash\  तथाऽधमः संप्रवृत्तकः} च नीचः \renewcommand{\thefootnote}{4}\footnote{ड \textendash\  प्रवर्तकः}प्रवृत्तकश्चैव~।\\
\renewcommand{\thefootnote}{5}\footnote{य \textendash\  स्त्रीणां प्रयोग}स्त्रीसंप्रयोगविषये ज्ञेयाः पुरुषास्त्वमी पञ्च~॥~५३

\renewcommand{\thefootnote}{6}\footnote{य \textendash\  दुःखक्लेश\textendash\ सहिष्णुः प्रियवाग्दाता}समदुःखक्लेशसहः प्रणयक्रोधप्रसादने कुशलः~।\\
\renewcommand{\thefootnote}{7}\footnote{ड \textendash\  प्रत्युपचारे निपुणो (च \textendash\  रत्युप)}योऽर्थी नात्मच्छन्दो दक्षश्चतुरः \renewcommand{\thefootnote}{8}\footnote{ड \textendash\  तु}स बोद्धव्यः~॥~५४

यो विप्रियं न कुरुते \renewcommand{\thefootnote}{9}\footnote{य \textendash\  धीरोदात्तः प्रियंवदो मानी}नार्याः किंचिद्विरागसंज्ञातम्\renewcommand{\thefootnote}{10}\footnote{प \textendash\  संजननम्}~।\\
\renewcommand{\thefootnote}{11}\footnote{य \textendash\  अज्ञातहृदय\textendash\ तत्त्वः}अज्ञातेप्सितहृदयः \renewcommand{\thefootnote}{12}\footnote{च \textendash\  ज्ञेयः स्मृतिमान्}स्मृतिमान्धृतिमान् स तु ज्येष्टः\renewcommand{\thefootnote}{13}\footnote{ब \textendash\  तथा चैव}~॥~५५

मधुरस्त्यागी रागं न\renewcommand{\thefootnote}{14}\footnote{ढ \textendash\  नयति च\ldots नापि}याति मदनस्य चापि वशमेति~।\\
अवमानितश्च नार्या विरज्यते चोत्तमः स पुमान्15~॥~५६

\renewcommand{\thefootnote}{16}\footnote{प \textendash\  सर्वावस्थास्वपि सद्}सर्वार्थैर्मथ्यस्थो भावग्रहणं करोति यो नार्याः\renewcommand{\thefootnote}{17}\footnote{च \textendash\  नारीणाम्}~।\\
किंचिद्दोषं दृष्ट्वा विरज्यते मध्यमः स भवेत्\renewcommand{\thefootnote}{18}\footnote{ड \textendash\  पुरुषः}~॥~५७}
\end{quote}

\hrule

\vspace{2mm}
उपचारार्थं पुरुषभेदो ज्ञेय इत्याशयेनाह \underline{चतुरोत्तमा}वित्यादि~। \underline{पञ्च} क्रमेण लक्षयति \underline{समदुःख} इति~। \underline{अज्ञातेप्सितहृदय} इति गम्भीर इत्यर्थः~। \underline{ज्येष्ठ}

\newpage
% त्रयोविंशोऽध्यायः २४३ 

\begin{quote}
{\na \renewcommand{\thefootnote}{1}\footnote{अयं श्लोको भ \textendash\  मातृकायां नास्ति}काले दाता ह्यवमानितोऽपि न क्रोधमतितरामेति~।\\
दृष्ट्वा\renewcommand{\thefootnote}{2}\footnote{ड \textendash\  अप्यलीक}व्यलीकमात्रं विरज्यते मध्यमोऽयमपि~॥~५८ 

अवमानितोऽपि नार्या\\
निर्लज्जतयाभ्युपैत्यविकृतास्यः\renewcommand{\thefootnote}{3}\footnote{य \textendash\  निर्लज्जः समुपस\textendash\ र्पति तथैव ड\textendash\ उपसर्पति य एनाम्}~।\\
\renewcommand{\thefootnote}{4}\footnote{य \textendash\  अन्यनरं संक्रान्तामारूढस्नेहभावतया (न\textendash\ श्च) ड \textendash\  संक्रान्तान्तमन्यः भ \textendash\  संक्रान्तान्तिकमन्यव्यावृत्तस्नेहभावरसः}अन्यतरं संक्रान्तां\\
स्नेहपरावृत्तभावश्च~॥~५९

अभिनवकृते\renewcommand{\thefootnote}{5}\footnote{य \textendash\  कृत} व्यलीके प्रत्यक्षं रज्यते दृढतरं यः~।\\
\renewcommand{\thefootnote}{6}\footnote{य \textendash\  सुहृदापि}मित्रैर्निवार्यमाणो विज्ञेयः सोऽधर्मः पुरुषः\renewcommand{\thefootnote}{7}\footnote{य \textendash\  नाम}~॥~६०

\renewcommand{\thefootnote}{8}\footnote{ब \textendash\  अविगलित ज \textendash\  अवगणित}अविगणितभयामर्षो मूर्खप्रकृतिः प्रसक्तहासश्च\renewcommand{\thefootnote}{9}\footnote{ड \textendash\  प्रकृष्टभावश्च (ड \textendash\  हृष्ट) ब \textendash\  प्रमुक्तहासश्च}~।\\
एकान्तदृढग्राही निर्लज्जः कामतन्त्रेषु~॥~६१

रतिकलहसंप्रहारे\textendash \\
ष्वकर्कशः\renewcommand{\thefootnote}{10}\footnote{च \textendash\  हारेषु कर्कशः} क्रीडनीयकः स्त्रीणाम्\renewcommand{\thefootnote}{11}\footnote{न \textendash\  नीयकस्त्रीणाम्}~। \\
एवंविधस्तु\renewcommand{\thefootnote}{12}\footnote{न \textendash\  विधिज्ञैः च॒ विधिज्ञः प्रवृत्ततो नाम विज्ञेयः}तज्ज्ञै \\
र्विज्ञेयः संप्रवृत्तस्तु\renewcommand{\thefootnote}{13}\footnote{ड \textendash\  स्यात्}~॥~६२}
\end{quote}

\hrule

\vspace{2mm}
\noindent
इति यावत्~। नन्वन्यदीयं कार्यवशात्संक्रामेदित्याह \underline{स्नेहेति} अन्यदीयेन स्नेहेन प्रेम्णा परावृत्तः तद्विषयो भावोऽभिप्रायो यस्याः~।

\newpage
% २४४ नाट्यशास्त्रम् 

\begin{quote}
{\na नानाशीलाः\renewcommand{\thefootnote}{1}\footnote{ढ \textendash\  लीलाः} स्त्रियो ज्ञेया गूढार्थहृदयेप्सिताः\renewcommand{\thefootnote}{2}\footnote{न \textendash\  गूढार्थहृदयाश्च ताः}~।\\
विज्ञाय तु यथा\renewcommand{\thefootnote}{3}\footnote{च \textendash\  तत्त्वं}सत्त्वमुपसर्पेत्तथैव ताः\renewcommand{\thefootnote}{4}\footnote{च \textendash\  तत\textendash\ श्च ताः न \textendash\  तु ताः पुनः}~॥~६३

भावाभावौ विदित्वाऽथ तत्र\renewcommand{\thefootnote}{5}\footnote{य \textendash\  तु ततः ड \textendash\  च ततः} तैस्तैरुपक्रमैः~।\\
पुमानुपचरेन्नारीं कामतन्त्रं समीक्ष्य तु~॥~६४

साम \renewcommand{\thefootnote}{6}\footnote{ड \textendash\  चैव}चोपप्रदानं च \renewcommand{\thefootnote}{7}\footnote{य \textendash\  दण्डो भेदः}भेदो दण्डस्तथैव च~।\\
उपेक्षा चैव कर्तव्या नारीणां विषयं प्रति~॥~६५

तवास्मि मम चैवासि\renewcommand{\thefootnote}{8}\footnote{य \textendash\  चैव त्वं} दासोऽहं त्वं च मे प्रिया~।\\
आत्मोपक्षेपणकृतं\renewcommand{\thefootnote}{9}\footnote{न \textendash\  युतं} \renewcommand{\thefootnote}{10}\footnote{य \textendash\  तत्सामेति हि संज्ञितम् ड \textendash\  तत्सामेत्यभिधीयते}यत्तत्सामेति कीर्तितम्~॥~६६

काले काले प्रदातव्यं धनं विभवमात्रया~।\\
\renewcommand{\thefootnote}{11}\footnote{य \textendash\  निमित्तान्तरसंभूतं ज \textendash\  नियुक्तान्तर ड \textendash\  सनिमि\textendash\ त्तान्तर}यन्निमित्तान्तरकृतं प्रदानं नाम तत्स्मृतम्\renewcommand{\thefootnote}{12}\footnote{ड \textendash\  भवेत्}~॥~६७}
\end{quote}

\hrule

\vspace{2mm}
ननु किमनेन स्त्रीणां भावज्ञानेनेत्याह \underline{नानाशीला} इति अर्थनमर्थः अभि\textendash\ प्रायः गूढाभिप्रायं हृदयमासाम्~। \underline{यथासत्त्व}मिति यथाशयम्~। \underline{भावाभावा}विति अनुरागविरागौ \underline{नारीणां विषये} बन्धनं स्वीकारः तं प्रतीति तस्मिन्साध्ये सामादयः उपेक्षान्ताः पञ्चोपाया इत्यर्थः~। तान् क्रमेण व्याचष्टे \underline{तवा}स्मीत्यादि~। \underline{आत्मन उपक्षेपो} निजभावप्रदर्शनम्, \underline{काले} दिवसे \underline{दातव्य}मिति 

\newpage
% त्रयोविंशोऽध्यायः २४५

\begin{quote}
{\na भेदः स्यात्तत्प्रियस्येह सोपायं दोषदर्शनम्~।\\
बन्धनं ताडनं \renewcommand{\thefootnote}{1}\footnote{य \textendash\  वाऽपि}चापि दण्ड इत्यभिधीयते~॥~६८

मध्यस्था\renewcommand{\thefootnote}{2}\footnote{न \textendash\  मानयेत्}मानयेत्साम्ना \renewcommand{\thefootnote}{3}\footnote{ड \textendash\  लुब्धामर्थ न \textendash\  लुब्धां चार्थ}लुब्धां चोपप्रदानतः~।\\
अन्यावबद्धभावां च भेदेन\renewcommand{\thefootnote}{4}\footnote{प \textendash\  भावायां भेदनं} प्रतिपादयेत्~॥~६९

दुष्टाचारे समारब्धे \renewcommand{\thefootnote}{5}\footnote{ड \textendash\  मध्यमावे}त्वन्यभावसमुत्थिते~।\\
दण्डः पातयितव्यस्तु\renewcommand{\thefootnote}{6}\footnote{य \textendash\  हि} मृदुताडनबन्धनैः~॥~७०

[\renewcommand{\thefootnote}{7}\footnote{अयं ज \textendash\  मातृकायामेव दृश्यते}नायकः पुरुषो वाच्यो नायिकां ताडयेच्च ताम्~।\\
ताडयेत्तां बुधो नारीं रज्ज्वा वेणुदलेन वा~॥]~७१

सामादीनां प्रयोगे तु परिक्षीणे यथाक्रमम्~।\\
न \renewcommand{\thefootnote}{8}\footnote{न \textendash\  भावाद्वशमापन्नां (च \textendash\  भवेद्वशं) ड \textendash\  भवेद्वशगा या तु}स्याद्या च समापन्ना तामुपेक्षेत बुद्धिमान्~॥~७२}
\end{quote}

\hrule

\vspace{2mm}
\noindent
नियमे न सति निमित्तविशेषकृतेन प्रमोदव्यसनादिनिबन्धनेन दानेन वर्तयती\textendash\ त्यर्थः~। तस्या योऽन्यः प्रियः तस्य दोषास्तथा दृश्यन्ते यथा तया सत्यत्वेन प्रतीयेरन्निति भेदः तदाह \underline{सोपायमिति}~। चतुर्णामुपायानां स्वं स्वं विषयमाह \underline{मध्यस्थामिति} किञ्चित् स्निह्यन्तीमित्यर्थः~। \underline{आनयेत्} स्वीकृर्यात्~। \underline{प्रतिपादये}दिति आत्मनि संमुखीभावं गमयेत्~। \underline{दुष्टाचार} इति देशात्पलायनं पुरुषा\textendash\ न्तरगृह एव वास इत्यादिके~। तत्रापि च स्त्रीषु निरपेक्षः स्यादिति \underline{मृदुताड}नबन्धनैरिति~। उपेक्षाया विषयमाह \underline{सामादीनामिति} दण्डेनापि हि तत्साम्मु\textendash

\newpage
% २४६ नाट्यशास्त्रम् 

\begin{quote}
{\na मुखरागेण नेत्राभ्यां विज्ञेयो\renewcommand{\thefootnote}{1}\footnote{न \textendash\  त्वङ्ग} भावचेष्टितैः~।\\
द्वेष्यो वापि प्रियो वापि मध्यस्थो वापि योषिताम्~॥~७३

अर्थहेतोस्तु वेश्यानां \renewcommand{\thefootnote}{2}\footnote{ड \textendash\  अप्रियो यदि वा प्रियः}प्रियो वा यदि वाप्रिया~।\\
\renewcommand{\thefootnote}{3}\footnote{ड \textendash\  गम्यो हि पुरुषो नित्यं (ढ \textendash\  नाम्नो भ \textendash\  नरो भवति नित्यं तु}गम्य एव नरो नित्यं मुक्त्वा दिव्यनृपस्त्रियः~॥~७४

द्वेष्यं तु प्रियमित्याहुः प्रियं प्रियतरं \renewcommand{\thefootnote}{4}\footnote{ज \textendash\  अप्यप्रियं}तथा~।\\
सुशीलमिति\renewcommand{\thefootnote}{5}\footnote{न \textendash\  इव य \textendash\  सुशील इति}\renewcommand{\thefootnote}{5}\footnote{न \textendash\  इव य \textendash\  सुशील इति} दुश्शीलं गुणाढ्यमिति निर्गुणम्\renewcommand{\thefootnote}{6}\footnote{ड \textendash\  निर्गुणं गुणवानिति}~॥~७५

प्रहसन्ती च नेत्राभ्यां यं दृष्ट्वोत्फुल्लतारका~।\\
प्रसन्नमुखरागा\renewcommand{\thefootnote}{7}\footnote{च \textendash\  रागा}च्च लक्ष्यते भावरूपणैः~॥~७६ 

\renewcommand{\thefootnote}{8}\footnote{ड \textendash\  मातृकायामयं न वर्तते}भावाभावौ विदित्वैव निरस्तैस्तैरुपक्रमैः~।\\
यत्नादुपचरेन्नारीं कामतन्त्रं प्रतीक्ष्य तु~॥~७७}
\end{quote}

\hrule

\vspace{2mm}
\noindent
ख्यं त्यजति या तस्याः किञ्चिदुपेक्षेत बुद्धिमानित्युक्तम्~। किं संबुध्यत इत्याह \underline{मुखरागेणेत्यादि}~। वेश्याचित्तं तु दुर्लक्षमिति प्रयत्नपरीक्ष्यमित्याशयेनाह \underline{अर्थहेतोस्त्विति}~। किं सर्वासां वेश्यानामयं विधिः नेत्याह \underline{मुक्त्वा दिव्यनृपस्त्रिय} इति~। ननु वचने वेश्याहृदयमुपलक्षेतेत्याह \underline{द्वेष्ये तु प्रिय इत्याहुरिति}~। तुरप्यर्थे~।\\

ननु किमस्वकार्यहृदया एव ताः, नेत्याह \underline{उपचारबलत्वादिति} अर्थकाम\textendash\ मयादित्यर्थः मध्ये वा नो सेवया (?)~। ननु यद्येवंभूताः कथं ताः काम्यन्ते

\newpage
% त्रयोविंशोऽध्यायः २४७ 

\begin{quote}
{\na उपचार\renewcommand{\thefootnote}{1}\footnote{य \textendash\  छलत्वाच्च ड \textendash\  फलत्वाच्च}बलत्वाच्च विप्रलम्भात्तथैव च\renewcommand{\thefootnote}{2}\footnote{च \textendash\  लम्भकृतेन च}~।\\
तासु निष्पद्यते कामः काष्ठादग्निरिवोत्थितः\renewcommand{\thefootnote}{3}\footnote{भ \textendash\  इव हुताशनः}~॥~७८

योषितामुपचारोऽयं यथोक्तो वैशिकाश्रयः\renewcommand{\thefootnote}{4}\footnote{ज \textendash\  आश्रये}~।\\
कार्यः प्रकरणे सम्यग्यथायोगं\renewcommand{\thefootnote}{5}\footnote{ब \textendash\  योग्यं} च नाटके~॥~७९

एवं \renewcommand{\thefootnote}{6}\footnote{न \textendash\  वेशोप}वेश्योपचारोऽयं तज्ज्ञैः कार्यो\renewcommand{\thefootnote}{7}\footnote{न \textendash\  ज्ञेयो} द्विजोत्तमाः~।\\
अत ऊर्ध्वं\renewcommand{\thefootnote}{8}\footnote{च \textendash\  परं}प्रवक्ष्यामि प्रकृतीनां तु लक्षणम्\renewcommand{\thefootnote}{9}\footnote{ज \textendash\  चित्रस्याभिनयं प्रति}~॥~८०

इति भारतीये नाट्यशास्त्रे त्रयोविंशोऽध्यायः\renewcommand{\thefootnote}{10}\footnote{ब भ य न \textendash\  मातृकासु अध्यायविभाग एव नास्ति म \textendash\  त्रयोविंशः च \textendash\  चतुर्विंशः अन्यस्मिन् पञ्चविंशः}~।}
\end{quote}

\hrule

\vspace{2mm}
\noindent
जनैरित्याशङ्कयावृत्त्यैतदेवाह \underline{उपचारबलत्वाच्चेति}~। यतो हृदयग्रहणोचितमुपचारं निन्दती मध्ये च विप्रलम्भयन्ती तस्मात् काम उत्सुको भवति~। कामाभिनिवेशी स इत्युक्तम्~। \underline{काष्ठादग्निरिति} प्रत्युत दुश्चिकित्स इत्यर्थः~।\\

वैशिकपुरुषाधिकारे प्रवृत्तमध्यायं प्रकृते उपयोजयति \underline{योषितामिति}~। \underline{नाटक} इति दिव्यवेश्यानां तत्र भावात् पताकानायकादिगतत्वेन चेति शिवम्~।

\begin{quote}
{\qt अध्यायो वैशिकः सोऽयं त्रयोविंशतिपूरणः~।\\
कृतोऽभिनवगुप्तेन भद्रग्रन्थिपदक्रमः~॥}
\end{quote}

\begin{center}
इति श्रीमहामाहेश्वराचार्याभिनवगुप्तविरचितायां नाट्यवेदवृत्तावभिनवभारत्यां वैशिकस्त्रयोविंशोऽध्यायः~॥\\

\vspace{1cm}
\rule{0.2\linewidth}{0.5pt}
\end{center}

\newpage
\thispagestyle{empty}
\begin{center}
\textbf{\large श्रीः}\\

\vspace{2mm}
\textbf{\huge नाट्यशास्त्रम्}\\

\vspace{2mm}
चतुर्विंशोऽध्यायः\renewcommand{\thefootnote}{*}\footnote{अस्याध्यायस्य पाठक्रमो बहुधा भिद्यते जा॒द्यादर्शेषु चतुस्त्रिंशतितम इत्य\textendash\ वनद्धभूमिकाध्याययोर्मध्येऽयं पठ्यते चनमयभादिषु चतुर्विंश एव बहुपाठभेदोऽप्ययं निर्वचनवाक्यसाम्येन द्विधाऽत्र दीयते प्रथमं वृत्तिकारपाठानुसारी यनमभादर्शपाठः, अनन्तरमनुबन्धरूपेण चफ \textendash\  पाठस्तु निवेश्यते पाठान्तरप्रदर्शनसौलभ्यात् काश्यां मुद्रितकोशः पूर्वं पाठं, काव्यमालायां मुद्रितस्तु द्वितीयं क्रममनुवर्तते~।}\\

\rule{0.2\linewidth}{0.5pt}
\end{center}

\begin{quote}
{\na \renewcommand{\thefootnote}{1}\footnote{अत ऊर्ध्वं प्रवक्ष्यामि प्रकृतीनां तु लक्षणम्\textendash\ इति चतुस्त्रिंशाध्यायप्रारम्भे (ज, ठ, ड, ढाद्यादर्शेषु) वर्तते~। वृत्तिकारीयपाठे तदर्धं वैशिकाध्यायान्त एव पठितम्~। बनभादिष्वयमध्यायो वैशिकेनैव संमेलितः~।}समासतस्तु प्रकृतिस्त्रिविधा परिकीर्तिता~।\\
पुरुषाणामथ स्त्रीणामुत्तमाधममध्यमा~॥~१}
\end{quote}

\hrule

\begin{center}
अभिनवभारती \textendash\  चतुर्विंशोऽध्यायः 
\end{center}

\begin{quote}
{\qt त्रिधा विकल्पनं यस्यां पुमान् यत्रोपचर्यते~।\\
तां वन्दे प्रकृतिं शम्भोः शक्तित्रयविजृम्भणात्~॥}
\end{quote}

इह कामोपचारः पूर्वं दर्शितः कामश्च स्त्रीपुरुषहेतुक इत्युक्तम्~। स्त्रीणां च पुंसां च यद्यपि विचित्राः स्वभावास्तथापि ते प्रतिपदमशक्यकलना इति प्रकृतित्रयेण ते सर्वे शक्यसंग्रहा इति प्रकृतित्रयं वक्तव्यम्~। तथा चाह \underline{समासत} इति कामोपचारश्च श्रृङ्गारपर्यवसायी नायकविशेष एवेति नायकभेदा वक्तव्याः~। तस्य च नायकस्यान्तःपुरो बहिर्वा किन्नामधेयः कियान्वा परिवार इति सर्वं कविना ज्ञातव्यं नटेन च~। तदेवं प्रकृतिनायकपरिवारभेदानभिधाय कोऽयमध्यायोऽस्याभिचारमारभ्यते प्रकृत्यादिभेदोपचारो हि

\newpage
\fancyhead[CO]{चतुर्विंशोऽध्यायः}
% चतुर्विंशोऽध्यायः २४९

\begin{quote}
{\na \renewcommand{\thefootnote}{1}\footnote{भ \textendash\  \begin{quote}
{\qt नरशीलगुणोपेताह्युत्तमाधममध्यमाः~। \\
तस्मात्पृथक्पृथग्भावैर्विज्ञेयाः प्रकृतीर्बुधैः~॥}
\end{quote}}जितेन्द्रियज्ञानवती नानाशिल्पविचक्षणा~।\\
दक्षिणाधमहालक्ष्या \renewcommand{\thefootnote}{2}\footnote{द \textendash\  दीनानां}भीतानां परिसान्त्वनी~॥~२

नानाशास्त्रार्थसंपन्ना गाम्भीर्यौदार्यशालिनी~।\\
\renewcommand{\thefootnote}{3}\footnote{द \textendash\  धैर्य}स्थैर्यत्यागगुणोपेता ज्ञेया प्रकृतिरुत्तमा~॥~३

\renewcommand{\thefootnote}{4}\footnote{भ \textendash\  किंचिल्लोकोपकारज्ञा}लोकोपचार\renewcommand{\thefootnote}{5}\footnote{द \textendash\  कार}चतुरा शिल्पशास्त्रविशारदा~।\\
\renewcommand{\thefootnote}{6}\footnote{भ \textendash\  साधारणगुणोपेता}विज्ञानमाधुर्ययुता मध्यमा प्रकृतिः स्मृता~॥~४

\renewcommand{\thefootnote}{7}\footnote{द \textendash\  रूक्षा}रूक्षवाचोऽथ दुःशीलाः कुसत्त्वाः\renewcommand{\thefootnote}{8}\footnote{द \textendash\  शल्प भ \textendash\  स्वल्प} स्थलबुद्धयः~।\\
क्रोधना घातकाश्चैव \renewcommand{\thefootnote}{9}\footnote{भ \textendash\  कृत}मित्रघ्नाश्चिद्रमानिनः\renewcommand{\thefootnote}{10}\footnote{द \textendash\  घातकाः भ \textendash\  दर्शनाः}~॥~५

पिशुनास्तूद्धतै\renewcommand{\thefootnote}{11}\footnote{द \textendash\  ता}र्वाक्यैरकृतज्ञास्तथालसाः~।\\
मान्यामान्या\renewcommand{\thefootnote}{12}\footnote{द \textendash\  मान्य} विशेषज्ञा स्त्रीलोलाः \renewcommand{\thefootnote}{13}\footnote{म \textendash\  कुहक}कलहप्रियाः~॥~६}
\end{quote}

\hrule

\vspace{2mm}
\noindent
स्त्रीणां नपुंसकस्य (काम) विरहत्वादुपचारः स्नेहव्यवहार इत्यध्यायसङ्गतिः~। तत्र प्रकृतिव्यवहारं तावदाह \underline{समासतस्त्विति}~। तुर्व्यतिरेके\textendash\ पूर्वं विस्तरेण स्वभावो दर्शितोऽधुना तु संक्षेपत इति~।\\

(\underline{लोकोपचारेति})~। लोकोपचारो व्यवहार (रतस्मिन् च तु) पतत्यव\textendash\ श्यम्)~। कृतमुपकारं ये विस्मरन्त्यकृतज्ञास्ते~। मान्यामान्ययोरविशेषज्ञा इति समासः~।

\lfoot{32}

\newpage
% २५० नाट्यशास्त्रम् 
\lfoot{}

\begin{quote}
{\na सूचकाः पापकर्माणः परद्रव्यापहारिणः~।\\
एभिर्दोषैस्तु संपन्ना भवन्तीहाधमा नराः~॥~७

एवं \renewcommand{\thefootnote}{1}\footnote{म \textendash\  हि शिल्प}तु शीलतो नॄणां प्रकृतिस्त्रिविधा स्मृता~।\\
\renewcommand{\thefootnote}{2}\footnote{भ \textendash\  एवमेव तु बोद्धव्या स्त्रीणामपि यथाक्रमम्}स्त्रीणां पुनश्च प्रकृतिं व्याख्यास्याम्यनुपूर्वशः~॥~८

\renewcommand{\thefootnote}{3}\footnote{भ \textendash\  स्मितभाषिणी}मृदुभावा चाचपला स्मित\renewcommand{\thefootnote}{4}\footnote{भ \textendash\  हासिनी}भाषिण्यनिष्ठुरा~।\\
गुरूणां वचने दक्षा सलज्जा विनयान्विता~॥~९

रूपाभि\renewcommand{\thefootnote}{5}\footnote{न \textendash\  नय}जनमाधुर्यैर्गुणैः स्वाभाविकैर्युता~।\\
\renewcommand{\thefootnote}{6}\footnote{भ \textendash\  गंभीरा धीरसत्त्वा वा योत्तमा प्रकृतिः स्मृता}गाम्भीर्य धैर्यसंपन्ना विज्ञेया प्रमदोत्तमा~॥~१०

नात्युत्कृष्टैरनिखिलैरेभिरेवान्विता गुणैः~।\\
अल्पदोषानुविद्धा च मध्यमा प्रकृतिः स्मृता\renewcommand{\thefootnote}{7}\footnote{भ \textendash\  मध्या प्रकृतिरिष्यते}~॥~११}
\end{quote}

\hrule

\vspace{2mm}
\underline{पुन}श्चेति पूर्वं यद्यप्युक्ता तथापीत्यर्थः~। तत्र हि कामोपचाराभिप्रायेण प्रकृतित्रैविध्यं व्याख्यातम्~। इह तु सर्वव्यवहारविषयमिति विशेषो दृश्यते~। (तत्र तु) विषयभेदादुत्तमादित्वं (सत्त्वसमुद्भवत्वात्) रस्याः स्निग्धा इत्या\textendash\ दय आहाराः सात्त्विकस्य प्रिया इत्युच्यन्ते~। तत्र हि सत्त्वमाहारविषयमेव सहधूमाभ्यवहारादिति~। कृतकटुकाहारव्रतो मुनिर्न सात्त्विक इति नापि चोरो (?) घृतगुड (प) योन्नभोजी सात्त्विक इति~। \underline{वचने दक्षासती गुरूणां} विषये \underline{सलज्जा~। नात्युत्कृष्टैर}तिश्रेष्ठताहीनैः, अनिखिलैः असमग्रेः~। \underline{सङ्कीर्ण}

\newpage
% चतुर्विंशोऽध्यायः २५१ 

\begin{quote}
{\na अधमा प्रकृतिर्या तु पुरुषाणां प्रकीर्तिता~।\\
विज्ञेया सैव नारीणामधमानां समासतः~॥~१२

\renewcommand{\thefootnote}{1}\footnote{भ \textendash\  नपुंसकं\ldots यं\ldots र्णमधमं तथा}नपुंसकस्तु विज्ञेयः संकीर्णोऽधम एव च~।\\
प्रेष्यादिरपि विज्ञेया संकीर्णा प्रकृतिद्विजाः\renewcommand{\thefootnote}{2}\footnote{न \textendash\  द्विधा}~॥~१३

शकारश्च विटश्चैव ये चान्येप्येवमादयः~।\\
संकीर्णास्तेऽपि विज्ञेया ह्यधमा नाटके बुधैः\renewcommand{\thefootnote}{3}\footnote{भ \textendash\  प्यधमाश्चैव विज्ञेयास्तेपि नाटके}~॥~१४

एता ज्ञेयाः प्रकृतयः पुरुषस्त्रीनपुंसकैः\renewcommand{\thefootnote}{4}\footnote{द \textendash\  काः}~।\\
आसां तु संप्रवक्ष्यामि विधानं शीलसंश्रयम्~॥~१५

\renewcommand{\thefootnote}{5}\footnote{द \textendash\  तत्र न \textendash\  अथ}अत्र चत्वार एव स्युर्नायकाः परिकीर्तिताः~।\\
\renewcommand{\thefootnote}{6}\footnote{द \textendash\  मध्योत्तमाया}मध्यमोत्तमप्रकृतौ नानालक्षणलक्षिताः~॥~१६

धीरोद्धता धीरललिता धीरोदात्तास्तथैव च~।\\
धीरप्रशान्तकाश्चैव नायकाः परिकीर्तिताः~॥~१७

देवा धीरोद्धता ज्ञेयाः स्युर्धीरललिता नृपाः~।\\
सेनापतिरमात्यश्च धीरोदात्तौ प्रकीर्तितौ~॥~१८}
\end{quote}

\hrule

\vspace{2mm}
\noindent
इति~। कश्चिन्मिश्रप्रकृतिः कश्चिदधमप्रकृतिरेव~। \underline{प्रेष्याश्च संकीर्णा} इति स्वामि\textendash\ चित्तानुरोधात्~। विटोऽप्येवं शकरोऽप्य (नुभूत) विभवत्वादुत्तममध्यमचेष्टित\textendash\ माचरति सङ्कीणः~। परमार्थतस्तु प्रेष्यविटशकारा अधमा एव~।\\

प्रकृतिभेदमभिधाय नायकभेदमाह \underline{अत्र चत्वार} इति~। सुरतविषये संबन्धिग्रहणे~। विग्रहं वा सन्धिना दूषयतीति विदूषकः विप्रलम्भनत्वे (कथा)

\newpage
% २५२ नाट्यशास्त्रम् 

\begin{quote}
{\na धीरप्रशान्ता विज्ञेया ब्राह्मणा वणिजस्तथा~।\\
एतेषां तु पुनर्ज्ञेयाश्चत्वारस्तु विदूषकाः~॥~१९

लिंगी द्विजो राजजीवी शिष्यश्चेति यथाक्रमम्~।\\
देवक्षितिश्रृतामात्यब्राह्मणानां प्रयोजयेत्~॥~२०

विप्रलंभसुहृदोमी संकथालापपेशलाः~।\\
व्यसनी प्राप्य दुःखं वा युज्यतेऽभ्युदयेन यः~॥~२१

\renewcommand{\thefootnote}{1}\footnote{कथा}तथा पुरुषमाहुस्तं प्रधानं नायकं बुधाः~।\\
यत्रानेकस्य \renewcommand{\thefootnote}{2}\footnote{न \textendash\  जायेते}भवतो व्यसनाभ्युदयौ पुनः\renewcommand{\thefootnote}{3}\footnote{न \textendash\  समौ}~॥~२२

सपुष्टौ यत्र तौ स्यातां न भवेत्तत्र नायकः~।\\
दिव्या च नृपपत्नी च कुलस्त्री गणिका तथा~॥~२३

एतास्तु नायिका ज्ञेया नानाप्रकृतिलक्षणाः~।\\
धीरा च ललिता च स्यादुदात्ता निश्रृता तथा~॥~२४}
\end{quote}

\hrule

\vspace{2mm}
\noindent
विनोदने (नैः) दूषयन्ति विस्मारयन्ति~। \underline{यथाक्रममिति} क्रमिकमौचित्यमत्र यथोचितं योजना, तद्यथा लिङ्गी ऋषिः देवानाम्, द्विजो वीरः सेनापते, राजा जीवी राज्ञः, शिष्यो ब्राह्मणस्य~। तेषां व्यापारमाह विप्रलम्भसुहृद इति विदूषकः~।\\

नन्वेकपुरुषसंभव इतिवृत्ते को नायक इत्याह \underline{व्यसनीति~। प्राप्यदुखं} चेति~। पूर्वमेव न व्यसनपतितो व्यसनी वा अयमपि तु सुखीभूत्वा दुखं प्राप्तः~।\\

न्वेतल्लक्षणमात्रेतिवृत्तेऽनेकस्यापि रामचरित इव सुग्रीवविभीषणयोरपीत्याह \underline{यत्रानेकस्येति}~। एवं नायकभेदं निरूप्य नायिकाभेदमाह \underline{दिव्या\textendash }

\newpage
% चतुर्विंशोऽध्यायः २५३ 

\begin{quote}
{\na दिव्या राजांगनाश्चैव गुणैर्युक्ता भवन्ति हि~।\\
उदात्ता निश्रृता चैव भवेत्तु कुलजांगना~॥~२५

ललिते चाभ्युदात्ते च गणिकाशिल्पकारिके~।\\
प्रकृतीनां तु सर्वासामुपचाराद् द्विधा स्मृताः~॥~२६

बाह्यश्चाभ्यन्तरश्चैव तयोर्वक्ष्यामि लक्षणम्~।\\
तत्र राजोपचारो यो भवेदाभ्यन्तरो हि सः~॥~२७

ततो वाक्योपचारस्तु यस्य बाह्यः स उच्यते~।\\
अथ राजोपचारे च राज्ञामन्तःपुराश्रितम्~॥~२८

स्त्रीविभागं प्रवक्ष्यामि विभक्तमुपचारतः~।\\
राजोपचारं वक्ष्यामि ह्यन्तःपुरसमाश्रयम्~॥~२९

महादेवी तथा देव्यः स्वामिन्यः स्थापिता अपि~।\\
भोगिन्य शिल्पकारिण्यो नाटकीयाः सनर्तकाः~॥~३०

अनुचारिकाश्च विज्ञेयास्तथा च परिचारिकाः~।\\
तथा संचारिकाश्चैव तथा प्रेषणकारिकाः~॥~३१

महत्तर्यः प्रतीहार्यः कुमार्यः स्थविरो अपि~।\\
आयुक्तिकाश्च नृपतेरयमन्तःपुरो जनः~॥~३२}
\end{quote}

\hrule

\vspace{2mm}
\noindent
चेति~। अथ परिवारभेदमाह \underline{राजोपचारमित्यादि}~। महादेवीप्रभृत्यायुक्ति\textendash\ कान्तः सप्तदशकः स्त्रीगणः, नपुंसकादिवर्गोऽष्टादश~। अत एव वक्ष्यति एतद\textendash\ ष्टादशविधं प्रोक्तमन्तःपुरमिति, तद्विषयः परिवार इत्यर्थः~। महादेवीत्येकत्वं\textendash

\newpage
% २५४ नाट्यशातस्त्रम् 

\begin{quote}
{\na अत्र मूर्धाभिषिक्ता या कुलशीलसमन्विता\renewcommand{\thefootnote}{1}\footnote{न \textendash\  विभूषिता दर्शनीया}~।\\
गुणैर्युक्ता वयस्स्था च मध्यस्था क्रोधना तथा~॥~३३

\renewcommand{\thefootnote}{2}\footnote{न \textendash\  अभीष्टानृत्त}मुक्तेर्ष्या नृपशीलज्ञा सुखदुःखसहा समा\renewcommand{\thefootnote}{3}\footnote{न \textendash\  तथा}~।\\
शान्तिस्वस्त्ययनैर्भर्त्तुस्सततं मङ्गलैषिणी~॥~३४

\renewcommand{\thefootnote}{4}\footnote{न \textendash\  शुचिः}शान्ता पतिव्रता धीरा अन्तःपुरहिते रता~।\\
एभिर्गुणैस्तु संयुक्ता महादेवीत्युदाहृता~॥~३५

एभिरेव गुणैर्युक्तास्तत्संस्कारविवर्जिताः~।\\
गर्विताश्चातिसौभाग्याः पतिसंभोगतत्पराः~॥~३६

शुचिनित्योज्ज्वलाकाराः पतिपक्षाभ्यसूयकाः~।\\
वयोरूपगुणाढ्या यास्ता देव्य इति भाषिताः~॥~३७

सेनापतेरमात्यानां श्रृत्यानामथवा पुनः~।\\
भवेयुस्तनया यास्तु प्रतिसम्मानवर्जिताः~॥~३८

शीलरूपगुणैर्यास्तु संपन्ना नृपतेर्हिताः~।\\
स्वगुणैर्लब्धसम्माना स्वामिन्य इति ताः स्मृताः~॥~३९}
\end{quote}

\hrule

\vspace{2mm}
\noindent
विवक्षितम्~। महादेवीनां क्रमेण लक्षणान्याह \underline{अत्र मूर्धाभिषिक्तेत्यादि}~। सर्वेषां मूर्धनि प्रधानस्थानेत्यभिषिक्ता~। वयसि मध्यमे तिष्ठतीति मध्यस्था~। (\underline{अन्तःपुरेति}) अन्तःपुरिके बाह्ये च वर्गे अन्तःपुराय हितं कौशल्यसंपादनम्~। \underline{देव्य} इति~। महादेवी तु श्रृङ्गारोचिता नातीव भवति साभिमुख्यमभिप्रयातीत्याशयेन वासवदत्तादिषु कवयो देवी वाचोयुक्त्या व्यवहरन्ति~।

\newpage
% चतुर्विंशोऽध्यायः २५५ 

\begin{quote}
{\na रूपयौवनशालिन्यः\renewcommand{\thefootnote}{1}\footnote{न \textendash\  शालिन्यः} कर्कशा विभ्रमान्विताः~।\\
रतिसंभोगकुशलाः प्रतिपक्षाभ्यसूयिकाः~॥~४०

दक्षा भर्तुश्च चित्तज्ञा गन्धमाल्योज्ज्वलास्सदा~।\\
नृपतेश्छन्दवर्त्तिन्यो न हीर्ष्यामानगर्विताः\renewcommand{\thefootnote}{2}\footnote{न \textendash\  ईर्ष्याकोपागमहते}~॥~४१

उत्थिताश्च प्रमत्ताश्च त्यक्तालस्या न निष्ठुराः~।\\
\renewcommand{\thefootnote}{3}\footnote{न \textendash\  अप्रमत्ता विवेकिन्यः}मान्यामान्यविशेषज्ञाः स्थापिता इति ताः स्मृताः~॥~४२

\renewcommand{\thefootnote}{4}\footnote{न \textendash\  रूप}कुलशील\renewcommand{\thefootnote}{5}\footnote{न \textendash\  कुलैः पूजा}लब्धपूजामृदवो नातिचोद्भटाः~।\\
मध्यस्था निश्रृताः क्षान्ता भोगिन्य इति ताः स्मृताः\renewcommand{\thefootnote}{6}\footnote{न \textendash\  विश्रुताः}~॥~४३

नाना\renewcommand{\thefootnote}{7}\footnote{ट \textendash\  काल}कलाविशेषज्ञा नानाशिल्पविचक्षणाः~।\\
गन्धपुष्पविभागज्ञा लेख्यालेख्यविकल्पिकाः~॥~४४

शयनासनभागज्ञाश्चतुरा \renewcommand{\thefootnote}{8}\footnote{न \textendash\  मुदिता}मधुरास्तथा~।\\
दक्षाः सौम्याः स्फुटाः श्लिष्टा निश्रृताः शिल्पकारिकाः~॥~४५

ग्रहमोक्षलयज्ञा या रसभावविकल्पिकाः~।\renewcommand{\thefootnote}{9}\footnote{ट \textendash\  स्वरताललयज्ञाश्च तथाचार्योपसेविताः न \textendash\  परभावेङ्गितज्ञाश्च}\\
चतुरा नाट्यकुशलाश्चोहापोहविचक्षणाः~॥~४६}
\end{quote}

\hrule

\vspace{2mm}
\noindent
(\underline{स्वगुणै}रिति समानं) लम्भिता गुणैर्योजिता (कलाशीलब)योभिः~। \underline{कर्कशा} सौभाग्यगर्वेण~।

\newpage
% २५६ नाट्यशास्त्रम् 

\begin{quote}
{\na रूपयौवनसंपन्ना नाटकीयास्तु ताः स्मृताः~।\\
हेलाभावविशेषाढ्या सत्त्वेनाभिनयेन च~॥~४७

माधुर्येण च संपन्ना ह्यातोद्यकुशला तथा~।\\
अङ्गप्रत्यङ्गसंपन्ना\renewcommand{\thefootnote}{1}\footnote{न \textendash\  संयुक्ता} चतुष्षष्ठिकलान्विता~॥~४८

चतुराः प्रश्रयोपेताः स्त्रीदोषैश्च विवर्जिताः~।\\
\renewcommand{\thefootnote}{2}\footnote{न \textendash\  प्रियंवदा सलज्जा च प्रगल्भा विजितश्रमा}सदा प्रगल्भा च तथा त्यक्तालस्या जितश्रमा~॥~४९

नानाशिल्प\renewcommand{\thefootnote}{3}\footnote{न \textendash\  शील}प्रयोगज्ञा नृत्तगीतविचक्षणा~।\\
\renewcommand{\thefootnote}{4}\footnote{न \textendash\  अनु}अथ रूपगुणौदार्यधैर्य\renewcommand{\thefootnote}{5}\footnote{न \textendash\  वीर}सौभाग्य\renewcommand{\thefootnote}{6}\footnote{न \textendash\  संयुता}शीलसंपन्ना~॥~५०

\renewcommand{\thefootnote}{7}\footnote{न \textendash\  अनुवादिकलायुक्ता रक्तकण्ठा मनोरमा}पेशलमधुरस्निग्धानुनादिकलचित्रकण्ठा च\renewcommand{\thefootnote}{8}\footnote{ट \textendash\  नीकलत्रकण्ठी च}~।\\
समागतासु नारीषु\renewcommand{\thefootnote}{9}\footnote{न \textendash\  बह्वीषु} रूपयौवनकान्तिभिः~॥~५१

न दृश्यते गुणैस्तुल्या यस्याः सा नर्तकी स्मृता~।\\
सर्वावस्थोपचारेषु या न मुञ्चति पार्थिवम्~॥~५२

विज्ञेया दक्षिणा दक्षा नाट्यज्ञैरनुचारिका~।\\
शय्यापाली छत्रधारी\renewcommand{\thefootnote}{10}\footnote{न \textendash\  छत्राधि\textendash\  कारिण्यः} तथा व्यजनधारिणी~॥~५३

संवाहिका गन्धयोक्त्री\renewcommand{\thefootnote}{11}\footnote{न \textendash\  अथ गन्धज्ञा} तथा चैव प्रसाधिका~।\\
तथाभरणयोक्त्री च माल्यसंयोजिका तथा\renewcommand{\thefootnote}{12}\footnote{न \textendash\  तथा माल्यविचक्षणा.}~॥~५४}
\end{quote}

\newpage
% चतुर्विंशोऽध्यायः २५७ 

\begin{quote}
{\na एवं विधा भवेयुर्याः ता ज्ञेयाः परिचारिकाः~।\\
नानाकक्ष्या विचारिण्यः तथोपवनसंचराः~॥~५५

देवतायतनक्रीडा प्रासादपरिचारिकाः~।\\
यामकिन्यस्तथा चैव याश्चैवं लक्षणाः स्त्रियः~॥~५६

संचारिकास्तु विज्ञेया नाट्यज्ञैः समुदाहृताः\renewcommand{\thefootnote}{1}\footnote{द \textendash\  उपचारतः त \textendash\  भोगवारिताः}~।\\
प्रेषणेऽकाम\renewcommand{\thefootnote}{2}\footnote{न \textendash\  काम्य}संयुक्ते \renewcommand{\thefootnote}{3}\footnote{द \textendash\  गुह्यागुह्य}गूह्यगुह्यसमुत्थिते~॥~५७

नृपैर्यास्तु नियुज्यन्ते ताः ज्ञेयाः परिचारिकाः\renewcommand{\thefootnote}{4}\footnote{ट \textendash\  प्रेषणकारिकाः}~।\\
सर्वान्तःपुररक्षासु स्तुति\renewcommand{\thefootnote}{5}\footnote{ट \textendash\  रक्षायामाशीः}स्वस्त्ययनेन च~॥~५८

या वृद्धिमभिनन्दन्ति ता विज्ञेया महत्तराः~।\\
सन्धिविग्रहसंबद्धनानाचारसमुत्थितम्~॥~५९

\renewcommand{\thefootnote}{6}\footnote{ट \textendash\  राज्ञो हरन्ति}निवेदयन्ति याः कार्यं प्रतिहार्यस्तु ताः स्मृताः~।\\
अप्राप्त\renewcommand{\thefootnote}{7}\footnote{ट \textendash\  रति}रससंभोगा न संभ्रान्ता न चोद्भटाः\renewcommand{\thefootnote}{8}\footnote{ट \textendash\  अविश्रान्ता न चोत्कटाः}~॥~६०

निश्रृताश्च सलज्जाश्चा कुमार्यो बालिकाः स्मृताः\renewcommand{\thefootnote}{9}\footnote{ट इति कीर्तिताः त \textendash\  पालिकाः स्मृताः}~।\\
पूर्व\renewcommand{\thefootnote}{10}\footnote{द \textendash\  राजेन संज्ञा त \textendash\  रङ्ग}राजनयज्ञा याः पूर्वराजाभिपूजिताः~॥~६१

पूर्वराजानुचरितास्ता वृद्धा इति सुज्ञिताः~।\renewcommand{\thefootnote}{11}\footnote{ट \textendash\  स्थविरा इति ताः स्मृताः}}
\end{quote}

\hrule

\vspace{2mm}
\noindent
\underline{अथे}त्यपिचेत्यर्थः~। \underline{यामकिन्यः} प्रतिदहरं जाग्रति याः~।

\lfoot{33}

\newpage
\lfoot{}
% २५८ नाट्यशास्त्रम्

\begin{quote}
{\na भाण्डागारेष्वधिकृता\renewcommand{\thefootnote}{1}\footnote{ट \textendash\  गारनियुक्ता याः}श्चायुधाधिकृतास्तथा~॥~६२

फलमूलौषधीनां च \renewcommand{\thefootnote}{2}\footnote{ट \textendash\  बीजानां चान्व}तथा चैवान्ववेक्षिकी~।\\
गन्धाभरणवस्त्राणां माल्यानां चैव चिन्तिका~॥~६३

बह्वाश्रये तथा युक्ता \renewcommand{\thefootnote}{3}\footnote{ट \textendash\  वृसक्ता च}ज्ञेया ह्यायुक्तिकास्तु ताः~।\\
इत्यन्तःपुरचारिण्यः स्त्रियः प्रोक्ताः समासतः~॥~६४

\renewcommand{\thefootnote}{4}\footnote{ट \textendash\  विशेषणं तु शेषाणां त \textendash\  विशेषाणां}विशेषणविशेषेण तासां वक्ष्यामि वै द्विजाः~।\\
अनुरक्ताश्च भक्ताश्च नानापार्श्वसमुत्थिताः~॥~६५

या नियुक्ता नियोगेषु कार्येषु विविधेषु च~।\\
न चोद्भटा असंभ्रान्ता न लुब्धा नापि निष्ठुराः~॥~६६

दान्ताः क्षान्ताः प्रसन्नाश्च जितक्रोधा जितेन्द्रियाः~।\\
अकामा लोभडीनाश्च स्त्रीदोषैश्च विवर्जिताः~॥~६७

\renewcommand{\thefootnote}{5}\footnote{\begin{quote}
{\qt [एवं विधास्तु कर्तव्या नियोगिन्यो नियोक्तृभिः~।\\
अतः परं प्रवक्ष्यामि तृतीयं प्रकृतिं द्विजाः~॥

यस्यान्नपुंसकं नाम द्वितीया प्रकृतिः स्मृता~।]}
\end{quote}}सा त्वन्तःपुरसंचारे योज्या पार्थिववेश्मनि~।\\
कारुकाः कञ्चुकीयाश्च तथा वर्षवराः पुनः~॥~६८}
\end{quote}

\hrule

\vspace{2mm}
\underline{विशेषण}मिति विशेषमन्याभ्यः, आयुक्तिकानां वक्ष्यामीत्यर्थः~। \underline{आकाम्य} इति परस्य कामयितुमनर्हा अशक्याश्च कामिता हि~। सर्वे शंसन्ति \underline{आयोजनेष्विति}~।

\newpage
% चतुर्विंशोऽध्यायः २५९ 

\begin{quote}
{\na औपस्थायिकनिर्मुण्डा \renewcommand{\thefootnote}{1}\footnote{म \textendash\  अभ्यगारास्तथैवच~।

\begin{quote}
{\qt अपुमांसोऽथ ये ज्ञात्वा स्त्रीलीलाधारिणश्च ये~।\\
अन्तःपुरचरास्तेऽपि कार्याः कार्यनियोगिनः~॥

द्वारस्थांस्नातकान् कुर्यादायाचारसमन्वितान्~।\\
प्रेषणेष्ठार्यरूपेण कञ्चुकीयान् प्रयोजयेत्~॥

अन्तःपुराणां रक्षार्थं योज्या वर्षवरा नृपैः~।\\
ज्ञानविज्ञानसंपन्नाः स्त्रीसंभोगविवर्जिताः~।\\
अमत्सरा ये पुरुषाः काञ्चुकीयास्तु ते स्मृताः~॥}
\end{quote}}स्त्रीणां प्रेषणकर्मणि~।\\
रक्षणं च कुमारीणां बालिकानां प्रयोजयेत्~॥~६९

अन्तःपुराधिकारेषु राजचर्यानुवर्त्तिनाम्~।\\
सर्ववृत्तान्तसंवाहाः पत्यागारे नियोजयेत्~॥~७०

विनीताः स्वल्पसत्त्वा ये क्लीबा वै स्त्रीस्वभाविकाः~।\\
जात्या न दोषिणश्चैव ते वै वर्षवराः स्मृताः~॥~७१

ब्राह्मणाः कुशला वृद्धाः कामदोषविवर्जिताः~।\\
प्रयोजनेषु देवीनां प्रयोक्तव्या नृपैः सदा~॥~७२

एतदष्टादशविधं प्रोक्तमन्तःपुरं मया~।\\
अतः परं प्रवक्ष्यामि बाह्यं पुरुषसंभवम्\renewcommand{\thefootnote}{2}\footnote{म \textendash\  पुरुषांस्तु बहिश्चरान्}~॥~७३

राजा सेनापतिश्चैव पुरोधा मन्त्रिणस्तथा~।\\
सचिवाः प्राड्विवाकाश्च कुमाराधिकृतास्तथा~॥~७४}
\end{quote}

\hrule

\vspace{2mm}
\noindent
अथ बाह्यपरिवारमाह \underline{राजेति}~। युवराजोऽत्र राजशब्देनोक्तः~। (अमात्य\textendash

\newpage
% २६० नाट्यशास्त्रम् 

\begin{quote}
{\na एके चान्ये च बहवो मान्या ज्ञेया नृपस्य तु~।\\
विशेषमेषां वक्ष्यामि लक्षणेन निबोधत~॥~७५

बलवान् बुद्धिसंपन्नः सत्यवादी जितेन्द्रिय~।\\
दक्षः प्रगल्भो धृतिमान् विक्रान्तो मतिमांञ्छुचिः~॥~७६

दीर्धदर्शी महोत्साहः कृतज्ञः प्रियवाङ्मृदुः~।\\
लोकपालव्रतधरः \renewcommand{\thefootnote}{1}\footnote{म \textendash\  शूरो दाक्षिण्यसंयुतः~। कलाज्ञश्चाप्रमत्तश्र वृद्धसेव्यथ नीतिवित्~।}कर्ममार्गविशारदः~॥~७७

उत्थितश्चाप्रमत्तश्च वृद्धसेव्यर्थशास्त्रवित्~।\\
परभावेङ्गिताभिज्ञः शूरो रक्षासमन्वितः~॥~७८

ऊहापोहविचारी च नानाशिल्पप्रयोजकः~।\\
नीतिशास्त्रार्थकुशलस्तथा चैवानुरागवान्~॥~७९

धर्मज्ञोऽव्यसनी चैव गुणैरेतैर्भवेन्नृपः~।\\
कुलीना बुद्धिसंपन्ना नानाशास्त्रविपश्चिताः~॥~८०

स्निग्धाः परैरहार्यश्च न प्रमत्ताश्च देशजाः~।\\
अलुब्धाश्च विनीताश्च शुचयो धार्मिकास्तथा~॥~८१

पुरोधो मन्त्रिणस्त्वेभिर्गुणैर्युक्ता भवन्ति हि~।\\
बुद्धिमान्नीतिसंपन्नस्त्यक्तालस्यः प्रियवदः~॥~८२}
\end{quote}

\hrule

\vspace{2mm}
\noindent
इति) अमेति सहार्थे सहभवाः सहचारिणोऽमात्या इत्येकोऽर्थः~। \underline{अनुराग}\textendash\ वानिति प्रजासु, प्रजास्वयत्नानुरक्ताः, अनुरागोऽहि सार्वगुण्यमिति कौटल्यः

\newpage
% चतुर्विंशोऽध्यायः २६१

\begin{quote}
{\na पररन्ध्रविधिज्ञश्च यात्राकालविशेषवित्~।\\
अर्थशास्त्रार्थकुशलो\renewcommand{\thefootnote}{1}\footnote{म \textendash\  तत्वज्ञो} ह्यनुरक्तः कुलोद्भवः~॥~८३

देशवित्कालविच्चैव \renewcommand{\thefootnote}{2}\footnote{म \textendash\  भवेत्सेनापतिद्विजाः~। 
\begin{quote}
{\qt कुलीना बुद्धिसंपन्ना नानाशास्त्रविपश्चितः~।\\
 स्निग्धाः परैरहार्याश्च न प्रमत्ताश्च देशजाः~॥

 अलुब्धाश्च विनीताश्च शुचयो धार्मिकास्तथा~।\\
 पुरोधोमन्त्रिणस्त्वेभिर्गुणैर्युक्ता भवन्ति हि~॥

 दक्षाः प्रियंवदा भक्ताः शुचयः श्रमवर्जिताः~।\\
 विनीताः कुशला दान्ताः प्रभवः सचिवाः स्मृताः~॥}
\end{quote}}कर्तव्यः क्षितिपैः सदा~।\\
व्यवहारार्थतत्त्वज्ञा बुद्धिमन्तो बहुश्रुताः~॥~८४

\renewcommand{\thefootnote}{3}\footnote{म \textendash\  वाग्मिनो}मध्यस्था धार्मिका धीराः कार्याकार्यविवेकिनः~।\\
क्षान्ता दान्ता जितक्रोधा सर्वत्र समदर्शिनः~॥~८५

ईदृशः प्राड्विवाकास्तु स्थाप्या धर्मासने द्विजाः~।\\
उत्थिताश्चाप्रमत्ताश्च त्यक्तालस्या जितश्रमाः~॥~८६}
\end{quote}

\hrule

\vspace{2mm}
\noindent
अव्यसनिभिः स्त्रीमद्यमृगयाक्षादावसक्ताः~। (\underline{अर्थशास्त्रस्येति}) अर्थशास्त्रस्य योऽर्थ एकवाक्यतात्पर्यादिना तज्जानाति~। \underline{परैरहार्या} इति अभेद्याः~। \underline{ज्ञानं} अर्थशास्त्रस्य, (तस्य) विधौ (ज्ञानं) विज्ञानं तदधिष्ठानकौशलं लक्ष्यलक्षणज्ञा इति यावत्~। (\underline{प्राड्विवाक} इति) पृच्छति विवादपदे निर्णयमिति प्रायो विवदितारस्तेषां विवेक उच्यते यैस्तैः प्राड्विवाकः, पृच्छेः क्किबचीति क्विपि दीर्घे प्राडिति रूपम्~। विपूर्वस्य ब्रूतेर्घञि विवाक इति~। नयोऽ\textendash

\newpage
% २६२ नाट्यशास्त्रम् 

\begin{quote}
{\na स्निग्धाः क्षान्ता विनीताश्च मध्यस्था निपुणास्तथा~।\\
नयज्ञा विनयज्ञाश्च ऊहापोहविचक्षणाः~॥~८७

\renewcommand{\thefootnote}{1}\footnote{म \textendash\  क्रमायाता हितासक्ताः कुमाराधिकृताः स्मृताः~।}सर्वशास्त्रार्थसंपन्नाः कुमाराधिकृतास्तथा~।\\
बृहस्पतिमतादेषां गुणांश्चाभिकांक्षयेत्~॥~८८

विज्ञेयं चोपहार्यं च सभ्यानां च विकल्पनम्~।\\
इत्यष वो मया प्रोक्तः \renewcommand{\thefootnote}{2}\footnote{म \textendash\  सर्वप्रकृति\textendash\ लक्षणम्}प्राड्विवाकविनिर्णयः\renewcommand{\thefootnote}{3}\footnote{म \textendash\  ज्ञानविज्ञानवेदिनः~। स्वदेशजा विनीताश्च शुचयो धार्मिकास्तथा~।}~॥~८९}
\end{quote}

\hrule

\vspace{2mm}
\noindent
त्रार्थशास्त्रं नयहेतुत्वात्~। एवं विनयोऽत्र धर्मशास्त्रम्~। कुमाराणां राजपुत्राणां रक्षार्थमधिकृताः~।\\

एतच्चैषां संक्षेपेण स्वरूपमुक्तं वितत्य तु तत्प्रधानेभ्य एव शास्त्रेभ्योऽ\textendash\ वधारयेदिति दर्शयति \underline{बृहस्पतिमता}दिति बार्हस्पत्यौशनसादेरित्यर्थः~। \underline{एषा}मिति राजसेनापत्यादीनां कुमाराधिकृतपर्यन्तानां यदेतैर्ज्ञातव्यमुपहरणीयं वा संपाद्य तदपि बार्हस्पत्यादेरभिलक्षयेदिति संबन्धः~।\\

ननु कियन्तः प्राड्विवाका इत्याह \underline{सभ्यानां} चेति~। तन्मतादेव जानीयादि\textendash\ त्यर्थः~। प्रजानां मात्स्यन्यायाद्रक्षितुं राज्ञोऽधिकारः, मात्स्यो न्यायश्च विवाद\textendash\ निर्णयेन रक्ष्यते~। तत्र च प्राड्विवाका एव प्रधानम्~। तथा च प्राड्विवाको राजस्थानीय इति लोके प्रसिद्धम्~। प्रधाने च व्यपदेशं कुर्वन्नुपसंहरति \underline{इत्येष वो मया} \underline{प्रोक्तः प्राड्विवाकविनिर्णय} इति~। राजोपयोगीति बाह्याभ्यन्तरपरि\textendash\ वारनिर्णय इति यावत्~।

\newpage
% चतुर्विंशोऽध्यायः २६३

\begin{quote}
{\na अत ऊर्ध्वं प्रवक्ष्यामि चित्राभिनयनं पुनः~॥}
\end{quote}

\begin{center}
\textbf{इति भारतीये नाट्यशास्त्रे पुंस्त्र्यपचारो नामाध्यायश्चतुर्विंशः~॥}
\end{center}

\hrule

\vspace{2mm}
अध्यायान्तरमासूत्रयति \underline{अत ऊर्ध्व}मिति~। पुनःपुनरयमभिप्रायो यद्यपि सामान्याभिनये चित्राभिनयोऽस्ति तथापि रसस्वभावविशेषो दर्शितः~। श्रृङ्गार\textendash\ तद्वयभिचार्यादीनां हि प्राधान्यं पूर्वमेव दर्शितम्~। चित्राभिनये तु रसाद्युपयोगिबाह्यवस्तुविषयमेवाभिनयानां भावनारूपं मिश्रीकरणात्मकं समानीकरणमुच्यते~। तेन चित्राभिनयः सामान्याभिनयविशेषभूत एवेति शिवम्~॥

\begin{quote}
{\qt चतुर्विंशोऽयमध्यायः किञ्चित्कृतविवेचनः~।\\
मयाभिनवगुप्तेन शिवदास्यैकशालिना~॥}
\end{quote}

\begin{center}
इति श्रीमहामाहेश्वराचार्याभिनवगुप्तविरचितायां नाट्यवेदवृत्तावभिनवभारत्यां स्त्रीपुंसोपचारोऽध्यायश्चतुर्विंशः\\

\vspace{2cm}
\rule{0.2\linewidth}{0.5pt}
\end{center}

\newpage
\thispagestyle{empty}
\begin{center}
\textbf{\large श्रीः}\\

\vspace{2mm}
\textbf{\huge नाट्यशास्त्रम्}\\

\vspace{2mm}
पञ्चविंशोऽध्यायः\\

\vspace{2mm}
\rule{0.2\linewidth}{0.5pt}
\end{center}

\begin{quote}
{\na अंगाद्यभिनयस्यैव यो विशेषः क्वचित् क्वचित्~।\\
\renewcommand{\thefootnote}{1}\footnote{भ \textendash\  अव्यक्त}अनुक्त उच्यते चित्रः\renewcommand{\thefootnote}{2}\footnote{ज \textendash\  यस्मात्त} स चित्राभिनयस्स्मृतः~॥~१}
\end{quote}

\hrule

\begin{center}
अथ पञ्चविंशोऽध्यायः
\end{center}

\begin{quote}
{\qt वागङ्गसत्त्वचेष्टाचित्राभिनयप्रयोगरचनचणः~।\\
संसारनाट्यनायकपुरुषाकारः शिवो जयति~॥}
\end{quote}

सामान्याभिनयस्य चित्राभिनयः शेष इत्युक्तम्~। केवलसामान्याध्याये रसात्मकप्रधानं पदार्थविशेषमभिनयानां समानीकरणम्~। इह तु तदुपयोगी विभावादिविषयम्~। किञ्चैक एवाभिनयः पूर्वं यो निरूपितः स एव कार्या न्तरालाभे तद्विरुद्धमर्थमभिनयतीति चित्र उच्यते~। तद्वैदग्ध्यमध्याये निरूप्यत इति सङ्गतिः~। तदेतदाह अङ्गाद्यभिनयस्यैवेति~। अङ्गमिति करणाङ्गहारास्तेषा मभिनयत्वं नोक्तम्~। तर्हि वक्ष्यते\textendash

\begin{quote}
{\qt शिखिसारसहंसाद्या स्थले ये च स्वभावतः~।\\
रेचकैरङ्गहारैश्च तेषामभिनयः\ldots \ldots \ldots \ldots ~॥} इत्यादि~।
\end{quote}

\noindent
आदिग्रहणे विभावादि विभावोऽपि विभावस्याभिनय इति वक्ष्यते~। अभिनीयतेऽनेनेत्यभिनयः~। तस्य च विशेषः उत्तमोत्तमेत्यादिना वक्ष्यते~। अभिनीयत इति चाभिनयः~। यो ह्यन्येनाभिनय उक्तः स इहाभिनयान्तरस्याप्यभिनयत्वे\textendash\ नोक्तो यथा {\qt अलपद्मकपीडाभिः सर्वार्थग्रहण} मिति~। अङ्गाभिनयस्येति समाहारे वृत्तिस्तस्य विशेषो य उच्यत इति संबन्धः~। क्वचिदिति~। न सर्वत्र~। 

\newpage
\fancyhead[CO]{पञ्चविंशोऽध्यायः}
% पञ्चविंशोऽध्यायः २६५

\begin{quote}
{\na उत्तानौ तु करौ कृत्वा स्वस्तिकौ पार्श्वसंस्थितौ\renewcommand{\thefootnote}{1}\footnote{ज \textendash\  पताकौ स्वस्तिकस्थितौ}~।\\
उद्वाहितेन शिरसा तथा चोर्ध्व\renewcommand{\thefootnote}{2}\footnote{ज \textendash\  उच्च}निरीक्षणात्~॥~२

प्रभातं गगनं रात्रिः प्रदोषं दिवसं तथा~।\\
ऋतून् घनान् वनान्तांश्च विस्तीर्णांश्च जलाशयान्~॥~३

दिशो ग्रहान् सनक्षत्रान् किञ्चित् स्वस्थं च यद्भवेत्\renewcommand{\thefootnote}{3}\footnote{ब \textendash\  स्वस्थं दिव्यार्थमेव च}~।\\
\renewcommand{\thefootnote}{4}\footnote{व \textendash\  अनेनाभिनयेन ह्यनेकान् भावान् प्रदर्शयेत्~। ज \textendash\  अभिनेयं तत्र सर्वं}तस्य त्वभिनयः कार्यो नानादृष्टिसमन्वितः\renewcommand{\thefootnote}{5}\footnote{च \textendash\  भावरसार्द्रतः}~॥~४

\renewcommand{\thefootnote}{6}\footnote{ब \textendash\  अनेनैव क्रमेणेह नानाभावसमाश्रयम्}एभिरेव करैर्भूयस्तेनैव शिरसा पुनः~।\\
अधो निरीक्षणेनाथ भूमिस्थान् संप्रदर्शयेत्~॥~५

स्पर्शस्य \renewcommand{\thefootnote}{7}\footnote{च \textendash\  ग्रहणात्\ldots \ldots उल्लुकसनादपि}ग्रहणेनैव तथोल्लुकसनेन च~।\\
चन्द्रज्योत्स्नां सुखं वायुं \renewcommand{\thefootnote}{8}\footnote{च \textendash\  रसगन्धौ}रसं गन्धं च निर्दिशेत्~॥~६}
\end{quote}

\hrule

\vspace{2mm}
\noindent
अत्राभेदतः अस्मात्क्वचिदनुक्तोऽसौ विशेषः क्वचित्तूक्त एव {\qt स्वस्तिकविच्युतिकरणा} (९\textendash\ २१) दित्यादि~। य एवासावधिको विशेषः तदेव चित्रमभिनयम्~। यतः संपादयति तत उपचारात्~। चित्राभिनयस्तत्र तत्र प्रभातादयः सर्वलोक\textendash\ साधारणा इति तद्विषयस्याभिनयस्य सहकारियोगेन चित्रत्वं दर्शयितुमाह उत्तानौ तु करौ कृत्वेति (९\textendash\  १३६) {\qt स्वस्तिकविच्युतिकरणाद्दिशो घनाः खं वनं समुद्राश्च~। ऋतवो महीतलोच्चं विस्तीर्णं चाभिनेयं स्या} दित्युक्तम्~। संयुक्त\textendash\ हस्ते स्वस्तिके~। अत्रोत्तानत्वं पार्श्वस्थता शिरसो उद्वाहनं दृष्टेरूर्ध्वता इत्यादि\textendash\ विंशेष उच्यते~। एवमुत्तरत्र योज्यम्~। नानादृष्टीति~। कदाचिद्विस्मिता क्वचिच्च विहीनेत्यादिक्रमेण ऊर्ध्वं स्वविरुद्धमधस्स्थं त्वित्थमभिनेयमित्याह अधोनिरी\textendash\ क्षणेनेति~। स्वस्थमित्युक्तं गगने च मृगाङ्कादयः पदार्थास्ते कथमभिनेया इत्याह स्पर्शस्य ग्रहणेनैवेति~।

\lfoot{34}

\newpage
\lfoot{}
% २६६ नाट्यशास्त्रम् 

\begin{quote}
{\na वस्त्रावकुण्ठनात्सूर्यं रजोधूमानिलांस्तथां\renewcommand{\thefootnote}{1}\footnote{ज \textendash\  अनलान् ब \textendash\  अनिलौ}~।\\
भूमितापमथोष्णं च कुर्याच्छायाभिलाषतः~॥~७

ऊर्ध्वाकेकरदृष्टिस्तु मध्याह्ने सूर्यमादिशेत्~।\\
उदयास्तगतं\renewcommand{\thefootnote}{2}\footnote{ज \textendash\  मये चैव गम्भीरार्थैः ब \textendash\  संभ्रमेण} चैव विस्मयार्थैः प्रदर्शयेत~॥~८

यानि सौम्यार्थयुक्तानि सुखभावकृतानि च~।\\
गात्रस्पर्शैस्सरोमाञ्चैस्तेषामभिनयो भवेत्~॥~९

यानि स्युस्तीक्ष्णरूपाणि तानि चाभिनयेत्सुधीः~।\\
असंस्पर्शैस्तथोद्वेगैस्तथा मुखविकुण्ठनैः~॥~१०

गम्भीरोदात्तसंयुक्तानर्थानभिनयैद्बुधः~।\\
\renewcommand{\thefootnote}{3}\footnote{ब \textendash\  साहसैश्च}साटोपैश्च सगर्वैश्च गात्रैः सौष्ठवसंयुतैः~॥~११}
\end{quote}

\hrule

\begin{quote}
{\qt किञ्चिदाकुञ्चिते नेत्रे कृत्वा भ्रूक्षेपमेव च~।\\
तथांसगण्डयोः स्पर्शादिति~॥}
\end{quote}

(२२\textendash\ २८) उलूकवदंसना यस्योर्ध्वं विधूननं~। रज इति धूलिः~।उदयास्तमयोः पूर्वपश्चिययोः पर्वतयोः गतं सूर्यं स्मयोऽभिनयोऽर्थोऽभिनयत्वेन प्रयोजनं येषामिति विस्मयाभिनयैर्निदर्शयेदिति यावत्~। सर्वग्राहकं लक्षणमाह यानि सौम्यार्थयुक्तानीति~। सौम्यं येषां प्रयोजनं एतदेव स्फुटयति सुखप्रधानस्य भावस्य कृतं संपत्तिर्येभ्यः सुखप्रधानो भावः सोम इव सौम्यसुखायेत्वाद्यः~। विकुण्ठनैः सङ्कोचनैः~। गम्भीरोदात्तसंयुक्तानिति भावप्रधानो निर्देशः~। तेन यद्विषयं गांभीर्यमुदायुक्तत्वं च~। आदावत्र शाखाविस्तरः गन्धर्व इत्यात्मनिर्देशादिगम्भीरोदात्तप्रसंगाद्राजोचितहाराभिनय \ldots \ldots \ldots पवीतदेशस्थमरालमिति~। आद्या धनुर्नताकुञ्चितोंऽगुष्ठकः शेष\textendash

\newpage
% पञ्चविंशोऽध्यायः २६७

\begin{quote}
{\na यज्ञोपवीत\renewcommand{\thefootnote}{1}\footnote{च \textendash\  देशे तु कृत्वारालौ करावुभौ}देशस्थमरालं \renewcommand{\thefootnote}{2}\footnote{ब \textendash\  हस्तं}हासमादिशेत्~।\\
\renewcommand{\thefootnote}{3}\footnote{ब \textendash\  हारार्थहारयोगेषु}स्वस्तिकौ विच्युतौ हारस्रग्दामार्थान् निदर्शयेत्~॥~१२

भ्रमणेनं प्रदेशिन्या दृष्टेः परिगमेन च~।\\
\renewcommand{\thefootnote}{4}\footnote{च \textendash\  पीडनाच्चालपद्मस्य}अलपद्मकपीडायाः सर्वार्थग्रहणं भवेत्~॥~१३

\renewcommand{\thefootnote}{5}\footnote{ज \textendash\  श्राव्यं}श्रव्यं श्रवणयोगेन दृश्यं दृष्टिविलोकनैः\renewcommand{\thefootnote}{6}\footnote{ज \textendash\  उपपातनात्}~।\\
आत्मस्थं परसंस्थं वा मध्यस्थं वा विनिर्दिशेत्~॥~१४

विद्युदुल्काघनरवाविष्फुलिंगा\renewcommand{\thefootnote}{7}\footnote{ज \textendash\  दीप्तयः}र्चिषस्तथा~।\\
त्रस्तांगाक्षिनिमेषैश्च तेऽभिनेयाः प्रयोक्तृभिः~॥~१५}
\end{quote}


\hrule

\vspace{2mm}
\noindent
भिन्नोर्ध्ववलिता ह्यन्तराले भारो देहस्य भूषणमपहारी वक्षसः पुनरपि तमेव गत्या सर्वग्रहणं तथैव लोकस्येति कर्मोक्तं सूचीमुखस्य तादृशः परितो भ्रमणेन गमनविशेषमाह~। भ्रमणेन प्रदेशिन्याः दृष्टेः सर्वार्थेऽभिनयेऽभिनयान्तरमप्याह अलपद्मकपीडादिरिति~। अलपद्मकशब्देन तदङ्गुल्यः पीडा करतलेन तासां संयोगः, तत्र च बहुत्वं विवक्षितं तेनायमर्थः {\qt आवर्तिन्यः करतल} (९\textendash\ ९१) इति यः अलपल्लव उक्तस्तस्य क्रमेण कनीयः स्यादिति कांगुलीति हस्ततलेन संयोजयेदिति~। सर्वार्थाभिनयः~। एतच्च रूपमलपद्मस्य प्राङ्नोक्तम्~। अथ यदा कार्त्स्न्येन सर्वशब्दः प्रवर्तते~। तद्यथा सर्वः शब्द इत्यादौ तदा विशेषसहकारि\textendash\ णमाह श्रव्यं श्रवणयोगेनेति~। (२२\textendash\ ७६) {\qt कृत्वा साचीकृत्वा} तां दृष्टिं शिरः\textendash\ पार्श्वानतमित्याश्रवणयोगः तत्सहितशब्दविषयः सर्वाभिनयः विद्युदादिविषये येऽभिनया उक्ताः~। तद्यथा {\qt सूची विरलांगुली विद्युति चक्रं च तटिल्लता} (९\textendash\ ६६) इत्युक्तत्वात्~। तेषां विशेषमाह त्रस्तेति~। वस्तुत्रासाभिनर्योऽगानामक्ष्णोश्च

\newpage
% २६८ नाट्यशास्त्रम् 

\begin{quote}
{\na उद्वेष्टितपरावृत्तौ करौ कृत्वा नतं शिरः~।\\
असंस्पर्शे तथानिष्टे जिह्मदृष्टेन कारयेत्\renewcommand{\thefootnote}{1}\footnote{ज \textendash\  असंस्पर्शात्तथानिष्टं मा स्पृशेति च निर्दिशेत्}~॥~१६

वायु\renewcommand{\thefootnote}{2}\footnote{प \textendash\  ऊष्मं नभ}मुष्णं तमस्तेजो मुख\renewcommand{\thefootnote}{3}\footnote{च \textendash\  सं}प्रच्छादनेन च~।\\
रेणुतोयपतंगांश्च \renewcommand{\thefootnote}{4}\footnote{ज \textendash\  तमो भङ्गांश्च च \textendash\  पतङ्गानां भ्रमराणां च वारणम्}भ्रमरांश्च निवारयेत्~॥~१७

कृत्वा स्वस्तिकसंस्थानौ\renewcommand{\thefootnote}{5}\footnote{ज \textendash\  संस्थौ तु} पद्मकोशावधोमुखौ~।\\
सिंहर्क्षवानरव्याघ्रश्वापदांश्च निरूपयेत्\renewcommand{\thefootnote}{6}\footnote{ज \textendash\  रूपवेद् द्विजयत्तमाः}~॥~१८

स्वस्तिकौ त्रिपताकौ तु गुरूणां पादवन्दने~।\\
खटकस्वस्थिकौ चापि प्रतोद\renewcommand{\thefootnote}{7}\footnote{च \textendash\  प्रग्रहादिषु}ग्रहणे स्मृतौ~॥~१९}
\end{quote}

\hrule

\vspace{2mm}
\noindent
निमेषः संकोच इति~। विद्युदादिव्यङ्गयमप्यभिनयमाह उद्वेष्टितपरावृत्ताविति~। परावृत्तौ समन्तादुद्वेष्टितौ क्रमात् मुक्तकनीयस्याङ्गुलिगुणावित्येवं प्राग्भावे मुष्टिः पश्चाद्भागे तु पराङ्मुखोऽरालोभ इति~। तत्रा सादितुः स्थितमेवात्र जिह्म\textendash\ ध्ष्टेनेति~। जिह्मया दृष्ट्या {\qt लंबिता कुञ्चितपुटा शनैस्तिर्यङ्निरीक्षणैः निगूढा गूढतारा च जिह्मा दृष्टि} (८\textendash\ ८३) रिति {\qt सिंहव्याघ्रेष्वभिनय} (९\textendash\ १२१) इत्यूर्णनाभस्य कर्मोक्तम्~। तत्र विशेषमाह कृत्वा स्वस्तिकसंस्थानाविति~। पद्मकोशावित्यत्रोत्तराभावे चोक्ते एतत्प्रकृतत्वात् {\qt पद्मकोशस्य हस्तांगुल्यः कुञ्चिता} इत्यूर्णनाभः~। अन्ये पद्मकोशस्यैव तमनुक्तं कर्मेत्याहुः~। त्रिपताकस्या\textendash\ नुक्तं कर्म दर्शयति\textendash\ स्वस्तिकौ त्रिपताकौ च गुरूणां पादवन्दन इति~। खटका\textendash\ मुखस्य कर्मोक्तं {\qt मन्थानशरापकर्षणपुष्पापचयप्रतोदकार्याणीति} (९\textendash\ ६३)~। तत्र प्रतोदकृत्ये विशेषमाह खटकस्वस्तिकौ चापि प्रतोदग्रहण इति~॥

\newpage
% पञ्चविंशोऽध्यायः २६९ 

\begin{quote}
{\na एकं द्वि त्रीणि चत्वारि पञ्च षट् सप्त चाष्टधा\renewcommand{\thefootnote}{1}\footnote{ज \textendash\  वा}~।\\
नव वा दश वापि स्युर्गणनांगुलिभिर्भवेत्\renewcommand{\thefootnote}{2}\footnote{ज \textendash\  गण्डान्यङ्गुलिभिर्बुधाः}~॥~२०

दशाख्याश्च शताख्याश्च सहस्राख्यास्तथैव च~।\\
पताकाभ्यां तु हस्ताभ्यां प्रयोज्यास्ताः\renewcommand{\thefootnote}{3}\footnote{ज \textendash\  अभिनेयाः} प्रयोक्तृभिः~॥~२१

\renewcommand{\thefootnote}{4}\footnote{ज \textendash\  संख्यायास्तु दशभ्यस्तु परतोऽभ्यधिका यदा~। वाचिकेनैव साध्याः}दशाख्यगणनायास्तु परतो या भवेदिह~।\\
वाक्यार्थेनैव साध्यासौ परोक्षाभिनयेन च~॥~२२

छत्रध्वजपताकाश्च निर्देश्या दण्डधारणात्~।\\
\renewcommand{\thefootnote}{5}\footnote{ज \textendash\  स्वधारणैश्च रूक्ष्याणि नानाप्रहरणान्वपि}नानाप्रहरणं चाथ निर्देश्यं धारणाश्रयम्~॥~२३}
\end{quote}

\hrule

\vspace{2mm}
अथ सर्वव्यवहारोपयोगिसंख्याभिनयं प्रदर्शयन्नुक्तपूर्वाणामेव हस्तानामनुक्तं कर्म दर्शयति~। एतदुक्तं भवति सूचीमुखप्रदेशिन्या त्रिपताकांगुलिकांगुलांगुलिविधौ ऊर्ध्वलताष्टकांगुलिचतुष्केण मुकुलहस्तविकासतया चैकादितया पञ्चता संख्या, षडादिका तु द्वाभ्यां यथा मुकुले सूच्यास्येन षट् यावन्मुकुले यौ दशपर्यन्ता गणना यावतीति दशपर्यन्ता गणना यावत् ततः परन्तु बहुत्वा\textendash\ भिनय एवेत्याह दशा इत्यादि~। दशभिराख्या येषां तेऽत्र विंशत्यादयः विंशेति हि विंशति तस्मादत्र पताकाभ्यामभिनय इत्याह दशाख्येति~। आख्या\textendash\ शद्वोऽवधिवाची~। आख्यति प्रतते भावा यावतीति~। दशपर्यन्ता गणना यावतीति दशपर्यन्ता गणना यावर्ततः परा सा संक्षेपेणाभिनेया बहुत्वमात्रेण वाक्यार्थशद्वोऽत्र संक्षिप्तमुपलक्षयति~। पदार्था हि तत्र संक्षिप्यन्ते~। एतच्च परोक्षाभिनये~। प्रत्यक्षे त्वेकैकस्य निर्देशेनैव गणनेति यावत्~। आयतदण्ड\textendash\ ग्रहणमिति(९\textendash\ ६२)~। खटकामुखस्य सामान्येन कर्मोक्तम्~। तद्विशेष्यं दर्शयितुमाह छत्रध्वजपताकाश्चेति~। दण्डधारणे एव तत्प्रकारमन्तर्भूतमित्यर्थः~। चकारेणा\textendash\ भिनयान्तरमप्यत्र सूचयति~। अथशब्दे च~। तद्यथा शिरस उपर्यधोमुखः

\newpage
% २७० नाट्यशास्त्रम् 

\begin{quote}
{\na एकचित्तो ह्यधोदृष्टिः किञ्चिन्नतशिरास्तथा\renewcommand{\thefootnote}{1}\footnote{प \textendash\  शिरः किंचिन्नतं भवेत् ज \textendash\  शनैराकम्पयेच्छिरः}~।\\
सव्यहस्तश्च सन्दंशः स्मृते\renewcommand{\thefootnote}{2}\footnote{ज \textendash\  सन्दंशं कुर्यात्} ध्याने वितर्किते~॥~२४

उद्वाहितं शिरः कृत्वा हंसपक्षौ प्रदक्षिणौ~।\\
\renewcommand{\thefootnote}{3}\footnote{ज \textendash\  अपत्पाभिनयं कार्यमुच्छ्र्यश्च}अपत्यरूपणे कार्यावुछ्रयौ च प्रयोक्तृभिः~॥~२५

उद्वाहितं शिरः कृत्वा हंसवक्त्रं तथोर्ध्वगम्~।\\
प्रसादयच्च यं मानं दीर्घसत्वं च निर्दिशेत्~॥~२६

अरालं च शिरस्स्थाने समुद्वाह्य तु वामकम्~।\\
गते निर्वृत्ते ध्वस्ते च श्रान्तवाक्ये च योजयेत्~॥~२७

सर्वेन्द्रियस्वस्थतया प्रसन्नवदनस्तथा\renewcommand{\thefootnote}{4}\footnote{ज \textendash\  प्रसादेन मुखस्य च}~।\\
विचित्र\renewcommand{\thefootnote}{5}\footnote{ज \textendash\  कुसुम}भूतलालोकैः शरदन्तु\renewcommand{\thefootnote}{6}\footnote{ज \textendash\  न्त्वभि} विनिर्दिशेत्~॥~२८

गात्रसंकोचनाच्चापि सूर्याग्निपटुसेवनात्\renewcommand{\thefootnote}{7}\footnote{ज \textendash\  तथा शुल्काभिलाषतः प \textendash\  तथाग्न्य}~।\\
हेमन्तस्त्वभिनेतव्यः पुरुषैर्मध्यमाधमैः\renewcommand{\thefootnote}{8}\footnote{म \textendash\  उत्तमैः}~॥~२९}
\end{quote}

\hrule

\vspace{2mm}
\noindent
पताकछत्रे पञ्चांगुल्यूर्ध्वध्वजे सूचीमुख्यांगुलिपताकायां मुष्टिशिखरकपित्थमृगशीर्षकाद्या यथायोगं प्रहरणेषु योगध्याने स्तोक इति(९\textendash\ ११४) क्वचिद्विस्मिता अपूर्वशालिसंपद्दर्शनात् क्वचित् {\qt ग्लानाभीष्टादर्शना} दित्यादि दृष्टिभेदादि चाद्यो व्यसनसंभवस्तस्मिन्निमित्ता यावस्थान्तरप्राप्तिः ततो हेतोरिति संबन्धः~। रूक्षस्येति~। उद्वेजनस्य प्रावृङ्कर्षारात्रस्य यथा प्रथमे भागे वर्षितुं प्रावृड्मेघा\textendash

\newpage
% पञ्चविंशोऽध्यायः २७१ 

\begin{quote}
{\na शिरोदन्तोष्ठकम्पेन गात्रसंकोचनेन च~।\\
कूजितैश्च सशीत्कारैरधमश्शीतमादिशेत्~॥~३०

अवस्थान्तर\renewcommand{\thefootnote}{1}\footnote{ज \textendash\  संप्राप्तः}मासाद्य कदाचित्तूत्तमैरपि\renewcommand{\thefootnote}{2}\footnote{ज \textendash\  उत्तमोऽपि कदाचन}~।\\
शीताभिनयनं कुर्याद्देवा\renewcommand{\thefootnote}{3}\footnote{ज \textendash\  शीतवातस्य निर्देशं कुर्यादू\ldots संभवे}द्वयसनसंभवम्~॥~३१

\renewcommand{\thefootnote}{4}\footnote{ज \textendash\  अनिलस्य सुखस्पर्शादृतुजानां तथैव च~। गन्धाघ्राणेन पुष्पाणां}ऋतुजानां तु पुष्पाणां गन्धाघ्राणैस्तथैव च~।\\
रूक्षस्य वायोः स्पर्शाच्च शिशिरं रूपयेद्बुधः~॥~३२

\renewcommand{\thefootnote}{5}\footnote{ज\textendash\  आमोदजननैर्गन्धैः}प्रमोदजननारम्भैरुपभोगैः पृथग्विधिः\renewcommand{\thefootnote}{6}\footnote{ज \textendash\  ससंभ्रमैः}~।\\
वसन्तस्त्वभिनेतव्यो नानापुष्पप्रदर्शनात्\renewcommand{\thefootnote}{7}\footnote{ज \textendash\  नैः}~॥~३३

स्वेद\renewcommand{\thefootnote}{8}\footnote{ज \textendash\  अवमार्जनाच्चापि}प्रमार्जनैश्चैव भूमितापैः सवीजनैः\renewcommand{\thefootnote}{9}\footnote{ज \textendash\  तापोपवीजनैः~। स्पर्शनादृतुवायोश्च}~।\\
उष्णस्य वायोः स्पर्शेन ग्रीष्मं त्वभिनयेद्बुधः~॥~३४

कदम्बनीपकुटपैः शाद्वलैः सेन्द्रगोपकैः~।\\
\renewcommand{\thefootnote}{10}\footnote{ज \textendash\  मेघैर्म\textendash\ यूरनादैश्च प्रावृषं संनिरूपपेत्~। गम्भीरैर्मेघनादैश्च}मेघवातैः सुखस्पर्शैः प्रावृट्कालं प्रदर्शयेत्~॥~३५}
\end{quote}

\hrule

\vspace{2mm}
\noindent
यत्रेति~। निर्घातः आकाशस्फोटः शब्दो गर्जितादन्य एव चिह्नमिति पुष्पपुष्पादिशेषषङ्विशेष इति~। तत्कालोचितो वस्त्वाभरणप्रसाधनादिः यस्मिन्निति वक्तव्ये संबन्धमात्रापेक्षया षष्ठी कर्मेति क्वचिद्रूपानुसरणमित्यादिरूपं यथा क्वचिदातपः क्वचिच्चन्द्रोद्योतः~। भावानां विभावैरभिनयो यथा क्रोधस्य परस्थस्य

\newpage
% २७२ नाट्यशास्त्रम् 

\begin{quote}
{\na मेघौघनादैर्गम्भीरैर्धाराप्रपतनैस्तदा~।\\
विद्युन्निर्घातघोषैश्च वर्षारात्रं समादिशेत्~॥~३६

यद्यस्य\renewcommand{\thefootnote}{1}\footnote{ज \textendash\  अत्र} चिह्नं वेषो वा कर्म वा रूपमेव वा~।\\
निर्देश्यः स ऋतुस्तेन इष्टानिष्टार्थदर्शनात्\renewcommand{\thefootnote}{2}\footnote{ज \textendash\  नैः}~॥~३७

एतानृतूनर्थवशा\renewcommand{\thefootnote}{3}\footnote{ज \textendash\  प्रयुञ्जीत विचक्षणः~। दुःखोपेतांस्तु दुःखेन सुखोपेतान्सुखेषु च}द्दर्शयेद्धि रसानुगान्~।\\
सुखिनस्तु सुखोपेतान् दुःखार्थान् दुःखसंयुतान्~॥~३८

यो येन भावेनाविष्टः सुखदेनेतरेण वा~।\\
स तदाहितसंस्कारः सर्वं पश्यति तन्मयम्~॥~३९

\renewcommand{\thefootnote}{4}\footnote{च \textendash\  विभावेनाहृतं कार्यमनुभावे निरूपणात्}भावाभिनयनं कुर्याद्विभावानां निदर्शनैः~।\\
तथैव चानुभावानां भावसिद्धिः प्रवर्तिता\renewcommand{\thefootnote}{5}\footnote{ज \textendash\  प्रवर्तितेः}~॥~४०

विभावेनाहृतं कार्यमनुभावेन नीयते\renewcommand{\thefootnote}{6}\footnote{ज \textendash\  भावेन प्रतिदर्शयेत्}~।}
\end{quote}

\hrule

\vspace{2mm}
\noindent
सूचीमुखांगुल्यादिपरः सन्निर्दिश्यते तद्द्वारेण क्रोधः~। अनेन च सहाध्याये यदुक्तं तत्राभिप्रायविशेषो दर्शितः~। तत्र ह्युक्तं {\qt रिपुर्देशे तथैव क्रोध} (२२\textendash\ अध्या.)इति~। एवं स्नेहाख्यपद्यरूपकेण हंसपक्षेणानुभावानां भावसिद्धया प्रवर्त्तितमभिनयं कुर्यादिति संबन्धः~। \ldots \ldots णोऽभिनयः शोकोचितेन मुखविकूणनादिना~। ननु विभावः कथमभिनय इत्याशंक्याह विभावेनाहृतं कार्यमिति~। विभावः करणत्वाद्गमक इति यावत्~। विभावेन हि कार्यमाहृतः~। सामग्री हि कार्यं व्यभिचरन्ती गमयत्येव~।

\newpage
% पञ्चविंशोऽध्यायः २७३ 

\begin{quote}
{\na आत्मानुभवनं भावो विभावः परदर्शनम्~॥~४१

गुरुर्मित्रं सखा स्निग्धः संबन्धी बन्धुरेव वा~।\\
आवेद्यते हि यः प्राप्तः स विभाव इति स्मृतः~॥~४२

यत्त्वस्य संभ्रमोत्थानैरर्घ्यपाद्यासनादिभिः\renewcommand{\thefootnote}{1}\footnote{ज \textendash\  अर्ध्यासनपरिग्रहैः}~।\\
पूजनं क्रियते\renewcommand{\thefootnote}{2}\footnote{ब \textendash\  वाचा} भक्तया \renewcommand{\thefootnote}{3}\footnote{ज \textendash\  स्वभाव इति कीर्तितः}सोऽनुभावः प्रकीर्तितः~॥~४३

एवमन्येष्वपि ज्ञेयो नानाकार्यप्रदर्शनात्~।}
\end{quote}

\hrule

\vspace{2mm}
ननु भावः कथमनुभावस्य गमक इत्याह अनुभावेनेति~। अनुभवान्तरे साहचर्यानुभवात् गमकमित्येतदमुत्र तत्त्वम्~। यदुक्तं भावसिद्धिप्रवर्तितमनुभावानामभिनयं कुर्यादिति~। अथ विस्मरणशीलान् प्रति श्रृंगग्राहिकया भावविभावानुभावस्वरूपं दर्शयति आत्मानुभवनं भाव इत्यादि~। आत्मविश्रान्तं यदनुभवनं सुखदुःखसंविद्रूपं स भाव इत्यर्थः~। आत्मग्रहणात् घटाद्यनुभवनं न भाव इत्युक्तं भवति~। णिचमन्ये पठन्ति~। तत्रार्थः आत्मानुभाव्यते येन न च तादृगर्थस्तदस्तीति प्रकर्षो गम्यते~। तेन यल्लब्धसत्तार्थकं चेत् तदवश्यमनुभूयते सुखादिरूपम्~। तदेव भाव इत्युक्तं भवति~।\\

यत्तु व्यतिरिक्तवस्तुज्ञानं तत्सर्वं सुखादिजनकत्वाद्विभावः~। तदाह विभावः परदर्शनमिति~। तदुदाहरति गुरुर्मित्रमित्यादि~। गुरुदर्शने सति विनयग्रहणे आदाबुत्साह एव मित्रादेर्यथोचितं हर्षादनुविभावत्वं योज्यम्~। मित्रं कार्यवशात् समानख्यातियोगत्वात् सखा सहपांसुक्रीडनापरिचितः आवेद्यत इत्यनेन दर्शनविषयस्यैव भावतेति दर्शयति~। प्राप्त इत्यनेन चित्तवृत्ति\textendash\ जन्मनि गुर्वादेरन्वयव्यतिरेकौ सूचयन् कारणमाह~। यत्त्वस्येति गुर्वादेः संभ्रमेण यदुत्थानं प्रत्युद्गमनं बहुवचनाद्यततं सूचकौ यथोचितं भावे संग्रहालिङ्गनादिवचनेनेत्यर्थः~। उदाहरणमात्रमेतदिति दर्शयति एवमन्येष्वपीति~।

\lfoot{35}

\newpage
\lfoot{}
% २७४ नाट्यशास्त्रम् 

\begin{quote}
{\na विभावो वापि भावो वा विज्ञेयोऽर्थवशाद्बुधैः~॥~४४

यस्त्वपि प्रतिसंदेशो दूतस्येह प्रदीयते~।\\
सोऽनुभाव इति ज्ञेयः \renewcommand{\thefootnote}{1}\footnote{ज \textendash\  पर}प्रतिसन्देशदर्शितः~॥~४५

एवं भावो विभावो वाप्यनुभावश्च कीर्तितः~।\\
पुरुषैरभिनेयः स्यात्प्रमदाभिरथापि वा~॥~४६

स्वभावाभिनये स्थानं पुंसां कार्यं तु वैष्णवम्~।\\
आयतं वावहित्थं वा स्त्रीणां कार्यं स्वभावतः~॥~४७

प्रयोजनवशाच्चैव शेषाण्यपि भवन्ति हि~।\\
नानाभावाभिनयनैः प्रयोगैश्च पृथग्विधैः\renewcommand{\thefootnote}{2}\footnote{ज \textendash\  भावानभिनयैः कार्यं नित्यं प्रयोक्तृभिः}~॥~४८}
\end{quote}

\hrule

\vspace{2mm}
\noindent
रसेषु श्रृङ्गारादिष्विति भावः एकेन वा ग्रहणेनानुभावः सूचितः~। द्वितीयो विकल्पार्थः अपिशब्देन स्थायिव्यभिचारिरूपतां समुच्चिनोति~। अर्थवशादिति प्रयोजनवशात्~।\\

ननु नियमेनायमस्याश्चित्तवृत्तौ विभाव इति शक्यं वक्तुम्~। प्रयोजनान्तरयोगे तस्यैवान्यत्र विभावत्वदर्शनात्~। न केवलं प्रत्यक्षेण दृश्य एवानुभवश्चित्तवृत्तिं गमयति~। यावत्प्रमाणान्तरेण शब्दादिनाप्यविदित इति दर्शयितुमाह यस्त्वपि प्रतिसन्देश इति~। एतच्चानुमानस्याप्युपलक्षणम्~। सन्तमसे हि गद्गदगुरुसज्जनवचनानुमेयात्~। बाष्पादपि हि भवति शोकावगमः~। एतदुपसंहरति एवमित्यादि~। पुरुषैः प्रमदाभिर्वेत्युक्तं तत्रानावेश्यादेषां स्थानभावेदयति~। गतिं च दर्शयति स्वभावाभिनय इत्यादिना~। शेषाणीति~। स्थानान्त राण्यपीति यावत्~। यदुक्तम्\textendash

\newpage
% पञ्चविंशोऽध्यायः २७५ 

\begin{quote}
{\na \renewcommand{\thefootnote}{1}\footnote{प \textendash\  स्थैर्य ज \textendash\  धैर्योदात्त}धैर्यलीलांगसंपन्नं \renewcommand{\thefootnote}{2}\footnote{ज \textendash\  नराणामपि}पुरुषाणां विचेष्टितम्~।\\
मृदुलीलांगहारैश्च\renewcommand{\thefootnote}{3}\footnote{ज \textendash\  हारं तु} स्त्रीणां कार्यं तु चेष्टितम्~॥~४९

करपादांग\renewcommand{\thefootnote}{4}\footnote{ज \textendash\  अग्र}सञ्चारास्स्त्रीणां तु ललिताः स्मृताः~।\\
\renewcommand{\thefootnote}{5}\footnote{ज \textendash\  धीरोदात्तास्तु विज्ञेया}सुधीरश्चोद्धतश्चैव पुरुषाणां प्रयोक्तृभिः~॥~५०

यथा रसं यथाभावं स्त्रीणां भावप्रदर्शनम्~।\\
नराणां प्रमदानां च भावाभिनयनं पृथक्\renewcommand{\thefootnote}{6}\footnote{ज \textendash\  शब्दार्थाभिनयं बुधाः (न \textendash\  पृथक्)}~॥~५१

भावानुभावनं युक्तं\renewcommand{\thefootnote}{7}\footnote{न \textendash\  संयुक्तं} व्याख्यास्याम्यनुपूर्वशः~।\\
आलिंगनेन गात्राणां सस्मितेन च चक्षुषा~॥~५२

तथोल्लुकसनाच्चापि \renewcommand{\thefootnote}{8}\footnote{प \textendash\  प्रहर्षं संप्रयोजयेत् द \textendash\  हर्षं पुंसां प्र\ldots \ldots }हर्षं सन्दर्शयेन्नरः~।\\
क्षिप्रसञ्जातरोमाञ्चात्\renewcommand{\thefootnote}{9}\footnote{भ \textendash\  रोमाञ्चा} बाष्पेणावृतलोचना~॥~५३

कुर्वीत नर्तकी हर्षं प्रीत्या वाक्यैश्च सस्मितैः\renewcommand{\thefootnote}{10}\footnote{प \textendash\  प्रीतियुक्ता स्मितेन च}~।\\
उद्वृत्तरक्तनेत्रश्च\renewcommand{\thefootnote}{11}\footnote{प \textendash\  नेत्रास्यास्स} सन्दष्टाधर एव च~॥~५४}
\end{quote}

\hrule

\begin{quote}
{\qt धैर्यलीलाङ्गसंपन्नं कृत्वा पुरुषचेष्टितम्~।\\
प्रयोक्तृभिः प्रयोक्तव्यं स्त्रीणां चेष्टित मन्यथा~॥}
\end{quote}

\noindent
मार्दवलीलाप्रधानैरङ्गविक्षेपैरिति तत्र तथा सिद्धत्वं हेतुमाहकरपादाङ्गसञ्चार इति~। तुर्हेतौ~। आलिङ्गनेन गात्राणामिति~। स्वात्मीयानामेवान्योन्यमासे\ldots \ldots णेत्यर्थः~। क्रोधस्त्वभिनयेदिति व्यभिचरितः प्राप्तिमिति मन्तव्यम्~। रसेषु हि

\newpage
% २७६ नाट्यशास्त्रम् 

\begin{quote}
{\na निश्वासकम्पितांगश्च क्रोधं चाभिनयेन्नरः~।\\
\renewcommand{\thefootnote}{1}\footnote{प \textendash\  बाष्पपूर्णैः क्षणत्वाच्च}नेत्राभ्यां बाष्पपूर्णाभ्यां चिबुकौष्ठप्रकम्पनात्~॥~५५

शिरसः कम्पनाच्चैव भ्रकुटीकरणेन च~।\\
मौनेनांगुलिभंगेन माल्याभरणवर्जनात्~॥~५६

आयतस्थानकस्थाया ईर्ष्या क्रोधे\renewcommand{\thefootnote}{2}\footnote{ज \textendash\  क्रोधो} भवेत्स्त्रियाः~।\\
निश्वासोच्छ्वासबहुलैरधोमुखविचिन्तनैः~॥~५७

आकाश\renewcommand{\thefootnote}{3}\footnote{द \textendash\  वीक्षणात्}वचनाच्चापि दुःखं पुंसां तु योजयेत्~।\\
रुदितैः\renewcommand{\thefootnote}{4}\footnote{द \textendash\  सस्वरैः} श्वसितैश्चैव शिरोभिहननेन च~॥~५८

भूमि\renewcommand{\thefootnote}{5}\footnote{प \textendash\  हस्त}पाताभिघातैश्च दुःखं स्त्रीषु प्रयोजयेत्~।\\
आनन्दजं चार्तिजं वा\renewcommand{\thefootnote}{6}\footnote{प \textendash\  आनन्दार्थलमुत्पन्नं} ईर्ष्यासंभूतमेव वा~॥~५९

यत्पूर्वमुक्तं रुदितं तत्स्त्रीनीचेषु योजयेत्~।\\
संभ्रमावेगचेष्टाभिश्शस्त्रसंपातनेन च~॥~६०

पुरुषाणां भयं कार्यं\renewcommand{\thefootnote}{7}\footnote{प \textendash\  रिपोर्धैर्य} धैर्यावेगबलादिभिः~।\\
\renewcommand{\thefootnote}{8}\footnote{द \textendash\  नेत्रप्रचलहस्तभ्रूगात्रस्फुरणकम्पनैः प \textendash\  तथा चलितनेत्रत्वाद्गात्राणां कम्पनादपि}चलतारकनेत्रत्वाद्गात्रैः स्फुरितकम्पितैः~॥~६१}
\end{quote}

\hrule

\vspace{2mm}
\noindent
सामान्याभिनयः श्रृङ्गारद्वारेण दर्शितः~। अत्र तु व्यभिचारिषु दर्श्यते~। दुःख\textendash\ मिति शोकः भूम्यां हस्ताभ्यां च ये घाता हस्तताडनानि तथाभूतैर्हस्ताभ्यां घातास्तैराकाशस्येति शून्योऽप्यवलम्बनप्रवृत्तेत्यर्थः~। विलग्नं कलासंकथितानि

\newpage
% पञ्चविंशोऽध्यायः २७७ 

\begin{quote}
{\na सन्त्रस्तहृदयत्वाच्च \renewcommand{\thefootnote}{1}\footnote{प \textendash\  तथा पार्श्वावलोकनात्}पार्श्वाभ्यामवलोकनैः~।\\
\renewcommand{\thefootnote}{2}\footnote{न \textendash\  त्रातुः द \textendash\  भ्रातुः}भर्तृरन्वेषणाच्चैवमुच्चैराक्रन्दनादपि\renewcommand{\thefootnote}{3}\footnote{द \textendash\ नेन च}~॥~६२

\renewcommand{\thefootnote}{4}\footnote{द \textendash\  पुरुष}प्रियस्यालिंगनाच्चैव भयं कार्यं भवेत्स्त्रियाः~।\\
\renewcommand{\thefootnote}{5}\footnote{न \textendash\  मया}मदा येऽभिहिताः पूर्वं तो स्त्रीनीचेषु योजयेत्~॥~६३

मृदुभिः \renewcommand{\thefootnote}{6}\footnote{न \textendash\  ललितैः कार्यं}स्खलितैर्नित्यमाकाशस्यावलंबनात्\renewcommand{\thefootnote}{7}\footnote{प \textendash\  सुकुमारैस्तु ललितैः पार्श्वानतविलम्बितैः}~।\\
नेत्रावघूर्णनैश्चैव \renewcommand{\thefootnote}{8}\footnote{द \textendash\  विलग्नैः न \textendash\  विलासकलितैः}सालस्यैः कथितैस्तथा~॥~६४

गात्राणां कम्पनैश्चैव मदः कार्यो भवेत्स्त्रियाः~।\\
अनेन विधिना कायः प्रयोगाः \renewcommand{\thefootnote}{9}\footnote{द \textendash\  करणान्वितः}कारणोत्थिताः~॥~६५

\renewcommand{\thefootnote}{10}\footnote{ज \textendash\  स्त्रीकृताः पौरूषा ये वा भावाभिनयनं प्रति}पौरुषः स्त्रीकृतो वापि भावा ह्यभिनयं प्रति~।\\
सर्वे सललिता भावास्स्त्रीभिः कार्याः प्रयत्नतः\renewcommand{\thefootnote}{11}\footnote{न \textendash\  स्त्रीणां कार्याः प्रयोक्तृभिः}~॥~६६

धैर्यमाधुर्यसंपन्ना भावाः कार्यास्तु पौरुषाः\renewcommand{\thefootnote}{12}\footnote{ज \textendash\  पुरुषाणां प्रयोजयेत्}~।\\
त्रिपताकां\renewcommand{\thefootnote}{13}\footnote{द \textendash\  काभ्यां तु हस्ताभ्यां} गुलीभ्यां तु वलिताभ्यां प्रयोजयेत्~॥~६७

शुकाश्च शारिकार्श्चेव सूक्ष्मा ये चापि पक्षिणः~।\\
शिखिसारसहंसाद्याः स्थूला येऽपि स्वभावतः~॥~६८}
\end{quote}

\hrule

\vspace{2mm}
\noindent
चलिताभ्यामिति मन्थरं चरं चरन्तीत्यारेचितकैरङ्गहारैरिति तुर्याध्यायनिरूपितैर्गतिप्रचारैरिति तदुचितैरेव शिरोग्रीवादिकर्मभिः भयोद्वेगौ स्त्रीनीचानां

\newpage
% २७८ नाट्यशास्त्रम् 

\begin{quote}
{\na \renewcommand{\thefootnote}{1}\footnote{ज \textendash\  पक्षाङ्गहारैर्विविधैस्तेऽभिनेयाः प्रयोक्तृभिः}रेचकैरंगहारैश्च तेषामभिनयो भवेत्~।\\
खरोष्ट्राश्वतरासिंहव्याघ्रगोमहिषादयः~॥~६९

\renewcommand{\thefootnote}{2}\footnote{द \textendash\  महापशूनङ्गहारैर्गतिभिश्च प्रयोजयेत्}गतिप्रचारैरंगैश्च तेऽभिनेयाः प्रयोक्तृभिः~।\\
भूताः पिशाचा यक्षाश्च दानवाः सहराक्षसैः~॥~७० 

अंगहारैर्विनिर्देश्या \renewcommand{\thefootnote}{3}\footnote{द \textendash\  कर्म}नामसंकीर्तनादपि~।\\
अंगहारैर्विनिर्देश्या अप्रत्यक्षा भवन्ति ये~॥~७१

प्रत्यक्षास्त्वभिनेतव्या भयोद्वेगैः सविस्मयैः~।\\
देवाश्च चिह्नैश्च प्रणामकरणैर्भावैश्च विचेष्टितैः~॥~७२

\renewcommand{\thefootnote}{4}\footnote{ज \textendash\  अनुकरणादिविधानादप्रत्यक्षानभिनयेत्ताः}अभिनेयो ह्यर्थवशादप्रत्यक्षाः प्रयोगज्ञैः~।\\
\renewcommand{\thefootnote}{5}\footnote{ज \textendash\  पार्श्वोत्थितेन मध्येन}सव्योत्थितेन हस्तेन ह्यरालेन शिरः स्पृशेत्~॥~७३

\renewcommand{\thefootnote}{6}\footnote{ज \textendash\  वन्दनं पुरुषाणां तु परोक्षं संप्रयोजयेत्}नरेऽभिवादनं ह्येतदप्रत्यक्षे विधीयते~।\\
खटकावर्धमानेन \renewcommand{\thefootnote}{7}\footnote{ज \textendash\  पताकाल्येन वा तथा}कपोताख्येन वा पुनः~॥~७४

दैवतानि गुरूंश्चैव प्रमदाश्चाभिवादयेत्~।\\
दिवौकसश्च ये पूज्याः प्रत्यक्षाश्च भवन्ति ये~॥~७५}
\end{quote}

\hrule

\vspace{2mm}
\begin{sloppypar}
\noindent
राक्षसादिदर्शने विस्मयस्तूत्तमानाम्~। अत्र च भावादिगता अनुभावाः\ldots \ldots \ldots दिशद्वैरक्ता भावैः \ldots \ldots यथा गदतो रुद्रस्य रौद्राभिनयस्य चेष्टितानि यथा
\end{sloppypar}

\newpage
% पञ्चविंशोऽध्यायः २७९ 

\begin{quote}
{\na तान् प्रमाणैः प्रभावैश्च गम्भीरार्थैश्च योजयेत्~।\\
महाजनं सखीवर्गं विटधूर्तजनं तथा~॥~७६

परिमण्डलसंस्थेन हस्तेनाभिनयेन्नरः~।\\
पर्वतान् प्रांशुयोगेन\renewcommand{\thefootnote}{1}\footnote{ज \textendash\  भावेन} वृक्षांश्चैव समुच्छ्रितान्~॥~७७

प्रसारिताभ्यां बाहुभ्यामुत्क्षिप्ताभ्यां प्रयोजयेत्~।\\
समूहसागरं सेनां बहुविस्तीर्णमेव च~॥~७८

पताकाभ्यां तु हस्ताभ्यामुत्क्षिप्ताभ्यां प्रदर्शयेत्\renewcommand{\thefootnote}{2}\footnote{ज \textendash\  विक्षिप्ताभ्यां प्ररूपयेत्}~।\\
शौर्यं धैर्यं च \renewcommand{\thefootnote}{3}\footnote{ज \textendash\  पूजां}गर्वं च दर्पमौदार्यमुच्छ्र्यम्~॥~७९

ललाटदेश\renewcommand{\thefootnote}{4}\footnote{ज \textendash\  स्थान}स्थानेन त्वरालेनाभिदर्शयेत्~।\\
वक्षोदेशादपाविद्धौ \renewcommand{\thefootnote}{5}\footnote{ज \textendash\  कृत्वा}करौ तु मृगशीर्षकौ~॥~८०

विस्तीर्णप्रद्रुतोत्क्षेपौ योज्यौ यत्स्यादपावृतम्~।\\
अधोमुखोत्तानतलौ हस्तौ किञ्चित्प्रसारितौ~॥~८१

कृत्वा त्वभिनयेद्वेलां\renewcommand{\thefootnote}{6}\footnote{ज \textendash\  क्षिप्रं कुर्यात्} बिलद्वारं गृहं गुहान्~।\\
कामं शापग्रहग्रस्तान् ज्वरोपहतचेतसः~॥~८२

\renewcommand{\thefootnote}{7}\footnote{ज \textendash\  एवं विधा नरा ये तु कुर्यात्तेषां विचेष्टितम् न \textendash\  तेषामभिनयः कार्यो मुखगात्रविचेष्टितैः}एतेषां चेष्टितं कुर्यादंगाद्यैः सदृशैर्बुधैः~।}
\end{quote}

\hrule

\vspace{2mm}
\noindent
रुद्रस्य संचिह्नानि यथास्य त्रिशूलं परिमण्डलत्वेन सम्यग् ज्ञातं यस्येति प्रकरणादत्र पताक एव विशेषो मन्तव्यः~। अङ्गाद्यैरिति~। आदिग्रहणाददृष्टिसात्त्विकपरिग्रहः~। सदृशैरिति डोलाहस्तादिरूपैः~। विलोलनैरिति चञ्चलैरिति

\newpage
% २८० नाट्यशास्त्रम् 

\begin{quote}
{\na दोलाभिनयनं कुर्याद्दोलायास्तु विलोलनैः\renewcommand{\thefootnote}{1}\footnote{ज \textendash\  कनैः भ \textendash\  कार्ये डोलान्दोलनलीलया}~॥~८३

संक्षोभेण च\renewcommand{\thefootnote}{2}\footnote{ज \textendash\  भणेन} गात्राणां \renewcommand{\thefootnote}{3}\footnote{ज \textendash\  रज्जुप्र प \textendash\  रज्वासं}रज्वश्वाग्रहणेन च~।\\
यदा \renewcommand{\thefootnote}{4}\footnote{ज \textendash\  चाङ्क}चांगवती डोल \renewcommand{\thefootnote}{5}\footnote{न \textendash\  भवेत्प्रत्यक्षसंश्रयाः~। उदात्तारोहणं कार्ये लोकानुकरणाश्रयम~। न चेदङ्गवती डोला}प्रत्यक्षा पुस्तजा भवेत्\renewcommand{\thefootnote}{6}\footnote{ज \textendash\  तदा त्वारोहणं कार्ये लोकानुकरणाश्रयम्}~॥~८४

आसनेषु\renewcommand{\thefootnote}{7}\footnote{ज \textendash\  तूष} प्रविष्टानां कर्तव्यं तत्र डोलनम्~।\\
आकाशवचनानीह वक्ष्याम्पात्मगतानि च~॥~८५

अपवारितकं चैव जनान्तिकमथापि च~।\\
दूरस्थाभाषणं यत्स्यादशरीरनिवेदनम्~॥~८६

परोक्षान्तरितं वाक्यमाकाशवचन तु तत्~।\\
तत्रोत्तरकृतैर्वाक्यैः संलापं संप्रयोजयेत्~॥~८७

नानाकारणसंयुक्तैः काव्यभावसमुत्थितैः~।\\
हृदयस्य वचो यत्तू तदात्मगतमिष्यते~॥~८८

सवितर्कं च तद्योज्यं प्रायशो नाटकादिषु~।}
\end{quote}

\hrule

\vspace{2mm}
\noindent
भावः~। अथ वाचिकप्रसंगाच्चत्राभिनयं वक्तुं प्रतिजानीते आकाशवचनानीत्यादि~। दूरस्थेन रङ्गमप्रविष्टेनैव पात्रेण सहाभाषणमत एवाह अशरीरं यन्निवेदनमिति परोक्तेन प्रविष्ष्टपालसंबन्धिन्यान्तहितं व्यवहितं~। नन्वप्रविष्टस्य संबन्धिवचनं केनोदीर्यत इत्याशंक्याह तत्रोत्तरकृतैरिति~। उत्तरत्वेन यानि कृतानि वाक्यानि {\qt मैवं ब्रवीषि} इति तैः प्रयोजयेदिति प्राक् प्रविष्टस्यैव पात्रस्य कर्तृत्वं परोक्तवचनमनुभाषणच्छायाप्रविष्ट एवं ब्रूयादिति तात्पर्यम्~।

\newpage
% पञ्चविंशोऽध्यायः २८१ 

\begin{quote}
{\na निगूढभावसंयुक्तमपवारितकं स्मृतम्~॥~८९

कार्यवशादश्रवणं पार्श्वगतैर्यज्जनान्तिकं तत्स्यात्~।\\
हृदयस्थं सविकल्पं भावस्थं चात्मगतमेव~॥~९०

\renewcommand{\thefootnote}{1}\footnote{प \textendash\  यानि गुह्यर्थ}इति गुढार्थयुक्तानि वचनानीह नाटके~।\\
\renewcommand{\thefootnote}{2}\footnote{प \textendash\  तानि कर्णनिवेद्यानि एवमित्यभिधाय च}जनान्तिकानि कर्णे तु तानि योज्यानि योक्तृभिः~॥~९१

पूर्ववृत्तं तु यत्कार्यं भूयंः कथ्यं तु कारणात्~।\\
\renewcommand{\thefootnote}{3}\footnote{ज \textendash\  वाच्यं कृत्वा प्रदेशे तु न स्यादुक्तं पुनर्यथा}कर्णप्रदेशे तद्वाच्यं मागात्तत्पुनरुक्तताम्~॥~९२

अव्यभिचारेण पठेदाकाशजनान्तिकात्मगतपाठ्यम्~।\\
प्रत्यक्षपरोक्षकृतानात्मसमुत्थान् परकृतांश्च~॥~९३

हस्तमन्तरितं कृत्वा त्रिपताकं प्रयोक्तृभिः~।\\
जनान्तिकं प्रयोक्तव्यमपवारितकं तथा~॥~९४

स्वप्नायितवाक्यार्थर्त्वभिनेयो न खलु हस्तसंचारैः~।\\
सुप्ताभिहितैरेव तु वाक्यार्थैः सोऽभिनेयः स्यात्~॥~९५}
\end{quote}

\hrule

\vspace{2mm}
\noindent
निगूढभावो निगूढनं सर्वेषां यन्निगूह्यते एक एव श्रृणुयादिति तदपवारितं जनान्तिकं एकान्तिकत्वं चैकस्यैव निगूह्यत इति विशेषः~॥\\

अन्ये त्वाहुः~। उभयमित्येतज्जनान्तिकमेव~। यावतो हि जनस्य तद्वक्तव्यं तावतोऽन्तिके सामीप्ये तदुच्यते~। अपवारितकं तु तदुच्यते यत्र तूहात्परमुद्दिश्य नोच्यते~। अथ च परः श्रुणोत्वित्ययमेवाशयो वचने तदपवारितकं तेन निगूढेन भावेनाशयेन संयुक्तमव्याभिचारेणेत्युक्तपूर्वे कालादिसर्वमन्त्रानुसरेदिति यावत्~। न खल्विति~। न तत्र हस्ताभिनय इत्यर्थः~। सुप्ताभिहितैरित्युक्तं

\lfoot{36}

\newpage
% २८२ नाट्यशास्त्रम् 
\lfoot{}

\begin{quote}
{\na मन्दस्वरसञ्चारैर्व्यक्ताव्यक्तं पुनरुक्तवचनार्थम्~।\\
पूर्वानुस्मरणकृतं कार्यं स्वप्नाञ्जिते पाठ्यम्~॥~९६

प्रशिथिलगुरुकरुणाक्षरघण्टानुस्वरितवाक्यगद्गदजैः~।\\
हिक्काश्वासोपेतां काकुं कुर्यान्मरणकाले~॥~९७

हिक्काश्वासोपेतां मूर्च्छोपगमे मरणवत्कथयेत्~।\\
अतिमत्तेष्वपि कार्यं तद्वत्स्वप्नायिते यथा पाठ्यम्\renewcommand{\thefootnote}{1}\footnote{ज \textendash\  पाठ्यं पुनरुक्तसंयुक्तम्}~॥~९८

वृद्धानां योजयेत्पाठ्यं गद्गदस्खलिताक्षरम्~।\\
असमाप्ताक्षरं चैव बालानां तु कलस्वनम्~॥~९९

नानाभावोपगतं मरणाभिनये बहुकीर्तितं तु~।\\
विक्षिप्तहस्तपादैर्निभृतैः सन्नैस्तथा कार्यम्\renewcommand{\thefootnote}{2}\footnote{प \textendash\  तदा गात्रैः}~॥~१००

व्याधिप्लुते च मरणं निषण्णगात्रैस्तु संप्रयोक्तव्यम्~।\\
हिक्काश्वासोपेतं \renewcommand{\thefootnote}{3}\footnote{प \textendash\  त्वनवेक्षित}तथा पराधीनमात्रसंचारम्~॥~१०१}
\end{quote}

\hrule

\vspace{2mm}
\noindent
तानि लक्षणतः कथयति मन्दस्वरसंचारैरित्यादि~। पूर्वानुस्मरणेन कृतं प्रयुक्तं प्रशिथिलानि स्वस्थानतो भ्रंशमानानि गुरूणि स्वकर्माण्यचतुराणि यानि जिह्माग्रोपाग्रमध्यमूलानि तेषां संबन्धीनि यान्यक्षराणि तथा ते चलद्घण्टावदनुकरणं प्रधानं यद्वाक्यं तत्र यो गद्गदस्वरभेदः ततो ये जातास्तारमन्द्रादयः उदात्तानुदात्तादयश्च तैरुपलक्षितां मरणकाले काकुं कुर्यादिति संबन्धः~। कलस्वनमिति मधुरस्वरम्~।\\

एवमाकाशभाषितमात्मगतमपवारितं जनान्तिकं कर्णोक्तं स्वभायितोक्तमरणमूर्छामदभाषितं वृद्धबालोक्तमिति वचनगतं चित्राभिनयमुक्त्वा मरणप्रसङ्गेन~। तद्नतमपि कथयितुमाह नानाभावोपगतमिति~। विषं पीतमनेनेति विषपीतः~।

\newpage
% पञ्चविंशोऽध्यायः २८३ 

\begin{quote}
{\na विषपीतेऽपि च मरणं कार्यं विक्षिस्तगात्रकरचरणम्~।\\
विषवेगसंप्रयुक्तं विस्फुरितागक्रियोपेतम्~॥~१०२

प्रथमे वेगे कार्श्यं त्वभिनेये वेपथुर्द्वितीये तु\renewcommand{\thefootnote}{1}\footnote{ज \textendash\  विषस्य कुर्यात् प्रकम्पनं परतः}~।\\
दाहस्तथा तृतीये विलल्लिका स्या\renewcommand{\thefootnote}{2}\footnote{ज \textendash\  हिक्का कुर्यात्} च्चतुर्थे तु~॥~१०३

फेनस्तु पञ्चमस्थे तु\renewcommand{\thefootnote}{3}\footnote{प \textendash\  पञ्चमं कुर्यात्} ग्रीवा षष्ठे भज्यते~।\\
जडता सप्तमे तु स्यान्मरणं त्वष्टमे भवेत्~॥~१०४

तत्र प्रथमवेगे तु क्षामवक्रकपोलता~।\\
कृशत्वेऽभिनयः कार्यो वाक्यानामल्पभाषणम्~॥~१०५

सर्वागवेपथुं च कण्डूयनं तथांगानान्\renewcommand{\thefootnote}{4}\footnote{ज \textendash\  च गात्राणां}~।\\
विक्षितहस्तगात्रं दाहं चैवाप्यभिनयेत्तु~॥~१०६

\renewcommand{\thefootnote}{5}\footnote{ज\textendash\  पादं}उद्दृत्तनिमेषत्वादुद्गारच्छर्दनैरतथाक्षेपैः~।\\
अव्यक्ताक्षरकथनैः\renewcommand{\thefootnote}{6}\footnote{प \textendash\  उन्मेष} विलल्लिकामभिनयेदेवम्\renewcommand{\thefootnote}{7}\footnote{प \textendash\  धिकामेवं प्रयुञ्जीत}~॥~१०७

उद्गारवमनयोगैः शिरसश्च विलोलनैरनेकविधैः\renewcommand{\thefootnote}{8}\footnote{प \textendash\  सकोल्लेहैर्विलोलनैः शिरसः}~।\\
फेनस्त्वभिनेतव्यो निःसंज्ञतया निमेषैश्च~॥~१०८}
\end{quote}

\hrule

\vspace{2mm}
\noindent
तत्रैतस्य दष्टकस्याप्युपलक्षणम्~। विषस्य वेगाक्रमणेन धातुषु रसादिष्वोजः\textendash\ पर्यन्तेषु सञ्चरणं प्रथमे वेगे यत्कृशत्वमत्नाभिनयः कार्य इति संबम्धः~। चलति कामिला छर्दिः प्रारंभ इवान्तरो दाह्नं वार्युद्रेकः अंसयोः कपोलाभ्यां स्पर्श\textendash

\newpage
% २८४ नाट्यशास्त्रम् 

\begin{quote}
{\na अंसकपोलस्पर्शः शिरसोऽथ विनामनं शिरोऽपांगः\renewcommand{\thefootnote}{1}\footnote{प \textendash\  स्पर्शात् ग्रीवाभङ्गाद्विवर्तनाच्छिरसः~। बाह्येन्द्रिय}~।\\
सर्वेन्द्रियसंमोहाज्जडतामेवं त्वभिनयेत्तु~॥~१०९

संमीलितनेत्रत्वात् \renewcommand{\thefootnote}{2}\footnote{प \textendash\  प्रचारणाद्वा मद्दीनिपतनाश्च~। व्याधिविषाभ्यां मरणं कर्तव्यं नर्तकैरेवम्~। एतेऽभिनयविशेषाः कर्तव्या भावसत्वसंयुक्ताः~। अन्ये च लौकिका ये तु ते सर्वे लोकतः कार्याः}व्याधिविवृद्धौ भुजंगदशनाद्वा~।\\
एवं हि नाट्यधर्मे मरणानि बुधैः प्रयोज्यानि~॥~११०

संभ्रमेष्वथ रोषेषु शोकावेशकृतेषु च~।\\
यानि वाक्यानि युज्यन्ते पुनरुक्तं न तेष्विह~॥~१११

साध्वहो मां च हेहेति किं त्वं मामावदेति च~।\\
एवंविधानि कार्याणि द्वित्रिसंख्यानि कारयेत्~॥~११२

प्रत्यंगहीनं यत्काव्यं विकृतं च प्रयुज्यते~।\\
न लक्षणकृतस्तत्र कार्यस्त्वभिनयो बुधैः~॥~११३

भावो यत्रोत्तमानां तु न तं मध्येषु योजयेत्~।\\
यो भावश्चैव मध्यानां न तं नीचेषु योजयेत्~॥~११४}
\end{quote}

\hrule

\vspace{2mm}
\noindent
संबन्धादित्यर्थः~। शिरसो भङ्गो ग्रीवासन्धिविच्युतिस्ततः~। पुनरुक्तं न तेष्विति दोषायेति शेषः~। तदुदाहरति साध्वहो इत्यादि~। तत्र शब्दपुनरुक्तं साधुसाध्वित्यादि~। अथपुनरुक्तमहो साधु भद्रं चेत्यादि~। प्रत्यङ्गहीनमित्यादि~। प्रहसनप्रधानतया यथा प्रत्यङ्गेन केनचित्संस्काराशेनाहीनं कार्यम्~। अत एव विकृतत्वाद्धासप्रधानं तत्राप्यभिनयोऽप्यलाक्षणिको हासायैव यथातथापि~॥

\newpage
% पञ्चविंशोऽध्यायः २८५ 

\begin{quote}
{\na पृथक् पृथग्भावरसैरात्मचेष्टासमुत्थितैः~।\\
ज्येष्ठमध्यमनीचेषु नाट्यं रागं हि गच्छति~॥~११५

एतेऽभिनयविशेषाः कर्तव्याः सत्त्वभावसंयुक्ताः~।\\
अन्ये तु लौकिका ये तु ते सर्वे लोकवत्कार्याः~॥~११६

नानाविधैर्यथा पुष्पैर्मालां \renewcommand{\thefootnote}{1}\footnote{बध्नाति}ग्रथ्नाति माल्यकृत्~।\\
अंगोपांगै रसैर्भावैस्तथा नाट्यं प्रयोजयेत्~॥~११७

या यस्य लीला नियता गतिश्च\\
रंगप्रविष्टस्य निधानयुक्तः~।\\
तामेव कुर्यादविमुक्तसत्त्वो\\
यावन्नरांगात्प्रतिनिर्वृतः स्यात्\renewcommand{\thefootnote}{2}\footnote{प \textendash\  निर्गतोऽसौ}~॥~११८}
\end{quote}

\hrule

\vspace{2mm}
अथ सर्वानुग्राहकं सामान्यलक्षणमाह भावो य उत्तमानामित्यादि~। रागं गच्छतीति~। सर्वस्य रञ्जकं भवतीति यावत्~। सामान्यभिनयशेषत्वं तदुक्तार्थात्तन्मुखेनोपसंहारदिशा चित्राभिनयस्य दर्शयति एतेऽभिनयविशेषा इति~। सत्त्वभावसंयुक्ता इत्यनेन सत्त्वातिरिक्तोऽभिनय इत्यादि स्मरति~। नानाविधेरित्यभिनयानां समानीकरणं चित्रत्वं च दर्शितम्~। त्क्रमं श्रमवशादुपचितत्वपरित्यागः प्रयत्नेन परिरक्ष्य इत्येतत्तात्पर्येण श्लोकं पठति या यस्य लीला नियता गति श्चेति~। विमुक्तसत्त्वो त्यक्तावष्टंभं प्रति निर्वत इति निर्वत्या निवृत्तिर्वक्ष्यते इति~। वृत्तो विषयाज्जनादेर्निवर्तते~। तेन

\newpage
% २८६ नाट्यशास्त्रम्

\begin{quote}
{\na एवमेते मया प्रोक्ता \renewcommand{\thefootnote}{1}\footnote{प \textendash\  भावो नाट्यसमाश्रयाः~। नोक्ता ये तु मया तेऽपि}नाट्ये चाभिनयाः क्रमात्~।\\
अन्ये तु लौकिका ये ते लोकाद्गाह्याः सदा बुधैः\renewcommand{\thefootnote}{2}\footnote{प \textendash\  प्रयोगतः}~॥~११९

लोको वेदस्तथाध्यात्मं प्रमाणं त्रिविधं स्मृतम्~।\\
वेदाध्यात्मपदार्थेषु प्रायो नाट्यं प्रतिष्ठितम्~॥~१२०

वेदाध्यात्मोपपन्नं तु\renewcommand{\thefootnote}{3}\footnote{प \textendash\  उपनिषदा} शब्दच्छन्दस्समन्वितम्~।\\
लोकसिद्धं भवेत्सिद्धं नाट्यं लोकात्मकं तथा~॥~१२१}
\end{quote}

\hrule

\vspace{2mm}
\noindent
रङ्गाद्यावन्निर्वतो निष्क्रान्तः स्यादित्यर्थः~। किमेतावानभिनयप्रकारः नेत्याह अन्ये तु लौकिका इति~। ननु किमत्र लोकः प्रमाणमित्याशंक्याह लोको वेदस्तथाध्यात्ममिति~। लोकसिद्धानि प्रत्यक्षानुमानागमप्रमाणानि लोकशब्देनोच्यन्ते~। वेद इति तु यथास्वं नियतरूपो लोकप्रसिद्धोऽध्यागमो यथा न्यायेषु धनुर्वेदः स्वरतालादौ गान्धर्ववे द इत्यादि~। अध्यात्मं तु संस्थं वेदाध्यात्माभ्यां प्रमिता ये पदार्थाः तेषु नाट्यं प्रतीमित्यत्र हेतुमाह वेदाध्यात्मोपपन्नं त्विति~। तुर्हेतो~। समन्वितमिति भावे~। एतदुक्तं शब्दसमन्वयो व्याकरणाभिधानेनागमेन सिद्धः~। छन्दस्समन्वयस्तु स्वसंवेदनेन~। श्रव्यता हि तद्विदा ससंवित्सिद्धावृत्तेषु प्रगीतानामिव रागभाषादीन् नीयते~। एतच्चागमस्ववेदनयोः प्रमोपलक्षणमातत्रं~। अथ लोकं प्रमाणयितुमाह यल्लोकसिद्धमिति~। यल्लोकेसिद्धं तत्सिद्धं न~। तत्कस्यचिदसिद्धभिति यावत्~। नहि लोकप्रसिद्धिमपह्नोति कश्चित्समर्थः~। सुविप्रतिपन्नस्यापि तदपह्नवे काष्ठपाषाणतापत्तिप्रसंगात्~। तथेति~। तत एव व प्रकाराद्धेर्तोर्लोकात्मकं लोकानुकीर्तनरूपं नाट्यमित्युक्तम्~।

\newpage
% पञ्चविंशोऽध्यायः २८७ 

\begin{quote}
{\na \renewcommand{\thefootnote}{1}\footnote{प \textendash\  देवतानामृषीणां च राज्ञां जनपदम्य च~। पूर्ववृत्तानुचरितं नाट्यमित्यभिधीयते~। एवं लोकस्य या वार्ता नानावस्रान्तरात्मिका~। सा नाट्ये संविधातव्या नाट्यवेदविचक्षणैः~। यानि शास्त्राणि ये धर्मा याति शिल्पानि या क्रिया~। लोकधर्मप्रवृत्तानि तन्नाट्यमिति संज्ञितम्~।}न च शक्यं हि लोकस्य स्थावरस्य चरस्य च~।\\
शास्त्रेण \renewcommand{\thefootnote}{2}\footnote{ज \textendash\  नियमं}निर्णयं कर्तुं \renewcommand{\thefootnote}{3}\footnote{ज \textendash\  नांना}भावचेष्टाविधिं प्रति~॥~१२२

नानाशीलाः प्रकृतयः \renewcommand{\thefootnote}{4}\footnote{ज \textendash\  तासु}शीले नाट्यं प्रतिष्ठितम्~।\\
तस्माल्लोकप्रमाणं हि विज्ञेयं नाट्ययोक्तृभिः\renewcommand{\thefootnote}{5}\footnote{ज \textendash\  अभ्यन्तरं च बाह्यं च द्विविधं नाट्यमिष्यते}~॥~१२३

एतान् विधीं श्चाभिनयस्य सम्य\textendash \\
ग्विज्ञाय रंगे मनुजः प्रयुंक्ते~।\\
स नाट्यतत्त्वाभिनयप्रयोक्ता\\
संमानमग्र्यं लभते हि लोके~॥~१२४}
\end{quote}

\hrule

\vspace{2mm}
ननु लोकेन च यत्प्रत्ययं तदागमेनैव प्रमितम्~। तत्किं पुनर्लोकेनोक्तेनेत्याशङ्क्याह न शक्यो लोकस्येति~। शीलः स्वभावः~। प्रकृतमुपसंहरति तस्माल्लोकप्रमाणं हि विज्ञेयमिति~। एतान् विधीनिति~। सामान्याभिनयात् प्रभृत्येतदध्यायपर्यन्तं ये कर्तव्यतारूपाभिनयानां विधय उक्ताः तान् सम्यग् विज्ञायेति वदन् कोहलादिशास्त्रलक्ष्यप्रवाहसिद्धमपि चित्राभिनयं सूचयति~। ततश्चोदाहरणार्थान् दर्शयामो माभूत्सम्प्रदायप्रवाहविच्छेद इति~।

\begin{quote}
{\qt मुख्याभ्याशे हंसपक्षात्स्कन्दो वा शक्तिदर्शनात्~।\\
संमुखौ खटकौ पाश्र्वद्वये शार्ङ्गिनिरूपणम्~॥}
\end{quote}

\newpage
% २८८ नाट्यशास्त्रम्

\begin{quote}
{\qt लीलालोकितसंदंशयुग्मेन कुसुमायुधम्~।\\
रुद्रवद्रुपयेद्दुर्गां चतुरेण सरस्वतीम्~॥

\ldots \ldots \ldots \ldots \ldots \ldots \ldots खटकेन तथा श्रियम्~।\\
गौरी च दष्ट्रया देवीं वाराहीमिति मातरः~॥

प्रदर्श्यात्तत्तदुचितब्राहभ्यादिगतलक्षणैः~।\\
सुचीहस्ताङ्गुलिकाद्रिनन्दिनीस्यान्नयोन्तता~॥

तत्प्रोत्तानाधोमुखेन त्रिपताकामुखेन तु~।\\
गङ्गा तथैव चतुरेणान्या सर्पशिरोद्वयम्~॥

उपर्युपरि डोलं स \ldots ले तिर्यग्विलोलितः~।\\
अरालत्रिपताकौ च पताकद्वयकम्पनम्~॥

अब्धि\ldots पुष्पपुटास्त्रिपताकौ तपस्विनाम्~।\\
प्रसृतोध्वेपराचीनौ शिखरौ बाह्यदन्तरे~॥

कूर्परोर्ध्वस्थितेनापि त्रिपताकेन योषितः~।\\
पार्वऽर्धकटकेन स्याल्लोकपालास्मलक्ष्मभिः~॥

स्तब्धकार्यौ मुक्तहस्तौ जिनं विद्याधरान् प्रजाः~।\\
अग्निता वाथ रक्षांसि नष्टं या सूचिकामुखात्~॥

विद्यात्तु त्रिपताकाभ्याम्मूर्ध्निराजप्लवङ्गमान्~।\\
पताकाभ्यामथो सर्पशिरोभ्यां स्वस्तिकस्थितेः~॥

घर्मं सितादिभिश्चाहीनृतुवो मणयः पुनः~।\\
\ldots काङ्गुलेन रिपुर्न्नाथ ग्रन्थतर्जनिकेऽङ्गना~॥

कूर्पगकुञ्चितां कम्प्रपताकाभ्यां च सारसः~।\\
प्रसारितं च बाहुभ्यां वृश्चिकस्थलपक्षिषु~॥

शृङ्गयां च मध्यमाङ्गुष्ठपताकामस्तकोपरि~।\\
उत्क्षेपादश्चितस्यांध्रिं नतोन्नतकरदूयात्~॥}
\end{quote}

\newpage
% पञ्चविंशोऽध्यायः २८९ 

\begin{quote}
{\na \renewcommand{\thefootnote}{1}\footnote{ज \textendash\  चत्वारो ह्यभिनया}एवमेते ह्यभिनया वाङ्नेपथ्यांगसंभवाः\renewcommand{\thefootnote}{2}\footnote{ज \textendash\  सत्त्वाः प \textendash\  त्रयोविंशः न \textendash\  षङ्विशः}~।\\
प्रयोगज्ञेन कर्तव्या नाटके सिद्धिमिच्छता~॥~१२५}
\end{quote}

\begin{center}
\textbf{इति भारतीये नाट्यशास्त्रे चित्राभिनयो नाम 3पञ्चविंशोऽध्यायः~॥}
\end{center}

\hrule

\begin{quote}
{\qt गरुडं चतुराभ्यां तु कण्डमूलोभयादधः~।\\
पताककूर्परे कुञ्च्य चालीढो क्रोधरूपणे~॥

मुखान्तिके तर्जनीं तु विश्लिष्टां वाक्यरूपणे~।\\
चूडायां मूर्ध्न्युपाङ्गेषु स्त्रीविषादे तथोद्वहम्~॥

वक्षः पार्श्वान्नूर्ध्वतः खं पताकस्वस्तिकेन तु~।\\
तथा प्रभातहस्ताभ्यामावेगोद्वर्तिताङ्गुलिः~॥

पराङ्मुखाभ्यां रात्रिर्वा पताकस्वस्तिकादिभिः~।\\
मुखाच्छादात्खलत्यादौ शलभाधूलिधूम्रकः~॥

पताकेनोरसि सुहृदरालेन सुतादयः~।\\
अभिमूर्ध्राथ तद्युग्मं कुब्जवामनबालकाः~॥

पताका मूर्ध्नि खण्डः स्यादथ स्वस्तिकविच्युते~।\\
निधिमाकुञ्चिते वामकूर्परे भूधरादिषु~॥

उत्तानं च शिरस्तेषां भेदास्तु न निरीक्षणात्~।\\
वामकस्त्रिपताकः स्यात् कण्ठमूलोऽपरोऽपरः~॥

करिणीं गण्डविचलञ्चतुराभ्यां मदश्रिताम्~।\\
पद्मोर्णनाभमुकुलैः स्वस्तिकैर्वृश्चिकेन तु~॥}
\end{quote}

\lfoot{37}

\newpage
\lfoot{}
% २९० नाट्यशास्त्रम्

\begin{quote}
{\qt सिंहगोपायुशरभा यथास्वं दृष्टिभेदतः~।\\
ललाटे सर्पशिरसा खङ्गिं श्रवणमूलतः~॥

खङ्गिकास्त्री तथान्यच्च शिखरं स्यात्प्रसारितम्~।\\
मुष्टिर्मल्लस्य शल्यश्च खदकेन हृदन्तरे~॥

संदंशेन मतिर्नाभेरुद्यता वक्षसि स्थितिः~।\\
पल्लवेन स्वमूर्धानं स्पृशता स्वेचरानतिः~॥

पराङ्मुखपताकाभ्यां मुखे स्वस्तिकविच्युते~।\\
कवाटाभ्यां करिघटान्मीक्षं (?) च करयुग्मतः~॥

मृगशीर्षे कनिष्ठायां निर्देशश्चतुरेण वा~।\\
चलः सूच्यास्वयुगलं हयसैन्येदृशान्वितम्~॥

शूलपाण्यादिशब्देषु केचिद्वर्तिपदाश्रितम्~।\\
कुब्जन्त्वभिनयन्त्यन्ये विशेष्येऽन्यद्वयाश्रितम्~॥

अव्याहतायां वाक्यार्थप्रतीतौ स्यात्पदेष्वथ~।\\
स्नानं मूर्ध्नि पताकाभ्यां शकटेऽन्योन्यसंमुखो~॥

कूर्पराकुञ्चितौ कार्यो त्रिपताककरेण तु~।\\
अयस्कारादिनिर्देशे खटकः करिविद्रवे~॥

गणेशे मुकुलास्योऽथ विक्षेपात्स्युर्मरीचयः~।\\
शिरसः पार्श्वयोः\textendash\ त्रिपताकद्वये जटा~॥

उपर्यृपरि युक्ताभ्यां शिखराभ्यां महेश्वरी~।\\
ब्रीह्मादिचतुरेण स्यान्मुष्टिना वाथलेखकाः~॥

खटकेन तथामूकाश्शून्योत्तानोपनो (?) क्रमात्~।\\
अधस्स्वलितयोगेन चतुरेणोपमीलनात्~॥}
\end{quote}

\newpage
% पञ्चविंशोऽध्यायः २९१ 

\begin{quote}
{\qt शिखरे वामकेऽधस्तः पताकस्थेन दक्षिणे~।\\
सङ्गतौ खटकौ सूर्ये सारथौ पृष्ठपूर्वगौ~॥

प्रमाणो मानपरिमाः सूची संदृष्ट्यरालकैः~।\\
एलाक्रीडा धनुर्योगात् पुलिन्दाभिनयो मतः~॥

यामाद्यौ खटकारालो समपादः कपालिनि~।\\
स्वबाहूर्ध्वे तु खटकौ पार्श्वक्षेश्र्च पादगः~॥

महाभैरवनाथस्य खटकावंसजानुगौ~।\\
अनूर्ध्वकर्मणः पादः कर्मान्तं यदुदीरितम्~॥

तस्य स्वबुध्द्या घटनं चित्राभिनयनं विदुः~।\\
तस्योदाहरणं किञ्चिदिदमूहाबिवृद्धये~॥

मयाभिनवगुप्तेन दर्शितं धीमतः प्रति~।\\
यथालिखितवस्तूनां प्रतिपत्संस्थितान् प्रति~॥

अपि वाचस्पतेर्वाणी कुण्ठा किमुत मादृशाम्~॥}
\end{quote}

एवं प्रमाणत्रयेणाभिनयान् विज्ञाय यो रङ्गे सभायां प्रयुङ्क्ते स एव च नाट्ये तत्त्वतोऽभिनयान् प्रयुङ्क्ते स च संपानं लभत इति योज्यम्~। अभिनयशेषभूतोऽयमितिर्कर्तव्यतारूपः परस्परसंमीलनात्मा प्रयोगस्तेन च विना न काचित् सिद्धिरित्युपसंहारव्याजेनैवमिति श्लोकेनांगीशब्दस्वीकृतसात्तिवकेन दर्शयन् सामान्याभिनयनायकान् सर्वान् दर्शयति सिद्धमिति शिवम्~॥

\begin{quote}
{\qt विचित्राभिनयाध्यायः सोऽयं व्याकृतसारकः~।\\
कृतोऽभिनवगुप्तेन शिवानुग्रहशालिना~॥}
\end{quote}

\begin{center}
इति श्रीमहामाहेश्वराचार्यामिनवगुप्तविरचितायां नाट्यवेदवृत्तावभिनवभारत्यां चित्नाभिनयः पञ्चविंशोऽध्यायः समाप्तः~॥\\

\vspace{2mm}
\rule{0.2\linewidth}{0.5pt}
\end{center}

\newpage
\thispagestyle{empty}
\begin{center}
\textbf{\large श्रीः}\\

\vspace{2mm}
\textbf{\huge नाट्यशास्त्रम्}\\

\vspace{2mm}
षङ्विंशोऽध्यायः

\vspace{2mm}
\rule{0.2\linewidth}{0.5pt}
\end{center}

\begin{quote}
{\na अनुरूपा विरूपा च तथा रूपानुरूपिणी~।\\
त्रिप्रकारेह पात्राणां प्रकृतिश्च विभाविता~॥~१

नानावस्थाक्रियोपेता भूमिका प्रकृतिस्तथा~।\\
भृशमुद्योतयेन्नाट्यं स्वभावकरणाश्रयम्~॥~२}
\end{quote}

\hrule

\begin{center}
॥~अभिनवभारती\textendash\ षड्विंशोऽध्यायः~॥
\end{center}

\begin{quote}
{\qt यस्मिन् सति प्रकृतिभूमिविकल्प एष\\
स्त्रेधास्य याति हृदयादरणीयभावः~।\\
रागः स यस्य महिमा महनीयधाम्नि\\
भूयात्स नित्यमपि तत्र च रागवन्तः~॥}
\end{quote}

समानीकरणलक्षणः सामान्येऽभिनयः प्रस्तुतः~। तत्र यथाभिनयानामन्योन्यं समानीकरणयुपदेश्य तथाभिनेतुरभिनेयस्य च~। एवं सोऽभिनेयद्वारेणाभिनयोऽभिनेत्ना समानीकृतो भवति~। तदेतदिति~। अभिनेत्रभिनेययोः समानीकरणं सामान्याभिनयरूपमनेन प्रकृत्यध्यायोऽभिधीयते~। तदर्थसूचनायैव सङ्गतिप्रदर्शनाभिप्रायो वृत्ताध्यायपरिसमाप्तौ यः प्रयुङ्क्त इति प्रयोक्तेति प्रयोगज्ञेति च निरूपितम्~। स एव हि प्रयोगं जानाति यः प्रयोज्य प्रयोजकस्वरूपवित्~। तत्र कृतिः प्रयोज्यानुकरणीयो अनुकीर्तनीय इति

\newpage
\fancyhead[CO]{षड्विंशोऽध्यायः}
% षड्विंशोऽध्यायः २९३ 

\begin{quote}
{\na बहुबाहूबहुमुखास्तथा च विकृताननाः~।\\
पशुश्वापदवक्त्राश्च खरोष्ट्राश्वः गजाननाः~॥~३

एते चान्ये च बहवो नानारूपा भवन्ति ये~।\\
आचार्येण तु ते कार्या मृत्काष्ठजतुचर्मभिः~॥~४ 

स्वाभाविकेन रूपेण प्रविशेद्रंगमण्डलम्~।\\
आत्मरूपमवच्छाद्य वर्णकैर्भूषणैरपि~॥~५

यादृशं यस्य यद्रूपं प्रकृत्या तत्र तादृशम्~।\\
वयोवेषानुरूपेण प्रयोज्यं नाट्यकर्मणि~॥~६ 

यथा जीवत्स्वभावं हि परित्यज्यान्यदेहिकम्~।\\
परभावं प्रकुरुते परभावं समाश्रितः~॥~७ 

एवं बुधः परंभावं सोऽस्मीति मनसा स्मरन्~।\\
येषां वागंगलीलाभिश्चेष्टाभिस्तु समाचरेत्~॥~८}
\end{quote}

\hrule

\vspace{2mm}
\noindent
पर्यायाः~। तत्र प्रयोक्ता प्रयोज्यसदृशो वा भवति विसदृशो वा उभयात्मको वा~। तत्व सदृशोऽनुरूपः उभयात्मरूपानुरूपः स्वलक्षणं तेना तदेतौ तवैव (?) सादृश्यं लक्षितम्~। आनुरूप्यं सादृश्यात्~। तदुभययोगाद्रूपानुरूपा यद्यपि कथंचित् सर्वत्रैव त्रैविध्यं संभवति~। तथाप्युद्रिक्ता सकललोकसंवादिनी अस्मादृशबुद्धिर्यथा पुरुषस्य प्रयोक्तुः पुरुषेण प्रयोज्येन योषितो योषिता तत्र सदृशव्यवहारः~। स्त्रिया पुरुषस्य तु कैसादृश्यं~। सा सिंहवदनदशवदनादिर्यस्तु प्रयोज्यैरन्यसादृश्यमेव~। तदपि प्रकृतित्रैविध्यं दर्शयिवुमाह अनुरूपेत्यादि~। पात्नाणामिति~। धीयते रसो यत इत्यनेन नटबुद्धितिरोधानं सूचयन्नटबुद्धे\textendash

\newpage
% २९४ नाट्यशास्त्रम् 

\begin{quote}
{\na सुकुमारप्रयोगो यो राज्ञामामोदसंभवः~।\\
शृंगाररसमासाद्य तन्नारीषु प्रयोजयेत्~॥~९ 

युद्धोद्धताविद्धकृता संरंभारभटाश्च ये~।\\
न ते स्त्रीभिः प्रयोक्तव्याः योक्तव्याः पुरुषेषु ते~॥~१०

अनुद्भटमसंभ्रान्तमनाविद्धांगचेष्टितम्~।\\
लयतालकलापातप्रमाणनियताक्षरम्~॥~११

सुविभक्तपदालापमनिष्टुरमुकाहलम्~।\\
ईदृशं यद्भवेन्नाट्यं नारीभिश्च प्रयोजयेत्~॥~१२

एवं कार्यं प्रयोगज्ञैर्भूमिकाविनिवेशनम्~।\\
स्त्रियो हि स्त्रीगतो भावः पौरुषः पुरुषस्य च~॥~१३

यथा वयो यथावस्थमनुरूपेति सा स्मृता~।\\
पुरुषः स्त्रीकृतं भावं रूपात्प्रकृर्ते तु यः~॥~१४}
\end{quote}

\hrule

\vspace{2mm}
\noindent
रप्यपायतामाह~। प्रकर्षेण क्रियते साक्षात्कारकल्पनानुव्यवसायगोचरत्वमीयत इति प्रकृतिः,~। सेयं त्रिप्रकारप्रकृतिप्रविभागेन भाविना सती नाट्यं भृशं द्योतयेदिति ज्ञानस्य प्रयोजनमुक्तम्~। कुतः सामान्यं द्योतयेदिति विशेषणद्वारेण हेतुमाह स्वभावेति~। प्रयोज्यस्वभावो हि नाट्यकर्तव्यः~। तत्र सुकुमारस्वभावे पुंसि प्रयोज्ये लालित्यसौकुमार्ये स्त्रीजनस्यायत्नसिद्धे इति स एव तत्र योक्तः युक्तो रूपानुरूपाणि वा सा तस्य प्रकृतिः , उद्धते तु प्रयोज्ये पुमानेव प्रयोक्ता युक्तः सानुरूपा प्रकृतिरूपायां तु प्रकृतौ कथंचित् यत्नेन संपाद्यमनुकीर्तनमिति तत्र सावधानेन प्रतियोक्ता भाव्यम्~।

\newpage 
% षड्विंशोऽध्यायः २९५ 

\begin{quote}
{\na रूपानुरूपा सा ज्ञेया प्रयोगे प्रकृतिर्बुधः~।\\
छन्दतः पौरुषीं भूमिं स्त्री कुर्यादनुरूपतः~॥~१५

न परस्परचेष्टासु कार्यौ स्थविरबालिशौ~।\\
पाठयप्रयोगे पुरुषाः प्रयोक्तव्या हि संस्कृते~॥~१६

स्त्रीणां स्वभावमधुराः कण्ठाः पुंसां तु बलवन्तः~।\\
यद्यपि पुरुषो विद्यात् गीतविधानं च लक्षणोपेतम्~॥~१७

माधुर्यगुणविहीनं शोभां जनयेन्न तद्गीतम्~।\\
यत्र स्त्रीणां पाठ्याद्गुणैर्नराणां च कण्ठमाधुर्यम्~॥~१८

प्रकृतिविपर्ययजनितौ विज्ञेयौ तावलंकारौ~।\\
प्रायेण देवपार्थिवसेनापतिमुख्यपुरुषभवनेषु~॥~१९}
\end{quote}

\hrule

\vspace{2mm}
नन्वनुरूपैव प्रकृतिर्युक्तेत्याशङ्क्याह नानेति (२ श्लोक)~। नानाप्रकाराभिरवस्थाभिः विलासलालित्यौद्धत्यादिभिर्धर्म्यैः क्रियाभिश्च सुकुमारोद्धतात्मिकाभिरुद्यानगमनयुद्धसन्नाहनादिभिरुपेता तस्मान्नानारूपैव प्रकृतिर्युक्ता~। तस्याः प्रकृतेः पर्यायेण स्वरूपं स्पष्टयति भूमिकेति~। भूमिखष्टं स्थानं~। यथा च ध्यानपटगते दशभुजपश्चवक्त्वादिरूप एव भगवति सदाशिवे धिषणानिवेशः क्रियते न तु तत्र तद्देशत्वतत्कालत्वे आदर्तव्यम्~। नापि तत्सिन्दूरहरितालादिकृतं तद्द्रव्यं केवलमवष्टंभस्थानम्~। तदेवं रामादयोऽवष्टंभस्थानमात्रम्~। एतश्च रसाध्यायादौ वितत्य निरूपितम्~। तेन भूमिरिव भूमिका~। इवार्थे अण्~। अरूपायाः प्रकृतेरसंभावव्यत्वात्~। तत्संपाद्यत्वाच्च पूर्वे स्वरूपमाह बहुबाहु इत्यादि~। विकृताधाराः तेष्वप्रावरणादयः पशुवक्ता यथा

\newpage
% २९६ नाट्यशास्त्रम् 

\begin{quote}
{\na स्त्रीजनकृताः प्रयोगा भवन्ति पुरुषस्वभावेन~।\\
रम्भोर्वशीप्रशृतिषु स्वर्गे नाट्यं प्रतिष्ठितम्~॥~२०

तथैव मानुषे लोके राज्ञामन्तःपुरेष्विह~।\\
उपदेष्टव्यमाचार्यैः प्रयत्नेनांगनाजने~॥~२१

न स्वयं भूमिकाभ्यासो बुधैः कार्यस्तु नाटके~।\\
स्त्रीषु योज्यः प्रयत्नेन प्रयोगः पुरुषाश्रयः~॥~२२

यस्मात्स्वभावोपगतो विलासः स्त्रीषु विद्यते~।\\
तत्मात्स्वभावमधुरमंगं सुलभसौष्ठवम्~॥~२३

ललितं सौष्ठवं यच्च सोऽलंकारः परो मतः~।\\
प्रयोगो द्विविधश्चैव विज्ञेयो नाटकाश्रयः~॥~२४}
\end{quote}

\hrule

\vspace{2mm}
\noindent
गोमुखाः अश्वमुखाः श्वापदवक्रः यथा सिंहवक्राः खरोष्ट्रेत्यादिना सर्ववपुषा तद्द्रूपान् कार्य इति~। आवश्यकेनेति शेषः~। अत्न हेतुमाह स्वाभाविकेनेति~। अनुकीर्तनीयस्य यः स्वभावः तदुचितेन रूपेणेति यावत्~। आत्परूपमिति~। नटरूपमपिशब्दात्~। मृत्काष्टादिनिर्मितबाहुवक्त्रादिरपि~। अवष्टंभयोगस्य प्राधान्यं दर्शयितुमेकविंशत्यध्यायोक्तं हेतुं स्मारयति यथा जीवस्वभावमिति~। षरं भावं रामादिकं वेषादिभिः समाचरेदिति संबन्धः~। सोऽस्मीत्यनेन स्वात्मावष्टंभस्यात्याज्यतामाह~। अन्यथा लयाद्यनुसरणमशक्यम्~। अथ रूपानुरूपिणी प्रकृतिकेत्याशंक्याह सुकुमारप्रयोग इति राज्ञामित्युपलक्षणम् आमोदो विभावपरिपूर्णता~। तन्नारीष्विति~। प्रयोक्रीषु प्रयोज्यतयात्र विषयत्वेन विवक्षितः~। प्रयोजयेदिति~। नाट्याचार्यः~।

\newpage
% षड्विंशोऽध्यायः २९७ 

\begin{quote}
{\na सुकुमारस्तथाविद्धो नानाभावरसाश्रयः~।\\
नाटकं सप्रकरणं भाणो वीथ्यङ्क एव च~॥~२५

ज्ञेयानि सुकुमाराणि मानुषैराश्रितानि तु~।\\
सुकुमारप्रयोगोऽयं राज्ञामामोदकारकः~॥~२६

श्चंगाररसमासाद्य स्त्रीणां तत्तु प्रयोजयेत्~।\\
युद्धोद्धताविद्धकृता संरंभारभटाश्च ये~॥~२७

न ते स्त्रीणां प्रकर्तव्याः कर्तव्याः पुरुषैहि ते~।\\
यथाविद्धांगहारं तु भेद्यभेद्याहवात्मकम्~॥~२८}
\end{quote}

\hrule

\vspace{2mm}
नन्वेवमनुरूषा पुरुषविषये प्रकृतिः किं नास्तीत्याशंक्याह युद्धोद्वतोविद्धकृताविति~। प्रयोगा इति शेषः~। युद्धोद्धतैराविद्धैश्च कृताः व्याप्ताः अत एव संरम्भप्रधाना आरभटादयः~। स्त्रीपुरुषभूमिकेत्यस्यार्थस्य व्यापकत्वमाह अनुद्भटमिति~। अनुरूपां प्रकृतिं लक्षयितुमाह स्त्रिया हीति~। प्रयोक्रयानुरूपाननुरूपां लक्षयति पुरुषस्त्रीगतपिति~। यत्र पुरुषत्वमालम्ब्य स्त्रीवर्तते यथा सांकृत्यायिनी~। न तु सर्वत्रेत्यर्थः~। स्त्रीपुरुषं प्रयुक्तमित्यतेत् सर्वत्र युक्तमिति दर्शयति छन्दत इति~। स्थविरबालिशाविति~। संबन्धिशब्दाः संबन्ध्यन्तरमाक्षिपन्तीति स्थविरो युवभुमिकायां युवा च वृद्धभूमिकायां न योज्यः~। बालिशोऽत्र विरूपः स विरूपभूमावायोज्यः~। एतच्चोपलक्षणम्~। यत्र यत्प्रयोजनो न श्लिष्यति न स तत्र योज्य इत्यर्थः~। पाठ्यप्रयोग इति~। संस्कृतपाट्य\textendash\ प्रधाने सुकुमारप्राकृतपाट्यप्रधाने~। गीत इति गीतप्रधाने प्रयोग इत्यर्थः~। अत्र हेतुमाह प्रायः प्रकृतिरिति~। बलवन्त इति~। रङ्गपूरणोचितगम्भीरस्वरा~। ननु पुमांसोऽपि भावयन्त्येव तत्कथयुक्तं स्त्रीणां गेयं प्रकृतिरित्याशंक्याह यद्यपि पुरुष इति~।

\newpage
% २९८ नाट्यशास्त्रम् 

\begin{quote}
{\na मायेन्द्रजालबहुलं पुस्तनैपथ्यदीपितम्~।\\
पुरुषप्रायसंचारमल्पस्त्रीकमथोद्धतम्~॥~२९

सात्त्वत्यारभटीयुक्तं नाट्यमाविद्धसंज्ञितम्~।\\
डिमः समवकारश्च व्यायोगेहामृगौ तथा~॥~३०

एतान्याविद्धसंज्ञानि विज्ञेयानि प्रयोक्तृभिः~।\\
एषां प्रयोगः कर्तव्यो देवदानवराक्षसैः~॥~३१

उद्धता ये च पुरुषाः शौर्यवीर्यसमान्विताः~।\\
योग्यः स च प्रयत्नः कर्तव्यः सततमप्रमादेन~॥~३२ }
\end{quote}

\hrule

\vspace{2mm}
ननु विपर्ययोऽपि दृष्ट इत्याशंक्याह तावलङ्कारमविति~। कदाचित्कापीति यावत्~। तत्र स्त्रीपुरुषप्रयोगमनुकरोतीत्ययमेव प्रचुरः प्रकार इति दर्शयति~। प्रायेणेति~। पुरुषस्वभावेन प्रयोज्येनोपलक्षितोऽयं स्त्रीभिः कृताः प्रयोज्याः~। अत्रानुवादं दर्शयति रंभोर्वशीग्रप्रभृतिष्विति~। अत्र हेतुमाह उपदेष्टव्यमिति~। उपदेष्टुं शक्यमित्यर्थः~। स्वयमिति पुरुषैः न्यासोऽत्र प्रयोगः~।\\

ननु पुमानपि भावबुद्धिमाश्रित्याशंक्य हेत्वन्तरमाह यस्मादिति~। पुरुषसंबन्धिबलितं च यद्वस्तु तदतीवहृद्यं प्रतिभाति~। तदाहसोऽलंकार इति~। यदुक्तमनुद्भट इत्यादिना युद्धोद्धता इत्यादिना प्रयोगद्वैविध्यं तद्रूपकभेदेन विभजंस्तद्रूपके सप्रयोग उचित इति दर्शयति नाटकमित्यादिना~। शौर्यवीर्यसमन्विता इत्यनेन श्लोकसप्तकेन प्रायः प्रकृतिः स्त्रीणां गेयं नृणां तु पाठ्यविधिरित्युक्तम्~। तत्रायत्नसिद्धेऽर्थे को नाट्याचार्यप्रवर्तितस्य गुणनिकाभ्यासव्यापार इत्याशंक्याह संगीतपरिक्लेश इति~। मधुरत्वं स्वाभाविक 

\newpage
% षड्विंशोऽध्यायः २९३ 

\begin{quote}
{\na न हि योग्यया विना भवति च भावरससौष्ठवं किंचित्\\
संगीतपरिक्लेशो नित्यं प्रमदाजनस्य गुण एव~॥~३३

यन्मधनुरकर्कशत्वं लभते नाट्यप्रयोगेण~।

प्रमदाः नाट्याविलासैर्लभते यत् कुसुमैर्विचित्रलावण्यम्~।\\
कामोपचारकुशला भवति च काम्या विशेषेण~॥~३४

गीतं नृत्तं तथा वाद्यं प्रस्तरगमनक्रिया~।\\
शिष्यनिष्पादनं चैव षडाचार्यगुणाः स्मृताः~॥~३५}
\end{quote}

\hrule

\vspace{2mm}
\noindent
कर्कशत्वं सविघ्नत्वं कलाभ्यासकृतः~। अथ नाट्याभ्यासप्रोत्साहनार्थमाह प्रमदा नाट्यविलासैरिति~।\\

न स्वयं भूमिकान्यासो बुधैः कार्यस्तु नाटके इत्युक्तम्~। तत्र नाट्याचार्यः किं बुद्धयते येन बुधा इति संशये स इत्याह गीतं नृत्तमित्यादि~। स्वरज्ञोंऽग्रहारक्रियापि चतुर्विधातोद्यकुशलस्तालज्ञः लोकोपकारविच्चेत्यर्थः~। प्रस्तारोऽत्र तालः~। गमने क्रियाङ्कस्य कीदृशी गतिरित्यनेनोपचारकौशलं तद्वक्ष्यते~। एतैर्विना नाट्याचार्यनामापि न लभत इत्यर्थः~। ऊहादयस्तत्पृष्ठे भवन्तस्तदुत्कृष्टं कुर्वन्तीति ते गुणा इति विभागेनोक्ता~। ऊहोऽनुक्तस्य कल्पननमपोहो अनुक्तस्य अनुसरणमिति पूर्वोन्मेषरूपा प्रतिभा स्मृतिरुपदिष्टस्याविस्फुरणं मेधाः उपदिष्टस्य झटिति ग्रहणं शिष्यनिष्पादनं शिष्याशयौचित्यान्नोपदेश्यत्वं गुणप्रख्यानोद्यमः प्रगल्भते इत्यर्थः~। राग इति~। प्रयोजनानभिसन्धिना तत्र कलायाश्चासंघर्षाभ्यधिकं प्रतिपच्चिः

\newpage
% ३०० नाट्यशास्त्रम् 

\begin{quote}
{\na एतानि पञ्च यो वेत्ति स आचार्य प्रकीर्तितः~।\\
ऊहापोहौ मतिश्चैव स्मृतिर्मेधा तथैव च~॥~३६

मेधास्मृतिर्गुणश्लाघाराग संघर्ष एव च~।\\
उत्साहश्च षडेवैतान् शिष्यस्यापि गुणान् विदुः~॥~३७

एवं कार्यं प्रयोगज्ञैर्नानाभूमिविकल्पनम्~।\\
अत ऊर्ध्वं प्रवक्ष्यामि सिद्धीनामपि लक्षणम्~॥~३८}
\end{quote}

\begin{center}
\textbf{इति भारतीये नाट्यशास्त्रे विकृतिविकल्पो नाम षड्विंशोऽध्यायः~॥}
\end{center}

\hrule

\noindent
स्पर्धा~। एतदुपसंहरन्नध्यायान्तरमासुत्रयति एवमिति~। अत ऊर्ध्वमिति चेति~। सिद्धेर्दैविध्येऽप्यवान्तरभेदेन बहुत्वमिति सिद्धीनामित्युक्तमिति शिवम्~॥

\begin{quote}
{\qt प्रकृतिविकल्पाध्याये विषमपदालोचनं समारचितम्~।\\
अभिनवगुप्तेन मया विषमविलोचनपदाब्जभृङ्गेण~॥}
\end{quote}

\begin{center}
\textbf{इति श्रीमहामाहेश्वराचार्याभिनवगुप्तेन विरचितायां भारतीयनाट्यवेदवृत्तावभिनवभारत्यां प्रकृतिविकल्पाध्यायः षड्विंशः~॥}\\

\vspace{2mm}
\rule{0.2\linewidth}{0.5pt}
\end{center}

\newpage
\thispagestyle{empty}
\begin{center}
\textbf{\large श्रीः}\\

\vspace{2mm}
\textbf{\huge नाट्यशास्त्रम्}

\vspace{2mm}
सप्तविंशोऽध्यायः

\vspace{2mm}
\rule{0.2\linewidth}{0.5pt}
\end{center}

\begin{quote}
{\na सिद्धीनां तु प्रवक्ष्यामि लक्षणं नाटकाश्रयम्~।\\
यस्मात्प्रयोगः सर्वोऽयं सिद्धयर्थं संप्रदर्शितः\renewcommand{\thefootnote}{1}\footnote{न \textendash\  प्रतिष्ठितम्}~॥~१

सिद्धिस्तु द्विविधा ज्ञेया \renewcommand{\thefootnote}{2}\footnote{ट \textendash\  वाक्सत्त्व च \textendash\  मानुषीदैविकी तथा~। वाढ्यनः\textendash\ क्षयसंभूता नानाभावरसाश्रया}वाङ्मनोंगसमुद्भवा~।\\
\renewcommand{\thefootnote}{3}\footnote{भ \textendash\ पुनश्च}दैवी च मानुषी चैव नानाभावसमुत्थिता\renewcommand{\thefootnote}{4}\footnote{न \textendash\  सममाश्रया भ\textendash\  विविधा नाट्यभाविनी}~॥~२}
\end{quote}

\hrule

\begin{center}
अभिनवभारती \textendash\ सप्तविंशोऽध्यायः
\end{center}

\begin{quote}
{\qt सत्त्वमित्यमलरङ्गमण्डले दैवमानुषविभेदभेदिता~।\\
सिद्धिमानयति यः स्वविद्यया तं नमामि गिरिजार्धधारिणम्~॥}
\end{quote}

इह यो यथाभिनये यस्मिन् योक्तव्यः सिद्धिमिच्छतेति सर्वमभिनयानां तावत्सिद्धिपर्यन्तमुक्तम्~। अभिनयप्रक्रमेणैवोपाङ्गाभिनयाध्याये सप्तमे~। तथाभिनयसमानीकरणात्मकमेलनिका संपादनात्मकस्सामान्याभिनयस्यापि सिद्धिफलत्वमेव दर्शितम्~। चित्नाभिनयाध्यायान्ते {\qt एवमेते ह्यभिनया वाङ्नैपथ्याङ्गसंभवाः~। प्रयोगज्ञेन कर्त्तव्या नाटके सिद्धिमिच्छतेति} श्लोकेन तत एव रसा भावा इत्यत्न सिद्धिरुद्दिष्टा तत्र केयं सिद्धिर्नामेति भवितव्यमधुना जिज्ञासया तदभिप्रायेणानन्तरवृत्तावध्यायपर्यन्ते दर्शितः~॥

\newpage
% ३०२ नाट्यशास्त्रम् 

\begin{quote}
{\na दशांगा मानुषी सिद्धिर्दैवी तु द्विविधा स्मृता~।\\
नानासत्त्वाश्रयकृता वाङनैपथ्यशरीरजा\renewcommand{\thefootnote}{1}\footnote{न \textendash\  शारोरो वाङ्मयी तया}~॥~३

स्मितापहासिनी हासा\renewcommand{\thefootnote}{2}\footnote{ट \textendash\  हासातिहासा} साध्वहो \renewcommand{\thefootnote}{3}\footnote{प \textendash\  हाहतेति च~। भवेत्प्रवृद्धानन्दाया}कष्टमेव च~।\\
प्रबद्धनादा च तथा सिद्विर्ज्ञेयाथ वाङ्मयी~॥~४

पुलकैश्च सरोमाञ्चैरभ्युत्थानैस्तथैव च~।\\
चेलदानांगुलिक्षेपैः शारीरी सिद्धिरिष्यते~॥~५

किञ्चिच्छिष्टो रसो हास्यो नृत्यद्भिर्यत्र युज्यते~।\\
स्मितेन् \renewcommand{\thefootnote}{4}\footnote{ष \textendash\  संपरि}स प्रतिग्राह्यः प्रेक्षकैर्नित्यमेव च~॥~६

किञ्चिदस्पष्टहास्यं यत्तथा वचनमेव च~।\\
अर्धहास्येन तद्ग्राह्यं प्रेक्षकैर्नित्यमेव हि~॥~७}
\end{quote}

अत ऊर्ध्वे प्रवक्ष्यामि सिद्धीनामपि लक्षणम्~॥~इति~। तत्र सिद्धिर्नामासाध्यप्रयोजनसंपत्तिः~। सा च नटानां सामाजिकानां च~। तत्र कतरा वक्तव्येत्याशंक्याह

\begin{quote}
{\qt सिद्धीनां तु प्रवक्ष्यामि लक्षणं नाटकाश्रयमिति~।}
\end{quote}

\noindent
तुर्व्यतिरेके~। यद्यपि सामाजिकाश्रयं नाटकाश्रयं च सिद्धीनां लक्षणं वक्तव्यं तथापि नटाश्रयमेव वक्ष्यामि~। नेतरदिति~। नाटकोऽत्ञ नटः नटतीति अपि हि व्युत्पत्तिः~। कस्मात्पुनरितरं नोच्यत इत्याशङ्क्या\textendash

\newpage
\fancyhead[CO]{सप्तविंशोऽध्यायः}
% सप्तविंशोऽध्यायः ३०३

\begin{quote}
{\na विदूषकोच्छेद\renewcommand{\thefootnote}{1}\footnote{प \textendash\  पदं}कृतं भवेच्छिल्पकृतं च यत्~।\\
अतिहास्येन तदूग्राह्यं प्रेक्षकैर्नित्यमेव तु~॥~८

\renewcommand{\thefootnote}{2}\footnote{प हीकारो नियतं}अहोकारस्तथा कार्यो नृणां प्रकृतिसंभवः~।\\
यद्धर्मपदसंयुक्तं तथातिशयसंभवम्~॥~९

तत्र साध्विति यद्वाक्यं प्रयोक्तव्यं हि साधकैः~।\\
विस्मयाविष्टभावेषु\renewcommand{\thefootnote}{3}\footnote{प \textendash\  यादिषु भावेषु ज \textendash\  येंषु भायेषु शृङ्गाराभ्दुतविक्रमैः~।} प्रहर्षार्थेषु चैव हि~॥~१०

करुणेऽपि प्रयोक्तव्यं कष्टं शास्त्रकृतेन तु~।\\
प्रबद्धनादा च तथा विस्मयार्थेषु नित्यशः~॥~११

\renewcommand{\thefootnote}{4}\footnote{प \textendash\  अविच्छेदेषु}साधिक्षेपेषु वाक्येषु प्रस्पन्दिततनूरुहैः~।\\
कुतूहलोत्तरावेधैर्बदुमानेन साधयेत्~॥~१२}
\end{quote}

\hrule

\begin{quote}
{\qt यस्मात्प्रयोगः सर्वोऽयं सिद्ध्यर्यः संप्रदर्शितः~॥} इति~।
\end{quote}

\noindent
सामाजिकानां सिद्ध्यर्थो यः प्रयोगः स इति विशेषणभागे विश्रान्तिः~। दण्डीप्रेषादस्याह लोहितोष्णीषा: प्रचरन्तीति~। यथा विधिविवक्तेः तेनायमर्थः~। यथा सिर्ध्यी प्रयोगः सप्रयोजनः सामाजिकगतया तदुद्देशेनैव नाट्योत्पत्तौ न वेदव्यवहारोऽयं संश्राव्यः शूद्रजातिष्वित्यादेशेषु प्रयोगदर्शित्वात् सा सिद्धिः पूर्वे दर्शितैव~। तदुपयोगेनान्तरीयकतया क्क परा सिद्धिः~। अनेनाध्यायेन दर्श्यते~। एतदुक्तं भवति~। सामाजिकानां तावदभि\textendash

\newpage
% ३०४ नाट्यशास्त्रम् 

\begin{quote}
{\na दीप्तप्रदेशं यत्कार्यं छेद्यभेद्याहवात्मकम्~।\\
सविद्रवमथोत्फुल्लं \renewcommand{\thefootnote}{1}\footnote{प \textendash\  पूर्णे}तथा युद्धनियुद्धजम्~॥~१३

प्रकम्पितांसशीर्षं च साश्रं सोत्थानमेव च\renewcommand{\thefootnote}{2}\footnote{ज\textendash\  प्रत्यत्थानास्रसंभवः}~।\\
तत्प्रेक्षकैस्तु कुशलस्साध्यमेवं विधानतः\renewcommand{\thefootnote}{3}\footnote{ज \textendash\  चेलस्य चालनात्}~॥~१४

एवं साधयितव्यैषा तज्ज्ञैः सिद्धिस्तु मानुषी~।\\
दैविकीं च पुनः सिद्धिं संप्रवक्ष्यामि तत्त्वतः~॥~१५

या भावातिशयोपेता सत्वयुक्ता तथैव च~।\\
\renewcommand{\thefootnote}{4}\footnote{ज \textendash\  नाट्यसंप्रेक्षकै र्ज्ञेया नित्यं सिद्धिस्तु दैविकी~।}सा प्रेक्षकैस्तु कर्तव्या दैवी सिद्धिः प्रयोगतः~॥~१६

न शब्दो यत्र न क्षोभो न चोत्पातनिदर्शनम्~।\\
संपूर्णता च रंगस्य दैवी सिद्धिस्तु सा स्मृता~॥~१७}
\end{quote}

\hrule

\vspace{2mm}
\noindent
संहितफलाप्तिलक्षणा सिद्धिः~। व्रीह्यात्मिका व्रीह्यादिष्वधिगच्छन्तीत्यादौ तत्र स्थाने निरूपितपूर्वैव~। सा च मानुषेण सामाजिकेनाभिसंहितत्वान्मानुषीत्युच्यते~। तत्रापि तया न संहितोऽसौ यथा निर्विषयस्वकपरमानन्दाविर्भावस्वरूपापत्तिवर्गब्रह्मचारिणी गीतादेर्विषयस्य नाट्यान्तरुपरञ्जकतया निमग्नस्य विषयसमत्वान्नदटादेः निह्नुतत्वाद्रामादेस्तुच्छत्वाद्देशकालनिर्यन्त्रणतया एव विषयत्वापादनात्तस्याश्चात्र असंभवस्योपादानात् सोऽप्ययं सिद्धयंशो दैवशान्त्याः समस्त रसप्रकृतितां\textendash

\newpage
% सप्तविंशोऽध्यायः ३०५

\begin{quote}
{\na \renewcommand{\thefootnote}{1}\footnote{ज \textendash\  एवं सिद्धिस्तु विज्ञेया प्रेक्षकैर्दिव्यमानुषी~। प \textendash\  एवं साधयितव्यैषा तज्ञैः सिद्धिस्तु मानुषी}दैवी च मानुषी चैव सिद्धिरेषा मयोदिता~।\\
अत ऊर्ध्वं प्रवक्ष्यामि घातान्दैवसमुत्थितान्~॥~१८

दैवात्मपरसमुत्था त्रिविधा घाता बुधैस्तु विज्ञेया~।\\
औत्पातिकश्चतुर्थः कदाचिदथ संभवत्येषु~॥~१९

वाताग्निवर्षकुञ्जरभुजङ्ग\renewcommand{\thefootnote}{2}\footnote{प \textendash\  संक्षोभणं}मण्डपनिपाताः~।\\
कीटव्यालपिपीलिकपशुप्रवेशनाश्च दैवककृता~॥~२०

घातानतः परमहं परयुक्तान् संप्रवक्ष्यामि~।\\
वैवर्ण्यं चाचेष्टं विभ्रमितत्वं स्मृतिप्रमोहश्च~॥~२१

अन्यवचनं च काव्यं तथांगदोषो विहस्तत्वम्~।\\
एते त्वात्मसमुत्था घाता ज्ञेया प्रयोगज्ञैः~॥~२२}
\end{quote}

\hrule

\vspace{2mm}
{\qt स्वं स्वं निमित्तमासाद्य शान्तादुत्पद्यते रसः}\textendash\ वदतामुचित एव स्पष्टतयानुरोधो रहस्यार्थस्यान्यपरत्वाच्च शास्त्रस्य~। तदुक्तं भट्टनायकेन\textendash

\begin{quote}
{\qt प्रधाने सिद्धिभागेऽस्य प्रयोगाङ्गत्वमागताः~।\\
गेयादयस्तथैवैते त्रैधैनं (?) ह्युपयोगिनः~॥

सोपानपदपङ्क्तया च सा च मोक्षस्पृशात्मिका~।\\
सा तु नोक्ता गुह्यमृषयोऽन्यपदे कथम्~॥

शास्त्रे प्रकटयेयुर्हि तालमानकृते यथा~॥} इति~॥
\end{quote}

\lfoot{39}

\newpage
\lfoot{}
% ३०६ नाट्यशास्त्रम्

\begin{quote}
{\na मात्सर्याद्द्वेषाद्वा तत्पक्षत्वात्तथार्थभेदत्वात्~।\\
एते तु परसमुत्था ज्ञेया घाता बुधैर्नित्यम्~॥~२३

अतिहसितरुदितविस्फोटितान्यथोत्कृष्टनालिकापाताः \\
गोमयलोष्ट\renewcommand{\thefootnote}{1}\footnote{ष \textendash\  धूली}पिपीलिकविक्षेपाश्चारिसंभूताः\renewcommand{\thefootnote}{2}\footnote{प \textendash\  स्यु परसमुत्थाः}~॥~२४

औत्पातिकाश्च घाता मत्तोन्मत्तप्रवेशलिङ्गकृताः~।\\
पुनरात्मसमुत्था ये घातांस्तांस्तान् प्रवक्ष्यामि~॥~२५

वैलक्षण्यमचेष्टितविभूमिकत्वं स्मृतिप्रमोषश्च~।\\
अन्यवचनं च काव्यं तथार्तनादो विहस्तत्वम्~॥~२६

\renewcommand{\thefootnote}{3}\footnote{ट \textendash\  आरटितरुदितविहसितस्त्रासक्षताङ्गकंपाद्या}अतिहसितरुदितविस्वरपिपीलिकाकीटपशुविरावाश्च~।\\
मुकुटाभरणनिपाता\renewcommand{\thefootnote}{4}\footnote{ट \textendash\  प्रपतन} पुष्कर\renewcommand{\thefootnote}{5}\footnote{ट \textendash\  रवागतित दोषाश्च}जाः काव्यदोषाश्चः~॥~२७}
\end{quote}

\hrule

\vspace{2mm}
यस्तु प्रस्फुटो दैवसिद्ध्यंशः पुरुषार्थव्युत्पत्तिलक्षणः सोऽपि {\qt धर्मो धर्मप्रवृत्तानां कामं कामोपसेविना} मित्यादिना प्रदर्शित एवेति सामाजिकाश्रया सिद्धिर्न वक्तव्या~। लक्षणतस्तदाह \underline{सम्यक् प्रकर्षेण प्रकटित} इति~। नटस्य तु या सम्यक् प्रयोगनिष्पत्षिलक्षणा सिद्धिः सा प्रयोगसिद्धिरुपयोगिनी प्रयोगनिष्पत्या हि विना नाट्यतथैव नेति कुतः सा भवेत्प्रयोगनिष्पत्तिश्च सामान्याभिनयस्यैव सम्यक्तापत्तिः~। परमार्थतस्तु परकीयप्रोत्साहनतारतम्योदितप्रकृतिभानप्रत्ययबलेन वा स्वतः प्रतिभानमाहात्म्येन वा तत्र पूर्वा मनुष्यनिष्पादितत्वान्मानुषीत्युच्यते~। दृश्यतेऽपि प्रोत्साहनबलेना\textendash

\newpage
% सप्तविंशोऽध्यायः ३०७ 

\begin{quote}
{\na अतिहसितरुदितहसितानि सिद्धर्भावस्य दूषकाणि स्युः\renewcommand{\thefootnote}{1}\footnote{ट \textendash\  सिद्धिवादप्रणामकरणानि प \textendash\  दरहसितरुदितयोगैः सिद्धिविभागं प्रणाशमुपयाति}~।\\
कीटपिपीलिकपाता सिद्धिं सर्वात्मना घ्नन्ति~॥~२८

\renewcommand{\thefootnote}{2}\footnote{ट \textendash\  विशसनमपि ज्ञेयं बाधाजननं प्रयोगस्य}विस्वरमजाततालं वर्णस्वरसंपदा च परिहीणम्~।\\
अज्ञातस्थानलयं स्वरगतमेवंविधं हन्यात्~॥~२९

मुकुटाभरणनिपातः प्रबद्धनादश्च नाशनो भवति~।\\
\renewcommand{\thefootnote}{3}\footnote{प \textendash\  विस्वरमरक्तरागं}पशुविशसनं तथा स्याद्बहुवचनघ्नं प्रयोगेषु~॥~३०

विषमं मानविहीनं विमार्जनं चाकुलप्रहारं च~।\\
अविभक्तग्रहमोक्षं पुष्करगतमीदृशं हन्ति~॥~३१

पुनरुक्तो ह्यसमासो विभक्तिभेदो विसन्धयोऽपार्थः~।\\
त्रैलिंगजश्च दोषः प्रत्यक्षपरोक्षसंमोहाः~॥~३२}
\end{quote}

\hrule

\vspace{2mm}
\noindent
प्रबोधो हनूमत एव सागरलंघने~। तत्र प्रोत्साहनं वाचिकं~। पञ्चधा सा त्वहो कष्टमित्येकं स्थानं शारीरं पञ्चषधेति दशधा~। अन्ये तु विभागमाहुः~। तथा हि प्रोत्साहनं वचसा वा सात्त्विकदर्शनेन वा शरीरव्यापारेण वा~। वचनं सप्तधा~। तद्यथा मध्यमारूपतत्प्ररोहालाकं सामान्यवैखर्यात्मकं तत् प्ररोहात्मकं विशेषशब्दात्मकं वैखरीस्वभावम्~। आवेशोचितविशेषवैखरीरूपं तत्प्रबन्धं विच्छेदं च~। तदाह~। स्मितंह्यन्तःसञ्जल्परूपा मध्यमां सूचयति~। संविदो

\newpage
% ३०८ नाट्यशास्त्रम् 

\begin{quote}
{\na छन्दोवृत्तत्यागो गुरुलाघवसङ्करो यतेर्भेदः~।\\
एतानि यथा स्थूलं घातस्थानानि काव्यस्य~॥~३३

ज्ञेयौ तु काव्यजातौ द्वौ घातावप्रतिक्रियौ नित्यम्~।\\
प्रकृतिव्यसनसमुत्थः शेषोदकेनालिकत्वं च~॥~३४

अप्रतिभागं स्खलनं विस्वरमुच्चारणं च काव्यस्य~।\\
अस्थानभूषणत्वं पतनं मुकुटस्य विभ्रंशः~॥~३५

\renewcommand{\thefootnote}{1}\footnote{प \textendash\  रथनाग}वाजिस्यन्दनकुञ्जरखरोष्ट्रशिबिकाविमानयानानाम्~।\\
आरोहणावतरणेष्वनभिज्ञत्वं विहस्तत्वम्~॥~३६

प्रहरणकवचानामप्ययथाग्रहणं विधारणं चापि~।\\
अमुकुटभूषणयोगश्चिरप्रवेशोऽथवा रङ्गे~॥~३७

एभिः स्थानविशेषैर्घाता लक्ष्यास्तु सूरिभिः कुशलैः~।\\
यूपाग्निचयनदर्भस्रग्भाण्डपरिग्रहान्मुक्त्वा~॥~३८}
\end{quote}

\hrule

\vspace{2mm}
\noindent
\underline{हि हासविकासानुपः} सुन्दरस्पन्दो यदाहं वृत्रहणं स्मितेनेति~। सात्त्विकं तु पुलकादिरूपमेकं चैव~। शरीरविकारोऽपि द्विधा~। अनभिसन्धिपूर्वक एव यथा झटिति हयुत्थानः~। अभिसन्धानकृतो वा यथा चेलादिप्रक्षेपं चेलाद्यभावे चोर्ध्वोगुलिकरणादिभिस्तेनेयं दशविधा मानुषी सिद्धिः तत्र तत्र प्रयोगौचित्यात्सभेदेन प्रवर्तते~। यदाह हास्यं स्मितेनेत्यादि~। स्वप्रतिभानतारतम्यकृता तु सिद्धिर्द्विविधा~। कदाचित्तु प्रतिभवन्त्यपि स्वप्नयोगादतीव

\newpage
% सप्तविंशोऽध्यायः ३०९ 

\begin{quote}
{\na सिध्द्या मिश्रो घातस्सर्वगतश्चैकदेशजो वापि~।\\
नाट्यकुशलैःसलेख्यासिद्धिर्वाख्याद्विघातोवा~॥~३९

\renewcommand{\thefootnote}{1}\footnote{प \textendash\  सिद्धिर्वा घातो वा सर्वगतोष्यक्ष लक्षणो बहुशः}नालेख्यो बहुदिनजः सर्वगतोऽव्यक्तलक्षणविशेषः~।\\
यस्त्वैकदिवसजातस्स प्रत्यवरोऽपि\renewcommand{\thefootnote}{2}\footnote{प \textendash\  रोऽहि~। भ \textendash\  सोल्पतरत्वान्न} लेख्यस्स्यात्~॥~४०

जर्जरमोक्षस्यान्ते सिद्धेर्मोक्षस्तु नालिकायास्तु\renewcommand{\thefootnote}{3}\footnote{ज \textendash\  नालिकलेख्याञ्च सिर्द्धिलेख्यं च}~।\\
कर्तव्यस्त्विह सततं नाट्यज्ञैः प्राश्निकैर्विधिना~॥~४१

दैन्ये दीनत्वमायान्ति ते नाट्ये प्रेक्षकाः स्मृताः~।\\
ये तुष्टौ तुष्टिमायान्ति शोके शोकं व्रजन्ति~॥~४२}
\end{quote}

\hrule

\vspace{2mm}
\noindent
मन्दीभवति~। तत्संभाव्यमानमध्यात्मिकाधिदैविकानां शरीरादिगता व्याधिरूपप्रक्षोभबाह्यभूतजनितकलकलशब्दादिभूकम्पवातवर्षादीनां विघ्नानां दैवपरपर्यायादृष्टकृताददृष्टप्रेरितं न पुरुषव्यापारोपनतादपसारणाद्वा भवति~। यत्रेदमाह \underline{न शब्दो यत्र न क्षोभ} इत्यादि~। मूलत एव विघ्नानामसंभवाद्वा यदाशयेनाह \underline{यो भावातिशेयोपेतेति}~।\\

तदेतदाहुर्यन्मुनिराह सिद्धिस्तु द्विविधा ज्ञेयेति~। अत एवास्या रसः प्रत्यङ्गत्वान्नाट्याङ्गमध्ये रसा वा इत्यत्र गुणाङ्गयुक्तं सामाजिकाश्रिता तु फलमाहुः~।\\

यत्तु भट्टनायकेनोक्तं {\qt सिद्धेरपि नटादेरङ्गत्वं व्रजन्त्यास्तत्पक्षेऽयमिति} तेन नाट्याङ्गता समर्थितफलञ्च पुरुषार्थत्वादिति केवलं जैमिनिरनुसृत

\newpage
%३१० नाट्यशास्त्रम् 

\begin{quote}
{\na योऽन्यस्य महे मूर्धो नांदीश्लोकं पठेध्दि देवस्य~।\\
स्ववशेन पूर्वरङ्गे सिद्धेर्घातः प्रयोगस्य\renewcommand{\thefootnote}{1}\footnote{प \textendash\  योऽन्यस्थ कवेः काव्ये काव्यं संभिश्रयेत्तथान्येन~। तस्यापि बलद्रङ्गे तज्ज्ञैर्घातो विलेख्यस्तु योऽन्यस्य कवेर्नाम्ना काव्यं काव्येन मिश्रयेन्मोहात्~। निर्दिष्टदोषतस्तस्मिन् सिद्धा लेख्यो बुधैः क्रमशः}~॥~४३

\renewcommand{\thefootnote}{2}\footnote{भ \textendash\  निर्दिष्टः स बुधै र्ज्ञातः काव्यापहारिणो बन्धः}यो देशभा\renewcommand{\thefootnote}{3}\footnote{भ \textendash\  वेषहीनं प \textendash\  विषयभाव्वेतमपि च प्रयोजयेक्ताव्यम् भ \textendash\  भाषावयवं च योजयेत् काव्ये}वरहितं भाषाकाव्यं प्रयोजयेद्बुद्ध्या~।\\
तस्याप्यभिलेख्यः स्याद्घातो देशः प्रयोगज्ञैः~॥~४४

कः शक्तो नाट्यविधौ यथावदुपपादनं प्रयोगस्य~।\\
\renewcommand{\thefootnote}{4}\footnote{प भ्रष्टो व्यग्रमनो वा}कर्तुं व्यग्रमना वा यथावदुक्तं परिज्ञातम्~॥~४५

\renewcommand{\thefootnote}{5}\footnote{प \textendash\  गम्भीराः शब्दा ये व्याकरणे वेदशास्रसंप्रोक्ताः}तस्माद्गम्भीरार्थाः शब्दा ये लोकवेदसंसिद्धाः~।\\
सर्वजनेन ग्राह्यास्ते योज्या नाटके विधिवत्~॥~४६}
\end{quote}

\hrule

\vspace{2mm}
\noindent
इत्यलमनेन~। \underline{वाङ्मनोंऽगसमुद्भवेति} सर्वाभिनयैकीकारसंपत्तिरपीत्यर्थः मनोरूपत्वात्~।\\

अन्ये तु वाङ्मनोंगसमुद्भवा दशाङ्गा मानुषीति संबन्धयन्ति~। तदन्येऽप्यभिनयविषयैवेति दर्शितं \underline{ननासत्त्वाश्रयकृता} वाङ्मय्यथशरीरजेति~। \underline{विदूषकच्छेदकृत}मिति~। च्छेदोऽत्र वचनभङ्गी प्रकम्पितस्कन्धना\ldots श्रयोच्छाटनं च कृत्वा तत्साध्यं प्रोत्साहने बृंहितव्यमितिं संबन्धः~। अथानेन साभ्युत्थानै\textendash

\newpage
% सप्तविंशोऽध्यायः ३११ 

\begin{quote}
{\na न च किञ्चिद्गुणहीनं दोषैः परिवर्जितं न चाकिंचित्~।\\
तस्मान्नाट्यप्रकृतौ दोषा नाव्यार्थतो ग्राह्याः~॥~४७

\renewcommand{\thefootnote}{1}\footnote{म \textendash\  नानादरस्तु}न च नादरस्तु कार्यो नटेन वागङ्गसत्त्वनेपथ्ये~।\\
रसभावयोश्च गीतेष्वातोद्ये लोकयुक्तयां च~॥~४८

एवमेतत्तु विज्ञेयं सिद्धीनां लक्षणं बुधैः~।\\
अत ऊर्ध्वं प्रवक्ष्यामि \renewcommand{\thefootnote}{2}\footnote{न \textendash\  प्रेक्षकाणां}प्राश्निकानां तु लक्षणम्~॥~४९

चारित्रामि जनोपेताः शान्तवृत्ताः कृतश्रमाः\renewcommand{\thefootnote}{3}\footnote{म \textendash\  श्रुतान्विताः}~।\\
यशोधर्मपराश्चैव मध्यस्थवयसान्विताः~॥~५०

षडङ्गनाट्यकुशलाः प्रबुद्धाः शुचयः समाः~।\\
चतुरातोद्यकुशला \renewcommand{\thefootnote}{4}\footnote{भ \textendash\  नेपथ्यज्ञाः सुधार्मिकाः}वृत्तज्ञास्तत्त्वदर्शिनः~॥~५१ 

देशभाषाविधानज्ञाः कलाशिल्पप्रयोजकाः~।\\
चतुर्थाभिनयोपेता रसभावविकल्पकाः\renewcommand{\thefootnote}{5}\footnote{म \textendash\  सूक्ष्मज्ञा रसभावयोः~।}~॥~५२}
\end{quote}

\hrule

\vspace{2mm}
\noindent
रिति नवमो भेदो व्याख्यातः~। \underline{प्रकम्पितांसशीर्षमित्यनेन} वचनाङ्गुलिक्षेपात् दशमोऽपि भवेदस्पृष्टो मन्तव्यः~। सास्त्रमिति वदन्नेवं सूचयन्ति~।\\

यद्यपि भेदान्तरमप्यत्रानुप्रविष्टं तथापि बाहुल्यादभ्युत्थानेन व्यपदेश इति~। एवं भेदात्क्रियते~। \underline{द्वेषः} सहजैवाप्री\underline{तिर्मात्सर्यं} तु कार्यार्थमेकद्रव्याभिलाषात्~। \underline{पिपीिकानिक्षेपः} सुकुमारप्रकृतेः स्त्रीपात्रप्रायस्य त्रासनोत्पादेन सिद्धिविघातः औत्पातिकाश्च घाताः पशुवेगोन्मत्तलिङ्गकृता इति~। अशंकितं

\newpage
% ३१२ नाट्यशास्त्रम् 

\begin{quote}
{\na शब्दच्छन्दोविधानज्ञा नानाशास्त्रविचक्षणाः~।\\
एवं विधास्तु कर्तव्याः प्राश्निका दशरूपके\renewcommand{\thefootnote}{1}\footnote{भ \textendash\  नाट्ययोक्तृभिः न \textendash\  नाट्यदर्शने}~॥~५३ 

अव्यग्रैरिन्द्रियैः शुद्ध ऊहापोहविशारदः~।\\
त्यक्तदोषोऽनुरागी च स नाट्ये प्रेक्षकः स्मृतः~॥~५४

न चैवैते गुणाः सम्यक् सर्वस्मिन् प्रेक्षके स्मृताः\renewcommand{\thefootnote}{2}\footnote{य \textendash\  ये तुष्टे तुष्टिमायान्ति शोके शोकं व्रजन्ति च~। दैन्ये दीनत्वमायाति ते नाटये प्रेक्षकाः स्मृताः}~।\\
विज्ञेयस्याप्रमेयत्वात्संकीर्णानां च पर्षदि~॥~५५

यद्यस्य शिल्यं नेपथ्यं कर्मचेष्टितमेव वा~।\\
तत्तथा तेन कार्यं तु स्वकर्मविषयं प्रति~॥~५६

नानाशीलाः प्रकृतयः शीले नाट्यं विनिर्मितम्~।\\
उत्तमाधममध्यानां वृद्धबालिशयोषिताम्~॥~५७

तुष्यन्ति तरुणाः कामे विदग्धाः समयान्विते~।\\
अर्थेष्वर्थपराश्चैव मोक्षे चाथ विरागिणः~॥~५८

शूरास्तु वीररौद्रेषु नियुद्धेष्वाहवेषु च~।\\
धर्माख्याने पुराणेषु वृद्धास्तुष्यन्ति नित्यशः~॥~५९ 

न शक्यमधरमैर्ज्ञातुमुत्तमानां विचेष्टितम्~।\\
तत्त्वभावेषु सर्वेषु तुष्यन्ति सततं बुधाः~॥~६०}
\end{quote}

\newpage
% सप्तविंशोऽध्यायः ३१३ 

\begin{quote}
{\na बाला मूर्खाः स्त्रियश्चैव हास्यनैपथ्ययोः सदा\\
यस्तुष्टो तुष्टिमायाति शोके शोकमुपैति च~॥~६१ 

क्रुद्धः क्रोधे भये भीतः स श्रेष्ठः प्रेक्षकः स्मृतः~।\\
एवं भावानुकरणे यो यस्मिन् प्रविशेन्नरः~॥~६२

स तत्र प्रेक्षको ज्ञेयो गुणैरेभिरलंकृतः~।\\
एवं हि प्रेक्षका ज्ञेयाः प्रयोगे दशरूपतः~॥~६३ 

संघर्षे तु ससुत्पन्ने प्राश्निकान् संनिबोधत~।\\
यज्ञविन्नर्तकश्चैव छ्दोविच्छ्ब्दवित्तथा~॥~६४

अस्त्रविच्चिकृद्वेश्या गान्धर्वो राजसेवकः~।\\
यज्ञविद्यज्ञयोगे तु नर्तकोऽभिनये स्मृतः~॥~६५

छ्न्दोविद्वृत्तबन्धेषु शब्दवित्पाट्यविस्तरे~।\\
इष्वस्तवित्सौष्वे तु नेपथ्ये चैव चित्रकृत्~॥~६६

कामोपचारे वेश्या च गान्धर्वः स्वरकर्मणि~।\\
सेवकस्तूपचरे स्यादेते वै प्राश्निकाः स्मृताः~॥~६७}
\end{quote}

\hrule

\vspace{2mm}
\noindent
पशोः सिंहादेर्वेषं कृत्वा सुकुमारं प्रयोक्तारं भीषयति सामाजिकं वा~। एवं मात्सर्यादुन्पत्तलिङ्मपि कश्चित्करोति हासानयनेन प्रकृतप्रयोगविप्रसंवादनायेति वैलक्षण्ये लक्षणविस्मरणमन्यभूमिकोचितसत्त्वस्विकारोऽपि विभूमिकस्तूष्णीकता~। अनन्येन पठनीयमन्यः पठती\underline{त्यन्यवचनं काव्यमिति} बहुव्रीहिः (२२)~। \underline{आर्तनाद} इति~। इतः प्रभृति परद्वेषप्रयुक्ताः सिद्धिविघाताः~। आर्तत्वं हि छद्मना प्रदर्श्य नादं सिद्धिविघातकं करोति~। एवं व्याधिदर्शनेन \underline{मकुटाभरण\textendash }

\lfoot{40}

\newpage
\lfoot{}
% ३१४ नाट्यशास्त्रम्

\begin{quote}
{\na एभिर्दृष्टान्तसंयुक्तैर्दोषा वाच्यास्तथा गुणाः~।\\
अशास्त्रज्ञा विवादेषु यथा प्रकृतिकर्मतः~॥~६८

अथैते प्राश्निका ज्ञेयाः कथिता ये मयानघा:~।\\
शास्त्रज्ञानां यदा तु स्यात्संघर्षः शास्त्रसंश्रयः~॥~६९

शास्त्रप्रमाणनिर्माणैर्व्यवहारो भवेत्तदा~।\\
भर्तृनियोगादन्योऽन्यविग्रहात्स्पर्धयापि भरतानाम्~॥~७०

अर्थपताका हेतोस्संघर्षो नाम संभवति~।\\
तेषां कार्यं व्यवहारदर्शनं पक्षपातविरहेण~॥~७१

कृत्वा पणं पताकां व्यवहारः स भवितव्यस्तु~।\\
सर्वैरनन्यमतिभिः सुखोपविर्ष्टैश्च शुद्धभावैश्च~॥~७२

यैर्लेखकगमकसहायास्सह सिद्धिभिर्घाताः~।\\
नात्यासनैर्नदूरसंस्थितैः प्रेक्षकैस्तु भवितव्यम्~॥~७३

तेषामासनयोगो द्वादशहस्तस्थित कार्यः~।\\
यानि विहितानि पूर्वं सिद्धिस्थानानि तानि लक्ष्याणि~॥७४}
\end{quote}

\hrule

\vspace{2mm}
\noindent
\underline{निपात} इति (२७) नेपथ्यभ्रंशः~। अन्यः \underline{पुष्करावाहितदोषा} इत्यनेनेदमाह न कवलमभिनयानामेव समानीकरणं सामान्याभिनयानां यावदातोद्यगीतयोरप्यन्योन्यमभिनयैश्च समं मीलनं सोऽपि सामान्याभिनयः~। \underline{अन्यदिति}~। गीतादि~। तच्चाभिनयाश्चेति दन्द्वः~। \underline{समशद्वेन} कर्मधारयः तत्र \underline{भवः} प्रयोग इति~।

\newpage
% सप्तविंशोऽध्यायः ३१५ 

\begin{quote}
{\na घाताश्च लक्षणीयाः प्रयोगतो नाट्ययोगे तु~।\\
दैवाद्धातसमुत्थाः परोत्थिता वा बुधैर्नवैर्लेख्याः~॥~७५

घाता नाव्यसमुत्था ह्यात्मससमुत्थास्तु लेख्या: स्यु:~।\\
घाता यस्य त्वल्पाः संख्याताः सिद्धयश्च बहुलाः स्युः~॥~७६

विदितं कृत्वा राज्ञस्तस्मै देया पताका हि~।\\
सिध्यतिशयात्पताका समसिद्धौ पार्थिवाज्ञया देया~॥~७७

अथ नरपतिः समः स्यादुभयोरपि सा तदा देया~।\\
एवं विधिज्ञैर्यष्टव्यो व्यवहारः समञ्जसाम्~॥~७८

स्वस्थचित्तसुखासीनैः सुविशिष्टैटर्गुणार्थिभिः~।\\
विमृश्य प्रेक्षरैर्ग्रह्यं सर्वरागपराङ्गुखैः~॥~७९

साधन दूषणाभासः प्रयोगसमयाश्रितैः~।\\
समत्वमङ्गमाधुर्यं पाठ्यं प्रकृतयो रसाः~॥~८०}
\end{quote}

\hrule

\vspace{2mm}
नन्वञ्जितादिभिः को दोषो जायते इत्याह~। भावस्य प्रयोगस्यानुभावादिरूपस्य दूषणानि तेषु सत्सु तदवस्थावचनात् तत्र सर्वात्मने रसादिदोषाः~। प्रथमेऽध्यायेऽत्राह\textendash

\begin{quote}
{\qt त्रासं सञ्जनयन्ति स्म शेषा विघ्नास्तु नृत्यताम्~।}
\end{quote}

\noindent
इति तत्र स एव सिद्धिविधाते प्रधानतमत्वेनोक्तः~। \underline{विस्वरमजाततालमित्येव} स्पटीकृतं \underline{वर्णे}त्यादिना~। एतत्स्वरूपं च वितत्य गेयाधिकारे निरूपयिष्याम इतीह नोक्तम्~। एवं विधं \underline{स्वरगतं} कर्तृहन्या\underline{द्विहन्ति प्रयोगं} नाशयन्तीत्यर्थः~। एवं पुष्करगतं कर्तृनियोज्यं विधिप्रयोगं हन्तीत्यकाव्यकृतवातस्य स्थानानि

\newpage
page missing

\newpage
% सप्तविंशोऽध्यायः ३१७ 

\begin{quote}
{\na यानि स्थानानि सिद्धीनां तैः सिद्धिं तु प्रकाशयेत्~।\\
हर्षादङ्ग\renewcommand{\thefootnote}{1}\footnote{र \textendash\  वयोभूतां} समुद्भूतां नानारससमुत्थिताम्~॥~८६

वारकालस्तु विज्ञेया नाट्यज्ञैर्विविधाश्रयाः~।\\
दिवसश्चैव रात्रिश्च तयोर्वारान् निबोधत\renewcommand{\thefootnote}{2}\footnote{र \textendash\  विशेषाश्चर्तवस्तु ये~।}~॥~८७

पूर्वाह्नस्त्वथ मध्याह्न स्त्वपराह्णस्तथैव च~।\\
दिवा समुत्था विज्ञेया \renewcommand{\thefootnote}{3}\footnote{र \textendash\  वारकालाः प्रयोक्तृभः}नाट्यवाराः प्रयोगतः~॥~८८

प्रादोषिकार्धरात्रिश्च तथा प्राभातिकोऽपरः\renewcommand{\thefootnote}{4}\footnote{र \textendash\  प्रभातसमयस्तथा}~।\\
नाट्यवारा भवन्त्येते रात्रावित्यनुपूर्वशः~॥~८९

एतेषां यत्र यद्योज्यं नाव्यकार्यं रसाश्रयम्~।\\
तदहं संप्रवक्ष्यामि वारकालसमाश्रयम्~॥~९०}
\end{quote}

\hrule

\vspace{2mm}
\noindent
तत्सर्वमैव सिद्धिविधातकमिति~। \underline{उत्तमव्यतिक्रियामीति}~। \underline{स्थानविशेषैरिति}~। स्थानविशेषैरिति निमित्तैरित्यर्थः~। \underline{यूपाग्निपावनेति}~। तच्छिह्नात् नटो~। दुर्लभश्च तत्र यथात्वं सर्वजनेन सुज्ञातं न च लोकवृत्तोपयोगिनीति भावः~। \underline{सर्वगत} इति~। सर्वत्र प्रयोग एकदेशज इत्यंशे बहुदिनजा\ldots \ldots वरुद्धत्यसौ सिद्धिविघातकः~। \underline{अव्यक्त} इति~। प्रयोगान्तेऽस्य प्रयोगान्तरेण संबन्धो रङ्गे कार्य इत्यर्थः~। \underline{एकदिवसजात} इति~। एकप्रयोगो लक्षणं प्रत्यपर इत्यन्वेति~। \underline{जर्जरमोक्षस्यान्त} इति पूर्वरङ्गपयोगोऽपि परीक्ष्य इति दर्शयति~। नाट्यविधौ यथावदुक्तं ज्ञातुं प्रयोगस्य चोपपादनं कर्तुमशक्तोऽपि व्यग्रमनस्कत्वात् देशवेषाद्यनौचित्येन यो यं प्रयोगं कुर्यात्तस्य सर्वस्य

\newpage
% ३१८ नाट्यशास्त्रम् 

\begin{quote}
{\na यच्छ्रोत्ररमणीयं स्याद्धर्मोत्थान\renewcommand{\thefootnote}{1}\footnote{र \textendash\  धर्माख्यान}कृतं च यत्~।\\
पूर्वाह्णे तत्प्रयोक्तव्यं शुद्धं वा विकृतं तथा~॥~९१

सत्त्वोत्थानगुर्णैर्युक्तं वाद्यभूयिष्ठमेव च~।\\
पुष्कलं \renewcommand{\thefootnote}{2}\footnote{र \textendash\  सिद्धिबहुलं}सत्त्वयुक्तं च अपराह्णे प्रयोजयेत्~॥~९२

कैशिकीवृत्तिसंयुक्तं शृङ्गाररससंश्रयम्\renewcommand{\thefootnote}{3}\footnote{घ \textendash\  ललिताभिनयात्मकम्}~।\\
नृत्तवादित्रगीताढ्यं प्रदोषे नाव्यमिष्यते~॥~९३

यन्नर्महास्यबहुलं\renewcommand{\thefootnote}{4}\footnote{र \textendash\  संयुक्तं} करुणप्रायमेव च~।\\
प्रभातकाले तत्कार्यं नाढ्यं निद्राविनाशनम्~॥~९४

अर्धरात्रे नियुञ्जीत समध्याह्ने तथैव च~।\\
सन्ध्याभोजनकाले च नाट्यं नैव प्रयोजयेत्~॥~९५}
\end{quote}

\hrule

\vspace{2mm}
\noindent
घाताः~। नन्वज्ञस्यैवेत्यार्याद्वयस्य योजना दोषप्रत्यर्थे \underline{न ग्राह्य} इत्युक्ते प्रयोक्तुरवलेपोऽवतरेदित्याशयेनाह \underline{न च नादसित्वति~। तज्ज्ञैरि}त्युक्तम्~। तान् वशीकर्तुमाह \underline{अतः ऊर्ध्वे प्रवक्ष्यामि प्राश्निकानां} त्विति~। प्रश्ने भवा मध्यस्थत्वेनाभिनयचतुष्कगीतातोद्ये चेति~। \underline{षडङ्गत्वान्नाव्यं सन्तोष} इत्यादिना विमलाशयत्वेन सहृदयत्वमेषां परमो गुण इति दर्शयिति \underline{कार्यमिति}~। तेन सह विचार्यमित्यर्थः~। शीलमेव दर्शयति~। \underline{तुष्यन्ति तरुणाः काम} इत्यादिना~। \underline{प्रविशेदिति} साधारणीभावमेवं सूचयनि तत्समीपे भवेदिति यावत्~। एवं वस्तुमात्रविचारे विधिरुक्तः~। प्रयोक्तॄणां परस्परकलहहेतुविधिर्वक्तव्यः~। स च कदाचिल्लक्ष्यमात्रविषयो भवति~। कदाचिल्लक्ष्यणविषयोऽपि पूर्वमधिकृत्याह~।

\newpage
% सप्तविंशोऽध्यायः ३१९ 

\begin{quote}
{\na एवं कालं च देशं च \renewcommand{\thefootnote}{1}\footnote{र \textendash\  पर्षरं च}समीक्ष्य च बलाबलम्~।\\
\renewcommand{\thefootnote}{2}\footnote{र \textendash\  नाय्यवारं}नित्यं नाट्यं प्रयुञ्जीत यथाभावं यथारसम्~॥~९६

\renewcommand{\thefootnote}{3}\footnote{र \textendash\  कदाचित्}अथवा देशकालौ च न परीक्ष्यौ प्रयोक्तृभिः~।\\
यथैवाज्ञापयेद्भर्ता तदा योज्यमसंशयम्~॥~९७

तथा समुदिताश्चैव विज्ञेया नाटकाश्रिताः~।\\
पात्रं प्रयोगमृद्धिश्च विज्ञेयास्तु त्रयो गुणाः~॥~९८

बुद्धिमत्वं सुरूपत्वं लयतालज्ञता तथा~।\\
रसभावज्ञता चैव वयस्स्थत्वं कुतूहलम्~॥~९९

ग्रहणं धारणं चैव गात्रावैकल्यमेव च~।\\
\renewcommand{\thefootnote}{4}\footnote{र \textendash\  जित}निजसाध्वसतोत्साह इति पात्रगतो विधिः~॥~१००

सुवाद्यता सुगानत्वं सुपाठ्यत्वं तथैव च~।\\
शास्त्रकर्मसमायोगः प्रयोग इति संज्ञितः~॥~१०१}
\end{quote}

\hrule

\vspace{2mm}
\begin{sloppypar}
\underline{संघर्षेत्विति~। उपचार} इति~। राजोचित इति भावः~। द्वितीयमधिकृत्याह \underline{शास्त्रज्ञानात्विति~। भर्तृनियोगादिति}~। लेखको लिखति गणकः पिण्डयति~। द्वयोरपि यर्थैकस्य न घातः तेनायमस्याधिकारसिद्धिः~। स चेत्याह \underline{सिद्ध्यतिशयात्पताकेति}~। यदि तु न कुत्रचिदतिशयः~। तदा कथमित्याह~। \underline{इति} शब्दोऽध्याहार्यः~। तत्तद्रसप्रधानं नाट्यं तत्र तत्र कालेषु रसः संभवतीत्यभिप्रायेणाह \underline{देशकाला}विति~। नेपथ्ये पाठः यत्कर्तुं जानाति \underline{तेन योग} इति \underline{पात्रसंपाद्य}न्तर्भावोऽस्य~। पात्रं हि रसप्रविष्टमेव~। \underline{यदा समुदिता}
\end{sloppypar}

\newpage
% ३२० नाट्यशास्त्रम् 

\begin{quote}
{\na शुचिभूषणतायां तु \renewcommand{\thefootnote}{1}\footnote{सुमाल्याम्बरता तथा}माल्याभरणवाससाम्~।\\
विचित्ररचना चैव समृद्धिरिति संज्ञिता~॥~१०२

यदा समुदिताः सर्वे एकीभूता भवन्ति हि~।\\
अलङ्काराः सकुतपा मन्तव्यो नाटकाश्रयाः~॥~१०३

एतदुत्तं द्विजश्रेष्ठाः सिद्धीनां लक्षणं मया~।\\
अत ऊर्ध्वं प्रवक्ष्याम्यातोद्यानां च विकल्पनम्~॥~१०४}
\end{quote}

\begin{center}
\textbf{इति भारतीये नाट्यशास्त्रे सिद्धिव्यञ्जको नाम \renewcommand{\thefootnote}{2}\footnote{र \textendash\  षड्विंशः}सप्तविंशोऽध्यायः}
\end{center}

\hrule

\vspace{2mm}
\noindent
इति सामान्याभिनयत्वमाह~। अध्यायपञ्चकेन हि तदेवोक्तम्~। नाट्योत्पत्तिरिव पूर्वरङ्गान्तेनेत्युक्तम्~। वक्ष्यति च अलङ्कार इति शोभापूर्णतायाश्चतुरित्यर्थः~।

\begin{center}
वक्तव्यशेषं सूचयति अत ऊर्ध्वमित्यादि~। शिवम्~॥
\end{center}

\begin{quote}
{\qt नृसिंहगुप्तापरनामधेय\textendash\  \\
विद्यावदातः मुखलाभिधानः~। \\
यं देहविद्यभिरयूयुजत्सः \\
प्रयोगसिद्धिं कृतवान् महार्थाम्~॥}
\end{quote}

\begin{center}
इति श्रीकाश्मीरमहामाहेश्वराचार्याभिनवगुप्ताचार्यविरचितायां अभिनवभारत्यां नाट्यवेदवृत्तौ सिध्द्यध्यायः सप्तविंशः~॥
\end{center}

\end{document}