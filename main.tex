\documentclass{article}
\usepackage[text={4.65in,7.45in}, centering, includefoot]{geometry}
\usepackage[table, x11names]{xcolor}
\usepackage{fontspec,realscripts}
\usepackage{polyglossia}
\usepackage[utf8]{inputenc}
\usepackage{enumerate}
\pagestyle{plain}
\usepackage{fancyhdr}
\pagestyle{fancy}
\renewcommand{\headrulewidth}{0pt}
\usepackage{afterpage}
\usepackage{multirow}
\usepackage{multicol}
\usepackage{wrapfig}
\usepackage{vwcol}
\usepackage{microtype}
\usepackage{amsmath,amsthm, amsfonts,amssymb}
\usepackage{mathtools}% <-- new package for rcases
\usepackage{graphicx}
\usepackage{longtable}
\usepackage{setspace}
\usepackage{footnote}
\usepackage{perpage}
\MakePerPage{footnote}
\usepackage{xspace}
\usepackage{array}
\usepackage{emptypage}
\usepackage{hyperref}
\hypersetup{colorlinks, citecolor=black, filecolor=black, linkcolor=blue, urlcolor=black}

\begin{document}
\chead{}
\lhead{}
\rhead{}

\begin{flushright}
21101030\\
K Madhulaalasa
\end{flushright}

\begin{center}

\textbf{AE681A: Composite Materials}\\

\vspace{3mm}
{\large Assignment-1}\\

\vspace{4mm}
\hrule

\vspace{4mm}
\textbf{\LARGE Report On Different Manufacturing Process For Composite Materials}\\

\vspace{4mm}
\hrule
\end{center}

\begin{multicols}{2}
\vspace{3mm}
\section{Introduction}

Composite refers to something that is made up of two or more separate pieces. As a result, a composite material is one that has two or more unique constituent materials or phases. One or more discontinuous phases are embedded in a continuous phase in composites. The reinforcement or reinforcing material is the discontinuous phase, which is usually harder and stronger than the continuous phase, whereas the matrix is the continuous phase. The qualities of the constituent materials, their distribution, and their interaction all have a significant impact on the properties of composites.\\

One of the most significant processes in the application of composite materials is the fabrication process. Rather than using generic material forms, structural pieces are made with comparatively basic tooling. Material and structure creation are two different processes in the case of traditional materials. Structures with several components and joints frequently demand extensive tooling and elaborate assembly. In an ideal world, the production method would be chosen at the same time as the material and structural design in a unified and interactive process. The matrix that is employed controls the production process. The fabrication technology was initially developed in a semi-empirical manner, with military uses leading the way, followed by a slew of commercial applications.\\

A number of fabrication processes are available, each of which is ideal for a different use. Autoclave moulding, filament winding, pultrusion, fibre implantation, and resin transfer moulding (RTM) are only a few of them. The so-called co-curing procedure allows structural components made of diverse materials, such as honeycomb sandwich constructions, to be created in one step.

\section{Manufacturing Processes}

\subsection{Wet/ Hand Lay-Up}

Hand lay-up is the oldest, simplest, and most widely utilised method for producing small and big reinforced goods. It's possible to utilise a flat surface, a cavity (female) or a positive (male) mould made of wood, metal, plastics, reinforced polymers, or a mix of these materials. Against the mould surface, fibre reinforcements and resin are manually applied. 

\begin{center}
\includegraphics[width=0.9\linewidth]{hand-layup-process.jpg}\\

{\small Figure 1: Hand Lay-Up Process}
\end{center}

The layers of materials placed against the mould govern the thickness. This method of moulding thermo-setting resins in conjunction with fibres, also known as contact lay-up, is an open-mold technique. Hand lay-up procedures are best suited to low-volume applications where other methods of production would be impractical due to costs or size constraints. Boat and boat hulls, radomes, ducts, pools, tanks, furnishings, and corrugated and flat sheets are all typical applications.

\subsection{Spray Lay-Up}

Spray-up is a semi-automated hand lay-up method. On an open mould, chopped glass fibres and resin are applied simultaneously. Fiberglass roving is fed through a chopper on the spray gun and blown into a resin stream that is directed at the mould by one of two spray systems: one is an external mixing system with two nozzles that ejects resin premixed with catalyst or catalyst alone, while the other ejects resin premixed with accelerator. The internal mixing system, on the other hand, just has one nozzle. Prior to the spray nozzle, resin and catalyst are delivered into a single gun mixing chamber.

\begin{center}
\includegraphics[width=0.9\linewidth]{spray-layup-process.jpg}\\

{\small Figure 2: Spray Lay-Up Process}
\end{center}

\subsection{Vacuum Bagging}

Bag moulding is one of the oldest and most versatile composite part manufacturing methods. The laminae are arranged in a mould, covered with a flexible diaphragm or bag, and cured under heat and pressure. The materials become an integrated moulded item contoured to the appropriate configuration after the requisite cure cycle. The steps of laying up and bagging have a significant impact on the quality of par output. As a result, the ability and knowledge of a worker have a significant impact on the quality of a part. Only the curing equipment, notably the size of the curing oven or autoclave, limits the size of a product that can be manufactured using bag moulding.

\begin{center}
\includegraphics[width=0.9\linewidth]{vacuum_bagging.png}\\

{\small Figure 3: Vacuum Bagging}
\end{center}

The majority of bag-molded parts are made using vacuum-bag and autoclave processes. Their key advantages are the low cost of tooling and the ability to employ standard curing equipment (oven and autoclave) to produce an endless range of shaped parts. Because it is integrated with the curing pressure system, the pressure-bag system has the disadvantage of being rather expensive tooling. Furthermore, the tooling can only be utilised for the item for which it was intended.

\subsection{Filament Winding}

Filament winding is a technique for making surfaces of revolution like as pipes, tubes, cylinders, and spheres, and it's commonly employed in the chemical industry to build enormous tanks and piping. Fibers are impregnated with a resin by drawing them through an in-line resin bath (wet winding)  or prepregs (dry winding) are wound over a mandrel. Wet winding is inexpensive and controls the properties of the composite. Dry winding is cleaner, but more expensive and thus quite uncommon. Depending on the desired properties of the product, winding patterns such as hoop, helical, and polar can be developed. The product is then cured with or without heat and pressure. Depending on the application, mandrels are made of wood, aluminum, steel, plaster, or salts. For example, steel mandrels are chosen for manufacturing large quantities of open-ended cylinders, and low-melting alloys or water-soluble salts are used for closed-ended cylinders so that one can easily remove the mandrel.

\begin{center}
\includegraphics[width=0.9\linewidth]{filament-winding-process.png}\\

{\small Figure 4: Filament Winding}
\end{center}

\subsection{Pultrusion}

Pultrusion is an automated procedure for creating continuous, constant-cross-section profiles out of composite materials. This method resembles aluminium or thermoplastic extrusion in several ways. Pultrusion, on the other hand, pulls the product from the die rather than forcing it out. Using the right dies, you may make a wide range of profiles, including rods, tubes, and various structural shapes. Pultrusion is a method that involves dragging continuous rovings and/or continuous glass mats through a resin bath or impregnator, then into preforming fixtures to partially shape the section and remove excess resin and/or air. The portion is then placed in a heated die, where it is continually cured. Pultrusion is best suited to thermosetting resins that cure without forming a condensation by-product.

\begin{center}
\includegraphics[width=0.9\linewidth]{pultrusion.jpg}\\

{\small Figure 5: Pultrusion}
\end{center}

\subsection{Autoclave Molding}

For military, aerospace, transportation, maritime, and infrastructure applications, the autoclave moulding method is utilised to fabricate high-performance advanced composites. The process has few size and shape limits and produces items with good dimensional tolerances. It's a low-volume, labor-intensive procedure that's also expensive. Typically, thermoset and thermoplastic resins are utilised, which are reinforced with glass, carbon, and aramid fibres and fabrics. Materials in prepreg form are used in the autoclave process. To make a layup, prepreg sheets are cut to size, oriented as needed, and stacked. To absorb excess resin and allow volatiles to escape during curing, a bleeded breather system made of dry glass fibre fabric or mat is utilised.

\begin{center}
\includegraphics[width=0.9\linewidth]{Autoclave.jpg}\\

{\small Figure 6: Autoclave Molding}
\end{center}

\subsection{Resin Transfer Moulding}

A low viscosity resin, such as polyester or epoxy resin, is injected under low pressure into a closed mould containing the fibre preform in resin transfer moulding (RTM), also known as liquid moulding. The resin flow is halted, and the part is given time to cure. The cure can be carried out at room temperature or at high temperatures. If the part is to be utilised in a high-temperature application, the latter is done. RTM has the advantages of being less expensive than hand lay-up, being automated, and not requiring refrigerated prepreg storage. The capital expense of having two moulds instead of one is one of the major disadvantages.

\subsection{Vacuum Assisted RTM}

Vacuum-assisted resin transfer moulding is a version of the RTM technique (VARTM). A peel ply and/or a resin distribution cloth are used to cover the preform when it is laid over an open mould surface. A vacuum bag is used to cover the stack and is sealed around the mold's perimeter. At one place, the resin is injected while vacuum is drawn at another. The vacuum aids resin flow through the preform (and resin distribution blanket). The VARTM process requires less expensive tooling and is well suited to big component fabrication. Only one side of the composite, however, has a moulded finish. Because the resins employed must have a very low viscosity, mechanical qualities must be sacrificed.

\subsection{Braiding}

Brading is an automated, cost-effective method of interlacing fibres into complex designs. Braiding is the mechanical intertwining of three or more yarns in such a way that no two strands are wrapped around each other. Because the braids are continuous, the load is distributed equally throughout the structure. These fibres are coiled into a helix, much like a spring's wire. Braiding can be used to make ropes, tubes, narrow flat strips, countered curved and solid three-dimensional shapes such as I-beams and T-beams.

\begin{center}
\includegraphics[width=0.9\linewidth]{brading.jpg}\\

{\small Figure 7: Braiding}
\end{center}

\begin{center}
\includegraphics[width=0.9\linewidth]{braiding.jpg}\\

{\small Figure 8: Braided Composite}
\end{center}

\subsection{Centrifugal Casting}

Reinforcements and resin are deposited against the interior surface of a revolving mould in centrifugal casting. The materials are held in place by centrifugal force until the portion is cured. The "finished" surface of a centrifugal casting part is the exterior surface of the component that is cured against the inside surface of the mould. To improve surface appearance and provide further chemical resistance in centrifugally cast objects, an additional coating of "neat" or pure resin can be applied to the internal surface. Centrifugal casting is used in the commercial production of large diameter composite pipe and tanks. A polished outer surface and the containment of volatiles during processing are two advantages of centrifugal casting. The capacity to spin huge moulds and the comparatively low productivity per tool are two of centrifugal casting's main drawbacks.

\subsection{Fabrication of Resin Matrix Composites}

Injection moulding is the most common method for producing items made of short-fiber reinforced thermoplastics. For this, traditional mold-and-plunger or reciprocating screw machines are employed. The reinforced material is moulded using the same standard moulding cycle as unfilled thermoplastics, however the production circumstances are substantially different. The fibre length, volume proportion, and degree of dispersion of the fibres throughout the matrix all influence the compounding process chosen. Extruder compounding and strand coating are the two most prevalent compounding processes.

\subsection{Fabrication of Metal Matrix Composites}

Process temperatures for metals are substantially greater than for polymers. Fibers may react with the metal matrix material at these higher temperatures, which always has a negative impact on the composite's characteristics. To keep the interaction between the fibres and the metal matrix to a minimum, further caution must be taken. Liquid infiltration or hot-pressing of solid matrix on fibres are the most common methods for creating metal matrix composites. Process temperatures for metals are substantially greater than for polymers.Fibers may react with the metal matrix material at these higher temperatures, which always has a negative impact on the composite's characteristics. To keep the interaction between the fibres and the metal matrix to a minimum, further caution must be taken.\\

Liquid infiltration is the most common method of fabricating metal matrix composites. Each individual fibre can be coated by drawing it singly through a bead of molten metal in the case of a mutually reactive matrix-fiber combination (e.g., aluminium and silica). Because the coating on a single fibre cools quickly, continuous single fibres can be passed quickly through small metal beads to provide appropriate coating thicknesses with little time for the chemical reaction. To make the composite with this process, the coated fibres must be hot-pressed.\\

A chemical vapour deposition procedure, coextrusion of matrix and pellet-shaped particles to produce fibres, or generating fibres in situ during the controlled solidification of particular off-eutectic or eutectic alloys can also be used to create metal matrix composites.

\subsection{Fabrication of Ceramic Matrix Composites}

A two-stage procedure is used to make ceramic matrix composites, glass, glass-ceramic, and oxide-ceramic matrices. Fibers are integrated into an unconsolidated matrix in the first stage. The slurry infiltration process, in which a fibre tow is passed through a slurry tank (containing the matrix powder, a carrier liquid, and an organic binder) and wound on a drum and dried, is the most typical method for this purpose. Cutting and stacking of tows, as well as consolidation, make up the second step. The most typical method for consolidating ceramic matrix composites is hot pressing or fire at temperatures above 1200°C. High temperatures are necessary to facilitate fast diffusion and recrystallization, allowing densification to reach the desired level in a reasonable amount of time. Porosity is a common and serious defect in ceramic materials. To reduce porosity, the fugitive binder must be entirely removed, and the matrix powder particle size must be lower than the fibre diameter.

\begin{thebibliography}{9}
\bibitem{textbook}
Kaw, Autar K.: Mechanics of composite materials, 2nd ed.

\bibitem{textbook}
Isaac M. Daniel, Ori Ishai: Engineering mechanics of composite materials - 2nd ed.

\bibitem{textbook}
Bhagwan D. Agarwal, Lawrence J. Broutman, K. Chandrashekhara: ANALYSIS AND PERFORMANCE OF 
FIBER COMPOSITES, 3rd ed.

\bibitem{textbook}
Carl.T.Herakovich: Mechanics of Fibrous Composites

\bibitem{lamport94}
Faizan S. Awan, Mohsin A. Fakhar, Laraib A. Khan, Usama Zaheer, Abdul F.Khan \& Tayyab Subhani (2018): Interfacial mechanical properties of carbon nanotube-deposited carbon fiber epoxy matrix hierarchical composites, Composite Interfaces

\bibitem{lamport94}
Ma Quanjin,et.al: Filament Winding Technique: SWOTAnalysis andApplied Favorable Factors

\bibitem{lamport94}
Hom Nath Dhakal, Sikiru Oluwarotimi Ismail: Design, manufacturing processes and their effects on bio-composite properties

\end{thebibliography}

\end{multicols}
\end{document}